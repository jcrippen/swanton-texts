%!TEX root = ../swanton-texts.tex
%%
%% 92. Mountain Dweller (280–288)
%%

\resetexcnt
\chapter{Shaakanaayí: Mountain Dweller}\label{ch:92-mountain-dweller}

\section{Swanton’s abstract}\label{sec:92-swanton-abstract}

Another version of 65.

Two girls ate between meals, contrary to the tabus, and their mother scratched the inside of the mouth of the elder and scolded them both.
Among other things she told them that they could not marry Mountain Dweller.
Then the girls ran away, and after wandering for some time, came to Mountain Dweller, who married them.
While they were there their mother-in-law killed them becauser they looked at her while she was eating, but Mountain Dweller killed her in turn and restored them to life.
After that they went to their father’s town, and their husband accompanied them, carrying a magic basket which contained an enormous amount of food, and yet was made small enough to be carried on his thumb.
Afterward they killed their mother in revenge.

\section{Swanton’s translation}\label{sec:92-swanton-translation}

A chief was living with his two children in the middle of a long town.
People were always visiting him, and he kept tallow stored away for strangers.
By and by a big canoe came to him, and [the peoples’] things were taken up.
{}[The children’s] grandmother had charge of the tallow.
She always had things stored away for strangers.
Then she would give these to her grandchildren.
Afterward the old woman would say,
“The old shaggy dog took it away from me.”
After that he invited the foreign people up.
He ordered the tallow in the big box to be brought for them.
Now there was nothing inside of the big box.
The foreign people, however, were all seated.
It was thought that his children had done it.
They had invited them for the food that was all eaten up.
This is why people say even now,
\{281\} “They came to invite for the food that was gone.”
It was entirely empty, and great was the shame that the chief felt.
Afterward he questioned his children.
Their dishes had hair on them.
There was a dish apiece, which always lay by them.
Then their mother came in to them.
“Did you do this?”
she said.
When they kept on crying, she raised the face of the older girl.
She scratched her daughter’s cheek, and also that of the younger one.
She scratched on both of their cheeks because they ate up the tallow for which [her husband] had invited strangers.
When the people went to bed that night the girls made a hole under the boards.
Then they put the hairy dishes in their places.
Afterward they went back into a hollow tree.

Next morning [their mother] said,
“I wonder where they have gone.”
She said to them,
“Getupnow.”
Then the long dishes moved [as she pulled at the covers].
It was the dishes they had put in their places.
They, however, had dug a hole underneath and were gone.
Then their mother came out from behind the screens.
No one knew \{282\} whither they had gone.
Afterward they went straight up into the woods.
And after they had started [the people] rushed up to hunt for them, but they hid themselves.
The younger kept saying to the elder,
“Let us make some kind of noise for our mother.”
She answered,
“How does the inside of your cheeks feel?”
She kept saying to her younger sister,
“Oh! we can not do it.
She said to us,
‘Let Mountain Dweller marry both of you.’
I know what she was saying to us.”

For this reason they went far up into the woods.
They wandered along, aimlessly crying.
The younger sister wanted her elder sister to go back to the place from which they had started, but she did not want her mother to see her down there.
After they had gone a long distance they saw a small mouse running across a log.
The mouse went into a little hill.
Then her younger sister said,
“Grandmother mouse, people have seen you.”
So said her younger sister.
“Put me quickly across this log,”
said the little mouse.
“My grandmother says
‘Call them into the house.’”
On account of that it had run out.
\{283\}
Then the door flew open.
They [entered and] sat down.%
\footnote{\{Swanton’s footnote \textit{a}\}
The story is very much condensed here.
The mouse’s “grandmother” had sent it to invite them in.
The mouse asks to be put over the log because the entrance to her grandmother’s house was on the other side.
“On account of that she had run out” refers to the mouse’s first appearance.}
“Why did you come?”
she said to them.
After they had been seated for some time she pushed something between her teeth, and got something out.
It was a piece of dried fish.
She shook it.
It was now a spring salmon taken from between her teeth, and they placed it by the fire.
She set it before them, and they consumed it.
She took a cranberry out from between her teeth.
She placed it before them, and they consumed that.
After they had eaten she said again,
“Why did you come, my little grandchildren?”
and the elder replied,
“My mother said we could not marry Mountain Dweller.”
“He is a very difficult person to get near.
Go now my little grandchildren.”
Then she told them what to do.
“Crushing-mountain is before the place, granddaughters, and also the fighting dogs (cᴀk!).”
She also said,
“Kelps float together in front of it.
Take your knife and a whetstone with you,”
she said.
After she had instructed them they started out.
When they had gone along for
\{284\}
some time they saw the fighting dogs.
They threw a piece of dried fish bone to them, and the dogs began to divide it.
Again they went forward.
Before they had gone far they came upon kelps floating together.
They threw moss between.
Then they passed through.
After that they saw Crushing-mountain.
(Just the way people tell this I am telling you, my opposite clansman.)
They threw a whetstone between 
They went through.
Now they saw the camp.
They came to the house door.

Mountain Dweller’s mother was at home.
Nothing could be seen inside of this house, there was so much fat.
They were told they could not get into Mountain Dweller’s house.
That is why they went there.
After they had been seated for some time they were given something to eat.
By and by the hunter brought in a load of food.
He asked his mother,
“What are those people that have come to you doing?”
“They came to marry you because it was said that they could not.”
So Mountain Dweller married both of them.

\{285\}
After they had been there for some time he started off.
He said to his wives,
“My mother does not let the person that stays with me last long.”
For this reason they kept sticks in their hands while he was away from them.
Some time afterward their mother-in-law put a side of mountain sheep into the fire.
She stood it up on end.
Then it caught fire.
This was the way she killed her son’s wives.
After that they kept watch on her.
When it was burning she pushed it toward her son’s wives.
Then they pushed it back upon her, and killed her.
They pulled her body outside and put something over it.
They let it stand out of the ground a very little.

Meanwhile her son was away.
When he arrived he was carrying a big mountain sheep.
Then he asked for his mother.
“She did to us just as you said.
We threw it over upon her.
We pulled her outside.”
He said to them,
“What you have done to her is well.
My mother would not let a person who lived with me last long.”
After that he collected sides of mountain sheep, inside fat, and tallow.

\{286\}
After many years had passed Mountain Dweller said to his wives,
“Wouldn’t you like to go home?”
“Yes,” said they.
{}[The elder] said to him,
“My mother said we could not marry you.
That is why we came to find you.”
“Weave some baskets,”
he said.
So they wove them.
“Weave two that you can just put on your thumbs” [he said].
They were going to start.
There were many mountains between.
After they had put many canoe loads of things inside of the baskets he put them both on his thumb, and they started along with them.
They were gone for a very few days.

When they were going along with him he seemed to be changed suddenly.
Mountain Dweller began to shine from within.
By and by they sighted their father’s town.
The town was long.
In the evening they came in front of the house.
He had the small baskets on his thumb.
Then they wished that their little brother might run out to them.
They called him to them.
The people had already
\{287\}
given a mourning feast for them there.
A year was now past.
For this reason he ran into the 
Then he said to his mother,
“My sisters have come and are outside.”
At this she became angry with her young son, who had longed for his sisters.
“You lie,”
she said to him.
At once he went back to them, crying.
When he came into the house again he said to his mother,
“They are there.
It is well that you go out to them.”
“Take a piece off of their marten blankets and bring it here,”
she said.
So he told them.
(The way I am telling you is the way people always tell old stories.)
Then he brought it into the house.
At that time his mother started out.
She looked.
Her children were really there.
“Come into the house,” she said.
So they came into the house to her.
Afterward the elder girl told her mother about the baskets.
Mountain Dweller having shaken the baskets, she said,
“There are big baskets outside.
Let them be brought in.”
Then two persons went out.
The baskets were too heavy for them.
More went out.
All the men in the house tried to bring them in.
\{288\}
When they could not, Mountain Dweller rose to get the baskets.
Although they were unable to get them, Mountain Dweller put the baskets on his third finger.
Inside was fat from the inside of a mountain sheep.
Because her mother had scratched the inside of her daughters’ cheeks, [the elder girl] invited the people for nothing but fat.
The things in the baskets were too much for them.
The baskets in which these things were contained, were called World-renowned-baskets.

\clearpage
\section{Paragraph 1}\label{sec:92-para-1}

\ex\label{ex:92-1-aristocrat-dwells-in-length-of-town}%
\exmn{280.1}%
\begingl
	\glpreamble	Qōwaū′ayu yuānqā′wo yū′ān qołayê′tq!//
	\glpreamble	Ḵuwa.óo áyú yú aanḵáawu, yú aan kulayátʼxʼ. //
	\gla	\rlap{Ḵuwa.óo} @ {} @ {} @ {} \rlap{áyú} @ {}
		{} yú \rlap{aanḵáawu,} @ {} @ {} {} +
		{} {} yú aan \rlap{kulayátʼxʼ.} @ {} @ {} @ {} @ {} @ {} {} {} {} //
	\glb	ḵu- i- \rt[²]{.u} -μμH á -yú
		{} yú aan ḵáaʷ -í {}
		{} {} yú aan k- u- l- \rt[¹]{ÿatʼ} -μH {} {} -xʼ {} //
	\glc	\xx{areal}- \xx{stv}- \rt[²]{own} -\xx{var} \xx{foc} -\xx{dist}
		{}[\pr{DP} \xx{dist} town- man -\xx{pss} {}]
		{}[\pr{PP} {}[\pr{DP} \xx{dist} town
			\xx{cmpv}- \xx{irr}- \xx{xtn}- \rt[¹]{long} -\xx{var} \·\xx{nmz} {}]
			-\xx{loc} {}] //
	\gld	\rlap{\xx{impfv}.dwell} {} {} {} \rlap{it.is} {}
		{} that \rlap{aristocrat} {} {} {}
		{} {} that town \rlap{\xx{gcnj}.\xx{stv}·\xx{impfv}.long} {} {} {} {} -th {}
			-in {} //
	\glft	‘He dwells, that aristocrat, in the length of a town.’
		//
\endgl
\xe

\FIXME{Compare the similar initial sentences in chs.\ \ref{ch:89-origin-of-copper} and \ref{ch:90-man-abandoned} as well as in \citeauthor{swanton:1909}’s number 10 (p.\ 38).
All of these narratives are from Deikeenaakʼw and suggest this was a favoured trope of his for starting stories.
Aren’t there a few similar starts in Tsimshian stories?}

\ex\label{ex:92-2-children-live-there}%
\exmn{280.1}%
\begingl
	\glpreamble	a duỵê′tq!î qo′dzîtî.  //
	\glpreamble	Áa du ÿátxʼi ḵudzitee. //
	\gla	{} \rlap{Áa} @ {} {} {} du \rlap{ÿátxʼi} @ {} @ {} {}
		\rlap{ḵudzitee.} @ {} @ {} @ {} @ {} @ {} //
	\glb	{} á -μ {} {} du ÿát -xʼ -í {}
		ḵu- d- s- i- \rt[¹]{tiʰ} -μμL //
	\glc	{}[\pr{DP} \xx{3n} -\xx{loc} {}]
		{}[\pr{DP} \xx{3h·pss} child -\xx{pl} -\xx{pss} {}]
		\xx{areal}- \xx{mid}- \xx{appl}- \xx{stv}- \rt[¹]{be} -\xx{var} //
	\gld	{} there -at {} {} his \rlap{children} {} {} {}
		\rlap{\xx{impfv}.live} {} {} {} {} {} //
	\glft	‘His children live there.’
		//
\endgl
\xe

\ex\label{ex:92-3-his-children-are-two}%
\exmn{280.2}%
\begingl
	\glpreamble	Dᴀxᴀnᴀ′x ỵᴀtî′ duỵê′tq!î. //
	\glpreamble	Dáx̱a̬náx̱ ÿatee du ÿátxʼi. //
	\gla	{} \rlap{Dáx̱a̬náx̱} @ {} {} \rlap{ÿatee} @ {} @ {}
		{} du \rlap{ÿátxʼi.} @ {} @ {} {} //
	\glb	{} déix̱ -náx̱ {} i- \rt[¹]{tiʰ} -μμL
		{} du ÿát -xʼ -í {} //
	\glc	{}[\pr{AdvP} two -\xx{hum} {}] \xx{stv}- \rt[¹]{be} -\xx{var}
		{}[\pr{DP} \xx{3h·pss} child -\xx{pl} -\xx{pss} {}] //
	\gld	{} \rlap{two} {} {} \rlap{\xx{impfv}.be} {} {}
		{} his \rlap{children} {} {} {} //
	\glft	‘His children are two.’
		//
\endgl
\xe

\ex\label{ex:92-4-travellers-always-visit}%
\exmn{280.2}%
\begingl
	\glpreamble	Doxᴀ′nde nahā′ỵe tc!aʟᴀ′k usᴀ′ttc. //
	\glpreamble	Du x̱ánde naháaÿi chʼa tlákw us.átch. //
	\gla	{} Du \rlap{x̱ánde} @ {} {}
		{} \rlap{naháaÿi} @ {} @ {} @ {} {}
		chʼa tlákw \rlap{us.átch.} @ {} @ {} @ {} @ {} @ {} //
	\glb	{} du x̱án -dé {}
		{} n- \rt[¹]{ha} -μμH -í {}
		chʼa tlákw u- d- s- \rt[¹]{.at} -μH -ch //
	\glc	{}[\pr{PP} \xx{3h·pss} near -\xx{all} {}]
		{}[\pr{DP} \xx{ncnj}- \rt[¹]{mv·invis} -\xx{var} -\xx{nmz} {}]
		just always \xx{zpfv}- \xx{pasv}- \xx{csv}-
			\rt[¹]{go·\xx{pl}} -\xx{var} -\xx{rep} //
	\gld	{} his near -to {}
		{} \rlap{traveller} {} {} {} {}
		just always \rlap{\xx{zcnj}.\xx{hab}.go·\xx{pl}} {} {} {} {} {} //
	\glft	‘Travellers always come near him.’
		//
\endgl
\xe

\FIXME{Discuss problems with interpreting \fm{d-s-\rt[¹]{.at}}.
The root \fm{\rt[¹]{saʼt}} ‘tight’ is nonsensical so this must be \fm{s-\rt[¹]{.at}}
and then either the form is \fm{oos.átch} with \fm{a-u-} or \fm{us.átch} with \fm{u-d-}.}

\ex\label{ex:92-5-put-away-tallow}%
\exmn{280.3}%
\begingl
	\glpreamble	G̣ona′n qoa′ne q!ēs ᴀt yī′akutcā′kᵘtc yutū′. //
	\glpreamble	G̱una.aan ḵwáani x̱ʼéis át yee akoocháakch yú toow. //
	\gla	{} {} \rlap{G̱una.aan} @ {} \rlap{ḵwáani} @ {} \rlap{x̱ʼéis} @ {} {}
			át \{yee\} {}
		\rlap{akoocháakch} @ {} @ {} @ {} @ {} @ {}
		{} yú toow. {} //
	\glb	{} {} g̱una- aan ḵwáan -í x̱ʼé =yís {} 
			át \{ÿee\} {}
		a- k- u- \rt[²]{chaʼk} -μμH -ch
		{} yú toow {} //
	\glc	{}[\pr{DP} {}[\pr{PP} other- town people -\xx{pss} mouth =\xx{ben} {}]
			thing {}]
		\{below\} \xx{arg}- \xx{qual}- \xx{zpfv}- \rt[²]{store} -\xx{var} -\xx{rep}
		{}[\pr{DP} \xx{dist} tallow {}] //
	\gld	{} {} \rlap{foreign} {} \rlap{people} {} mouth \•for {} thing {}
		\{below\} \rlap{3>3.\xx{hab}.store} {} {} {} {} {}
		{} that tallow {} //
	\glft	‘As things for foreigners to eat, he would always store it, that tallow.’
		//
\endgl
\xe

\FIXME{\orth{yī′} is apparently something like \fm{ÿee} but syntactically it’s difficult unless there’s something missing like \fm{neilÿee} or \fm{a kax̱ÿee}.
See \cite[03/187]{leer:1973}.
Sentence (\ref{ex:92-9-}) has the same thing so it’s not a mistake here.}

\ex\label{ex:92-6-big-canoe-came-near}%
\exmn{280.3}%
\begingl
	\glpreamble	Wananī′sayu doxᴀ′nt uwaqo′x yū′yākᵘ ʟēn. //
	\glpreamble	Wáa nanée sáyú du x̱ánt uwaḵúx̱ yú yaakw tlein. //
	\gla	{} Wáa \rlap{nanée} @ {} @ {} @ {} {}
		\rlap{sáyú} @ {} @ {}
		{} du \rlap{x̱ánt} @ {} {}
		\rlap{uwaḵúx̱} @ {} @ {} @ {}
		{} yú yaakw tlein. {} //
	\glb	{} wáa n- \rt[¹]{ni} -μμH {} {} 
		s= á -yú
		{} du x̱án -t {}
		u- i- \rt[¹]{ḵux̱} -μH
		{} yú yaakw tlein {} //
	\glc	{}[\pr{CP} how \xx{ncnj}- \rt[¹]{happen} -\xx{var} \·\xx{sub} {}]
		\xx{q}= \xx{foc} -\xx{dist}
		{}[\pr{PP} \xx{3h·pss} near -\xx{pnct} {}]
		\xx{zpfv}- \xx{stv}- \rt[¹]{go·boat} -\xx{var}
		{}[\pr{DP} \xx{dist} boat big {}] //
	\gld	{} how \rlap{\xx{csec}.happen} {} {} \·while {}
		ever\· \rlap{it.is} {}
		{} his near -to {}
		\rlap{\xx{zcnj}.\xx{pfv}.go·boat} {} {} {}
		{} that canoe big {} //
	\glft	‘At some point it came near him, that big canoe.’
		//
\endgl
\xe

\ex\label{ex:92-7-ppl-brought-them-inland}%
\exmn{280.4}%
\begingl
	\glpreamble	Dᴀ′qde wuduˈʟ̣îāt. //
	\glpreamble	Dáḵde wududli.aat. //
	\gla	{} \rlap{Dáḵde} @ {} {} \rlap{wududli.aat.} @ {} @ {} @ {} @ {} @ {} @ {} //
	\glb	{} dáaḵ -dé {} wu- du- d- l- i- \rt[¹]{.at} -μμL //
	\glc	{}[\pr{PP} inland -\xx{all} {}]
		\xx{pfv}- \xx{4h·s}- \xx{mid}- \xx{csv}- \xx{stv}-
			\rt[¹]{go·\xx{pl}} -\xx{var} //
	\gld	{} inland -to {} \rlap{\xx{ncnj}.\xx{pfv}.handle·\xx{pl}} {} {} {} {} {} {} //
	\glft	‘People brought them inland.’
		//
\endgl
\xe

The verb in (\lastx) has an unmarked third person object.
This is notable because \citeauthor{swanton:1909}’s gloss “the things were taken” implies that there should be the indefinite (fourth person) nonhuman object \fm{at=} rather than the unmarked third person object.
The Tlingit sentence is thus not explicit whether the things moved are inanimate entities and thus the posessesions of the visitors or whether they are animate entities and thus the visitors themselves.
The English translation “them” used here is deliberately ambiguous because it can refer to either an animate or inanimate referent.
