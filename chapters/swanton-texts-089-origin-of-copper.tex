%!TEX root = ../swanton-texts.tex
%%
%% 89. Origin of copper (252–261)
%%

\resetexcnt
\chapter{Tináa Shagóon: Origin of coppers}\label{ch:89-origin-of-copper}

This narrative was told to \citeauthor{swanton:1909} by \fm{Deikeenaakʼw} John Morris in Sitka in 1904.
In the original publication it is number 89, running from page 252 to 261 and totalling 125 lines of glossed transcription.
\citeauthor{swanton:1909}’s original title is “The Origin of Copper”.
This story has not been presented elsewhere with any conventional title.
The Tlingit name given here, \fm{Tináa Shagóon}, is an idiomatic translation of \citeauthor{swanton:1909}’s title.
The \fm{tináa} is the highly valued copper plate or ‘shield’ traded across the Northwest Coast \parencites{keithahn:1964}[353–354]{de-laguna:1972}{jopling:1989}[179–183]{emmons:1991}[237–243]{kan:2016}.
\citeauthor{swanton:1909}’s ‘origin’ is idiomatically translated as \fm{shagóon} ‘source, origin, background; fundament, element, component; cause and effect; ancestor, background, heredity’ \parencite[\textsc{t}·63]{leer:2001}; \textcite[71 ff.]{kan:2016} provides a detailed discussion of \fm{shagóon} and associated metaphysical concepts.

This story probably comes from Coast Tsimshian culture as it is recorded several times from different Coast Tsimshian people.
This is suggested in the narrative here by explicit references to \fm{Tsʼootsxán} ‘Coast Tsimshian’ in sentences (\ref{ex:89-57-above-Tsimshian-land}) and (\ref{ex:89-60-Tsimshian-town}).
Assistance from an old mouse woman as in sentences (\ref{ex:89-33-deermouse-super-help})–(\ref{ex:89-38-she-explain-to-her}) is also a common feature in several Coast Tsimshian stories.
In addition, \citeauthor{emmons:1991}’s consultant places the story “in a more southern Tlingit village” which supports an origin among more southern peoples.
Confirming this, \citeauthor{boas:1895} recorded a longer and more detailed version of the same story in 1886 as “\fm{Tsag·atilâ′o}” \parencite[581–589]{boas:2002} from a Coast Tsimshian consultant nammed Mathew or Matthias who was from Old Metlakatla \parencites[69]{boas:1912a}[547]{boas:2002}.
\citeauthor{boas:1912a} also recorded the same story from Henry Tate of Port Simpson, with editing by Archie Dundas of New Metlakatla, as the “Story of \fm{Gunaxnēsᴇmg·a′d}” \parencite[147–191]{boas:1912a}.
\FIXME{Check Ralph Maud’s discussions for other references.}
Both of these Coast Tsimshian narratives begin with a plot very similar to the Tlingit one here from \fm{Deikeenaakʼw} but they move on to other episodes not present in the Tlingit narrative.
\FIXME{Haida versions? Also review Barbeau’s “native Orpheus” discussion.}

The parts of this narrative before paragraph 7 can be identified elsewhere as independent narratives.
\FIXME{maybe quote Boas’s discussion; \cite[835]{boas:1916}}

The first part of this narrative is clearly identifiable as \fm{Xóots X̱ʼayaa Ḵuwdlig̱adi Shaawát} ‘Woman Who Offended The Bear’ which is also known as \fm{Wé Xóotsch Uwashayi Shaawát} ‘The Woman Whom A Bear Married’.
Two versions of this story in Tlingit were published by the Dauenhauers, one from \fm{Yéil Naawú} Tom Peters \parencite[166–193]{dauenhauer:1987} and one from \fm{Naakil.aan} Frank Dick Sr.\ \parencite[194–217]{dauenhauer:1987}.
There is also an unpublished manuscript in Tlingit from \fm{Yaanalchéen} Ben Watson of Hoonah which was transcribed by \fm{L.udashánx̱} John Fawcett from an apparently lost audio recording \parencite{fawcett:1973a}.
Several other versions of the same story have been published in English.
\citeauthor{swanton:1909} has a version from \fm{Ḵaadashaan} \parencite[126–129]{swanton:1909} which could be identified as the start of a sequence like this one since it is followed by \fm{Shamgigétk} (pp.\ 129–130), \fm{Náaḵwch Uwasháyi Shaawát} ‘Woman Whom A Devilfish Married’ (pp.\ 130–132), and then \fm{Tináa Shagóon} (pp.\ 132–133).
\citeauthor{de-laguna:1972} provides a version of \fm{Xóots X̱ʼayaa Ḵuwdlig̱adi Shaawát} ‘Woman Who Offended The Bear’ in English from \fm{Sʼoow Dalé} Minnie Johnson and another from \fm{Ḵaajax̱daḵeinaa} Sheldon James Sr.\ in Yakutat \parencite[880–883]{de-laguna:1972}.
McClellan recorded at least eight versions in English: four from the Southern Tutchone elders Johnny Fraser, Maggie Jim, Lily Birckel, and Mary Jacquot \parencite[39–41, 143–146, 163–167, 181–186]{mcclellan-cruikshank:2007a}; three from \fm{Stéew} Angela Sidney, \fm{Yeilkʼidáa} Jimmy Scotty James, and \fm{La.oos Tláa} Maria Johns in Carcross and Tagish \parencite[302–308, 435–437, 465–474]{mcclellan-cruikshank:2007b}; and one from \fm{Yeilshaan} Jake Jackson in Teslin \parencite[498–512]{mcclellan-cruikshank:2007c}.
\citeauthor{birket-smith-de-laguna:1938} recorded a version from their Eyak consultant Galushia Nelson \parencite[277–279]{birket-smith-de-laguna:1938}.

The second part starting with paragraph 3 is reminiscent of the story \fm{G̱agaanch Uwashayi Shaawát} ‘Woman Whom The Sun Married’ which was recorded by McClellan from \fm{Stéew} Angela Sidney \parencite[288–289]{mcclellan-cruikshank:2007a} and twice by \citeauthor{olson:1967} from unspecified consultants \parencite[44, 67 col.\ 2 ¶3]{olson:1967}.
Paragraphs 4 through 6 are reminiscent of \fm{G̱agaan Yátxʼi} ‘Children of the Sun’.
\citeauthor{de-laguna:1972} recorded this story in English from \fm{Ḵuchein} Frank Italio, \fm{Sʼoow Dalé} Minnie Johnson, and \fm{X̱ʼadanéikʼ} Nick Milton \parencite[873–875]{de-laguna:1972}.
\citeauthor{birket-smith-de-laguna:1938} also recorded this story in English from their Eyak consultant Galushia Nelson \parencite[294–300]{birket-smith-de-laguna:1938}.

The last part of the narrative starting with paragraph 7 is essentially the core of the \fm{Tináa Shagóon} story.
The same story can be identified in \fm{Ḵaadashaan}’s English narrative \fm{Ḵaa Eetí Shuká Ḵáa} ‘In Front of Garbage Man’ \parencite[132–133]{swanton:1909} which has the creature in the lake explicitly identified as a brown bear canoe.
Emmons provides another version of this story from an unknown consultant in Wrangell. De Laguna reported this from Emmons’s notes, pointing out that “it is a variant of the stories recorded by Swanton at Wrangell and Sitka, although in these the monster killed by the widow’s son was her father’s miraculous brown bear canoe, which was made of copper plates” \parencite[180]{emmons:1991}. As this version is fairly short it is included for reference on page \pageref{sec:89-emmons-version}.

The story called “The Discovery of Copper” recorded by De Laguna from \fm{Dlaasx̱ee} Harry Bremner \parencite[899–900]{de-laguna:1972} might seem to be related to this one based on its title, but they are actually distinct stories.
Among other things, in \fm{Dlaasx̱ee}’s narrative the copper is discovered in the ashes of a fire after a spiritual experience, and the protagonist is male.
This narrative has the copper coming from a canoe instead and the initial protagonist is female.
The two narratives both provide origins for copper, but they have irreconcilable differences and so must be different stories.

The term \fm{lukanáa} appears in sentence (\ref{ex:89-52-married-lukanaa}) in paragraph 4 on page \pageref{ex:89-52-married-lukanaa} as well as in a few subsequent sentences.
This is a very rare noun that refers to spiritual practices that spread north along the coast from Vancouver Island to southern Tlingit country.
Today these practices are extinct and are known only today from ethnographic records and occasional mentions in some traditional narratives.
\citeauthor{swanton:1909} glosses \fm{lukanáa} as “cannibal”, but this is misleading because the term does not refer to cannibals per se.
The \fm{lukanáa} practices do include (feigned) cannibalism but this does not seem to be the primary focus and the term \fm{lukanáa} is never used to cannibals in other contexts.
In the analysis here it has been glossed simply as ‘\xx{name}’ and given without translation since there is no straightforward equivalent of it in English and because its meaning in Tlingit is unclear.

\citeauthor{leer:1978b} gives this word as \fm{lookanáa} and Tongass \fm{loòkanaa} [\ipa{ɬuʰ.kʰa.ˈnaː}] and defines it as “member of secret society whose trademark is acting bestial and eating dogs” \parencite[16]{leer:1978b}.
Earlier he has the definitions “demoniac” and “person who acts crazy” \parencite[04/33]{leer:1973} which are likely direct quotes from his consultants.
\citeauthor{naish-story:1976} give \fm{lookanáa} defined as “demoniac (secret society)” \parencite[93]{naish-story:1976} which is likely from \citeauthor{leer:1973}.
I have only heard \fm{lukanáa} [\ipa{ɬù.kʰà.ˈnáː}] with a short initial vowel which accords with \citeauthor{swanton:1909}’s transcription.
\citeauthor{swanton:1909}’s \orth{Łuqᴀnā′} suggests \fm{luḵanáa} with a uvular and this is also reflected in \citeauthor{mcclellan:1975b}’s \orth{łuqanau} \parencite[567]{mcclellan:1975b}.
Since the word is borrowed and thus morphologically opaque, shifts in pronunciation are unsurprising.

\citeauthor{de-laguna:1972} quotes \fm{Sʼoow Dalé} Minnie Johnson in a discussion of \fm{yéik} ‘shamanic spirit’ behaviour as mentioning a word \orth{Łukᴀ} which she glosses as “power” \parencite[704 col.\ 1 para.\ 2]{de-laguna:1972}.
This is apparently a word \fm{luká} [\ipa{ɬù.kʰá}] that is not otherwise attested.
\citeauthor{leer:1973} has \fm{ḵaa luká} ‘someone’s nose-ridge’ \parencite[f06/5]{leer:1973} but this is a straightforward compound of \fm{lú} ‘nose’ and \fm{ká} ‘horizontal surface’ and so is probably unrelated.
It is tempting to connect \citeauthor{de-laguna:1972}’s \fm{luká} to the noun \fm{lukanáa}: \fm{luká} ‘power’ could conceivably be backformed from the loanword \fm{lukanáa} by reanalysis of the latter as a compound \fm{lu-ka-náa} with a stem \fm{\rt{na}-μμH}, then deleting the stem.
This hypothesis is difficult to support without additional evidence of similar backformations, and \citeauthor{de-laguna:1972} expressly states that there is no evidence for a \fm{lukanáa} tradition in Yakutat \parencite[628]{de-laguna:1972}.
Until we have more data on the purported noun \fm{luká} ‘power’ we cannot say whether it is connected to the better attested \fm{lukanáa}.

The word \fm{lukanáa} in Tlingit is borrowed from either Haida or Coast Tsimshian along with its associated spiritual practices.
The ultimate source is a Kwakʼwala term transcribed by Boas as \fm{ʟō′koala} and glossed as “the supernatural one” \parencite[693 line 1]{boas-hunt:1897}.
\citeauthor{lincoln-rath:1980} give Kwakwʼala and Oowekyala \fm{λugʷala} and Heiltsuk \fm{λúgʷálá} as “one who has found supernatural power, lucky” along with Oowekyala \fm{λukʷa} “to prepare for the acquisition of supernatural power” and Kwakwʼala and Oowekyala \fm{λaλuxʷsila} “people assisting the shaman with his incantations” \parencite[180]{lincoln-rath:1980}. \citeauthor{grubb:1977} gives Kwakwʼala \fm{dlúgwala} as “lucky, fortunate” along with \fm{dlúgwi} “good fortune”, \fm{dli7dlúgwi} “treasured possessions, fortunes”, and \fm{dlúgway̓u} “sweetheart, precious one” \parencite[165]{grubb:1977}.
\citeauthor{nater:1990} lists \fm{lhukwala} “student of supernatural power, shaman-to-be” in Nuxalk along with some derivations \parencite[64]{nater:1990}.

As noted earlier, the \fm{lukanáa} practices are no longer known today, so it is difficult to interpret the term.
\textcite{swanton:1908} offers a short discussion of \fm{lukanáa} from his Sitka and Wrangell consultants which is repeated here for reference.

\begin{quote}\small
Secret society dances were imported from the south, as the name łuqᴀna′, evidently from Kwakiutl ʟū′koala, testifies, but their observance had by no means reached the importance attained among the Kwakiutl and Tsimshian.
At Sitka the writer heard of but one man who had become a łuqᴀna′, a Kîksᴀ′dî named Maawᴀ′n [JC: perh.\ \fm{Maa.aawán}?].
He said that the łuqᴀna′ were spirits who came from the body of the łuqᴀana′ wife of the Sun’s son, a cannibal woman referred to in one of the chief Tlingit stories, who was broken to pieces and thrown down by her husband.
When they came upon him, they would fly along through the air with him.
They forced him to eat dogs and do various other things, and they made him cry “Hai, hai, hai, hai.” Once, as they were flying along, they left him suddenly, and he dropped upon the side of a cliff where he hung on the point of a rock by his cheek.
At the time of his possession people ran around with him with rattles and sang certain songs to keep him from going away, and they also sat on the tops of the houses singing.
All this was to restore him to his right mind.
At Wrangell the łuqᴀna′ performances seem to have been better known and to have existed in greater variety.
A man could imitate any animal except a crest of some other family.
As was the case farther south, whistles (łuqᴀna′ doᴀ′t-cî [i.e.\ \fm{lukanáa du at.sheeyí} ‘\fm{lukanáa} song’]) were essential concomitants of the secret society dances.
\sourceatright{\parencite[436]{swanton:1908}}
\end{quote}

\citeauthor{swanton:1909}’s mention of the \fm{lukanáa} wife is probably a reference to this particular story \fm{Tináa Shagóon}. 
\citeauthor{swanton:1909} also recorded from \fm{Ḵaadashaan} in Wrangell an origin story for \fm{lukanáa} in English which is part of \fm{Ḵaadashaan}’s long Raven story sequence \parencite[133–135]{swanton:1909}.
Included in this \fm{lukanáa} origin story is some commentary about them which is repeated here.

\begin{quote}\small
There are many kinds of łuqᴀna′s.
Some are dog-eaters and some pretend to eat the arms of people.
It is previously arranged between the luqᴀna′ and his father what he is to do and whom he is to injure, and, after the spirit has come out, the father has to pay a great deal of money for damages.
The łuqᴀna′s are always found at feasts, and high-caste people stand around them.
The people who learned from this boy first are those in the direction of Victoria, and there they think that a person who has performed many times is very high.
It is only very lately that we Alaskans have had łuqᴀna′s.
Łuqᴀna′ is a Tsimshian word meaning yēk. [Footnote: Actually it is from the Kwakiutl word ʟū′koala.
Katishan calls it Tsimshian because the Tlingit received their secret societies through them.] When they perform up here, the southern Tlingit dance Tsimshian dances and the northern Tlingit Athapascan dances.
\sourceatright{\parencite[134]{swanton:1909}}
\end{quote}

According to \citeauthor{de-laguna:1972}, Emmons witnessed a performance at Sitka but failed to recognize it as \fm{lukanáa} since he says “There were no secret societies” \parencite[21]{emmons:1991}; she notes other mentions of the practices from \textcite[98–100, 118–121]{olson:1967} and suggests that the dances were called \fm{yéik sʼaatí} ‘spirit master’ \parencite[319]{emmons:1991}.
The appearance of this term in both Emmons’s and Olson’s materials from consultants in Ketchikan, Wrangell, Klawock, and Sitka suggests that it was widely understood.
But the term \fm{yéik sʼaatí} literally refers to a person rather than to a dance, so it is not entirely clear if this term really only denoted the person doing the dances or if it was used metonymically for the dances themselves.

\citeauthor{olson:1967}’s description of the \fm{lukanáa} phenomenon includes the incongruous statement “None of the Tlingit north of the Tantakwan and Wrangell groups knew of them” \parencite[98]{olson:1967}.
This contradicts the mention here of \fm{lukanáa} in this narrative by \fm{Deikeenaakʼw} in Sitka well before \citeauthor{olson:1967} ever visited Tlingit country.
It is unclear from the context of this discussion whether this is a claim from \citeauthor{olson:1967} or from his consultant George Mackay,\footnote{George Mackay or “GM” was a member of the \fm{G̱aanax̱.ádi} clan of \fm{Taantʼá Ḵwáàn} \parencite[82–83]{olson:1967}.
\citeauthor{olson:1967} gives his name as “Cántacu´h” \parencite[98]{olson:1967} but this has not been clearly identified in Tlingit.} but later \citeauthor{olson:1967} notes that his consultant is in error about denying the use of the word \fm{lukanáa} in Tlingit \parencite[100]{olson:1967}.
He later cites \fm{Yaandusgéi} Amelia Sloan Cameron of Sitka\footnote{\citeauthor{olson:1967} only mentions her as \fm{Daawoolsʼéesʼ} John Cameron’s wife – “Mrs.\ DC of Sitka”, “Mrs.\ Don Cameron” – but \textcite[130]{jones:2017} identifies her as \fm{Yaandusgéi} Amelia Sloan Cameron.} for additional information about the \fm{lukanáa}, confirming that it was known at least as far north as Sitka and so that \fm{Deikeenaakʼw} had probably witnessed \fm{lukanáa} performances at the same time as \fm{Yaandusgéi}.

\citeauthor{garfield-forrest:1948} recount a story from \fm{Gunyaa} George Gunyah of a dog-eater spirit (\fm{keitl ax̱á yéik}) which “caused those under its influence to lose all control over their actions” and that “only by eating dog flesh could the spirit be quieted” \parencite[143]{garfield-forrest:1948}.
In this story the spirit was obtained from Tsimshians and there is a potlatch where a young man becomes possessed by a spirit and implicitly consumes dog meat \parencite[141–145]{garfield-forrest:1948}, fitting other descriptions of \fm{lukanáa} events.

\citeauthor{mcclellan:1975b} offers a short mention of \fm{lukanáa} among the Inland Tlingit under the heading of “Cannibal Possession” in a larger discussion of possession by spirits.

\begin{quote}\small
Cannibal possession was known to occur among the coastal Tlingit, but only one Inland Tlingit woman had heard her grandmother actually describe people possessed by cannibal spirits.
It had happened long ago on the coast.
Evidently those who were cannibal dancers put on exhibitions that only men and old women were allowed to see – not middle-aged or young women, or children either.

\begin{quote}
\fm{luqanau} is the cannibal.
When the kids are crazy in their play the women say, “You’re just like łuqanau.”
\end{quote}

Only men of noble birth became “cannibal”.
Apparently they were always sent into a frenzy by the smell of a dead bear or the sight of its blood.
\sourceatright{\parencite[567]{mcclellan:1975b}}
\end{quote}

\citeauthor{mcclellan:1954} also says “We have heard one report of a chief’s son at Sitka who ‘went cannibal crazy’ from time to time, but always before a limited gathering of old men.” \parencite[96]{mcclellan:1954}.
She suggests that the \fm{lukanáa} practices failed to spread beyond the southern Tlingit area because ceremonies were tightly connected to kinship in Tlingit culture \parencites[96]{mcclellan:1954}[567]{mcclellan:1975b} but she does not elaborate on this idea.

For more details on the \fm{lukanáa} practices in other cultures, the seminal publication is by Boas and Hunt on the Kwakwakaʼwakw \parencite{boas-hunt:1897}.
Other early discussions include Boas on Coast Tsimshian \parencite[546–558]{boas:1916}, Barbeau on the Coast Tsimshian \parencite[xi–xii]{barbeau-beynon:1987a}, \citeauthor{swanton:1905a} on Haida \parencite[155–181]{swanton:1905a} \FIXME{etc}.
\FIXME{List some notable retrospective analyses from the later 20th century.}
Narratives that discuss \fm{lukanáa} practices include Coast Tsimshian material from Henry Tate
\parencite[350–353, 353–354]{boas:1916}, Helen Clifford \parencite[93–95, 97–99]{barbeau-beynon:1987a}, Nathan Shaw \parencite[5–6, 8–12, 206–208]{barbeau-beynon:1987b}, and John Tate \parencite[206–208]{barbeau-beynon:1987b}; Haida material recorded by Swanton from an unspecified Masset consultant \parencite[793–795]{swanton:1908a}[Eng.\ trans.][158–160]{swanton:1905a} and from Abraham of Skidegate \parencite[425, 429–433, 434–443]{swanton:1905b}; \FIXME{others}.

\clearpage
\begin{pairs}
\begin{Leftside}
\beginnumbering
\pstart
\noindent
\snum{1}Aan kulayátʼ digeeÿeegé aya.óo aanḵáawu.
\snum{2}Du sée ḵukʼéetʼ aku̬shitán.
\snum{3}Ḵukʼéetʼ aan woo\-.aat, du éesh goox̱xʼú tin.
\snum{4}A káa yan kawdliÿásʼ yú xóots háatlʼi yú dáaḵxʼ ḵukʼéetʼi.
\snum{5}Yéi aÿawsi\-ḵaa yú xóots háatlʼi
\snum{6}«\!Tsʼas ḵaa x̱ʼoos ÿeedé has aléelʼ, tóoḵ ḵákw.\!»
\snum{7}Aatx̱ íḵde has aÿada.áat áwé ÿawlikʼoots du ḵágu.
\snum{8}Du éesh goox̱xʼúch áwé ÿa\-sahéix̱ a kaadé du jeeÿís.
\snum{9}Tlax̱ de yá du éesh neilí x̱ánxʼ áwé tsu ÿawlikʼoots.
\snum{10}Chʼu tle yéi aÿaw\-si\-ḵaa
\snum{11}«\!Chʼa wa.éich déi ÿasahá.\!»
\snum{12}A kaadé chʼa tléináx̱ áwé de yéi adaané;
\snum{13}du x̱ánt uwagút yú ḵáa;
\snum{14}wásʼ yaa ashakanalÿén.
\snum{15}«\!Iḵashaa\!» tle yóo ash yawsiḵaa.
\snum{16}Chʼu tle ash een g̱unéi uwa.át.
\snum{17}Daḵdachóon ásíyú déix̱ x̱áaw taÿeenáx̱ ash een ÿaawa.át.
\snum{18}X̱ách shaa áyú x̱áaw yáx̱ ash tuwáa ÿatee.
\pend
%2
\pstart
\snum{19}Du eetéex̱ ḵuÿawduwashée, yú shaawát yú aantḵeiních.
\snum{20}Yan yóo ḵudushée áwé du eetéexʼ yoo x̱ʼawu̬duwatán.
\snum{21}X̱ách xóots ḵwáani ásíyú ash uwasháa, yú áx̱ x̱ʼanyaa ḵuwdlig̱ádi, yú aanÿádi.
\snum{22}X̱áatg̱aa na.ádi na\-.átch yú xóots ḵwáani.
\snum{23}Yú x̱áatg̱aa na.ádi eetéexʼ áwé héen táakw shaag̱í yéi adaanéi nuch.
\snum{24}Hú ḵu.aa tsʼas xook alʼéexʼ nuch.
\snum{25}Kei ag̱a.ádín áwé, x̱áat aaní dáx̱, ḵaa kʼoo\-dásʼi, kaax̱ kínde dug̱ích nuch.
\snum{26}Kadukíksʼi nuch.
\snum{27}A tóotx̱een áwé tle eix̱ yáx̱ át akoogaanch yú shaaḵ x̱ooxʼ.
\snum{25}Du aaÿí ḵu.aa áwé tsʼas koolkéesʼch yú xook, yú shaawát.
\snum{29}A kaax̱ áwé tléil unalé wáa sá wdusneeyí, yú aanyátkʼu.
\pend
%3
\pstart
\snum{30}Tsu ana.áat áwé tsu has woo.aat gáng̱aa.
\snum{31}Chʼa yá du x̱ʼusÿeedé áwé awsiteen, yú shaa\-wátch, sʼeiḵ.
\snum{32}Yú goochkʼi tóonáx̱ naashóo.
\snum{33}Kag̱áak ḵusháanákʼw ásíyú ash eeg̱áa woo\-soo.
\snum{34}«\!Neil gú, chx̱ánkʼ.
\snum{35}Tleil \{niyaa kushi\-g̱aneix̱ át i ÿát\} áwé;
\snum{36}xóots ḵwáani áwé iwsineix̱\!»
\snum{37}ash een ḵunáax̱ daaḵ akaawaník.
\snum{38}Chʼu tle ashukaawajáa «\!Yóodu ee éesh aa\-ní\!».
\snum{39}A yáx̱ áwé, chʼu tsʼootaat x̱áatg̱aa na\-.aadí g̱unayéi .áat áwé,
\snum{40}a dakádeen áwé yóot wujixíx.
\snum{41}Yagiyee kei a.áat áwé du eetéex̱ ḵuyawduwashée xóots ḵwáanich.
\snum{42}Yáaxʼ kei uwalʼáḵw du lʼaagí, yú shaawát.
\snum{43}De tléixʼ shaa kaanáx̱ ÿawusheexí áwé ḵux̱ awdlig̱én du ítde.
\snum{44}Tle kag̱ít yáx̱ gwáawé ÿatee du ít, xóots ḵwáani.
\snum{45}Ash káa yax̱ ÿaa g̱a.áat áwé sh yaÿeedé kdag̱áax̱.
\snum{46}Áa x̱ʼayaax̱í daak wujixíx.
\snum{47}Yú áa tlein, a digeeyeegéit gwáayú wlihaash yú yaakw; shadaakóox̱ʼ, a shá.
\snum{48}«\!Haandé héent isheex!\!» yóo ash yawsiḵaa.
\snum{49}Tle a kaa\-dé héent wujixíx.
\snum{50}Yaax̱ wuduwaÿeiḵ.
\snum{51}Chʼu tle ash een deikéet wudzix̱ák g̱agaan tóot.
\pend
%4
\pstart
\snum{52}Lukanáa ásíyú has aawasháa yú g̱agaan ÿátxʼi.
\snum{53}Has ag̱asháanín tléil sdu jee has ool\-tsáakw.
\snum{54}Tle s ajáḵx̱.
\snum{55}Ÿeedádi aaÿí ḵu.aa áwé sh tóog̱aa s awdisháa.
\snum{56}A ya.áakdáx̱ áwé has aawajáḵ yú lukanáa.
\snum{57}Tsʼootsxán aaní kináaxʼ áyú has aawajáḵ.
\snum{58}Chʼa ÿéi gusgéi wóoshdáx̱ awu̬lisʼóow.
\snum{59}Ách áwé lukanáa áa shaÿandihéin.
\snum{60}Tsʼootsxán aaní, tle áxʼ áwé l gooháa.
\snum{61}Du éesh aaní, a kináa wug̱axíxín, yú g̱agaan yéi yandusḵéich:
\snum{62}«\!Héidu i éesh aaní.\!»
\snum{63}Wáa nanée sáwé ÿát has aawa.oo.
\snum{64}Hasdu shukát áwé ÿatán hasdu éesh yaagú, xóots yaakw.
\snum{65}Ḵuwa.áx̱ch yú yaakw.
\snum{66}A ÿík s at kawligaa.
\snum{67}Hasdu wóo x̱ánde daneit a ÿeedé yéi wududzinei.
\snum{68}Hasdu een g̱una\-yéi uwagút.
\snum{69}Chʼáakw yaa nagúdi áwé ḵux̱ akoodadzéeych.
\snum{70}X̱ách óot yaan g̱ahéinín áwé wé yaakw, daneit has akoostʼéix̱ʼch a yatʼákwxʼ, yú yaakw.
\pend
%5
\pstart
\snum{71}A eig̱ayáat has uwaḵúx̱ du wóo.
\snum{72}Awu̬si\-kóo du éesh hídi.
\snum{73}Tle a eig̱ayáa daaḵ uwagút.
\snum{74}Du éekʼch áwé neil tʼaa.uwagút.
\snum{75}«\!Ax̱ dlaakʼ gáant uwagút.\!»
\snum{76}A káaxʼ áwé dujáaḵw du tláach, chʼáakw ḵut wudzigeedi du dlaakʼ sh waḵkawdaneegéech.
\snum{77}Áa yux̱ woogoot du tláa.
\snum{78}X̱ách xʼéig̱aa ásíyú dáḵde has dul.aat.
\snum{79}Hás ḵu.aa tléil has duteen.
\snum{80}X̱ách de chʼa á ásíyú yú aldís x̱ʼusyee yáx̱ ḵaa tuwáa ÿatee. 
\snum{81}Daaḵ kadujéil áwé yú at.la.át, áa yux̱ aawagoot.
\snum{82}«\!Tléil daa át\!» yóo s yawdudzi\-ḵaa.
\snum{83}Du shát yéi ÿaawaḵaa:
\snum{84}«\!De chʼa á áwé wé aldís x̱ʼusÿee yáx̱ ÿatee.
\snum{85}Yéi ÿanaÿsaḵá ‹\!áa daaḵ ÿi.aadí.\!›\!»
\snum{86}Yéi ÿawdudziḵaa.
\snum{87}Daaḵ uwa.át.
\pend
%6
\pstart
\snum{88}Chʼu tle g̱agaan x̱ʼoos, wáa sá neil kax̱du\-gúg̱ún, yú g̱agaan x̱ʼoos yú shaawát tuwánxʼ;
\snum{89}hasdu ÿéet kʼátskʼu tsú x̱ʼaseiÿí, á tsú g̱agaan x̱ʼoos yáx̱ ÿatee.
\snum{90}Chʼu tle neilxʼ yan has ḵéi áwé tsa wáa sá a tóonáx̱ \{keis yánáx̱ has ÿi yáx̱\} áwé yéi s yatee, yú ḵugwáasʼ.
\snum{91}«\!At g̱ax̱aa dé ax̱ séekʼ\!» yóo ÿaawaḵaa yú aanḵáawu.
\snum{92}Tlax̱ sh kastáa xwáa áwé wujixeex hasdu x̱ʼéis héeng̱aa.
\snum{93}Áxʼ kei aawatán gijook ḵínaÿi.
\snum{94}A kaadé aawatsaaḵ.
\snum{95}Yoo yan kaawatán x̱éilʼ kaax̱, tléil sh kawushkook yú ḵáa.
\snum{96}\{Shunaaÿát\} yandayéiḵ áwé tsá, du éekʼ kʼátskʼu akaawaḵaa.
\snum{97}Chʼu tleix̱ de héen hasdu x̱ʼéi\-de aawayaa hasdu éek kʼátskʼu.
\snum{98}Ḵut g̱agóot áwé du éekʼ, héeng̱aa aawataan xʼeesháa du x̱úx̱\-xʼu wán x̱ʼéis.
\snum{99}Dax̱dahéen héeng̱aa nagóot áwé ash jín awu̬lisháat ḵáa héen x̱ʼéixʼ.
\snum{100}Chʼu tle neil awu̬sinéi áwé du x̱úx̱xʼu wán x̱ánxʼ a kaadé wduwatsaaḵ gijook ḵínaÿi.
\snum{101}Chʼu yú du jín wudulsháadi áwé tle yoo yan kaawatán x̱éilʼ kaax̱.
\snum{102}Tle áwé wudináḵ du x̱úx̱xʼu wán gánde, du náḵ.
\snum{103}Chʼu héitʼaa áwé ag̱ashátch, tle a tóonáx̱ wujélch.
\snum{104}Chʼu tle tléil has wudusteen.
\snum{105}Hasdu yaagú ḵu.aa áwé áa kát wujixeex.
\pend
%7
\pstart
\snum{106}Hasdu ÿéet ḵu.aa áwé yéi át has awdi\-shée, kaháasʼch ÿaa kg̱ajaaḵ.
\snum{107}Ách áwé du éesh nag̱a\-náanín atkʼátskʼu, ḵʼanashgidéich ujaaḵch.
\snum{108}Tlax̱ wáa yóo kashóo sáwé de du yátkʼu, du tláa tin gáani yux̱ kawdudli.oo.
\snum{109}Aan shukáxʼ áwé cháash hít a káa awliyéx̱.
\snum{110}Du yátkʼu, áxʼ aan yéi wootee.
\snum{111}Ashóoch nuch du yátkʼu yú cháash hít ÿík.
\snum{112}Deisgwách tlʼag̱áa yaa nalgéin.
\snum{113}Ḵaa x̱ʼa.eetí áwé du kaadé kdug̱éich nuch.
\snum{114}«\!Yáatʼaa áa aya.óowu ḵáa\!» yóo áwé daaÿaduḵáa nuch.
\snum{115}Óox̱ udulshooḵ nuch.
\snum{116}Wáa nanée sáwé yux̱ wujixeex yú atkʼátskʼu ash koolÿádi x̱ooxʼ.
\snum{117}«\!Chʼa ée, X̱ʼa.eetí Shuÿee Ḵáa\!» tle yóo wdu\-wasáa.
\snum{118}Du tláa yéi aÿawsiḵaa «\!Sáḵs ax̱ jee\-ÿís layéx̱\!».
\snum{119}Aa tle yéi anasnée áwé chʼu yaa akandagáni áwé aan nagútch atʼúkt.
\snum{120}Lda\-kát át áwé atʼúkdi nuch.
\snum{121}Ḵáax̱ ÿaa ksatée áwé deisgwách yú áakʼw a ÿaax̱í daak gútch.
\pend
%8
\pstart
\snum{122}Xʼoon aa áa daak góot sáwé ash ÿís a ÿeenáx̱ kei x̱ʼeiwaxíx.
\snum{123}X̱ʼaan yáx̱ ÿatee a laká.
\snum{124}Dax̱dahéen yéi ash nasnée, du tláa ax̱ʼeiwawóosʼ
\snum{125}«\!Daa sáyú, atléi?\!»
\snum{126}Chʼu tle yan awsinée ÿées tláaḵ.
\snum{127}«\!Deikée xʼwán daak eesheex i yáx̱ x̱ʼawutʼaax̱í;
\snum{128}a x̱aakw ḵaa jigala.átch.
\snum{129}I éesh yaagú áwé.\!»
\snum{130}Áxʼ áa yaax̱t wugoodí áwé ash yáx̱ x̱ʼeiwatʼaax̱.
\snum{131}«\!Du laká chʼa tʼúk.\!»
\snum{132}Chʼu tle awutʼóogu áwé yéi wduwa.áx̱ «\!G̱áa!\!», yéil yáx̱.
\snum{133}A yáx̱ shaÿawdlixáshi yé yáx̱ áwé woonei a yax̱a\-kʼáawu.
\snum{134}X̱ách eiḵ yaagú áyú;
\snum{135}yéi kwdiwóox̱ʼ a yax̱akʼáawu.
\snum{136}X̱ách chʼas tle yéi ÿatee eiḵ áyú;
\snum{137}tle kaawawálʼ yú yaakw, ldakát á.
\snum{138}Taat ÿeennáx̱ áwé aawayaa du hídidé, du tláa x̱ánde.
\snum{139}Tléil leengítch wuskú.
\pend
%9
\pstart
\snum{140}Chʼu tle aatlein hítxʼ áwé ÿaa analyéx̱ yú eeḵx̱.
\snum{141}Yú cháash taÿeexʼ, aadé áwé aa tʼéx̱ʼ nuch, tláaḵ sákw, kées sákw.
\snum{142}Tléil g̱ayéisʼ ḵusteeyín, ḵachʼu eiḵ yáx̱ ÿateeÿi át.
\snum{143}Tináa yáx̱ tsú atʼéix̱ʼ.
\snum{144}Tle neil ÿée yaa ashakanajél.
\snum{145}Chʼu tle du kát kudux̱eichch x̱ʼa.eetí.
\snum{146}«\!Yá datʼéx̱ʼ aanḵáawu.\!»
\snum{147}Yan asnée wé hít ka yú tináa, de shaÿadihéin yú neilxʼ aadé atʼéx̱ʼ át.
\snum{148}Chʼa yéi óox̱ ÿanax̱dulshooḵt áwé, kʼisáani x̱oo yux̱ nashíxch.
\snum{149}«\!Chʼa ée, X̱ʼa.eetí Shuyee Ḵáa.\!»
\pend
%10
\pstart
\snum{150}Yú aan sʼaatí sée tléil du jeedé yéi ḵaa saduhá.
\snum{151}Ldakát yéitx̱ dusháaxʼw áwé chʼu tle a ÿís yan uwanée.
\snum{152}Hóoch ḵu.aa taat áwé sh daadé yéi jiwu̬dinei.
\snum{153}Eiḵ katíx̱ʼ awsitee.
\snum{154}Áa teix̱ ÿé awsikóo yú aanÿádi.
\snum{155}Tʼáa x̱ʼáa\-náx̱ áwé ách yoo aklitsáḵk yú shaawát, yú eiḵ katíx̱ʼch.
\snum{156}Yú shaawátch wulisháat.
\snum{157}Aw\-dziníxʼ.
\snum{158}Tléil aag̱áa awuskú yú eiḵ.
\snum{159}Tléil leen\-gít aaníxʼ áx̱ dustínji áyú eiḵ.
\snum{160}Chʼu tle aawax̱oox̱ «\!Haagú, gáanxʼi\!».
\snum{161}X̱áni yux̱ woogoot.
\snum{162}«\!Ax̱ hídi ÿeedé x̱áan na.á dé.
\snum{163}Ax̱ x̱áni ÿéi ikg̱watée\!»
\snum{164}yóo aÿawsiḵaa.
\snum{165}Goodáx̱ ḵáax̱ sáyú oowajée;
\snum{166}tléil yéi awus\-kú.
\snum{167}Yú du éex̱ dunéekw ḵáax̱ sateeyí, tsʼas yú éeḵde áyú ash een ÿaa na.át.
\snum{168}Chʼa du jishukát áwé neil shuwjix̱ín, yú eiḵ x̱ʼaháat.
\snum{169}Du yát kawdigán a ÿee.
\snum{170}Chʼu tle gootx̱ át sáyú tle neilÿee ashaÿakaawajél yú tináa.
\snum{171}Chʼu tle aawasháa du hídixʼ.
\pend
%11
\pstart
\snum{172}Du eeg̱áa ḵudushee yú shaawát.
\snum{173}Wu\-dudzihaa, xʼoon ÿagiÿee kaanáx̱ sá.
\snum{174}Déix̱ ux̱ei aag̱áa, óog̱aa ḵudushee ÿé.
\snum{175}Wáa nanée sáwé du éesh goox̱ yéi aÿawsiḵaa
\snum{176}«\!Kʼé ig̱inaat ḵeeshee.\!»
\snum{177}Át ḵoowashée yú goox̱ du x̱ánt.
\snum{178}Chʼu tle áa neil ÿawus.aayí áwé yú goox̱ gáani ḵux̱ wujiḵáḵ.
\snum{179}Du ÿát kawdigán.
\snum{180}Yú hít ÿeedáx̱, «\!Neil gú\!» yóo aÿawsiḵaa yú shaawát x̱úx̱ch.
\snum{181}«\!Líl keeneegéeḵ yá ax̱ hídi\!» tle yéi aÿawsiḵaa.
\snum{182}Yéi ḵu.aa yan aÿawsiḵáa,
\snum{183}«\!‹\!X̱ʼa.eetí Shuyee Ḵáach uwa\-sháa\!› yóo xʼwán sh kaneelneek.\!»
\snum{184}Tle neil wugoodí áwé akaawaneek.
\snum{185}«\!X̱ʼa.eetí Shu\-yee Ḵáach uwasháa\!» yóo sh kalneek.
\snum{186}Chʼu tle áwé yux̱ has jiwdi.át.
\snum{187}«\!Ax̱ séekʼ!\!» yóo x̱ʼayaḵá du tláa.
\snum{188}Chʼu tle a x̱ʼawoolt has loowagúḵ.
\snum{189}Yú cháash hít géit la.á hít neil ashukawlitséx̱.
\snum{190}«\!Dám\!» yóo wuduwa.áx̱.
\snum{191}Du yát áyú kawdigán.
\snum{192}Yú hít ÿee aanáx̱ gáani ḵux̱ has wudikélʼ.
\snum{193}Goosú a ÿís xʼaant wunoogú?
\snum{194}Chʼu tle kaawadéixʼ.
\pend
%12
\pstart
\snum{195}Neildé has na.áat áwé aag̱áa ḵukaawaḵaa du wóo.
\snum{196}Du x̱ánt has .áat áwé nasʼgadooshú tináa ash náa ÿéi awsinei, a sée awushaaÿích.
\snum{197}Tle a daadáx̱ dé shawdudlig̱éch yú cháash hít ÿéeÿi.
\snum{198}Yóot kawdigán yú eiḵ.
\snum{199}Á ḵu.aa du éesh áwé yéi ash eet tuwditán du eeg̱áa at nag̱asoot.
\snum{200}Ách áwé yá ÿeedádi ḵáawu tsú ḵʼanashkidéix̱ nax̱satéenín óog̱aa áyú at yaseik.
\snum{201}Ách áwé héinax̱.á eiḵ áa x̱ʼaw\-li\-tseen, áxʼ yéi at wuneeÿéech.
\snum{202}Chʼu yá ÿeedát x̱áng̱aa tsú déix̱ goox̱ \{woosh kaadé yatseenín\} tináa.
\snum{203}Chʼa tlákw ḵudziteeÿi átx̱ sitee, áxʼ.
\pend
\endnumbering
\end{Leftside}
%%
%% Column break.
%%
\begin{Rightside}
\beginnumbering
\pstart
\noindent
\snum{1}In the middle of the length of a town lives an aristocrat.
\snum{2}His daughter is wont to go berry picking.
\snum{3}She went berry picking with them, with her father’s slaves.
\snum{4}She stepped on it, that brown bear dung, while she is inland picking berries.
\snum{5}She said to the brown bear dung
\snum{6}\qqk{}“They just de\-fecate under people’s feet, butt baskets.”
\snum{7}Later it was when they returned to the beach that it snapped, her basket.
\snum{8}It is her father’s slaves that are gathering it up there for her.
\snum{9}It is just right near to her father’s home that it broke again.
\snum{10}So then one said to her
\snum{11}\qqk{}“Now just you gather it up.”
\snum{12}So there it is alone that she is working on it,
\snum{13}and he came near to her, that man,
\snum{14}and he is waving a bush.
\snum{15}\qqk{}“Let me marry you” he then says to her.
\snum{16}So then he started off with her.
\snum{17}It seems that it is directly inland where he goes off with her below two logs.
\snum{18}Actually it is mountains that are like logs to her mind.
\pend
%2
\pstart
\snum{19}They were searching after her, that girl, those townspeople.
\snum{20}Having finished searching arou\-nd, they talked in her absence.
\snum{21}Apparently it was actually the brown bear that married her, which she had offended, that aristocrat.
\snum{22}They always went going for salmon, those brown bear people.
\snum{23}It is after that going for salmon that he always gets driftlogs below the water.
\snum{24}Her however, she always breaks just dry wood.
\snum{25}Then having gone up from the salmon land, people always throw their shirts off upward.
\snum{26}They always shake them repeatedly.
\snum{27}It is from within them that they burn it there like grease, among those logs.
\snum{25}It was hers however that always just went out, the dry wood, that girl.
\snum{29}It was following that that not long after they did something to her, that little aristocrat.
\pend
%3
\pstart
\snum{30}People having gone out again, they went again for firewood.
\snum{31}It was below her foot that she saw it, that girl, smoke.
\snum{32}It extends through a little hill.
\snum{33}It is apparently a little old lady deermouse who gave supernatural help to her.
\snum{34}\qqk{}“Come inside, grandchild.
\snum{35}It is not \{unintelligible\};
\snum{36}it is the brown bear people who have rescued you”
\snum{37}she explains to her.
\snum{38}She instructed her “There is your father’s land”.
\snum{39}So like that, them having started going to go for salmon in the morning,
\snum{40}it is in the opposite direction of them that she ran away.
\snum{41}It was having come up in the day that they searched in her absence, the brown bear people.
\snum{42}Here it is all worn out, her dress, that girl.
\snum{43}It was now that she had run across one mountain that she looked back after herself.
\snum{44}Then it seems like it is darkness following her, the bear people.
\snum{45}Them all coming down to her, she is wailing in anticipation.
\snum{46}She ran out at the edge of a lake.
\snum{47}That big lake, apparently it was in the middle of it that it floated, a canoe; a clan hat (on) its head.
\snum{48}\qqk{}“Run this way into the water!” it said to her.
\snum{49}She ran across into the water.
\snum{50}She was pulled aboard.
\snum{51}Just then it was snapped back up with her into the sun.
\pend
%4
\pstart
\snum{52}Apparently it was a \fm{lukanáa} that they had married, those children of the sun.
\snum{53}Whenever they marry her, they do not have her lasting in their possession.
\snum{54}They just repeatedly kill her.
\snum{55}It is the one of now however that they are pleased to have married.
\snum{56}It is for room for her that they killed that \fm{lukanáa}.
\snum{57}Above Tsim\-shian country is where they killed her.
\snum{58}They just chopped her up into small pieces.
\snum{59}That is why \fm{lukanáa} have come to be so numerous there.
\snum{60}The Tsimshian town, it is obvious there.
\snum{61}Above her father’s town, whenever they fly around above it, the sun says to her:
\snum{62}\qqk{}“Here is your father’s town.”
\snum{63}At some point they have a child.
\snum{64}It is ahead of them that there lies their father’s canoe, a brown bear canoe.
\snum{65}It can hear people, that canoe.
\snum{66}They distribute things within it.
\snum{67}They put boxed grease within it for their father-in-law.
\snum{68}It began going with them.
\snum{69}Having been going along for a long time it would stop short.
\snum{70}Actually whenever he gets hungry, that canoe, they smash a grease box on its temple, that canoe.
\pend
%5
\pstart
\snum{71}They boated to the beach below him, their father in-law.
\snum{72}She knew her father’s house.
\snum{73}She went up on the beach in front of it.
\snum{74}It was her brother who went up inside.
\snum{75}\qqk{}“My sister has come outside.”
\snum{76}Consequently he is beaten by his mother, because of reporting seeing of his sister who was long lost.
\snum{77}His mother went outside there.
\snum{78}Actually it apparently is true that people are coming ashore.
\snum{79}Them however people can’t see.
\snum{80}Actually now it just is apparently that they are like those beams of moon-shining to people’s minds.
\snum{81}Them having hauled it ashore, that baggage, she went outside.
\snum{82}“There is nothing” they said to her.
\snum{83}His wife said:
\snum{84}\qqk{}“Now it is they who are like beams of moon-shining.
\snum{85}Tell them ‘come ashore there’.”
\snum{86}Thus they told them.
\snum{87}They came up.
\pend
%6
\pstart
\snum{88}So then sunbeams, however they poke them inside, those sunbeams in that girl’s mind;
\snum{89}their little son also in front, it too is like a sunbeam.
\snum{90}\{After they were seated inside of the house they began to appear as if coming out of a fog.\}
\snum{91}\qqk{}“Let my daughter eat things now” that aristocrat said.
\snum{92}Right then a fellow lying down ran to get water for them.
\snum{93}He picked up there a fishhawk pinion.
\snum{94}He poked it into it.
\snum{95}\{If it bent over on account of the wet the man had not behaved himself.\}
\snum{96}Everyone having examined it then, she sent her younger brother.
\snum{97}Forever after he packed water for them, their younger brother.
\snum{98}Having gone lost, her brother, for water she carried a bucket for her husbands.
\snum{99}Having gone twice for water, a man grabbed her hand at the mouth of the river.
\snum{100}It was when she brought it inside that next to her husbands one poked a fishhawk’s pinion in it.
\snum{101}Since one had grabbed her hand, the pinion bent over from slime.
\snum{102}Then they stood up, her husbands, going outside and leaving her.
\snum{103}Just when she would grab this one, then her hands would move through it.
\snum{104}Then people could not see them.
\snum{105}It was their canoe however that was running around on the lake.
\pend
%7
\pstart
\snum{106}Their son however, they hoped that vomit should kill him.
\snum{107}That is why when his father dies, a boy, poverty kills him.
\snum{108}Having somehow become very intoxicated now, her little child, people made him live outside with his mother.
\snum{109}It is at the end of town that she made a brush house there.
\snum{110}Her little child, she was there with him.
\snum{111}She always bathed her little child in that brush house.
\snum{112}Gradually he gets very tall.
\snum{113}It is people’s garbage that people were always throwing on top of him.
\snum{114}\qqk{}“This one man who lives there” is what people say about him.
\snum{115}People always laugh at him.
\snum{116}At some point he ran out among the boys playing.
\snum{117}\qqk{}“Just yuck, Below Garbage Man” they called him.
\snum{118}He said to his mother “Make me a bow”.
\snum{119}So then having made it, just as it is getting bright, he would always go out with it shooting arrows.
\snum{120}It is everything that he would shoot.
\snum{121}Having gradually become a man, eventually he comes out at the shore of that little lake.
\pend
%7
\pstart
\snum{122}Having gone out there however many times, it ran up along below for him.
\snum{123}The inside of its mouth is red.
\snum{124}Having done so twice to him, he asked his mother
\snum{125}\qqk{}“What is it, mother?”
\snum{126}So then he made a new spear point.
\snum{127}\qqk{}“Be sure to run out to the water when it opens its mouth at you;
\snum{128}its claws always attack people.
\snum{129}It is your father’s canoe.”
\snum{130}So it was when he went there to the lakeshore that it opened its mouth at him.
\snum{131}“Just shoot the inside of its mouth.”
\snum{132}It was right when he shot it that there was heard “G̱áa!”, like a raven.
\snum{133}Like that, it happened as if they had been cut out, its thwarts.
\snum{134}Actually it was a canoe of copper;
\snum{135}they were wide, its thwarts.
\snum{136}Actually it was just that way being copper;
\snum{137}it broke apart, that canoe, all of it.
\snum{138}It was through the middle of the night that he packed it to his house, near to his mother.
\snum{139}No person knew about it.
\pend
%9
\pstart
\snum{140}So then he was building many houses from that copper.
\snum{141}Below the brush, it was there that he was always hammering some, for arrow points, for bracelets.
\snum{143}No iron used to exist, nor anything that is like copper.
\snum{143}He is also hammering (a thing) like a \fm{tináa}.
\snum{144}Then he is carrying it all inside.
\snum{145}Then people were throwing it on him, garbage.
\snum{146}“This hammering aristocrat.”
\snum{147}Having finished making that house and those coppers, there is already much of those things he hammers that way inside.
\snum{148}It was so that they would laugh at him that he would always run outside.
\snum{149}“Just yuck, Below Garbage Man.”
\pend
%10
\pstart
\snum{150}The daughter of the town leader, he is not willing for people to go to her possession.
\snum{151}People having tried to marry her from everywhere, he got ready.
\snum{152}Him however, it was at night that he dressed himself.
\snum{153}He took a twist of copper.
\snum{154}He knew the place where she sleeps, that aristocrat.
\snum{155}It was through a hole in the wall that he kept poking her with it, that girl, with that copper twist.
\snum{156}The girl caught it.
\snum{157}She smelled it.
\snum{158}She doesn’t know what it is for, that copper.
\snum{159}Which it was that nobody on earth had ever seen a thing of it, copper.
\snum{160}So then he summoned her “Come here, outside”.
\snum{161}She went out to him.
\snum{162}\qqk{}“Come with me now to my house.
\snum{163}You will be with me”
\snum{164}he said to her.
\snum{165}She thought that he was a man from somewhere;
\snum{166}she didn’t know him.
\snum{167}Being that man who people teased, it was to the beach that she was going along with him.
\snum{168}It is just as it is in front of her hand that it opens inside, that copper door;
\snum{169}it shone on her face inside.
\snum{170}It was things from wherever that he had lugged indoors, those \fm{tináa}.
\snum{171}So then he married her in his house.
\pend
%11
\pstart
\snum{172}People search for her, that girl.
\snum{173}People missed her, for however many days.
\snum{174}They overnight twice around there, that place where people search for her.
\snum{175}At some point her father said to a slave
\snum{176}“Best that you search around the beach.”
\snum{177}He searched there, that slave, near her.
\snum{178}It was just as he stuck his face inside there that the slave fell back.
\snum{179}It shone on his face.
\snum{180}From within that house, “Come inside” he said to him, that girl’s husband.
\snum{181}“Don’t tell about my house” he said to him then.
\snum{182}He finished saying to him however,
\snum{183}“Report that ‘Below Garbage Man married her’.”
\snum{184}Then it is when he came inside that he told about it.
\snum{185}“Below Garbage Man married her” he reports.
\snum{186}Right then they went out to fight.
\snum{187}Her mother says “My daughter!”.
\snum{188}Then they ran to the doorway.
\snum{189}They kicked in the house that sits against the brush house.
\snum{190}\qqk{}“Boom” it sounded.
\snum{191}It is upon her face that it shone.
\snum{192}They fled back outside from within that house.
\snum{193}Where is the feeling of anger for him?
\snum{194}They were ashamed.
\pend
%12
\pstart
\snum{195}Them having gone home, he sent people for him, his father-in-law.
\snum{196}Them having come to him, he put eight coppers on him because he married his daughter.
\snum{197}Then from around it people finally took it down, that former brush house.
\snum{198}It shone out, that copper.
\snum{199}But it was his father who had thought of him to supernaturally help him.
\snum{200}That is why men of today also, whenever they become poor, it supernaturally helps them.
\snum{201}That is why copper is so valuable over here, because of things happening there.
\snum{202}Just recently also it \{was worth\} two slaves, a copper.
\snum{203}It is always a thing that is significant, there.
\pend
\endnumbering
\end{Rightside}
\end{pairs}
\Columns

\vspace{1\baselineskip}

\section{Swanton’s abstract}\label{sec:89-swanton-abstract}

A woman was carried away by the grizzly-bear people, escaped, and impeded her pursuers by throwing small objects behind her which changed into great obstructions.
Finally she was taken up into the sun in a canoe and married the sun’s sons, who made way for her by killing their former cannibal wife above a Tsimshian town.
Therefore there are many cannibals among the Tsimshian.
At last the woman returned to her parents in a canoe which was a live grizzly-bear.
By and by her husbands became angry with her and left her.
Then she and her child lived in a brush house covered with filth, at one end of the town.
When he got larger her boy shot something in the lake which proved to be his father’s canoe, and pounded out all kinds of copper objects from the metal of which it was composed.
Then he married the daughter of the town chief and became a great man.

\section{Swanton’s translation}\label{sec:89-swanton-translation}

\snum{1}A chief lived in the middle of a very long town.
\snum{2}His daughter was fond of picking berries.
\snum{3}Once she went for berries with her father’s slaves,
\snum{4}and while picking far up in the woods she stepped upon some grizzly-bear’s dung.
\snum{6}\qqk{}“They always leave things under people’s feet, those wide anuses,”
\snum{5}she said.
\snum{7}When they wanted to go down her basket broke,
\snum{8}and her father’s slaves picked up the berries and put them back for her.
\snum{9}Very close to her father’s house it broke again.
\snum{10}Then one said to her
\snum{11}\qqk{}“Now pick them up yourself.”
\snum{12}While she was putting them in
\snum{13}a man came to her
\snum{14}whirling a stick in his hand.
\snum{15}\qqk{}“Let me marry you,” he said to her.
\snum{16}Then he started off with her.
\snum{17}He went up toward the woods with her and passed under two logs.
\snum{18}These things which looked like logs were mountains.

\snum{19}The people missed this woman.
\snum{20}For that the people were called together, and they searched everywhere for her.
\snum{21}It was the grizzly bear to which the high-caste woman had spoken angrily that married her.
\snum{22}The grizzly-bear people kept going after salmon.
\snum{23}After they had gone her husband went out after wet wood.
\snum{24}She, however, always collected dry wood.
\snum{25}When they came up from the salmon place they threw off their coats.
\snum{26}They shook them.
\snum{27}Something in these like grease would burn in the soaked wood.
\snum{28}The woman’s dry wood, however, always went out.
\snum{29}It was not long before they did something to the high-caste woman on account of it.

\snum{30}When they went out again,
\snum{31}the woman saw smoke right under her foot.
\snum{32}A grandmother mouse was coming out from under a little hill.
\snum{33}It was that which was going to help her.
\snum{34}\qqk{}“Come in, grandchild,” ⸢she said,⸣
\snum{35}\qqk{}“These are very dangerous animals you are among.
\snum{36}The grizzly-bear people have carried you away.”
\snum{37}She told her the truth.
\snum{38}Then she gave her advice. “Over there is your father’s home.”
\snum{39}So next morning when they were gone after salmon
\snum{40}she started running in the opposite direction.
\snum{41}When they came home at midday the grizzly-bear people missed her.
\snum{42}The woman’s dress had rotted up there.
\snum{43}After she had crossed one mountain she glanced behind her.
\snum{44}It looked dark with grizzly bears.
\snum{45}When they gained on her she began crying for her life.
\snum{46}She came out on the edge of a lake.
\snum{47}In the middle of this big lake a canoe was floating wearing a dance hat.
\snum{48}It said to her “Run this way into the water.”
\snum{49}Then she ran into the water toward it.
\snum{50}She was pulled in,
\snum{51}and it went up with her into the sun.

\snum{52}The sun’s sons had married a cannibal.
\snum{53}Whomsoever they married never lasted long before
\snum{54}they killed her.
\snum{55}Now, however, they liked the one they had just married.
\snum{56}To make way for her they killed the cannibal.
\snum{57}They killed her over a Tsimshian town.
\snum{58}They chopped her into very fine pieces.
\snum{59}This is why there came to be so many cannibals there.
\snum{60}They could see the Tsimshian town.
\snum{61}When the sun got straight up over her father’s town they said,
\snum{62}\qqk{}“Here is your father’s town.”
\snum{63}Very soon they had a child.
\snum{64}Their father’s canoe, a grizzly-bear canoe, stood at the end of this town.
\snum{65}The canoe could hear.
\snum{66}They loaded it with things.
\snum{67}They put grease inside of it for their father in law.
\snum{68}Then it walked away with them.
\snum{69}After it had walked on for a long time it would stop suddenly.
\snum{70}This was because it was hungry, and they would then break up a box of grease in front of the bow.
\snum{71}They came in front of their father-in-law’s house.
\snum{72}Then she recognized her father’s house,
\snum{73}and went up in front of it.
\snum{74}Then her brother came into the house and said,
\snum{75}\qqk{}“My sister has come and is outside.”
\snum{76}But his mother beat him because he claimed to see his sister who had been long dead.\snum{77}His mother went out.
\snum{78}It was indeed true, and they were coming ashore.
\snum{79}They did not see them (her husbands), however,
\snum{80}for they were like streaks of moonlight.
\snum{81}Now, after they had brought all their things up, one went out and said, \snum{82}\qqk{}“There is nothing there.”
\snum{83}The wife said,
\snum{84}\qqk{}“That moonlight down there is they.
\snum{85}Tell them to come up.”
\snum{86}So people went to tell them.
\snum{87}They came up.
\snum{88}Then the sunbeams lay alongside of the woman in streaks,
\snum{89}and their little son in front of them was also like a sunbeam.
\snum{90}After they were seated inside of the house they began to appear as if coming out of a fog.
\snum{91}\qqk{}“Eat something, my daughter,” said the chief.
\snum{92}Then a very young man ran to get water for them.
\snum{93}But her husband took a fishhawk’s quill out,
\snum{94}and put this into it.
\snum{95}If it bent over on account of the wet the man had not behaved himself.
\snum{96}After they had examined everyone she sent her little brother,
\snum{97}and her little brother always brought water for them.
\snum{98}When her brother went away she took her husband’s bucket for the water herself.
\snum{99}But after she had been twice a man near the water seized her hand.
\snum{100}And, when she brought it into the house and set it close beside her husbands, they put the fishhawk’s quill into it.
\snum{101}This time, after her hand had been caught, the quill bent over with slime.
\snum{102}Then they started to get up to go outside, away from her.
\snum{103}She would catch the first one and then the other, but her hands passed right through them.
\snum{104}Then they ceased to see them.
\snum{105}Their canoe, however, ran about on the lake.

\snum{106}After that the sun’s children began to wish that filth would kill their son.
\snum{107}This is why poverty always kills a little boy when his father dies.
\snum{108}After her little child had begun to suffer very much they compelled him to go outside with his mother.
\snum{109}She made a house with branches at the other end of the town.
\snum{110}There she stayed with her little child.
\snum{111}She continually bathed her little child inside of the house of branches,
\snum{112}and he grew larger there.
\snum{113}People kept throwing the leavings of food on top of their house.
\snum{114}They always called him “This man living here.”
\snum{115}They would laugh at him.
\snum{116}Whenever the little boy ran out among the boys who were playing
\snum{117}they said “Uh! Garbage-man.”
\snum{118}Now he said to his mother, “Make a bow and arrows for me.”
\snum{119}And, after she had made them, he went out shooting just at daybreak.
\snum{120}He shot all kinds of things.
\snum{121}When he was getting to be a man, he kept going up close by the lake.

\snum{122}After he had gone up there many times something came up quickly toward him.
\snum{123}Its mouth was red.
\snum{124}After it had done so twice he asked his mother,
\snum{125}\qqk{}“What is that, mother?”
\snum{126}Then he prepared a new spear.
\snum{127}\qqk{}“When it opens its mouth for you
\snum{128}and puts its forefeet up on land
\snum{127}run down to it.
\snum{129}It is your father’s canoe.”
\snum{130}So he went there and it opened its mouth for him. ⸢His mother had said,⸣ \snum{131}“Shoot it in the mouth,” and,
\snum{132}when he had shot it, it was heard to say “G̣a,” like a raven.
\snum{133}It was as if its seats had been all cut off.
\snum{134}It was a copper canoe
\snum{135}in which were wide seats.
\snum{136}The canoe was nothing but copper
\snum{137}and broke entirely up.
\snum{138}Throughout the night he carried it into his house to his mother.
\snum{139}No person knew of it.

\snum{140}Now he began making a big house out of copper.
\snum{141}He would pound out spears and bracelets under the branches.
\snum{142}In those days there was no iron or copper.
\snum{143}He also pounded out copper plates.
\snum{144}Then he set them all round the inside of the house.
\snum{145}When they threw garbage upon his house [they kept calling him] \snum{146}\qqk{}“Pounding-chief.”
\snum{147}After he had finished the house there were plenty of copper plates which he kept pounding out.
\snum{148}When they laughed at him and he ran outside they would say,
\snum{149}\qqk{}“Uh!
Garbage-man.”
\snum{150}There was a chief’s daughter whom they would let no one marry.
\snum{151}After people from all places had tried to get her he prepared himself.
\snum{152}He dressed himself at night.
\snum{153}He took a piece of twisted copper.
\snum{154}He knew where the chief’s daughter slept.
\snum{155}He poked the woman through a hole with this copper roll,
\snum{156}and the woman caught hold of it.
\snum{157}She smelt it.
\snum{158}She did not know what the copper was,
\snum{159}no person in the world having ever seen copper.
\snum{160}Then he called to her saying, “Come outside,”
\snum{161}and she went outside to him.
\snum{162}\qqk{}“Go down to my house with me.
\snum{163}With me you shall stay,”
\snum{164}he said to her.
\snum{165}She did not know whence the man came.
\snum{167}The man that used to be called dirty was only going to the beach with her.
\snum{168}Just before she touched the door it opened inward.
\snum{169}The copper door shone in her face.
\snum{170}Whence were all those coppers that stood around inside of the house?
\snum{171}Then he married her in his house.

\snum{172}By and by the people began searching for that woman.
\snum{173}They missed her for many days.
\snum{174}Two days were passed in searching for her.
\snum{175}Then her father said to a slave,
\snum{176}\qqk{}“Search below here.”
\snum{177}The slave searched there for her.
\snum{178}When he had looked into the house the slave backed out.
\snum{179}It began shining in his face.
\snum{180}Then the woman’s husband from inside the house said to him,
\snum{181}\qqk{}“Come in. Do not tell about my house,”
\snum{182}he said.
\snum{183}\qqk{}“Say Garbage-man has married her.”
\snum{184}When he came into the house he told about it.
\snum{185}He said, “Garbage-man has married her.”
\snum{186}Then they started to rush out.
\snum{187}Her mother cried “My daughter!”
\snum{188}Then they rushed to his door.
\snum{189}They kicked into the house, under the house made of branches.
\snum{190}\qqk{}“Dᴀm” it sounded.
\snum{191}It shone out into her face,
\snum{192}and they started back from the house door.
\snum{193}Where was their anger against him?
\snum{194}Then she became ashamed.
\snum{195}After they got home he sent for his father-in-law,
\snum{196}and he put eight coppers on him because he had married his daughter.
\snum{197}Then they threw the branch house away,
\snum{198}letting the copper shine out.
\snum{199}But his father had done this purposely to him in order to help him.
\snum{200}So even now, when a man is poor, something comes to help him.
\snum{201}This shows how valuable copper was at the place where this happened.
\snum{202}Even lately a copper plate used to cost two slaves.
\snum{203}It has since become an everlasting thing here (i.e., it is now used there all the time).

\section{Emmons’s version from Wrangell}\label{sec:89-emmons-version}

The version of the story below was recorded by George T.\ Emmons some time in the late 19th century from an unknown consultant in Wrangell.
It was edited and published as part of Frederica De Laguna’s collection of Emmons’s ethnographic writings, included in a section on the \fm{tináa} \parencite[180]{emmons:1991}.
The comments in [square brackets] are from De Laguna.

\begin{quote}
In early days in a more southern Tlingit village there was a widow and her son.
She was a malicious gossip and in her evil talk created much trouble that culminated in a conflict in which many were killed.
This led to her exile and she wandered far away with her son.
He was a hunter and supported them both.
One day before he went out, his mother told him that when he saw an animal he should shoot his arrow and call like a raven.
[This cry was common magic to avert evil.]
When he came to a lake, he saw a large fishlike monster coming toward him.
Shooting and calling as he had been told, he killed it.
Upon returning home, he told his mother he did not know what he had killed but that it looked like a huge shark and was covered with great scales or plates.
The mother told him to take these off and to keep them, as they would make him rich.
With his stone tools he cut down the middle of the monster’s back and removed all the heavy plates which he stacked like wood by the house.
When this was done, a visiting Tsimshian stopped, and, seeing the pile of shining plates, evidently of copper, bought one for twenty blankets, one mink skin, and one [sea] otter skin.
Later, others came to buy and the hunter became wealthy.
The plates were bright at first, but later turned almost black.
The plates were used as ornaments on or in the house [of the hunter?], which was given the name \fm{Ik-yatz hit} [\ipa{ʔiqỵé·s̓ hít} “dark copper house”], and the lineage became the \fm{Ik-yatz hit-tan} “Copper House Family”.
In some unexplained way the \fm{Ga-yatze hit-tan} [\ipa{g̣aỵé·s̓} “iron”, derived from \ipa{ʔiqỵé·s̓}], “Iron House Family”, was looked upon as the same line.
\end{quote}

The name that Emmons gives as “\fm{Ik-yatz hit}” is clearly \fm{Iḵyéisʼ Hít}.
The noun \fm{iḵyéisʼ} [\ipa{ʔìq.jéːsʼ}] is a conservative form of \fm{g̱ayéisʼ} [\ipa{qà.jéːsʼ}] ‘iron’ and is a compound of \fm{eeḵ} ‘copper’ and \fm{ÿéisʼ} ‘obsidian; black stone’ from the root \fm{\rt[¹]{ÿesʼ}} ‘dark, dusky; discolour, bruise; stain, dye’.
The form \fm{iḵyéisʼ} is attested primarily from Tongass and Southern Tlingit dialects as well as from Transitional Northern varieties; its use here reflects the Transitional Northern variety of Wrangell.

\clearpage
\section{Paragraph 1}\label{sec:89-para-1}

\ex\label{ex:89-1-long-town-chief}%
\exmn{252.1}%
\begingl
	\glpreamble	An kułayᴀ′t! dîgī′ỵīga a′ya u ānqā′wo. //
	\glpreamble	Aan kulayátʼ digeeÿeegé aya.óo aanḵáawu. //
	\gla	{} Aan \rlap{kulayátʼ} @ {} @ {} @ {} @ {} @ {} \rlap{digeeÿeegé} @ {} {}
		\rlap{aya.óo} @ {} @ {} @ {}
		{} \rlap{aanḵáawu.} @ {} @ {} {} //
	\glb	{} aan k- u- l- \rt[¹]{ÿatʼ} -μH {} digeeÿgé {} {}
		a- i- \rt[²]{.u} -μμH
		{} aan= ḵáaʷ -í {} //
	\glc	{}[\pr{PP} town \xx{cmpv}- \xx{irr}- \xx{xtn}- \rt[¹]{long} -\xx{var} \·\xx{nmz} middle \·\xx{loc} {}]
		\xx{arg}- \xx{stv}- \rt[²]{own} -\xx{var}
		{}[\pr{DP} land= man -\xx{pss} {}] //
	\gld	{} town \rlap{\xx{gcnj}.\xx{stv}·\xx{impfv}.long} {} {} {} {} -th middle -in {}
		\rlap{3>3.\xx{ncnj}.\xx{stv}·\xx{impfv}.live} {} {} {}
		{} \rlap{aristocrat} {} {} {} //
	\glft	‘In the middle of the length of a town lives an aristocrat.’
		//
\endgl
\xe

\ex\label{ex:89-2-daughter-berry-picking}%
\exmn{252.1}%
\begingl
	\glpreamble	Dusī′ qok!ī′t! akucîtᴀ′n. //
	\glpreamble	Du sée ḵukʼéetʼ aku̬shitán. //
	\gla	{} Du sée {}
		{} \rlap{ḵukʼéetʼ} @ {} @ {} {} {}
		\rlap{aku̬shitán.} @ {} @ {} @ {} @ {} @ {} @ {} //
	\glb	{} du sée {} 
		{} ḵu- \rt[²]{kʼitʼ} -μμH {} {}
		a- k- u- sh- i- \rt[²]{tan} -μH //
	\glc	{}[\pr{DP} \xx{3h·pss} daughter {}]
		{}[\pr{DP} \xx{areal}- \rt[²]{eat·up} -\xx{var} \·\xx{nmz} {}]
		\xx{arg}- \xx{qual}- \xx{irr}- \xx{pej}- \xx{stv}- \rt[²]{habit} -\xx{var} //
	\gld	{} his daughter {}
		{} \rlap{\xx{zcnj}.\xx{impfv}.berry·pick} {} {} -ing {}
		\rlap{3>3.\xx{g̱cnj}.\xx{stv}·\xx{impfv}.accustomed} {} {} {} {} {} {} //
	\glft	‘His daughter is wont to go berry picking.’
		//
\endgl
\xe

The verb \fm{aku̬shitán} [\ipa{ʔà.kʰʷù.ʃì.ʹtʰán}] is an instance of a full vowel which would be syncopated in modern Tlingit as \fm{akwshitán} [\ipa{ʔàkʷ.ʃì.ʹtʰán}].
\citeauthor{swanton:1909}’s transcription \orth{akucîtᴀ′n} could potentially be a mistake for ?\orth{akᵘcîtᴀ′n}, but similarly unsyncopated vowels occur in song lyrics and occasionally in other contexts.
It is thus plausible that \citeauthor{swanton:1909}’s transcription is accurate and that the amount of vowel syncopation in verb prefixes has increased since the early 20th century.
Crucially, if this was a mistake in reading \citeauthor{swanton:1909}’s transcription then we would expect other similar cases to be correctly transcribed, but these unsyncopated vowels are relatively common and consistent.

\ex\label{ex:89-3-berry-picking-with-dads-slaves}%
\exmn{252.2}%
\begingl
	\glpreamble	Qok!ī′t! ān ū′at duī′c guxq!ᵘ tîn. //
	\glpreamble	Ḵukʼéetʼ aan woo.aat, du éesh goox̱xʼú tin. //
	\gla	{} \rlap{Ḵukʼéetʼ} @ {} @ {} {} {}
		{} \rlap{aan} @ {} {}
		\rlap{woo.aat,} @ {} @ {} @ {} +
		{} du éesh \rlap{goox̱xʼú} @ {} @ {} tin. {} //
	\glb	{} ḵu- \rt[²]{kʼitʼ} -μμH {} {}
		{} á -n {}
		wu- i- \rt[¹]{.at} -μμL
		{} du éesh goox̱ -xʼ -í teen {} //
	\glc	{}[\pr{DP} \xx{areal}- \rt[²]{eat·up} -\xx{var} \·\xx{nmz} {}]
		{}[\pr{PP} \xx{3n} -\xx{instr} {}]
		\xx{pfv}- \xx{stv}- \rt[¹]{go·\xx{pl}} -\xx{var}
		{}[\pr{PP} \xx{3h·pss} father slave -\xx{pl} -\xx{pss} \xx{instr} {}] //
	\gld	{} \rlap{\xx{zcnj}.\xx{impfv}.berry·pick} {} {} -ing {}
		{} it -with {}
		\rlap{\xx{ncnj}.\xx{pfv}.they.go·\xx{pl}} {} {} {}
		{} her father slave -s -of with {} //
	\glft	‘She went berry picking with them, with her father’s slaves.’
		//
\endgl
\xe

\ex\label{ex:89-4-step-dung-picking-berries}%
\exmn{252.2}%
\begingl
	\glpreamble	Akā′yan kaoʟ̣îỵᴀ′s! yux̣ū′ts! hā′ʟ!î yudā′qq! qok!ī′t!ê. //
	\glpreamble	A káa yan kawdliÿásʼ yú xóots háatlʼi yú dáaḵxʼ ḵukʼéetʼi. //
	\gla	{} A \rlap{káa} @ {} {}
		yan @ \rlap{kawdliÿásʼ} @ {} @ {} @ {} @ {} @ {} @ {}
		{} yú xóots \rlap{háatlʼi} @ {} {} +
		{} {} yú \rlap{dáaḵxʼ} @ {} {}
			\rlap{ḵukʼéetʼi.} @ {} @ {} {} {} //
	\glb	{} a ká -μ {}
		ÿán= k- wu- d- l- i- \rt[¹]{ÿasʼ} -μH
		{} yú xóots háatlʼ -í {}
		{} {} yú dáaḵ -xʼ {} 
			ḵu- \rt[²]{kʼitʼ} -μμH -í {} //
	\glc	{}[\pr{PP} \xx{3n·pss} \xx{hsfc} -\xx{loc} {}]
		\xx{term}= \xx{hsfc}- \xx{pfv}- \xx{mid}- \xx{xtn}- \xx{stv}- \rt[¹]{mv·leg} -\xx{var}
		{}[\pr{DP} \xx{dist} br·bear dung -\xx{pss} {}]
		{}[\pr{CP} {}[\pr{PP} \xx{dist} inland -\xx{loc} {}]
			\xx{areal}- \rt[²]{eat·up} -\xx{var} -\xx{sub} {}] //
	\gld	{} its atop -on {}
		down \rlap{\xx{zcnj}.\xx{pfv}.step} {} {} {} {} {} {}
		{} that br·bear dung -of {}
		{} {} that inland -at {}
			\rlap{\xx{zcnj}.\xx{impfv}.berry·pick} {} {} -while {} //
	\glft	‘She stepped on it, that brown bear feces, while she is inland picking berries.’
		//
\endgl
\xe

The phrase \fm{yú dáaḵxʼ} ‘on that inland’ in (\lastx) is a rare example of the noun \fm{dáaḵ} ‘inland’ used as an ordinary noun.
Usually this noun is found as part of a compound like \fm{daḵ-ká} ‘inland-surface’ and as the preverb \fm{daaḵ=} ‘inland, up from shore’.
Its presence with both a determiner \fm{yú} and a postposition \fm{-xʼ} confirms that \fm{dáaḵ} is an ordinary alienable noun in this context.

\ex\label{ex:89-5-said-to-the-dung}%
\exmn{252.3}%
\begingl
	\glpreamble	Yē aỵa′osîqa yūx̣ū′ts! hā′ʟ!î, //
	\glpreamble	Yéi aÿawsiḵaa yú xóots háatlʼi //
	\gla	Yéi @ \rlap{aÿawsiḵaa} @ {} @ {} @ {} @ {} @ {} @ {}
		{} yú xóots \rlap{háatlʼi} @ {} {} //
	\glb	yéi a- ÿ- wu- s- i- \rt[¹]{ḵa} -μμL
		{} yú xóots háatlʼ -í {} //
	\glc	thus= \xx{arg}- \xx{qual}- \xx{pfv}- \xx{csv}- \xx{stv}- \rt[¹]{say} -\xx{var}
		{}[\pr{DP} \xx{dist} br·bear dung -\xx{pss} {}] //
	\gld	thus \rlap{3>3.\xx{ncnj}.\xx{pfv}.say·to} {} {} {} {} {} {}
		{} that br·bear dung -of {} //
	\glft	‘She said to the brown bear dung’
		//
\endgl
\xe

\ex\label{ex:89-6-basket-butts}%
\exmn{252.4}%
\begingl
	\glpreamble	“Ts!ᴀs qa′q!osi ỵidê′ hᴀs aʟīʟ! toq qᴀkᵘ.” //
	\glpreamble	«\!Tsʼas ḵaa x̱ʼusÿeedé has aléelʼ, tuḵḵákw.\!» //
	\gla	\rlap{«\!Tsʼas} @ {} 
		{} ḵaa \rlap{x̱ʼusÿeedé} @ {} @ {} {}
		has @ \rlap{alʼéelʼ,} @ {} @ {}
		{} \rlap{tuḵḵákw.\!»} @ {} {} //
	\glb	\pqp{}tsʼa =s
		{} ḵaa x̱ʼoos- ÿee -dé {}
		has= a- \rt[²]{lʼilʼ} -μμH
		{} tóoḵ- ḵákw {} //
	\glc	\pqp{}just =\xx{dub}
		{}[\pr{PP} \xx{4h·pss} foot- below -\xx{all} {}]
		\xx{plh}= \xx{xpl}- \rt[²]{defecate} -\xx{var}
		{}[\pr{DP} butt- basket {}] //
	\gld	\pqp{}just \•maybe
		{} one’s foot- below -at {}
		they \rlap{\xx{zcnj}.\xx{impfv}.defecate} {} {}
		{} butt- basket {} //
	\glft	‘“They just defecate under people’s feet, butt baskets.”’
		//
\endgl
\xe

\ex\label{ex:89-7-basket-broke}%
\exmn{252.5}%
\begingl
	\glpreamble	Ātxê′qdê hᴀs aỵᴀ′ daā′dawe ỵa′ołik!ūts dukᴀ′gu. //
	\glpreamble	Aatx̱ íḵde has aÿada.áat áwé ÿawlikʼoots du ḵágu. //
	\gla	{} {} \rlap{Aatx̱} @ {} {}
			{} \rlap{íḵde} @ {} {}
			has @ \rlap{aÿada.áat} @ {} @ {} @ {} @ {} @ {} @ {} {}
		\rlap{áwé} @ {}
		\rlap{ÿawlikʼoots} @ {} @ {} @ {} @ {} @ {}
		{} du \rlap{ḵágu.} @ {} {} //
	\glb	{} {} á -dáx̱ {}
			{} éeḵ -dé {}
			has= a- ÿ- {} d- \rt[¹]{.at} -μμH {} {}
		á -wé
		ÿ- wu- l- i- \rt[¹]{kʼuts} -μμL
		{} du ḵákw -í {} //
	\glc	{}[\pr{CP} {}[\pr{PP} \xx{3n} -\xx{abl} {}]
			{}[\pr{PP} beach -\xx{all} {}]
			\xx{plh}= \xx{xpl}- \xx{qual}- \xx{zcnj}\· \xx{mid}-
				\rt[¹]{go·\xx{pl}} -\xx{var} \·\xx{sub} {}]
		\xx{foc} -\xx{mdst}
		\xx{qual}- \xx{pfv}- \xx{xtn}- \xx{stv}- \rt[¹]{break·lf} -\xx{var}
		{}[\pr{DP} \xx{3h·pss} basket -\xx{pss} {}] //
	\gld	{} {} that -after {}
			{} beach -to {}
			they \rlap{back.\xx{csec}.go·\xx{pl}} {} {} {} {} {} {} {}
		\rlap{it.is} {}
		\rlap{\xx{ncnj}.\xx{pfv}.break·long·flexible} {} {} {} {} {}
		{} her \rlap{basket} {} {} //
	\glft	‘Later it was when they returned to the beach that it snapped, her basket.’
		//
\endgl
\xe

There are two notable features of (\lastx) that deserve some comment.
One is the the verb \fm{has aÿada.áat}.
The presence of \fm{d-} in this verb without a change in argument structure indicates that it is revertive, describing movement back to a previous location.
There are two kinds of revertive motion derivations: (i) the direct revertive with \fm{ḵux̱=…d-} (\fm{∅}; \fm{-ch} repetitive) ‘back along the same path’ and (ii) the perambulative revertive \fm{a-ÿ-…-d-} (\fm{∅}; \fm{-x̱} repetitive) ‘back along a different path’.
The verb \fm{has aÿada.áat} in (\lastx) uses the perambulative revertive.
This means that, although \fm{a-} in a motion verb frequently represents the fourth person (indefinite) human subject ‘someone, people’, in this particular case the \fm{a-} is part of the revertive motion derivation and does not code for the subject.
Instead the subject is covert with the modifier \fm{has=} making it plural.

The other notable feature of (\lastx) is the verb \fm{ÿawlikʼoots} ‘it broke’.
The root \fm{\rt[¹]{kʼuts}} ‘break, snap’ specifically refers to the breaking of a long, flexible object \parencite[780]{leer:1976}.
A typical verb based on this root is \fm{wookʼoots} (\fm{n}; achievement) ‘it (long, flexible) snapped, broke’.
This root implies that it is not the basket itself which broke, but rather the neck strap of the basket hung around the woman’s neck.
This detail has been left out of the English translation because it is difficult to render concisely without explicitly mentioning a strap; since there is no noun phrase in the original Tlingit form that describes a strap, including it in the English translation would be misleading.
The \fm{l-} in \fm{ÿawlikʼoots} is the extensional which reflects the strap being extended through space.
The \fm{ÿ-} qualifier in \fm{ÿawlikʼoots} probably contributes a meaning of ‘obliquely, off at an angle’ similar to its function in motion derivations like \fm{NP-x̱ ÿa-u-} (\fm{∅}; \fm{-ch} repetitive) ‘obliquely, circuitously, indirectly along NP’ and the perambulative revertive mentioned above.
The meaning of \fm{ÿawlikʼoots} is then ‘it (long, flexible thing) snapped off at an angle extended through space’ which gives an image of the strap breaking apart at a point with the ends flying off.

\ex\label{ex:89-8-slaves-picked-up}%
\exmn{252.5}%
\begingl
	\glpreamble	Duī′c guxq!ū′tcawe ỵᴀsahē′x akā′dê du djiỵîˈs. //
	\glpreamble	Du éesh goox̱xʼúch áwé ÿasahéix̱ a kaadé du jeeÿís. //
	\gla	{} Du éesh \rlap{goox̱xʼúch} @ {} @ {} @ {} {}
		\rlap{áwé} @ {}
		\rlap{ÿasahéix̱} @ {} @ {} @ {} @ {} @ {} +
		{} a \rlap{kaadé} @ {} {}
		{} du \rlap{jeeÿís.} @ {} {} //
	\glb	{} du éesh goox̱ -xʼ -í -ch {}
		á -wé
		ⱥ- ÿ- s- \rt[²]{ha} -eμH -x̱
		{} a ká -dé {}
		{} du jee =ÿís {} //
	\glc	{}[\pr{DP} \xx{3h·pss} father slave -\xx{pl} -\xx{pss} -\xx{erg} {}]
		\xx{foc} -\xx{mdst}
		\xx{arg}- \xx{qual}- \xx{xtn}- \rt[²]{mv·mass} -\xx{var} -\xx{rep}
		{}[\pr{PP} \xx{3n·pss} \xx{hsfc} -\xx{all} {}]
		{}[\pr{PP} \xx{3h·pss} poss’n -\xx{ben} {}] //
	\gld	{} her father slave -s -of {} {}
		\rlap{it.is} {}
		\rlap{3>3.\xx{zcnj}.\xx{impfv}.gather·up.\xx{rep}} {} {} {} {} {}
		{} its atop -to {}
		{} her poss’n -for {} //
	\glft	‘It is her father’s slaves that are gathering it up there for her.’
		//
\endgl
\xe

\ex\label{ex:89-9-broke-again}%
\exmn{252.6}%
\begingl
	\glpreamble	ʟᴀxdê′ yā′duīc nełixᴀ′n-q!awe ts!u ỵa′olik!ūts. //
	\glpreamble	Tlax̱ dé yá du éesh neilí x̱ánxʼ áwé tsu ÿawlikʼoots. //
	\gla	Tlax̱ de
		{} yá du éesh \rlap{neilí} @ {} \rlap{x̱ánxʼ} @ {} {}
		\rlap{áwé} @ {} +
		tsu \rlap{ÿawlikʼoots.} @ {} @ {} @ {} @ {} @ {} //
	\glb	tlax̱ de
		{} yá du éesh neil -í x̱án -xʼ {}
		á -wé
		tsu ÿ- wu- l- i- \rt[¹]{kʼuts} -μμL //
	\glc	very now
		{}[\pr{PP} \xx{prox} \xx{3h·pss} father home -\xx{pss} near -\xx{loc} {}]
		\xx{foc} -\xx{mdst}
		again \xx{qual}- \xx{pfv}- \xx{xtn}- \xx{stv}- \rt[¹]{break·lf} -\xx{var} //
	\gld	very now
		{} this her father’s home {} near -at {}
		\rlap{it.is} {}
		again \rlap{\xx{ncnj}.\xx{pfv}.break·long·flex} {} {} {} {} {} //
	\glft	‘It is just right near to her father’s home that it broke again.’
		//
\endgl
\xe

\ex\label{ex:89-10-so-he-said-to-her}%
\exmn{252.7}%
\begingl
	\glpreamble	Tc!uʟe′ yē aỵa′osîqa //
	\glpreamble	Chʼu tle yéi aÿawsiḵaa //
	\gla	Chʼu tle yéi @ \rlap{aÿawsiḵaa} @ {} @ {} @ {} @ {} @ {} @ {} //
	\glb	chʼu tle yéi= a- ÿ- wu- s- i- \rt[¹]{ḵa} -μμL //
	\glc	just then thus= \xx{arg}- \xx{qual}- \xx{pfv}- \xx{csv}- \xx{stv}- \rt[¹]{say} -\xx{var} //
	\gld	just then thus \rlap{3>3.\xx{ncnj}.\xx{pfv}.say·to} {} {} {} {} {} {} //
	\glft	‘So then one said to her’
		//
\endgl
\xe

The referent of the subject in (\lastx) is not stated explicitly in the Tlingit sentence.
\citeauthor{swanton:1909} clarifies this in his gloss “he (i.e.\ one) said to her” and his translation also makes this explicit: “Then one said to her”.
Thus the referent of the subject in (\lastx) is therefore one of the father’s slaves last mentioned in (\ref{ex:89-8-slaves-picked-up}) and crucially is not the father mentioned in (\ref{ex:89-9-broke-again}).
The English translation for (\lastx) should most accurately be either “he said to her” or “she said to her”, but this requires selecting a gender for the subject that cannot be determined from the Tlingit original.
To avoid ascribing an arbitrary gender, \citeauthor{swanton:1909}’s technique of using the non-gendered ‘one’ is maintained.

\ex\label{ex:89-11-you-gather-it-up}%
\exmn{252.7}%
\begingl
	\glpreamble	“Tc!a wae′tc dē′ ỵᴀsaha′.” //
	\glpreamble	«\!Chʼa wa.éich déi ÿasahá.\!» //
	\gla	«\!Chʼa {} \rlap{wa.éich} @ {} {} déi
		\rlap{ÿasahá.\!»} @ {} @ {} @ {} @ {} @ {} //
	\glb	\pqp{}chʼa {} wa.é -ch {} dé
		ÿ- {} {} s- \rt[²]{ha} -μH //
	\glc	\pqp{}just {}[\pr{DP} \xx{2sg} -\xx{erg} {}] now
		\xx{qual}- \xx{zcnj}\· \xx{2sg}\· \xx{xtn}- \rt[²]{mv·mass} -\xx{var} //
	\gld	\pqp{}just {} \rlap{you·\xx{sg}} {} {} now
		\rlap{\xx{imp}.you·\xx{sg}.gather·up} {} {} {} {} {} //
	\glft	‘“Now just you gather it up.”’
		//
\endgl
\xe

\ex\label{ex:89-12-working-alone}%
\exmn{252.8}%
\begingl
	\glpreamble	Akā′dê tc!a ʟē′ nᴀ′xawe de ᴀt a′na //
	\glpreamble	A kaadé chʼa tléináx̱ áwé de yéi adaané; //
	\gla	{} A \rlap{kaadé} @ {} {}
		{} chʼa \rlap{tléináx̱} @ {} {}
		\rlap{áwé} @ {}
		de
		yéi @ \rlap{adaané;} @ {} @ {} @ {} //
	\glb	{} a ká -dé {}
		{} chʼa tléixʼ -náx̱ {}
		á -wé
		de
		yéi= a- daa- \rt[²]{ne} -μH //
	\glc	{}[\pr{PP} \xx{3n} \xx{hsfc} -\xx{all} {}]
		{}[\pr{AdvP} just one -\xx{hum} {}]
		\xx{foc} -\xx{mdst}
		now
		thus= \xx{arg}- around- \rt[²]{work} -\xx{var} //
	\gld	{} its atop -to {}
		{} just \rlap{alone} {} {}
		\rlap{it.is} {}
		now
		thus \rlap{3>3.\xx{ncnj}.\xx{impfv}.work·on} {} {} {} //
	\glft	‘So there it is alone now that she is working on it,’
		//
\endgl
\xe

\ex\label{ex:89-13-came-near-to-her}%
\exmn{252.8}%
\begingl
	\glpreamble	doxᴀ′nt ū′wagut yuqā′ //
	\glpreamble	du x̱ánt uwagút yú ḵáa; //
	\gla	{} du \rlap{x̱ánt} @ {} {}
		\rlap{uwagút} @ {} @ {} @ {}
		{} yú ḵáa; {} //
	\glb	{} du x̱án -t {}
		u- i- \rt[¹]{gut} -μH
		{} yú ḵáaʷ {} //
	\glc	{}[\pr{PP} \xx{3h·pss} near -\xx{pnct} {}]
		\xx{zpfv}- \xx{stv}- \rt[¹]{go·\xx{sg}} -\xx{var}
		{}[\pr{DP} \xx{dist} man {}] //
	\gld	{} her near -to {}
		\rlap{\xx{zcnj}.\xx{pfv}.go·\xx{sg}} {} {} {}
		{} that man {} //
	\glft	‘and he came near to her, that man,’
		//
\endgl
\xe

\ex\label{ex:89-14-waving-a-bush}%
\exmn{252.9}%
\begingl
	\glpreamble	wᴀs!-ya acakᴀ′nᴀłỵên. //
	\glpreamble	wásʼ yaa ashakanalÿén.  //
	\gla	{} wásʼ {}
		yaa @ \rlap{ashakanalÿén.} @ {} @ {} @ {} @ {} @ {} @ {} //
	\glb	{} wásʼ {}
		ÿaa= a- sha- k- n- l- \rt[²]{ÿen} -μH //
	\glc	{}[\pr{DP} bush {}]
		along= \xx{arg}- head- \xx{qual}- \xx{ncnj}- \xx{xtn}- \rt[²]{shake} -\xx{var} //
	\gld	{} bush {}
		along \rlap{3>3.\xx{zcnj}.\xx{prog}.wave} {} {} {} {} {} {} //
	\glft	‘and he is waving a bush.’
		//
\endgl
\xe

The material in (\ref{ex:89-12-working-alone})–(\ref{ex:89-14-waving-a-bush}) is given by \citeauthor{swanton:1909} as a single sentence.
Each of the three units appears however to be a main clause with no indication of embedding.
They have been represented here as a single long sentence with three paratactically conjoined main clauses separated by semicolons.
They are translated with overt conjunctions (‘and’) because this provides something of the feel of a long continuous utterance; without conjunctions they would seem like completely separate sentences in English.

\ex\label{ex:89-15-lemme-marry-you-he-says}%
\exmn{252.9}%
\begingl
	\glpreamble	“Iqâca′” ʟe yū′ᴀcia′osîqa. //
	\glpreamble	«\!Iḵashaa\!» tle yóo ash yawsiḵaa. //
	\gla	{} \llap{«\!}\rlap{Iḵashaa\!»} @ {} @ {} @ {} @ {} @ {} @ {} {}
		tle yóo @ ash @ \rlap{yawsiḵaa.} @ {} @ {} @ {} @ {} @ {} //
	\glb	{} i- {} g̱- x̱- \rt[²]{sha} -μμL {} {}
		tle yóo= ash= ÿ- wu- s- i- \rt[¹]{ḵa} -μμL //
	\glc	{}[\pr{CP} \xx{2sg·o}- \xx{zcnj}\· \xx{mod}- \xx{1sg·s}-
			\rt[²]{woman} -\xx{var} \·\xx{sub} {}]
		then \xx{quot}= \xx{3prx·o}= \xx{qual}- \xx{pfv}- \xx{csv}- \xx{stv}-
			\rt[²]{say} -\xx{var} //
	\gld	{} \rlap{you·\xx{sg}.\xx{hort}.I.marry} {} {} {} {} {} {} {}
		then \xx{quot} her \rlap{\xx{ncnj}./\xx{pfv}.say·to} {} {} {} {} {} //
	\glft	‘“Let me marry you” he then says to her.’
		//
\endgl
\xe

\ex\label{ex:89-16-he-started-off-with-her}%
\exmn{252.10}%
\begingl
	\glpreamble	Tc!uʟe′ ᴀcī′n g̣one′ uwaᴀ′t. //
	\glpreamble	Chʼu tle ash een g̱unéi uwa.át. //
	\gla	Chʼu tle {} ash \rlap{een} @ {} {}
		g̱unéi @ \rlap{uwa.át.} @ {} @ {} @ {} //
	\glb	chʼu tle {} ash ee -n {}
		g̱unaÿéi= u- i- \rt[¹]{.at} -μH //
	\glc	just then {}[\pr{PP} \xx{3prx} \xx{base} -\xx{instr} {}]
		\xx{incep}= \xx{zpfv}- \xx{stv}- \rt[¹]{go·\xx{pl}} -\xx{var} //
	\gld	just then {} her \rlap{with} {} {}
		begin \rlap{\xx{zcnj}.\xx{pfv}.go·\xx{pl}} {} {} {} //
	\glft	‘So then he started to go with her.’
		//
\endgl
\xe

Note the use of the plural \fm{\rt[¹]{.at}} ‘plural go’ in (\lastx) and not the singular \fm{\rt[¹]{gut}} ‘singular go’.
This means that the man does not pick up the woman and walk off with her, but rather that the woman and the man both walk independently together.
An interesting semantic consequence of this is that the subject and \fm{ash} are partly coreferential.

\ex\label{ex:89-17-went-off-inland}%
\exmn{252.10}%
\begingl
	\glpreamble	Dᴀq datcū′n ᴀsiyu′ dēx xao taỵinᴀ′x ᴀcī′n ỵā′waᴀt. //
	\glpreamble	Daḵdachóon ásíyú déix̱ x̱áaw taÿeenáx̱ ash een ÿaawa.át. //
	\gla	{} \rlap{Daḵdachóon} @ {} {}
		\rlap{ásíyú} @ {} @ {}
		{} déix̱ x̱áaw \rlap{taÿeenáx̱} @ {} {}
		{} ash \rlap{een} @ {} {}
		\rlap{ÿaawa.át.} @ {} @ {} @ {} @ {} //
	\glb	{} dáaḵ- dachóon {}
		á -sí -yú
		{} déix̱ x̱áaw taÿee -náx̱ {}
		{} ash ee -n {}
		ÿ- wu- i- \rt[¹]{.at} -μH //
	\glc	{}[\pr{AdvP} inland- directly {}]
		\xx{foc} -\xx{dub} -\xx{dist}
		{}[\pr{PP} two log below -\xx{perl} {}]
		{}[\pr{PP} \xx{3prx} \xx{base} -\xx{instr} {}]
		\xx{qual}- \xx{pfv}- \xx{stv}- \rt[¹]{go·\xx{pl}} -\xx{var} //
	\gld	{} inland- directly {}
		\rlap{it.is.maybe} {} {}
		{} two log below -thru {}
		{} her \rlap{with} {} {}
		\rlap{off.\xx{zcnj}.\xx{pfv}.go·\xx{pl}} {} {} {} {} //
	\glft	‘It seems that it is directly inland where he goes off with her below two logs.’
		//
\endgl
\xe

\ex\label{ex:89-18-actually-mountains}%
\exmn{252.11}%
\begingl
	\glpreamble	Xᴀtc cā′ayu xao yᴀx ᴀc tuwā′ỵatî. //
	\glpreamble	X̱ách shaa áyú x̱áaw yáx̱ ash tuwáa ÿatee. //
	\gla	X̱ách {} shaa {} \rlap{áyú} @ {}
		{} x̱áaw yáx̱ {}
		{} ash \rlap{tuwáa} @ {} @ {} {}
		\rlap{ÿatee.} @ {} @ {} //
	\glb	x̱ách {} shaa {} á -yú
		{} x̱áaw yáx̱ {}
		{} ash tú- ÿá -μ {}
		i- \rt[¹]{tiʰ} -μμL //
	\glc	actually {}[\pr{DP} mountain {}] \xx{foc} -\xx{dist}
		{}[\pr{PP} log \xx{sim} {}]
		{}[\pr{PP} \xx{3prx} mind- face -\xx{loc} {}]
		\xx{stv}- \rt[¹]{be} -\xx{var} //
	\gld	actually {} mountain {} \rlap{it.is} {}
		{} log like {}
		{} her mind- face -at {}
		\rlap{\xx{ncnj}.\xx{stv}·\xx{impfv}.be} {} {} //
	\glft	‘Actually it is mountains that are like logs to her mind.’
		//
\endgl
\xe

\section{Paragraph 2}

\ex\label{ex:89-19-searching-for-girl}%
\exmn{253.1}%
\begingl
	\glpreamble	Duītē′x qoỵa′odū′waci yucawᴀ′t yū′antqenītc. //
	\glpreamble	Du eetéex̱ ḵuÿawduwashée yú shaawát, yú aantḵeiních. //
	\gla	{} Du \rlap{eetéex̱} @ {} {}
		\rlap{ḵuÿawduwashée} @ {} @ {} @ {} @ {} @ {} @ {}
		{} yú \rlap{shaawát,} @ {} {} +
		{} yú \rlap{aantḵeiních.} @ {} @ {} @ {} @ {} @ {} @ {} {} //
	\glb	{} du eetí -x̱ {}
		ḵu- ÿ- wu- du- i- \rt[²]{shi} -μμH
		{} yú sháaʷ- ÿát {}
		{} yú aan- d- \rt[¹]{ḵi} -μμL -n -í -ch {} //
	\glc	{}[\pr{PP} \xx{3h·pss} remains -\xx{pert} {}]
		\xx{areal}- \xx{qual}- \xx{pfv}- \xx{4h·s}-- \xx{stv}- \rt[²]{search} -\xx{var}
		{}[\pr{DP} \xx{dist} woman- child {}]
		{}[\pr{DP} \xx{d}·\xx{dist} town- \xx{mid}- \rt[¹]{sit·\xx{pl}} -\xx{var} -\xx{nsfx} -\xx{nmz} -\xx{erg} {}] //
	\gld	{} her place -of {}
		\rlap{\xx{zcnj}.\xx{pfv}.people.search} {} {} {} {} {} {}
		{} that \rlap{girl} {} {}
		{} those \rlap{townspeople} {} {} {} {} {} {} {} //
	\glft	‘They were searching after her, that girl, those townspeople.’
		//
\endgl
\xe

The sentence in (\lastx) is a good example of two given DPs dislocated to the right periphery.
The girl \fm{yú shaawát} is given because she has been under discussion since the beginning of the narrative.
The townspeople \fm{yú aantḵeiní} are implicitly given because they are the people who would be expected to search for her after her absence.
The PP \fm{du eetéex̱} would be \fm{yú shaawát eetéex̱} if the right dislocation of \fm{yú shaawát} was undone, and so the whole sentence without right dislocation would be \fm{yú aantḵeiních yú shaawát eetéex̱ ḵuÿawduwashée}.

\ex\label{ex:89-20-done-searching-talked}%
\exmn{253.1}%
\begingl
	\glpreamble	Yên yu′qodūciawa duite′q! yug̣ā′ wuduwatᴀ′n. //
	\glpreamble	Yan yóo ḵudushée áwé du eetéexʼ yoo x̱ʼawu̬duwatán. //
	\gla	{} Yan @ yoo @ \rlap{ḵudushée} @ {} @ {} @ {} @ {} @ {} {}
		\rlap{áwé} @ {}
		{} du \rlap{eetéexʼ} @ {} {}
		yoo @ \rlap{x̱ʼawu̬duwatán.} @ {} @ {} @ {} @ {} @ {} @ {} //
	\glb	{} ÿán= yoo= ḵu- {} du- \rt[²]{shi} -μμH {} {}
		á -wé
		{} du eetí -xʼ {}
		yoo= x̱ʼe- wu- du- {} i- \rt[²]{tan} -μH //
	\glc	{}[\pr{CP} \xx{term}= \xx{alt}= \xx{areal}- \xx{zcnj}\· \xx{4h·s}- \rt[²]{search} -\xx{var} \·\xx{sub} {}]
		\xx{foc} -\xx{mdst}
		{}[\pr{PP} \xx{3h·pss} remains -\xx{loc} {}]
		\xx{alt}= mouth- \xx{pfv}- \xx{4h·s}- \xx{mid}- \xx{stv}- \rt[²]{hdl·w/e} -\xx{var} //
	\gld	{} done \xx{alt} \rlap{people.\xx{csec}.search} {} {} {} {} {} {}
		\rlap{it.is} {}
		{} her place -at {}
		\xx{alt} \rlap{\xx{zcnj}.\xx{pfv}.people.speak} {} {} {} {} {} {} //
	\glft	‘Having finished searching around, they talked in her absence.’
		//
\endgl
\xe

\citeauthor{swanton:1909}’s transcription \orth{yug̣ā′ wuduwatᴀ′n} in (\lastx) suggests something like \fm{yóog̱aa wuduwatán} as indeed \citeauthor{leer:1977} interpreted it \parencite[2]{leer:1977}.
This is nonsensical because it means something like ‘people carried it (empty or wooden object) for/near that far away place’ which is incongruous in this context.
\citeauthor{swanton:1909}’s gloss “for that the people were called” and his translation “For that the people were called together” suggest that the verb instead involves speech and so should be something like \fm{x̱ʼawduwatán} ‘people spoke’.
This leaves the \orth{yu} as likely being the alternating preverb \fm{yoo} ‘back and forth, up and down, to and fro’.
The original pronunciation may have been something like [\ipa{ˌ\!jùː ˌχʼàˑ.wù.ˌtù.wà.ˈtán}] with the unusually preserved vowel in [\ipa{wù}] causing secondary stress to appear on [\ipa{χʼàˑ}] which in turn lengthened its vowel somewhat. 

\ex\label{ex:89-21-actually-brown-bear}%
\exmn{253.2}%
\begingl
	\glpreamble	Xᴀtc x̣ūts! qoa′nî ᴀsiyu′ ᴀcū′waca yua′xk!ᴀnya-ka′oʟ̣îgᴀdî yuānỵê′dî. //
	\glpreamble	X̱ách xóots ḵwáani ásíyú ash uwasháa, yú áx̱ x̱ʼanyaa ḵuwdlig̱ádi, yú aanÿádi. //
	\gla	X̱ách
		{} xóots \rlap{ḵwáani} @ {} {}
		\rlap{ásíyú} @ {} @ {}
		ash @ \rlap{uwasháa,} @ {} @ {} @ {} +
		{} yú {} {} \rlap{áx̱} @ {} {}
			{} \rlap{x̱ʼanyaa} @ {} {}
			\rlap{ḵuwdlig̱ádi,} @ {} @ {} @ {} @ {} @ {} @ {} @ {} {} {} +
		{} yú \rlap{aanÿádi.} @ {} @ {} {} //
	\glb	x̱ách
		{} xóots ḵwáan -í {}
		á -sí -yú
		ash= u- i- \rt[²]{shaʷ} -μμH
		{} yú {} {} á -x̱ {}
			{} x̱ʼé- niÿaa {}
			ḵu- wu- d- l- i- \rt[¹]{g̱at} -μH -í {} {}
		{} yú aan- ÿát -í {} //
	\glc	actually
		{}[\pr{DP} br·bear people -\xx{pss} {}]
		\xx{foc} -\xx{dub} -\xx{dist}
		\xx{3prx·o}= \xx{zpfv}- \xx{stv}- \rt[²]{woman} -\xx{var}
		{}[\pr{DP} \xx{dist} {}[\pr{CP} {}[\pr{PP} \xx{3n} -\xx{pert} {}]
			{}[\pr{NP} mouth- direction {}]
			\xx{areal}- \xx{pfv}- \xx{mid}- \xx{xtn}- \xx{stv}- \rt[¹]{wander} -\xx{var} -\xx{nmz} {}] {}]
		{}[\pr{DP} \xx{dist} town- child -\xx{pss} {}] //
	\gld	actually
		{} br·bear people -of {}
		\rlap{apparently.it.is} {} {}
		her \rlap{\xx{zcnj}.\xx{pfv}.marry} {} {} {}
		{} that {} {} it -of {}
			{} mouth- direction {}
			\rlap{\xx{zcnj}.\xx{pfv}.offend} {} {} {} {} {} {} -which {} {}
		{} that \rlap{aristocrat} {} {} {} //
	\glft	‘Apparently it was actually the brown bear that married her, that which she offended, that aristocrat.’
		//
\endgl
\xe

The phrase \fm{yú áx̱ x̱ʼanyaa ḵuwdlig̱ádi} in (\lastx) is a nominalized clause which has been translated as a demonstrative-headed relative clause ‘that which she offended’.
Although nominalized clauses and subordinate clauses share morphology in Tlingit, this must be a nominalized clause because it is preceded by the distal determiner \fm{yú} ‘the, that’.
This nominalized clause is coreferential with the subject of the verb \fm{ash uwasháa} ‘he married her’.
The following DP \fm{yú aanÿádi} could plausibly be interpreted as coreferential to the subject in which case it would entail that the brown bear who married her is an aristocrat.
But because the girl is described in (\ref{ex:89-2-daughter-berry-picking}) as the daughter of an aristocratic leader (\fm{aanḵáawu}) she is more likely to be the referent of the DP \fm{yú aanÿádi} here.
This interpretation is supported by \citeauthor{swanton:1909}’s gloss “the high-caste girl” and his translation “the high-caste woman”.

The verb \fm{ash uwasháa} ‘he married her’ in (\lastx) is based on the root \fm{\rt[²]{shaʷ}} ‘marry’.
This root is homophonous with \fm{\rt{shaʷ}} ‘woman’ found in e.g.\ \fm{sháa} ‘woman’ and \fm{shaawát} ‘girl, woman’, and in fact the two roots can be considered identical.
The occult labialization – cf.\ \fm{ḵáaʷ} ‘man’, \fm{naaʷ} ‘corpse’ – is one key sign that these two roots are actually the same root.
A second piece of evidence supporting the identity of the two roots here is that the social gender of the arguments of \fm{aawasháa} ‘he married her’ is fixed: if one argument is male and the other female then the subject must be the male and the object the female.
Thus \fm{ḵáach shaawát aawasháa} ‘a man married a woman’ is felicitous while \fm{shaawátch ḵáa aawasháa} ‘a woman married a man’ is normally infelicitous.
Although social practices around marriage are currently changing in the Tlingit community, the language has apparently not yet adapted to these changes.

\ex\label{ex:89-22-go-for-salmon}%
\exmn{253.3}%
\begingl
	\glpreamble	Xāt g̣ā naᴀdî′ naᴀ′ttc yux̣ū′ts! qoạ′nî. //
	\glpreamble	X̱áatg̱aa na.aadi na.átch yú xóots ḵwáani. //
	\gla	{} {} \rlap{X̱áatg̱aa} @ {} {} 
			\rlap{na.aadí} @ {} @ {} @ {} {}
		\rlap{na.átch} @ {} @ {} @ {} +
		{} yú xóots \rlap{ḵwáani.} @ {} {} //
	\glb	{} {} x̱áat -g̱áa {}
			n- \rt[¹]{.at} -μμL -í {}
		n- \rt[¹]{.at} -μH -ch
		{} yú xóots ḵwáan -í {} //
	\glc	{}[\pr{NP} {}[\pr{PP} salmon -\xx{ades} {}]
			\xx{ncnj}- \rt[¹]{go·\xx{pl}} -\xx{var} -\xx{nmz} {}]
		\xx{ncnj}- \rt[¹]{go·\xx{pl}} -\xx{var} -\xx{rep}
		{}[\pr{DP} \xx{dist} br·bear people -\xx{pss} {}] //
	\gld	{} {} salmon -for {}
			\rlap{\xx{ncnj}.\xx{prog}.go·\xx{pl}} {} {} -ing {}
		\rlap{\xx{hab}.go·\xx{pl}} {} {} {}
		{} those br·bear people -of {} //
	\glft	‘They always went going for salmon, those brown bear people.’
		//
\endgl
\xe

The phrase \fm{x̱áatg̱aa na.aadí na.átch} ‘they always went going for salmon’ in (\lastx) is a nominalized clause \fm{x̱áatg̱aa na.aadí} ‘going for salmon’ as the adjunct of the main verb \fm{na.átch} ‘they always go’.
See (\ref{ex:89-39-going-for-salmon-morning}) for further discussion.

\ex\label{ex:89-23-go-for-wet-wood}%
\exmn{253.4}%
\begingl
	\glpreamble	Yuxā′t g̣a naᴀ′dî itī′q!awe hīn tākᵘcā′gê yᴀdanē′nutc. //
	\glpreamble	Yú x̱áatg̱aa na.ádi eetéexʼ áwé héen táakw shaag̱í yéi adaanéi nuch. //
	\gla	{} {} Yú {} {} \rlap{x̱áatg̱aa} @ {} {}
				\rlap{na.ádi} @ {} @ {} @ {} {} {}
			\rlap{eetéexʼ} @ {} {} 
		\rlap{áwé} @ {} +
		{} héen táak \rlap{shaag̱í} @ {} {}
		yéi @ \rlap{adaanéi} @ {} @ {} @ {} @ \•nuch //
	\glb	{} {} yú {} {} x̱áat -g̱áa {} n- \rt[¹]{.at} -μH -í {} {} eetée -xʼ {}
		á -wé
		{} héen táaᵏ shaaḵ -í {}
		yéi= a- daa- \rt[²]{ne} -μμH =nooch //
	\glc	{}[\pr{PP} {}[\pr{DP} \xx{dist} {}[\pr{NP} {}[\pr{PP} salmon -\xx{ades} {}]
				\xx{ncnj}- \rt[¹]{go·\xx{pl}} -\xx{var} -\xx{nmz} {}] {}]
			remains -\xx{loc} {}]
		\xx{foc} -\xx{mdst}
		{}[\pr{DP} water below driftlog -\xx{pss} {}]
		thus= \xx{arg}- around- \rt[¹]{work} -\xx{var} =\xx{hab·aux} //
	\gld	{} {} that {} {} salmon -for {}
				\rlap{\xx{ncnj}.\xx{prog}.go·\xx{pl}} {} {} -ing {} {}
			after -at {}
		\rlap{it.is} {}
		{} water below driftlog -of {}
		thus \rlap{3>3.\xx{ncnj}.\xx{impfv}.work·on} {} {} {} \•always //
	\glft	‘It is after that going for salmon that he always gets driftlogs below the water.’
		//
\endgl
\xe

\ex\label{ex:89-24-break-dry-wood}%
\exmn{253.5}%
\begingl
	\glpreamble	Ho′ qo′a ts!ᴀs x̣ūk ᴀʟī′q!anutc. //
	\glpreamble	Hú ḵu.aa tsʼas xook alʼéexʼ nuch. //
	\gla	{} Hú {} ḵu.aa \rlap{tsʼas} @ {}
		{} xook {} 
		\rlap{alʼéexʼ} @ {} @ {} @ \•nuch. //
	\glb	{} hú {} ḵu.aa tsʼa =s
		{} xook {}
		a- \rt[²]{lʼixʼ} -μμH\hspace{1.25em} =nooch //
	\glc	{}[\pr{DP} \xx{3h} {}] \xx{contr} just =\xx{dub}
		{}[\pr{DP} dry {}]
		\xx{arg}- \rt[²]{break} -\xx{var} =\xx{hab·aux} //
	\gld	{} her {} however just \•maybe
		{} dry {}
		\rlap{3>3.\xx{ncnj}.\xx{impfv}.break} {} {} \•always //
	\glft	‘Her however, she always breaks just dry wood.’
		//
\endgl
\xe

\ex\label{ex:89-25-threw-off-shirts}%
\exmn{253.5}%
\begingl
	\glpreamble	Kē ag̣aᴀ′dînawe xāt ā′ni dᴀx qāk!udᴀ′s! kāxkî′nde du′qêtcnutc. //
	\glpreamble	Kei ag̱a.ádín áwé, x̱áat aaní dáx̱, ḵaa kʼoodásʼi, kaax̱ kínde dug̱ích nuch. //
	\gla	{} Kei @ \rlap{ag̱a.ádín} @ {} @ {} @ {} @ {} @ {} @ {} {}
		\rlap{áwé,} @ {}
		{} x̱áat \rlap{aaní} @ {} @ \•dáx̱, {} +
		{} ḵaa \rlap{kʼoodásʼi,} @ {} {}
		{} {} \rlap{kaax̱} @ {} {}
		{} \rlap{kínde} @ {} {}
		\rlap{dug̱ích} @ {} @ {} @ {} @ \•nuch. //
	\glb	{} kei= a- {} g̱- \rt[¹]{.at} -μH -n -ín {}
		á -wé
		{} x̱áat aan -í =dáx̱ {}
		{} ḵaa kʼoodásʼ -í {} 
		{} {} ká =dáx̱ {}
		{} kín -dé {}
		du- d- \rt[²]{g̱ich} -μH =nooch //
	\glc	{}[\pr{CP} up= \xx{4h·s}- \xx{zcnj}\· \xx{mod}- \rt[¹]{go·\xx{pl}} -\xx{var} -\xx{nsfx} -\xx{ctng} {}]
		\xx{foc} -\xx{mdst}
		{}[\pr{PP} salmon land -\xx{pss} =\xx{abl} {}]
		{}[\pr{DP} \xx{4h·pss} shirt -\xx{pss} {}]
		{}[\pr{PP} \xx{rflx·pss} \xx{hsfc} =\xx{abl} {}]
		{}[\pr{PP} up -\xx{all} {}]
		\xx{4h·s}- \xx{mid}- \rt[²]{throw·\xx{pl}} -\xx{var} =\xx{hab·aux} //
	\gld	{} up \rlap{\xx{ctng}.they.go·\xx{pl}} {} {} {} {} {} {} {}
		\rlap{it.is} {}
		{} salmon land -of \•from {}
		{} their shirt {} {}
		{} self atop \•from {}
		{} \rlap{upward} {} {}
		\rlap{\xx{ncnj}.\xx{impfv}.they.throw·\xx{pl}} {} {} {} \•always //
	\glft	‘Then having gone up from the salmon land, people always throw their shirts off upward.’
		//
\endgl
\xe

The postposition phrase \fm{x̱áat aaní dáx̱} ‘from the salmon land’ is syntactically interesting.
It is apparently associated with the contingent clause \fm{kei ag̱a.ádín} ‘whenever they go up’, but it appears after the focus particle \fm{áwé}.
It seems to be an inserted parenthetical appearing in a topic-like position, but its precise syntactic position is still unclear.
It is also possible that this is a case of right dislocation out of a consecutive clause, and there are signs of the same thing in other narratives that suggest something similar, but the syntactic details of this remain to be worked out.

\ex\label{ex:89-26-shake-them}%
\exmn{253.6}%
\begingl
	\glpreamble	Kᴀdukî′ksînutc. //
	\glpreamble	Kadukíksʼi nuch. //
	\gla	\rlap{Kadukíksʼi} @ {} @ {} @ {} @ {} @ \•nuch //
	\glb	k- du- \rt[²]{kik} -μH -sʼ =nooch //
	\glc	\xx{qual}- \xx{4h·s}- \rt[²]{shake} -\xx{var} -\xx{rep} =\xx{hab·aux} //
	\gld	\rlap{\xx{zcnj}.\xx{impfv}.they.shake.\xx{rep}} {} {} {} {} \•always //
	\glft	‘They always shake them repeatedly.’
		//
\endgl
\xe

\ex\label{ex:89-27-like-grease-burn-it}%
\exmn{253.7}%
\begingl
	\glpreamble	Atūtxī′nawe ʟe ex yêx ᴀt akug̣ā′ntc yū′caq xōq!ᵘ. //
	\glpreamble	A tóotx̱een áwé tle eix̱ yáx̱ át akoogaanch yú shaaḵ x̱ooxʼ. //
	\gla	{} A \rlap{tóotx̱een} @ {} @ {} @ {} {}
		\rlap{áwé} @ {}
		tle {} eix̱ yáx̱ {}
		{} \rlap{át} @ {} {}
		\rlap{akoogaanch} @ {} @ {} @ {} @ {} @ {}
		{} yú shaaḵ \rlap{x̱ooxʼ.} @ {} {} //
	\glb	{} a tú =dáx̱ =ee -n {}
		á -wé
		tle {} eix̱ yáx̱ {}
		{} á -t {}
		a- k- u- \rt[²]{gan} -μμL -ch
		{} yú shaaḵ x̱oo -xʼ {} //
	\glc	{}[\pr{PP} \xx{3n·pss} inside =\xx{abl} =\xx{base} -\xx{instr} {}]
		\xx{foc} -\xx{mdst}
		just {}[\pr{PP} grease like {}]
		{}[\pr{PP} \xx{3n} -\xx{pnct} {}]
		\xx{arg}- \xx{qual}- \xx{zpfv}- \rt[²]{burn} -\xx{var} -\xx{rep}
		{}[\pr{PP} \xx{dist} driftlog among -\xx{loc} {}] //
	\gld	{} its inside \•from {} -with {}
		\rlap{it.is} {}
		just {} grease like {}
		{} there -at {}
		\rlap{3>3.\xx{zcnj}.\xx{hab}.burn} {} {} {} {} {}
		{} that driftlog among -at {} //
	\glft	‘It is from within them that they burn it there like grease, among those logs.’
		//
\endgl
\xe

The phrase \citeauthor{swanton:1909} transcribes as \orth{Atūtxī′n} and glosses as “From into it (clothing)” in (\lastx) is difficult to interpret.
Given his gloss, the first part of this is likely \fm{a tóotx̱} ‘from inside of it’ which is the relational noun \fm{tú} ‘inside’ and the ablative postposition \fm{dáx̱} ‘from’ in its syncopated form \fm{tx̱} [\ipa{tχ}].
The remaining \fm{een} seems to be the instrumental postposition \fm{-n} ‘with’ attached to the meaningless postpositional base \fm{ee}.
This implies the normally ungrammatical structure where one postposition is followed by another postposition, i.e.\ postposition stacking.
Another possibility is that \citeauthor{swanton:1909} missed a vowel between the two PPs so that the phrase should instead be two phrases \fm{a tóotx̱ a een} and thus mean something like ‘from within it, with it’.
Further study is needed to clarify this phrase.

\ex\label{ex:89-28-dry-went-out}%
\exmn{253.7}%
\begingl
	\glpreamble	Doaỵē′ qo′a awe′ ts!ᴀs kułkī′stc yū′x̣ūk yū′cāwat. //
	\glpreamble	Du aaÿí ḵu.aa áwé tsʼas koolkéesʼch yú xook, yú shaawát. //
	\gla	{} Du \rlap{aaÿí} @ {} {} ḵu.aa \rlap{áwé} {}
		\rlap{tsʼas} @ {}
		\rlap{koolkéesʼch} @ {} @ {} @ {} @ {} @ {} +
		{} yú xook {}
		{} yú \rlap{shaawát.} @ {} {} //
	\glb	{} du aa -í {} ḵu.aa á -wé
		tsʼa =s
		k- u- l- \rt[¹]{kisʼ} -μμH -ch
		{} yú xook {}
		{} yú sháaʷ- ÿát {} //
	\glc	{}[\pr{DP} \xx{3h·pss} \xx{part} -\xx{pss} {}] however \xx{foc} -\xx{mdst}
		just =\xx{dub}
		\xx{qual}- \xx{zpfv}- \xx{xtn}- \rt[¹]{expire} -\xx{var} -\xx{rep}
		{}[\pr{DP} \xx{dist} dry {}]
		{}[\pr{DP} \xx{dist} woman- child {}] //
	\gld	{} her \rlap{ones} {} {} however \rlap{it.is} {}
		only \•maybe
		\rlap{\xx{zcnj}.\xx{hab}.burn·out} {} {} {} {} {}
		{} that dry {} 
		{} that \rlap{girl} {} {} //
	\glft	‘It was hers however that always just went out, the dry wood, that girl.’
		//
\endgl
\xe

\ex\label{ex:89-29-did-something-to-her}%
\exmn{253.8}%
\begingl
	\glpreamble	Akā′q!awe ʟēł unałᴀ′ wâsa′ odusniyî′ yuānyê′tq!ᵘ. //
	\glpreamble	A kaax̱ áwé tléil unalé wáa sá wdusneeyí, yú aanyátkʼu. //
	\gla	{} A \rlap{kaax̱} @ {} {} \rlap{áwé} @ {}
		tléil \rlap{unalé} @ {} @ {} @ {}
		{} {} wáa sá {}
			\rlap{wdusneeyí,} @ {} @ {} @ {} @ {} @ {} @ {}
			{} yú \rlap{aanyátkʼu.} @ {} @ {} @ {} {} {} //
	\glb	{} a ká =dáx̱ {} á -wé 
		tléil u- n- \rt[¹]{leʰ} -μH
		{} {} wáa sá {}
			wu- du- d- s- \rt[¹]{niʰ} -μμL -í
			{} yú aan- ÿát -kʼw -í {} {} //
	\glc	{}[\pr{PP} \xx{3n·pss} \xx{hsfc} =\xx{abl} {}] \xx{foc} -\xx{mdst}
		\xx{neg} \xx{irr}- \xx{ncnj}- \rt[¹]{far} -\xx{var}
		{}[\pr{CP} {}[\pr{QP} how \xx{q} {}]
			\xx{pfv}- \xx{4h·s}- \xx{mid}- \xx{csv}- \rt[¹]{happen} -\xx{var} -\xx{sub}
			{}[\pr{DP} \xx{dist} town- child -\xx{dim} -\xx{pss} {}] {}] //
	\gld	{} its atop -from {} \rlap{it.is} {}
		not \rlap{\xx{ext}·\xx{stv}·\xx{impfv}.far} {} {} {}
		{} {} how ever {}
			\rlap{\xx{ncnj}.\xx{pfv}.they.make.happen} {} {} {} {} {} {}
			{} that \rlap{aristocrat} {} -little {} {} {} //
	\glft	‘It was following that that not long after they did something to her, that little aristocrat.’
		//
\endgl
\xe

The noun \fm{aanyátkʼu} ‘little aristocrat’ displays unexpected labialization of the possessive suffix \fm{-í} as seen in \citeauthor{swanton:1909}’s transcription of \orth{yuānyê′tq!ᵘ} with the final \orth{q!ᵘ} representing [\ipa{kʼʷù}] rather than expected \fm{kʼi} [\ipa{kʼì}]; compare \fm{ax̱ yátxʼi} [\ipa{ʔàχ ˈját.xʼì}] ‘my children’ with the plural suffix \fm{-xʼ} and possessive \fm{-í}.
This unexpected labialization also occurs elsewhere in this narrative with the word \fm{yátkʼu} in e.g.\ (\ref{ex:89-108-son-sick-forced-out}), (\ref{ex:89-110-live-there-with-child}), and (\ref{ex:89-111-always-bathed-child}) so it is unlikely to be a transcription error.
This unexpected labialization may be associated with the diminutive suffix \fm{-kʼ} since labialization occurs in the diminutive \fm{-ákʼw} allomorph where the base word does not have labialization: \fm{héen-ákʼw} ‘little water’, \fm{xʼáatʼ-ákʼw} ‘little island’, \fm{shaan-ákʼw} ‘little old person’, \fm{g̱éel-ákʼw} ‘little mountain pass, little saddle’, \fm{yád-ákʼw} ‘little child’.
One form that maintains this unexpected labialization of the diminutive + possessive as \fm{-kʼu} in modern Tlingit is \fm{atkʼátskʼu} [\ipa{ʔàt.ˈkʼáts.kʼʷù}] ‘young boy’ which is derived from \fm{at kʼí-ÿáts-kʼʷ-í} ‘sth’s base-child-\xx{dim}-\xx{pss}’; see (\ref{ex:89-89-son-sunbeam}) and (\ref{ex:89-107-boy-killed-by-poverty}) for some examples.

\section{Paragraph 3}

\ex\label{ex:89-30-they-went-again}%
\exmn{253.10}%
\begingl
	\glpreamble	Ts!u ᴀnaā′dawe ts!u hᴀs wuā′t g̣ᴀ′ng̣a //
	\glpreamble	Tsu ana.áat áwé tsu has woo.aat gáng̱aa. //
	\gla	{} Tsu \rlap{ana.áat} @ {} @ {} @ {} @ {} {} \rlap{áwé} @ {}
		tsu has @ \rlap{woo.aat} @ {} @ {} @ {} +
		{} \rlap{gáng̱aa.} @ {} {} //
	\glb	{} tsu a- n- \rt[¹]{.at} -μμH {} {} á -wé
		tsu has= wu- i- \rt[¹]{.at} -μμL
		{} gán -g̱áa {} //
	\glc	{}[\pr{CP} again \xx{4h·s}- \xx{ncnj}- \rt[¹]{go·\xx{pl}} -\xx{var} \·\xx{sub} {}] \xx{foc} -\xx{mdst}
		again \xx{plh}= \xx{pfv}- \xx{stv}- \rt[¹]{go·\xx{pl}} -\xx{var}
		{}[\pr{DP} firewood -\xx{ades} {}] //
	\gld	{} again \rlap{people.\xx{csec}.go·\xx{pl}} {} {} {} \·which {} \rlap{it.is} {}
		again they \rlap{\xx{ncnj}.\xx{pfv}.go·\xx{pl}} {} {} {}
		{} firewood -for {} //
	\glft	‘People having gone again, they went again for firewood.’
		//
\endgl
\xe

The sentence in (\ref{ex:89-30-they-went-again}) has an embedded consecutive clause and a main clause with different subjects, but this is somewhat unclear in the English translation.
The subject of the consecutive clause \fm{tsu ana.áat} ‘people having gone’ is the fourth person (indefinite) human \fm{a-} ‘someone, people’.
The subject of the main clause \fm{has woo.aat} ‘they went’ is a covert third person which is pluralized by the modifier \fm{has=} as well as the plural root \fm{\rt[¹]{.at}} ‘pl.\ go’.
These two subjects could have the same referent, but the resulting interpretation ‘people\ix{i} having gone again, they\ix{i} went again’ is repetitious.
The interpretation as ‘people\ix{i} having gone again, they\ix{j} went again’ is more coherent, with ‘people\ix{i}’ referring to the bear community and ‘they\ix{j}’ referring to the girl and her bear husband.

\ex\label{ex:89-31-she-saw-smoke}%
\exmn{253.10}%
\begingl
	\glpreamble	tc!a yā′doq!osî ỵêdê′ a′we aositī′n yucā′wᴀttc s!ēq. //
	\glpreamble	Chʼa yá du x̱ʼusÿeedé áwé awsiteen, yú shaawátch, sʼeiḵ. //
	\gla	{} Chʼa yá du \rlap{x̱ʼusÿeedé} @ {} @ {} {} \rlap{áwé} @ {}
		\rlap{awsiteen,} @ {} @ {} @ {} @ {} @ {} +
		{} yú \rlap{shaawátch,} @ {} @ {} {}
		{} sʼeiḵ. {} //
	\glb	{} chʼa yá du x̱ʼoos- ÿee -dé {} á -wé
		a- wu- s- i- \rt[²]{tin} -μμL
		{} yú sháaʷ- ÿát -ch {}
		{} sʼeiḵ {} //
	\glc	{}[\pr{PP} just \xx{prox} \xx{3h·pss} foot- below -\xx{all} {}] \xx{foc} -\xx{mdst}
		\xx{arg}- \xx{pfv}- \xx{xtn}- \xx{stv}- \rt[²]{see} -\xx{var}
		{}[\pr{DP} \xx{dist} woman- child -\xx{erg} {}]
		{}[\pr{DP} smoke {}] //
	\gld	{} just this her foot- below -to {} \rlap{it.is} {}
		\rlap{3>3.\xx{g̱cnj}.\xx{pfv}.see} {} {} {} {} {}
		{} that \rlap{girl} {} {} {}
		{} smoke {} //
	\glft	‘It was below her foot that she saw it, that girl, smoke.’
		//
\endgl
\xe

\citeauthor{swanton:1909} gives (\ref{ex:89-30-they-went-again}) and (\ref{ex:89-31-she-saw-smoke}) as a single sentence, but they appear to be two separate sentences.
The form in (\ref{ex:89-30-they-went-again}) has both a focused consecutive clause and a main clause along with a right dislocated PP, all of which strongly imply that this is a single sentential unit.
The form in (\ref{ex:89-31-she-saw-smoke}) has a focused PP as well as two right dislocated DPs, once more strongly implying that this is a complete sentential unit.
Another suggestive factor is that the subject of the main verb in (\ref{ex:89-30-they-went-again}) is plural and is therefore presumably the girl and her bear husband, whereas the subject of the verb in (\ref{ex:89-31-she-saw-smoke}) is explicitly the girl alone with the DP \fm{yú shaawátch} ‘that girl’.

\ex\label{ex:89-32-extends-through-little-hill}%
\exmn{253.11}%
\begingl
	\glpreamble	Yū′gutc kîtū′nᴀx nacu′  //
	\glpreamble	Yú goochkʼi tóonáx̱ naashóo. //
	\gla	{} Yú \rlap{goochkʼi} @ {} \rlap{tóonáx̱} @ {} {}
		\rlap{naashóo.} @ {} @ {} @ {} //
	\glb	{} yú gooch -kʼ tú -náx̱ {}
		n- i- \rt[¹]{shuʰ} -μμH //
	\glc	{}[\pr{PP} \xx{dist} hill -\xx{dim} inside -\xx{perl} {}]
		\xx{ncnj}- \xx{stv}- \rt[¹]{end} -\xx{var} //
	\gld	{} that hill -little inside -thru {}
		\rlap{\xx{ext}·\xx{stv}·\xx{impfv}.extend} {} {} {} //
	\glft	‘It extends through a little hill.’
		//
\endgl
\xe

\ex\label{ex:89-33-deermouse-super-help}%
\exmn{253.11}%
\begingl
	\glpreamble	qᴀg̣ᴀ′qqocā-nᴀk! ᴀsiyu′ ᴀcigā′ wusu′. //
	\glpreamble	Kag̱ákkʼw sháanákʼw ásíyú ash eeg̱áa woosoo. //
	\gla	{} \rlap{Kag̱ákkʼw} @ {} \rlap{shaanákʼw} @ {} @ {} {}
		\rlap{ásíyú} @ {} @ {}
		{} ash \rlap{eeg̱áa} @ {} {}
		\rlap{woosoo.} @ {} @ {} @ {} //
	\glb	{} kag̱áak -kʼw \rt[¹]{shan} -μμL -kʼw {}
		á -sí -yú
		{} ash ee -g̱áa {}
		wu- i- \rt[¹]{suʰ} -μμL //
	\glc	{}[\pr{DP} deer·mouse -\xx{dim} \rt[¹]{old} -\xx{var} -\xx{dim} {}]
		\xx{foc} -\xx{dub} -\xx{dist}
		{}[\pr{PP} \xx{3prx} \xx{base} -\xx{ades} {}]
		\xx{pfv}- \xx{stv}- \rt[¹]{sup·help} -\xx{var} //
	\gld	{} deer·mouse -little \rlap{old·person} {} -little {}
		\rlap{it.is.apparently} {} {}
		{} her {} -for {}
		\rlap{\xx{ncnj}.\xx{pfv}·super·help} {} {} {} //
	\glft	‘It is apparently a little old lady deermouse who gave supernatural help to her.’
		//
\endgl
\xe

\citeauthor{swanton:1909} also gives (\ref{ex:89-32-extends-through-little-hill}) and (\ref{ex:89-33-deermouse-super-help}) as a single sentence, though in his English translation it is presented as two sentences.
\citeauthor{swanton:1909}’s translation of (\ref{ex:89-32-extends-through-little-hill}) says that the one extending out is the little old lady deermouse, probably because he takes the phrase \fm{kag̱ákkʼw sháanákʼw ásíyú} at the beginning of (\ref{ex:89-33-deermouse-super-help}) to be part of the end of (\ref{ex:89-32-extends-through-little-hill}).
This is syntactically odd however because it places a focused phrase at the end of a sentence where normally focused phrases appear at the beginning.
As such, the sentence in (\ref{ex:89-32-extends-through-little-hill}) actually has an implicit subject which could be interpreted either as the little old lady deermouse or alternatively as the smoke mentioned in (\ref{ex:89-31-she-saw-smoke}).
The interpretation as the deermouse is suggested by the use of \fm{n}-conjugation with the extensional state imperfective verb \fm{naashóo} ‘it extends horizontally’, but it is nonetheless plausible for smoke to extend horizontally.

The noun \fm{kag̱áak} in (\ref{ex:89-33-deermouse-super-help}) is translated here as ‘deermouse’, though it is often translated elsewhere as ‘field mouse’ or just ‘mouse’.
This term specifically refers to the North American deermouse (\species{Peromyscus}{maniculatus}[Wagner 1845]).
A related term \fm{kutsʼeen} refers to the Northwestern deermouse (\species{Peromyscus}{keeni}[Rhoads 1894]), but this is also used generically for any mouse or rat including introduced species and so is also used for computer mice as a calque from English.
Similar terms for small rodents include \fm{ḵʼeichʼeedí} ‘packrat’ (bush-tailed woodrat, \species{Neotoma}{cinerea}[Ord 1815]) and \fm{tseidukaadé} ‘rock rat’ (collared pika, \species{Ochotona}{collaris}[Nelson 1893]), the latter apparently borrowed from Tagish.
The noun \fm{kag̱áak} ‘deermouse’ is also found in the noun \fm{kag̱aklʼeedí} ‘yarrow’ (\species{Achillea}{millefolium}[L.]) attached to the stem \fm{lʼeet} ‘tail’ and so yarrow is sometimes known locally as ‘rattail’.
\citeauthor{paul:1930} suggests both “field mouse” and “mole” \parencite[253]{paul:1930}.

The noun \fm{sháanákʼw} in (\ref{ex:89-33-deermouse-super-help}) is translated by \citeauthor{swanton:1909} as ‘grandmother’ but it is not a kinship term.
The noun \fm{shaan} refers either to an old person or to the abstract concept of old age.
It is also used as a postnominal modifier as in \fm{ḵáa shaan} ‘old man’, and in a few compounds like \fm{shaan gúgu} ‘old people’s ear fungus’ (unid., perh.\ \species{Spongiporus}{leucospongia}[(Cooke \&\ Harkn.)\ Murrill 1905] or \species{Trametes}{suaveolens}[L.]).
The Tlingit term \fm{sháanákʼw} is not gender specific, but \citeauthor{swanton:1909}’s translation as ‘grandmother’ suggests a woman so ‘old lady’ is used here in English.

\ex\label{ex:89-34-come-inside}%
\exmn{253.12}%
\begingl
	\glpreamble	“Nēł gu tcxᴀnk!. //
	\glpreamble	«\!Neil gú, chx̱ánkʼ. //
	\gla	{} \llap{«\!}\rlap{Neil} @ {} {}
		\rlap{gú,} @ {} @ {} @ {}
		{} \rlap{chx̱ánkʼ.} @ {} {} //
	\glb	{} neil {} {}
		{} {} \rt[¹]{gut} -⊗
		{} (da)chx̱án -kʼ {} //
	\glc	{}[\pr{PP} inside \·\xx{pnct} {}] 
		\xx{zcnj}\· \xx{2sg·s}- \rt[¹]{go·\xx{sg}} -\xx{var}
		{}[\pr{DP} grandchild·\xx{voc} -\xx{dim} {}] //
	\gld	{} inside -to {}
		\rlap{\xx{imp}.you·\xx{sg}.go·\xx{sg}} {} {} {}
		{} \rlap{grandchild} {} {} //
	\glft	‘“Come inside, grandchild.’
		//
\endgl
\xe

The verb \fm{gú} in (\lastx) demonstrates a well known irregularity where an imperative stem based on one of the three roots \fm{\rt[¹]{gut}} ‘sg.\ go’, \fm{\rt[¹]{.at}} ‘pl.\ go’, or \fm{\rt[¹]{nuk}} ‘sg.\ sit’ lacks the final consonant of the root.
This is represented by the stem variation symbol -⊗ and occurs without exception in all recorded varieties of Tlingit.
Although the /\ipa{tʃ}/ at the beginning of the following word could potentially obscure a final /\ipa{t}/ in the verb stem, \citeauthor{swanton:1909}’s transcription \orth{gu} is most likely correct.

The noun \fm{chx̱ánkʼ} in (\lastx) is the vocative form of the kinship term \fm{dachx̱án} ‘grandchild’.
This term, like its inverse \fm{léelkʼw} ‘grandparent’, is not moiety dependent and so can be used to refer both to members of the same moiety and to members of the opposite moiety.
As such, it can be generalized beyond actual kinship to any elder–youth relationship.
Thus the old lady deermouse is not necessarily asserting that the protagonist is her actual grandchild, but instead could be using \fm{chx̱ánkʼ} as a term of endearment and comfort.

The noun \fm{dachx̱án} ‘grandchild’ has an interesting and incompletely understood etymology.
The \fm{ch} is probably a contraction of \fm{ji-} where \fm{ji-x̱án} is ‘hand-near’ and means roughly ‘near at hand, convenient’.
This contraction is recorded from speakers of Southern and Transitional varieties and occurs in the name \fm{Ḵaachx̱an.áakʼw} ‘Convenient Pond’ for Wrangell which is etymologically \fm{ḵaa ji-x̱án áa-kʼw} ‘one’s hand-near lake-little’.
The \fm{da} portion is still unidentified, but could be from \fm{daa} ‘around’ or alternatively could reflect the \fm{d-} voice prefix that is frozen in a handful of relational nouns like \fm{daÿéen} ‘facing’, \fm{dachóon} ‘straight, directly’, \fm{dakádin} ‘opposite, other way’, and \fm{daséixʼán} ‘exchanging places’ as well as in several directional nouns like \fm{dikée} ‘up’, \fm{diÿee} ‘down’, \fm{diyáa} ‘across’, and \fm{digeeÿigé} ‘middle, centre’.

\ex\label{ex:89-35-}%
\exmn{253.12}%
\begingl
	\glpreamble	ʟēł niya′ kucîgᴀnē′x ᴀt iỵᴀ′dawe, //
	\glpreamble	Tléil niyaa ḵushiganeix̱ át i ÿát áwé; //
	\gla	Tléil  {} {} \{niyaa\} \{ḵushig̱aneix̱\} {} át {}
		{} i \rlap{ÿát} @ {} {}
		\rlap{áwé;} @ {} //
	\glb	tléil {} {} \{niyaa\} \{ḵushig̱aneix̱\} {} át {}
		{} i ÿá -t {}
		á -wé //
	\glc	\xx{neg} {}[\pr{DP} {}[\pr{CP} ?? ?? {}] thing {}]
		{}[\pr{PP} \xx{2sg·pss} face -\xx{pnct} {}]
		\xx{foc} -\xx{mdst} //
	\gld	not {} {} ?? ?? {} thing {}
		{} your face -at {} 
		\rlap{it.is} {} //
	\glft	‘’\newline
		“not easy what saved you things around you”\newline
		“These are very dangerous animals you are among”
		//
\endgl
\xe

\FIXME{First part \orth{ʟēł} is obviously \fm{tléil} ‘not’.
Last part \orth{ᴀt iỵᴀ′dawe} looks like it’s \fm{át i ÿát áwé}.
The middle bit is unclear.
\cite{leer:1977} gives this as \orth{Tléil niyaa \fm{kucîgᴀnē′x}\textsuperscript{?} át i ÿát áwé\textsuperscript{?}} so he interprets the first unclear part as the noun \fm{niÿaa} ‘direction’ but leaves the rest uninterpreted.}

\FIXME{\citeauthor{paul:1930} reads \orth{niya′} as “on your side” suggesting a missing possessor and thus \fm{i niyaa} ‘your direction’ \parencite[253]{paul:1930}.}

\FIXME{Apparently \fm{\rt[¹]{nex̱}} ‘safe’; this is \fm{g̱}-conjugation so perhaps \fm{g̱aneix̱}.}

\FIXME{Read \orth{kucî} as \fm{ḵushée} ‘searching’; \orth{gᴀnē′x} as \fm{g̱aneix̱} ‘rescuing, saving; saviour, survivor’.
The gloss suggests a relative clause headed by \fm{át} ‘thing’.
\fm{ḵu-} + \fm{sha-/shu-/se-} + \fm{g̱-} + \rt[¹]{nex̱} + \fm{-μμL}}

\FIXME{\orth{niya′} = \fm{ni}? + \fm{yaa=}?}

\FIXME{\fm{Tléil [[i niyaa ḵushee] g̱aneix̱₁] [á₁-t] yee.aat}}

\FIXME{\fm{ḵushée-g̱aa} \fm{neix̱}?}

\ex\label{ex:89-36-brown-bear-rescue}%
\exmn{253.13}%
\begingl
	\glpreamble	xūts! qoa′ni awe′ ī′usinē′x.” //
	\glpreamble	xóots ḵwáani áwé iwsineix̱\!» //
	\gla	{} xóots \rlap{ḵwáani} @ {} {}
		\rlap{áwé} @ {} 
		\rlap{iwsineix̱.} @ {} @ {} @ {} @ {} @ {} //
	\glb	{} xóots ḵwáan -í {}
		á -wé
		i- wu- s- i- \rt[¹]{nex̱} -μμL //
	\glc	{}[\pr{DP} br·bear people -\xx{pss} {}]
		\xx{foc} -\xx{mdst}
		\xx{2sg·o}- \xx{pfv}- \xx{csv}- \xx{stv}- \rt[¹]{safe} -\xx{var} //
	\gld	{} br·bear people -of {}
		\rlap{it.is} {}
		\rlap{you·\xx{sg}.\xx{g̱cnj}.\xx{pfv}.make.safe} {} {} {} {} {} //
	\glft	‘it is the brown bear people who have rescued you”’
		//
\endgl
\xe

\ex\label{ex:89-37-she-explain-to-her}%
\exmn{253.14}%
\begingl
	\glpreamble	ᴀcī′n qonā′xdaq aka′wanîk. //
	\glpreamble	ash een ḵunáax̱ daaḵ akaawaník. //
	\gla	{} ash \rlap{een} @ {} {}
		ḵunáax̱ @ daaḵ @ \rlap{akaawaník.} @ {} @ {} @ {} @ {} @ {} //
	\glb	{} ash ee -n {}
		ḵunáax̱= daaḵ= a- k- wu- i- \rt[²]{nik} -μH //
	\glc	{}[\pr{PP} \xx{3prx} \xx{base} -\xx{instr} {}]
		\xx{explan}= inland= \xx{arg}- \xx{qual}- \xx{pfv}- \xx{stv}- \rt[²]{tell} -\xx{var} //
	\gld	{} her {} -to {}
		explain out \rlap{3>3.\xx{zcnj}.\xx{pfv}.tell} {} {} {} {} {} //
	\glft	‘she explains to her.’
		//
\endgl
\xe

The verb \fm{ḵunáax̱ daaḵ akaawaník} in (\lastx) is an instance of the explanatory derivation \vbderiv{ḵunáax̱ daaḵ}{∅}{no rep.}{explaining, clarifying} as described by \textcite[220]{leer:1991}.
Apparently the same derivation is attested with this same root \rt[²]{nik} by \textcite[83]{story-naish:1973} where \fm{daak} ‘out to sea, offshore’ appears instead of \fm{daaḵ} ‘inland region, area up from shore’.
\textcite{eggleston:2017} also has \fm{daak} rather than \fm{daaḵ} and further notes that one of her consultants says \fm{wanáax̱} instead of \fm{ḵunáax̱}.
Since this derivation assigns the \fm{∅}-conjugation class it accounts for the short high stem \fm{–ník} rather than the expected \fm{–neek}.
The preverb \fm{ḵunáax̱} is not attested anywhere outside of this derivation.
It appears to be contain the areal \fm{ḵu-} and the pertingent postposition \fm{-x̱}.
The \fm{náa} is unidentified but could be connected to \fm{náa} ‘covering’ \parencite[04/28, 03/121]{leer:1973}, to the root \fm{\rt[²]{na}} \~\ \fm{\rt[²]{naʼÿ}} ‘order, send’ \parencite[04/22–25]{leer:1973}, to the phrase \fm{chʼu g̱unáa} \~\ \fm{chʼu ḵunáa} ‘even though, despite, in spite of’ \parencite[04/27]{leer:1973}, or perhaps to the noun \fm{niÿaa} ‘direction’ \parencite[03/120–125]{leer:1973}.
Another potentially related stem is the poorly documented root \fm{\rt{ÿax̱}} ‘desire, plan’ \parencite[03/172]{leer:1973} or the similarly poorly documented \fm{ÿaax̱} ‘side’ \parencites[03/173]{leer:1973}[10]{leer:1978b}.

\ex\label{ex:89-38-she-explain-to-her}%
\exmn{253.15}%
\begingl
	\glpreamble	Tc!uʟe′ ᴀcu-kā′wadjᴀ. “Yū′do ỵīī′c ānîˈ.” //
	\glpreamble	Chʼu tle ashukaawajáa «\!Yóodu ee éesh aaní\!». //
	\gla	Chʼu tle \rlap{ashukaawajáa} @ {} @ {} @ {} @ {} @ {} @ {}
		{} \llap{«\!}\rlap{Yóodu} @ {} @ {}
			{} ee éesh \rlap{aaní.} @ {} @ {} @ {} //
	\glb	chʼu tle a- shu- k- wu- i- \rt[²]{jaʰ} -μμH
		{} yú -t -ú
			{} ee éesh aan -í {} {} //
	\glc	just then \xx{arg}- end- \xx{qual}- \xx{pfv}- \xx{stv}- \rt[²]{instruct} -\xx{var}
		{}[\pr{CP} \xx{dist} -\xx{link} -\xx{locp}
			{}[\pr{DP} \xx{2sg·pss} father land -\xx{pss} {}] {}] //
	\gld	just then \rlap{3>3.\xx{zcnj⁺}.\xx{pfv}.instruct} {} {} {} {} {} {}
		{} \xx{dist} -\rlap{is.at} {}
			{} your·\xx{sg} father land -of {} //
	\glft	‘She instructed her “There is your father’s land.”’
		//
\endgl
\xe

The phrase \fm{ee éesh} in (\lastx) is given by \citeauthor{swanton:1909} as \orth{ỵīī′c} which implies \fm{ÿee éesh} ‘your (pl.)\ father’, i.e.\ that the possessor is plural.
This does not make sense in context because there is only the girl whom the old lady deermouse is talking to, so it has been changed to \fm{ee éesh} ‘your (sg.)\ father’.

\ex\label{ex:89-39-going-for-salmon-morning}%
\exmn{253.15}%
\begingl
	\glpreamble	Ayᴀ′xawe tc!u ts!utā′t xāt g̣a naadê′, g̣onaye′ ā′dawe //
	\glpreamble	A yáx̱ áwé, chʼu tsʼootaat x̱áatg̱aa na.aadí g̱unayéi .áat áwé, //
	\gla	{} A yáx̱ {}
		\rlap{áwé,} @ {}
		{} chʼu {} tsʼootaat
			{} \rlap{x̱áatg̱aa} @ {} {}
			\rlap{na.aadí} @ {} @ {} @ {} {}
		g̱unayéi @ \rlap{.áat} @ {} @ {} @ {} {}
		\rlap{áwé,} @ {} //
	\glb	{} a yáx̱ {}
		á -wé
		{} chʼu {} tsʼootaat
			{} x̱áat -g̱áa {}
			n- \rt[¹]{.at} -μμL -í {}
		g̱unayéi= {} \rt[¹]{.at} -μμH {} {}
		á -wé //
	\glc	{}[\pr{PP} \xx{3n} \xx{sim} {}]
		\xx{foc} -\xx{mdst}
		{}[\pr{CP} just {}[\pr{NP} morning
			{}[\pr{PP} salmon -\xx{ades} {}]
			\xx{ncnj}- \rt[¹]{go·\xx{pl}} -\xx{var} -\xx{nmz} {}]
		\xx{incep}= \xx{zcnj}\· \rt[¹]{go·\xx{pl}} -\xx{var} \·\xx{sub} {}]
		\xx{foc} -\xx{mdst} //
	\gld	{} it like {}
		\rlap{it.is} {}
		{} just {} morning
			{} salmon -for {}
			\rlap{\xx{ncnj}.\xx{prog}.go·\xx{pl}} {} {} -ing {}
		start \rlap{\xx{csec}.they.go·\xx{pl}} {} {} {} {}
		\rlap{it.is} {} //
	\glft	‘So like that, them having started going to go for salmon in the morning,’
		//
\endgl
\xe 

\ex\label{ex:89-40-she-ran-away-opposite-dirn}%
\exmn{254.1}%
\begingl
	\glpreamble	ᴀdakᴀ′dīnawe yūt wudjix̣ī′x̣. //
	\glpreamble	a dakádeen áwé yóot wujixíx. //
	\gla	{} a dakádeen {}
		\rlap{áwé} @ {}
		{} \rlap{yóot} @ {} {}
		\rlap{wujixíx.} @ {} @ {} @ {} @ {} @ {} //
	\glb	{} a dakádin {}
		á -wé
		{} yú -t {}
		wu- d- sh- i- \rt[¹]{xix} -μH //
	\glc	{}[\pr{DP} \xx{3n·pss} opp·dir’n {}]
		\xx{foc} -\xx{mdst}
		{}[\pr{PP} \xx{dist} -\xx{pnct} {}]
		\xx{pfv}- \xx{mid}- \xx{pej}- \xx{stv}- \rt[¹]{fall·\xx{sg}} -\xx{var} //
	\gld	{} their opp·dir’n {}
		\rlap{it.is} {}
		{} \rlap{away} {} {}
		\rlap{\xx{zcnj}.\xx{pfv}.she.run·\xx{sg}} {} {} {} {} {} //
	\glft	‘it is in the opposite direction of them that she ran away.’
		//
\endgl
\xe

The material in (\ref{ex:89-39-going-for-salmon-morning}) and (\ref{ex:89-40-she-ran-away-opposite-dirn}) are a single sentence which has been broken into two units for easier analysis.
The first focus phrase \fm{a yáx̱ áwé} ‘it is like it’ in (\ref{ex:89-39-going-for-salmon-morning}) is a discourse element which serves to link this whole sentence to the preceding situation where the old lady doormouse gives instructions to the girl, implying that she behaves just as she has been instructed.

The remainder of (\ref{ex:89-39-going-for-salmon-morning}) is a focused embedded clause with a consecutive verb form identified by the \fm{∅}-conjugation (lack of an overt aspectual prefix) and the \fm{-μμH} stem variation.
If we only include the verb then \fm{g̱unayéi áat áwé} means roughly ‘them having started going’.
The rest of the material in this clause is further modification of that verb.
The phrase \fm{x̱áatg̱aa na.aadí} is a nominalization ‘going for salmon’ which was also seen earlier in (\ref{ex:89-22-go-for-salmon}) as the adjunct of a verb ‘go’.
In English the literal translation of \fm{x̱áatg̱aa naa.aadí g̱unayéi áat} ‘them having started going to go for salmon’ seems awkward, but the Tlingit form is not a mistake since the narrator did the same thing before in (\ref{ex:89-22-go-for-salmon}) with \fm{x̱áatg̱aa na.aadi na.átch} ‘they always went going for salmon’.
Compare this with another much more common nominalized clause adjunct to a motion verb: \fm{alʼóoni x̱waagoot} ‘I went hunting’ versus \fm{ax̱waalʼóon} ‘I hunted’.
The same structure is used with \fm{alʼóoni} ‘hunting’ as a modifier of \fm{x̱waagoot} ‘I went’, but the nominalized clause \fm{alʼóoni} is simpler than \fm{x̱áatg̱aa na.aadí}.

The material in (\ref{ex:89-40-she-ran-away-opposite-dirn}) is the main clause which is modified by the adjuncts in (\ref{ex:89-39-going-for-salmon-morning}).
First there is a focused phrase \fm{a dakádeen áwé} ‘it is in the opposite direction of them’ where the possessor \fm{a} has the same referent as the subject of \fm{g̱unayéi áat} ‘them having started going’.
Note that the English translation has ‘\emph{in} the opposite direction’ but the Tlingit \fm{a dakádeen} ‘its opposite direction’ is an NP not a PP and so does not contain a postposition equivalent to ‘in’.
Then \fm{yóot wujixíx} means simply ‘she ran off’ or ‘she ran away’.
Literally \fm{yóot} is ‘over there (far away)’ with the distal \fm{yú}, but in this context it has a special interpretation where it means some nonspecific, far away location and thus is best translated as ‘away’ or ‘off’ in English.

\ex\label{ex:89-41-searched-for-her}%
\exmn{254.2}%
\begingl
	\glpreamble	Yī′gîỵî ke aā′dawe duitē′x qoya′oduwacî X̣ūts! qoa′nitc. //
	\glpreamble	Yagiyee kei a.áat áwé du eetéex̱ ḵuyawduwashée xóots ḵwáanich. //
	\gla	{} Yagiyee kei @ \rlap{a.áat} @ {} @ {} @ {} @ {} {} 
		\rlap{áwé} @ {}
		{} du \rlap{eetéex̱} @ {} {}
		\rlap{ḵuyawduwashee} @ {} @ {} @ {} @ {} @ {} @ {} @ {} 
		{} xóots \rlap{ḵwáanich.} @ {} @ {} {} //
	\glb	{} yagiyee kei= a- {} \rt[¹]{.at} -μμH {} {}
		á -wé
		{} du eetí -x̱ {}
		ḵu- ÿ- wu- du- {} i- \rt[²]{shiʰ} -μμL
		{} xóots ḵwáan -í -ch {} //
	\glc	{}[\pr{CP} day up= \xx{4h·s}- \xx{zcnj}\· \rt[¹]{go·\xx{pl}} -\xx{var} \·\xx{sub} {}]
		\xx{foc} -\xx{mdst}
		{}[\pr{PP} \xx{3h·pss} remains -\xx{pert} {}]
		\xx{areal}- \xx{qual}- \xx{pfv}- \xx{4h·s}- \xx{mid}\· \xx{stv}- \rt[²]{reach} -\xx{var}
		{}[\pr{DP} br·bear people -\xx{pss} -\xx{erg} {}] //
	\gld	{} day up \rlap{\xx{csec}.they.go·\xx{pl}} {} {} {} {} {}
		\rlap{it.is} {}
		{} her absence -in {}
		\rlap{\xx{ncnj}.\xx{pfv}.they.search} {} {} {} {} {} {} {}
		{} br·bear \rlap{people} {} {} {} //
	\glft	‘It was having come up in the day that they searched in her absence, the brown bear people.’
		//
\endgl
\xe

The phrase \fm{yagiyee kei a.áat} in (\lastx) is translated by \citeauthor{swanton:1909} as “When they came home at midday”, with the gloss “At midday up when they came”, but literally it means ‘their having gone up in the day’.
Like all Tlingit motion verbs, the verb here does not differentiate between ‘come’ and ‘go’.
The preverb \fm{kei} ‘up’ tells us that the brown bear people are going up somewhere.
The sentence in (\ref{ex:89-25-threw-off-shirts}) shows that they go up from the salmon land, and since their going for salmon is in the morning per (\ref{ex:89-39-going-for-salmon-morning}) we can conclude that in (\lastx) they are coming back from the salmon land in the middle of the day.
\citeauthor{swanton:1909}’s translation is not literal but does capture the apparently intended meaning: the brown bear people have come back up to their village to find the girl absent.

\ex\label{ex:89-42-here-her-dress}%
\exmn{254.2}%
\begingl
	\glpreamble	Yāq! kē uwaʟ!ᴀ′k! duʟ!ā′ke yucā′wᴀt. //
	\glpreamble	Yáaxʼ kei uwalʼáḵw du lʼaagí, yú shaawát. //
	\gla	{} \rlap{Yáaxʼ} @ {} {}
		kei @ \rlap{uwalʼáḵw} @ {} @ {} @ {}
		{} du \rlap{lʼaagí,} @ {} {}
		{} yú \rlap{shaawát.} @ {} {} //
	\glb	{} yá -xʼ {}
		kei= u- i- \rt[¹]{lʼaḵw} -μH
		{} du lʼaak -í {}
		{} yú sháaʷ- ÿát {} //
	\glc	{}[\pr{PP} \xx{prox} -\xx{loc} {}]
		up= \xx{zpfv}- \xx{stv}- \rt[¹]{worn·out} -\xx{var}
		{}[\pr{DP} \xx{3h·pss} dress -\xx{pss} {}]
		{}[\pr{DP} \xx{dist} woman- child {}] //
	\gld	{} here -at {}
		up \rlap{\xx{zcnj}.\xx{pfv}.worn·out} {} {} {}
		{} her \rlap{dress} {} {}
		{} that \rlap{girl} {} {} //
	\glft	‘Here it is all worn out, her dress, that girl.’
		//
\endgl
\xe

The verb that \citeauthor{swanton:1909} transcribes as \orth{kē uwaʟ!ᴀ′k!} is difficult to decipher.
His gloss is “had rotted” which suggests one of the roots \fm{\rt{naḵw}} ‘rot and fall apart’, \fm{\rt{tlʼuʼḵ}} ‘rot and smell, develop sore’, \fm{\rt{sʼix}} ‘spoil, rot’, \fm{\rt{tsʼix}} ‘damp, rotten’, \fm{\rt{kiʼch}} ‘putrefy, rot, decay’, or \fm{\rt{x̱ʼwaʼn}} ‘dry rot’, but none of these are a close phonological match with \citeauthor{swanton:1909}’s transcription which instead suggests a root like \fm{\rt{tlʼakʼ}}.
There is in fact a root \fm{\rt{tlʼakʼ}} ‘wet (on surface)’, but this does not exactly match the meaning suggested by \citeauthor{swanton:1909}’s gloss and translation and the root is normally found with either \fm{d-} or \fm{l-}.
\textcite{leer:1977} interpreted \citeauthor{swanton:1909}’s \orth{ʟ!ᴀ′k!} as \fm{\rt{tlʼuḵ}} but questioned this; the meaning ‘rot and smell, develop sore’ describes infected flesh and does not fit with a dress rotting.
There is one other root which could apply here, namely the poorly documented \fm{\rt{lʼaḵw}} ‘stuck in mud’ which is known from only one form \fm{yanax̱ woolʼaaḵw} ‘it got stuck in (a) mucky place’ \parencite[08/180]{leer:1973}.
This root might be connected to the relatively obscure noun \fm{lʼáaḵw} ‘worn out canoe’ and the related \fm{lʼáḵwti} ‘worn out, in bad condition’ \parencites[08/179]{leer:1973}[34]{leer:1978b}, and thus could be another candidate for \citeauthor{swanton:1909}’s \orth{ʟ!ᴀ′k!} with the intended meaning of either ‘stuck in the mud’ or ‘worn out’.
\FIXME{Finish deciding on the root and justify the decision and interpretation.}

\ex\label{ex:89-43-run-look-back}%
\exmn{254.3}%
\begingl
	\glpreamble	De ʟēq! cā kᴀnᴀ′x ỵawucîx̣ī′awe qox awuʟ̣îgê′n duî′tdê. //
	\glpreamble	De tléixʼ shaa kaanáx̱ ÿawusheexí áwé ḵux̱ awdlig̱én du ítde. //
	\gla	{} De {} tléixʼ shaa \rlap{kaanáx̱} @ {} {}
			\rlap{ÿawusheexí} @ {} @ {} @ {} @ {} @ {} @ {} {}
		\rlap{áwé} @ {}
		ḵux̱ @ \rlap{awdlig̱én} @ {} @ {} @ {} @ {} @ {} @ {}
		{} du \rlap{ítde.} @ {} @ {} {} //
	\glb	{} de {} tléixʼ shaa ká -náx̱ {} 
			ÿ- wu- d- sh- \rt[¹]{xix} -μμL -í {} 
		á -wé
		ḵúx̱= a- wu- d- l- i- \rt[²]{g̱eͥn} -μH
		{} du ít -dé {} //
	\glc	{}[\pr{CP} now {}[\pr{PP} one mountain \xx{hsfc} -\xx{perl} {}]
			\xx{qual}- \xx{pfv}- \xx{mid}- \xx{pej}- \rt[¹]{fall·\xx{sg}} -\xx{var} -\xx{sub} {}]
		\xx{foc} -\xx{mdst}
		\xx{rev}= \xx{arg}- \xx{pfv}- \xx{mid}- \xx{xtn}- \xx{stv}- \rt[²]{look} -\xx{var}
		{}[\pr{PP} \xx{3h·pss} after -\xx{all} {}] //
	\gld	{} now {} one mountain over -thru {}
			\rlap{\xx{ncnj}.\xx{pfv}.run·\xx{sg}} {} {} {} {} {} -that {}
		\rlap{it.is} {}
		back \rlap{3>3.\xx{zcnj}.\xx{pfv}.look} {} {} {} {} {} {}
		{} her after -to {} //
	\glft	‘It was now that she had run across one mountain that she looked back after herself.’
		//
\endgl
\xe

The translation of \fm{du ítde} in (\lastx) is given as ‘after herself’ but this is not exact.
The possessive pronoun \fm{du} is not actually reflexive but it can only be interpreted as referring to the girl doing the looking because there is no other singular human referent in this discourse context.
Also the noun \fm{ít} literally means ‘following’ and the allative postposition \fm{-dé} literally means ‘to, toward’, but the phrase ‘toward her following’ sounds peculiar in English and has incorrect implications.

\ex\label{ex:89-44-dark-with-bears}%
\exmn{254.4}%
\begingl
	\glpreamble	ʟē qag̣ê′t yᴀx g̣â′awe ỵatî′ duî′t x̣ūts! qoa′nî. //
	\glpreamble	Tle kag̱ít yáx̱ gwáawé ÿatee du ít, xóots ḵwáani. //
	\gla	Tle {} \rlap{kag̱ít} @ {} @ {} yáx̱ {}
		\rlap{gwáawé} @ {} @ {}
		\rlap{ÿatee} @ {} @ {}
		{} du ít, {} +
		{} xóots \rlap{ḵwáani.} @ {} {} //
	\glb	tle {} k- \rt[¹]{g̱it} -μH yáx̱ {}
		gwá= á -wé
		i- \rt[¹]{tiʰ} -μμL
		{} du ít {}
		{} xóots ḵwáan -í {} //
	\glc	then {}[\pr{PP} \xx{hsfc}- \rt[¹]{dark} -\xx{var} \xx{sim} {}]
		\xx{dub}= \xx{foc} -\xx{mdst}
		\xx{stv}- \rt[¹]{be} -\xx{var}
		{}[\pr{DP} \xx{3h·pss} after {}]
		{}[\pr{DP} br·bear people -\xx{pss} {}] //
	\gld	then {} \rlap{darkness} {} {} like {}
		maybe\• \rlap{it.is} {}
		\rlap{\xx{ncnj}.\xx{stv}·\xx{impfv}.be} {} {}
		{} her after {}
		{} br·bear people -of {} //
	\glft	‘Then it seems like it is darkness following her, the bear people.’
		//
\endgl
\xe

\ex\label{ex:89-45-wail-anticipation}%
\exmn{254.5}%
\begingl
	\glpreamble	ᴀckā′ yᴀx ỵāg̣āā′dawe ciaỵidē′kdag̣ā′x. //
	\glpreamble	Ash káa yax̱ ÿaa g̱a.áat áwé sh yaÿeedé kdag̱áax̱. //
	\gla	{} {} Ash \rlap{káa} @ {} {}
			yax̱ @ ÿaa @ \rlap{g̱a.áat} @ {} @ {} @ {} {}
		\rlap{áwé} @ {} +
		{} sh \rlap{yaÿeedé} @ {} @ {} {}
		\rlap{kdag̱áax̱.} @ {} @ {} @ {} //
	\glb	{} {} ash ká -μ {}
			ÿáx̱= ÿaa= g̱- \rt[¹]{.at} -μμH {} {}
		á -wé
		{} sh ÿá- ÿee -dé {}
		k- d- \rt[¹]{g̱ax̱} -μμH //
	\glc	{}[\pr{CP} {}[\pr{PP} \xx{3prx·pss} \xx{hsfc} -\xx{loc} {}]
			\xx{exh}= along= \xx{g̱cnj}- \rt[¹]{go·\xx{pl}} -\xx{var} \·\xx{sub} {}]
		\xx{foc} -\xx{mdst}
		{}[\pr{PP} \xx{rflx·pss} face- below -\xx{all} {}]
		\xx{qual}- \xx{mid}- \rt[¹]{cry} -\xx{var} //
	\gld	{} {} her atop -at {}
			all along \rlap{\xx{csec}.go·\xx{pl}} {} {} \·that {}
		\rlap{it.is} {}
		{} self’s face- below -to {}
		\rlap{\xx{gcnj}.\xx{impfv}.wail} {} {} {} //
	\glft	‘Them all coming down to her, she is wailing in anticipation.’
		//
\endgl
\xe

\ex\label{ex:89-46-lakeshore}%
\exmn{254.5}%
\begingl
	\glpreamble	ᴀk!ayaxê′ dāk udjix̣ī′x̣. //
	\glpreamble	Áa x̱ʼayaax̱í daak wujixíx. //
	\gla	{} Áa \rlap{x̱ʼayaax̱í} @ {} @ {} {} 
		daak @ \rlap{wujixíx.} @ {} @ {} @ {} @ {} @ {} //
	\glb	{} áa x̱ʼé- yaax̱ -í {}
		daak= wu- d- sh- i- \rt[¹]{xix} -μH //
	\glc	{}[\pr{PP} lake mouth- side -\xx{loc} {}]
		\xx{admar}= \xx{pfv}- \xx{mid}- \xx{pej}- \xx{stv}- \rt[¹]{fall·\xx{sg}} -\xx{var} //
	\gld	{} lake \rlap{shore} {} -at {}
		out \rlap{\xx{zcnj}.\xx{pfv}.run·\xx{sg}} {} {} {} {} {} //
	\glft	‘She ran out at the edge of a lake.’
		//
\endgl
\xe

The verb in (\lastx) includes the motion derivation with \vbderiv{daak=}{∅}{\fm{-ch} repetitive}{seaward, into open, down from sky}.
This does not mean that the protagonist runs out onto the water.
Rather, it means that she moves out into the open from a covered or enclosed area, implicitly being the covered darkness of a forest.
The translation in (\lastx) approximates this in English with the preposition ‘out’ which similarly implies motion from a covered or enclosed area.

The phrase \fm{áa x̱ʼayaax̱í} ‘at the edge of a lake’ in (\lastx) appears to have a locative \fm{-í} suffix which is rather unusual.
Normally this suffix is only found with a small handful of indefinite nouns in preverbs, namely in \fm{gági} ‘emerging’ with \fm{gáak} ‘outside’, in \fm{dáag̱i} ‘out of water’ with \fm{dáaḵ} ‘inland’, in \fm{héeni} ‘into water’ with \fm{héen} ‘(fresh) water, river’, and in \fm{éeg̱i} ‘abeach’ with \fm{éeḵ} ‘beach’.
Alternatively the \fm{-í} could be analyzed as the possessive suffix, but this would be unprecedented.
The noun \fm{x̱ʼayaax̱} ‘shoreline (of lake)’ is well documented as an inalienable noun in the phrase \fm{áa x̱ʼayaax̱} ‘lake’s edge’ as well as in \fm{héen x̱ʼayaax̱} ‘river’s edge’ and \fm{dei x̱'ayaax̱} ‘road’s edge’.
If \fm{-í} in (\lastx) were a possessive suffix then this would make the noun \fm{x̱ʼayaax̱} alienated, i.e.\ physically separated from the lake itself.
This is nonsensical, so the only other alternative would be that in this specific sentence the noun \fm{x̱ʼayaax̱} is actually alienable and so must be marked for possession by the noun \fm{áa} ‘lake’.
Given that this happens nowhere else in the documentation of Tlingit, it is more plausible that \fm{-í} is a locative postposition.
A final possibility is that \citeauthor{swanton:1909}’s transcription of a final \orth{ê′} is spurious or a misinterpretation of release noise from the uvular fricative /\ipa{χ}/.

\ex\label{ex:89-47-boat-floats-in-lake}%
\exmn{254.6}%
\begingl
	\glpreamble	Yū′a ʟen ᴀ′dî gīyīg̣ē′t gwâyu′ łix̣ā′c yū′yākᵘ cᴀadakū′q! ᴀca′. //
	\glpreamble	Yú áa tlein, a digeeyeegéit gwáayú wlihaash yú yaakw; shadaakóox̱ʼ, a shá. //
	\gla	{} Yú áa tlein, {} 
		{} a \rlap{digeeyeegéit} @ {} {}
		\rlap{gwáayú} @ {} @ {} 
		\rlap{wlihaash} @ {} @ {} @ {} @ {} +
		{} yú yaakw; {}
		{} \rlap{shadaakóox̱ʼ,} @ {} @ {} {}
		{} a shá. {} //
	\glb	{} yú áa tlein {}
		{} a digeeÿgé -t {}
		gwá= á -yú
		wu- l- i- \rt[¹]{hash} -μμL
		{} yú yaakw {}
		{} shá- daa- kóox̱ʼ {}
		{} a shá {} //
	\glc	{}[\pr{DP} \xx{dist} lake big {}]
		{}[\pr{PP} \xx{3n·pss} middle -\xx{pnct} {}]
		\xx{mir}= \xx{foc} -\xx{dist}
		\xx{pfv}- \xx{xtn}- \xx{stv}- \rt[¹]{float} -\xx{var}
		{}[\pr{DP} \xx{dist} boat {}]
		{}[\pr{DP} head- around- stalk {}]
		{}[\pr{DP} \xx{3n·pss} head {}] //
	\gld	{} that lake big {}
		{} its middle -at {}
		\rlap{apparently.it.is} {} {}
		\rlap{\xx{ncnj}.\xx{pfv}.float} {} {} {} {}
		{} that boat {}
		{} \rlap{hat·ring} {} {} {}
		{} its head {} //
	\glft	‘That big lake, apparently it was in the middle of it that it floated, a canoe; a clan hat (on) its head.’
		//
\endgl
\xe

The sentence in (\lastx) has an appendix \fm{shadaakóox̱ʼ, a shá} which is syntactically peculiar.
The gloss “a dance hat with high crown on its head” and the translation “wearing a dance hat” from \citeauthor{swanton:1909} suggest that this should be interpreted as given in the translation in (\lastx): ‘a clan hat (on) its head’.
But the phrase \fm{a shá} does not contain any locative postposition and there is neither a verb nor a non-verb predicate structure (locative predicate with \fm{-ú}, \fm{áwé} predicate).
Instead this appendix appears to be no more than two noun phrases in sequence with no elaboration: ‘a clan hat, its head’.
The translation in (\lastx) assumes that \citeauthor{swanton:1909}’s interpretation is correct and includes a parenthesized ‘(on)’ to reflect the missing locative and predicate.

The noun \fm{shadaakóox̱ʼ} in (\lastx) is fairly rare and deserves some comment.
It includes the two very common nouns \fm{shá} ‘head’ and \fm{daa} ‘around, about’.
The stem noun \fm{kóox̱ʼ} normally refers to a dry stalk of Indian celery (\species{Heracleum}{maximum}[H.Bartram 1791]); this plant is also known in English as ‘cow parsnip’ or in some parts of Alaska as ‘pushki’ from the plural of Siberian Russian пучка \fm{púčka} (\species{Heracleum}{sibiricum}[L.], a.k.a.\ борщевик сибирский \fm{borščevík sibírskiĭ} ‘Siberian hogweed’).
The compound \fm{shadaakóox̱ʼ} therefore literally means ‘dry Indian celery stalk around the head’ but it actually refers to the woven basketry rings that are fitted on top of a hat to indicate the number of potlatches held by the owner; compare the Haida \fm{sgíl} for the same thing \parencite[511]{enrico:2005}.
In (\lastx) it seems to be used in synecdoche to refer to a carved ceremonial clan hat judging by \citeauthor{swanton:1909}’s gloss “a dance hat with a high crown” and translation “dance hat”.

\ex\label{ex:89-48-run-into-water-it-said}%
\exmn{254.7}%
\begingl
	\glpreamble	“Hā′nde hīnt isî′x̣” yuaciao′sîqa. //
	\glpreamble	«\!Haandé héent isheex!\!» yóo ash yawsiḵaa. //
	\gla	{} {} \llap{«\!}\rlap{Haandé} @ {} {}
			{} \rlap{héent} @ {} {}
			\rlap{isheex!\!»} @ {} @ {} @ {} @ {} @ {} {}
		yóo @ ash @ \rlap{yawsiḵaa.} @ {} @ {} @ {} @ {} @ {} //
	\glb	{} {} haaⁿ -dé {}
			{} héen -t {}
			{} i- d- sh- \rt[¹]{xix} -μμL {}
		yóo= ash= ÿ- wu- s- i- \rt[¹]{ḵa} -μμL //
	\glc	{}[\pr{CP} {}[\pr{PP} here -\xx{all} {}]
			{}[\pr{PP} water -\xx{pnct} {}]
			\xx{zcnj}\· \xx{2sg·s}- \xx{mid}- \xx{pej}- \rt[¹]{fall·\xx{sg}} -\xx{var} {}]
		\xx{quot}= \xx{3prx·o}= \xx{qual}- \xx{pfv}- \xx{csv}- \xx{stv}- \rt[¹]{say} -\xx{var} //
	\gld	{} {} here -to {}
			{} water -to {}
			\rlap{\xx{imp}.you·\xx{sg}.run·\xx{sg}} {} {} {} {} {} {}
		thus her \rlap{\xx{ncnj}.\xx{pfv}.say} {} {} {} {} {} //
	\glft	‘“Run this way into the water!” it said to her.’
		//
\endgl
\xe

\ex\label{ex:89-49-she-ran-into-water}%
\exmn{254.7}%
\begingl
	\glpreamble	ʟe akā′de hīnt wudjix̣ī′x̣. //
	\glpreamble	Tle a kaadé héent wujixíx. //
	\gla	Tle {} a \rlap{kaadé} @ {} {}
		{} \rlap{héent} @ {} {}
		\rlap{wujixíx.} @ {} @ {} @ {} @ {} @ {} //
	\glb	tle {} a ká -dé {}
		{} héen -t {}
		wu- d- sh- i- \rt[¹]{xix} -μH //
	\glc	then {}[\pr{PP} \xx{3n·pss} \xx{hsfc} -\xx{all} {}]
		{}[\pr{PP} water -\xx{pnct} {}]
		\xx{pfv}- \xx{mid}- \xx{pej}- \xx{stv}- \rt[¹]{fall·\xx{sg}} -\xx{var} //
	\gld	then {} its atop -to {}
		{} water -to {}
		\rlap{\xx{zcnj}.\xx{pfv}.run·\xx{sg}} {} {} {} {} {} //
	\glft	‘She ran across into the water.’
		//
\endgl
\xe

\ex\label{ex:89-50-pulled-aboard}%
\exmn{254.8}%
\begingl
	\glpreamble	Yāx wuduwaỵē′q. //
	\glpreamble	Yaax̱ wuduwaÿeiḵ. //
	\gla	Yaax̱ @ \rlap{wuduwaÿeiḵ.} @ {} @ {} @ {} @ {} @ {} //
	\glb	yaax̱= wu- du- {} i- \rt[²]{ÿeḵ} -μμL //
	\glc	aboard= \xx{pfv}- \xx{4h·s}- \xx{mid}\· \xx{stv}- \rt[²]{pull} -\xx{var} //
	\gld	aboard \rlap{\xx{g̱cnj}.\xx{pfv}.people.pull} {} {} {} {} {} //
	\glft	‘She was pulled aboard.’
		//
\endgl
\xe

\ex\label{ex:89-51-snap-back}%
\exmn{254.8}%
\begingl
	\glpreamble	Tc!uʟe′ ᴀcī′n dekī′t wudzîxᴀ′q g̣ᴀgā′n tūt. //
	\glpreamble	Chʼu tle ash een deikéet wudzix̱ák g̱agaan tóot. //
	\gla	Chʼu tle {} ash \rlap{een} @ {} {}
		{} \rlap{deikéet} @ {} {}
		\rlap{wudzix̱ák} @ {} @ {} @ {} @ {} @ {} +
		{} \rlap{g̱agaan} @ {} @ {} \rlap{tóot.} @ {} {} //
	\glb	chʼu tle {} ash ee -n {}
		{} deikée -t {}
		wu- d- s- i- \rt[²]{x̱ak} -μH
		{} g̱- \rt[¹]{gan} -μμL tú -t {} //
	\glc	just then {}[\pr{PP} \xx{3prx} \xx{base} -\xx{instr} {}]
		{}[\pr{PP} upward -\xx{pnct} {}]
		\xx{pfv}- \xx{pasv}- \xx{xtn}- \xx{stv}- \rt[²]{sting} -\xx{var}
		{}[\pr{PP} \xx{g̱cnj}- \rt[¹]{burn} -\xx{var} inside -\xx{pnct} {}] //
	\gld	just then {} her {} -with {}
		{} upward -to {}
		\rlap{\xx{zcnj}.\xx{pfv}.\xx{pasv}.snap·back} {} {} {} {} {}
		{} \rlap{sun} {} {} inside -to {} //
	\glft	‘Just then it was snapped back up with her into the sun.’
		//
\endgl
\xe

The verb \fm{wudzix̱ák} in (\lastx) is semantically unusual, but there is no better match in the lexical documentation.
\citeauthor{leer:1973} likewise interprets \citeauthor{swanton:1909}’s \orth{wudzîxᴀ′q} as \fm{wudzix̱ák} and also queries its meaning in this sentence \parencite{leer:1977}.
\citeauthor{swanton:1909}’s \orth{xᴀ′q} suggests a root \fm[?]{\rt{xaḵ}} but there is no such root \parencite[52]{leer:1978b}.
Given the variability and relative inaccuracy of \citeauthor{swanton:1909}’s transcriptions, there are at least 34 potentially matching roots based on phonological similarity which are listed below.

\begin{itemize}
\item	with initial \fm{h} /\ipa{h}/ (2):
		\fm{\rt{hakw}} ‘suspicious’, 
		\fm{\rt{hakw}} ‘encourage’
\item	with initial \fm{sh} /\ipa{ʃ}/ (4):
		\fm{\rt{shaḵ}} ‘winded, exhausted’,
		\fm{\rt{shaḵ}} ‘deny’,
		\fm{\rt{shikʼw}} \~\ \fm{\rt{shukʼ}} ‘cramp, shock’,
		\fm{\rt{shuḵ}} ‘laugh’
\item	with initial \fm{x} /\ipa{x}/ (5):
		\fm{\rt{xak}} ‘dessicated; mouth open’,
		\fm{\rt{xak}} ‘fragrant’,
		\fm{\rt{xakw}} ‘grind, whip, beat’,
		\fm{\rt{xekʼw}} ‘sip; inhale smoke’,
		\fm{\rt{xuk}} ‘dry’
\item	with initial \fm{xʼ} /\ipa{xʼ}/ (5):
		\fm{\rt{xʼak}} ‘rough’,
		\fm{\rt{xʼak}} ‘fish swim’,
		\fm{\rt{xʼaḵw}} ‘die off’,
		\fm{\rt{xʼaḵw}} ‘situate comfortably’,
		\fm{\rt{xʼuk}} ‘steam’
\item	with initial \fm{x̱} /\ipa{χ}/ (8):
		\fm{\rt{x̱ak}} ‘sting; snap back; shrivel, wither’,
		\fm{\rt{x̱akw}} ‘claw, cling’,
		\fm{\rt{x̱axʼ}} ‘split into layers’,
		\fm{\rt{x̱ax̱ʼ}} ‘dew’,
		\fm{\rt{x̱exʼw}} ‘pl.\ sleep’,
		\fm{\rt{x̱eḵ}} ‘wake early’,
		\fm{\rt{x̱ik}} ‘shoulder; flap arms’,
		\fm{\rt{x̱ux̱}} ‘ask’
\item	with initial \fm{x̱ʼ} /\ipa{χʼ}/ (7):
		\fm{\rt{x̱ʼak}} ‘valley’,
		\fm{\rt{x̱ʼakw}} ‘salmon turn red’
		\fm{\rt{x̱ʼax̱ʼ}} ‘convulse’,
		\fm{\rt{x̱ʼixʼ}} ‘squeeze through’,
		\fm{\rt{x̱ʼixʼw}} ‘wedge’,
		\fm{\rt{x̱ʼixʼw}} ‘eat chitons, gumboots’,
		\fm{\rt{x̱ʼix̱ʼ}} ‘scald, burn flesh’
\item	with initial \fm{ḵʼ} /\ipa{qʼ}/ (3):
		\fm{\rt{ḵʼekʼw}} ‘cut flesh with knife’,
		\fm{\rt{ḵʼekʼw}} ‘stride, take long steps’,
		\fm{\rt{ḵʼikʼ}} ‘cram’
\end{itemize}

The vast majority of these roots can be eliminated on semantic grounds because they have no conceivable connection to the motion of a canoe into the sky.
The remaining semantically plausible roots are \fm{\rt{x̱ak}} ‘sting; snap back; shrivel, wither’, \fm{\rt{x̱akw}} ‘claw, cling’, and \fm{\rt{ḵʼekʼw}} ‘stride, take long steps’.
The root \fm{\rt{ḵʼekʼw}} ‘stride, take long steps’ does not fit well because there is no evidence that the canoe has legs.
The root \fm{\rt{x̱akw}} ‘claw, cling’ likewise does not fit well because there is no sign of either the canoe having claws or of a hand coming down to grab it; furthermore ,\fm{\rt{x̱akw}} as a verb normally refers to hanging or clinging by claws, not to grabbing or manipulating with claws \parencite[791]{leer:1976}.

The only remaining root is \fm{\rt{x̱ak}} ‘sting; snap back; shrivel, wither’.
This root is documented with a few different meanings: \fm{aawax̱ák} ‘it stung him/her/it’, \fm{wudix̱ák} ‘it became shrunken’, \fm{wudzix̱ák} ‘it (limb) is withered, crippled’, \fm{awsix̱ák} ‘s/he/it shrunk it’, and \fm{át wudzix̱ák} ‘it (elastic object) snapped back and struck there’ \parencite[791]{leer:1976}.
Out of these various meanings the most plausible for (\lastx) seems to be that of ‘snap back’, as illustrated by the sentence \fm{ax̱ jínt wudzix̱ák} ‘it sprang back and hit my hand’ \parencite[190.2631]{story-naish:1973}.
The situation in (\lastx) must then be that the canoe has been sitting on the surface of the lake with an invisible elastic tether connecting it to the sun, and that this tether snaps the canoe back up into the sky.
The stinging effect of the snap is apparently irrelevant, with only the snapping effect being salient in this context.

\section{Paragraph 4}\label{sec:89-para-4}

\ex\label{ex:89-52-married-lukanaa}%
\exmn{254.9}%
\begingl
	\glpreamble	Łuqᴀnā′ ᴀsiyu′ hᴀs ā′waca yū′g̣ᴀgān ỵê′tq!î. //
	\glpreamble	Lukanáa ásíyú has aawasháa yú g̱agaan ÿátxʼi. //
	\gla	{} Lukanáa {}
		\rlap{ásíyú} @ {} @ {}
		has @ \rlap{aawasháa} @ {} @ {} @ {} @ {}
		{} yú \rlap{g̱agaan} @ {} @ {} \rlap{ÿátxʼi} @ {} @ {} {} //
	\glb	{} lukanáa {}
		á -sí -yú
		has= a- wu- i- \rt[²]{shaʷ} -μμH
		{} yú g̱- \rt[¹]{gan} -μμL ÿát -xʼ -í {} //
	\glc	{}[\pr{DP} \xx{name} {}]
		\xx{foc} -\xx{dub} -\xx{dist}
		\xx{plh}= \xx{arg}- \xx{pfv}- \xx{stv}- \rt[²]{woman} -\xx{var}
		{}[\pr{DP} \xx{dist} \xx{g̱cnj}- \rt[¹]{burn} -\xx{var} child -\xx{pl} -\xx{pss} {}] //
	\gld	{} \xx{name} {}
		\rlap{maybe.it.is} {} {}
		they \rlap{3>3.\xx{zcnj⁺}.\xx{pfv}.marry} {} {} {} {}
		{} that \rlap{sun} {} {} child -ren -of {} //
	\glft	‘Apparently it was a \fm{lukanáa} that they had married, those children of the sun.’
		//
\endgl
\xe

\ex\label{ex:89-53-marry-not-last-long}%
\exmn{254.9}%
\begingl
	\glpreamble	Hᴀs ᴀ′g̣acān ʟē′łsdjî hᴀs ułsā′kᵘ. //
	\glpreamble	Has ag̱asháanín tléil sdu jee has ooltsáakw. //
	\gla	{} Has @ \rlap{ag̱asháanín} @ {} @ {} @ {} @ {} @ {} @ {} {}
		tléil {} \rlap{sdu} @ {} \rlap{jee} @ {} {}
		has @ \rlap{ooltsáakw.} @ {} @ {} @ {} @ {} //
	\glb	{} has= a- {} g̱- \rt[²]{shaʷ} -μμH -n -ín {}
		tléil {} has= du jee {} {}
		has a- u- l- \rt[¹]{tsakw} -μμH //
	\glc	{}[\pr{CP} \xx{plh}= \xx{arg}- \xx{zcnj}\· \xx{mod}- \rt[²]{woman} -\xx{var} -\xx{nsfx} -\xx{ctng} {}]
		\xx{neg} {}[\pr{PP} \xx{plh}= \xx{3h·pss} poss’n \·\xx{loc} {}]
		\xx{plh}= \xx{arg}- \xx{irr}- \xx{csv}- \rt[²]{lasting} -\xx{var} //
	\gld	{} they \rlap{3>3.\xx{ctng}.marry} {} {} {} {} {} {} {}
		not {} \rlap{their} {} poss’n -in {}
		they \rlap{3>3.\xx{gcnj}.\xx{impfv}.make.last} {} {} {} {}  //
	\glft	‘Whenever they marry her, they do not have her lasting in their possession.’
		//
\endgl
\xe

\ex\label{ex:89-54-repeatedly-kill}%
\exmn{254.10}%
\begingl
	\glpreamble	ʟē sadjᴀ′qx. //
	\glpreamble	Tle s ajáḵx̱. //
	\gla	Tle s @ \rlap{ajáḵx̱.} @ {} @ {} @ {} //
	\glb	tle has= a- \rt[²]{jaḵ} -μH -x̱ //
	\glc	then \xx{plh}= \xx{arg}- \rt[²]{kill} -\xx{var} -\xx{rep} //
	\gld	then they \rlap{3>3.\xx{zcnj}.\xx{impfv}.kill.\xx{rep}} {} {} {} //
	\glft	‘They just repeatedly kill her.’
		//
\endgl
\xe

\ex\label{ex:89-55-pleased-to-marry}%
\exmn{254.10}%
\begingl
	\glpreamble	Ỵīdᴀ′tî ā′ỵî qo′aawe ctū′gas a′odîca. //
	\glpreamble	Ÿeedádi aaÿí ḵu.aa áwé sh tóog̱aa s awdisháa. //
	\gla	{} \rlap{Ÿeedádi} @ {} \rlap{aaÿí} @ {} {}
		ḵu.aa \rlap{áwé} @ {}
		{} sh \rlap{tóog̱aa} @ {} {}
		s @ \rlap{awdisháa.} @ {} @ {} @ {} @ {} @ {} //
	\glb	{} ÿeedát -í aa -í {}
		ḵu.aa á -wé
		{} sh tú -g̱áa {}
		has= a- wu- d- i- \rt[²]{shaʷ} -μμH //
	\glc	{}[\pr{DP} moment -\xx{pss} \xx{part} -\xx{pss} {}]
		\xx{contr} \xx{foc} -\xx{mdst}
		{}[\pr{PP} \xx{rflx·pss} inside -\xx{ades} {}]
		\xx{plh}= \xx{arg}- \xx{pfv}- \xx{mid}- \xx{stv}- \rt[²]{woman} -\xx{var} //
	\gld	{} moment -of one -of {}
		however \rlap{it.is} {}
		{} self’s inside -for {}
		they \rlap{\xx{zcnj⁺}.\xx{pfv}.\xx{mid}.marry} {} {} {} {} {} //
	\glft	‘It is the one of now however that they are pleased to have married.’
		//
\endgl
\xe

\ex\label{ex:89-56-killed-lukanaa}%
\exmn{254.11}%
\begingl
	\glpreamble	A′ya aq dᴀ′xawe hᴀs ā′wadjᴀq yū′łūqᴀna′. //
	\glpreamble	A ya.áakdáx̱ áwé has aawajáḵ yú lukanáa. //
	\gla	{} A \rlap{ya.áakdáx̱} @ {} {}
		\rlap{áwé} @ {}
		has @ \rlap{aawajáḵ} @ {} @ {} @ {} @ {}
		{} yú lukanáa. {} //
	\glb	{} a ya.áak -dáx̱ {}
		á -wé
		has= a- wu- i- \rt[²]{jaḵ} -μH
		{} yú lukanáa {} //
	\glc	{}[\pr{PP} \xx{3n·pss} room -\xx{abl} {}]
		\xx{foc} -\xx{mdst}
		\xx{plh}= \xx{arg}- \xx{pfv}- \xx{stv}- \rt[²]{kill} -\xx{var}
		{}[\pr{PP} \xx{dist} \xx{name} {}] //
	\gld	{} it room·for -from {}
		\rlap{it.is} {}
		they \rlap{3>3.\xx{zcnj}.\xx{pfv}.kill} {} {} {} {} 
		{} that \xx{name} {} //
	\glft	‘It is for room for her that they killed that lukanáa.’
		//
\endgl
\xe

\ex\label{ex:89-57-above-Tsimshian-land}%
\exmn{254.11}%
\begingl
	\glpreamble	Ts!ūtsx̣ᴀ′n ā′nî kînā′q! ayu′ hᴀs ā′wadjᴀq. //
	\glpreamble	Tsʼootsxán aaní kináaxʼ áyú has aawajáḵ. //
	\gla	{} Tsʼootsxán \rlap{aaní} @ {} \rlap{kináaxʼ} @ {} {}
		\rlap{áyú} @ {}
		has @ \rlap{aawajáḵ.} @ {} @ {} @ {} @ {} //
	\glb	{} Tsʼootsxán aan -í kináa -xʼ {}
		á -yú
		has= a- wu- i- \rt[²]{jaḵ} -μH //
	\glc	{}[\pr{PP} Tsimshian land -\xx{pss} above -\xx{loc} {}]
		\xx{foc} -\xx{dist}
		\xx{plh}= \xx{arg}- \xx{pfv}- \xx{stv}- \rt[²]{kill} -\xx{var} //
	\gld	{} Tsimshian land -of above -at {}
		\rlap{it.is} {}
		they \rlap{3>3.\xx{zcnj}.\xx{pfv}.kill} {} {} {} {} //
	\glft	‘Above Tsimshian country is where they killed her.’
		//
\endgl
\xe

\ex\label{ex:89-58-chop-small-pieces}%
\exmn{254.12}%
\begingl
	\glpreamble	Tc!aỵēguskî wucdᴀ′x awułîsū′. //
	\glpreamble	Chʼa ÿéi gusgéi wóoshdáx̱ awu̬lisʼóow. //
	\gla	Chʼa {} ÿéi @ \rlap{gusgéi} @ {} @ {} @ {} @ {} @ {} @ {} {}
		{} \rlap{wóoshdáx̱} @ {} {}
		\rlap{awu̬lisʼóow.} @ {} @ {} @ {} @ {} @ {} //
	\glb	chʼa {} ÿéi= g- u- d- s- \rt[¹]{ge} -μμH {} {}
		{} wóosh -dáx̱ {}
		a- wu- l- i- \rt[²]{sʼuʼw} -μμH //
	\glc	just {}[\pr{NP} thus= \xx{cmpv}- \xx{irr}- \xx{mid}- \xx{xtn}- \rt[¹]{big} -\xx{var} \·\xx{nmz} {}]
		{}[\pr{PP} \xx{recip} -\xx{abl} {}]
		\xx{arg}- \xx{pfv}- \xx{appl}- \xx{stv}- \rt[²]{chop} -\xx{var} //
	\gld	just {} thus \rlap{\xx{cmpv}.\xx{gcnj}.\xx{stv}·\xx{impfv}.small} {} {} {} {} {} {} {}
		{} ea·oth -from {}
		\rlap{3>3.\xx{ncnj}.\xx{pfv}.chop} {} {} {} {} {} //
	\glft	‘They just chopped her up into small pieces.’
		//
\endgl
\xe

\ex\label{ex:89-59-many-lukanaa}%
\exmn{254.12}%
\begingl
	\glpreamble	ᴀtcawe′ łuqᴀna′ ā′caỵandihēn. //
	\glpreamble	Ách áwé lukanáa áa shaÿandihéin. //
	\gla	{} \rlap{Ách} @ {} {}
		\rlap{áwé} @ {}
		{} lukanáa {}
		{} \rlap{áa} @ {} {}
		\rlap{shaÿandihéin.} @ {} @ {} @ {} @ {} @ {} @ {} @ {} //
	\glb	{} á -ch {}
		á -wé
		{} lukanáa {}
		{} á -μ {}
		sha- ÿ- n- d- i- \rt[¹]{haʰ} -eμH -n //
	\glc	{}[\pr{PP} \xx{3n} -\xx{erg} {}]
		\xx{foc} -\xx{mdst}
		{}[\pr{DP} \xx{name} {}]
		{}[\pr{PP} \xx{3n} -\xx{loc} {}]
		head- \xx{qual}- \xx{ncnj}- \xx{mid}- \xx{stv}- \rt[¹]{many} -\xx{var} -\xx{nsfx} //
	\gld	{} that -why {}
		\rlap{it.is} {}
		{} \xx{name} {}
		{} there -at {}
		\rlap{\xx{rlzn}.be.many} {} {} {} {} {} {} {} //
	\glft	‘That is why lukanáa have come to be so numerous there.’
		//
\endgl
\xe

\ex\label{ex:89-60-Tsimshian-town}%
\exmn{254.14}%
\begingl
	\glpreamble	Ts!ū′tsx̣ᴀn ā′nî ʟe k!awê′łguha. //
	\glpreamble	Tsʼootsxán aaní, tle áxʼ áwé l gooháa. //
	\gla	{} Tsʼootsxán \rlap{aaní,} @ {} {}
		tle {} \rlap{áxʼ} @ {} {}
		\rlap{áwé} @ {}
		l \rlap{gooháa.} @ {} @ {} @ {} //
	\glb	{} Tsʼootsxán aan -í {}
		tle {} á -xʼ {}
		á -wé
		l g- u- \rt[¹]{ha} -μμH //
	\glc	{}[\pr{DP} Tsimshian land -\xx{pss} {}]
		then {}[\pr{PP} \xx{3n} -\xx{loc} {}]
		\xx{foc} -\xx{mdst}
		\xx{neg} \xx{gcnj}- \xx{irr}- \rt[¹]{appear} -\xx{var} //
	\gld	{} Tsimshian town -of {}
		then {} it -at {}
		\rlap{it.is} {}
		not \rlap{\xx{gcnj}.\xx{stv}·\xx{impfv}.unobvious} {} {} {} //
	\glft	‘The Tsimshian town, it is obvious there.’
		//
\endgl
\xe

\ex\label{ex:89-61-fly-around-sun-says}%
\exmn{254.14}%
\begingl
	\glpreamble	Duī′c ā′nî akînā′ wug̣ax̣îx̣în yū′g̣ᴀgān ye yên dosqê′tc, //
	\glpreamble	Du éesh aaní, a kináa wug̱axíxín, yú g̱agaan yéi yandusḵéich: //
	\gla	{} {} Du éesh \rlap{aaní,} @ {} {}
			{} a \rlap{kináa} @ {} {}
			\rlap{wug̱axíxin,} @ {} @ {} @ {} @ {} @ {} @ {} {} +
		{} yú \rlap{g̱agaan} @ {} @ {} {}
		yéi @ \rlap{yandusḵéich:} @ {} @ {} @ {} @ {} @ {} @ {} //
	\glb	{} {} du éesh aan -í {}
			{} a kinaa -H {}
			ÿ- u- {} g̱- \rt[¹]{xix} -μμL -ín {}
		{} yú g̱- \rt[¹]{gan} -μμL {}
		yéi= ÿ- n- du- s- \rt[¹]{ḵa} -eμH -ch //
	\glc	{}[\pr{CP} {}[\pr{DP} \xx{3h·pss} father town -\xx{pss} {}]
			{}[\pr{DP} \xx{3n·pss} above \·\xx{loc} {}]
			\xx{qual}- \xx{irr}- \xx{zcnj}\· \xx{mod}- \rt[¹]{fall} -\xx{var} -\xx{ctng} {}]
		{}[\pr{DP} \xx{dist} \xx{g̱cnj}- \rt[¹]{burn} -\xx{var} {}]
		thus= \xx{qual}- \xx{ncnj}- \xx{4h·s}- \xx{csv}- \rt[¹]{say} -\xx{var} -\xx{rep} //
	\gld	{} {} her father town -of {}
			{} its above -at {}
			\rlap{around.\xx{ctng}.fall} {} {} {} {} {} {} {}
		{} that \rlap{sun} {} {} {}
		thus= \rlap{\xx{hab}.one.say} {} {} {} {} {} {} //
	\glft	‘Above her father’s town, whenever they fly around above it, the sun says to her:’
		//
\endgl
\xe

The phrase \fm{a kináa wug̱axíxín} in (\lastx) features the motion verb root \fm{\rt[¹]{xix}} ‘fall; move through space’ with the motion derivation \vbderiv{DP-xʼ ÿaa= \~\ ÿ-u-}{∅}{\fm{-ch} rep.}{obliquely, circuitously at DP}.
The word \fm{kináa} is composed of \fm{kinaa} ‘up’ (cf.\ e.g.\ \fm{kinaa.át} ‘coat’, \fm{kinaayéigi} ‘overseer spirit’) and the high tone \fm{-μH} allomorph of the locative suffix \fm{-xʼ}.

\ex\label{ex:89-62-heres-your-fathers-town}%
\exmn{255.1}%
\begingl
	\glpreamble	“Hē duī′c ā′nî.” //
	\glpreamble	«\!Héidu i éesh aaní.\!» //
	\gla	«\!\rlap{Héidu} @ {}
		{} i éesh \rlap{aaní.\!»} @ {} {} //
	\glb	\pqp{}héi -du
		{} i éesh aan -í {} //
	\glc	\pqp{}\xx{mprx} -\xx{locp}
		{}[\pr{DP} \xx{2sg·pss} father land -\xx{pss} {}] //
	\gld	\pqp{}here -is.at
		{} your father town -of {} //
	\glft	‘“Here is your father’s town.”’
		//
\endgl
\xe

\ex\label{ex:89-63-have-a-child}%
\exmn{255.2}%
\begingl
	\glpreamble	Wānanī′sawe ỵêt hᴀs ā′wa-ū. //
	\glpreamble	Wáa nanée sáwé ÿát has aawa.oo. //
	\gla	{} Wáa \rlap{nanée} @ {} @ {} @ {} {} \rlap{sáwé} @ {} @ {}
		{} ÿát {} has @ \rlap{aawa.oo.} @ {} @ {} @ {} @ {} //
	\glb	{} wáa n- \rt[¹]{niʰ} -μμH {} {} s- á -wé
		{} ÿát {} has= a- wu- i- \rt[²]{.u} -μμL //
	\glc	{}[\pr{CP} how \xx{ncnj}- \rt[¹]{happen} -\xx{var} \·\xx{sub} {}] \xx{q}- \xx{foc} -\xx{mdst}
		{}[\pr{DP} child {}] \xx{plh}= \xx{arg}- \xx{pfv}- \xx{stv}- \rt[²]{own} -\xx{var} //
	\gld	{} how \rlap{\xx{csec}.happen} {} {} \·while {} ever- \rlap{it.is} {} 
		{} child {} they \rlap{3>3.\xx{ncnj}.\xx{pfv}.have} //
	\glft	‘At some point they have a child.’
		//
\endgl
\xe

\ex\label{ex:89-64-canoe-lies-ahead}%
\exmn{255.2}%
\begingl
	\glpreamble	Hᴀsdutcukᴀ′tawe ỵiatᴀ′n hasduī′c yā′gu x̣ūts! yākᵘ. //
	\glpreamble	Hasdu shukát áwé ÿatán hasdu éesh yaagú, xóots yaakw. //
	\gla	{} \rlap{Hasdu} @ {} \rlap{shukát} @ {} @ {} {} \rlap{áwé} @ {}
		\rlap{ÿatán} @ {} @ {} +
		{} \rlap{hasdu} @ {} éesh \rlap{yaagú,} @ {} {}
		{} xóots yaakw. {} //
	\glb	{} has= du shu- ká -t {} á -wé
		ÿ- \rt[¹]{tan} -μH
		{} has= du éesh yaakw -í {}
		{} xóots yaakw {} //
	\glc	{}[\pr{PP} \xx{plh}= \xx{3h·pss} end- \xx{hsfc} -\xx{pnct} {}] \xx{foc} -\xx{mdst}
		face- \rt[²]{hdl·w/e} -\xx{var}
		{}[\pr{DP} \xx{plh}= \xx{3h·pss} father boat -\xx{pss} {}]
		{}[\pr{DP} br·bear boat {}] //
	\gld	{} \rlap{their} {} end- atop -at {} \rlap{it.is} {}
		\rlap{\xx{pos}·\xx{impfv}.sit·wood} {} {}
		{} \rlap{their} {} father canoe -of {}
		{} br·bear canoe {} //
	\glft	‘It is ahead of them that there lies their father’s canoe, a brown bear canoe.’
		//
\endgl
\xe

\ex\label{ex:89-65-canoe-can-hear}%
\exmn{255.3}%
\begingl
	\glpreamble	Qō′waᴀxtc yū′yākᵘ. //
	\glpreamble	Ḵuwa.áx̱ch yú yaakw. //
	\gla	\rlap{Ḵuwa.áx̱ch} @ {} @ {} @ {} @ {} 
		{} yú yaakw. {} //
	\glb	ḵu- i- \rt[²]{.ax̱} -μH -ch
		{} yú yaakw {} //
	\glc	\xx{4h·o}- \xx{stv}- \rt[²]{hear} -\xx{var} -\xx{rep}
		{} \xx{dist} boat {} //
	\gld	\rlap{ppl.\xx{gcnj}.\xx{impfv}.hear.\xx{rep}} {} {} {} {}
		{} that canoe {} //
	\glft	‘It can hear people, that canoe.’
		//
\endgl
\xe

\ex\label{ex:89-66-distribute}%
\exmn{255.3}%
\begingl
	\glpreamble	Āỵî′s ᴀt ka′ołiga. //
	\glpreamble	A ÿík s at kawligaa. //
	\gla	{} A ÿík {}
		s @ at @ \rlap{kawligaa.} @ {} @ {} @ {} @ {} @ {} //
	\glb	{} a ÿíᵏ {}
		has= at= k- wu- l- i- \rt[¹]{ga} -μμL //
	\glc	{}[\pr{PP} \xx{3n} within {}]
		\xx{plh}= \xx{4n·o}= \xx{qual}- \xx{pfv}- \xx{xtn}- \xx{stv}- \rt[¹]{distrib} -\xx{var} //
	\gld	{} it within {} they things\• \rlap{\xx{ncnj}.\xx{pfv}.distribute} {} {} {} {} {} //
	\glft	‘They distribute things within it.’
		//
\endgl
\xe

\FIXME{Other interpretations of \fm{\rt[¹]{ga}}?}

\ex\label{ex:89-67-grease-for-father}%
\exmn{255.4}%
\begingl
	\glpreamble	Hᴀsduwū′ xᴀ′ndî dᴀnē′t aỵîde′ ye wududzî′nê. //
	\glpreamble	Hasdu wóo x̱ánde daneit a ÿeedé yéi wududzinei. //
	\gla	{} {} \rlap{Hasdu} @ {} wóo {} \rlap{x̱ánde} @ {} {}
		{} daneit {}
		{} a \rlap{ÿeedé} @ {} {}
		yéi @ \rlap{wududzinei.} @ {} @ {} @ {} @ {} @ {} @ {} //
	\glb	{} {} has= du wóo {} x̱án -dé {}
		{} daneit {} 
		{} a ÿee -dé {}
		yéi= wu- du- d- s- i- \rt[¹]{neʰ} -μμL //
	\glc	{}[\pr{PP} {}[\pr{DP} \xx{plh}= \xx{3h·pss} father·in·law {}] near -\xx{all} {}]
		{}[\pr{DP} grease·box {}]
		{}[\pr{PP} \xx{3n·pss} below -\xx{all} {}]
		thus= \xx{pfv}- \xx{4h·s}- \xx{mid}- \xx{csv}- \xx{stv}- \rt[¹]{happen} -\xx{var} //
	\gld	{} {} \rlap{their} {} father·in·law {} near -to {}
		{} grease·box {}
		{} its below -to {}
		thus\• \rlap{\xx{zcnj}.\xx{pfv}.they.handle·\xx{pl}} {} {} {} {} {} {} //
	\glft	‘They put boxed grease within it for their father-in-law.’
		//
\endgl
\xe

\FIXME{Apparently \fm{a x̱ánde} used as a benefactive here?}

\FIXME{Etymology of \fm{daneit}.}

\FIXME{Perhaps \orth{aỵîde′} is actually \fm{a ÿíkde} instead of \fm{a ÿeedé}.}

\ex\label{ex:89-68-began-going}%
\exmn{255.4}%
\begingl
	\glpreamble	Hᴀsduī′n g̣onaye′ ūwagu′t. //
	\glpreamble	Hasdu een g̱unayéi uwagút. //
	\gla	{} \rlap{Hasdu} @ {} \rlap{een} @ {} {}
		g̱unayéi @ \rlap{uwagút.} @ {} @ {} @ {} //
	\glb	{} has= du ee -n {}
		g̱unayéi= u- i- \rt[¹]{gut} -μH //
	\glc	{}[\pr{PP} \xx{plh}= \xx{3h} \xx{base} -\xx{instr} {}]
		\xx{incep}= \xx{zpfv}- \xx{stv}- \rt[¹]{go·\xx{sg}} -\xx{var} //
	\gld	{} \rlap{them} {} \rlap{with} {} {}
		begin \rlap{\xx{zcnj}.\xx{pfv}.go·\xx{sg}} {} {} {} //
	\glft	‘It began going with them.’
		//
\endgl
\xe

\ex\label{ex:89-69-go-long-stop-short}%
\exmn{255.5}%
\begingl
	\glpreamble	Tc!ākᵘ yā′nagu′tîawe qox ᴀkū′dadjītc. //
	\glpreamble	Chʼáakw yaa nagúdi áwé ḵux̱ akoodadzéeych. //
	\gla	{} Chʼáakw yaa @ \rlap{nagúdi} @ {} @ {} @ {} {} 
		\rlap{áwé} @ {}
		ḵux̱ @ \rlap{akoodadzéeych.} @ {} @ {} @ {} @ {} @ {} @ {} //
	\glb	{} chʼáakw ÿaa= n- \rt[¹]{gut} -μH -í {}
		á -wé
		ḵúx̱= a- k- u- d- \rt[¹]{dziʼÿ} -μμH -ch //
	\glc	{}[\pr{CP} long·ago along= \xx{ncnj}- \rt[¹]{go·\xx{sg}} -\xx{var} -\xx{sub} {}]
		\xx{foc} -\xx{mdst}
		\xx{rev}= \xx{xpl}- \xx{qual}- \xx{zpfv}- \xx{mid}- \rt[¹]{balk} -\xx{var} -\xx{rep} //
	\gld	{} long·time along \rlap{\xx{ncnj}.\xx{prog}.go·\xx{sg}} {} {} {} {}
		\rlap{it.is} {}
		back \rlap{\xx{zcnj}.\xx{hab}.stop·short} {} {} {} {} {} {} //
	\glft	‘Having been going along for a long time it would stop short.’
		//
\endgl
\xe

The use of \fm{chʼáakw} ‘long ago’ in (\lastx) seems to be a bit unusual.
Normally it refers to a time long since passed, and thus can be translated as ‘in the old days’ or ‘in ancient times’.
Here instead it appears to be used to describe a long stretch of current time which is now more commonly expressed as e.g.\ \fm{wáa yeekuwáatʼ sá} ‘how long a time’.

The verb \fm{ḵux̱ akoodadzéeych} in (\lastx) is transcribed by \citeauthor{swanton:1909} as \fm{qox ᴀkū′dadjītc} which suggests a stem like \fm{jéech} from his \orth{djītc}.
The root seems to be \fm{\rt[¹]{dziʼÿ}} ‘balk, back up suddenly’, but it has been variously written as \fm{–dzée} \parencite[74.884]{story-naish:1973}, \fm{–jée} \parencite[09/113]{leer:1973}, and \fm{–dzéey} \parencite[09/113]{leer:1973}.
It is not clear if these different forms reflect genuine dialect variation or if they are simply mistranscriptions, but the only form reported later by \citeauthor{leer:1978b} is the root \fm{\rt[¹]{dziʼÿ}} \parencite[39]{leer:1978b}. 

\ex\label{ex:89-70-smash-grease-box}%
\exmn{255.5}%
\begingl
	\glpreamble	Xᴀtc u′tiyāng̣ahē′n, awe′ wē′yākᵘ dᴀnē′t hᴀs akust!ē′q!ᴀtc ayat!ᴀ′kq!ᵘ.
Yū′yākᵘ //
	\glpreamble	X̱ách óot yaan g̱ahéinín áwé wé yaakw, daneit has akoostʼéix̱ʼch a yatʼákwxʼ, yú yaakw. //
	\gla	X̱ách {} {} \rlap{óot} @ {} {}
			yaan @ \rlap{g̱ahéinín} @ {} @ {} @ {} @ {} @ {} {}
			\rlap{áwé} @ {} +
		{} wé yaakw, @ {}
		{} daneit {}
		has @ \rlap{akoostʼéix̱ʼch} {} {} {} {} {} {} +
		{} a \rlap{yatʼákwxʼ,} @ {} @ {} @ {} {}
		{} yú yaakw. {}  //
	\glb	x̱áju {} {} ú -t {}
			ÿaan= {} g̱- \rt[¹]{haʰ} -eμH -n -ín {}
			á -wé
		{} wé yaakw {}
		{} daneit {}
		has= a- k- u- s- \rt[²]{tʼex̱ʼ} -μμH -ch
		{} a ÿá- \rt{tʼakw} -μH -xʼ {}
		{} yú yaakw {} //
	\glc	actually {}[\pr{CP} {}[\pr{PP} \xx{3h} -\xx{pnct} {}]
			hunger= \xx{zcnj}- \xx{mod}- \rt[¹]{appear} -\xx{var} -\xx{nsfx} -\xx{ctng} {}]
			\xx{foc} -\xx{mdst}
		{}[\pr{DP} \xx{mdst} boat {}]
		{}[\pr{DP} grease·box {}]
		\xx{plh}= \xx{arg}- \xx{qual}- \xx{zpfv}- \xx{appl}- \rt[²]{pound} -\xx{var} -\xx{rep}
		{}[\pr{PP} \xx{3n·pss} face- \rt{slap} -\xx{var} -\xx{loc} {}]
		{}[\pr{DP} \xx{dist} boat {}] //
	\gld	actually {} {} him -to {}
			hunger \rlap{\xx{ctng}.appear} {} {} {} {} {} {}
			\rlap{it.is} {}
		{} that canoe {}
		{} grease·box {}
		they \rlap{3>3.\xx{zcnj}.\xx{hab}.\xx{appl}.smash} {} {} {} {} {} {}
		{} its \rlap{temple} {} {} -on {}
		{} that boat {} //
	\glft	‘Actually whenever he gets hungry, that canoe, they smash a grease box on its temple, that canoe.’
		//
\endgl
\xe

\label{note:089-70-u-pronoun}%
The word \fm{óot} [\ipa{ʔúːt}] in (\lastx) is an example of the archaic third person human pronoun \fm{ú} together with the punctual postposition \fm{-t} ‘to, at’.
The \fm{ú} pronoun is today only found in set phrases and song lyrics and is not understood when used in ordinary speech; modern speakers uniformly use \fm{du ee} instead.
In \citeauthor{swanton:1909}’s time it seems to have been old fashioned but still comprehensible since it shows up in several narratives where it does not seem to be part of a fixed expression.
\citeauthor{leer:1978b} suggests an etymological connection to Eyak \fm{ʔəw} ‘he/she’ and \fm{ʔu-} ‘his/her’ and and to Proto-Dene \fm[*]{wə-} ‘his/her’ \parencite[9]{leer:1978b}.
The same archaic \fm{ú} pronoun occurs again as part of the word \fm{óox̱} with the pertingent postposition \fm{-x̱} ‘of, contacting’ in (\ref{ex:89-115-laugh-at-him}) and (\ref{ex:89-148-run-out-to-get-laughed-at}).

\section{Paragraph 5}\label{sec:89-para-5}

Paragraph 4 in \citeauthor{swanton:1909}’s transcription runs very long, across three pages.
For ease of study this paragraph has been broken into three paragraphs in this retranscription and analysis.
Thus \citeauthor{swanton:1909}’s paragraph 4 corresponds to this new sequence of paragraphs 4, 5, and 6.
The particular sentence in (\nextx) was chosen as a paragraph break because it reflects a shift from the background discussion of the brown bear canoe back to the narrative progression.

\ex\label{ex:89-71-go-beach-below-father-in-law}%
\exmn{255.7}%
\begingl
	\glpreamble	āeg̣ayā′t has ū′waqox dūwu′. //
	\glpreamble	A eig̱ayáat has uwaḵúx̱ du wóo. //
	\gla	{} A \rlap{eig̱ayáat} @ {} @ {} {}
		has @ \rlap{uwaḵúx̱} @ {} @ {} @ {}
		{} du wóo. {} //
	\glb	{} a eeḵ- ÿáᵏ -t {}
		has= u- i- \rt[¹]{ḵux̱} -μH
		{} du wóo {} //
	\glc	{}[\pr{PP} \xx{3n·pss} beach- face -\xx{pnct} {}]
		\xx{plh}= \xx{zpfv}- \xx{stv}- \rt[¹]{go·boat} -\xx{var}
		{}[\pr{DP} \xx{3h·pss} father·in·law {}] //
	\gld	{} its beach- face -to {}
		they \rlap{\xx{pfv}.boat} {} {} {}
		{} his father·in·law //
	\glft	‘They boated to the beach below him, their father-in-law.’
		//
\endgl
\xe

\citeauthor{swanton:1909} includes \orth{Yū′yākᵘ} \fm{yú yaakw} ‘that canoe’ as the start of sentence (\ref{ex:89-71-go-beach-below-father-in-law}), but he does not include it in his translation “They came in front of their father-in-law’s house”.
Including this phrase in (\ref{ex:89-71-go-beach-below-father-in-law}) gives a sentence like ‘that canoe, to its beach below they boated, his father-in-law’ where the phrase \fm{a eig̱ayáat} ‘to its beach below’ could be interpreted as referring to either the canoe or the father-in-law.
If it refers to the father-in-law as implied by \citeauthor{swanton:1909}’s translation then \fm{yú yaakw} ‘that canoe’ has no grammatical function – it is neither a subject nor an object – and so is ungrammatical.
When \fm{yú yaakw} is moved to the end of (\ref{ex:89-70-smash-grease-box}) the result is repetitious but grammatical.

\ex\label{ex:89-72-knew-fathers-house}%
\exmn{255.7}%
\begingl
	\glpreamble	Awusikū′ duī′c hî′tî. //
	\glpreamble	Awu̬sikóo du éesh hídi. //
	\gla	\rlap{Awu̬sikóo} @ {} @ {} @ {} @ {} @ {}
		{} du éesh \rlap{hídi.} @ {} {} //
	\glb	a- wu- s- i- \rt[²]{kuʰ} -μμH
		{} du éesh hít -í {} //
	\glc	\xx{arg}- \xx{pfv}- \xx{xtn}- \xx{stv}- \rt[²]{know} -\xx{var}
		{}[\pr{DP} \xx{3h·pss} father house -\xx{pss} {}] //
	\gld	\rlap{3>3.\xx{pfv}.know} {} {} {} {} {}
		{} her father house -of {} //
	\glft	‘She knew her father’s house.’
		//
\endgl
\xe

\ex\label{ex:89-73-she-went-inland}%
\exmn{255.7}%
\begingl
	\glpreamble	ʟe āeg̣ayā′ dāq ūwagu′t. //
	\glpreamble	Tle a eig̱ayáa daaḵ uwagút. //
	\gla	Tle {} a \rlap{eig̱ayáa} @ {} @ {} {}
		daaḵ @ \rlap{uwagút.} @ {} @ {} @ {} //
	\glb	tle {} a eeḵ- ÿáᵏ -μ {}
		daaḵ= u- i- \rt[¹]{gut} -μH //
	\glc	then {}[\pr{PP} \xx{3n·pss} beach- face -\xx{loc} {}]
		\xx{abmar}= \xx{zpfv}- \xx{stv}- \rt[¹]{go·\xx{sg}} -\xx{var} //
	\gld	then {} its beach- face -at {}
		inland \rlap{\xx{pfv}.go·\xx{sg}} {} {} {} //
	\glft	‘She went up on the beach in front of it.’
		//
\endgl
\xe

The sentence in (\ref{ex:89-73-she-went-inland}) has the directional preverb \fm{daaḵ} ‘inland, on land away from a body of water’.
This is translated as ‘up’ because in English this preposition is conventionally used for the direction toward the inland away from a shoreline.
A more scrupulously literal translation could be ‘She went inland on the face of its beach’ but this seems needlessly pedantic.

\ex\label{ex:89-74-brother-went-inside}%
\exmn{255.8}%
\begingl
	\glpreamble	Duī′k!cawe nēłt!ā′ uwagu′t //
	\glpreamble	Du éekʼch áwé neil tʼaa.uwagút. //
	\gla	{} Du \rlap{éekʼch} @ {} {} \rlap{áwé} @ {}
		{} neil @ {} {}
		\rlap{tʼaa.uwagút.} @ {} @ {} @ {} @ {} //
	\glb	{} du éekʼ -ch {} á -wé
		{} neil -t {}
		tʼaa- u- i- \rt[¹]{gut} -μH //
	\glc	{}[\pr{DP} \xx{3h·pss} f’s·brother -\xx{erg} {}] \xx{foc} -\xx{mdst}
		{}[\pr{PP} inside -\xx{pnct} {}]
		back- \xx{zpfv}- \xx{stv}- \rt[¹]{go·\xx{sg}} -\xx{var} //
	\gld	{} her \rlap{brother} {} {} \rlap{it.is} {}
		{} inside -to {}
		\rlap{up.\xx{pfv}.go·\xx{sg}} {} {} {} {} //
	\glft	‘It was her brother who went up inside.’
		//
\endgl
\xe

\FIXME{Discuss \fm{tʼaa-}.
Move discussion from Ḵaadishaan’s \fm{Aakʼwtaatseen} 1938–1949.}

\ex\label{ex:89-75-my-sister-outside}%
\exmn{255.8}%
\begingl
	\glpreamble	“ᴀxʟ̣ā′k! gānt ūwagu′t.” //
	\glpreamble	«\!Ax̱ dlaakʼ gáant uwagút.\!» //
	\gla	{} \llap{«\!}Ax̱ dlaakʼ {}
		{} \rlap{gáant} @ {} {}
		\rlap{uwagút} @ {} @ {} @ {} //
	\glb	{} ax̱ dlaakʼ {}
		{} gáan -t {}
		u- i- \rt[¹]{gut} -μH //
	\glc	{}[\pr{DP} \xx{1sg·pss} m’s·sister {}]
		{}[\pr{PP} outside -\xx{pnct} {}]
		\xx{zpfv}- \xx{stv}- \rt[¹]{go·\xx{sg}} -\xx{var} //
	\gld	{} my sister {}
		{} outside -to {}
		\rlap{\xx{pfv}.go·\xx{sg}} {} {} {} //
	\glft	‘“My sister has come outside.”’
		//
\endgl
\xe

A typical English translation of (\ref{ex:89-75-my-sister-outside}) would be something like ‘my sister has gone outside’.
But this would imply that the woman went from inside the house to the outside which is not the case in this context; she has yet to actually step inside.
This incongruity is because the English verbs ‘come’ and ‘go’ entail an origo and a motion toward or away from that origo.
Tlingit does not have this origo-dependent directionality in its verb \fm{\rt[¹]{gut}} ‘sg.\ go’ which can equally describe motion toward or away from an origo.
The English translation for (\ref{ex:89-75-my-sister-outside}) is given with ‘come’ because in this context the brother is describing his sister’s arrival outside of the house from beyond.

\ex\label{ex:89-76-beaten-report-saw-sister}%
\exmn{255.9}%
\begingl
	\glpreamble	Akā′q!awe dudjā′q duʟā′tc tc!ākᵘ qot wudzîgī′tî duʟ̣ā′k!ᴀtc wᴀq kaodᴀnigītc. //
	\glpreamble	A káaxʼ áwé dujáaḵw du tláach, chʼáakw ḵut wudzigeedi du dlaakʼ
				sh waḵkawdaneegéech.//
	\gla	{} A \rlap{káaxʼ} @ {} @ {} {} \rlap{áwé} @ {}
		\rlap{dujáaḵw} @ {} @ {}
		{} du \rlap{tláach} @ {} {} +
		{} {} chʼáakw
			{} {} ḵut @ \rlap{wudzigeedi} @ {} @ {} @ {} @ {} @ {} @ {} {}
				du dlaakʼ {}
			sh @ \rlap{waḵkawdaneegéech} @ {} @ {} @ {} @ {} @ {} @ {} {} {} {} //
	\glb	{} a ká -μ -xʼ {} á -wé
		du- \rt[²]{jaḵw} -μμH
		{} du tláa -ch {}
		{} {} chʼáakw
			{} {} ḵut= wu- d- s- i- \rt[¹]{git} -μμL -i {}
				du dlaakʼ {}
			sh= waḵ- k- wu- d- \rt[²]{nik} -μμL -í {} -ch {} //
	\glc	{}[\pr{PP} \xx{3n·pss} \xx{hsfc} -\xx{unkn} -\xx{loc} {}] \xx{foc} -\xx{mdst}
		\xx{4h·s}- \rt[²]{beat} -\xx{var}
		{}[\pr{DP} \xx{3h·pss} mother -\xx{erg} {}]
		{}[\pr{PP} {}[\pr{CP} long·ago
			{}[\pr{DP} {}[\pr{CP} lost= \xx{pfv}- \xx{mid}- \xx{csv}- \xx{stv}-
					\rt[¹]{fall} -\xx{var} -\xx{rel} {}]
				\xx{3h·pss} m’s·sister {}]
			\xx{rflx·o}= eye- \xx{qual}- \xx{pfv}- \xx{mid}- \rt[²]{tell} -\xx{var} -\xx{sub} {}] -\xx{erg} {}] //
	\gld	{} its \rlap{depending} {} {} {} \rlap{it.is} {}
		\rlap{\xx{impfv}.one.beat} {} {}
		{} his \rlap{mother} {} {}
		{} {} long·ago
			{} {} lost\· \rlap{\xx{pfv}.fall} {} {} {} {} {} {} {}
				his sister {}
			self \rlap{eye.\xx{pfv}.report} {} {} {} {} {} \·that {} \·cause {} //
	\glft	‘Consequently he is beaten by his mother, because of reporting seeing of his sister who was long lost.’
		//
\endgl
\xe

The sentence in (\lastx) has at least four interesting features: (i) the underdescribed phrase \fm{a káaxʼ} ‘depending on it’, (ii) the use of subject \fm{du-} with an overt DP subject \fm{du tláach}, (iii) the verb \fm{ḵut wudzigeet}, and (iv) the otherwise unattested verb \fm{sh waḵkawdineek}.
The phrase \fm{a káaxʼ} is analyzed here as \fm{ká} ‘horizontal surface’ with an unknown vowel lengthening \fm{-μ} and the locative postposition \fm{-xʼ}.
It is not especially well documented, listed in \citeauthor{leer:1973}’s stem list under \fm{ká} with the form \fm{du kwéiyi káaxʼ} ‘according to his sighting’ \parencite[f06/3]{leer:1973} and given in \textcite[\textsc{t}·33]{leer:2001} as \fm{a káaxʼ} ‘subsisting on it; depending on it; modeled on it; using it as a guide’.
It is distinct from the combination of \fm{ká} and locative \fm{-xʼ} which is realized as \fm{káxʼ} [\ipa{kʰáxʼ}] not \fm{káaxʼ} [\ipa{kʰáːxʼ}] and which has the predictably compositional meaning ‘on top of’.
It is also distinct from the noun \fm{káaxʼ} ‘grouse; chicken’.
It is attested in a number of texts and occurs in the example sentence \fm{ḵutí káaxʼ át haa kawdiyáa} ‘when we travel is dependent on the weather’ \parencite[233.3317]{story-naish:1973}.
Since the form and meaning of \fm{a káaxʼ} are not straightforwardly derived from \fm{ká} and \fm{-xʼ}, it may be that this word is based on an otherwise unknown \fm{káa} or that it is an indivisible unit \fm{káaxʼ}.

The main clause in (\lastx) contains the verb \fm{dujáaḵw} ‘someone/people beat him’ along with the subject DP \fm{du tláa-ch} ‘his mother-\xx{erg}’.
This is unusual because usually the fourth person subject \fm{du-} ‘someone, people’ is argument-saturating like the first and second person subjects, thus blocking the appearance of a subject DP outside of the verb word.
This combination of \fm{du-} and a subject DP is an instance of the ‘obviative’ use of \fm{du-} to represent a third person subject that has a backgrounded status in the discourse.

The verb \fm{ḵut wudzigeedi} is a relative clause form of \fm{ḵut wudzigeet}.
This verb is based on the root \fm{\rt[¹]{git}} ‘animate fall’ that has a secondary metaphoric meaning of ‘act in a certain (oft.\ negative) way’.
Examples of the latter meaning include \fm{hasdu x̱ʼayáx̱ x̱wadzigeet} ‘I acted according to their instructions’ \parencite[72.861]{story-naish:1973}, \fm{du yaadé yóo x̱wdzigít} ‘I saluted him’ \parencite[111.1446]{story-naish:1973}, \fm{du yoo x̱ʼatángi a géidei has wudzigeet} ‘they violated his word’ \parencite[239.3397]{story-naish:1973}, \fm{yee tugéit shákdé x̱wadzigít} ‘perhaps I have offended you’ \parencite[141.1899]{story-naish:1973}, \fm{ux̱ kei gasgítch} ‘he’s always getting into mischief’ \parencite[f05/93]{leer:1973}, and \fm{ax̱ waḵshiyeexʼ has koosgídákw} ‘they are showing off in front of me’  \parencite[f05/94]{leer:1973}.
The \fm{ḵut=} preverb is a frozen form that is etymologically from areal \fm{ḵú} \~\ \fm{ḵu-} ‘area, extent, space’ and the punctual postposition \fm{-t} ‘to, at, around’.
It assigns \fm{g}-conjugation (hence the \fm{-μμL} stem for perfective aspect) and generally indicates an eventuality where an individual becomes lost or goes astray from a path.
The resulting combination of \fm{ḵut=} with \fm{d-s-\rt[¹]{git}} has the meaning ‘become lost, go off track’.

The phrase \fm{sh waḵkawdaneegích} in (\lastx) illustrates a verb that is not described elsewhere.
It is based on the root \fm{\rt[²]{nik}} ‘report, tell about’ (cf.\ noun \fm{neek} ‘news, report, gossip’) with the subordinate clause suffix \fm{-í} and the ergative/instrumental postposition \fm{-ch} ‘by, with, because of’; the latter together form an explanatory adjuct clause.
The basic verb on which this is built is \fm{(du een) akanéek} ‘s/he tells about it (to him/her)’ \parencite[282]{leer:1976}.
To this is added the incorporated noun \fm{waḵ-} from \fm{waaḵ} ‘eye’, and then the reflexive object \fm{chush=} \~\ \fm{sh=} ‘self’ which triggers the presence of middle voice \fm{d-}.
The resulting structure appears to mean something like ‘report self’s eyes’, i.e.\ to give an eyewitness report of something.

\ex\label{ex:89-77-mother-went-outside}%
\exmn{255.10}%
\begingl
	\glpreamble	Ā′yux wugū′t duʟā′. //
	\glpreamble	Áa yux̱ woogoot du tláa. //
	\gla	{} \rlap{Áa} @ {} {}
		yux̱ @ \rlap{woogoot} @ {} @ {} @ {}
		{} du tláa. {} //
	\glb	{} á -μ {}
		yux̱= wu- i- \rt[¹]{gut} -μμL
		{} du tláa {} //
	\glc	{}[\pr{PP} \xx{3n} -\xx{loc} {}]
		out= \xx{pfv}- \xx{stv}- \rt[¹]{go·\xx{sg}} -\xx{var}
		{}[\pr{DP} \xx{3h·pss} mother {}] //
	\gld	{} there -at {}
		out \rlap{\xx{pfv}.go·\xx{sg}} {} {} {}
		{} his mother {} //
	\glft	‘His mother went outside there.’
		//
\endgl
\xe

\ex\label{ex:89-78-true-people-ashore}%
\exmn{255.10}%
\begingl
	\glpreamble	Xᴀtc q!ē′g̣a ᴀsiyu′ dᴀ′qde hᴀs duła′t . //
	\glpreamble	X̱ách xʼéig̱aa ásíyú dáḵde has dul.aat. //
	\gla	X̱ách xʼéig̱aa \rlap{ásíyú} @ {} @ {}
		{} \rlap{dáḵde} @ {} {} has @ \rlap{dul.aat.} @ {} @ {} @ {} @ {} //
	\glb	x̱áju xʼéig̱aa á -sí -yú
		{} dáaḵ -dé {}
		has= du- d- l- \rt[¹]{.at} -μμL //
	\glc	actually truly \xx{foc} -\xx{dub} -\xx{dist}
		{}[\pr{PP} inland -\xx{all} {}]
		\xx{plh}= \xx{4h·s}- \xx{mid}- \xx{xtn}- \rt[¹]{go·\xx{pl}} -\xx{var} //
	\gld	actually truly \rlap{apparently.it.is} {} {}
		{} inland -to {}
		they \rlap{\xx{impfv}.people.go·\xx{pl}.\xx{rep}} {} {} {} {} //
	\glft	‘Actually it apparently is true that people are coming ashore.’
		//
\endgl
\xe

The word \fm{xʼéig̱aa} in (\lastx) is a fairly common adverb which means ‘truly, really, factually’.
It appears to be composed of a stem \fm{xʼéi} and the adessive postposition \fm{-g̱áa} ‘nearby, for (obtaining)’, but the stem does not otherwise occur so the whole form is frozen.
Supporting its frozen and thus synchronically indivisible status is the form \fm{xʼeig̱aànax̱} [\ipa{xʼeː.qaʰ.naχ}] ‘really’ attested in Tongass Tlingit \parencite[f04/34]{leer:1973} where the perlative postposition \fm{-náx̱} would otherwise be ungrammatical since Tlingit prohibits multiple postpositions on a single noun.
The noun \fm{xʼaa} \~\ \fm{xʼaan} ‘point’ may be etymologically related but the connection is unclear.

\ex\label{ex:89-79-people-cant-see}%
\exmn{255.11}%
\begingl
	\glpreamble	Hᴀs qo′a ʟēł hᴀs dutī′n. //
	\glpreamble	Hás ḵu.aa tléil has duteen. //
	\gla	{} Hás {} ḵu.aa tléil has @ \rlap{duteen.} @ {} @ {} @ {} @ {} //
	\glb	{} hás {} ḵu.aa tléil has= u- du- d- \rt[²]{tin} -μμL //
	\glc	{}[\pr{DP} \xx{3h·pl} {}] \xx{contr} \xx{neg} \xx{plh}= \xx{irr}- \xx{4h·s}- \xx{mid}- \rt[²]{see} -\xx{var} //
	\gld	{} them {} however not them \rlap{\xx{stv}·\xx{impfv}.ppl.see} {} {} {} {} //
	\glft	‘Them however people can’t see.’
		//
\endgl
\xe

\ex\label{ex:89-80-like-moonbeams}%
\exmn{255.11}%
\begingl
	\glpreamble	Xᴀtc dē′tc!a ᴀ′sîyu yūᴀłdî′s q!os yêx kātuwā(ỵ)ati. //
	\glpreamble	X̱ách de chʼa á ásíyú yú aldís x̱ʼusyee yáx̱ ḵaa tuwáa ÿatee. //
	\gla	X̱ách de chʼa {} á {} \rlap{ásíyú} @ {} @ {}
		{} {} yú {} \rlap{aldís} @ {} @ {} @ {} @ {} @ {} {}
			\rlap{x̱ʼusyee} @ {} {} yáx̱ {}
		{} ḵaa \rlap{tuwáa} @ {} @ {} {}
		\rlap{ÿatee.} @ {} @ {} //
	\glb	x̱áju de chʼa {} á {} á -sí- yú
		{} {} yú {} a- d- l- \rt[¹]{dis} -μH {} {}
			x̱ʼoos- ÿee {} yáx̱ {}
		{} ḵaa tú- ÿá -μ {}
		i- \rt[¹]{tiʰ} -μμL //
	\glc	actually now just {}[\pr{DP} \xx{3n} {}] \xx{foc} -\xx{dub} -\xx{dist}
		{}[\pr{PP} {}[\pr{DP} \xx{dist} {}[\pr{CP} \xx{xpl}- \xx{xtn}- \xx{mid}- \rt[¹]{moon} -\xx{var} \·\xx{nmz} {}]
			foot- below {}] \xx{sim} {}]
		{}[\pr{PP} \xx{4h·pss} mind- face -\xx{loc} {}]
		\xx{stv}- \rt[¹]{be} -\xx{var} //
	\gld	actually now just {} it {} \rlap{it.is.apparently} {} {}
		{} {} those {} \rlap{\xx{impfv}.moon·shine} {} {} {} {} -ing {}
			foot- below {} like {}
		{} ppl’s mind- face -at {}
		\rlap{\xx{stv}·\xx{impfv}.be} {} {} //
	\glft	‘Actually now it just is apparently that they are like those beams of moon-shining to people’s minds.’
		//
\endgl
\xe

The parentheses around \orth{ỵ} in \orth{kātuwā(ỵ)ati} are present in the original publication from \citeauthor{swanton:1909}.
Because we do not have access to his original notes we cannot say whether this was added by his consultant during his time in Sitka or if it was added later during the preparation of the publication.
\citeauthor{swanton:1909} does not offer any explanation for the parenthesization, but we can reasonably infer that it reflects the common elision of \fm{ÿ} [\ipa{ɰ}] between two \fm{a} [\ipa{a}] vowels.

The phrase \fm{yú aldís x̱ʼusyee} in (\lastx) is an interesting example of using a deverbal nominalization rather than an ordinary noun.
The ordinary noun \fm{dís} ‘moon’ is the usual Tlingit term for Earth’s natural satellite both in myth and in modern description; like in many other languages it also refers to the timespan between new moons, i.e.\ a month.
The verb \fm{aldís} means something like ‘moon shines’ or ‘it moons’, similar to \fm{awdigaan} ‘sun has shone; it has become sunny’.
It is an activity imperfective belonging to the \fm{g̱}-conjugation class with the perfective form \fm{awdlidées} ‘moon shone’ \parencite[335]{leer:1976}.
The phrase \fm{yú aldís x̱ʼusyee} is descriptively equivalent to \fm{yú dís x̱ʼusyee} ‘below the moonbeams’ (lit.\ ‘below those feet of the moon’).
The use here of a nominalization is a poetic figure of speech.

\ex\label{ex:89-81-haul-go-outside}%
\exmn{255.12}%
\begingl
	\glpreamble	Dāq kᴀdudjē′ławe yū′ᴀtłaᴀt ā′yux ā′wagut. //
	\glpreamble	Daaḵ kadujéil áwé yú at.la.át, áa yux̱ aawagoot. //
	\gla	{} Daaḵ @ \rlap{kadujéil} @ {} @ {} @ {} @ {} @ {} @ {} {}
		\rlap{áwé} @ {} +
		{} yú {} \rlap{at.la.át,} @ {} @ {} @ {} @ {} {} {}
		{} \rlap{áa} @ {} {}
		yux̱ @ \rlap{aawagút.} @ {} @ {} @ {} @ {} //
	\glb	{} daaḵ= k- {} du- d- \rt[²]{jel} -μμH {} {}
		á -wé
		{} yú {} at= l- \rt[¹]{.at} -μH {} {} {}
		{} á -μ {}
		yux̱= a- wu- i- \rt[¹]{gut} -μμL //
	\glc	{}[\pr{CP} inland= \xx{qual}- \xx{zcnj}\· \xx{4h·s}- \xx{mid}- \rt[¹]{lug} -\xx{var} \·\xx{sub} {}]
		\xx{foc} -\xx{mdst}
		{}[\pr{DP} \xx{dist} {}[\pr{NP} \xx{4n·o}= \xx{csv}- \rt[¹]{go·\xx{pl}} -\xx{var} \·\xx{nmz} {}] {}]
		{}[\pr{PP} \xx{3n} -\xx{loc} {}]
		outside= \xx{4h·s}- \xx{pfv}- \xx{stv}- \rt[¹]{go·\xx{sg}} -\xx{var} //
	\gld	{} inland \rlap{\xx{csec}.ppl.haul} {} {} {} {} {} {} {}
		\rlap{it.is} {}
		{} that {} thing\· \rlap{\xx{impfv}.hdl·\xx{pl}} {} {} -ing {} {}
		{} there -to {}
		outside \rlap{one.\xx{pfv}.go·\xx{sg}} {} {} {} {} //
	\glft	‘Them having hauled it ashore, that baggage, she went outside.’
		//
\endgl
\xe

\ex\label{ex:89-82-there-is-nothing-said}%
\exmn{255.13}%
\begingl
	\glpreamble	“ʟēł da ᴀt,” yū′siaodudziqa. //
	\glpreamble	«\!Tléil daa át\!» yóo s yawdudziḵaa. //
	\gla	{} \llap{«\!}Tléil daa át {}
		yóo @ s @ \rlap{yawdudziḵaa.} @ {} @ {} @ {} @ {} @ {} @ {} @ {} //
	\glb	{} tléil daa át {}
		yóo= has= ÿ- wu- du- d- s- i- \rt[¹]{ḵa} -μμL //
	\glc	{}[\pr{CP} \xx{neg} what \xx{4n} {}]
		\xx{quot}= \xx{plh}= \xx{qual}- \xx{pfv}- \xx{4h·s}- \xx{mid}- \xx{csv}- \xx{stv}- \rt[¹]{say} -\xx{var} //
	\gld	{} not what thing {}
		\xx{quot}= they= \rlap{\xx{pfv}.ppl.say} {} {} {} {} {} {} {} //
	\glft	‘“There is nothing” they said to her.’
		//
\endgl
\xe

\ex\label{ex:89-83-his-wife-said}%
\exmn{255.13}%
\begingl
	\glpreamble	Ducᴀ′t ye ỵawaqa′, //
	\glpreamble	Du shát yéi ÿaawaḵaa: //
	\gla	{} Du shát {} yéi @ \rlap{ÿaawaḵaa:} @ {} @ {} @ {} @ {} //
	\glb	{} du shát {} yéi= ÿ- wu- i- \rt[¹]{ḵa} -μμL //
	\glc	{}[\pr{DP} \xx{3h·pss} wife {}] thus= \xx{qual}- \xx{pfv}- \xx{stv}- \rt[¹]{say} -\xx{var} //
	\gld	{} his wife {} thus \rlap{\xx{pfv}.say} {} {} {} {} //
	\glft	‘His wife said:’
		//
\endgl
\xe

\ex\label{ex:89-84-them-like-moonbeams}%
\exmn{255.13}%
\begingl
	\glpreamble	“Detc!a′a-awe′ weᴀłdî′s-q!os ỵi yêx ỵatî′.//
	\glpreamble	«\!De chʼa á áwé wé aldís x̱ʼusÿee yáx̱ ÿatee. //
	\gla	«\!De chʼa {} á {} \rlap{áwé} @ {} +
		{} {} wé {} \rlap{aldís} @ {} @ {} @ {} @ {} @ {} {}
			\rlap{x̱ʼusÿee} @ {} {} yáx̱ {}
		\rlap{ÿatee.} @ {} @ {} //
	\glb	\pqp{}de chʼa {} á {} á -wé
		{} {} yú {} a- d- l- \rt[¹]{dis} -μH {} {}
			x̱ʼoos- ÿee {} yáx̱ {}
		i- \rt[¹]{tiʰ} -μμL //
	\glc	\pqp{}already just {}[\pr{DP} \xx{3n} {}] \xx{foc} -\xx{mdst}
		{}[\pr{PP} {}[\pr{DP} \xx{dist} {}[\pr{CP} \xx{xpl}- \xx{xtn}- \xx{mid}- \rt[¹]{moon} -\xx{var} \·\xx{nmz} {}]
			foot- below {}] \xx{sim} {}]
		\xx{stv}- \rt[¹]{be} -\xx{var} //
	\gld	\pqp{}now just {} it {} \rlap{it.is} {}
		{} {} those {} \rlap{\xx{impfv}.moon·shine} {} {} {} {} -ing {}
			foot- below {} like {}
		\rlap{\xx{stv}·\xx{impfv}.be} {} {} //
	\glft	‘“Now it is they who are like beams of moon-shining.’
		//
\endgl
\xe

\ex\label{ex:89-85-them-like-moonbeams}%
\exmn{255.14}%
\begingl
	\glpreamble	Yē ỵana-isᴀqa a dāq ỵiᴀ′dî.” //
	\glpreamble	Yéi ÿanaÿsaḵá ‹\!áa daaḵ ÿi.aadí.\!›\!» //
	\gla	Yéi @ \rlap{ÿanaÿsaḵá} @ {} @ {} @ {} @ {} @ {}
		{} {} \llap{‹\!}\rlap{áa} @ {} {} daaḵ @ \rlap{ÿi.aadí.\!›\!»} @ {} @ {} @ {} @ {} {} //
	\glb	yéi= ÿ- n- ÿi- s- \rt[¹]{ḵa} -μH
		{} {} á -μ {}
			daaḵ= ÿi- \rt[¹]{.at} -μμL -í {} //
	\glc	thus= \xx{qual}- \xx{ncnj}- \xx{2pl·s}- \xx{csv}- \rt[¹]{say} -\xx{var}
		{}[\pr{CP} {}[\pr{PP} \xx{3n} -\xx{loc} {}]
			inland= \xx{zcnj}\· \xx{2pl·s}- \rt[¹]{go·\xx{pl}} -\xx{var} -\xx{sub} {}] //
	\gld	thus \rlap{\xx{imp}.you·\xx{pl}.say} {} {} {} {} {}
		{} {} there -at {}
			inland \rlap{\xx{imp}.you·\xx{pl}.go·\xx{pl}} {} {} {} {} {} //
	\glft	‘Tell them ‘come ashore there’.”’
		//
\endgl
\xe

\ex\label{ex:89-86-thus-they-told-them}%
\exmn{255.14}%
\begingl
	\glpreamble	Ye ỵa′odudzîqa. //
	\glpreamble	Yéi ÿawdudziḵaa. //
	\gla	Yéi @ \rlap{ÿawdudziḵaa.} @ {} @ {} @ {} @ {} @ {} @ {} @ {} //
	\glb	yéi= ÿ- wu- du- d- s- i- \rt[¹]{ḵa} -μμL //
	\glc	thus= \xx{qual}- \xx{pfv}- \xx{4h·s}- \xx{mid}- \xx{csv}- \xx{stv}- \rt[¹]{say} -\xx{var} //
	\gld	thus \rlap{\xx{pfv}.people.say} {} {} {} {} {} {} {} //
	\glft	‘Thus they told them.’
		//
\endgl
\xe

\ex\label{ex:89-87-came-up}%
\exmn{256.1}%
\begingl
	\glpreamble	Dāq uwaᴀ′t. //
	\glpreamble	Daaḵ uwa.át. //
	\gla	Daaḵ @ \rlap{uwa.át.} @ {} @ {} @ {} //
	\glb	daaḵ= u- i- \rt[¹]{.at} -μH //
	\glc	inland= \xx{zpfv}- \xx{stv}- \rt[¹]{go·\xx{pl}} -\xx{var} //
	\gld	inland \rlap{\xx{pfv}.go·\xx{pl}} {} {} {} //
	\glft	‘They came up.’
		//
\endgl
\xe

\section{Paragraph 6}\label{sec:89-para-6}

As noted at the beginning of paragraph 5, the original paragraph 4 in \citeauthor{swanton:1909}’s transcription runs very long, across three pages.
It has been split into three paragraphs, namely 4, 5, and 6 which is this one.
The particular sentence in (\nextx) was chosen as a paragraph break because it reflects a scene shift from outside on the shore to inside of the house in the village.

\ex\label{ex:89-88-sunbeams-poke-inside}%
\exmn{256.1}%
\begingl
	\glpreamble	Tc!uʟe′ g̣ᴀgā′n q!ōs wᴀ′sâ nēł kᴀx dugu′g̣un yū′g̣ᴀgan q!ōs yū′cāwᴀt tuwᴀ′nq!  //
	\glpreamble	Chʼu tle g̱agaan x̱ʼoos, wáa sá neil kax̱dugúg̱ún, yú g̱agaan x̱ʼoos yú shaawát tuwánxʼ; //
	\gla	Chʼu tle {} \rlap{g̱agaan} @ {} @ {} x̱ʼoos, {}
		{} wáa sá {}
		{} \rlap{neil} @ {} {}
		\rlap{kax̱dugúg̱ún,} @ {} @ {} @ {} @ {} @ {} @ {}
		{} yú \rlap{g̱agaan} @ {} @ {} x̱ʼoos {}
		{} yú \rlap{shaawát} @ {} \rlap{tuwánxʼ;} @ {} @ {} {} //
	\glb	chʼu tle {} g̱- \rt[¹]{gan} -μμL x̱ʼoos {}
		{} wáa sá {}
		{} neil -t {}
		k- {} g̱- du- \rt[²]{guḵ} -μH -ín
		{} yú g̱- \rt[¹]{gan} -μμL x̱ʼoos {}
		{} yú sháaʷ- ÿát tú- wán -xʼ {} //
	\glc	just then {}[\pr{DP} \xx{g̱cnj}- \rt[¹]{burn} -\xx{var} foot {}]
		{}[\pr{QP} how \xx{q} {}]
		{}[\pr{PP} inside -\xx{pnct} {}]
		\xx{qual}- \xx{zcnj}\· \xx{mod}- \xx{4h·s}- \rt[²]{poke} -\xx{var} -\xx{ctng}
		{}[\pr{DP} \xx{dist} \xx{g̱cnj}- \rt[¹]{burn} -\xx{var} foot {}]
		{}[\pr{DP} \xx{dist} woman- child mind- edge -\xx{loc} {}] //
	\gld	just then {} \rlap{sun} {} {} beam {}
		{} how ever {}
		{} inside -to {} \rlap{\xx{ctng}.ppl.poke} {} {} {} {} {} {}
		{} those \rlap{sun} {} {} beam {}
		{} that \rlap{girl} {} mind- edge -at {} //
	\glft	‘So then sunbeams, however they poke them inside, those sunbeams in that girl’s mind;’
		//
\endgl
\xe

\ex\label{ex:89-89-son-sunbeam}%
\exmn{256.2}%
\begingl
	\glpreamble	hᴀsduỵī′t k!ᴀtsk!ᵘ ts!u q!wᴀsỵê′ ᴀłts!u′ g̣ᴀgā′n q!ōs yêx ỵatî′. //
	\glpreamble	hasdu ÿéet kʼátskʼu tsú x̱ʼaseiÿí, á tsú g̱agaan x̱ʼoos yáx̱ ÿatee. //
	\gla	{} {} \rlap{hasdu} @ {} ÿéet \rlap{kʼátskʼu} @ {} @ {} @ {} {}
			tsú \rlap{x̱ʼaseiÿí} @ {} @ {} {}
		{} á {} tsú +
		{} {} \rlap{g̱agaan} @ {} @ {} x̱ʼoos {} yáx̱ {}
		\rlap{ÿatee.} @ {} @ {} //
	\glb	{} {} has= du ÿéet kʼí- ÿáts -kʼʷ -í {}
			tsú x̱ʼé- sei -í {}
		{} á {} tsú
		{} {} g̱- \rt[¹]{gan} -μμL x̱ʼoos {} yáx̱ {}
		i- \rt[¹]{tiʰ} -μμL //
	\glc	{}[\pr{DP} {}[\pr{NP} \xx{plh}= \xx{3h·pss} son base- child -\xx{dim} -\xx{pss} {}]
			also mouth- front -\xx{pss} {}]
		{}[\pr{DP} \xx{3n} {}] also
		{}[\pr{PP} {}[\pr{DP} \xx{g̱cnj}- \rt[¹]{burn} -\xx{var} foot {}] \xx{sim} {}]
		\xx{stv}- \rt[¹]{be} -\xx{var} //
	\gld	{} {} \rlap{their} {} son \rlap{little} {} {} {} {}
			also mouth- front -of {}
		{} it {} too
		{} {} \rlap{sun} {} {} beam {} like {}
		\rlap{\xx{stv}·\xx{impfv}.be} {} {}  //
	\glft	‘their little son also in front, it too is like a sunbeam.’
		//
\endgl
\xe

\FIXME{Unusual structure of initial DP.
Is this really \fm{tsú} additive focus on just the possessive NP?
Could this be instead [\pr{DP} \fm{hasdu ÿéet kʼátskʼu}] \fm{tsú} [\pr{DP} \fm{\_{} x̱ʼaseiyí}] where there’s a displaced possessor?
I don’t think that’s normally possible.}

The noun \fm{x̱ʼaseiyí} [\ipa{χʼʷà.ˈsèː.jí}] in (\lastx) is quite rare and not well documented.
It is a compound of the noun \fm{x̱ʼé} ‘mouth’ and the relational noun \fm{seiyí} ‘area in front and below of; shelter of’.
The labialization of \fm{x̱ʼé} as \fm{x̱ʼwa-} is a fairly common irregularity in many modern varieties of Tlingit but it has not been described or explored.
The noun \fm{seiyí} is well attested in structures like \fm{aas seiyí} ‘shelter of a tree’ and \fm{shaa seiyí} ‘area protected by mountain’ \parencite[09/48]{leer:1973}.
The combination with \fm{x̱ʼé} is attested only three times: here, in the \fm{Shaakanaayí} narrative \FIXME{xref}, and in \fm{Naakil.aan} Frank Dick Sr.’s telling of Woman Who Married the Bear in the line \fm{Tle yá du x̱ʼaseiyíxʼ áwé wdzigeet} “It dropped right in front of her” \parencite[217.443]{dauenhauer:1987}.
\citeauthor{leer:1978b} suggests an etymological connection between \fm{seiyí} and Proto-Dene fm[*]{tsəŋʸ} \parencite[37]{leer:1978b} and gives \fm{sé} ‘voice’ as a distinct noun.

\ex\label{ex:89-90-}%
\exmn{256.3}%
\begingl
	\glpreamble	Tc!uʟe′ nēłq! yên hᴀs qē′awe tsa wᴀ′sa ᴀtū′nᴀx kês yê′nᴀx hᴀs ỵî yᴀ′xawe ỵasiate′ yuqog̣ā′s!.//
	\glpreamble	Chʼu tle neilxʼ yan has ḵéi áwé tsa wáa sá a tóonáx̱ \{keis yánáx̱ has ÿi yáx̱\} áwé yéi s yatee, yú ḵugwáasʼ.//
	\gla	{} Chʼu tle {} \rlap{neilxʼ} @ {} {}
			yan @ has @ \rlap{ḵéi} @ {} @ {} @ {} {}
			\rlap{áwé} @ {} 
		tsa {} wáa sá {} 
		{} a \rlap{tóonáx̱} @ {} {}
		\{keis yánáx̱ has ÿi yáx̱\}
		\rlap{áwé} @ {}
		yéi @ s @ \rlap{yatee,} @ {} @ {}
		{} yú \rlap{ḵugwáasʼ.} @ {} @ {} {} //
	\glb	{} chʼu tle {} neil -xʼ {}
			ÿán= has= {} \rt[¹]{ḵe} -μμH {} {}
			á -wé
		tsa {} wáa sá {}
		{} a tú -náx̱ {}
		\{keis yánáx̱ has ÿi ÿáx̱\}
		á -wé
		yéi= has= i- \rt[¹]{tiʰ} -μμL
		{} yú ḵu- \rt[¹]{gwasʼ} -μμH {} //
	\glc	{}[\pr{CP} just then {}[\pr{PP} inside -\xx{loc} {}]
			\xx{term}= \xx{plh}= \xx{zcnj}\· \rt[¹]{sit·\xx{pl}} -\xx{var} \·\xx{sub} {}]
			\xx{foc} -\xx{mdst}
		then {}[\pr{QP} how \xx{q} {}]
		{}[\pr{PP} \xx{3n·pss} inside -\xx{perl} {}]
		{} {} {} {} {}
		\xx{foc} -\xx{mdst}
		thus= \xx{plh}= \xx{stv}- \rt[¹]{be} -\xx{var}
		{}[\pr{DP} \xx{dist} \xx{areal}- \rt[¹]{fog} -\xx{var} {}] //
	\gld	{} just then {} inside -at {}
			done \rlap{\xx{csec}.sit·\xx{pl}} {} {} {} {}
			\rlap{it.is} {}
		just {} how ever {} {}
		{} its inside -thru {}
		?? ?? ?? ?? ??
		\rlap{it.is} {}
		thus they \rlap{\xx{stv}·\xx{impfv}.be} {} {}
		{} that \rlap{fog} {} {} {} //
	\glft	‘\{…\}’\newline
		“out from.there they being like”\newline
		“After they were seated inside of the house they began to appear as if coming out of a fog.”
		//
\endgl
\xe

\FIXME{Figure out what \orth{kês yê′nᴀx hᴀs ỵî yᴀ′x} is supposed to be.}

\ex\label{ex:89-91-let-her-eat}%
\exmn{256.5}%
\begingl
	\glpreamble	“ᴀtg̣ᴀxā′ dê ᴀxsī′k!ᵘ” yū′ỵawaqa yuānqā′wo. //
	\glpreamble	«\!At g̱ax̱aa dé ax̱ séekʼ\!» yóo ÿaawaḵaa yú aanḵáawu. //
	\gla	{} \llap{«\!}At @ \rlap{g̱ax̱aa} @ {} @ {} @ {}
			dé {} ax̱ \rlap{séekʼ\!»} @ {} {} {}
		yóo @ \rlap{ÿaawaḵaa} @ {} @ {} @ {} @ {} +
		{} yú \rlap{aanḵáawu.} @ {} @ {} {} //
	\glb	{} at= {} g̱- \rt[²]{x̱a} -μμL
			dé {} ax̱ sée -kʼ {} {}
		yóo= ÿ- wu- i- \rt[¹]{ḵa} -μμL
		{} yú aan- ḵáaʷ -í {} //
	\glc	{}[\pr{CP} \xx{4n·o}= \xx{zcnj}\· \xx{mod}- \rt[²]{eat} -\xx{var}
			already {}[\pr{DP} \xx{1sg·pss} daughter -\xx{dim} {}] {}]
		\xx{quot}= \xx{qual}- \xx{pfv}- \xx{stv}- \rt[¹]{say} -\xx{var}
		{}[\pr{DP} \xx{dist} land- man -\xx{pss} {}] //
	\gld	{} thing= \rlap{\xx{hort}.eat} {} {} {}
			now {} my \rlap{daughter} {} {} {}
		\xx{quot} \rlap{\xx{pfv}.say} {} {} {} {}
		{} that \rlap{aristocrat} {} {} {} //
	\glft	‘“Let my daughter eat things now” that aristocrat said.’
		//
\endgl
\xe

\ex\label{ex:89-92-ran-for-water}%
\exmn{256.5}%
\begingl
	\glpreamble	ʟᴀx ckᴀstā′x̣wâ awe′ wudjîx̣ī′x̣ h`ᴀsduq!oe′s hī′ng̣a. //
	\glpreamble	Tlax̱ sh kastáa xwáa áwé wujixeex hasdu x̱ʼéis héeng̱aa. //
	\gla	Tlax̱ {} {} sh @ \rlap{kastáa} @ {} @ {} @ {} @ {} @ {} {} xwáa {} \rlap{áwé} @ {}
		\rlap{wujixeex} @ {} @ {} @ {} @ {} @ {}
		{} \rlap{hasdu} @ {} \rlap{x̱ʼéis} @ {} {}
		{} \rlap{héeng̱aa.} @ {} {} //
	\glb	tlax̱ {} {} sh k- d- s- \rt[¹]{taʰ} -μμH {} {} xwáa {} á -wé
		wu- d- sh- i- \rt[¹]{xix} -μμL
		{} has= du x̱ʼé =ÿís {}
		{} héen -g̱áa {} //
	\glc	very {}[\pr{DP} {}[\pr{CP} \xx{rflx·o}= \xx{qual}- \xx{mid}- \xx{csv}-
			\rt[¹]{sleep·\xx{sg}} -\xx{var} \·\xx{rel} {}] fellow {}] \xx{foc}- \xx{mdst}
		\xx{pfv}- \xx{mid}- \xx{pej}- \xx{stv}- \rt[¹]{fall·anim} -\xx{var}
		{}[\pr{PP} \xx{plh}= \xx{3h·pss} mouth =\xx{ben} {}]
		{}[\pr{PP} water -\xx{ades} {}] //
	\gld	very {} {} self\• \rlap{\xx{impfv}.make·lay} {} {} {} {} \·that {} fellow {} \rlap{it.is} {}
		\rlap{\xx{pfv}.run·\xx{sg}} {} {} {} {} {}
		{} \rlap{their} {} mouth -for {}
		{} water -for {} //
	\glft	‘Right then a fellow lying down ran to get water for them.’
		//
\endgl
\xe

\ex\label{ex:89-93-put-up-fishhawk-quill}%
\exmn{256.6}%
\begingl
	\glpreamble	Ax ke ā′watᴀn kîdjū′k qî′naỵî. //
	\glpreamble	Áxʼ kei aawatán gijook ḵínaÿi. //
	\gla	{} \rlap{Áxʼ} @ {} {} kei @ \rlap{aawatán} @ {} @ {} @ {} @ {}
		{} gijook \rlap{ḵínaÿi.} @ {} @ {} @ {} {} //
	\glb	{} á -xʼ {} kei= a- wu- i- \rt[²]{tan} -μH
		{} gijook \rt[¹]{ḵin} -μH -aa -í {} //
	\glc	{}[\pr{PP} \xx{3n} -\xx{loc} {}] up= \xx{arg}- \xx{pfv}- \xx{stv}- \rt[²]{hdl·w/e} -\xx{var}
		{}[\pr{DP} fishhawk \rt[¹]{fly·\xx{sg}} -\xx{var} -\xx{nmz} -\xx{pss} {}] //
	\gld	{} there -at {} up \rlap{3>3.\xx{pfv}.handle·long} {} {} {} {}
		{} fishhawk \rlap{pinion} {} {} -of {} //
	\glft	‘He picked up there a fishhawk pinion.’
		//
\endgl
\xe

\citeauthor{swanton:1909} translates (\lastx) as referring to “her husband”, presumably referring to the son of the sun that married the protagonist.
This is not present in the Tlingit translation and so is not given in the English translation here, but \citeauthor{swanton:1909}’s interpretation is plausible and likely informed by his consultant who in this case was also the narrator.

The bird represented by the word \fm{gijook} \~\ \fm{kijook} in (\lastx) is unclear and may differ depending on dialect.
\citeauthor{leer:1973} in earlier materials gives it as an unspecific “mountain hawk” \parencites[10.172]{leer:1973}[48]{leer:1978b}, but later identifies it as a golden eagle (\species{Aquila}{chrysaetos}[L.]).
\citeauthor{swanton:1909} translates it as “fish-hawk” which is a local English term for the osprey (\species{Pandion}{haliaetus}[L.]).
\citeauthor{leer:1978b} gives \fm{atsʼáts} as ‘osprey’ from Mabel Johnson \FIXME{ref?}; this has the appearance of a loan from a Dene language but there has been no further investigation of its etymology.
The noun \fm{shaayáal} has been described as a “kind of hawk” \parencites[03/24]{leer:1973}[\textsc{m}·138]{leer:2001} which my consultants have identified as an osprey or a Northern goshawk (\species{Accipiter}{gentilis}[L.]) or even a sharp-shinned hawk (\species{Accipiter}{striatus}[Viellot 1808]).
Other possibilities include the peregrine falcon (\species{Falco}{peregrinus}[Tunstall 1771]), the red-tailed hawk (\species{Buteo}{jamaicensis}[Gmelin 1788]), and rough-legged hawk (\species{Buteo}{lagopus}[Pontoppidan 1763]).
Additionally there is a noun \fm{g̱aay} which is said to refer to an unspecified kind of eagle.
The terminology for raptors aside from the bald eagle (\fm{chʼáakʼ}) is still unsettled and needs investigation.

The noun \fm{ḵínaa} in (\lastx) is a relatively uncommon noun that refers to the pinions or outermost primary wing feathers (remiges) of a bird.
The noun is derived from the root \fm{\rt[¹]{ḵin}} ‘sg.\ fly’ and so clearly refers to feathers specifically adapted for flight.
This specific term is often subsumed in ordinary usage under the more generic term \fm{tʼaaw} ‘feather’ which can refer to wing feathers (remiges), tail feathers (rectrices), and any other pennaceous (quilled) feathers in contrast with \fm{x̱ʼwáalʼ} ‘down’.
\citeauthor{swanton:1909} translates \fm{ḵínaa} as ‘quill’ which reflects current variation in the term’s application to the entire feather in some contexts and to the quill stripped of barbs.
In this context it is not clear whether the pinion has been stripped of barbs or not, so the translation here uses ‘pinion’.

\ex\label{ex:89-94-poke-in}%
\exmn{256.7}%
\begingl
	\glpreamble	ᴀqadê′ awatsā′q. //
	\glpreamble	A kaadé aawatsaaḵ. //
	\gla	{} A \rlap{kaadé} @ {} {} \rlap{aawatsaaḵ.} @ {} @ {} @ {} @ {} //
	\glb	{} a ká -dé {} a- wu- i- \rt[²]{tsaḵ} -μμL //
	\glc	{}[\pr{PP} \xx{3n·pss} \xx{hsfc} -\xx{all} {}] \xx{arg}- \xx{pfv}- \xx{stv}- \rt[²]{poke} -\xx{var} //
	\gld	{} its atop -to {} \rlap{3>3.\xx{pfv}.poke} {} {} {} {} //
	\glft	‘He poked it into it.’
		//
\endgl
\xe

\ex\label{ex:89-95-}%
\exmn{256.7}%
\begingl
	\glpreamble	Yū yên kā′watᴀn xēʟ! qāx ʟēł cka′ wucku′k yuqā′. //
	\glpreamble	Yoo yan kaawatán x̱éilʼ kaax̱, tléil sh kawushkook yú ḵáa. //
	\gla	Yoo @ yan @ \rlap{kaawatán} @ {} @ {} @ {} @ {}
		{} x̱éilʼ \rlap{kaax̱,} @ {} {} +
		tléil sh @ \rlap{kawushkook} @ {} @ {} @ {} @ {} @ {}
		{} yú ḵaa. {} //
	\glb	yoo= ÿán= k- wu- i- \rt[¹]{tan} -μH
		{} x̱éilʼ ká -dáx̱ {}
		tléil sh= k- u- wu- sh- \rt[²]{kuk} -μμL
		{} yú ḵáaʷ {} //
	\glc	\xx{alt}= \xx{term}= \xx{qual}- \xx{pfv}- \xx{stv}- \rt[¹]{bend} -\xx{var}
		{}[\pr{DP} slime \xx{hsfc} -\xx{abl} {}]
		\xx{neg} \xx{rflx·o}= \xx{qual}- \xx{irr}- \xx{pfv}- \xx{pej}- \rt[²]{??} -\xx{var}
		{}[\pr{DP} \xx{dist} man {}] //
	\gld	 //
	\glft	‘’\newline
		“If it bent over on account of the wet the man had not behaved himself.”
		//
\endgl
\xe

The noun \fm{x̱éilʼ} [\ipa{χéːɬʼ}] ‘slime’ in (\lastx) also occurs as \fm{x̱éelʼ} [\ipa{χíːɬʼ}] without uvular lowering.
It is based on the root \fm{\rt{x̱ilʼ}} \~\ \fm{\rt{x̱elʼ}} ‘slime’ which can form both intransitive and transitive verbs.
Compare intransitive \fm{tʼukanéiyi du x̱ʼéi shax̱éelʼ nooch} ‘a baby’s mouth is always slimy’ \parencite[196.2729]{story-naish:1973} with transitive \fm{ax̱ kʼoodásʼi akawshix̱éelʼ} ‘she slimed up my shirt’ \parencite[196.2731]{story-naish:1973}.
There is also an intransitive stative verb with \fm{l-} instead of \fm{sh-} as shown by \fm{ishḵeen lix̱éelʼi} ‘rock cod is slimy’ \parencite[196.2730]{story-naish:1973}.
There are a few roots which are possibly related: \fm{\rt{x̱ʼilʼ}} \~\ \fm{\rt{x̱ʼelʼ}} ‘slip, slide’ (erroneously given by \citeauthor{story-naish:1973} as \fm{\rt{x̱ilʼ}} i.e.\ homophonous with ‘slime’), \fm{\rt{x̱il}} \~\ \fm{\rt{x̱el}} ‘foam’,  \fm{\rt{x̱ish}} ‘foam’, and \fm{\rt{xelʼ}} ‘drool’.
The noun \fm{x̱éilʼ} ‘slime’ is stereotypically used for the mucus exuded by fish, but it is also used for the mucus of slugs and for the sticky scum of algae on and around water among other slimy substances.
The noun \fm{g̱eetlʼ} \~\ \fm{g̱eitlʼ} is more often used for mucus from human lungs and \fm{dlóoḵ} for dried mucus from the eyes or nose (also occ.
\fm{lukasʼeexí} or \fm{lukanóodzi}), but there is also \fm{lux̱éelʼ} specifically for nasal mucus.

\FIXME{Unidentified verb root – \fm{\rt{kuḵ}} ‘pile fall, fish rush’, \fm{\rt{kuḵ}} ‘pit’, \fm{\rt{ḵuk}} ‘box’, \fm{\rt{ḵuḵ}} ‘cough’ \fm{\rt{kuʰ}} ‘know’, \fm{\rt{gu}} ‘joy’, \fm{\rt{gu}} ‘poke, penetrate’ (note \fm{daa.itnagóowu} ‘actions’) \fm{\rt{guk}} ‘know how’, \fm{\rt{guḵ}} ‘push; run’, \fm{\rt{guḵ}} ‘dried fish eggs’, \fm{\rt{g̱uk}} ‘clutch, squeeze in hand’.}

\ex\label{ex:89-96-examined-sent-ygbro}%
\exmn{256.8}%
\begingl
	\glpreamble	Cunāỵê′t yên da yē′g̣awetsa, du ī′k! k!ᴀtsk!ᵘ kā′waqa. //
	\glpreamble	\{Shunaaÿát\} yandayéiḵ áwé tsá, du éekʼ kʼátskʼu akaawaḵaa. //
	\gla	{} {} \rlap{Shunaaÿát} @ {} {} \rlap{yandayéiḵ} @ {} @ {} @ {} @ {} @ {} {} 
		\rlap{áwé} @ {} tsá, +
		{} du éekʼ \rlap{kʼátskʼu} @ {} @ {} @ {} {}
		\rlap{akaawaḵaa.} @ {} @ {} @ {} @ {} @ {}//
	\glb	{} {} shunaaÿá -t {} ÿ- n- d- \rt[²]{ÿeḵ} -μμH {} {}
		á -wé tsá
		{} du éekʼ kʼí- ÿáts -kʼʷ -í {}
		a- k- wu- i- \rt[²]{ḵa} -μμL //
	\glc	{}[\pr{CP} {}[\pr{PP} \xx{unid} -\xx{pnct} {}] \xx{qual}- \xx{ncnj}- \xx{mid}- \rt[²]{\xx{unid}}
			-\xx{var} \·\xx{sub} {}]
		\xx{foc} -\xx{mdst} then
		{}[\pr{DP} \xx{3h·pss} f’s·bro base- child -\xx{dim} -\xx{pss} {}]
		\xx{arg}- \xx{qual}- \xx{pfv}- \xx{stv}- \rt[²]{say} -\xx{var} //
	\gld	{} {} “everyone” -at {} \rlap{\xx{csec}.“examine”} {} {} {} {} {} {}
		\rlap{it.is} {} then
		{} her brother \rlap{little} {} {} {} {}
		\rlap{3>3.\xx{pfv}.send} {} {} {} {} {} //
	\glft	‘Everyone having examined it then, she sent her younger brother.’
		//
\endgl
\xe

The word \citeauthor{swanton:1909} transcribes as \orth{Cunāỵê′t} in (\lastx) is mysterious.
It appears to reflect something like \fm{shunaaÿá}, but this noun is found only in a couple of compounds: \fm{tlʼátgi shunaayá} ‘where land drops off’ and \fm{héen shunaayá} ‘where ground undewater drops off’; these probably refer to the same thing from different deictic perspectives.
In these compound nouns \fm{shunaayá} may be analyzed as \fm{shú} ‘end’ with \fm{naa} and \fm{ÿá} ‘face’, but the \fm{naa} is unidentified; it is not \fm{\rt{naʰ}} ‘damp; drink; rub oil’, \fm{\rt{naʷ}} ‘die, corpse; inherit’, \fm{\rt{na}} ‘handle bundles’, \fm{\rt{naʼ(ÿ)}} ‘send, order’, \fm{náʼ} ‘here, take this’, \fm{naa} ‘clan, nation’, \fm{náa} ‘covering, draped’, nor \fm{naa} ‘upriver’.
This unidentified \fm{naa} could be a contracted form of \fm{niÿaa} ‘direction’.
In any case, the meaning of \fm{shunaayá} as something like ‘drop off of surface’ is not easily reconciled with \citeauthor{swanton:1909}’s gloss “everyone” which we would instead expect to be something like \fm{chʼa ldakát ḵáa} ‘just every man’.
The translation maintains \citeauthor{swanton:1909}’s use of ‘everyone’ but the gloss indicates that this word is still unidentified.

The verb \fm{yandayéiḵ} in (\lastx) is of uncertain interpretation.
The form appears to be a consecutive with \fm{n-} conjugation and \fm{-μμH} stem variation in an unmarked adjunct clause followed by \fm{áwé}.
\citeauthor{swanton:1909}’s gloss of his \orth{yên da yē′g̣awetsa} is “there when they examined” which suggests a verb like \fm{a daa yas.éix̱} ‘s/he examines it’ (lit.\ ‘moves face around it’), but this does not fit with his transcription.
The possible roots are \fm{\rt{ÿiḵ}} \~\ \fm{\rt{ÿeḵ}} ‘pull, hoist, draw, mark’ (also \fm{ÿéeḵ} ‘foremost/sternmost place in canoe’), \fm{\rt{ÿiḵ}} \~\ \fm{\rt{ÿeḵ}} ‘bite, carry in mouth’, \fm{\rt{yek}} ‘roomy’, \fm{\rt{yek}} ‘animated, alert; fast’ (also \fm{yéik} ‘shamanic spirit’, \fm{kayéik} ‘noise’), \fm{\rt{yeʼk}} ‘stagger’, and \fm{ÿík} ‘inside concavity’.
None of these seems to be related to \citeauthor{swanton:1909}’s “examine” gloss.
As with the puzzling noun \fm{shunaaÿá}, \citeauthor{swanton:1909}’s translation is maintained but the gloss indicates that this root is unidentified.

\ex\label{ex:89-97-forever-pack-water}%
\exmn{256.9}%
\begingl
	\glpreamble	Tc!uʟē′xdê hīn hᴀ′sduq!oē′dê ā′wayᴀ hᴀ′sduīk! k!ᴀ′tsk!ᵘ. //
	\glpreamble	Chʼu tleix̱ de héen hasdu x̱ʼéide aawayaa hasdu éek kʼátskʼu.  //
	\gla	Chʼu tleix̱ de {} héen {}
		{} \rlap{hasdu} @ {} \rlap{x̱ʼéide} @ {} {}
		\rlap{aawayaa} @ {} @ {} @ {} @ {} +
		{} \rlap{hasdu} @ {} éekʼ \rlap{kʼátskʼu.} @ {} @ {} @ {} {} //
	\glb	chʼu tleix̱ de {} héen {}
		{} has= du x̱ʼé -dé {}
		a- wu- i- \rt[²]{ya} -μμL
		 {} has= du éekʼ kʼí- ÿáts -kʼʷ -í {} //
	\glc	just forever now {}[\pr{DP} water {}]
		{}[\pr{PP} \xx{plh}= \xx{3h·pss} mouth -\xx{all} {}]
		\xx{arg}- \xx{pfv}- \xx{stv}- \rt[²]{pack} -\xx{var}
		{}[\pr{DP} \xx{plh}= \xx{3h·pss} f’s·bro base- child -\xx{dim} -\xx{pss} {}] //
	\gld	just forever now {} water {}
		{} \rlap{their} {} mouth -to {}
		\rlap{3>3.\xx{pfv}.pack} {} {} {} {}
		{} \rlap{their} {} brother \rlap{little} {} {} {} {} //
	\glft	‘Forever after he packed water for them, their younger brother.’
		//
\endgl
\xe

\ex\label{ex:89-98-bro-lost-she-carry-water}%
\exmn{256.10}%
\begingl
	\glpreamble	Qot g̣agū′dawe duī′k! hīn g̣ā a′watan q!īca′ duxo′xq!ᵘ wᴀ′nq!es. //
	\glpreamble	Ḵut g̱agóot áwé du éekʼ, héeng̱aa aawataan xʼeesháa du x̱úx̱xʼu wán x̱ʼéis. //
	\gla	{} Ḵut @ \rlap{g̱agóot} @ {} @ {} @ {} {}
		\rlap{áwé} @ {}
		{} du éekʼ, {}
		{} \rlap{héeng̱aa} @ {} {}
		\rlap{aawataan} @ {} @ {} @ {} @ {}
		{} \rlap{xʼeesháa} @ {} @ {} {}
		{} du \rlap{x̱úx̱xʼu} @ {} @ \•wán \rlap{x̱ʼéis.} @ {} {} //
	\glb	{} ḵut= g̱- \rt[¹]{gut} -μμH {} {}
		á -wé
		{} du éekʼ {}
		{} héen -g̱áa {}
		a- wu- i- \rt[²]{tan} -μμL
		{} \rt[]{xʼish} -μμL -áa {}
		{} du x̱úx̱ -xʼ =ÿán x̱ʼé =ÿís {} //
	\glc	{}[\pr{CP} lost= \xx{g̱cnj}- \rt[¹]{go·\xx{sg}} -\xx{var} \·\xx{sub} {}]
		\xx{foc} -\xx{mdst}
		{}[\pr{DP} \xx{3h·pss} f’s·bro {}]
		{}[\pr{DP} water -\xx{ades} {}]
		\xx{arg}- \xx{pfv}- \xx{stv}- \rt[²]{hdl·w/e} -\xx{var}
		{}[\pr{DP} \rt{contain} -\xx{var} -\xx{nmz} {}]
		{}[\pr{DP} \xx{3h·pss} husband -\xx{pl} =\xx{plk} mouth =\xx{ben} {}] //
	\gld	{} lost\• \rlap{\xx{csec}.go·\xx{sg}} {} {} {} {}
		\rlap{it.is} {}
		{} her brother {}
		{} water -for {}
		\rlap{3>3.\xx{pfv}.handle·w/e} {} {} {} {}
		{} \rlap{bucket} {} {} {}
		{} her \rlap{husbands} {} {} mouth -for {} //
	\glft	‘Having gone lost, her brother, for water she carried a bucket for her husbands.’
		//
\endgl
\xe

The noun \fm{xʼeesháa} ‘bucket’ in (\lastx) is of unknown etymology although it is a fairly common, everyday word.
It appears to be formed from a root \fm{\rt{xʼish}} with \fm{-μμL} stem variation and the instrument suffix \fm{-áa}, but that root is not otherwise attested.
There is a homophonous root \fm{\rt[²]{xʼish}} ‘remove skin, flay’ but this has a slightly different instrument noun \fm{xʼíshaa} ‘skinning knife’ and is probably unrelated.
The root \fm{\rt[¹]{xʼis}} ‘swell up; boil, abscess, folliculitis’ is potentially connected, perhaps by the traditional use of stomachs or skins as buckets, but this is merely conjecture.

\ex\label{ex:89-99-twice-man-grabs-hand}%
\exmn{256.11}%
\begingl
	\glpreamble	Dᴀ′xda hī′ng̣a gū′dawe ᴀcdjī′n awu′łîcāt qā hīn q!ēq!. //
	\glpreamble	Dax̱dahéen héeng̱aa nagóot áwé ash jín awu̬lisháat ḵáa héen x̱ʼéixʼ. //
	\gla	{} \rlap{Dax̱dahéen} @ {} {} \rlap{héeng̱aa} @ {} {}
			\rlap{nagóot} @ {} @ {} @ {} {}
		\rlap{áwé} @ {}
		{} ash jín {}
		\rlap{awu̬lisháat} @ {} @ {} @ {} @ {} @ {} 
		{} ḵáa {}
		{} héen \rlap{x̱ʼéixʼ.} @ {} {} //
	\glb	{} déix̱ -dahéen {} héen -g̱áa {}
			n- \rt[¹]{gut} -μμH {} {}
		á -wé
		{} ash jín {}
		a- wu- l- i- \rt[²]{shaʼt} -μμH
		{} ḵáa {}
		{} héen x̱ʼé -xʼ {} //
	\glc	{}[\pr{CP} two -times {}[\pr{PP} water -\xx{ades} {}]
			\xx{ncnj}- \rt[¹]{go·\xx{sg}} -\xx{var} \·\xx{sub} {}]
		\xx{foc} -\xx{mdst}
		{}[\pr{DP} \xx{3prx·pss} hand {}]
		\xx{arg}- \xx{pfv}- \xx{xtn}- \xx{stv}- \rt[²]{grab} -\xx{var}
		{}[\pr{DP} man {}]
		{}[\pr{PP} water mouth -\xx{loc} {}] //
	\gld	{} \rlap{twice} {} {} water -for {}
			\rlap{\xx{csec}.go·\xx{sg}} {} {} {} {}
		\rlap{it.is} {}
		{} her hand {}
		\rlap{3>3.\xx{pfv}.\xx{xtn}.grab} {} {} {} {} {}
		{} man {}
		{} river mouth -at {} //
	\glft	‘Having gone twice for water, a man grabbed her hand at the mouth of the river.’
		//
\endgl
\xe

\ex\label{ex:89-100-husbands-poke-quill}%
\exmn{256.11}%
\begingl
	\glpreamble	Tc!uʟe′ nēł awî′sīn′eawe duxo′xq!ᵘ awᴀ′n xᴀ′nq! aqadē′ uduwatsā′k kîdjū′k q!î′naỵî. //
	\glpreamble	Chʼu tle neil awu̬sinéi áwé du x̱úx̱xʼu wán x̱ánxʼ a kaadé wduwatsaaḵ gijook ḵínaÿi. //
	\gla	{} Chʼu tle {} \rlap{neil} @ {} {}
			\rlap{awu̬sinéi} @ {} @ {} @ {} @ {} @ {} @ {} {}
		\rlap{áwé} @ {} +
		{} du \rlap{x̱úx̱xʼu} @ {} @ \•wán \rlap{x̱ánxʼ} @ {} {}
		{} a \rlap{kaadé} @ {} {}
		\rlap{wduwatsaaḵ} @ {} @ {} @ {} @ {} @ {} +
		{} gijook \rlap{ḵínaÿi.} @ {} @ {} @ {} {} //
	\glb	{} chʼu tle {} neil -t {}
			a- wu- s- i- \rt[¹]{neʰ} -μμH {} {}
		á -wé
		{} du x̱úx̱ -xʼ =ÿán x̱án -xʼ {}
		{} a ká -dé {}
		wu- du- d- i- \rt[²]{tsaḵ} -μμL
		{} gijook \rt[¹]{ḵin} -μH -aa -í {} //
	\glc	{}[\pr{CP} just then {}[\pr{PP} inside -\xx{pnct} {}]
			\xx{arg}- \xx{pfv}- \xx{csv}- \xx{stv}- \rt[¹]{happen} -\xx{var} \·\xx{sub} {}]
		\xx{foc} -\xx{mdst}
		{}[\pr{PP} \xx{3h·pss} husband -\xx{pl} =\xx{plk} near -\xx{loc} {}]
		{}[\pr{PP} \xx{3n·pss} \xx{hsfc} -\xx{all} {}]
		\xx{pfv}- \xx{4h·s}- \xx{mid}- \xx{stv}- \rt[²]{poke} -\xx{var}
		{}[\pr{DP} fishhawk \rt[¹]{fly·\xx{sg}} -\xx{var} -\xx{nmz} -\xx{pss} {}] //
	\gld	{} just then {} inside -to {}
			\rlap{3>3.\xx{pfv}.cause.happen} {} {} {} {} {} {} {}
		\rlap{it.is} {}
		{} her \rlap{husbands} {} {} near -at {}
		{} it atop -to {}
		\rlap{\xx{pfv}.one.poke} {} {} {} {} {}
		{} fishhawk \rlap{pinion} {} {} -of {} //
	\glft	‘It was when she brought it inside that, next to her husbands, people poked a fishhawk’s pinion in it.’
		//
\endgl
\xe

\ex\label{ex:89-101-hand-caught-bent-quill}%
\exmn{256.13}%
\begingl
	\glpreamble	Tc!uyū′ du djī′n wudułcā′dî awe′ ʟa yū′yênkā′watᴀn xēʟ!qāx. //
	\glpreamble	Chʼu yú du jín wudulsháadi áwé tle yoo yan kaawatán x̱éilʼ kaax̱. //
	\gla	Chʼu {} {} yú du jín {}
			\rlap{wudulsháadi} @ {} @ {} @ {} @ {} @ {} @ {} {}
		\rlap{áwé} @ {}
		tle yoo @ yan @ \rlap{kaawatán} @ {} @ {} @ {} @ {}
		{} x̱éilʼ \rlap{kaax̱.} @ {} {} //
	\glb	chʼu {} {} yú du jín {}
			wu- du- d- l- \rt[²]{shaʼt} -μμH -í {}
		á -wé
		tle yoo= ÿán= k- wu- i- \rt[¹]{tan} -μH
		{} x̱éilʼ ká -dáx̱ {} //
	\glc	chʼu {}[\pr{CP} {}[\pr{DP} \xx{dist} \xx{3h·pss} hand {}]
			\xx{pfv}- \xx{4h·s}- \xx{mid}- \xx{xtn}- \rt[²]{grab} -\xx{var} -\xx{sub} {}]
		\xx{foc} -\xx{mdst}
		then \xx{alt}= \xx{term}= \xx{qual}- \xx{pfv}- \xx{stv}- \rt[¹]{bend} -\xx{var}
		{}[\pr{PP} slime \xx{hsfc} -\xx{abl} {}] //
	\gld	just {} {} that her hand {}
			\rlap{\xx{pfv}.one.\xx{xtn}.grab} {} {} {} {} {} -when {}
		\rlap{it.is} {}
		then \xx{alt} end \rlap{\xx{pfv}.bend} {} {} {} {}
		{} slime atop -from {} //
	\glft	‘Since one had grabbed her hand, the pinion bent over from slime.’
		//
\endgl
\xe

\ex\label{ex:89-102-husbands-leave}%
\exmn{256.13}%
\begingl
	\glpreamble	ʟe awē′ wudîna′q duxo′xq!ᵘ wᴀ′ng̣ᴀ′ndî dunᴀ′q. //
	\glpreamble	Tle áwé wudináḵ du x̱úx̱xʼu wán gánde, du náḵ. //
	\gla	Tle \rlap{áwé} @ {}
		\rlap{wudináḵ} @ {} @ {} @ {} @ {}
		{} du \rlap{x̱úx̱xʼu} @ {} @ \•wán {}
		{} \rlap{gánde,} @ {} {} +
		{} du náḵ. {} //
	\glb	Tle á -wé
		wu- d- i- \rt[¹]{naḵ} -μH
		{} du x̱úx̱ -xʼ =ÿán {}
		{} gáan -dé {}
		{} du náḵ {} //
	\glc	then \xx{foc} -\xx{mdst}
		\xx{pfv}- \xx{mid}- \xx{stv}- \rt[¹]{stand·\xx{pl}} -\xx{var}
		{}[\pr{DP} \xx{3h·pss} husband -\xx{pl} =\xx{plk} {}]
		{}[\pr{PP} outside -\xx{all} {}]
		{}[\pr{PP} \xx{3h} \xx{elat} {}] //
	\gld	then \rlap{it.is} {}
		\rlap{\xx{pfv}.stand·\xx{pl}} {} {} {} {}
		{} her \rlap{husbands} {} {} {}
		{} outside -to {}
		{} her away {} //
	\glft	‘Then they stood up, her husbands, going outside and leaving her.’
		//
\endgl
\xe

\ex\label{ex:89-103-grab-through}%
\exmn{257.1}%
\begingl
	\glpreamble	Ts!uhē′t!aawe ag̣acᴀ′ttc, ʟe atū′nᴀx wudjᴀ′łtc. //
	\glpreamble	Chʼu héitʼaa áwé ag̱ashátch, tle a tóonáx̱ wujélch. //
	\gla	Chʼu {} \rlap{héitʼaa} @ {} @ {} {} \rlap{áwé} @ {}
		\rlap{ag̱ashátch} @ {} @ {} @ {} @ {}
		tle {} a \rlap{tóonáx̱} @ {} {}
		\rlap{wujélch.} @ {} @ {} @ {} @ {} //
	\glb	chʼu {} hé- tʼ- aa {} á -wé
		a- g̱- \rt[²]{shaʼt} -μH -ch
		tle {} a tú -náx̱ {}
		ÿ- u- \rt[¹]{jel} -μH -ch //
	\glc	just {}[\pr{DP} \xx{mprx}- \xx{link}- \xx{part} {}] \xx{foc} -\xx{mdst}
		\xx{arg}- \xx{g̱cnj}- \rt[²]{grab} -\xx{var} -\xx{rep}
		then {}[\pr{PP} \xx{3n·pss} inside -\xx{perl} {}]
		\xx{qual}- \xx{zpfv}- \rt[¹]{mv·hand} -\xx{var} -\xx{rep} //
	\gld	just {} \rlap{this.one} {} {} {} \rlap{it.is} {}
		\rlap{3>3.\xx{hab}.grab} {} {} {} {}
		then {} it inside -thru {}
		\rlap{\xx{hab}.move·hand} {} {} {} //
	\glft	‘Just when she would grab this one, then her hands would move through it.’
		//
\endgl
\xe

\ex\label{ex:89-104-could-not-see}%
\exmn{257.1}%
\begingl
	\glpreamble	Tc!uʟe′ ʟēł hᴀs wudustī′n. //
	\glpreamble	Chʼu tle tléil has wudusteen. //
	\gla	Chʼu tle tléil has @ \rlap{wudusteen.} @ {} @ {} @ {} @ {} @ {} //
	\glb	chʼu tle tléil has= wu- du- d- s- \rt[²]{tin} -μμL //
	\glc	just then \xx{neg} \xx{plh}= \xx{pfv}- \xx{4h·s}- \xx{mid}- \xx{xtn}- \rt[²]{see} -\xx{var} //
	\gld	just then not them= \rlap{\xx{pfv}.people.see} {} {} {} {} {} //
	\glft	‘Then people could not see them.’
		//
\endgl
\xe

The verb in (\ref{ex:89-104-could-not-see}) with \fm{s-} and \fm{\rt[²]{tin}} is usually used to mean ‘see’; contrast this with the use of \fm{\rt[²]{tin}} without \fm{s-} to mean ‘can see’ as in \fm{ayatéen} ‘s/he can see him/her/it’ \parencite[242]{edwards:2009}.
But the verb without \fm{s-} is ungrammatical in the perfective aspect.
The speaker could have chosen to use the imperfective here and thus could have said \fm{Chʼu tle tléil has duteen} − compare the sentence \fm{Hás ḵu.aa tléil has duteen} in (\ref{ex:89-79-people-cant-see}) – but for some reason chose to use the perfective and so could not say \fm[*]{Chʼu tle tléil has wuduteen}.

\ex\label{ex:89-105-canoe-run-around}%
\exmn{257.2}%
\begingl
	\glpreamble	Hᴀ′sduyā′gu qo′a awe′ ā kᴀt wudjix̣ī′x̣. //
	\glpreamble	Hasdu yaagú ḵu.aa áwé áa kát wujixeex. //
	\gla	{} \rlap{Hasdu} @ {} \rlap{yaagú} @ {} {}
		ḵu.aa \rlap{áwé} @ {}
		{} áa \rlap{kát} @ {} {}
		\rlap{wujixeex.} @ {} @ {} @ {} @ {} @ {} //
	\glb	{} has= du yaakw -í {}
		ḵu.aa á -wé
		{} áa ká -t {}
		wu- d- sh- i- \rt[¹]{xix} -μμL //
	\glc	{}[\pr{DP} \xx{plh}= \xx{3h·pss} boat -\xx{pss} {}]
		\xx{contr} \xx{foc} -\xx{mdst}
		{}[\pr{PP} lake \xx{hsfc} -\xx{pnct} {}]
		\xx{pfv}- \xx{mid}- \xx{pej}- \xx{stv}- \rt[¹]{fall} -\xx{var} //
	\gld	{} \rlap{their} {} \rlap{canoe} {} {}
		however \rlap{it.is} {}
		{} lake atop -around {}
		\rlap{\xx{pfv}.run·\xx{sg}} {} {} {} {} {} //
	\glft	‘It was their canoe however that was running around on the lake.’
		//
\endgl
\xe

\clearpage
\section{Paragraph 7}\label{sec:89-para-7}

\ex\label{ex:89-106-hope-killed-by-filth}%
\exmn{257.3}%
\begingl
	\glpreamble	Hᴀsdūỵī′dî qo′a awe′ yēᴀt hᴀs aodîcî qahā′s!tc ỵāqgadjā′q. //
	\glpreamble	Hasdu ÿéet ḵu.aa áwé yéi át has awdishée, kaháasʼch ÿaa kg̱ajaaḵ.  //
	\gla	{} \rlap{Hasdu} @ {} ÿéet {} ḵu.aa \rlap{áwé} @ {}
		yéi {} \rlap{át} @ {} {}
		has @ \rlap{awdishée,} @ {} @ {} @ {} @ {} @ {} 
		{} {} \rlap{kaháasʼch} @ {} @ {} @ {} {}
			ÿaa @ \rlap{kg̱ajaaḵ.} @ {} @ {} @ {} @ {} @ {} {} //
	\glb	{} has= du ÿéet {} ḵu.aa á -wé
		yéi {} á -t {} has= a- wu- d- i- \rt[²]{shiʰ} -μμL
			{} {} k- \rt[¹]{hasʼ} -μμH -ch {}
			ÿaa= ⱥ- k- g̱- \rt[²]{jaḵ} -μμL {} {} //
	\glc	{}[\pr{DP} \xx{plh}= \xx{3h·pss} son {}] \xx{contr} \xx{foc} -\xx{mdst}
		thus {}[\pr{PP} \xx{3n}\ix{i} -\xx{pnct} {}]
		\xx{plh}= \xx{xpl}- \xx{pfv}- \xx{mid}- \xx{stv}- \rt[²]{hope} -\xx{var}
		{}[\pr{CP} {}[\pr{DP} \xx{qual}- \rt[¹]{vomit} -\xx{var} -\xx{erg} {}]
			along= \xx{arg}- \xx{gcnj}- \xx{mod}- \rt[²]{kill} -\xx{var} \·\xx{sub}\ix{i} {}] //
	\gld	{} \rlap{their} {} son {} however \rlap{it.is} {}
		thus {} it\ix{i} -for {}
		they\• \rlap{\xx{pfv}.hope} {} {} {} {} {}
		{} {} \rlap{vomit} {} {} {} {}
			along= \rlap{3>3.\xx{hort}.kill} {} {} {} {} {} {} //
	\glft	‘Their son however, they hoped that vomit should kill him.’
		//
\endgl
\xe

The analysis and translation of (\lastx) has a couple of problems.
One is \citeauthor{swanton:1909}’s \orth{Hᴀsdūỵī′dî} which appears to reflect a phrase \fm{hasdu ÿéedi} that is ungrammatical in modern Tlingit.
Kinship terms like \fm{ÿéet} ‘son’ are inalienable nouns and so are inherently (and so obligatorily) possessed, and so they cannot occur without a possessor.
Even though inalienable nouns must be possessed, they are not marked for possession with the possessive suffix \fm{-í}.
Some inalienable nouns – especially body parts – can be alienated with the addition of the possessive suffix.
The meaning changes so that the noun refers to a severed body part no longer attached to its normal owner.
Compare for example \fm{g̱áx̱ jín} ‘a rabbit’s foot’ (attached to the rabbit) with \fm{g̱áx̱ jíni} ‘a rabbit foot’ (not attached to the rabbit).
Kinship nouns like \fm{ÿéet} do not normally allow this, so a form like \fm[*]{ax̱ ÿéedi} is ungrammatical.
There are two possibilities for \citeauthor{swanton:1909}’s \orth{Hᴀsdūỵī′dî}: (i) it is mistranscribed, or (ii) there is a difference in grammar between this speaker and modern Tlingit.
The mistranscription case (i) is presumably a simple mishearing of an extra vowel.
The different grammar case (ii) is problematic because it cannot be verified since there are no speakers from the 19th century alive to ask, but also because there is no other evidence from the same time period that supports this pattern.

The second problem in (\lastx) is the interpretation of \citeauthor{swanton:1909}’s \orth{qahā′s!tc}.
He glosses and translates this as “filth”, but it is apparently \fm{kaháasʼ-ch} based on the root \fm{\rt[¹]{hasʼ}} ‘vomit’; the final \fm{-ch} is the ergative suffix and is not problematic.
There are a handful of words that can be translated as ‘filth’ such as \fm{tlʼeex} ‘garbage, trash, rubbish, dirt, filth’, \fm{sʼeex} ‘dirt, trash, lint, filth’, \fm{ḵoosh} ‘open sore; filth, impurity’, and \fm{ḵéetʼ} ‘pus, ooze, filth’.
None of these fit with \citeauthor{swanton:1909}’s transcription, for which \fm{\rt[¹]{hasʼ}} ‘vomit’ is by far the best candidate.
This root \fm{\rt[¹]{hasʼ}} has a few related roots, namely \fm{\rt[¹]{hatlʼ}} ‘crap, dung’, \fm{\rt[¹]{hachʼ}} ‘shameful’, \fm{\rt[¹]{hesh}} ‘disgrace’, and perhaps also \fm{\rt[¹]{helʼ}} ‘clumsy, helpless, weak’ and \fm{\rt{hetlʼ}} ‘huge, awful’.
It is plausible that \fm{\rt[¹]{hasʼ}} used to have a more general meaning of ‘filth’ which has become specifically ‘vomit’, but this is speculation.

\ex\label{ex:89-107-boy-killed-by-poverty}%
\exmn{257.3}%
\begingl
	\glpreamble	ᴀtcawe′ duī′c nᴀg̣anā′n ᴀtk!ᴀ′tsk!ᵘ q!ᴀnᴀskîdē′tc wudjā′qtc. //
	\glpreamble	Ách áwé du éesh nag̱anáanín atkʼátskʼu, ḵʼanashgidéich ujaaḵch. //
	\gla	{} \rlap{Ách} @ {} {} \rlap{áwé} @ {}
		{} {} du éesh {} 
			\rlap{nag̱anáanín} @ {} @ {} @ {} @ {} @ {} {} +
		{} \rlap{atkʼátskʼu,} @ {} @ {} @ {} @ {} {}
		{} \rlap{ḵʼanashgidéich} @ {} @ {} @ {} @ {} @ {} @ {} {}
		\rlap{ujaaḵch.} @ {} @ {} @ {} @ {} //
	\glb	{} á -ch {} á -wé
		{} {} du éesh {}
			n- g̱- \rt[¹]{naʰ} -μμH -n -ín {}
		{} at= kʼí- ÿáts -kʼʷ -í {}
		{} ḵʼe- n- sh- \rt{git} -i =yé -ch {}
		ⱥ- u- \rt[²]{jaḵ} -μμL -ch //
	\glc	{}[\pr{PP} \xx{3n} -\xx{erg} {}] \xx{foc} -\xx{mdst}
		{}[\pr{CP} {}[\pr{DP} \xx{3h·pss} father {}]
			\xx{ncnj}- \xx{mod}- \rt[¹]{die} -\xx{var} -\xx{nsfx} -\xx{ctng} {}]
		{}[\pr{DP} \xx{4n·pss}= base- child -\xx{dim} -\xx{pss} {}]
		{}[\pr{DP} mouth- \xx{ncnj}- \xx{pej}- \rt{\xx{unkn}} -\xx{rel} =way -\xx{erg} {}]
		\xx{arg}- \xx{zpfv}- \rt[²]{kill} -\xx{var} -\xx{rep} //
	\gld	{} that -why {} \rlap{it.is} {}
		{} {} his father {}
			\rlap{\xx{ctng}.die} {} {} {} {} {} {}
		{} \rlap{boy} {} {} {} {} {}
		{} \rlap{poverty} {} {} {} {} {} {} {}
		\rlap{3>3.\xx{hab}.kill} {} {} {} {} //
	\glft	‘That is why when his father dies, a boy, poverty kills him.’
		//
\endgl
\xe

Although the sentence in (\lastx) literally describes poverty killing a boy, it is very likely that this does not describe death and is instead a special case of a more general metaphor for being strongly affected by something.
Verbs based on \fm{\rt[²]{jaḵ}} ‘kill’ appear in metaphors like \fm{táach uwajáḵ} ‘s/he fell asleep’ (lit.\ ‘sleep killed him/her’), \fm{áatʼch uwajáḵ} ‘s/he has gotten very cold’ (lit.\ ‘cold killed him/her’), \fm{ta.aasch uwajáḵ} ‘s/he has gotten lonely’ (lit.\ ‘loneliness killed him/her’), \fm{tuxʼandagáaxʼch yaa najáḵ} ‘s/he is getting irritated’ (lit.\ ‘annoyance is killing him/her’), and \fm{ḵu.áx̱jikʼch yaa najáḵ} ‘s/he has gotten deaf’ (lit.\ ‘little hearing of people is killing him/her’).
The phrase \fm{ḵʼanashgidéich uwajáḵ} is not otherwise attested so the exact meaning is still unclear, but it is almost certain to be another instance of this metaphor structure.

\ex\label{ex:89-108-son-sick-forced-out}%
\exmn{257.4}%
\begingl
	\glpreamble	ʟᴀx wâ′yu kᴀcū′sawedê duyê′tk!ᵘ, duʟā′ tîn gā′nîyᴀx ka′oduʟ̣î-u′. //
	\glpreamble	Tlax̱ wáa yóo kashóo sáwé de du yátkʼu, du tláa tin gáani yux̱ kawdudli.oo. //
	\gla	{} Tlax̱ wáa yóo @ \rlap{kashóo} @ {} @ {} @ {} @ {} {}
		\rlap{sáwé} @ {} @ {} de +
		{} du \rlap{yátkʼu,} @ {} @ {} {}
		{} du tláa tin {} +
		{} \rlap{gáani} @ {} {}
		yux̱ @ \rlap{kawdudli.oo.} @ {} @ {} @ {} @ {} @ {} @ {} @ {} //
	\glb	{} tlax̱ wáa yóo= k- {} \rt[¹]{shuʰ} -μμH {} {}
		s- á -wé de
		{} du ÿát -kʼʷ -í {}
		{} du tláa teen {}
		{} gáan -í {}
		yux̱= k- wu- du- d- l- i- \rt[²]{.uʰ} -μμL //
	\glc	{}[\pr{CP} very how thus= \xx{qual}- \xx{zcnj}\· \rt[¹]{intox} -\xx{var} \·\xx{sub} {}]
		\xx{q}- \xx{foc} -\xx{mdst} now
		{}[\pr{DP} \xx{3h·pss} child -\xx{dim} -\xx{pss} {}]
		{}[\pr{PP} \xx{3h·pss} mother \xx{instr} {}]
		{}[\pr{PP} outside -\xx{loc} {}]
		out= \xx{qual}- \xx{pfv}- \xx{4h·s}- \xx{mid}- \xx{csv}- \xx{stv}- \rt[¹]{dwell} -\xx{var} //
	\gld	{} very how thus \rlap{\xx{csec}.intoxicate} {} {} {} {} {}
		ever- \rlap{it.is} {} now
		{} her son -little {} {}
		{} his mother with {}
		{} outside -at {}
		out\• \rlap{\xx{pfv}.ppl.make.live} {} {} {} {} {} {} {} //
	\glft	‘Having somehow become very intoxicated now, her little child,
		people made him live outside with his mother.’
		//
\endgl
\xe

\ex\label{ex:89-109-made-brush-house}%
\exmn{257.5}%
\begingl
	\glpreamble	Ān tcukᴀ′q!awe tcāc hît aka′ aołîyᴀ′x., //
	\glpreamble	Aan shukáxʼ áwé cháash hít a káa awliyéx̱. //
	\gla	{} Aan \rlap{shukáxʼ} @ {} @ {} {} \rlap{áwé} @ {}
		{} cháash hít {}
		{} a \rlap{káa} @ {} {}
		\rlap{awliyéx̱.} @ {} @ {} @ {} @ {} @ {} //
	\glb	{} aan shú- ká -xʼ {} á -wé
		{} cháash hít {}
		{} a ká -μ {}
		a- wu- l- i- \rt[²]{yex̱} -μH //
	\glc	{}[\pr{PP} town end- \xx{hsfc} -\xx{loc} {}] \xx{foc} -\xx{mdst}
		{}[\pr{DP} brush house {}]
		{}[\pr{PP} \xx{3n·pss} \xx{hsfc} -\xx{loc} {}]
		\xx{arg}- \xx{pfv}- \xx{xtn}- \xx{stv}- \rt[²]{make} -\xx{var} //
	\gld	{} town end- atop\ix{i} -at {} \rlap{it.is} {}
		{} brush house {}
		{} its\ix{i} atop -at {}
		\rlap{3>3.\xx{pfv}.build} {} {} {} {} {} //
	\glft	‘It is at the end of town that she made a brush house there.’
		//
\endgl
\xe

\ex\label{ex:89-110-live-there-with-child}%
\exmn{257.6}%
\begingl
	\glpreamble	Duyê′tk!ᵒ ᴀ′q! ān yē wutî′. //
	\glpreamble	Du yátkʼu, áxʼ aan yéi wootee. //
	\gla	{} Du \rlap{yátkʼu} @ {} @ {} {}
		{} \rlap{áxʼ} @ {} {}
		{} \rlap{aan} @ {} {}
		yéi @ \rlap{wootee.} @ {} @ {} @ {} //
	\glb	{} du yát -kʼ -í {}
		{} á -xʼ {}
		{} á -n {}
		yéi= wu- i- \rt[¹]{tiʰ} -μμL //
	\glc	{}[\pr{DP} \xx{3h·pss} child -\xx{dim} -\xx{pss} {}]
		{}[\pr{PP} \xx{3n} -\xx{loc} {}]
		{}[\pr{PP} \xx{3n} -\xx{instr} {}]
		thus= \xx{pfv}- \xx{stv}- \rt[¹]{be} -\xx{var} //
	\gld	{} her child -little {} {}
		{} there -at {}
		{} him -with {}
		thus\• \rlap{\xx{pfv}.be} {} {} {} //
	\glft	‘Her little child, she was there with him.’
		//
\endgl
\xe

\ex\label{ex:89-111-always-bathed-child}%
\exmn{257.7}%
\begingl
	\glpreamble	ᴀ′cutcnutc duyê′tk!ᵒ yū′tcāc hît ỵîk. //
	\glpreamble	Ashóoch nuch du yátkʼu yú cháash hít ÿík. //
	\gla	\rlap{Ashóoch} @ {} @ {} @ \•nuch
		{} du \rlap{yátkʼu} @ {} @ {} {}
		{} yú cháash hít \rlap{ÿík.} @ {} {} //
	\glb	a- \rt[²]{shuch} -μμH =nooch
		{} du ÿát -kʼʷ -í {}
		{} yú cháash hít ÿíᵏ -k {} //
	\glc	\xx{arg}- \rt[²]{bathe} -\xx{var} =\xx{hab·aux}
		{}[\pr{DP} \xx{3h·pss} child -\xx{dim} -\xx{pss} {}]
		{}[\pr{PP} \xx{dist} brush house within -\xx{oloc} {}] //
	\gld	\rlap{3>3.\xx{impfv}.bathe} {} {} \•always
		{} her child -little {} {}
		{} that brush house within -at {} //
	\glft	‘She always bathed her little child in that brush house.’
		//
\endgl
\xe

\ex\label{ex:89-112-gradually-get-big}%
\exmn{257.7}%
\begingl
	\glpreamble	Desgwᴀ′tc ʟ!agā′yan nᴀłgē′n. //
	\glpreamble	Deisgwách tlʼag̱áa yaa nalgéin. //
	\gla	Deisgwách tlʼag̱áa
		yaa @ \rlap{nalgéin.} @ {} @ {} @ {} @ {} //
	\glb	deisgwách tlʼag̱áa
		ÿaa= n- l- \rt[¹]{ge} -μμH -n //
	\glc	gradually lots
		along= \xx{ncnj}- \xx{xtn}- \rt[¹]{big} -\xx{var} -\xx{nsfx} //
	\gld	gradually lots
		along\• \rlap{\xx{prog}.big} {} {} {} {} //
	\glft	‘Gradually he gets very tall.’
		//
\endgl
\xe

Both of the adverbs \fm{deisgwách} [\ipa{tèːs.ˈkʷátʃ}] ‘gradually, after a while, pretty soon, eventually’ \parencite[f05/208]{leer:1973} and \fm{tlʼag̱áa} [\ipa{tɬʼà.ˈqáː}] ‘lots, much, more’ \parencite[08/217]{leer:1973} are fairly well known but their segmentations are still unclear so they have been given as unanalyzed lumps in (\lastx).
The adverb \fm{deisgwách} might start with \fm{de} ‘now, already’ and may contain the modal or mirative element \fm{gwá} < \fm[*]{gʷa}.
The latter is found in a variety of words like \fm{gu.aal} ‘hopefully’, \fm{gushí} ‘maybe’, and \fm{gwáa} ‘surprisingly’ \parencite[f05/185]{leer:1973}.
The \fm{gaa} \~\ \fm{gáa} element in the particle \fm{haagaashí} [\ipa{hàː.kàː.ˈʃí}] ‘it would be good if’ \parencite[10/48–50]{leer:1973} and in \fm{dágáa} [\ipa{tá.ˈkáː}] \parencite[f05/10]{leer:1973} could come from \fm[*]{gʷa} with delabialization.
The wh-words \fm{gwátk} [\ipa{kʷátk}] ‘when (past)’ and \fm{gwátgeen} [\ipa{ˈkʷát.kìːn}] ‘when (future)’ could also be based on \fm[*]{gʷa}.
The \fm{s} in \fm{deisgwách} is plausibly connected to the dubitative particle \fm{sí} and the wh-question particle \fm{sá}; these are possibly related to the modal element \fm{shí} found in \fm{gushí} ‘maybe’ and \fm{haagaashí} ‘it would be good if’ noted above, as well as perhaps the root \fm{\rt[²]{shi}} ‘hope’.
The final \fm{ch} in \fm{deisgwách} is unidentified; two unexplored possibilities are ergative \fm{-ch} and repetitive \fm{-ch}.

The adverb \fm{tlʼag̱áa} [\ipa{tɬʼà.ˈqáː}] ‘lots, much, more’ seen in (\lastx) appears to be composed of an element \fm{tlʼa} and the adessive postposition \fm{-g̱áa} ‘for, near’.
\textcite[08/217]{leer:1973} identifies the \fm{tlʼa} as identical to the \fm{tlʼaa} in \fm{tlʼaadéin} [\ipa{tɬʼàː.ˈtéːn}] ‘crosswise, sideways’, but provides no further details.
In his stem list \citeauthor{leer:1978b} gives the two as separate and so implicitly unrelated lexical items \parencite[35]{leer:1978b}; again he offers no further details.
If the final \fm{-g̱áa} is not the adessive postposition then another likely relation is the adverbial \fm{g̱áa} ‘all right, okay, fine’ \parencite[f02/98]{leer:1973}; both may have an origin in \fm[*]{ɢa-ˀ} meaning something like ‘near’ with the locative postposition allomorph \fm[*]{-ˀ}.

\ex\label{ex:89-113-ppl-throw-garbage}%
\exmn{257.8}%
\begingl
	\glpreamble	Qaq!aitē′awe dukadê′q dog̣ê′tcnutc. //
	\glpreamble	Ḵaa x̱ʼa.eetí áwé du kaadé kdug̱éich nuch. //
	\gla	{} Ḵaa \rlap{x̱ʼa.eetí} @ {} {} \rlap{áwé} @ {}
		{} du \rlap{kaadé} @ {} {}
		\rlap{kdug̱éich} @ {} @ {} @ {} @ {} @ \•nuch. //
	\glb	{} ḵaa x̱'é- eetí {} á -wé
		{} du ká -dé {}
		k- du- d- \rt[²]{g̱ech} -μμH =nooch //
	\glc	{}[\pr{DP} \xx{4h·pss} mouth- remains {}] \xx{foc} -\xx{mdst}
		{}[\pr{PP} \xx{3h·pss} \xx{hsfc} -\xx{all} {}]
		\xx{sro}- \xx{4h·s}- \xx{mid}- \rt[²]{throw} -\xx{var} =\xx{hab·aux} //
	\gld	{} ppl’s \rlap{garbage} {} {} \rlap{it.is} {}
		{} his atop -to {}
		\rlap{round.\xx{impfv}.ppl.throw} {} {} {} {} \•always //
	\glft	‘It is people’s garbage that people were always throwing on top of him.’
		//
\endgl
\xe

The noun \fm{x̱ʼa.eetí} in (\lastx) generally refers to food scraps or leftover bits that are fed to dogs, tossed in a fire, or otherwise discarded \parencite[02/248]{leer:1973}.
\citeauthor{swanton:1909} glosses and translates this as “garbage” which is maintained here.
Note however that ‘garbage’ in English has drifted over the 20th century from the specific meaning of ‘food waste’ to apply more generally to waste of any kind, thus coming to overlap with ‘trash’ and ‘waste’.
In Tlingit the literal meaning of \fm{x̱ʼa.eetí} is ‘mouth remains’ and so is only used to describe food waste.
\citeauthor{swanton:1909}’s transcription \orth{q!aitē′} suggests a long final vowel, thus \fm{x̱ʼa.eetée} [\ipa{χʼà.ˈʔìː.tʰíː}], but for example in (\ref{ex:89-117-call-him-below-garbage-man}) and (\ref{ex:89-145-throw-garbage-on-him}) the transcriptions have a short vowel so it has been ignored here as incidental variation probably due to intonation and prosody that was not recorded.

\ex\label{ex:89-114-this-guy-they-say}%
\exmn{257.8}%
\begingl
	\glpreamble	“Yā′t!ayauwaqā′,” yuawe′ daỵadoqā′nutc. //
	\glpreamble	«\!Yáatʼaa áa aya.óowu ḵáa\!» yóo áwé daaÿaduḵáa nuch. //
	\gla	{} \llap{«\!}\rlap{Yáatʼaa} @ {} @ {}
			{} {} \rlap{áa} @ {} {}
				\rlap{aya.óowu} @ {} @ {} @ {} @ {} {} ḵáa!» {}
		yóo \rlap{áwé} @ {}
		\rlap{daaÿaduḵáa} @ {} @ {} @ {} @ {} @ {} @ \•nuch. //
	\glb	{} yá- tʼ- aa
			{} {} á -μ {}
				a- i- \rt[²]{.u} -μμH -i {} ḵáa {}
		yóo á -wé
		daa- ÿ- du- d- \rt[¹]{ḵa} -μμH =nooch //
	\glc	{}[\pr{DP} \xx{prox}- \xx{link}- \xx{part}
			{}[\pr{CP} {}[\pr{PP} \xx{3n} -\xx{loc} {}]
				\xx{xpl}- \xx{stv}- \rt[²]{own} -\xx{var} -\xx{rel} {}] man {}]
		\xx{quot} \xx{foc} -\xx{mdst}
		around- \xx{qual}- \xx{4h·s}- \xx{mid}- \rt[¹]{say} -\xx{var} =\xx{hab·aux} //
	\gld	{} \rlap{this.one} {} {}
			{} {} there -at {}
				\rlap{\xx{stv}·\xx{impfv}.live} {} {} {} -who {} man {}
		thus \rlap{it.is} {}
		\rlap{about.\xx{impfv}.people} {} {} {} {} {} \•always //
	\glft	‘“This one man who lives there” is what people say about him.’
		//
\endgl
\xe


\ex\label{ex:89-115-laugh-at-him}%
\exmn{257.9}%
\begingl
	\glpreamble	U′x udułcu′qnutc. //
	\glpreamble	Óox̱ udulshooḵ nuch. //
	\gla	{} \rlap{Óox̱} @ {} {}
		\rlap{udulshooḵ} @ {} @ {} @ {} @ {} @ {} @ \•nuch. //
	\glb	{} ú -x̱ {}
		u- du- d- l- \rt[²]{shuḵ} -μμL =nooch //
	\glc	{}[\pr{PP} \xx{3h} -\xx{pert} {}]
		\xx{irr}- \xx{4h·s}- \xx{mid}- \xx{xtn}- \rt[²]{laugh} -\xx{var} =\xx{hab·aux} //
	\gld	{} him -at {}
		\rlap{\xx{impfv}.ppl.laugh} {} {} {} {} {} \•always //
	\glft	‘People always laugh at him.’
		//
\endgl
\xe

The sentence in (\lastx) reflects a poorly documented verb based on the root \fm{\rt[²]{shuḵ}} ‘laugh at’; compare \fm{óox̱ ux̱lashooḵ} ‘I laugh at him’ \parencite[10/117]{leer:1973}.
For the archaic pronoun \fm{ú} ‘him/her’ in \fm{óox̱} see the discussion of the form \fm{óot} in (\ref{ex:89-70-smash-grease-box}).

\ex\label{ex:89-116-ran-out-among-boys-playing}%
\exmn{257.9}%
\begingl
	\glpreamble	Wānanī′sawe yux wudjix̣ī′x̣ yuᴀtk!ᴀ′tsk!ᵒ ᴀckułỵê′tîxōq!. //
	\glpreamble	Wáa nanée sáwé yux̱ wujixeex yú atkʼátskʼu ash koolÿádi x̱ooxʼ. //
	\gla	{} Wáa \rlap{nanée} @ {} @ {} @ {} {} \rlap{sáwé} @ {} @ {}
		yux̱ @ \rlap{wujixeex} @ {} @ {} @ {} @ {} @ {} +
		{} {} yú {} \rlap{atkʼátskʼu} @ {} @ {} @ {} @ {}
			{} ash @ \rlap{koolÿádi} @ {} @ {} @ {} @ {} @ {} @ {} {} {} {}
			\rlap{x̱ooxʼ.} @ {} {} //
	\glb	{} wáa n- \rt[¹]{niʰ} -μμH {} {} s- á -wé
		yúx̱= wu- d- sh- i- \rt[¹]{xix} -μμL
		{} {} yú {} at= kʼí- ÿáts -kʼʷ -í
			{} ash= k- u- d- l- \rt[¹]{ÿat} -μH -í {} {} {}
			x̱oo -xʼ {} //
	\glc	{}[\pr{CP} how \xx{ncnj}- \rt[¹]{happen} -\xx{var} \·\xx{sub} {}] \xx{q}- \xx{foc} -\xx{mdst}
		out= \xx{pfv}- \xx{mid}- \xx{pej}- \xx{stv}- \rt[¹]{fall} -\xx{var}
		{}[\pr{PP} {}[\pr{DP} \xx{dist} {}[\pr{NP} \xx{4h·pss}= base- child -\xx{dim} -\xx{pss}
			{}[\pr{NP} \xx{rflx·o}= \xx{qual}- \xx{irr}- \xx{mid}- \xx{csv}-
				\rt[¹]{child} -\xx{var} -\xx{nmz} {}] {}] {}]
			among -\xx{loc} {}] //
	\gld	{} how \rlap{\xx{csec}.happen} {} {} \·while {} ever- \rlap{it.is} {}
		out\• \rlap{\xx{pfv}.run·\xx{sg}} {} {} {} {} {}
		{} {} that {} \rlap{boy} {} {} {} {}
			{} self\• \rlap{play} {} {} {} {} {} -ing {} {} {}
			among -at {} //
	\glft	‘At some point he ran out among the boys playing.’
		//
\endgl
\xe

\ex\label{ex:89-117-call-him-below-garbage-man}%
\exmn{257.10}%
\begingl
	\glpreamble	“Tca-ī′ Q!aī′tî-cūye′-qā” ʟa yūˈduwasa. //
	\glpreamble	«\!Chʼa ée, X̱ʼa.eetí Shuyee Ḵáa\!» tle yóo wduwasáa. //
	\gla	{} \llap{«\!}Chʼa ée, \rlap{X̱ʼa.eetí} @ {} \rlap{Shuyee} @ {} Ḵáa {}
		tle yóo @ \rlap{wduwasáa.} @ {} @ {} @ {} @ {} @ {} //
	\glb	{} chʼa ée x̱ʼé- eetí shú- ÿee ḵáa {}
		tle yóo= wu- du- d- i- \rt[²]{sa} -μμH //
	\glc	{}[\pr{CP} just yuck mouth- remains end- below man {}]
		then \xx{quot}= \xx{pfv}- \xx{4h·s}- \xx{mid}- \xx{stv}- \rt[²]{name} -\xx{var} //
	\gld	{} just yuck \rlap{garbage} {} \rlap{below} {} man {}
		then thus\• \rlap{\xx{pfv}.they.call} {} {} {} {} {} //
	\glft	‘“Just yuck, Below Garbage Man” they called him.’
		//
\endgl
\xe

The utterance in (\lastx) given by \citeauthor{swanton:1909} as \orth{Tca-ī′ Q!aī′tî-cūye′-qā} can be divided into two phrases as indicated by the comma in the retranscription \fm{Chʼa ée, X̱ʼa.eetí Shuyee Ḵáa}.
The first phrase here is \fm{chʼa ée} ‘just yuck’ where the \fm{ée} element in (\lastx) is an interjection uttered when a person finds something disgusting or distasteful.
The second phrase is \fm{x̱ʼa.eetí shuyee ḵáa} ‘man below garbage’ which is the name that the protagonst’s son is called by the boys playing.
Only the second phrase is a name although the speech verb with \fm{\rt[²]{sa}} ‘call, name’ could be taken to suggest that the whole utterance is the name.
\citeauthor{swanton:1909}’s translation “Uh!
Garbage-man” supports this division into two phrases with the first being an interjection corresponding to his “Uh!” (we would write “Ugh!” today).
The whole utterance is stereotyped because it appears later in (\ref{ex:89-149-just-yuck-below-garbage-man}) with an identical form, but the second phrase is used independently as a name in (\ref{ex:89-183-report-garbage-man-married-her}) and (\ref{ex:89-185-reporting-garbage-man-married-her}).

\ex\label{ex:89-118-tell-mother-make-me-bow}%
\exmn{257.11}%
\begingl
	\glpreamble	Duʟā′ ye aỵa′osîqa ,“Sᴀks ᴀ′xdjîỵîs łayᴀ′x.” //
	\glpreamble	Du tláa yéi aÿawsiḵaa «\!Sáḵs ax̱ jeeÿís layéx̱\!». //
	\gla	{} Du tláa {}
		yéi @ \rlap{aÿawsiḵaa} @ {} @ {} @ {} @ {} @ {} @ {} +
		{} {} \llap{«\!}Sáḵs {} {} ax̱ \rlap{jeeÿís} @ {} {}
			\rlap{layéx̱.\!»} @ {} @ {} @ {} @ {} {}//
	\glb	{} du tláa {}
		yéi= a- ÿ- wu- s- i- \rt[¹]{ḵa} -μμL
		{} {} sáḵs {} {} ax̱ jee =ÿís {}
			{} {} l- \rt[²]{yex̱} -μH {} //
	\glc	{}[\pr{DP} \xx{3h·pss} mother {}]
		thus= \xx{arg}- \xx{qual}- \xx{pfv}- \xx{csv}- \xx{stv}- \rt[¹]{say} -\xx{var}
		{}[\pr{CP} {}[\pr{DP} bow {}] {}[\pr{PP} \xx{1sg·pss} poss’n =\xx{ben} {}]
			\xx{zcnj}\· \xx{2sg·s}\· \xx{xtn}- \rt[²]{make} -\xx{var} {}] //
	\gld	{} his mother {}
		thus\• \rlap{3>3.\xx{pfv}.say} {} {} {} {} {} {}
		{} {} bow {} {} my poss’n =for {}
		\rlap{\xx{imp}.you·\xx{sg}.make} {} {} {} {} {} //
	\glft	‘He said to his mother “Make me a bow”.’
		//
\endgl
\xe

\ex\label{ex:89-119-made-getting-bright-go-shooting}%
\exmn{257.11}%
\begingl
	\glpreamble	Aʟe′ ye anᴀsnī′awe tc!ū′ya akᴀ′ndagᴀnêawe′ ānagu′ttc at!o′kt!. //
	\glpreamble	Aa tle yéi anasnée áwé chʼu yaa akandagáni áwé aan nagútch atʼúkt. //
	\gla	Aa {} tle yéi @ \rlap{anasnée} @ {} @ {} @ {} @ {} @ {} {}
		\rlap{áwé} @ {} +
		{} chʼu yaa @ \rlap{akandagáni} @ {} @ {} @ {} @ {} @ {} @ {} {}
		\rlap{áwé} @ {}
		{} \rlap{aan} @ {} {}
		\rlap{nagútch} @ {} @ {} @ {}
		{} \rlap{atʼúkt} @ {} @ {} @ {} {} //
	\glb	aa {} tle yéi= a- n- s- \rt[¹]{niʰ} -μμH {} {} á -wé
		{} chʼu ÿaa= a- k- n- d- \rt[²]{gan} -μH -í {} á -wé
		{} á -n {}
		n- \rt[¹]{gut} -μH -ch
		{} a- \rt[²]{tʼuʼk} -μH -t {} {} //
	\glc	so {}[\pr{CP} then thus= \xx{arg}- \xx{ncnj}- \xx{csv}- \rt[¹]{happen} -\xx{var} \·\xx{sub} {}]
		\xx{foc} -\xx{mdst}
		{}[\pr{CP} just along= \xx{xpl}- \xx{qual}- \xx{ncnj}- \xx{mid}- \rt[²]{burn} -\xx{var} -\xx{sub} {}]
		\xx{foc} -\xx{mdst}
		{}[\pr{PP} \xx{3n} -\xx{instr} {}]
		\xx{ncnj}- \rt[¹]{go·\xx{sg}} -\xx{var} -\xx{rep}
		{}[\pr{CP} \xx{arg}- \rt[²]{shoot·arrow} -\xx{var} -\xx{ict} {}] //
	\gld	so {} then thus\• \rlap{3>3.\xx{csec}.make.happen} {} {} {} {} {} {}
		\rlap{it.is} {}
		{} just along\• \rlap{\xx{prog}.bright} {} {} {} {} {} {} {}
		\rlap{it.is} {}
		{} it -with {}
		\rlap{\xx{hab}.go·\xx{sg}} {} {} {}
		{} \rlap{3>3.\xx{impfv}.shoot·arrow.\xx{rep}} {} {} {} {} //
	\glft	‘So then having made it, just as it is getting bright, he would always go out with it shooting arrows.’
		//
\endgl
\xe

\ex\label{ex:89-120-shoot-everything}%
\exmn{257.12}%
\begingl
	\glpreamble	Łdakᴀ′t ᴀ′dawe at!o′kt!înutc. //
	\glpreamble	Ldakát át áwé atʼúkdi nuch. //
	\gla	{} Ldakát át {} \rlap{áwé} @ {}
		\rlap{atʼúkdi} @ {} @ {} @ {} @ \•nuch. //
	\glb	{} ldakát át {} á -wé
		a- \rt[²]{tʼuʼk} -μH -t =nooch //
	\glc	{}[\pr{DP} every thing {}] \xx{foc} -\xx{mdst}
		\xx{arg}- \rt[²]{shoot·arrow} -\xx{var} -\xx{ict} =\xx{hab·aux} //
	\gld	{} every thing {} \rlap{it.is} {}
		\rlap{3>3.\xx{impfv}.shoot·arrow.\xx{rep}} {} {} {} {} \•always //
	\glft	‘It is everything that he would shoot.’
		//
\endgl
\xe

\ex\label{ex:89-121-became-man-came-lakeshore}%
\exmn{257.13}%
\begingl
	\glpreamble	Qāx ỵaqsatī′yawe desgwa′tc yū′āk! aỵahêtaqguttc. //
	\glpreamble	Ḵáax̱ ÿaa ksatée áwé deisgwách yú áakʼw a ÿaax̱í daak gútch. //
	\gla	{} {} \rlap{Ḵáax̱} {} {} ÿaa @ \rlap{ksatée} @ {} @ {} @ {} @ {} {}
		\rlap{áwé} @ {} +
		deisgwách
		{} yú \rlap{áakʼw} @ {} {}
		{} a \rlap{ÿaax̱í} @ {} {}
		daak @ \rlap{gútch.} @ {} @ {} //
	\glb	{} {} ḵáa -x̱ {} ÿaa= g- s- \rt[¹]{tiʰ} -μμH {} {}
		á -wé
		deisgwách
		{} yú áaʷ -kʼ {}
		{} a ÿaax̱ -í {}
		daak= \rt[¹]{gut} -μH -ch //
	\glc	{}[\pr{CP} {}[\pr{PP} man -\xx{pert} {}] along= \xx{gcnj}- \xx{appl}- \rt[¹]{be} -\xx{var} \·\xx{sub} {}]
		\xx{foc} -\xx{mdst}
		gradually
		{}[\pr{DP} \xx{dist} lake -\xx{dim} {}]
		{}[\pr{PP} \xx{3n·pss} side -\xx{loc} {}]
		\xx{admar}= \rt[¹]{go·\xx{sg}} -\xx{var} -\xx{rep} //
	\gld	{} {} man -of {} along\• \rlap{\xx{csec}.be·of} {} {} {} {} {}
		\rlap{it.is} {}
		eventually
		{} that lake -little {}
		{} its shore -at {}
		out\• \rlap{\xx{impfv}.go·\xx{sg}.\xx{rep}} {} {} //
	\glft	‘Having gradually become a man, eventually he comes out at the shore of that little lake.’
		//
\endgl
\xe

The phrase \fm{ḵáax̱ ÿaa ksatée} in (\lastx) is a rare example of the consecutive aspect applied to the verb \fm{áx̱ wusitee} ‘s/he/it became (of) it’.
This verb – symbolized \fm{NP-x̱ O-s-\rt[²]{tiʰ}}{g}{\fm{-μμL} stv}{O be (made, member) of NP} – is a member of the \fm{g}-conjugation class.
The consecutive is realized by the conjugation class prefix and the \fm{-μμH} stem, so in (\lastx) the \fm{g}-conjugation prefix here appears as the \fm{k} after the \fm{ÿaa} preverb: \fm{ÿaa ksatée} [\ipa{ɰàːk.sà.tíː}].
\citeauthor{swanton:1909}’s transcription with \orth{q} suggests \fm{g̱}-conjugation but this would be ungrammatical for this verb and so it must be a mishearing.

The labialization of the diminutive \fm{-kʼ} in \fm{áakʼw} [\ipa{ʔáːkʼʷ}] ‘little lake’ is irregular but well attested and universal among modern Tlingit speakers.
The noun \fm{áaʷ} ‘lake’ is one of a small number of stems with final /\ipa{aː}/ that act as though the vowel is labial; compare \fm{ax̱ ḵáawu} ‘my man’ with \fm{ḵáaʷ} ‘man’, \fm{ax̱ sháawu} ‘my woman’ with \fm{sháaʷ} ‘woman’, and \fm{ax̱ naawú} ‘my corpse’ with \fm{naaʷ} ‘corpse’.
\citeauthor{swanton:1909} transcribes \orth{āk!} suggesting that it is not labial, but this is very likely an error for \fm{āk!ᵘ} or \fm{āk!ᵒ} which he has elsewhere.

The transcription \orth{taqguttc} in (\lastx) suggests the preverb \fm{daaḵ=} ‘inland’ with \orth{q}, but \citeauthor{swanton:1909} is unreliable in differentiating velar and uvular so this could just as likely be \fm{daak=} ‘out to sea’.
Usually the context should differentiate the two, but in this particular case both are plausible.
The \fm{daaḵ=} ‘inland’ could apply because the lake is inland from the seashore where the village is located.
But the \fm{daak=} ‘out to sea’ could apply because the lakeshore is ‘out’ in contrast with the forest from where the character has emerged.
The sentence in (\ref{ex:89-46-lakeshore}) uses \fm{daak=} for what might be the same lake so \fm{daak=} is also used here.

\clearpage
\section{Paragraph 8}\label{sec:89-para-8}

\ex\label{ex:89-122-ran-up-below}%
\exmn{258.1}%
\begingl
	\glpreamble	Q!ū′na ā′daq gū′tsawe ᴀcỵîs ỵînᴀx ke q!ē′wax̣ix̣. //
	\glpreamble	Xʼoon aa áa daak góot sáwé ash ÿís a ÿeenáx̱ kei x̱ʼeiwaxíx. //
	\gla	{} Xʼoon aa \rlap{áa} @ {}
			daak @ \rlap{góot} @ {} @ {} @ {} {}
		\rlap{sáwé} @ {} @ {}
		{} ash ÿís {}
		{} a \rlap{ÿeenáx̱} @ {} {}
		kei @ \rlap{x̱ʼeiwaxíx.} @ {} @ {} @ {} @ {} //
	\glb	{} xʼoon aa á -μ
			daak= {} \rt[¹]{gut} -μμH {} {}
		s- á -wé
		{} ash ÿís {}
		{} a ÿee -náx̱ {}
		kei= x̱ʼe- u- i- \rt[¹]{xix} -μH //
	\glc	{}[\pr{CP} how·many \xx{part} \xx{3n} -\xx{loc}
			\xx{admar}= \xx{zcnj}\· \rt[¹]{go·\xx{sg}} -\xx{var} \·\xx{sub} {}]
		\xx{q}- \xx{foc} -\xx{mdst}
		{}[\pr{PP} \xx{3prx} \xx{ben} {}]
		{}[\pr{PP} \xx{3n·pss} below -\xx{perl} {}]
		up= mouth- \xx{zpfv}- \xx{stv}- \rt[¹]{fall} -\xx{var} //
	\gld	{} how·many one there -at out= \rlap{\xx{csec}.go·\xx{sg}} {} {} {} {}
		ever- \rlap{it.is} {}
		{} him for {}
		{} its below -thru {}
		up\• \rlap{mouth.\xx{pfv}.run} {} {} {} {} //
	\glft	‘Him having gone out there however many times, it ran up along below for him.’
		//
\endgl
\xe

\ex\label{ex:89-123-mouth-inside-red}%
\exmn{258.1}%
\begingl
	\glpreamble	Q!ānᴀ′x łatî′ ᴀłeq!ā′. //
	\glpreamble	X̱ʼaan yáx̱ ÿatee a laká. //
	\gla	{} X̱ʼaan yáx̱ {}
		\rlap{ÿatee} @ {} @ {}
		{} a laká. {} //
	\glb	{} x̱ʼaan yáx̱ {}
		i- \rt[¹]{tiʰ} -μμL
		{} a laká {} //
	\glc	{}[\pr{PP} fire \xx{sim} {}]
		\xx{stv}- \rt[¹]{be} -\xx{var}
		{}[\pr{DP} \xx{3n·\xx{pss}} mouth·inside {}] //
	\gld	{} red like {}
		\rlap{\xx{stv}·\xx{impfv}.be} {} {}
		{} its mouth·inside {} //
	\glft	‘The inside of its mouth is red.’
		//
\endgl
\xe

The form transcribed by \citeauthor{swanton:1909} in (\lastx) is difficult to interpret.
He gives \orth{Q!ānᴀ′x łatî′ ᴀłeq!ā′}, glosses this as “it was around mouth was red” and translates it as “Its mouth was red”.
At first glance the transcription appears to represent something like \fm{x̱ʼanáx̱ latí aleḵʼáa} but this is nonsense in modern Tlingit.
\citeauthor{leer:1977} reads the sentence as \fm{x̱ʼaan yáx̱ ÿatee a laká} \parencite[7]{leer:1977} which has been adopted here because it closely matches \citeauthor{swanton:1909}’s translation, but it does not follow clearly from \citeauthor{swanton:1909}’s transcription.
For example, the final \orth{ᴀłeq!ā′} looks suspiciously like it contains the root \fm{\rt{lex̱ʼw}} ‘red colour’ as in the verbs \fm{kawdiléx̱ʼw} ‘it (berry) turned red, ripened’ and \fm{ayaléix̱ʼw} ‘s/he paints his/her face red’ and the nouns \fm{léix̱ʼw} ‘red paint’ and \fm{kooléix̱ʼwaa} ‘walrus’ (\species{Odobenus}{rosmarus}[L.]); note also the related root \fm{\rt{seḵʼw}} ‘stain, dye’ and nouns \fm{shéix̱ʼw} ‘red alder’ (\species{Alnus}{rubra}[Bong.\ 1832]) and \fm{shoox̱ʼ} ‘robin’ (\species{Turdus}{migratorius}[L.]).

\ex\label{ex:89-124-twice-ask-mother}%
\exmn{258.2}%
\begingl
	\glpreamble	Dᴀxdanī′n ye ᴀc nᴀsnī′ duʟā′ q!ēwawūs!, //
	\glpreamble	Dax̱dahéen yéi ash nasnée, du tláa ax̱ʼeiwawóosʼ //
	\gla	{} \rlap{Dax̱dahéen} @ {}
			yéi @ ash @ \rlap{nasnée} @ {} @ {} @ {} @ {} {}
		{} du tláa {}
		\rlap{ax̱ʼeiwawóosʼ} @ {} @ {} @ {} @ {} @ {} //
	\glb	{} déix̱ -dahéen
			yéi= ash= n- s- \rt[¹]{niʰ} -μμH {} {}
		{} du tláa {}
		a- x̱'e- wu- i- \rt[²]{wuͣsʼ} -μμH //
	\glc	{}[\pr{CP} two -times
			thus= \xx{3prx·o}= \xx{ncnj}- \xx{csv}- \rt[¹]{happen} -\xx{var} \·\xx{sub} {}]
		{}[\pr{DP} \xx{3h·pss} mother {}]
		\xx{arg}- mouth- \xx{pfv}- \xx{stv}- \rt[²]{ask} -\xx{var} //
	\gld	{} \rlap{twice} {}
			thus= him= \rlap{\xx{csec}.make.happen} {} {} {} {} {}
		{} his mother {}
		\rlap{3>3.mouth.\xx{pfv}.ask} {} {} {} {} {} //
	\glft	‘Having done so twice to him, he asked his mother’
		//
\endgl
\xe

\ex\label{ex:89-125-what-is-it}%
\exmn{258.2}%
\begingl
	\glpreamble	“Dā′sayu aʟe′?” //
	\glpreamble	«\!Daa sáyú, atléi?\!» //
	\gla	{} \llap{«\!}Daa {} \rlap{sáyú} @ {} @ {}
		atléi?\!» //
	\glb	{} daa {} s- á -yú
		atléi //
	\glc	{}[\pr{DP} what {}] \xx{q}- \xx{foc} -\xx{dist}
		mom //
	\gld	{} what {} ?
\rlap{it.is} {}
		mom //
	\glft	‘“What is it, mother?”’
		//
\endgl
\xe

\ex\label{ex:89-126-made-new-point}%
\exmn{258.3}%
\begingl
	\glpreamble	Tcuʟe′ yên a′osînî ỵīs ʟāk. //
	\glpreamble	Chʼu tle yan awsinée ÿées tláaḵ. //
	\gla	Chʼu tle yan @ \rlap{awsinée} @ {} @ {} @ {} @ {} @ {}
		{} ÿées tláaḵ {} //
	\glb	chʼu tle ÿán= a- wu- s- i- \rt[¹]{niʰ} -μμH
		{} ÿées tláaḵ {} //
	\glc	just then \xx{term}= \xx{arg}- \xx{pfv}- \xx{csv}- \xx{stv}- \rt[¹]{happen} -\xx{var}
		{}[\pr{DP} new arrowpoint {}] //
	\gld	just then done \rlap{3>3.\xx{pfv}.make.happen} {} {} {} {} {}
		{} new arrowpoint {} //
	\glft	‘So then he made a new arrow point.’
		//
\endgl
\xe

The noun \fm{tláaḵ} in (\lastx) is translated here as ‘arrow point’, but its meaning is somewhat more complex.
\citeauthor{leer:1973} gives one definition as “sharp arrow for killing” and another as “point of spear”, noting also the compound noun \fm{lʼutʼtláaḵ} ‘snake, serpent’ which contains the noun \fm{lʼóotʼ} ‘tongue’ \parencite[08/138]{leer:1973}.
The root \fm{\rt{tlaḵ}} is only found in this noun and there are no roots with similar phonology that have any obviously related meaning.
\citeauthor{swanton:1909}’s translation as “spear” is probably a misunderstanding during the glossing review of his transcription, with the glossing consultant (probably \fm{Daawoolsʼéesʼ} Don/John Cameron) envisioning a spear point when given the noun \fm{tláaḵ} out of context.
This mistake is not obvious here, but sentences (\ref{ex:89-131-shoot-its-mouth}) and (\ref{ex:89-132-when-shot-caw-raven-like}) below clearly use the verb root \fm{\rt[²]{tʼuʼk}} ‘shoot arrow’ and not \fm{\rt[²]{tsaḵ}} ‘spear’ (cf.\ \fm{tsaag̱álʼ} \~\ \fm{tsaag̱áa} ‘spear’).

\ex\label{ex:89-127-run-down-when-open-mouth}%
\exmn{258.3}%
\begingl
	\glpreamble	“Dekī′ q!wᴀn dāq īcī′q îyᴀ′x q!aowut!ā′xe //
	\glpreamble	«\!Deikée xʼwán daak eesheex i yáx̱ x̱ʼawutʼaax̱í; //
	\gla	\llap{«\!}Deikée xʼwán 
		daak @ \rlap{eesheex} @ {} @ {} @ {} @ {} @ {}
		{} {} i \rlap{yáx̱} @ {} {}
			\rlap{x̱ʼawutʼaax̱í;} @ {} @ {} @ {} @ {} {}  //
	\glb	\pqp{}deikée xʼwán
		daak= {} i- d- sh- \rt{xix} -μμL
		{} {} i ÿá -x̱ {}
			x̱ʼe- wu- \rt[¹]{tʼax̱} -μμL -í {} //
	\glc	\pqp{}seaward \xx{imp}
		\xx{admar}= \xx{zcnj}\· \xx{2sg·s}- \xx{mid}- \xx{peg}- \rt[²]{fall} -\xx{var}
		{}[\pr{CP} {}[\pr{PP} \xx{2sg·pss} face -\xx{pert} {}]
			mouth- \xx{pfv}- \rt[¹]{open·mouth} -\xx{var} -\xx{sub} {}] //
	\gld	\pqp{}upward be·sure
		asea\• \rlap{\xx{imp}.you·\xx{sg}.run} {} {} {} {} {}
		{} {} your face -at {}
			\rlap{mouth.\xx{pfv}.mouth·open} {} {} {} -when {} //
	\glft	‘“Be sure to run out to the water when it opens its mouth at you;’
		//
\endgl
\xe

\clearpage
\ex\label{ex:89-128-claws-attack}%
\exmn{258.4}%
\begingl
	\glpreamble	x̣ākᵘ qâ′djî gᴀ′laat. //
	\glpreamble	a x̱aakw ḵaa jigala.átch. //
	\gla	{} a x̱aakw {}
		ḵaa @ \rlap{jigala.átch.} @ {} @ {} @ {} @ {} @ {} //
	\glb	{} a x̱aakw {}
		ḵaa= ji- g- l- \rt[¹]{.at} -μ -ch //
	\glc	{}[\pr{DP} \xx{3n·pss} claw {}]
		\xx{4h·o}= hand- \xx{gcnj}- \xx{csv}- \rt[¹]{go·\xx{pl}} -\xx{var} //
	\gld	{} its claws {}
		ppl\• \rlap{\xx{hab}.attack} {} {} {} {} {} //
	\glft	‘its claws always attack people.’
		//
\endgl
\xe

The sentences in (\ref{ex:89-127-run-down-when-open-mouth}) and (\ref{ex:89-128-claws-attack}) are given by \citeauthor{swanton:1909} as a single sentence but they have much more of the feel of two separate sentences with the subordinate clause in the middle being dependent on the first of the two sentences as a ‘while’ clause.
Since these sentences are given as a single utterance in \citeauthor{swanton:1909}’s transcription, the two sentences are here presented as a single utterance divided into two independent units by a semicolon.

\ex\label{ex:89-129-fathers-canoe}%
\exmn{258.4}%
\begingl
	\glpreamble	Iī′c yāgu′ awe′. //
	\glpreamble	I éesh yaagú áwé.\!» //
	\gla	{} I éesh \rlap{yaagú} @ {} {} \rlap{áwé.\!»} @ {} //
	\glb	{} i éesh yaakw -í {} á -wé //
	\glc	{}[\pr{DP} \xx{2sg·pss} father boat -\xx{pss} {}] \xx{cpl} -\xx{mdst} //
	\gld	{} your father’s canoe {} {} \rlap{it.is} {} //
	\glft	‘It is your father’s canoe.”’
		//
\endgl
\xe

\ex\label{ex:89-130-when-lakeshore-open-mouth}%
\exmn{258.4}%
\begingl
	\glpreamble	Aq! āyᴀ′x dugudē′awe ᴀcyᴀ′x q!ē′wat!āx. //
	\glpreamble	Áxʼ áa yaax̱t wugoodí áwé ash yáx̱ x̱ʼeiwatʼaax̱. //
	\gla	{} {} \rlap{Áxʼ} @ {} {}
			{} áa \rlap{yaax̱t} @ {} {} 
			\rlap{wugoodí} @ {} @ {} @ {} {} 
		\rlap{áwé} @ {} +
		{} ash \rlap{yáx̱} @ {} {} 
		\rlap{x̱ʼeiwatʼaax̱.} @ {} @ {} @ {} @ {} //
	\glb	{} {} á -xʼ {}
			{} áa yaax̱ -t {}
			wu- \rt[¹]{gut} -μμL -í {}
		á -wé
		{} ash ÿá -x̱ {}
		x̱ʼe- wu- i- \rt[¹]{tʼax̱} -μμL //
	\glc	{}[\pr{CP} {}[\pr{PP} \xx{3n} -\xx{loc} {}]
			{}[\pr{PP} lake side -\xx{pnct} {}]
			\xx{pfv}- \rt[¹]{go·\xx{sg}} -\xx{var} -\xx{sub} {}]
		\xx{foc} -\xx{mdst}
		{}[\pr{PP} \xx{3prx·pss} face -\xx{pert} {}]
		mouth- \xx{pfv}- \xx{stv}- \rt[¹]{mouth·open} -\xx{var} //
	\gld	{} {} there -at {}
			{} lake side -to {}
			\rlap{\xx{pfv}.go·\xx{sg}} {} {} -when {}
		\rlap{it.is} {}
		{} his face -at {}
		\rlap{mouth.\xx{pfv}.mouth·open} {} {} {} {} //
	\glft	‘So it was when he went there to the lakeshore that it openened its mouth at him.’
		//
\endgl
\xe

\ex\label{ex:89-131-shoot-its-mouth}%
\exmn{258.5}%
\begingl
	\glpreamble	“Dułēq!ᴀ′ tcᴀ t!u′k.” //
	\glpreamble	«\!Du laká chʼa tʼúk.\!» //
	\gla	{} \llap{«\!}Du laká {}
		chʼa \rlap{tʼúk.\!»} @ {} @ {} @ {} //
	\glb	{} du laká {}
		chʼa {} {} \rt[²]{tʼuʼk} -μH //
	\glc	{}[\pr{DP} \xx{3h·pss} mouth·inside {}]
		just \xx{zcnj}\· \xx{2sg·s}\· \rt[²]{shoot·arrow} -\xx{var} //
	\gld	{} its mouth·inside {}
		just \rlap{\xx{imp}.you·\xx{sg}.shoot·arrow} {} {} {} //
	\glft	‘“Just shoot the inside of its mouth.”’
		//
\endgl
\xe

The interpretation of \citeauthor{swanton:1909}’s \orth{Dułēq!ᴀ′} in (\lastx) as \fm{du laká} ‘his inside of the mouth’ is in keeping with (\ref{ex:89-123-mouth-inside-red}) where his \orth{ᴀłeq!ā′} was interpreted as \fm{a laká} ‘its inside of the mouth’.
As before, there is the alternative possibility of some kind of connection to the root \fm{\rt{lex̱ʼw}} ‘red colour’ and related red-associated vocabulary.
This connection is underlined by \citeauthor{swanton:1909}’s gloss of \orth{Dułēq!ᴀ′} in (\lastx) as “Its (mouth’s) redness”.
A plausible explanation for this confusion is that the speaker originally said \fm{laká} which \citeauthor{swanton:1909} then transcribed as \orth{łēq!ᴀ′} and then later while checking his transcription he offered his consultant a pronunciation like [\ipa{ɬe.qʼa}] which was interpreted as something like \fm{léx̱ʼwaa} and thus connected to the root \fm{\rt{lex̱ʼw}}.
\citeauthor{swanton:1909}’s translation completely lacks any mention of redness in both cases, further supporting this conjecture.

\ex\label{ex:89-132-when-shot-caw-raven-like}%
\exmn{258.5}%
\begingl
	\glpreamble	Tc!uʟe′ awut!ū′guawe ye uduwaᴀ′x “G̣ā” yēł yᴀ′x. //
	\glpreamble	Chʼu tle awutʼóogu áwé yéi wduwa.áx̱ «\!G̱áa!\!», yéil yáx̱. //
	\gla	{} Chʼu tle \rlap{awutʼóogu} @ {} @ {} @ {} @ {} {}
		\rlap{áwé} @ {}
		yéi @ \rlap{wduwa.áx̱} @ {} @ {} @ {} @ {}
		«\!G̱áa!\!»,
		{} yéil yáx̱. {} //
	\glb	{} chʼu tle a- wu- \rt[²]{tʼuʼk} -μμH -í {}
		á -wé
		yéi= wu- du- i- \rt[²]{.ax̱} -μH
		\pqp{}g̱áa
		{} yéil yáx̱ {} //
	\glc	{}[\pr{CP} just then \xx{arg}- \xx{pfv}- \rt[²]{shoot·arrow} -\xx{var} -\xx{sub} {}]
		\xx{foc} -\xx{mdst}
		thus= \xx{pfv}- \xx{4h·s}- \xx{stv}- \rt[²]{hear} -\xx{var}
		\pqp{}caw
		{}[\pr{PP} raven \xx{sim} {}] //
	\gld	{} just then \rlap{3>3.\xx{pfv}.shoot·arrow} {} {} {} -when {}
		\rlap{it.is} {}
		thus\• \rlap{\xx{pfv}.ppl.hear} {} {} {} {}
		\pqp{}caw
		{} raven like {} //
	\glft	‘It was right when he shot it that there was heard “G̱áa!”, like a raven.’
		//
\endgl
\xe

\ex\label{ex:89-133-like-its-thwarts-were-cut}%
\exmn{258.6}%
\begingl
	\glpreamble	Ayê′x caỵa′oʟ̣ix̣ᴀc yeyᴀ′x awe′ wūne′ ayêxak!ā′wu. //
	\glpreamble	A yáx̱ shaÿawdlixáshi yé yáx̱ áwé woonei a yax̱akʼáawu. //
	\gla	{} A yáx̱ {}
		{} {} {} \rlap{shaÿawdlixáshi} @ {} @ {} @ {} @ {} @ {} @ {} @ {} @ {} {}
			yé {} yáx̱ {}
		\rlap{áwé} @ {}
		\rlap{woonei} @ {} @ {} @ {} 
		{} a \rlap{yax̱akʼáawu.} @ {} {} //
	\glb	{} a yáx̱ {}
		{} {} {} sha- ÿ- wu- d- l- i- \rt[²]{xash} -μH -i {}
			yé {} yáx̱ {}
		á -wé
		wu- i- \rt[¹]{neʰ} -μμL
		{} a yax̱akʼáaʷ -í {} //
	\glc	{}[\pr{PP} \xx{3n} \xx{sim} {}]
		{}[\pr{PP} {}[\pr{DP} {}[\pr{CP}
				head- \xx{qual}- \xx{pfv}- \xx{pasv}- \xx{xtn}- \xx{stv}- \rt[²]{cut} -\xx{var} -\xx{rel} {}]
			way {}] \xx{sim} {}]
		\xx{foc} -\xx{mdst}
		\xx{pfv}- \xx{stv}- \rt[¹]{happen} -\xx{var}
		{}[\pr{DP} \xx{3n·pss} thwart -\xx{pss} {}] //
	\gld	{} it like {}
		{} {} {} \rlap{\xx{pfv}.\xx{pasv}.cut} {} {} {} {} {} {} {} -that {}
			way {} like {}
		\rlap{it.is} {}
		\rlap{\xx{pfv}.happen} {} {} {}
		{} its thwarts {} {} //
	\glft	‘Like that, it happened as if they had been cut out, its thwarts.’
		//
\endgl
\xe

The intended meaning of (\lastx) is not entirely clear.
It does not make sense literally given that (\ref{ex:89-135-thwarts-wide}) describes the canoe as still having thwarts.\footnote{Modern English \fm{thwart} [\ipa{θwɔɹt}] derives from Old Norse \fm{þver-t} ‘across, athwart’ < \fm{þverr} < PGmc \fm[*]{þwerhaz} < \fm[*]{þerh-} < PIE \fm[*]{terkʷ-} ‘spin, turn’.
Cognates include Danish \fm{tvær} ‘sullen’, West Frisian \fm{dwers} ‘across, beyond’, Dutch \fm{dwars} ‘crosswise; stubborn’, German \fm{quer} ‘crosswise; cross, angry’ (this may lead to \fm{queer} via Scots).} One possible interpretation is that (\lastx) describes the pain experienced by the canoe after having been shot in (\ref{ex:89-132-when-shot-caw-raven-like}).
Thus the raven-like \fm{g̱áa!} in (\ref{ex:89-132-when-shot-caw-raven-like}) is a cry of pain, and (\lastx) says that this pain was as though integral part of its body (clavicles? ribs? sternum?)\ had been severed.

The noun \fm{a yax̱akʼáawu} [\ipa{ʔà jà.χà.ˈkʼáː.wù}] ‘its thwart(s)’ in (\lastx) refers to the crosspieces within a traditional dugout canoe that keep the sides spread apart after the canoe has been steamed.
The same term is applied to crosspieces in skiffs and other small boats of European origin, but in a traditional Tlingit canoe the thwarts are not for seating unlike in European designs.
This noun has the apperance of a compound but its components are as yet unidentified \parencites[f04/84]{leer:1973}.
The initial syllable \fm{yax̱} could have partly come from \fm{yaakw} ‘canoe’; \textcite[66]{leer:1978b} gives the form \fm{ÿax̱akʼáawu} with an initial \fm{ÿ} which would be incompatible with \fm{yaakw} (Tongass \fm{yaàkw} < PT \fm[*]{yaʰgʷ} <?
PND), but he gives no evidence for this \fm{ÿ} and it appears here clearly without \citeauthor{swanton:1909}’s \orth{ỵ}.
The noun \fm{yax̱akʼáawu} only occurs with possession but the final syllable looks to be a labialized form of the possessive suffix \fm{-í}.
This implies that the stem is either \fm{–kʼáaw} or \fm{–kʼáaʷ}, the latter with an occult labial like the nouns \fm{ḵáaʷ} ‘man’ and \fm{naaʷ} ‘corpse’.
There are no obvious correspondences with this stem in the lexicon; only \fm{\rt[¹]{kʼa}} ‘too small’ bears a resemblance with a meaning perhaps from the narrow breadth of an unspread canoe.
\FIXME{Check Haida, Tsimshianic lexica for comparanda.}

\ex\label{ex:89-134-copper-boat}%
\exmn{258.7}%
\begingl
	\glpreamble	Xᴀtc ēq yāˈgu ayu′ //
	\glpreamble	X̱ách eiḵ yaagú áyú; //
	\gla	X̱ách {} eiḵ \rlap{yaagú} @ {} {} \rlap{áyú;} @ {} //
	\glb	x̱áju {} eiḵ yaakw -í {} á -yú //
	\glc	actually {}[\pr{DP} copper boat -\xx{pss} {}] \xx{foc} -\xx{dist} //
	\gld	actually {} copper canoe -of {} \rlap{it.is} {} //
	\glft	‘Actually it was a canoe of copper;’
		//
\endgl
\xe

\ex\label{ex:89-135-thwarts-wide}%
\exmn{258.7}%
\begingl
	\glpreamble	yēkᵘdīwuq! ayᴀxak!āw′o. //
	\glpreamble	yéi kwdiwóox̱ʼ a yax̱akʼáawu. //
	\gla	yéi @ \rlap{kwdiwóox̱ʼ} @ {} @ {} @ {} @ {} @ {}
		{} a \rlap{yax̱akʼáawu.} @ {} {} //
	\glb	yéi= k- u- d- i- \rt[¹]{wux̱ʼ} -μμH
		{} a yax̱akʼáaʷ -í {} //
	\glc	thus= \xx{cmpv}- \xx{irr}- \xx{mid}- \xx{stv}- \rt[¹]{wide} -\xx{var}
		{}[\pr{DP} \xx{3n·pss} thwart -\xx{pss} {}] //
	\gld	thus= \rlap{\xx{cmpv}.\xx{stv}·\xx{impfv}.\xx{pl}.wide} {} {} {} {} {}
		{} its thwart {} {} //
	\glft	‘they were wide, its thwarts.’
		//
\endgl
\xe

\citeauthor{swanton:1909} gives (\ref{ex:89-134-copper-boat}) and (\ref{ex:89-135-thwarts-wide}) as a single sente̞nce, but it is clear from the syntax and interpretation that these are two separate clauses with no embedding of either one within the other.
The clause in (\ref{ex:89-134-copper-boat}) could be mistaken for a focused phrase but \citeauthor{swanton:1909}’s translation “It was a copper canoe” clearly indicates that this is a verbless identity predication and not focus.
Furthermore, the verb in (\ref{ex:89-135-thwarts-wide}) can have only one argument because it is an unaccusative intransitive and since \fm{a yax̱akʼáawu} is apparently the argument, the material in (\ref{ex:89-134-copper-boat}) cannot be focused from the clause in (\ref{ex:89-135-thwarts-wide}).

The verb \fm{yéi kwdiwóox̱ʼ} in (\ref{ex:89-135-thwarts-wide}) is a comparative stative imperfective.
The presence of \fm{d-} in comparative stative imperfectives is an idiosyncratic marker of argument pluralization which does not occur anywhere else in the language.
This pluralizing \fm{d-} is stereotypically accompanied by the plural suffix \fm{-xʼ} but that suffix generally does not appear when the verb root has a coda consonant like the final \fm{x̱ʼ} /\ipa{χʼʷ}/ in \fm{\rt[¹]{wux̱ʼ}} ‘wide’.

\ex\label{ex:89-136-actually-that-way-of-copper}%
\exmn{258.8}%
\begingl
	\glpreamble	Xᴀtc tc!ᴀs ʟe yē′tî ē′qayu, //
	\glpreamble	X̱ách chʼas tle yéi ÿatee eiḵ áyú; //
	\gla	X̱ách \rlap{chʼas} @ {} tle
		yéi @ \rlap{ÿatee} @ {} @ {}
		{} eiḵ {} \rlap{áyú,} @ {} //
	\glb	x̱áju chʼa =s tle
		yéi= i- \rt[¹]{tiʰ} -μμL
		{} eiḵ {} á -yú //
	\glc	actually just =\xx{dub} then
		thus= \xx{stv}- \rt[¹]{be} -\xx{var}
		{}[\pr{DP} copper {}] \xx{foc} -\xx{dist} //
	\gld	actually just \•maybe then thus\• \rlap{\xx{stv}·\xx{impfv}.be} {} {}
		{} copper {} \rlap{it.is} {} //
	\glft	‘Actually it was just that way being copper;’
		//
\endgl
\xe

\ex\label{ex:89-137-broke-canoe-all}%
\exmn{258.8}%
\begingl
	\glpreamble	ʟe kā′wawᴀʟ! yū′yākᵘ łdakᴀ′t ā. //
	\glpreamble	tle kaawawálʼ yú yaakw, ldakát á. //
	\gla	tle \rlap{kaawawálʼ} @ {} @ {} @ {} @ {}
		{} yú yaakw, {}
		{} ldakát á. {} //
	\glb	tle k- wu- i- \rt[¹]{walʼ} -μH
		{} yú yaakw {}
		{} ldakát á {} //
	\glc	then \xx{qual}- \xx{pfv}- \xx{stv}- \rt[¹]{break} -\xx{var}
		{}[\pr{DP} \xx{dist} boat {}]
		{}[\pr{DP} all \xx{3n} {}] //
	\gld	then \rlap{\xx{pfv}.break} {} {} {} {}
		{} that canoe {}
		{} all it {} //
	\glft	‘it broke apart, that canoe, all of it.’
		//
\endgl
\xe

Like (\ref{ex:89-134-copper-boat}) and (\ref{ex:89-135-thwarts-wide}), the forms in (\ref{ex:89-136-actually-that-way-of-copper}) and (\ref{ex:89-137-broke-canoe-all}) are given by \citeauthor{swanton:1909} as a single sentence.
They are analyzed here as two independent clauses without either embedded in the other.
This is actually implied by \citeauthor{swanton:1909}’s gloss of \fm{tle} as ‘and’, reflecting that this particle is often used as a discourse connector between sentences.

\citeauthor{swanton:1909}’s transcription of \fm{yéi ÿatee} as \orth{yē′tî} in (\ref{ex:89-136-actually-that-way-of-copper}) is not a mistake.
Speakers even today often contract \fm{yéi yatee} [\ipa{jéː jà.ˈtʰìː}] as \fm{yéiatee} [\ipa{jéə.ˈtʰìː}] or \fm{yéitee} [\ipa{jéː.ˈtʰìː}] in very rapid speech.
It is interesting that this contraction was not ‘undone’ by his consultant during \citeauthor{swanton:1909}’s process of transcription and spoken verification, but if \citeauthor{swanton:1909} had developed his skill at imitating Tlingit by this point his consultants might have found it unremarkable.

The syntax of (\ref{ex:89-136-actually-that-way-of-copper}) is unusual and its fine-grained structure has not been confirmed.
Its peculiarity arises from the combination of a main verb \fm{yéi ÿatee} ‘it is thus’ and the phrase \fm{eiḵ áyú}.
The latter looks at first like a focused phrase given the particle \fm{áyú}, but focused phrases only occur before the verb in their clause and never after.
The alternative is that \fm{eiḵ áyú} is a verbless predicate ‘it is copper’ and this interpretation is supported by \citeauthor{swanton:1909}’s gloss of the phrase as “copper it was” and his consequent translation “it was a copper canoe”.

\ex\label{ex:89-138-night-pack-home-to-mother}%
\exmn{258.9}%
\begingl
	\glpreamble	Tāt ỵinᴀ′x awe′ ā′waya du hî′tî dê duʟā′ xᴀ′ndî. //
	\glpreamble	Taat ÿeennáx̱ áwé aawayaa du hídidé, du tláa x̱ánde. //
	\gla	{} Taat \rlap{ÿeennáx̱} @ {} {} \rlap{áwé} @ {}
		\rlap{aawayaa} @ {} @ {} @ {} @ {}
		{} du \rlap{hídidé} @ {} @ {} {} +
		{} du tláa \rlap{x̱ánde.} @ {} {} //
	\glb	{} taat ÿeen -náx̱ {} á -wé
		a- wu- i- \rt[²]{ya} -μμL
		{} du hít -í -dé {}
		{} du tláa x̱án -dé {} //
	\glc	{}[\pr{PP} night middle -\xx{perl} {}] \xx{foc} -\xx{mdst}
		\xx{arg}- \xx{pfv}- \xx{stv}- \rt[²]{pack} -\xx{var}
		{}[\pr{PP} \xx{3h·pss} house -\xx{pss} -\xx{all} {}]
		{}[\pr{PP} \xx{3h·pss} mother near -\xx{pss} {}] //
	\gld	{} night middle -thru {} \rlap{it.is} {}
		\rlap{3>3.\xx{pfv}.pack} {} {} {} {}
		{} his house {} -to {}
		{} his mother near -to {} //
	\glft	‘It was through the middle of the night that he packed it to his house, near to his mother.’
		//
\endgl
\xe

\ex\label{ex:89-139-nobody-knew}%
\exmn{258.9}%
\begingl
	\glpreamble	ʟēł Łīngî′ttc wusko′. //
	\glpreamble	Tléil leengítch wuskú. //
	\gla	Tléil {} \rlap{leengítch} @ {} {}
		\rlap{wuskú.} @ {} @ {} @ {} @ {} @ {} @ {} //
	\glb	tléil {} leengít -ch {}
		ⱥ- u- wu- d- s- \rt[²]{ku} -μH //
	\glc	\xx{neg} {}[\pr{DP} person -\xx{erg} {}]
		\xx{arg}- \xx{irr}- \xx{pfv}- \xx{mid}- \xx{xtn}- \rt[²]{know} -\xx{var} //
	\gld	no {} person {} {} 
		\rlap{3>3.\xx{irr}.\xx{pfv}.know} {} {} {} {} {} {} //
	\glft	‘No person knew about it.’
		//
\endgl
\xe

\section{Paragraph 9}\label{sec:89-para-9}

\ex\label{ex:89-140-building-lotsa-houses-from-copper}%
\exmn{258.11}%
\begingl
	\glpreamble	Tc!uʟe′ ā′ʟen hî′txawe ỵāˈnᴀłyᴀx yuī′q. //
	\glpreamble	Chʼu tle aatlein hítxʼ áwé ÿaa analyéx̱ yú eeḵx̱. //
	\gla	Chʼu tle {} \rlap{aatlein} @ {} \rlap{hítxʼ} @ {} {} \rlap{áwé} @ {}
		ÿaa @ \rlap{analyéx̱} @ {} @ {} @ {} @ {} +
		{} yú \rlap{eeḵx̱.} @ {} {} //
	\glb	chʼu tle {} aa =tlein hít -xʼ {} á -wé
		ÿaa= a- n- l- \rt[²]{yex̱} -μH
		{} yú eeḵ -x̱ {} //
	\glc	just then {}[\pr{DP} \xx{part} =big house -\xx{pl} {}] \xx{foc} -\xx{mdst}
		along= \xx{arg}- \xx{ncnj}- \xx{xtn}- \rt[²]{make} -\xx{var}
		{}[\pr{DP} \xx{dist} copper -\xx{pert} {}] //
	\gld	just then {} \rlap{many} {} house -s {} \rlap{it.is} {}
		along \rlap{3>3.\xx{prog}.build} {} {} {} {}
		{} that copper -of {} //
	\glft	‘So then he was building many houses from that copper.’
		//
\endgl
\xe

\citeauthor{swanton:1909}’s transcription \orth{yuī′q} in (\lastx) is surprising because it implies the speaker said \fm{eeḵ} instead of \fm{eiḵ} ‘copper’ as in (\ref{ex:89-134-copper-boat}) and (\ref{ex:89-136-actually-that-way-of-copper}).
This could be considered a mistake on \citeauthor{swanton:1909}’s part except for the fact that \fm{eeḵ} is exactly the form used in more conservative dialects without lexicalized uvular lowering.
Rather than potentially miscorrecting this and thereby losing potentially significant evidence, the retranscription maintains \citeauthor{swanton:1909}’s transcription as \fm{eeḵ} in this one sentence.

The end of \fm{eeḵx̱} in (\lastx) is not present in the original which implies only \fm{yú eeḵ} without the pertingent postposition.
This would be ungrammatical however because without a postposition this DP would have to be interpreted as either the subject or the object of the verb.
The object is certainly \fm{aatlein hítxʼ} ‘many houses’ and the subject is presumably the young man (son of the original protagonist), so this additional DP would have no grammatical role in the sentence.
Since the word \fm{eeḵ} ends with a uvular stop and the pertingent postposition \fm{-x̱} is a uvular fricative, it is entirely plausible that \citeauthor{swanton:1909} misheard the fricative as release noise from the stop.
The pertingent postposition is expected here since with \fm{l-\rt[²]{yex̱}} ‘make, build’ it predictably indicates the material from which objects are constructed.

The English translation of (\lastx) uses past tense and progressive aspect.
The Tlingit form is, like most sentences in Tlingit, unmarked for tense and only indicates progressive aspect (\fm{ÿaa=} + \fm{n-} + \fm{-μH}).
Thus the appearance of past tense in the translation should not be taken to mean that the Tlingit original is also past tense.
The interpretation of the Tlingit form is plausibly past tense because most of the narrative has taken a past-looking perspective on situations other than speech acts, but this is wholly implicit and is not expressed by any overt grammar in the sentence.

\ex\label{ex:89-141-always-hammering}%
\exmn{258.11}%
\begingl
	\glpreamble	Yutcā′ctaỵīq! ade′awe ā′tᴀq!anutc ʟāq sᴀkᵘ kīs sᴀkᵘ. //
	\glpreamble	Yú cháash taÿeexʼ, aadé áwé aa tʼéx̱ʼ nuch, tláaḵ sákw, kées sákw. //
	\gla	{} Yú cháash \rlap{taÿeexʼ,} @ {} {}
		{} \rlap{aadé} @ {} {} \rlap{áwé} @ {}
		aa @ \rlap{tʼéx̱ʼ} @ {} @ \•nuch, +
		{} tláaḵ sákw, {}
		{} kées sákw. {} //
	\glb	{} yú cháash taÿee -xʼ {}
		{} á -dé {} á -wé
		aa= \rt[²]{tʼex̱ʼ} -μH =nooch
		{} tláaḵ sákw {}
		{} kées sákw {} //
	\glc	{}[\pr{PP} \xx{dist} brush below -\xx{loc} {}]
		{}[\pr{PP} \xx{3n} -\xx{all} {}] \xx{foc} -\xx{mdst}
		\xx{part}= \rt[²]{pound} -\xx{var} =\xx{hab·aux}
		{}[\pr{PP} arrowpoint \xx{fut} {}]
		{}[\pr{PP} bracelet \xx{fut} {}] //
	\gld	{} that brush below -at {}
		{} there -to {} \rlap{it.is} {}
		some \rlap{\xx{impfv}.pound} {} \•always
		{} arrowpoint for {}
		{} bracelet for {} //
	\glft	‘Below the brush, it was there that he was always hammering some, for arrow points, for bracelets.’
		//
\endgl
\xe

\ex\label{ex:89-142-no-iron-no-copper}%
\exmn{258.12}%
\begingl
	\glpreamble	ʟēł g̣ayē′s! qōstī′ỵīn qᴀʹtcu ēq yᴀx ỵatī′ỵī ᴀt. //
	\glpreamble	Tléil g̱ayéisʼ ḵusteeyín, ḵachʼu eiḵ yáx̱ ÿateeÿi át. //
	\gla	Tléil {} \rlap{g̱ayéisʼ} @ {} @ {} {}
		\rlap{ḵusteeyín,} @ {} @ {} @ {} @ {} @ {} @ {} 
		\rlap{ḵachʼu} @ {} +
		{} {} {} eiḵ yáx̱ {}
			\rlap{ÿateeÿi} @ {} @ {} @ {} {} át. {} //
	\glb	tléil {} eeḵ- \rt[¹]{yesʼ} -μμH {}
		ḵu- u- d- s- \rt[¹]{tiʰ} -μμL -ín
		ḵa= chʼu 
		{} {} {} eiḵ yáx̱ {}
			i- \rt[¹]{tiʰ} -μμL -i {} át {} //
	\glc	\xx{neg} {}[\pr{DP} copper- \rt[¹]{dark} -\xx{var} {}]
		\xx{areal}- \xx{irr}- \xx{mid}- \xx{xtn}- \rt[¹]{be} -\xx{var} -\xx{past}
		and= just
		{}[\pr{DP} {}[\pr{CP} {}[\pr{PP} copper \xx{sim} {}]
			\xx{stv}- \rt[¹]{be} -\xx{var} -\xx{rel} {}] thing {}] //
	\gld	not {} \rlap{iron} {} {} {}
		\rlap{\xx{irr}.\xx{stv}·\xx{impfv}.exist.\xx{past}} {} {} {} {} {} {}
		\rlap{or} {}
		{} {} {} copper like {}
			\rlap{\xx{stv}·\xx{impfv}.be} {} {} -that {} thing {} //
	\glft	‘No iron used to exist, nor anything that is like copper.’
		//
\endgl
\xe

\ex\label{ex:89-143-hammering-tinaa}%
\exmn{258.13}%
\begingl
	\glpreamble	Tînna′ yᴀx ts!u at!ē′q!. //
	\glpreamble	Tináa yáx̱ tsú atʼéix̱ʼ. //
	\gla	{} {} \rlap{Tináa} @ {} @ {} {} yáx̱ {} tsú
		\rlap{atʼéix̱ʼ.} @ {} @ {} //
	\glb	{} {} \rt{tin} -μμL -áa {} yáx̱ {} tsú
		a- \rt[²]{tʼex̱ʼ} -μμH //
	\glc	{}[\pr{PP} {}[\pr{N} \rt{??} -\xx{var} -\xx{nmz} {}] \xx{sim} {}] also
		\xx{arg}- \rt[²]{pound} -\xx{var} //
	\gld	{} {} \rlap{copper} {} {} {} like {} also
		\rlap{3>3.\xx{impfv}.pound} {} {} //
	\glft	‘He is also hammering (a thing) like a \fm{tináa}.’
		//
\endgl
\xe

The sentence in (\lastx) is syntactically somewhat unusual and might be incorrectly analyzed.
The specific problem lies in the interpretation of \citeauthor{swanton:1909}’s transcription \orth{yᴀx}.
He glosses this as “like” which implies that it is the similative postposition \fm{yáx̱} ‘like, similar to’.
If this is correct then there is a phrase \fm{tináa yáx̱} ‘like a copper’ which must be an adjunct, leaving the object of the verb unspoken.
But if \orth{yᴀx} is instead interpreted as the terminative preverb \fm{yax̱=} the sentence would mean something like ‘he is also finishing hammering a copper’ and then \fm{tináa} is the object.
The problem with this analysis is that the adverb \fm{tsú} ‘also’ is a second position particle which would be expected to come immediately after \fm{tináa} and not after \fm{yax̱=}.
\citeauthor{swanton:1909}’s analysis has been maintained here but it is still in doubt and needs further study.

The noun \fm{tináa} in (\lastx) is typically given in local English as just ‘copper’ but it refers to a specific object and not to the metal.
Alternative translations include ‘copper shield’ (which \citeauthor{swanton:1909} used), ‘copper plate’, and ‘copper token’.
Other attested forms of the noun include \fm{teenáa} and \fm{tinaa}, and in Tongass Tlingit it is \fm{tinaà} [\ipa{tʰi.ˈnaʰ}].
The conservative form \fm{teenáa} implies a root \fm{\rt{tin}} with \fm{-μ} stem variation and the instrument nominalization suffix \fm{-áa}.
Identification of the root is difficult; there is at least \fm{\rt[²]{tin}} ‘see’, \fm{\rt[²]{tin}} ‘go away angry’, and \fm{\rt[¹]{tin}} ‘monstrous’, none of which is straightforwardly connected to the concept of a \fm{tináa} copper.
The Tongass form \fm{tinaà} is unusual because the expected cognate to Northern and Southern Tlingit forms would be \fm[*]{teènaa} [\ipa{ˈtʰiʰ.naː}].

\ex\label{ex:89-144-lug-inside}%
\exmn{258.13}%
\begingl
	\glpreamble	ʟe nēł ỵīˈya ᴀcᴀ′kᴀnadjᴀł. //
	\glpreamble	Tle neil ÿée yaa ashakanajél. //
	\gla	Tle {} neil \rlap{ÿée} @ {} {}
		yaa @ \rlap{ashakanajél.} @ {} @ {} @ {} @ {} @ {} //
	\glb	tle {} neil ÿee -H {}
		ÿaa= a- sha- k- n- \rt[²]{jel} -μH //
	\glc	then {}[\pr{PP} inside below -\xx{loc} {}]
		along= \xx{arg}- head- \xx{qual}- \xx{ncnj}- \rt[²]{mv·hand} -\xx{var} //
	\gld	then {} inside below -at {}
		along \rlap{3>3.\xx{prog}.carry·load} {} {} {} {} {} //
	\glft	‘Then he is carrying it all inside.’
		//
\endgl
\xe

\ex\label{ex:89-145-throw-garbage-on-him}%
\exmn{258.14}%
\begingl
	\glpreamble	Tc!uʟe′ dokᴀ′t ku-doxē′tc q!aīte′. //
	\glpreamble	Chʼu tle du kát kudux̱eichch x̱ʼa.eetí. //
	\gla	Chʼu tle {} du \rlap{kát} @ {} {}
		\rlap{kudux̱eichch} @ {} @ {} @ {} @ {} @ {} @ {}
		{} \rlap{x̱ʼa.eetí.} @ {} {} //
	\glb	chʼu tle {} du ká -t {}
		k- u- du- d- \rt[²]{x̱ech} -μμL -ch
		{} x̱ʼé- .eetí {} //
	\glc	just then {}[\pr{PP} \xx{3h·pss} \xx{hsfc} -\xx{pnct} {}]
		\xx{qual}- \xx{zpfv}- \xx{4h·s}- \xx{mid}- \rt[²]{throw} -\xx{var} -\xx{rep}
		{}[\pr{DP} mouth- remains {}] //
	\gld	just then {} his atop -to {}
		\rlap{\xx{hab}.ppl.throw·\xx{fc}} {} {} {} {} {} {}
		{} \rlap{garbage} {} {} //
	\glft	‘Then people were throwing it on him, garbage.’
		//
\endgl
\xe

\ex\label{ex:89-146-hammering-aristocrat}%
\exmn{258.14}%
\begingl
	\glpreamble	“Yā′dat!ᴀ′q!-anqā′wo.” //
	\glpreamble	«\!Yá datʼéx̱ʼ aanḵáawu.\!» //
	\gla	{} \llap{«\!}Yá {} \rlap{datʼéx̱ʼ} @ {} @ {} {}
			\rlap{aanḵáawu.} @ {} @ {} {} //
	\glb	{} yá {} d- \rt[²]{tʼex̱ʼ} -μH {}
			aan- ḵáaʷ -í {} //
	\glc	{}[\pr{DP} \xx{prox} {}[\pr{N} \xx{apsv}- \rt[²]{pound} -\xx{var} {}]
			town- man -\xx{pss} {}] //
	\gld	{} this {} \rlap{hammering} {} {} {}
			\rlap{aristocrat} {} {} {} //
	\glft	‘“This hammering aristocrat.”’
		//
\endgl
\xe

\ex\label{ex:89-147-having-finished-many-pounded-things-inside}%
\exmn{258.14}%
\begingl
	\glpreamble	Yên asnī′ wehî′t qa yu′tînna de cā′ỵadîhēn yū′nîłq! ade′ ᴀt!aq!ᴀ′t. //
	\glpreamble	Yan asnée wé hít ka yú tináa, de shaÿadihéin yú neilxʼ aadé atʼéx̱ʼ át. //
	\gla	{} Yan @ \rlap{asnée} @ {} @ {} @ {} @ {}
			{} wé hít {} ḵa {} yú tináa, {} {} +
		de \rlap{shaÿadihéin} @ {} @ {} @ {} @ {} @ {} @ {}
		{} yú {} {} \rlap{neilxʼ} @ {} {}
				{} \rlap{aadé} @ {} {} +
				\rlap{atʼéx̱ʼ} @ {} @ {} @ {} {} át {} //
	\glb	{} ÿán= a- {} s- \rt[¹]{niʰ} -μμH
			{} wé hit {} ḵa {} yú tináa {} {}
		de sha- ÿ- d- i- \rt[¹]{he} -eμH -n
		{} yú {} {} neil -xʼ {}
				{} á -dé {} 
				a- \rt[²]{tʼex̱ʼ} -μH {} {} át {} //
	\glc	{}[\pr{CP} \xx{term}= \xx{arg}- \xx{zcnj}\· \xx{csv}- \rt[¹]{happen} -\xx{var}
			{}[\pr{DP} \xx{mdst} house {}] and {}[\pr{DP} \xx{dist} copper {}] {}]
		already head- \xx{qual}- \xx{mid}- \xx{stv}- \rt[¹]{many} -\xx{var} -\xx{nsfx}
		{}[\pr{DP} \xx{dist} {}[\pr{CP} {}[\pr{PP} inside -\xx{loc} {}]
				{}[\pr{PP} \xx{3n} -\xx{all} {}]
				\xx{arg}- \rt[²]{pound} -\xx{var} \·\xx{rel} {}] thing {}] //
	\gld	{} done \rlap{3>3.\xx{csec}.do} {} {} {} {}
			{} that house {} and {} that copper {} {}
		already \rlap{\xx{stv}·\xx{impfv}.many} {} {} {} {} {} {} 
		{} that {} {} inside -at {}
				{} it -way {} 
				\rlap{3>3.\xx{impfv}.pound} {} {} \·that {} thing {} //
	\glft	‘Having finished making that house and those coppers, there is already much of those things he hammers that way inside.’
		//
\endgl
\xe

\ex\label{ex:89-148-run-out-to-get-laughed-at}%
\exmn{259.2}%
\begingl
	\glpreamble	Tc!aye′ u′xanᴀx dułcu′g̣tawe′, k!êsā′nî xō yux nacî′qtc. //
	\glpreamble	Chʼa yéi óox̱ ÿanax̱dulshooḵt áwé, kʼisáani x̱oo yux̱ nashíxch. //
	\gla	{} {} Chʼa yéi {} \rlap{óox̱} @ {} {}
			\rlap{ÿanax̱dulshooḵt} @ {} @ {} @ {} @ {} @ {} @ {} @ {} @ {} {} {} {}
		\rlap{áwé} @ {}
		{} kʼisáani \rlap{x̱oo} @ {} {}
		yux̱ @ \rlap{nashíxch.} @ {} @ {} @ {} @ {} @ {} //
	\glb	{} {} chʼa yéi {} ú -x̱ {}
			ÿ- n- g̱- du- d- l- \rt[²]{shuḵ} -μμL {} {} -t {}
		á -wé
		{} kʼisáani x̱oo {} {}
		yúx̱= n- d- sh- \rt[¹]{xix} -μH -ch //
	\glc	{}[\pr{PP} {}[\pr{CP} just thus {}[\pr{PP} \xx{3h} -\xx{pert} {}]
			\xx{qual}- \xx{ncnj}- \xx{mod}- \xx{4h·s}- \xx{mid}- \xx{appl}- \rt[²]{laugh}
				-\xx{var} \·\xx{sub} {}] -\xx{pnct} {}]
		\xx{foc} -\xx{mdst}
		{}[\pr{PP} yg·boys among \·\xx{loc} {}]
		out= \xx{ncnj}- \xx{mid}- \xx{pej}- \rt[¹]{fall} -\xx{var} -\xx{rep} //
	\gld	{} {} just thus {} him -at {}
			\rlap{\xx{hort}.ppl.laugh} {} {} {} {} {} {} {} \·that {} \·so {}
		\rlap{it.is} {}
		{} yg·boys among \·at {}
		out \rlap{\xx{hab}.run} {} {} {} {} {} //
	\glft	‘It was so that they would laugh at him that he would always run outside.’
		//
\endgl
\xe

\citeauthor{swanton:1909}’s translation of (\lastx) is incorrect and misleading.
The embedded clause with hortative mood (\fm{n-} + \fm{g̱-} + \fm{-μμL}) and the punctual postposition \fm{-t} forms a purpose clause with the meaning ‘in order to bring about’ \parencites[106]{story:1966}[28]{naish:1966}[427–430, 58 fn.\ 65, 431 fn.\ 52, 225 fn.\ 22, ]{leer:1991}.
Thus the protagonist habitually runs outside among the young boys in order to cause them to laugh at him.
\citeauthor{swanton:1909}’s translation “When they laughed at him and he ran outside” does not express the intentionality of the purpose clause and so obscures the agency of the protagonist.

The noun \fm{kʼisáani} [\ipa{kʼì.ˈsáː.nì}] ‘young boys’ in (\lastx) is very well known but its etymology is unclear.
Synchronically it is best analyzed as monomorphemic, but diachronically \fm{kʼisáani} is probably ends with the noun \fm{sáani} ‘medium, moderate amount of’ as in the sentence \fm{chʼa sáani áwé kwlisáa} ‘it is just a moderate amount that it is narrow’, i.e.\ ‘it is somewhat narrow’ \parencite[09/26]{leer:1973}.
This noun \fm{sáani} today is rarely used independently and is most often encountered as part of a periphrastic plural diminutive structure.
The plural suffix \fm{-xʼ} and the diminutive suffix \fm{-kʼ} cannot cooccur for many speakers, so to express a plural diminutive the base noun takes \fm{-xʼ} followed by \fm{sáani} as in \fm{tíxʼxʼi sáani x̱aat} ‘small strings of (spruce) roots’ with \fm{tíxʼ} ‘rope’ and \fm{x̱aat} ‘root’ \parencite[76.97]{dauenhauer:1987}.
The initial part of the noun \fm{kʼisáani} ‘young boys’ is more problematic.
Naively it appears to be from the relational noun \fm{kʼí} ‘base of’, but this does not make much sense for the description of a young male human.
A more plausible analysis is that it is a truncation of some other noun, perhaps for example \fm{kéekʼ} ‘younger sibling of the same sex’ or \fm{éekʼ} ‘woman’s brother’, but the exact etymon has yet to be determined.

\ex\label{ex:89-149-just-yuck-below-garbage-man}%
\exmn{259.2}%
\begingl
	\glpreamble	“Tca-ī′ q!a-ite′-cū′ye-qā.” //
	\glpreamble	«\!Chʼa ée, X̱ʼa.eetí Shuyee Ḵáa.\!» //
	\gla	«\!Chʼa ée, \rlap{X̱ʼa.eetí} @ {} \rlap{Shuyee} @ {} Ḵáa //
	\glb	\pqp{}chʼa ée x̱ʼé- eetí shú- ÿee ḵáa //
	\glc	\pqp{}just yuck mouth- remains end- below man //
	\gld	\pqp{}just yuck \rlap{garbage} {} \rlap{below} {} man //
	\glft	‘“Just yuck, Below Garbage Man.”’
		//
\endgl
\xe

This is the same utterance which appeared earlier in (\ref{ex:89-117-call-him-below-garbage-man}), but this time it occurs without a quoted speech verb.

\section{Paragraph 10}\label{sec:89-para-10}

This paragraph break does not exist in \citeauthor{swanton:1909}’s original transcription.
It has been added here because otherwise the paragraph runs on very long.
This is a relatively natural point for a break in the narrative because two new characters are introduced in (\nextx) and the protagonist has a new goal of marrying the leader’s daughter.

\ex\label{ex:89-150-town-leader-unwilling-for-daughter}%
\exmn{259.3}%
\begingl
	\glpreamble	Yū′ans!atî-si ʟēł dudjîde′ yē′qasado′ha. //
	\glpreamble	Yú aan sʼaatí sée tléil du jeedé yéi ḵaa saduhá. //
	\gla	{} Yú aan sʼaatí sée {}
		tléil {} du \rlap{jeedé} @ {} {}
		yéi @ ḵaa @ \rlap{saduhá.} @ {} @ {} @ {} @ {} //
	\glb	{} yú aan sʼaatí sée {}
		tléil {} du jee -dé {}	
		yéi= ḵaa= se- du- d- \rt[²]{haʰ} -μH //
	\glc	{}[\pr{DP} \xx{dist} town master daughter {}]
		\xx{neg} {}[\pr{PP} \xx{3h·pss} poss’n -\xx{all} {}]
		thus= \xx{4h·o}= voice- \xx{4h·s}- \xx{mid}- \rt[²]{will} -\xx{var} //
	\gld	{} that town master daughter {}
		not {} her poss’n -to {}
		thus= ppl= \rlap{\xx{stv}·\xx{impfv}.he.will} {} {} {} {} //
	\glft	‘The daughter of the town leader, he is not willing for people to go to her possession.’
		//
\endgl
\xe

The verb root \fm{\rt[²]{haʰ}} ‘will, desire, decide’ in (\lastx) is notable because it occurs in irregular stems with \fm{-eμH} stem variation where there is no suffix to trigger ablaut of the root vowel.
To illustrate this, \citeauthor{leer:1973} has a perfective form in the sentence \fm{yéi asaawahaa nag̱agoodít} ‘he willed or desired him to go’ and a state imperfective in the sentence \fm{yéi asayahéi} ‘he wills, desires that it happen’ \parencite[01/44]{leer:1973}.
The corresponding negative stem of a state imperfective is predicted to be short as in \fm{tléil yéi asahé} ‘he doesn’t will, desire it’ \parencite[01/44]{leer:1973}.
But \citeauthor{swanton:1909}’s transcription of (\lastx) has \orth{yē′qasado′ha} which implies a stem \fm{–há} not \fm{–hé}.
This could be a mistranscription, but it could instead signify that the speaker has a regularized stem paradigm for this root.
The retranscription follows \citeauthor{swanton:1909} and departs from \citeauthor{leer:1977}’s retranscription which gives \fm{yéi ḵaa saduhé} \parencite[8]{leer:1977}.

The interpretation of pronouns in (\lastx) is a bit complicated because there are three referents, none of which are discourse-local (1st or 2nd person).
The three referents are: (i) the town leader’s daughter, (ii) the town leader, and (iii) unspecified people who might marry the daughter.
There are three corresponding pronouns: (i) the third person possessive pronoun \fm{du} of \fm{du jeedé} ‘to his/her possession’, (ii) the fourth person human subject pronoun \fm{du-} ‘one, somebody; they; s/he’, and (iii) the fourth person human object pronoun \fm{ḵaa=} ‘one, somebody; they’.
The referent of the third person possessive is most plausibly the daughter because she is explicitly identified by the DP at the beginning of the sentence and so is topical and foregrounded matching the use of third person.
The other two pronouns are fourth person pronouns which can refer either to an 	indefinite/nonspecific human entity, or alternatively to a definite/specific but backgrounded human entity.
The town leader is explicitly mentioned as part of the DP \fm{yú aan sʼaatí sée}, but crucially he is the possessor rather than the primary noun of this phrase.
This implies that he has been backgrounded and therefore should be compatible with the fourth person human subject \fm{du-}.
This leaves the unspecified people who might marry the daughter to be the referent of the fourth person human object pronoun \fm{ḵaa=}.

\ex\label{ex:89-151-ppl-try-marry-got-ready}%
\exmn{259.3}%
\begingl
	\glpreamble	Łdakᴀ′t yētx ducā′q!awe tc!uʟe′ aỵîs yên ū′wani. //
	\glpreamble	Ldakát yéitx̱ dusháaxʼw áwé chʼu tle a ÿís yan uwanée. //
	\gla	{} {} Ldakát \rlap{yéitx̱} @ {} {}
			\rlap{dusháaxʼw} @ {} @ {} @ {} @ {} @ {} @ {} {}
		\rlap{áwé} @ {} +
		chʼu tle {} a ÿís {}
		yan @ \rlap{uwanée.} @ {} @ {} @ {} //
	\glb	{} {} ldakát yéi -dáx̱ {}
			{} du- d- \rt[²]{shaʷ} -μμH -xʼ {} {}
		á -wé
		chʼu tle {} a ÿís {}
		ÿán= u- i- \rt[¹]{niʰ} -μμH //
	\glc	{}[\pr{CP} {}[\pr{PP} all place -\xx{abl} {}]
			\xx{zcnj}\· \xx{4h·s}- \xx{mid}- \rt[²]{marry} -\xx{var} -\xx{rep} \·\xx{sub} {}]
		\xx{foc} -\xx{mdst}
		just then {}[\pr{PP} \xx{3n} \xx{ben} {}]
		\xx{term}= \xx{zpfv}- \xx{stv}- \rt[¹]{happen} -\xx{var} //
	\gld	{} {} all place -from {}
			\rlap{\xx{csec}.ppl.marry.\xx{rep}} {} {} {} {} {} {} {} 
		\rlap{it.is} {}
		just then {} her for {}
		done \rlap{\xx{pfv}.happen} //
	\glft	‘People having tried to marry her from everywhere, he got ready.’
		//
\endgl
\xe

The repetitive suffix \fm{-xʼ} in the consecutive aspect (\fm{∅}\pr{\xx{zcnj}} + \fm{-μμH}\pr{\xx{var}} + \fm{∅}\pr{\xx{sub}} + \fm{áwé}\pr{\xx{foc}}) verb \fm{dusháaxʼw} ‘people having tried to marry her’ has a conative (‘trying, attempting’) interpretation which is significant in this context.
Conation arises here because by default an attempt at marriage is expected to succeed, so a repetitive sequence of marriage events should only occur either because of repetitive divorces or because each event fails.
This conative interpretation for repetitives is particularly associated with achievement verbs where the event takes place instantaneously.

\ex\label{ex:89-152-dress-at-night}%
\exmn{259.4}%
\begingl
	\glpreamble	Hūtc qo′a tā′dawe ctā′de yē′djîwudîne. //
	\glpreamble	Hóoch ḵu.aa taat áwé sh daadé yéi jiwu̬dinei. //
	\gla	{} \rlap{Hóoch} @ {} {} ḵu.aa
		{} taat {} \rlap{áwé} @ {}
		{} sh \rlap{daadé} @ {} {}
		yéi @ \rlap{jiwu̬dinei.} @ {} @ {} @ {} @ {} @ {} //
	\glb	{} hú -ch {} ḵu.aa
		{} taat {} á -wé
		{} sh daa -dé {}
		yéi= ji- wu- d- i- \rt[²]{neʰ} -μμL //
	\glc	{}[\pr{DP} \xx{3h} -\xx{erg} {}] \xx{contr}
		{}[\pr{DP} night {}] \xx{foc} -\xx{mdst}
		{}[\pr{PP} \xx{rflx·pss} around -\xx{all} {}]
		thus= hand- \xx{pfv}- \xx{mid}- \xx{stv}- \rt[²]{work} -\xx{var} //
	\gld	{} \rlap{him} {} {} however
		{} night {} \rlap{it.is} {}
		{} self’s around -to {}
		thus \rlap{hand.\xx{pfv}.work} {} {} {} {} {} //
	\glft	‘Him however, it was at night that he dressed himself.’
		//
\endgl
\xe

\ex\label{ex:89-153-handle-twist-of-copper}%
\exmn{259.5}%
\begingl
	\glpreamble	Eq kᴀtî′q! aosîte′. //
	\glpreamble	Eiḵ katíx̱ʼ awsitee. //
	\gla	{} Eiḵ \rlap{katíx̱ʼ} @ {} @ {} {}
		\rlap{awsitee.} @ {} @ {} @ {} @ {} @ {} //
	\glb	{} eiḵ ka- \rt[²]{tix̱ʼ} -μH {}
		a- wu- s- i- \rt[²]{ti} -μμL //
	\glc	{}[\pr{DP} copper \xx{qual}- \rt[²]{twist} -\xx{var} {}]
		\xx{arg}- \xx{pfv}- \xx{xtn}- \xx{stv}- \rt[²]{handle} -\xx{var} //
	\gld	{} copper \rlap{twist} {} {} {}
		\rlap{3>3.\xx{pfv}.\xx{xtn}.handle} {} {} {} {} {} //
	\glft	‘He took a twist of copper.’
		//
\endgl
\xe

\label{note:89-153-twist-of-copper}
The \fm{eiḵ katíx̱ʼ} ‘twist of copper’ in (\lastx) describes a traditional copper object used for decoration and storage.
Copper would be drawn into long wires 1 or 2 millimeters thick and then twisted together in pairs to form something similar to the twisted cedar bark and wool warp of a Chilkat or Ravenstail blanket \parencites{emmons:1908}[349, 445]{de-laguna:1972}[176, 178–179]{emmons:1991}.
A copper twist might be bent into a ring and worn around the neck as a necklace, but it was also used as a form of storage that might be later broken and worked into tool points.
There are a few examples in museums of copper twists which were obtained in trade by Europeans in the 18th and 19th centuries, but in modern Tlingit culture they are nearly unknown.

\ex\label{ex:89-154-handle-twist-of-copper}%
\exmn{259.5}%
\begingl
	\glpreamble	Ātē′xỵa aosîku′ yuānỵê′dê. //
	\glpreamble	Áa teix̱ ÿé awsikóo yú aanÿádi. //
	\gla	{} {} {} \rlap{Áa} @ {} {} \rlap{teix̱} @ {} @ {} @ {} {} ÿé {}
		\rlap{awsikóo} @ {} @ {} @ {} @ {} @ {} +
		{} yú \rlap{aanÿádi.} @ {} @ {} {} //
	\glb	{} {} {} á -μ {} \rt[¹]{taʰ} -eμL -x̱ {} {} ÿé {}
		a- wu- s- i- \rt[²]{kuʰ} -μμH
		{} yú aan- ÿát -í {} //
	\glc	{}[\pr{DP} {}[\pr{CP} {}[\pr{PP} \xx{3n} -\xx{loc} {}]
			\rt[¹]{sleep·\xx{sg}} -\xx{var} -\xx{rep} \•\xx{sub} {}] place {}]
		\xx{arg}- \xx{pfv}- \xx{xtn}- \xx{stv}- \rt[²]{know} -\xx{var}
		{}[\pr{DP} \xx{dist} town- child -\xx{pss} {}] //
	\gld	{} {} {} there -at {} \rlap{\xx{impfv}.sleep·\xx{sg}.\xx{rep}} {} {} \•where {} place {}
		\rlap{3>3.\xx{pfv}.know} {} {} {} {} {}
		{} that \rlap{aristocrat} {} {} {} //
	\glft	‘He knew the place where she sleeps, that aristocrat.’
		//
\endgl
\xe

The referent of \fm{yú aanÿádi} ‘that aristocrat’ in (\lastx) is potentially ambiguous because it could refer to either the protagonist or to the daughter of the town leader.
In the narrative so far the protagonist has been established as low class, but it is plausible that the narrator here anticipated his future elevation to high class status and so deserved the label \fm{aanÿádi}.
\citeauthor{swanton:1909}’s gloss clarifies this ambiguity since “the rich man’s daughter” can refer only to the daughter of the town leader.
Since this gloss is not literal it was probably provided by a consultant familiar with the story who would know the intended meaning.

\ex\label{ex:89-155-poked-her-thru-wall}%
\exmn{259.6}%
\begingl
	\glpreamble	T!aq!ā′nᴀxawe ᴀtc yu-ᴀqłî′tsᴀqk yucā′wat yuē′q-kᴀtî′q!tc. //
	\glpreamble	Tʼáa x̱ʼáanáx̱ áwé ách yoo aklitsáḵk yú shaawaát, yú eiḵ katíx̱ʼch. //
	\gla	{} Tʼáa \rlap{x̱ʼáanáx̱} @ {} {} \rlap{áwé} @ {}
		{} \rlap{ách} @ {} {}
		yoo @ \rlap{aklitsáḵk} @ {} @ {} @ {} @ {} @ {} @ {}
		{} yú \rlap{shaawát,} @ {} {}
		{} yú eiḵ \rlap{katíx̱ʼch.} @ {} @ {} @ {} {} //
	\glb	{} tʼáa x̱'é -náx̱ {} á -wé
		{} á -ch {}
		yoo= a- k- l- i- \rt[²]{tsaḵ} -μH -k
		{} yú sháaʷ- ÿát {}
		{} yú eiḵ ka- \rt[²]{tix̱ʼ} -μH -ch {} //
	\glc	{}[\pr{PP} board mouth -\xx{perl} {}] \xx{foc} -\xx{mdst}
		{}[\pr{PP} \xx{3n} -\xx{instr} {}]
		\xx{alt}= \xx{arg}- \xx{qual}- \xx{appl}- \xx{stv}- \rt[²]{poke} -\xx{var} -\xx{rep}
		{}[\pr{DP} \xx{dist} woman- child {}]
		{}[\pr{PP} \xx{dist} copper \xx{qual}- \rt[²]{twist} -\xx{var} -\xx{instr} {}] //
	\gld	{} wall hole -thru {} \rlap{it.is} {}
		{} it -with {}
		\xx{alt} \rlap{3>3.\xx{stv}·\xx{impfv}.poke.\xx{rep}} {} {} {} {} {} {}
		{} that \rlap{girl} {} {}
		{} that copper \rlap{twist} {} {} -with {} //
	\glft	‘It was through a hole in the wall that he kept poking her with it, that girl, with that copper twist.’
		//
\endgl
\xe

The sentence in (\lastx) has a good example of postposition doubling due to dislocation.
The applicative is formed in this sentence by the combination of applicative \fm{l-} and the applicative instrumental postposition \fm{-ch} ‘by means of’.
The applied argument PP is fixed in its position immediately before the verb complex and cannot be dislocated to either the left or right periphery.
This PP’s complement DP \fm{yú eiḵ katíx̱ʼ} ‘that copper twist’ has been right dislocated out of the PP leaving the resumptive pronoun \fm{á} ‘it’ behind.
But then the right dislocated DP occurs in the right periphery with a copy of the same postposition \fm{-ch}.
The syntactic mechanisms underlying this have yet to be studied, but it is presumably similar to better known cases of adposition doubling in other languages such as English.
\FIXME{Note difference between applicative instrumental and ergative \fm{-ch} as in (\nextx).}

\ex\label{ex:89-156-girl-caught-it}%
\exmn{259.6}%
\begingl
	\glpreamble	Yucawā′ttc aołicā′t. //
	\glpreamble	Yú shaawátch wulisháat. //
	\gla	{} Yú \rlap{shaawátch} @ {} @ {} {}
		\rlap{wulisháat.} @ {} @ {} @ {} @ {} @ {} //
	\glb	{} yú sháaʷ- ÿát -ch {}
		ⱥ- wu- l- i- \rt[²]{shaʼt} -μμH //
	\glc	{}[\pr{DP} \xx{dist} woman- child -\xx{erg} {}]
		\xx{arg}- \xx{pfv}- \xx{xtn}- \xx{stv}- \rt[²]{grab} -\xx{var} //
	\gld	{} that \rlap{girl} {} {} {}
		\rlap{3>3.\xx{pfv}.catch} {} {} {} {} {} //
	\glft	‘The girl caught it.’
		//
\endgl
\xe

The sentence in (\lastx) as transcribed by \citeauthor{swanton:1909} would be \fm{yú shaawátch awlisháat}.
This is ungrammatical because the presence of \fm{-ch} immediately before the verb should trigger the disappearance of \fm{a-}.
Setting aside the possibility that this was grammatical for the speaker, there are mutually exclusive corrections: either the \fm{-ch} can be deleted or the \fm{a-} can be deleted.
Neither is really more plausible than the other, but deleting the \fm{a-} was chosen because it results in the presence of an ergative suffix explicitly identifying the subject.

\ex\label{ex:89-157-she-smelled-it}%
\exmn{259.7}%
\begingl
	\glpreamble	Aodzînî′q!. //
	\glpreamble	Awdziníxʼ. //
	\gla	\rlap{Awdziníxʼ.} @ {} @ {} @ {} @ {} @ {} @ {} //
	\glb	a- wu- d- s- i- \rt[²]{nix} -μH //
	\glc	\xx{arg}- \xx{pfv}- \xx{mid}- \xx{xtn}- \xx{stv}- \rt[²]{smell} -\xx{var} //
	\gld	\rlap{3>3.\xx{pfv}.smell} {} {} {} {} {} {} //
	\glft	‘She smelled it.’
		//
\endgl
\xe

\ex\label{ex:89-158-doesnt-know-it}%
\exmn{259.7}%
\begingl
	\glpreamble	ʟēł ag̣a′ wus-ha yuē′q. //
	\glpreamble	Tléil aag̱áa awuskú yú eiḵ. //
	\gla	Tléil {} \rlap{aag̱áa} @ {} {}
		\rlap{awuskú} @ {} @ {} @ {} @ {} @ {}
		{} yú eiḵ. {} //
	\glb	tléil {} á -g̱áa {}
		a- u- wu- s- \rt[²]{kuʰ} -μH
		{} yú eiḵ {} //
	\glc	\xx{neg} {}[\pr{PP} \xx{3n} -\xx{ades} {}]
		\xx{arg}- \xx{irr}- \xx{pfv}- \xx{xtn}- \rt[²]{know} -\xx{var}
		{}[\pr{DP} \xx{dist} copper {}] //
	\gld	not {} it -for {}
		\rlap{3>3.\xx{pfv}.know} {} {} {} {} {}
		{} that copper {} //
	\glft	‘She doesn’t know what it is for, that copper.’
		//
\endgl
\xe

\citeauthor{swanton:1909}’s transcription \orth{wus-ha} in (\lastx) suggests a verb like \fm{wus.há}.
But this makes no sense in context; none of the candidate roots \fm{\rt[¹]{haʰ}} ‘be many’, \fm{\rt[²]{haʰ}} ‘will, desire’, \fm{\rt[¹]{haʰ}} ‘wrestle’, \fm{\rt[¹]{ha}} ‘move invisibly’, \fm{\rt[²]{ha}} ‘move mass’, or \fm{\rt[¹]{ha}} ‘become located’ could denote something that the girl might do in this situation.
Furthermore, \citeauthor{swanton:1909}’s gloss of the sentence as “Not what (for it) it was [she knew] the copper” does not even approach the meaning of any root with the shape /\ipa{ha}/.
A plausible explanation is that \orth{ha} is a misreading of handwritten \orth{ku}.
Then the verb would be \fm{awuskú} ‘she doesn’t know it’ with the initial [\ipa{ʔà}] merged by liason with the preceding [\ipa{ʔàː.ˈqáː}], i.e.\ [\ipa{ʔàː.ˈqáː‿a.wùs.ˈkʰʷú}].

\ex\label{ex:89-159-nobody-seen-copper}%
\exmn{259.7}%
\begingl
	\glpreamble	ʟēł łīngî′t-ānē′q! ax dustî′ndjîayu′ ēq. //
	\glpreamble	Tléil leengít aaníxʼ áx̱ dustínji áyú eiḵ. //
	\gla	Tléil {} leengít \rlap{aaníxʼ} @ {} @ {} {}
		{} \rlap{áx̱} @ {} {}
		\rlap{dustínji} @ {} @ {} @ {} @ {} @ {} @ {} @ {} @ {}
		\rlap{áyú} @ {}
		{} eiḵ. {} //
	\glb	tléil {} leengít aan -í -xʼ {}
		{} á -x̱ {}
		u- u- du- d- s- \rt[²]{tin} -μH -ch -í
		á -yú
		{} eiḵ {} //
	\glc	\xx{neg} {}[\pr{PP} Tlingit land -\xx{pss} -\xx{loc} {}]
		{}[\pr{PP} it -\xx{pert} {}]
		\xx{irr}- \xx{zpfv}- \xx{4h·s}- \xx{mid}- \xx{xtn}- \rt[²]{see} -\xx{var} -\xx{rep} -\xx{sub}
		\xx{foc} -\xx{dist}
		{}[\pr{DP} copper {}] //
	\gld	not {} \rlap{earth} {} {} -on {}
		{} it -of {}
		\rlap{\xx{hab}.ppl.see} {} {} {} {} {} {} {} -which
		\rlap{it.is} {}
		{} copper {} //
	\glft	‘Which it was that nobody on earth had ever seen a thing of it, copper.’
		//
\endgl
\xe

\ex\label{ex:89-160-summoned-cmere-outside}%
\exmn{259.8}%
\begingl
	\glpreamble	Tc!uʟe′ ā′waxox, “Hāgu gā′nq!a.” //
	\glpreamble	Chʼu tle aawax̱oox̱ «\!Haagú, gáanxʼi\!». //
	\gla	Chʼu tle \rlap{aawax̱oox̱} @ {} @ {} @ {} @ {}
		{} \llap{«\!}\rlap{Haagú,} @ {} {} \rlap{gáanxʼi\!».} @ {} {} //
	\glb	chʼu tle a- wu- i- \rt[²]{x̱ux̱} -μμL
		{} haaⁿ= gú {} gáan -xʼ {} {} //
	\glc	just then \xx{arg}- \xx{pfv}- \xx{stv}- \rt[²]{summon} -\xx{var}
		{}[\pr{CP} \xx{cis}= go·\xx{imp} {}[\pr{PP} outside -\xx{loc} {}] {}] //
	\gld	just then \rlap{3>3.\xx{pfv}.summon} {} {} {} {}
		{} \rlap{come·here} {} {} outside -at {} //
	\glft	‘So then he summoned her “Come here, outside”.’
		//
\endgl
\xe

\ex\label{ex:89-161-went-out-to-him}%
\exmn{259.8}%
\begingl
	\glpreamble	Xᴀ′ni yux wugū′t. //
	\glpreamble	X̱áni yux̱ woogoot. //
	\gla	{} \rlap{X̱áni} @ {} {} yux̱ @ \rlap{woogoot} @ {} @ {} @ {} //
	\glb	{} x̱án -í {} yúx̱= wu- i- \rt[¹]{gut} -μμL //
	\glc	{}[\pr{PP} near -\xx{loc} {}] out= \xx{pfv}- \xx{stv}- \rt[¹]{go·\xx{sg}} -\xx{var} //
	\gld	{} near -at {} out \rlap{\xx{pfv}.go·\xx{sg}} {} {} {} //
	\glft	‘She went out to him.’
		//
\endgl
\xe

The sentence in (\lastx) has an unusual preverb \fm{x̱áni} ‘near to’.
\citeauthor{leer:1977} added what he felt was a missing \fm{du} to give \fm{du x̱áni} \parencite[8]{leer:1977}, but there is no sign of this in \citeauthor{swanton:1909}’s transcription and its necessity is dubious.
The use of the locative postposition allomorph \fm{-í} is generally limited to preverbs and is only found with a small number of particular nouns like \fm{gáan} ‘outside’ and \fm{éeḵ} ‘beach’.
The form \fm{x̱áni} here is plausibly grammaticalized as a postpositional phrase preverb and has lost its internal noun phrase structure so that \fm{x̱án} is no longer a regular inalienable noun and so no longer requires a possessor.
Contrast this with (\ref{ex:89-163-youre-gonna-be-with-me}) where the phrase occurs with a possessor as \fm{ax̱ x̱áni} ‘by me’.

\ex\label{ex:89-162-come-home-with-me}%
\exmn{259.9}%
\begingl
	\glpreamble	“ᴀxhî′tîỵīdê′ xā′naᴀde. //
	\glpreamble	«\!Ax̱ hídi ÿeedé x̱áan na.á dé. //
	\gla	{} \llap{«\!}Ax̱ \rlap{hídi} {} \rlap{ÿeedé} @ {} {}
		{} \rlap{x̱áan} @ {} {} 
		\rlap{na.á} @ {} @ {} @ {} dé. //
	\glb	{} ax̱ hít -í ÿee -dé {}
		{} x̱á -n {}
		n- {} \rt[¹]{.at} -⊗ dé //
	\glc	{}[\pr{PP} \xx{1sg·pss} house -\xx{pss} below -\xx{all} {}]
		{}[\pr{PP} \xx{1sg} -\xx{instr} {}]
		\xx{ncnj}- \xx{2sg·s}\· \rt[¹]{go·\xx{pl}} -\xx{var}
		now //
	\gld	{} my \rlap{house} {} below -to {}
		{} me -with {}
		\rlap{\xx{imp}.you·\xx{sg}.go·\xx{pl}} {} {} {} now //
	\glft	‘“Come with me now to my house.’
		//
\endgl
\xe

The verb in (\lastx) is remarkable because it combines the root \fm{\rt[¹]{.at}} ‘plural go’ with the second person singular subject.
This presumably arises because the speaker – the protagonist – includes himself in the event of going but the target of the imperative is only the addressee – the daughter of the town leader – not including the speaker.
We know that the subject must be the second person singular because the only other possible subject for an imperative is the second person plural.
The second person plural is always an overt morpheme \fm{ÿi-}.
The second person singular in contrast is covert by default in imperatives and appears as overt \fm{i-} only when the \fm{d-} voice prefix is present.
Since there is no overt subject in (\lastx) the subject can only be the second person singular.
A similar pattern occurs in (\ref{ex:89-167-man-people-teased-with-him-to-beach}) where the subject is again the girl and the protagonist is referred to in a PP but the verb is still plural.

\ex\label{ex:89-163-youre-gonna-be-with-me}%
\exmn{259.9}%
\begingl
	\glpreamble	Axaniỵe′ îq-gwâte′” //
	\glpreamble	Ax̱ x̱áni ÿéi ikg̱watée\!» //
	\gla	{} Ax̱ \rlap{x̱áni} @ {} {}
			ÿéi @ \rlap{ikg̱watée\!»} @ {} @ {} @ {} @ {} @ {} //
	\glb	{} ax̱ x̱án -í {}
			ÿéi= i- w- g- g̱- \rt[¹]{tiʰ} -μμH //
	\glc	{}[\pr{PP} \xx{1sg·pss} near -\xx{loc} {}]
			thus= \xx{2sg·o}- \xx{irr}- \xx{gcnj}- \xx{mod}- \rt[¹]{be} -\xx{var} //
	\gld	{} my near -at {}
			thus \rlap{you·\xx{sg}.\xx{prsp}.be} {} {} {} {} {} //
	\glft	‘You will be with me”’
		//
\endgl
\xe

\ex\label{ex:89-164-he said}%
\exmn{259.9}%
\begingl
	\glpreamble	yū′aỵaosîqa. //
	\glpreamble	yóo aÿawsiḵaa. //
	\gla	yóo @ \rlap{aÿawsiḵaa.} @ {} @ {} @ {} @ {} @ {} @ {} //
	\glb	yóo= a- ÿ- wu- s- i- \rt[¹]{ḵa} -μμL //
	\glc	\xx{quot}= \xx{arg}- \xx{qual}- \xx{pfv}- \xx{csv}- \xx{stv}- \rt[¹]{say} -\xx{var} //
	\gld	\xx{quot} \rlap{3>3.\xx{pfv}.say} {} {} {} {} {} {} //
	\glft	‘he said to her.’
		//
\endgl
\xe


\ex\label{ex:89-165-she-thought-man-from-somewhere}%
\exmn{259.10}%
\begingl
	\glpreamble	Gudᴀxqā′x sayu′ ū′wadjî //
	\glpreamble	Goodáx̱ ḵáax̱ sáyú oowajée; //
	\gla	{} {} \rlap{Goodáx̱} @ {} {} \rlap{ḵáax̱} @ {} {} \rlap{sáyú} @ {} @ {} 
		\rlap{oowajée;} @ {} @ {} @ {} @ {} //
	\glb	{} {} góo -dáx̱ {} ḵáa -x̱ {} s= á -yú
		a- u- i- \rt[²]{jiʰ} -μμH //
	\glc	{}[\pr{PP} {}[\pr{PP} where -\xx{abl} {}] man -\xx{pert} {}] \xx{q}= \xx{foc} -\xx{dist}
		\xx{arg}- \xx{irr}- \xx{stv}- \rt[²]{think} -\xx{var} //
	\gld	{} {} where -from {} man -of {} some- \rlap{it.is} {}
		\rlap{3>3.\xx{stv·impfv}.think} {} {} {} {} //
	\glft	‘She thought that he was a man from somewhere;’
		//
\endgl
\xe

\ex\label{ex:89-166-she-didnt-know}%
\exmn{259.10}%
\begingl
	\glpreamble	ʟēł ye′awusku. //
	\glpreamble	tléil yéi awuskú. //
	\gla	tléil yéi @ \rlap{awuskú.} @ {} @ {} @ {} @ {} @ {} //
	\glb	tléil yéi= a- u- wu- s- \rt[²]{ku} -μH //
	\glc	\xx{neg} thus= \xx{arg}- \xx{irr}- \xx{pfv}- \xx{xtn}- \rt[²]{know} -\xx{var} //
	\gld	not thus \rlap{3>3.\xx{pfv}.know} {} {} {} {} {} //
	\glft	‘she didn’t know him.’
		//
\endgl
\xe

\ex\label{ex:89-167-man-people-teased-with-him-to-beach}%
\exmn{259.10}%
\begingl
	\glpreamble	Yū′duīqonī′k qāx sateỵî, ts!ᴀs yuî′qtê ayu′ ᴀcī′n ỵā′naᴀt. //
	\glpreamble	Yú du éex̱ dunéekw ḵáax̱ sateeyí, tsʼas yú éeḵde áyú ash een ÿaa na.át. //
	\gla	{} {} Yú {} {} du \rlap{éex̱} @ {} {}
					\rlap{dunéekw} @ {} @ {} @ {} @ {} {} 
				\rlap{ḵáax̱} @ {} {} +
			\rlap{sateeyí} @ {} @ {} @ {} {}
		\rlap{tsʼas} @ {} {} yú \rlap{éeḵde} @ {} {} \rlap{áyú} @ {} +
		{} ash \rlap{een} @ {} {}
		ÿaa @ \rlap{na.át.} @ {} @ {} //
	\glb	{} {} yú {} {} du ee -x̱ {} 
					du- d- \rt[¹]{niʼkw} -μμH {} {}
				ḵáaʷ -x̱ {}
			s- \rt[¹]{tiʰ} -μμL -í {}
		tsʼa =s {} yú éeḵ -dé {} á -yú
		{} ash ee -n {}
		ÿaa= n- \rt[¹]{.at} -μH //
	\glc	{}[\pr{CP} {}[\pr{PP} \xx{dist} {}[\pr{CP} {}[\pr{PP} \xx{3h} \xx{base} -\xx{pert} {}]
					\xx{4h·s}- \xx{mid}- \rt[¹]{bother} -\xx{var} \·\xx{rel} {}]
				man -\xx{pert} {}]
			\xx{appl}- \rt[¹]{be} -\xx{var} -\xx{sub} {}]
		just =\xx{dub} {}[\pr{PP} \xx{dist} beach -\xx{all} {}] \xx{foc} -\xx{dist}
		{}[\pr{PP} \xx{3prx} \xx{base} -\xx{instr} {}]
		along= \xx{ncnj}- \rt[¹]{go·\xx{pl}} -\xx{var} //
	\gld	{} {} that {} {} him {} -at {}
					\rlap{\xx{impfv}.ppl.tease} {} {} {} {} {}
				man -of {}
			\rlap{\xx{stv·impfv}.be} {} {} -ing {}
		just =apparently {} that beach -to {} \rlap{it.is} {}
		{} him {} -with {}
		along \rlap{\xx{prog}.go·\xx{pl}} {} {} //
	\glft	‘Being that man who people teased, it was to the beach that she was going along with him.’
		//
\endgl
\xe

\ex\label{ex:89-168-it-opened-before-her-copper-door}%
\exmn{259.11}%
\begingl
	\glpreamble	Tc!a dudjî′ cukᴀdawe′ nēł cū′djîxîn, yuē′q q!axā′t //
	\glpreamble	Chʼa du jishukát áwé neil shuwjix̱ín, yú eiḵ x̱ʼaháat. //
	\gla	Chʼu {} du \rlap{jishukát} @ {} @ {} @ {} {} \rlap{áwé} @ {}
		neil @ \rlap{shuwjix̱ín,} @ {} @ {} @ {} @ {} @ {} @ {} +
		{} yú eiḵ \rlap{x̱ʼaháat.} @ {} {} //
	\glb	chʼu {} du jín- shú- ká -t {} á -wé
		neil= shu- u- d- sh- i- \rt[²]{x̱in} -μH
		{} yú eiḵ x̱ʼé- háat {} //
	\glc	just {}[\pr{PP} \xx{3h·pss} hand- end- \xx{hsfc} -\xx{pnct} {}] \xx{foc} -\xx{mdst}
		inside= end- \xx{zpfv}- \xx{mid}- \xx{pej}- \xx{stv}- \rt[¹]{fall} -\xx{var}
		{}[\pr{DP} \xx{dist} copper mouth- cover {}] //
	\gld	just {} her hand- end- atop -to {} \rlap{it.is} {}
		inside \rlap{end.\xx{pfv}.open} {} {} {} {} {} {}
		{} that copper \rlap{door} {} {} //
	\glft	‘It is just as it is in front of her hand that it opens inside, that copper door;’
		//
\endgl
\xe

\ex\label{ex:89-169-shone-on-her-face}%
\exmn{259.11}%
\begingl
	\glpreamble	duyê′t kaodîgᴀ′naỵî′. //
	\glpreamble	Du yát kawdigán a ÿee. //
	\gla	{} Du \rlap{yát} @ {} {}
		\rlap{kawdigán} @ {} @ {} @ {} @ {} @ {}
		{} a \rlap{ÿee.} @ {} {} //
	\glb	{} du ÿá -t {}
		k- wu- d- i- \rt[¹]{gan} -μH
		{} a ÿee {} {} //
	\glc	{}[\pr{PP} \xx{3h·pss} face -\xx{pnct} {}]
		\xx{hsfc}- \xx{pfv}- \xx{mid}- \xx{stv}- \rt[¹]{burn} -\xx{var}
		{}[\pr{PP} \xx{3n·pss} below \·\xx{loc} {}] //
	\gld	{} her face -on {}
		\rlap{\xx{pfv}.shine} {} {} {} {} {}
		{} its below -at {} //
	\glft	‘it shone on her face inside.’
		//
\endgl
\xe

\ex\label{ex:89-170-from-somewhere-lugged-inside-tinaa}%
\exmn{259.12}%
\begingl
	\glpreamble	Tc!uʟe′ gutxᴀ′tsayu ʟe nēłỵī′ caỵaqā′wadjᴀł yū′tînna. //
	\glpreamble	Chʼu tle gootx̱ át sáyú tle neilÿee ashaÿakaawajél yú tináa. //
	\gla	Chʼu tle {} {} \rlap{gootx̱} @ {} {} át {} \rlap{sáyú} @ {} @ {}
		tle {} \rlap{neilÿee} @ {} @ {} {}
		\rlap{ashaÿakaawajél} @ {} @ {} @ {} @ {} @ {} @ {} @ {}
		{} yú tináa. {} //
	\glb	chʼu tle {} {} goo -dáx̱ {} át {} s= á -yú
		tle {} neil- ÿee {} {}
		a- sha- ÿ- k- wu- i- \rt[²]{jel} -μH
		{} yú tináa {} //
	\glc	just then {}[\pr{DP} {}[\pr{PP} where -\xx{abl} {}] thing {}] \xx{q}= \xx{foc} -\xx{dist}
		then {}[\pr{PP} inside- below \·\xx{loc} {}]
		\xx{arg}- head- face- \xx{sro}- \xx{pfv}- \xx{stv}- \rt[²]{lug} -\xx{var}
		{}[\pr{DP} \xx{dist} copper {}] //
	\gld	just then {} {} where -from {} thing {} ever= \xx{it.is} {}
		then {} inside- below -at {}
		\rlap{3>3.head.face.round.\xx{pfv}.lug} {} {} {} {} {} {} {}
		{} that copper {} //
	\glft	‘It was things from wherever that he had lugged indoors, those \fm{tináa}.’
		//
\endgl
\xe

\ex\label{ex:89-171-he-married-her}%
\exmn{259.13}%
\begingl
	\glpreamble	Tc!uʟe′ ā′waca duhî′tîq!. //
	\glpreamble	Chʼu tle aawasháa du hídixʼ. //
	\gla	Chʼu tle \rlap{aawasháa} @ {} @ {} @ {} @ {}
		{} du \rlap{hídixʼ.} @ {} @ {} {} //
	\glb	chʼu tle a- wu- i- \rt[²]{shaʷ} -μμH
		{} du hít -í -xʼ {} //
	\glc	just then \xx{arg}- \xx{pfv}- \xx{stv}- \rt[²]{woman} -\xx{var}
		{}[\pr{PP} \xx{3h·pss} house -\xx{pss} -\xx{loc} {}] //
	\gld	just then \rlap{3>3.\xx{pfv}.marry} {} {} {} {}
		{} his house {} -at {} //
	\glft	‘So then he married her in his house.’
		//
\endgl
\xe

\section{Paragraph 11}\label{sec:89-para-11}

\ex\label{ex:89-172-ppl-search-for-her}%
\exmn{260.1}%
\begingl
	\glpreamble	Du-īg̣ā′ qodicī′ yū′cawᴀt. //
	\glpreamble	Du eeg̱áa ḵudushee yú shaawát. //
	\gla	{} Du \rlap{eeg̱áa} @ {} {}
		\rlap{ḵudushee} @ {} @ {} @ {}
		{} yú \rlap{shaawát.} @ {} {} //
	\glb	{} du ee -g̱áa {}
		ḵu- du- \rt[²]{shiʰ} -μμL
		{} yú sháaʷ- ÿát {} //
	\glc	{}[\pr{PP} \xx{3h} \xx{base} -\xx{ades} {}]
		\xx{areal}- \xx{4h·s}- \rt[²]{reach·for} -\xx{var}
		{}[\pr{DP} \xx{dist} woman- child {}] //
	\gld	{} her {} -for {}
		\rlap{\xx{impfv}.ppl.search} {} {} {}
		{} that \rlap{girl} {} {} //
	\glft	‘People search for her, that girl.’
		//
\endgl
\xe

\ex\label{ex:89-173-ppl-missed-her-some-days}%
\exmn{260.1}%
\begingl
	\glpreamble	Wudū′dziha k!ū′nỵagīỵî.
Kᴀnaxsa′ //
	\glpreamble	Wududzihaa, xʼoon ÿagiÿee kaanáx̱ sá. //
	\gla	\rlap{Wududzihaa,} @ {} @ {} @ {} @ {} @ {} @ {}
		{} {} xʼoon ÿagiÿee \rlap{kaanáx̱} @ {} {} sá {} //
	\glb	wu- du- d- s- i- \rt[¹]{haʰ} -μμL
		{} {} xʼoon ÿagiÿee ká -náx̱ {} sá {} //
	\glc	\xx{pfv}- \xx{4h·s}- \xx{mid}- \xx{csv}- \xx{stv}- \rt[¹]{mv·invis} -\xx{var}
		{}[\pr{QP} {}[\pr{PP} how·many day \xx{hsfc} -\xx{perl} {}] \xx{q} {}] //
	\gld	\rlap{\xx{pfv}.ppl.miss} {} {} {} {} {} {}
		{} {} how·many day \rlap{across} {} {} ever {} //
	\glft	‘People missed her, for however many days.’
		//
\endgl
\xe

\ex\label{ex:89-174-two-overnight-place-search-for-her}%
\exmn{260.2}%
\begingl
	\glpreamble	dēx oxe′ ᴀg̣a′ ug̣a′qoduciỵa′. //
	\glpreamble	Déix̱ ux̱ei aag̱áa, óog̱aa ḵudushee ÿé. //
	\gla	Déix̱ \rlap{ux̱ei} @ {} @ {}
		{} \rlap{aag̱áa,} @ {} {} +
		{} {} {} \rlap{óog̱aa} @ {} {}
			\rlap{ḵudushee} @ {} @ {} @ {} @ {} @ {} {} ÿé. {} //
	\glb	déix̱ u- \rt[¹]{x̱e} -μμL
		{} á -g̱áa {}
		{} {} {} ú -g̱áa {}
			ḵu- du- {} \rt[²]{shiʰ} -μμL {} {} ÿé {} //
	\glc	two \xx{irr}- \rt[¹]{overnight} -\xx{var}
		{}[\pr{PP} \xx{3n} -\xx{ades} {}]
		{}[\pr{DP} {}[\pr{CP} {}[\pr{PP} \xx{3h} -\xx{ades} {}]
			\xx{areal}- \xx{4h·s}- \xx{mid}- \rt[¹]{reach·for} -\xx{var} \·\xx{rel} {}] place {}] //
	\gld	two \rlap{\xx{impfv}.overnight} {} {}
		{} there -near {}
		{} {} {} her -for {}
			\rlap{\xx{impfv}.search} {} {} {} {} \·where {} place {} //
	\glft	‘They overnight twice around there, the place where people search for her.’
		//
\endgl
\xe

\citeauthor{swanton:1909} broke the two sentences in (\ref{ex:89-173-ppl-missed-her-some-days}) and (\ref{ex:89-174-two-overnight-place-search-for-her}) incorrectly.
The phrases he transcribes as \orth{k!ū′nỵagīỵî} and \orth{Kᴀnaxsa′} are in fact all one single wh-question phrase (QP) containing a single PP headed by the postposition \fm{-náx̱}.
The noun \fm{ká} in \fm{kaanáx̱} is inalienable and so must have a possessor, in this case \fm{ÿagiÿee} ‘day’.
\citeauthor{leer:1977} analyzed the phrase \fm{xʼoon ÿagiÿee kaanáx̱ sá} as the start of (\ref{ex:89-174-two-overnight-place-search-for-her}) but this conflicts with the \fm{déix̱} ‘two’ because the former phrase describes an unspecified period of time and the latter a specific period of time.

\ex\label{ex:89-175-fathers-slave-says}%
\exmn{260.2}%
\begingl
	\glpreamble	Wānanī′sawe duī′c gux, ye′ aỵaosîqa, //
	\glpreamble	Wáa nanée sáwé du éesh goox̱ yéi aÿawsiḵaa //
	\gla	{} Wáa \rlap{nanée} @ {} @ {} @ {} {} \rlap{sáwé} @ {} @ {}
		{} du éesh {}
		{} goox̱ {}
		yéi @ \rlap{aÿawsiḵaa} @ {} @ {} @ {} @ {} @ {} @ {} //
	\glb	{} wáa n- \rt[¹]{niʰ} -μμH {} {} s- á -wé
		{} du éesh {}
		{} goox̱ {}
		yéi= a- ÿ- wu- s- i- \rt[¹]{ḵa} -μμL //
	\glc	{}[\pr{CP} how \xx{ncnj}- \rt[¹]{happen} -\xx{var} \·\xx{sub} {}] \xx{q}- \xx{foc} -\xx{mdst}
		{}[\pr{DP} \xx{3h·pss} father {}]
		{}[\pr{DP} slave {}]
		thus= \xx{arg}- \xx{qual}- \xx{pfv}- \xx{csv}- \rt[¹]{say} -\xx{var} //
	\gld	{} how \rlap{\xx{csec}.happen} {} {} \·while {} ever- \rlap{it.is} {} 
		{} her father {}
		{} slave {}
		thus \rlap{3>3.\xx{pfv}.say} {} {} {} {} {} {} //
	\glft	‘At some point her father said to a slave’
		//
\endgl
\xe

The comma in \citeauthor{swanton:1909}’s transcription of (\lastx) suggests that the speaker may have paused between \fm{goox̱} ‘a slave’ and the verb.
This pause would imply that the subject \fm{du éesh} ‘her father’ and the object \fm{goox̱} ‘a slave’ are both topics.
Taking this into account, a close rendering of the whole sentence in English could be ‘It was at some point that, her father, a slave, to him he said’.
The simpler translation ignoring the topicalization is given merely because it shorter and easier.

\ex\label{ex:89-176-search-around-the-beach}%
\exmn{260.3}%
\begingl
	\glpreamble	“K!ē g̣êna′t qe′cî.” //
	\glpreamble	«\!Kʼé ig̱inaat ḵeeshee.\!» //
	\gla	\rlap{«\!Kʼé} @ {}
		{} \rlap{ig̱inaat} @ {} @ {} {}
		\rlap{ḵeeshee.\!»} @ {} @ {} @ {} //
	\glb	\pqp{}\rt[¹]{kʼe} -μH
		{} eiḵ- niÿaa -t {}
		ḵu- i- \rt[²]{shiʰ} -μμL //
	\glc	\pqp{}\rt[¹]{good} -\xx{var}
		{}[\pr{PP} beach- dir’n -\xx{pnct} {}]
		\xx{areal}- \xx{2sg·s}- \rt[²]{reach·for} -\xx{var} //
	\gld	\pqp{}\rlap{good·\xx{imp}} {}
		{} beach- dir’n -around {}
		\rlap{\xx{impfv}.you·\xx{sg}.search} {} {} {} {} //
	\glft	‘“Best that you search around the beach.”’
		//
\endgl
\xe

\ex\label{ex:89-177-searched-there-near}%
\exmn{260.3}%
\begingl
	\glpreamble	ᴀt kū′wacî yū′gux doxᴀ′nt. //
	\glpreamble	Át ḵoowashée yú goox̱ du x̱ánt. //
	\gla	{} \rlap{Át} @ {} {}
		\rlap{ḵoowashée} @ {} @ {} @ {} @ {}
		{} yú goox̱ {}
		{} du \rlap{x̱ánt.} @ {} {} //
	\glb	{} á -t {}
		ḵu- wu- i- \rt[²]{shiʰ} -μμH
		{} yú goox̱ {}
		{} du x̱án -t {} //
	\glc	{} \xx{3n} -\xx{pnct} {}
		\xx{areal}- \xx{pfv}- \xx{stv}- \rt[²]{reach·for} -\xx{var}
		{}[\pr{DP} \xx{dist} slave {}]
		{}[\pr{PP} \xx{3h·pss} near -\xx{pnct} {}]  //
	\gld	{} there -at {}
		\rlap{\xx{pfv}.search} {} {} {} {}
		{} that slave {}
		{} her near -at {} //
	\glft	‘He searched there, that slave, near her.’
		//
\endgl
\xe

\clearpage
\ex\label{ex:89-178-face-in-squat-back}%
\exmn{260.3}%
\begingl
	\glpreamble	Tc!uʟe′ ā′nēł ỵawusaye′awe yū′gux g̣ā′nî qo′xodiîqᴀq. //
	\glpreamble	Chʼu tle áa neil ÿawus.aayí áwé yú goox̱ gáani ḵux̱ wujiḵáḵ. //
	\gla	{} Chʼu tle 
			{} \rlap{áa} @ {} {}
			{} \rlap{neil} @ {} {}
			\rlap{ÿawus.aayí} @ {} @ {} @ {} @ {} @ {} @ {} {} 
		\rlap{áwé} @ {}
		{} yú goox̱ {}
		ḵux̱ @ \rlap{wujiḵáḵ.} @ {} @ {} @ {} @ {} @ {} //
	\glb	{} chʼu tle
			{} á -μ {}
			{} neil -t {}
			ÿ- wu- d- s- \rt[¹]{.a} -μμL -í {}
		á -wé
		{} yú goox̱ {}
		ḵúx̱= wu- d- sh- i- \rt[¹]{ḵaḵ} -μH //
	\glc	{}[\pr{CP} just then
			{}[\pr{PP} there -\xx{loc} {}]
			{}[\pr{PP} inside -\xx{pnct} {}]
			face- \xx{pfv}- \xx{mid}- \xx{csv}- \rt[¹]{end·mv} -\xx{var} -\xx{sub} {}]
		\xx{foc} -\xx{mdst}
		{}[\pr{DP} \xx{dist} slave {}]
		back= \xx{pfv}- \xx{mid}- \xx{csv}- \xx{stv}- \rt[¹]{squat} -\xx{var} //
	\gld	{} just then
			{} there -at {}
			{} inside -to {}
			\rlap{face.\xx{pfv}.move} {} {} {} {} {} -when {}
		\rlap{it.is} {}
		{} that slave {}
		back\• \rlap{\xx{pfv}.squat} {} {} {} {} {} //
	\glft	‘It was just as he stuck his face inside there that the slave fell back.’
		//
\endgl
\xe

The translation of (\lastx) is somewhat loose because it is difficult to precisely render the meaning of \fm{ḵux̱ wujiḵáḵ} in any succinct manner in English.
The verb root \fm{\rt[¹]{ḵaḵ}} describes squatting as in \fm{du káa x̱wajiḵaaḵ} ‘I squatted on him’ \parencite[193.2678]{story-naish:1973} as well as birds landing as in \fm{gáaxw át wujiḵáḵ} ‘the duck landed there’ \parencite[121.1595]{story-naish:1973}.
The phrase \fm{yú goox̱ ḵux̱ wujiḵáḵ} thus describes the slave falling backward on his haunches into a squatting or seated position.
This could be explicitly indicated with an additional PP in translation like ‘he fell back into a squat’ or ‘he fell back on his haunches’, but no such PP is present in Tlingit.
Another possibility is the very literal ‘he squatted back’ but this does not convey the implicit loss of control due to surprise.

\ex\label{ex:89-179-shone-on-his-face}%
\exmn{260.4}%
\begingl
	\glpreamble	Duỵê′t ka′odigᴀn. //
	\glpreamble	Du ÿát kawdigán. //
	\gla	{} Du \rlap{ÿát} @ {} {}
		\rlap{kawdigán.} @ {} @ {} @ {} @ {} @ {} //
	\glb	{} du ÿá -t {}
		k- wu- d- i- \rt[¹]{gan} -μH //
	\glc	{}[\pr{PP} \xx{3h·pss} face -\xx{pnct} {}]
		\xx{hsfc}- \xx{pfv}- \xx{mid}- \xx{stv}- \rt[¹]{burn} -\xx{var} //
	\gld	{} his face -on {}
		\rlap{\xx{pfv}.shine} {} {} {} {} {} //
	\glft	‘It shone on his face.’
		//
\endgl
\xe

\ex\label{ex:89-180-come-inside-that-girls-husband}%
\exmn{260.4}%
\begingl
	\glpreamble	Yuhî′t ỵī′dᴀx, “Nēł gu′” yū′aỵaosîqa yū′cāwᴀt xoxtc. //
	\glpreamble	Yú hít ÿeedáx̱, «\!Neil gú\!» yóo aÿawsiḵaa yú shaawát x̱úx̱ch. //
	\gla	{} Yú hít \rlap{ÿeedáx̱,} @ {} {}
		{} {} \llap{«\!}\rlap{Neil} @ {} {} \rlap{gú\!»} @ {} @ {} @ {} {}
		yóo @ \rlap{aÿawsiḵaa} @ {} @ {} @ {} @ {} @ {} @ {}
		{} yú \rlap{shaawát} @ {} \rlap{x̱úx̱ch.} @ {} {} //
	\glb	{} yú hít ÿee -dáx̱ {}
		{} {} neil -t {} {} {} \rt[¹]{gut} -⊗ {}
		yóo= a- ÿ- wu- s- i- \rt[¹]{ḵa} -μμL
		{} yú sháaʷ- ÿát x̱úx̱ -ch {} //
	\glc	{}[\pr{PP} \xx{dist} house below -\xx{abl} {}]
		{}[\pr{CP} {}[\pr{PP} inside -\xx{pnct} {}] \xx{zcnj}\· \xx{2sg·s}- \rt[¹]{go·\xx{sg}} -\xx{var} {}]
		\xx{quot}= \xx{arg}- \xx{qual}- \xx{pfv}- \xx{csv}- \xx{stv}- \rt[¹]{say} -\xx{var}
		{}[\pr{DP} \xx{dist} woman- child husband -\xx{erg} {}] //
	\gld	{} that house below -from {}
		{} {} inside -to {} \rlap{\xx{imp}.you·\xx{sg}.go·\xx{sg}} {} {} {} {}
		\xx{quot}\• \rlap{3>3.\xx{pfv}.say} {} {} {} {} {} {}
		{} that \rlap{girl’s} {} \rlap{husband} {} {} //
	\glft	‘From within that house, “Come inside” he said to him, that girl’s husband.’
		//
\endgl
\xe

\ex\label{ex:89-181-dont-tell-about-my-house}%
\exmn{260.5}%
\begingl
	\glpreamble	“Łîł kīnigī′q ya ᴀxhî′tî” ʟe yū′aỵaosîqa. //
	\glpreamble	«\!Líl keeneegéeḵ yá ax̱ hídi\!» tle yéi aÿawsiḵaa. //
	\gla	{} \llap{«\!}Líl \rlap{keeneegéeḵ} @ {} @ {} @ {} @ {} @ {}
			{} yá ax̱ \rlap{hídi\!»} @ {} {} {} +
		tle yéi @ \rlap{aÿawsiḵaa.} @ {} @ {} @ {} @ {} @ {} @ {} //
	\glb	{} líl k- u- i- \rt[²]{nik} -μμL -ḵ
			{} yá ax̱ hít -í {} {}
		tle yéi= a- ÿ- wu- s- i- \rt[¹]{ḵa} -μμL //
	\glc	{}[\pr{CP} \xx{phib} \xx{qual}- \xx{irr}- \xx{2sg·s}- \rt[²]{tell} -\xx{var} -\xx{phib}
			{}[\pr{DP} \xx{prox} \xx{1sg·pss} house -\xx{pss} {}] {}]
		then thus= \xx{arg}- \xx{qual}- \xx{csv}- \xx{stv}- \rt[¹]{say} -\xx{var} //
	\gld	{} don’t \rlap{\xx{impfv}.you·\xx{sg}.tell·of} {} {} {} {} {}
			{} this my house {} {} {}
		then thus= \rlap{3>3.\xx{pfv}.say} {} {} {} {} {} {}  //
	\glft	‘“Don’t tell about my house” he said to him then.’
		//
\endgl
\xe  

\ex\label{ex:89-182-but-he-said-to-him}%
\exmn{260.6}%
\begingl
	\glpreamble	Yē qo′a yên aỵa′osîqa, //
	\glpreamble	Yéi ḵu.aa yan aÿawsiḵáa, //
	\gla	Yéi ḵu.aa yan @ \rlap{aÿawsiḵáa,} @ {} @ {} @ {} @ {} @ {} @ {} //
	\glb	yéi= ḵu.aa ÿán= a- ÿa- wu- s- i- \rt[¹]{ḵa} -μμH //
	\glc	thus= \xx{contr} \xx{term}= \xx{arg}- \xx{qual}- \xx{pfv}- \xx{csv}- \xx{stv}- \rt[¹]{say} -\xx{var} //
	\gld	thus however done\• \rlap{3>3.\xx{pfv}.say} {} {} {} {} {} {} //
	\glft	‘He finished saying to him however,’
		//
\endgl
\xe

\ex\label{ex:89-183-report-garbage-man-married-her}%
\exmn{260.6}%
\begingl
	\glpreamble	“Q!a-ī′tîcuye-qātc uwaca′” yuq!wᴀ′nskāniłnîk. //
	\glpreamble	«\!‹\!X̱ʼa.eetí Shuyee Ḵáach uwasháa\!› yóo xʼwán sh kaneelneek.\!» //
	\gla	{} {} \llap{«\!‹\!}\rlap{X̱ʼa.eetí} @ {} \rlap{Shuyee} @ {} \rlap{Ḵáach} @ {} {}
			\rlap{uwasháa\!›} @ {} @ {} @ {} @ {} {} +
		yóo xʼwán sh @ \rlap{kaneelneek.\!»} @ {} @ {} @ {} @ {} @ {} @ {} //
	\glb	{} {} x̱ʼé- eetí shú- ÿee ḵáa -ch {}
			ⱥ- u- i- \rt[²]{shaʷ} -μμH {}
		yóo= xʼwán sh= k- n- i- d- l- \rt[²]{nik} -μμL //
	\glc	{}[\pr{CP} {}[\pr{DP} mouth- remains end- below man -\xx{erg} {}]
			\xx{arg}- \xx{zpfv}- \xx{stv}- \rt[²]{woman} -\xx{var} {}]
		\xx{quot}= \xx{imp} \xx{rflx·o}= \xx{qual}- \xx{ncnj}- \xx{2sg·s}- \xx{mid}- \xx{xtn}-
			\rt[²]{tell} -\xx{var} //
	\gld	{} {} \rlap{garbage} {} \rlap{below} {} man {} {}
			\rlap{3>3.\xx{pfv}.marry} {} {} {} {} {}
		\xx{quot}= \xx{imp} self= \rlap{\xx{imp}.report} {} {} {} {} {} {} //
	\glft	‘“Report that ‘Below Garbage Man married her’.”’
		//
\endgl
\xe

\ex\label{ex:89-184-when-inside-told-about-it}%
\exmn{260.7}%
\begingl
	\glpreamble	Lᴀ nēł wugudī′awe aka′wanêk. //
	\glpreamble	Tle neil wugoodí áwé akaawaneek. //
	\gla	{} Tle {} \rlap{neil} @ {} {}
			\rlap{wugoodí} @ {} @ {} @ {} {}
		\rlap{áwé} @ {}
		\rlap{akaawaneek.} @ {} @ {} @ {} @ {} @ {} //
	\glb	{} tle {} neil -t {}
			wu- \rt[¹]{gut} -μμL -í {}
		á -wé
		a- k- wu- i- \rt[²]{nik} -μμL //
	\glc	{}[\pr{CP} then {}[\pr{PP} inside -\xx{pnct} {}]
			\xx{pfv}- \rt[¹]{go·\xx{sg}} -\xx{var} -\xx{sub} {}]
		\xx{foc} -\xx{mdst}
		\xx{arg}- \xx{qual}- \xx{pfv}- \xx{stv}- \rt[²]{tell} -\xx{var} //
	\gld	{} then {} inside -to {}
			\rlap{\xx{pfv}.go·\xx{sg}} {} {} -when {}
		\rlap{it.is} {}
		\rlap{3>3.\xx{pfv}.tell·about} {} {} {} {} {} //
	\glft	‘Then it is when he came inside that he told about it.’
		//
\endgl
\xe

\ex\label{ex:89-185-reporting-garbage-man-married-her}%
\exmn{260.8}%
\begingl
	\glpreamble	“Q!a-ī′îcuye-qātc uwaca′” yūckᴀłnīk. //
	\glpreamble	«\!X̱ʼa.eetí Shuyee Ḵáach uwasháa\!» yóo sh kalneek. //
	\gla	{} {} \llap{«\!‹\!}\rlap{X̱ʼa.eetí} @ {} \rlap{Shuyee} @ {} \rlap{Ḵáach} @ {} {}
			\rlap{uwasháa} @ {} @ {} @ {} @ {} {}
		yóo @ sh @ \rlap{kalneek.} @ {} @ {} @ {} @ {} //
	\glb	{} {} x̱ʼé- eetí shú- ÿee ḵáa -ch {}
			ⱥ- u- i- \rt[²]{shaʷ} -μμH {}
		yóo= sh= k- d- l- \rt[²]{nik} -μμL //
	\glc	{}[\pr{CP} {}[\pr{DP} mouth- remains end- below man -\xx{erg} {}]
			\xx{arg}- \xx{zpfv}- \xx{stv}- \rt[²]{woman} -\xx{var} {}]
		\xx{quot}= \xx{rflx·o}= \xx{qual}- \xx{mid}- \xx{xtn}- \rt[²]{tell} -\xx{var} //
	\gld	{} {} \rlap{garbage} {} \rlap{below} {} man {} {}
			\rlap{3>3.\xx{pfv}.marry} {} {} {} {} {}
		\xx{quot} self\• \rlap{\xx{impfv}.report} {} {} {} {} //
	\glft	‘“Below Garbage Man married her” he reports.’
		//
\endgl
\xe

\citeauthor{swanton:1909}’s transcription \orth{Q!a-ī′îcuye-qātc} is missing a \orth{t} in \orth{ī′î} which corresponds to the \orth{ī′tî} seen earlier in (\ref{ex:89-183-report-garbage-man-married-her}).
This is certainly a typo or an error in copying the manuscript for print since his transcription elsewhere has \orth{t} and there is no linguistic reason for its absence here.

\ex\label{ex:89-186-went-to-fight}%
\exmn{260.8}%
\begingl
	\glpreamble	Tc!uʟe′ awe′ yūx hᴀs dju′deᴀt. //
	\glpreamble	Chʼu tle áwé yux̱ has jiwdi.át. //
	\gla	Chʼu tle \rlap{áwé} @ {}
		yux̱ @ has @ \rlap{jiwdi.át.} @ {} @ {} @ {} @ {} @ {} //
	\glb	chʼu tle á -wé
		yúx̱= has= ji- wu- d- i- \rt[¹]{.at} -μH //
	\glc	just then \xx{foc} -\xx{mdst}
		out= \xx{pl}= hand- \xx{pfv}- \xx{mid}- \xx{stv}- \rt[¹]{go·\xx{pl}} -\xx{var} //
	\gld	just then \rlap{it.is} {}
		out they\• \rlap{hand.\xx{pfv}.go·\xx{pl}} {} {} {} {} {} //
	\glft	‘Right then they went out to fight.’
		//
\endgl
\xe

The form \fm{jiwdi.át} in (\lastx) is a somewhat obscure verb whose meaning is not immediately obvious from its parts.
The combination of \fm{ji-} ‘hand’ and the root \fm{\rt[¹]{.at}} ‘plural go’ together give rise to an idiomatic meaning ‘go to fight, go to make war’.
The intransitive form with \fm{d-} is seen in (\lastx) as well as \fm{hasdu aanít jiwduwa.át} ‘people attacked their town’ \parencite[23.114]{story-naish:1973}; there is also a causative form without \fm{d-} and with \fm{s-} as in \fm{has ajiwsi.aat} ‘s/he had them go fight’ \parencite[02/75]{leer:1973}.
A second meaning of the same structures is ‘go to help at a potlatch’ as in \fm{has jiwdi.aat} ‘they came to help at potlatch (no relationship implied)’ as well as the derived noun \fm{jinda.aadí} ‘helpers at potlatch’ \parencite[02/75]{leer:1973}.

\ex\label{ex:89-187-my-daughter-she-says-her-mother}%
\exmn{260.9}%
\begingl
	\glpreamble	“ᴀxsī′k!” yū′q!oyaqa duʟa′. //
	\glpreamble	«\!Ax̱ séekʼ!\!» yóo x̱ʼayaḵá du tláa. //
	\gla	{} \llap{«\!}Ax̱ séekʼ\!» yóo @ \rlap{x̱ʼayaḵá} @ {} @ {} @ {}
		{} du tláa. {} //
	\glb	{} ax̱ séekʼ {} yóo= x̱ʼe- ÿ- \rt[¹]{ḵa} -μH
		{} du tláa {} //
	\glc	{}[\pr{DP} \xx{1sg·pss} daughter {}] \xx{quot}= mouth- \xx{qual}- \rt[¹]{say} -\xx{var}
		{}[\pr{DP} \xx{3h·pss} mother {}] //
	\gld	{} my daughter {} thus \rlap{\xx{impfv}.say} {} {} {}
		{} her mother {} //
	\glft	‘Her mother says “My daughter!”.’
		//
\endgl
\xe

\ex\label{ex:89-188-ran-to-doorway}%
\exmn{260.9}%
\begingl
	\glpreamble	Tc!uʟe′ aq!a′wułt hᴀs lū′waguq. //
	\glpreamble	Chʼu tle a x̱ʼawoolt has loowagúḵ. //
	\gla	Chʼu tle {} a \rlap{x̱ʼawoolt} @ {} @ {} @ {} {}
		has @ \rlap{loowagúḵ.} @ {} @ {} @ {} @ {} //
	\glb	chʼu tle {} a x̱ʼe- \rt[¹]{waͧl} -μμL -t {}
		has= lu- wu- i- \rt[¹]{guḵ} -μH //
	\glc	just then {}[\pr{PP} \xx{3n·pss} mouth- \rt[¹]{hole} -\xx{var} -\xx{pnct} {}]
		\xx{plh}= nose- \xx{pfv}- \xx{stv}- \rt[²]{push} -\xx{var} //
	\gld	just then {} its \rlap{doorway} {} {} -to {}
		they \rlap{\xx{pfv}.run·\xx{pl}} {} {} {} {} //
	\glft	‘Then they ran to the doorway.’
		//
\endgl
\xe

The noun \fm{x̱ʼawool} in (\lastx) describes the opening of a cave, chamber, or other large covered and enclosed structure.
This is conventionally translated as ‘doorway’ but the meaning in Tlingit is broader and encompasses other terms like ‘entry’, ‘entrance’, ‘exit’, and ‘opening’.
The root \fm{\rt[¹]{waͧl}} is phonologically unusual because it has different vowels depending on its context, specifically \fm{a} /\ipa{a}/ and \fm{u} /\ipa{u}/ as represented by the symbol \fm{aͧ}.
The simplest noun based on this root is \fm{wool} [\ipa{wùːɬ}] ‘hole; cave’.
One simple verb with the same root is the state imperfective \fm{yawool} [\ipa{jà.ˈwùːɬ}] ‘it has a hole, it is holed’, but then the perfective counterpart of this verb is \fm{woowaal} [\ipa{wùː.ˈwàːɬ}] ‘it got a hole, it became holed’ and \fm{woowool} is ungrammatical.
The synchronic and diachronic explanations for this variation in root vowels are still uncertain.

The verb \fm{has loowagúḵ} in (\lastx) is very common but it is semantically and syntactically unusual.
In everyday use it is the plural counterpart of the singular \fm{wujixeex} ‘s/he ran’ based on the root \fm{\rt[¹]{xix}} ‘fall, move through space; run’.
One peculiarity of \fm{loowagooḵ} ‘they ran’ lies in the idiomatic combination of \fm{lu-} ‘nose’ and the root \fm{\rt[²]{guḵ}} ‘push, penetrate’.
\citeauthor{leer:1991} suggests that this appears because the combination originally meant ‘push enlongated end forward’ which “evokes the elongated ovoid shape characteristic of a group running” \parencite[51]{leer:1991}.
Another peculiarity is that the root \fm{\rt[²]{guḵ}} is bivalent: \fm{aawagooḵ} ‘s/he pushed it’, \fm{sh wudigooḵ} ‘it (water, rain) pushed itself through’ \parencite[675]{leer:1976}.

\ex\label{ex:89-189-kick-in-against-brush-house}%
\exmn{260.10}%
\begingl
	\glpreamble	Yū′tcac-hît g̣etła′a hît nēł ᴀcuka′ołîtsᴀx. //
	\glpreamble	Yú cháash hít géit la.á hít neil ashukawlitséx̱. //
	\gla	{} Yú {} {} {} cháash hít {} \rlap{géit} @ {} {}
				\rlap{la.á} @ {} @ {} @ {} {} hit {}
		neil @ \rlap{ashukawlitséx̱.} @ {} @ {} @ {} @ {} @ {} @ {} @ {} //
	\glb	{} yú {} {} {} cháash hít {} géi -t {}
				l- \rt[¹]{.a} -μH {} {} hít {}
		neil= a- shu- k- wu- l- i- \rt[²]{tsex̱} -μH //
	\glc	{}[\pr{DP} \xx{dist} {}[\pr{CP} {}[\pr{PP} {}[\pr{DP} brush house {}] against -\xx{pnct} {}]
				\xx{xtn}- \rt[¹]{sit·\xx{sg}} -\xx{var} \·\xx{rel} {}] house {}]
		inside= \xx{arg}- end- \xx{qual}- \xx{pfv}- \xx{xtn}- \xx{stv}- \rt[²]{kick} -\xx{var} //
	\gld	{} that {} {} {} brush house {} against -at {}
				\rlap{\xx{pos}·\xx{impfv}.sit·\xx{sg}} {} {} -that {} house {}
		inside \rlap{3>3.\xx{pfv}.kick·in} {} {} {} {} {} {} {} //
	\glft	‘They kicked in the house that sits against the brush house.’
		//
\endgl
\xe

\ex\label{ex:89-190-boom-it-sounded}%
\exmn{260.10}%
\begingl
	\glpreamble	“Dᴀm” yū′yudowa.ᴀx. //
	\glpreamble	«\!Dám\!» yóo wuduwa.áx̱. //
	\gla	«\!Dám\!» yóo @ \rlap{wuduwa.áx̱.} @ {} @ {} @ {} @ {} @ {} //
	\glb	\pqp{}dám yóo= wu- du- d- i- \rt[²]{.ax̱} -μH //
	\glc	\pqp{}\xx{onomat} \xx{quot}= \xx{pfv}- \xx{4h·s}- \xx{mid}- \xx{stv}- \rt[²]{hear} -\xx{var} //
	\gld	\pqp{}boom \xx{quot} \rlap{\xx{pfv}.ppl.hear} {} {} {} {} {} //
	\glft	‘“Boom” it sounded.’
		//
\endgl
\xe

The verb \fm{wuduwa.áx̱} in (\lastx) is conventionally translated into English as ‘it was heard’, ‘it made noise’, or ‘it sounded’.
But in Tlingit it is literally a transitive verb with a fourth person (indefinite, nonspecific) human subject and so is most literally translated as ‘people heard it’.
Following \citeauthor{swanton:1909}’s lead, the conventional translation is maintained here because the literal translation is awkward and unconventional.

\ex\label{ex:89-191-shone-on-her-face}%
\exmn{260.11}%
\begingl
	\glpreamble	Duyê′tayu kaodigᴀ′n. //
	\glpreamble	Du yát áyú kawdigán. //
	\gla	{} Du \rlap{yát} @ {} {} \rlap{áyú} @ {}
		\rlap{kawdigán.} @ {} @ {} @ {} @ {} @ {} //
	\glb	{} du ÿá -t {} á -yú
		k- wu- d- i- \rt[¹]{gan} -μH //
	\glc	{}[\pr{PP} \xx{3h·pss} face -\xx{pnct} {}] \xx{foc} -\xx{dist}
		\xx{hsfc}- \xx{pfv}- \xx{mid}- \xx{stv}- \rt[¹]{burn} -\xx{var} //
	\gld	{} his face -on {} \rlap{it.is} {}
		\rlap{\xx{pfv}.shine} {} {} {} {} {} //
	\glft	‘It is upon her face that it shone.’
		//
\endgl
\xe

\ex\label{ex:89-192-flee-back-outside}%
\exmn{260.11}%
\begingl
	\glpreamble	Yuhî′tỵīanᴀ′q gā′nîqox hᴀs wu′diqêʟ!. //
	\glpreamble	Yú hít ÿee aanáx̱ gáani ḵux̱ has wudikélʼ. //
	\gla	{} Yú hít ÿee {}
		{} \rlap{aanáx̱} @ {} {}
		{} \rlap{gáani} @ {} {}
		ḵux̱ @ has @ \rlap{wudikélʼ.} @ {} @ {} @ {} @ {} //
	\glb	{} yú hít ÿee {}
		{} á -náx̱ {}
		{} gáan -í {}
		ḵúx̱= has= wu- d- i- \rt[¹]{kelʼ} -μH //
	\glc	{}[\pr{DP} \xx{dist} house below {}]
		{}[\pr{PP} \xx{3n} -\xx{perl} {}]
		{}[\pr{PP} outside -\xx{loc} {}]
		\xx{rev}= \xx{plh}= \xx{pfv}- \xx{mid}- \xx{stv}- \rt[¹]{flee} -\xx{var} //
	\gld	{} that house below {}
		{} it -thru {}
		{} outside -at {}
		back\• they\• \rlap{\xx{pfv}.flee} {} {} {} {} //
	\glft	‘They fled back outside from within that house.’
		//
\endgl
\xe

\ex\label{ex:89-193-where-anger}%
\exmn{260.11}%
\begingl
	\glpreamble	Gūsū′ aỵî′s k!ānt wunū′gu. //
	\glpreamble	Goosú a ÿís xʼaant wunoogú? //
	\gla	\rlap{Goosú} @ {} @ {}
		{} {} a ÿís {}
			{} \rlap{xʼaant} @ {} {}
			\rlap{wunoogú?} @ {} @ {} @ {} {} //
	\glb	goo =s =ú
		{} {} a ÿís {}
			{} xʼaan -t {}
			wu- \rt[¹]{nuk} -μμL -í {} //
	\glc	where =\xx{q} =\xx{locp}
		{}[\pr{DP} {}[\pr{PP} \xx{3n} \xx{ben} {}]
			{} anger -\xx{pnct} {}
			\xx{pfv}- \rt[¹]{feel} -\xx{var} -\xx{nmz} {}] //
	\gld	where \•\xx{q} \•is·at
		{} {} him for {}
			{} anger -of {}
			\rlap{\xx{pfv}.feel} {} {} -ing {} //
	\glft	‘Where is the feeling of anger for him?’
		//
\endgl
\xe

\clearpage
\ex\label{ex:89-194-ashamed}%
\exmn{260.12}%
\begingl
	\glpreamble	Tc!uʟe′ kāwadī′q!. //
	\glpreamble	Chʼu tle kaawadéixʼ. //
	\gla	Chʼu tle \rlap{kaawadéixʼ.} @ {} @ {} @ {} @ {} //
	\glb	chʼu tle k- wu- i- \rt[¹]{dexʼ} -μμH //
	\glc	just then \xx{qual}- \xx{pfv}- \xx{stv}- \rt[¹]{shame} -\xx{var} //
	\gld	just then \rlap{\xx{pfv}.ashamed} {} {} {} {} //
	\glft	‘They were ashamed.’
		//
\endgl
\xe

It is not entirely clear in (\lastx) who is the one who has become ashamed by the preceding events because the sentence refers to a third person without explicit indication of who that third person should be.
\citeauthor{swanton:1909} translates (\lastx) as “Then she became ashamed” which implies that the newly married daughter of the town leader is the one ashamed.
But the sentence in (\ref{ex:89-193-where-anger}) seems to imply that it is the angry townspeople who are ashamed.
This perspective is adopted for the new English translation of (\lastx), but it should be noted that this is not entirely certain, especially since there is no explicit indication of plurality (i.e.\ not \fm{has kaawadéixʼ}).
Other possible people who could be ashamed here include the town leader alone (thus “he was ashamed”), the protagonist (again “he was ashamed”), the slave of the town leader, etc.

\section{Paragraph 12}\label{sec:89-para-12}

There is no paragraph break here in \citeauthor{swanton:1909}’s original text.
Once is introduced here because the original paragraph is fairly long.
This particular location also reflects a change of scene because the time described in (\lastx) is different from that in (\nextx) as indicated by the consecutive clause which starts that sentence.

\ex\label{ex:89-195-send-for-father-in-law}%
\exmn{260.12}%
\begingl
	\glpreamble	Nēłdê′ hᴀs naā′dawe ag̣ā′ qoqāwaqa duwu′. //
	\glpreamble	Neildé has na.áat áwé aag̱áa ḵukaawaḵaa du wóo. //
	\gla	{} {} \rlap{Neildé} @ {} {}
			has @ \rlap{na.áat} @ {} @ {} @ {} {}
		\rlap{áwé} @ {}
		{} \rlap{aag̱áa} @ {} {}
		\rlap{ḵukaawaḵaa} @ {} @ {} @ {} @ {} @ {}
		{} du wóo. {} //
	\glb	{} {} neil -dé {}
			has= n- \rt[¹]{.at} -μμH {} {}
		á -wé
		{} á -g̱áa {}
		ḵu- k- wu- i- \rt[¹]{ḵa} -μμL
		{} du wóo {} //
	\glc	{}[\pr{CP} {}[\pr{PP} home -\xx{all} {}]
			\xx{plh}= \xx{ncnj}- \rt[¹]{go·\xx{pl}} -\xx{var} \·\xx{sub} {}]
		\xx{foc} -\xx{mdst}
		{}[\pr{PP} \xx{3n} -\xx{ades} {}]
		\xx{4h·o}- \xx{qual}- \xx{pfv}- \xx{stv}- \rt[²]{say} -\xx{var}
		{}[\pr{DP} \xx{3h·pss} father·in·law {}] //
	\gld	{} {} home -to {}
			they \rlap{\xx{csec}.go·\xx{pl}} {} {} {} {}
		\rlap{it.is} {}
		{} him -for {}
		\rlap{ppl.\xx{pfv}.send} {} {} {} {} {}
		{} his father·in·law {} //
	\glft	‘Them having gone home, he sent people for him, his father-in-law.’
		//
\endgl
\xe

\ex\label{ex:89-196-paid-eight-coppers}%
\exmn{260.13}%
\begingl
	\glpreamble	Doxᴀ′nt hᴀs ā′dawe nᴀs!gaducu′ tînna′ ᴀcnā′ỵe aosî′ne ᴀsī′ awucā′ỵetc. //
	\glpreamble	Du x̱ánt has .áat áwé nasʼgadooshú tináa ash náa ÿéi awsinei, a sée awushaaÿích. //
	\gla	{} {} Du \rlap{x̱ánt} @ {} {}
			has @ \rlap{.áat} @ {} @ {} @ {} {}
		\rlap{áwé} @ {} +
		{} nasʼgadooshú tináa {}
		{} ash \rlap{náa} @ {} {}
		ÿéi @ \rlap{awsinei,} @ {} @ {} @ {} @ {} @ {} +
		{} {} {} a sée {}
			\rlap{awushaaÿích} @ {} @ {} @ {} @ {} {} {} {} //
	\glb	{} {} du x̱án -t {}
			has= {} \rt[¹]{.at} -μμH {} {}
		á -wé
		{} nasʼgadooshú tináa {}
		{} ash náa {} {}
		ÿéi= a- wu- s- i- \rt[¹]{neʰ} -μμL
		{} {} {} a sée {}
			a- wu- \rt[²]{shaʷ} -μμL -í {} -ch {}  //
	\glc	{}[\pr{CP} {}[\pr{PP} \xx{3h·pss} near -\xx{pnct} {}]
			\xx{plh}= \xx{zcnj}\· \rt[¹]{go·\xx{pl}} -\xx{var} \·\xx{sub} {}]
		\xx{foc} -\xx{mdst}
		{}[\pr{DP} eight copper {}]
		{}[\pr{PP} \xx{3prx·pss} drape \·\xx{loc} {}]
		thus= \xx{arg}- \xx{pfv}- \xx{csv}- \xx{stv}- \rt[¹]{happen} -\xx{var}
		{}[\pr{PP} {}[\pr{CP} {}[\pr{DP} \xx{3n·pss} daughter {}]
			\xx{arg}- \xx{pfv}- \rt[²]{woman} -\xx{var} -\xx{sub} {}] -\xx{erg} {}] //
	\gld	{} {} his near -to {}
			they \rlap{\xx{csec}.go·\xx{pl}} {} {} {} {}
		\rlap{it.is} {}
		{} eight copper {}
		{} his drape \·on {}
		thus \rlap{3>3.\xx{pfv}.make.happen} {} {} {} {} {}
		{} {} {} his daughter {}
			\rlap{3>3.\xx{pfv}.marry} {} {} {} {} {} -cause {} //
	\glft	‘Them having come to him, he put eight coppers on him because he married his daughter.’
		//
\endgl
\xe

\ex\label{ex:89-197-take-down-brush-house}%
\exmn{260.14}%
\begingl
	\glpreamble	ʟe adadᴀ′xdê caoduʟ̣îg̣ê′tc yutcā′c-hît ỵīỵî. //
	\glpreamble	Tle a daadáx̱ dé shawdudlig̱éch yú cháash hít ÿéeÿi. //
	\gla	Tle {} a \rlap{daadáx̱} @ {} {} dé
		\rlap{shawdudlig̱éch} @ {} @ {} @ {} @ {} @ {} @ {} @ {} +
		{} yú cháash hít ÿéeÿi. {} //
	\glb	tle {} a daa -dáx̱ {} dé
		sha- wu- du- d- l- i- \rt[²]{g̱ech} -μH
		{} yú cháash hít ÿéeÿi {} //
	\glc	then {}[\pr{PP} \xx{3n·pss} around -\xx{abl} {}] now
		head- \xx{pfv}- \xx{4h·s}- \xx{mid}- \xx{xtn}- \xx{stv}- \rt[²]{throw} -\xx{var}
		{}[\pr{DP} \xx{dist} brush house \xx{past} {}] //
	\gld	then {} its around -from {} now
		\rlap{\xx{pfv}.ppl.take·down} {} {} {} {} {} {} {}
		{} that brush house former {} //
	\glft	‘Then from around it people finally took it down, that former brush house.’
		//
\endgl
\xe

\ex\label{ex:89-198-shone-out-copper}%
\exmn{261.1}%
\begingl
	\glpreamble	Yut ka′odigᴀn yū′ēq. //
	\glpreamble	Yóot kawdigán yú eiḵ. //
	\gla	Yóot @ \rlap{kawdigán} @ {} @ {} @ {} @ {} @ {}
		{} yú eiḵ. {} //
	\glb	yóot= k- wu- d- i- \rt[¹]{gan} -μH
		{} yú eiḵ {} //
	\glc	off \xx{hsfc}- \xx{pfv}- \xx{mid}- \xx{stv}- \rt[¹]{burn} -\xx{var}
		{}[\pr{DP} \xx{dist} copper {}] //
	\gld	out \rlap{\xx{pfv}.shine} {} {} {} {} {} //
	\glft	‘It shone out, that copper.’
		//
\endgl
\xe

\ex\label{ex:89-199-father-supernatural-help}%
\exmn{261.1}%
\begingl
	\glpreamble	Qo′a duī′c awe′ ye ᴀcī′t ta′odıtᴀn duīg̣ā′ ᴀt nᴀg̣asū′t. //
	\glpreamble	Á ḵu.aa du éesh áwé yéi ash eet tuwditán du eeg̱áa at nag̱asoot. //
	\gla	Á ḵu.aa
		{} du éesh {} \rlap{áwé} @ {}
		yéi {} ash \rlap{eet} @ {} {}
			\rlap{tuwditán} @ {} @ {} @ {} @ {} @ {}
		{} {} {} du \rlap{eeg̱áa} @ {} {}
			at @ \rlap{nag̱asoot.} @ {} @ {} @ {} @ {} {} {} {} //
	\glb	á ḵu.aa
		{} du éesh {} á -wé
		yéi= {} ash ee -t {}
			tu- wu- d- i- \rt[²]{tan} -μH
		{} {} {} du ee -g̱áa {}
			at= n- g̱- \rt[²]{suʰ} -μμL {} {} -t {} //
	\glc	\xx{3n} \xx{contr}
		{}[\pr{DP} \xx{3h·pss} father {}] \xx{foc} -\xx{mdst}
		thus= {}[\pr{PP} \xx{3prx} \xx{base} -\xx{pnct} {}]
			mind- \xx{pfv}- \xx{mid}- \xx{stv}- \rt[²]{hdl·w/e} -\xx{var}
		{}[\pr{PP} {}[\pr{CP} {}[\pr{PP} \xx{3h} \xx{base} -\xx{ades} {}]
			\xx{4n·o}= \xx{ncnj}- \xx{mod}- \rt[²]{sup·help} -\xx{var} \·\xx{sub} {}] -\xx{pnct} {}] //
	\gld	it however
		{} his father {} \rlap{it.is} {}
		thus {} him {} -to {}
			\rlap{\xx{pfv}.think} {} {} {} {} {}
		{} {} {} him {} -for {}
			sth= \rlap{\xx{hort}.super·help} {} {} {} {} {} -for {} //
	\glft	‘But it was his father who had thought of him to supernaturally help him.’
		//
\endgl
\xe

The phrase \fm{du eeg̱áa at nag̱asoot} ‘in order to supernaturally help him’ is an example of a purposive clause.
Purposive clauses are specialized adjunct clause structures which describe the purpose or intended goal for some eventuality described by the rest of the main clause.
Purposive clauses are formed by the combination of hortative mood in a subordinate clause which is followed by a punctual postposition \fm{-t} \parencites[91–92]{naish:1966}[106, 187]{story:1966}[427–428]{leer:1991}.
This can be represented schematically as [\pr{PP} [\pr{CP} \xx{hortative}…\xx{sub}] \fm{-t}] The subordinate clause type is usually implicit without an overt \fm{-í} suffix so that the punctual postposition is attached directly to the end of the hortative verb stem.
Typical English translations of purposive clauses are ‘in order to …’ and ‘so that …’, although sometimes ‘for …’ can be used unambiguously.
The gloss of the punctual postposition can be either ‘-to’ or ‘-for’ depending on which is less ambiguous in context.
\FIXME{Point to other discussions with 90 \#52 (\ref{ex:90-52-sent-people-to-burn-him}) and 100 \#70 (\ref{ex:100-70-got-hungry-went-to-spear}).}

\ex\label{ex:89-200-men-today-become-poor-super-help}%
\exmn{261.2}%
\begingl
	\glpreamble	ᴀtcawe′ ỵiỵidᴀ′de qawu ts!u q!anᴀckîdē′x nᴀ′xsᴀtīn yug̣ā′ayu ᴀt yᴀsē′k. //
	\glpreamble	Ách áwé yá ÿeedádi ḵáawu tsú ḵʼanashkidéix̱ nax̱satéenín óog̱aa áyú at yaseik. //
	\gla	{} \rlap{Ách} @ {} {} \rlap{áwé} @ {}
		{} {} yá \rlap{ÿeedádi} @ {} \rlap{ḵáawu} @ {} {} tsú +
			{} \rlap{ḵʼanashgidéix̱} @ {} @ {} @ {} @ {} @ {} @ {} {}
			\rlap{nax̱satéenín} @ {} @ {} @ {} @ {} @ {} @ {} {} +
		{} \rlap{óog̱aa} @ {} {} \rlap{áyú} @ {}
		at @ \rlap{yaseik.} @ {} @ {} @ {} //
	\glb	{} á -ch {} á -wé
		{} {} yá ÿeedát -í ḵáaʷ -í {} tsú
			{} ḵʼe- n- sh- \rt{git} -i =yé -x̱ {}
			n- g̱- s- \rt[¹]{tiʰ} -μμH -n -ín {}
		{} ú -g̱áa {} á -yú
		at= i- \rt[²]{suʰ} -eμL -k //
	\glc	{}[\pr{PP} \xx{3n} -\xx{erg} {}] \xx{foc} -\xx{mdst}
		{}[\pr{CP} {}[\pr{DP} \xx{prox} moment -\xx{pss} man -\xx{pss} {}] also
			{}[\pr{PP} mouth- \xx{ncnj}- \xx{pej}- \rt{\xx{unkn}} -\xx{rel} =way -\xx{pert} {}]
			\xx{ncnj}- \xx{mod}- \xx{appl}- \rt[¹]{be} -\xx{var} -\xx{nsfx} -\xx{ctng} {}]
		{}[\pr{PP} \xx{3h} -\xx{ades} {}] \xx{foc} -\xx{dist}
		\xx{4h·o}= \xx{stv}- \rt[¹]{sup·help} -\xx{var} -\xx{rep} //
	\gld	{} that -why {} \rlap{it.is} {}
		{} {} this moment {} man -of {} also
			{} \rlap{poverty} {} {} {} {} {} -of {}
			\rlap{\xx{ctng}.be·of} {} {} {} {} {} {} {}
			{} him -for {} \rlap{it.is} {}
			sth\• \rlap{\xx{stv}·\xx{impfv}.super·help.\xx{rep}} {} {} {} //
	\glft	‘That is why men of today also, whenever they become poor, it supernaturally helps them.’
		//
\endgl
\xe

\ex\label{ex:89-201-copper-valuable-because}%
\exmn{261.3}%
\begingl
	\glpreamble	ᴀtcawe′ hē′nᴀxa ēq ā′q!aołitsīn. ᴀq! ye ᴀt wuniỵī′tc. //
	\glpreamble	Ách áwé héinax̱.á eiḵ áa x̱ʼawlitseen, áxʼ yéi at wuneeÿéech. //
	\gla	{} \rlap{Ách} @ {} {} \rlap{áwé} @ {}
		{} \rlap{héinax̱.á} @ {} @ {} {}
		{} eiḵ {}
		{} \rlap{áa} @ {} {}
		\rlap{x̱ʼawlitseen} @ {} @ {} @ {} @ {} @ {} +
		{} {} {} \rlap{áxʼ} @ {} {}
			yéi @ at @ \rlap{wuneeÿích} @ {} @ {} @ {} {} {} {} //
	\glb	{} á -ch {} á -wé
		{} hé- nax̱- á {}
		{} eiḵ {}
		{} á -μ {}
		x̱ʼe- wu- l- i- \rt[¹]{tsin} -μμL
		{} {} {} á -xʼ {}
			yéi= at= wu- \rt[¹]{niʰ} -μμL -í {} -ch {} //
	\glc	{}[\pr{PP} \xx{3n} -\xx{erg} {}] \xx{foc} -\xx{mdst}
		{}[\pr{DP} \xx{mprx}- side- \xx{3n} {}]
		{}[\pr{DP} copper {}]
		{}[\pr{PP} \xx{3n} -\xx{loc} {}]
		mouth- \xx{pfv}- \xx{xtn}- \xx{stv}- \rt[¹]{alive} -\xx{var}
		{}[\pr{PP} {}[\pr{CP} {}[\pr{PP} \xx{3n} -\xx{loc} {}]
			thus= \xx{4n·o}= \xx{pfv}- \rt[¹]{happen} -\xx{var} -\xx{sub} {}] -\xx{erg} {}] //
	\gld	{} that -why {} \rlap{it.is} {}
		{} \rlap{over·here} {} {} {}
		{} copper {}
		{} there -at {}
		\rlap{\xx{pfv}.expensive} {} {} {} {} {}
		{} {} {} there- at {}
			thus\• sth\• \rlap{\xx{pfv}.happen} {} {} {} {} -cause {} //
	\glft	‘That is why copper is so valuable over here, because of things happening there.’
		//
\endgl
\xe

\citeauthor{swanton:1909} erroneously split the sentence in (\lastx) into two separate sentences.
But the second clause \fm{áxʼ yéi at wuneeÿích} ‘because of things happening there’ is clearly an adjunct clause.
Specifically it is an explanatory clause formed by the combination of a subordinate clause with \fm{-í} and the ergative postposition \fm{-ch}.
This would be ungrammatical on its own, and it would make no sense with the following sentences.

\ex\label{ex:89-202-worth-two-slaves-copper}%
\exmn{261.4}%
\begingl
	\glpreamble	Tc!uya′ ỵîdᴀt xᴀ′ng̣āt ts!u dēx gux ckᴀ′teᴀtsinen tînna′. //
	\glpreamble	Chʼu yá ÿeedát x̱áng̱aa tsú déix̱ goox̱ \{woosh kaadé yatseenín\} tináa. //
	\gla	Chʼu {} yá ÿeedát \rlap{x̱áng̱aa} @ {} {} tsú
		{} déix̱ goox̱ {}
		{}\{
			{} woosh \rlap{kaadé} @ {} {}
			\rlap{ÿatseenín} @ {} @ {} @ {}
		{}\}
		{} tináa. {} //
	\glb	chʼu {} yá ÿeedát x̱án -g̱áa {} tsú
		{} déix̱ goox̱ {}
		{}
			{} sh ká -dé {}
			i- \rt[¹]{tsin} -μμL -ín
		{}
		{} tináa {} //
	\glc	just {}[\pr{DP} \xx{prox} moment near -\xx{ades} {}] also
		{}[\pr{DP} two slave {}]
		{} 
			{}[\pr{PP} \xx{recip·pss} \xx{hsfc} -\xx{all} {}]
			\xx{stv}- \rt[¹]{alive} -\xx{var} -\xx{past}
		{} 
		{}[\pr{DP} copper {}] //
	\gld	just {} this moment near -around {} also
		{} two slaves {}
		{} 
			{} ea·oth {} -to {}
			\rlap{\xx{stv}·\xx{impfv}.alive.\xx{past}} {} {} {}
		{} 
		{} copper {} //
	\glft	‘Just recently also it \{was worth\} two slaves, a copper.’
		//
\endgl
\xe

The form given by \citeauthor{swanton:1909} as \orth{ckᴀ′teᴀtsinen} is difficult to interpret and \textcite[10]{leer:1977} leaves it unanalyzed.
\citeauthor{swanton:1909} glosses the form as “used to cost” which suggests that the past tense \fm{-ín} that could correspond to the final \orth{en} portion.
This implies that \orth{tsin} would represent a verb stem like \fm{tseen} and thus the root \fm{\rt[¹]{tsin}} which as seen in (\ref{ex:89-201-copper-valuable-because}) appears in \fm{x̱ʼalitseen} ‘it is expensive, valuable’.
The initial \orth{c} could reflect either the reflexive \fm{sh} or the reciprocal \fm{woosh} which could be a possessive pronoun.
The following \orth{kᴀ} would then be the inalienable noun \fm{ká}.
Then if the \orth{t} of \orth{′teᴀ} is actually \fm{d} then we might have something like \fm{woosh kaadé ÿatseenín} with the \orth{ᴀ} as either the qualifier \fm{ÿ-} or the stative \fm{i-}.
Although tantalizingly close to a solution, this interpretation is still problematic because it would mean something like ‘it used to be alive opposite each other’.
The English translation given follows \citeauthor{swanton:1909}’s translation because this seems to be the intended meaning from his consultant, but this translation ignores what is suggested by the tentative analysis.

\ex\label{ex:89-203-always-thing-that-exists}%
\exmn{261.4}%
\begingl
	\glpreamble	Tcᴀ ʟᴀ′kᵘ qo′dzîtī′ỵī-ᴀtx sîtî′ ᴀq!. //
	\glpreamble	Chʼa tlákw ḵudziteeÿi átx̱ sitee, áxʼ. //
	\gla	Chʼa tlákw
		{} {} {} \rlap{ḵudziteeÿi} @ {} @ {} @ {} @ {} @ {} @ {} {} \rlap{átx̱} {} {} {}
		\rlap{sitee,} @ {} @ {} @ {} +
		{} \rlap{áxʼ.} @ {} {} //
	\glb	chʼa tlákw
		{} {} {} ḵu- d- s- i- \rt[¹]{tiʰ} -μμL -i {} át {} -x̱ {}
		s- i- \rt[¹]{tiʰ} -μμL
		{} á -xʼ {} //
	\glc	just always
		{}[\pr{PP} {}[\pr{DP} {}[\pr{CP} \xx{areal}- \xx{mid}- \xx{xtn}- \xx{stv}- \rt[¹]{be} -\xx{var} -\xx{rel} {}]
			thing {}] -\xx{pert} {}]
		\xx{appl}- \xx{stv}- \rt[¹]{be} -\xx{var}
		{}[\pr{PP} \xx{3n} -\xx{loc} {}] //
	\gld	just always
		{} {} {} \rlap{\xx{stv}·\xx{impfv}.exist} {} {} {} {} {} -that {} thing {} -of {}
		\rlap{\xx{stv}·\xx{impfv}.be·of} {} {} {}
		{} there -at {} //
	\glft	‘It is always a thing that is significant, there.’
		//
\endgl
\xe

The phrase \fm{ḵudziteeÿi átx̱ sitee} in (\lastx) is an idiom.
Literally it means something like ‘it is of things which exist’, but the idiomatic interpretation of ‘a thing that exists’ is something like ‘a thing that is worth considering’, ‘a thing which is remarkable’, or ‘a thing that is important’.
\citeauthor{swanton:1909}’s analysis seems to be clouded by translation problems since he glosses this as \fm{ḵudziteeÿi át} as “living thing”, leading to his misleading translation “It has since become an everlasting thing there (i.e.\ it is now used there all the time)” which goes quite wide of the mark.
