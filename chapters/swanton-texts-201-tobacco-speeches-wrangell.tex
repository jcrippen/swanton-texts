%!TEX root = ../swanton-texts.tex
%%
%% 201. Tobacco speeches (pp. 372–373)
%%

\resetexcnt
\chapter{Tobacco speeches}\label{ch:201-tobacco-speeches}

\vspace{1\baselineskip}

\begin{pairs}
\begin{Leftside}
\beginnumbering
\pstart\noindent
\snum{1}Aaá, ax̱ léelkʼw hás a daat haa tuwatee ÿee tula.eesháani.
\pend

%2
\pstart
\snum{14}Hó hó gunalchéesh ásgí.
\pend
\endnumbering
\end{Leftside}
%%
%% Column break.
%%
\begin{Rightside}
\beginnumbering
\pstart\noindent
\snum{1}Yes, my grandfathers, our minds are about your suffering.
\pend

%2
\pstart
\snum{13}Well, thank you very much.
\pend
\endnumbering
\end{Rightside}
\end{pairs}
\Columns

\vspace{1\baselineskip}

\section{Paragraph 1}\label{sec:201-para-1}

\ex
\begingl
	\glpreamble	A′a ᴀxłī′łk!-hᴀs ᴀdā′t hatū′watī ỵītū′ła ỵīcā′nî. //
	\glpreamble	Aaá, ax̱ léelkʼw hás, a daat haa tuwatee ÿee tula.eesháani. //
	\gla	Aaá,
		{} ax̱ léelkʼw hás, {}
		{} a \rlap{daat} @ {} {} 
		haa @ \rlap{tuwatee} @ {} @ {} @ {}
		{} ÿee {} \rlap{tula.eesháani.} @ {} @ {} @ {} {} {} //
	\glb	aaá
		{} ax̱ léelkʼw hás {}
		{} a daa -t {}
		haa= tu- i- \rt[¹]{tiʰ} -μμL 
		{} ÿee {} tu- l- eesháan -í {} {} //
	\glc	yes 
		{}[\pr{DP} \xx{1sg·pss} g’parent \xx{plh} {}]
		{}[\pr{PP} \xx{3n·pss} around -\xx{pnct} {}]
		\xx{1pl·o}= mind- \xx{stv}- \rt[¹]{be} -\xx{var}
		{}[\pr{DP} \xx{2pl·pss} {}[\pr{NP} mind- \xx{intr}- poverty -\xx{pss} {}] {}] //
	\glft	‘’
		//
\endgl
\xe

\section{Paragraph 2}\label{sec:201-para-2}

\ex
\begingl
	\glpreamble	Ho′ho gunᴀłtcī′c ᴀ′skî. //
	\glpreamble	Hó hó gunalchéesh ásgí. //
	\gla	Hó hó gunalchéesh ásgí. //
	\glb	 //
	\glc	 //
	\glft	‘Well, thank you very much.’
		//
\endgl
\xe

\section{Paragraph 3}\label{sec:201-para-3}

\ex
\begingl
	\glpreamble	Hᴀsduq!ᴀnā′t kîd¯á′n îxō′xq!ᵘyên. //
	\glpreamble	Hasdu x̱ʼanaatx̱ gidaan, i x̱úx̱xʼu yán. //
	\gla	 //
	\glb	 //
	\glc	 //
	\glft	‘Get up from your husbands’ way.’
		//
\endgl
\xe


\section{Swanton’s description and translation}\label{sec:201-swanton-translation}

\begin{quote}\small
If one of the family of the writer’s informant, the Kasq!ague′dî [\fm{Kaasx̱ʼaagweidí}] had married a Nanỵaā′ỵî [\fm{Naanÿaa.aaÿí}] woman and she died, the Nanỵaā′ỵî would invite his people for tobacco.
They invited them there to mourn.
This feast was different from the pleasure feasts, when dancing and such things took place.
The people asked them while the dead body was still lying in the house.
Then the other Kasq!ague′dî would ask the bereaved man to deliver a speech.
The Nanỵaā′ỵî would be very quiet because they were mourning.
Then he would rise and speak as follows:

“Yes, yes, my grandfathers, we remember you are mourning.
We are not smoking this tobacco for which you have invited us.
These long-dead uncles of ours and our mothers are the ones who smoke it.
Do not mourn, my grandfathers.
She is not dead.
Her aunts are holding her on their laps.
All her father’s brothers are shaking hands with her.
Our (dead) chief has come back because he has seen you mourning.
Now, however, he has wiped away your tears.
That is all.”

One of those giving the feast would now reply:

“I thank you deeply, deeply for the things you have done to these grandfathers of yours  with your words.
A person will always take his shell to a dry place.
So you have done to this dead of ours.
All these, your grandfathers, were as if sick.
But now you are good medicine to us.
These words of yours have cured us.

Then they would say to the dead woman:

“Get up from your husbands’ path (so that they may pass out).”

The spirits of the dead of both phratries are supposed to be smoking while their friends on earth smoke, and they also share the feast.
People of the opposite phratry took care of the dead, ebcause it was thought men would be wanting in respect to their opposites if members of their own phratry were invited to do it.
For this service the opposites were well paid.
\end{quote}
