%!TEX root = ../swanton-texts.tex
%%
%% 106. Origin of a Low Caste Name (pp. 369–371)
%%

\resetexcnt
\chapter{Eechká Ḵáawu: Man of the Reef}\label{ch:106-low-caste-name}

This narrative was told to \citeauthor{swanton:1909} by Ḵaadishaan in Wrangell.
In the original publication it is number 106, running from page 369 to 371 and totalling 29 lines of glossed transcription.
\citeauthor{swanton:1909}’s original title is “Origin of a Low Caste Name” which seems to have little to do with the story.
We can infer from this title that this story is supposed to be an aetiological explanation for the origin of the name \fm{Eechká Ḵáawu} ‘Man of the Reef’ and that this name was used among lower class people (of which clan?).
But this is only an educated guess and we have no other information to support this conclusion, nor does \citeauthor{swanton:1909} provide any explanation.
To avoid any unwarranted implications the original title replaced here by the main character’s name.

There are no versions of this story in the well known collections of Tlingit narratives \parencites{de-laguna:1972}{dauenhauer:1987}{nyman:1993}{mcclellan-cruikshank:2007c}.
\citeauthor{swanton:1905} does not list this story in his comparison of Tlingit and Haida myths \parencite{swanton:1905} which suggests that there are no recorded Haida stories that are obviously comparable to this story.
A cursory review of \citeauthor{boas:1895}’s \textit{Indianische Sagen} \parencites{boas:1895}{boas:2002} does not turn up any immediately similar stories.
Given the fairly regular interaction between \fm{Shtaxʼhéen Ḵwáan} and the various Tsimshianic peoples, the Nisg̱aʼa and Coast Tsimshian narrative collections \parencites{boas:1902}{boas:1916} need to be reviewed for any potentially related stories.
The lack of independent attestation for this story does not mean that it is unique, but it does imply that it has not been very widely told among Tlingit people.

\clearpage
\begin{pairs}
\begin{Leftside}
\beginnumbering
\pstart\noindent
\snum{1}Tléixʼ aan áyú.
\snum{2}Atnatée has akwshitán, yú aan ḵu.oowú.
\snum{3}Wáa nanée sáyú násʼgináx̱ ḵáa has wooḵoox̱, atnatée.
\snum{4}Wáa nanée sáyú aadé ÿaa has naḵúx̱, yú has alʼóon nuch eechxʼ x̱oo.
\snum{5}Atx̱ áyú át has uwaḵúx̱, yú eechxʼ x̱oo.
\snum{6}Daa sáyú áxʼ has awsiteen, atkʼátskʼu.
\snum{7}Aadáx̱ yaax̱ has awsigoot;
\snum{8}chʼa g̱una.át has oowajée.
\snum{9}Aadáx̱ áyú chʼa g̱ég̱aa áx̱ has x̱ʼataan.
\snum{10}Tléil hasdu éexʼ ḵʼeishgú.
\snum{11}Aadáx̱ neil has uwaḵúx̱.
\snum{12}Hasdu x̱ooní sákw has awsineex̱.
\pend

%2
\pstart
\snum{13}Aadáx̱ áyú a x̱ʼéixʼ has at téex̱ nuch.
\snum{14}Tléil hasdu jeexʼ at kooshtán.
\snum{15}Aadáx̱, tsá chʼa ang̱a\-x̱éixʼun, tsá hasdu yatʼéixʼ atx̱áa nuch.
\snum{16}Aadáx̱, chʼa daa sá át uwashée, chʼu tle yax̱ kashx̱ínx̱.
\snum{17}Aadáx̱ da.oos aku̬shitán;
\snum{18}tlákw tléil tooshkʼéi nuch.
\snum{19}Tléil yoo kdu.áḵuk.
\snum{20}Oodzikaa.
\snum{21}Aadáx̱ gán ax̱óotʼ g̱anóogún, chʼu tle taÿees yoo aÿa\-lʼéexʼk.
\snum{22}Yú taÿees x̱ʼalitseen.
\snum{23}Aadáx̱ yú ash wusineex̱i ḵu.oo, wáa sá hasdu toowú unéekw nuch.
\snum{24}Aadáx̱ yú atÿátxʼi tin ash koolÿádi chʼu tle yoo ḵuÿalisʼélʼk.
\snum{25}Aadáx̱ yú ash wusineex̱i ḵu.ooch chʼu tle ḵuÿasagéÿxʼ.
\snum{26}Chʼu lítaa aan at layéix̱i chʼu tle yoo aÿalʼéexʼk.
\snum{27}Aadáx̱ chʼa yeisú du náaxʼ yéi ndu.eich atdoogú kʼoodásʼ chʼu tle akg̱asʼélʼch.
\snum{28}Aadáx̱ téel du x̱ʼoosí yéi ndu.eich chʼu tle akg̱asʼélʼch.
\snum{29}Héen alitsʼéx̱.
\snum{30}Ḵa tláx̱ yatléḵwk.
\snum{31}Du náa.at lichʼéx̱ʼwk.
\snum{32}Kadig̱áx̱kw.
\snum{33}Yaa anatéen át chʼu tle ḵut kei ag̱íxʼch.
\snum{34}Aadáx̱ kaxéelʼ ḵaa jée yéi aÿa.óo.
\pend

%3
\pstart
\snum{35}Aadáx̱ yéi wduwasaa «\!Eechká Ḵáawu\!».
\snum{36}Aa\-dáx̱ chʼa ldakát yú aantḵeení yéi x̱ʼaÿaḵá áa ḵux̱ yax̱dux̱aa.
\snum{37}Chʼu wudusneex̱í dáx̱ chʼa tlákw kʼeeljáa ḵa séew áa yéi ÿatee.
\snum{38}Aadáx̱ yú ḵu.oo, ash wusineex̱i chʼa ldakát hás, tsu yú yaakw ÿíkxʼ has woo.aat
		áa ḵux̱ has aÿaawax̱áa.
\snum{39}Yú eech, tsá ḵúnáx̱ a kaax̱ has awusnoogu eech, tsu a káa yan has awsinook.
\snum{40}Aadáx̱ a náḵ ḵux̱ has wudiḵoox̱.
\snum{41}Aadáx̱ neil has uwaḵúx̱.
\snum{42}Yú leengít aaní kanduwayéilʼ.
\snum{43}Yú séew tsú akawaataan.
\snum{44}Aadáx̱ a daa yoo x̱ʼadudli.átk, yú aan ḵu.oowúch.
\snum{45}Yéi ḵux̱ʼaÿaḵá «\!Daa sáyú?\!»;
\snum{46}tléil wuduskú.
\snum{47}Aadáx̱ yú aan ḵu.oowú yéi has x̱ʼaÿaḵá
\snum{48}«\!Tleigíl ÿisakú Eechká Ḵáawu ÿádi áyú.\!»
\pend
\endnumbering
\end{Leftside}
%%
%% Column break.
%%
\begin{Rightside}
\beginnumbering
\pstart\noindent
\snum{1}There is one village.
\snum{2}The townspeople are fond of hunting.
\snum{3}At some point three men went out by boat to go hunting.
\snum{4}At some point they are going along there, among those reefs where they always hunt.
\snum{5}So then they got there, among those reefs.
\snum{6}There is something that they saw there, a young boy.
\snum{7}So they made him go aboard;
\snum{8}they think it’s strange.
\snum{9}So then they’re speaking to him in vain.
\snum{10}He doesn’t want to talk to them.
\snum{11}After that they boated home.
\snum{12}They rescued him to be their future family.
\pend

%2
\pstart
\snum{13}They are always giving him things to eat.
\snum{14}He is not accustomed to things that they have.
\snum{15}After that, whenever they are sleeping, he is always eating things behind their backs.
\snum{16}And then, whatever he has touched, it spills.
\snum{17}He has a habit of sulking;
\snum{18}he is always unhappy.
\snum{19}People cannot get him to do anything.
\snum{20}He is lazy.
\snum{21}And then whenever he is splitting firewood he just breaks the stone axes.
\snum{22}Those stone axes are valuable.
\snum{23}After that, the people who had saved him, how upset they always were.
\snum{24}Then when he plays with the children, he tears their faces.
\snum{25}And then those people who saved him pay back people for the injuries.
\snum{26}When he makes things with knives, he just breaks them.
\snum{27}And then the skin shirts that people have just put on him, he always tears them.
\snum{28}And when people put shoes on his feet, he always tears them.
\snum{29}He has a proclitivity for water.
\snum{30}And he is very gluttonous.
\snum{31}His clothing is repeatedly dirty.
\snum{32}He is constantly wailing.
\snum{33}He keeps losing things that they are giving him.
\snum{34}So he causes a lot of trouble for people.
\pend

%3
\pstart
\snum{35}After that they called him “Man Of The Reef”.
\snum{36}So then all of the townspeople are saying that someone should take him back there.
\snum{37}Ever since he was rescued there are always storms and rain.
\snum{38}So then those people, all of them who rescued him, again went into their boats
		to take him back there.
\snum{39}That reef, just the very reef which they had rescued him from, they put him back down on it again.
\snum{40}After that they went back away from him.
\snum{41}And then they went home.
\snum{42}At last the world became calm.
\snum{43}The rain also let up.
\snum{44}After that people are talking about it, those townspeople.
\snum{45}People are saying “What was he?”;
\snum{46}they don’t know.
\snum{47}So then the townspeople say
\snum{48}“Well don’t you know, it’s a child of Man Of The Reef.”
\pend
\endnumbering
\end{Rightside}
\end{pairs}
\Columns

\section{Swanton’s abstract}\label{sec:106-swanton-abstract}

Some people found a rock man’s son on some rocks and adopted him, but he got them into so much trouble that they carried him back there.
Then the weather, which had been bad, immediately cleared.
Since that time a low-caste person has been called a “man of the rocks”.

\section{Swanton’s translation}\label{sec:106-swanton-translation}

\snum{1}There was a certain village in the north \snum{2}from which the people were fond of going hunting.
\snum{3}By and by three men went out, \snum{4}and finally came to the rocks among which they always hunted.
\snum{5}After they reached the rocks \snum{6}they saw a little boy.
\snum{7}Then they took him aboard, \snum{8}thinking it was strange that he should be there.
\snum{9}When they spoke to him \snum{10}he did not reply.
\snum{11}After that they came home.
\snum{12}They kept him as their friend.

\snum{13}Whenever they gave him something to eat \snum{14}he ate nothing.
\snum{15}Only after everyone had gone to bed did he eat.
\snum{16}Whatever thing he touched would spill on him.
\snum{17}\!\snum{18}He was whimsical and \snum{19}they could do nothing with him.
\snum{20}He was also lazy.
\snum{21}When he was asked to chop wood he broke all of their stone axes.
\snum{22}The axes were then valuable.
\snum{23}Then the people who had kept him were very sorry.
\snum{24}When he played with the children he hurt them badly.
\snum{25}Afterward the people who kept him would have to pay for the injuries.
\snum{26}If he made something with a knife he would break it.
\snum{27}Right after a skin shirt had been put upon him it was in rags.
\snum{28}If shoes were put on his feet they were soon in pieces.
\snum{29}He drank a great deal of water.
\snum{30}He was a great eater.
\snum{31}He was a dirty little fellow.
\snum{32}He was a crybaby.
\snum{33}If they gave him anything to take to another place he lost it.
\snum{34}So he made a great deal of trouble for the people.

\snum{35}Then they said of him, “He is really a man of the rocks.” \snum{36}All the town people agreed to take him back to the place where he had been found.
\snum{37}After he had been brought in it was very rainy.
\snum{38}Then the people who had saved him got into their canoe and carried him back.
\snum{39}They put him on the very same rock from which they had taken him.
\snum{40}Then they went back.
\snum{41}They reached home.
\snum{42}The world was now calm.
\snum{43}The rain had also ceased.
\snum{44}Then the town people were all talking about it.
\snum{45}They said to one another, “What could it have been?” \snum{46}and no one knew.
\snum{47}Finally the town people said, \snum{48}“Don’t you see it was a rock-man’s son?”

\section{Paragraph 1}\label{sec:106-para-1}
\resetexcnt

\ex\label{ex:106-1-one-town}%
\exmn{369.1}%
\begingl
	\glpreamble	ᴀ!ēq! ān ayu′ //
	\glpreamble	Tléixʼ aan áyú. //
	\gla	{} Tléixʼ aan {} \rlap{áyú.} @ {} //
	\glb	{} tléixʼ aan {} á -yú //
	\glc	{}[\pr{DP} one village {}] \xx{cpl} -\xx{dist} //
	\gld	{} one village {} \rlap{there.is} //
	\glft	‘There is one village.’
		//
\endgl
\xe

The first sentence in (\lastx) is divided incorrectly by \citeauthor{swanton:1909}.
He wrongly includes \orth{ᴀt natī′} – which he erroneously glosses “there (up north) was” – at the end of this sentence, but it is clear from his gloss of (\nextx) as “They were fond of hunting” that this word belongs at the beginning of the next sentence.
See the discussion of (\nextx) below for further details.

The transcription of the first word \orth{ᴀ!ēq!} in (\lastx) if read literally would be impossible because the \orth{!} means that the preceding symbol represents an ejective consonant but a vowel like \orth{ᴀ} cannot be ejective.
This is probably a misreading or typo for \orth{ʟ} and then \orth{ʟ!} would represent \fm{tlʼ} /\ipa{tɬʼ}/.
But \fm{tlʼéixʼ} is nonsense and \citeauthor{swanton:1909} glosses \orth{ᴀ!ēq!} as “one” so \fm{tléixʼ} [\ipa{tɬʰéːxʼ}] ‘one’ must be the actual utterance.

\ex\label{ex:106-2-fond-of-hunting}%
\exmn{369.1}%
\begingl
	\glpreamble	ᴀt natī′.
Hᴀs ᴀkᵘcîtᴀ′n yū′ān qa-ū′wu. //
	\glpreamble	Atnatée has akwshitán, yú aan ḵu.oowú. //
	\gla	{} \rlap{Atnatée} @ {} @ {} @ {} {}
			has @ \rlap{akwshitán,} @ {} @ {} @ {} @ {} @ {} @ {}
			{} yú aan \rlap{ḵu.oowú.} @ {} @ {} @ {} {} //
	\glb	{} at= n- \rt[¹]{tiʰ} -μμH {}
			has= a- k- u- sh- i- \rt[²]{tan} -μH
			{} yú aan ḵu- \rt[²]{.u} -μμL -í {} //
	\glc	{}[\pr{DP} \xx{4n·o}= \xx{ncnj}- \rt[¹]{be} -\xx{var} {}]
			\xx{plh}= \xx{arg}- \xx{qual}- \xx{irr}- \xx{pej}- \xx{stv}- \rt[²]{habit} -\xx{var}
			{}[\pr{DP} \xx{dist} town \xx{areal}- \rt[²]{own} -\xx{var} -\xx{pss} {}] //
	\gld	{} \rlap{hunting} {} {} {} {} 
			they \rlap{3>3.\xx{impfv}.be.habit·of} {} {} {} {} {} {}
			{} the town \rlap{dwellers} {} {} \·of {} //
	\glft	‘The townspeople are fond of hunting.’
		%\exvblex{akwshitán}{g}{\fm{-H} state}{s/he has habit of (doing) it}
		//
\endgl
\xe

The word \fm{atnatée} in (\lastx) and (\nextx) is an avoidance term for \fm{alʼóon} ‘hunting’ \parencite[06/158]{leer:1973}.
It is based on the root \fm{\rt[¹]{tiʰ}} ‘be, exist’ which forms \fm{n}-conjugation verbs.
Given the overt \fm{n-} conjugation prefix and the long high tone \fm{-μH} stem, the word \fm{atnatée} is a nominalization of the realizational \fm{at naatée} ‘something has at last come to exist’.
Nominalization in this instance overtly does no more than suppress \fm{i-}.
Avoidance terms are still used today to some extent, particularly among Inland Tlingit people.
One better known example is \fm{yatseeneit} which is used to avoid saying \fm{xóots} ‘brown bear’ and which is derived from the relative clause \fm{yatseeni át} ‘thing which is alive’.

The verb \vblex{akwshitán}{g}{\fm{-H} state}{s/he has a habit of (doing) it} in (\lastx) contains a lexically specified irrealis \fm{u-} and the pejorative \fm{sh-} classifier prefix.
This verb has irrealis because it denotes the possibility of behaviour rather than actual behaviour.
The presence of pejorative \fm{sh-} is likely because it was originally used for behaviours that are disapproved of or dispreferred, but it has lost this disapproval entailment and is now often used for positive behaviours.
Hunting is not generally frowned upon, so the positive usage despite pejorative \fm{sh-} at least dates back to Ḵaadishaan’s time in the late 19th century.

\ex\label{ex:106-3-three-men-went-hunting}%
\exmn{369.1}%
\begingl
	\glpreamble	Wananī′sayu nᴀs!ginᴀ′x qā hᴀs wuqo′x ᴀt nᴀtī′. //
	\glpreamble	Wáa nanée sáyú násʼgináx̱ ḵáa has wooḵoox̱, atnatée. //
	\gla	{} Wáa \rlap{nanée} @ {} @ {} @ {} {} \rlap{sáyú} @ {} @ {}
			{} \rlap{násgʼináx̱} @ {} ḵáa {}
			has @ \rlap{wooḵoox̱,} @ {} @ {} @ {} +
			{} \rlap{atnatée} @ {} @ {} @ {} {} //
	\glb	{} wáa n- \rt[¹]{niʰ} -μμH {} {} s= á- yú
			{} násʼk -náx̱ ḵáa {}
			has= wu- i- \rt[¹]{ḵux̱} -μμL
			{} at= n- \rt[¹]{tiʰ} -μμH {} //
	\glc	{}[\pr{CP} how \xx{ncnj}- \rt[¹]{occur} -\xx{var} \·\xx{sub} {}] \xx{q}= \xx{foc} -\xx{mdst}
			{}[\pr{DP} three -\xx{hum} man {}]
			\xx{plh}= \xx{pfv}- \xx{stv}- \rt[¹]{go·boat} -\xx{var}
			{}[\pr{DP} \xx{4n·o}= \xx{ncnj}- \rt[¹]{be} -\xx{var} {}] //
	\gld	{} how \rlap{\xx{rlzn}.happen} {} {} {} {} {} \rlap{it.is} {}
			{} \rlap{three} {} man {}
			they \rlap{\xx{pfv}.go·by·boat} {} {} {}
			{} \rlap{hunting} {} {} {} {} //
	\glft	‘At some point three men went out by boat to go hunting.’
		//
\endgl
\xe

The phrase \fm{wáa nanée sáyú} ‘somehow; at some point’ in (\lastx) is one example of a few different temporal ordering phrases based on the root \fm{\rt[¹]{niʰ}} \~\ \fm{\rt[¹]{neʰ}} ‘occur, happen’.
Similar phrases include \fm{wáa ng̱aneen sá} ‘whenever it happens’ and \fm{wáa néexʼ sá} \~\ \fm{wáanxʼás} ‘whenever, perhaps’ \parencite[04/120]{leer:1973}.
This specific structure is composed of a wh-question phrase \fm{wáa … sá} ‘how’ and the consecutive aspect form of the verb with \fm{n-} and the long high tone \fm{-μH} stem.
\citeauthor{dauenhauer:1987} sometimes translate this with “at what point was it?” as a question \parencite[35–36]{dauenhauer:1987}, but it is actually a wh-indefinite like \fm{chʼa daa sá} ‘whatever’ rather than a wh-question (rhetorical or otherwise).
Their other suggested translations “after a while” and “a while after this” are closer to the mark.
An especially literal translation is ‘somehow it having finally happened’ which takes into account the wh-indefiniteness (‘somehow’), the covert object of the verb (‘it’), and the consecutive aspect as an adjunct clause form (‘having’) of the realizational aspect (‘has finally happened’).

\ex\label{ex:106-4-rocks-among}%
\exmn{369.2}%
\begingl
	\glpreamble	Wananī′sayu ade′ỵa hᴀs naqo′x yū-hᴀs-aʟ!ū′nnutc-ītcq! xō. //
	\glpreamble	Wáa nanée sáyú aadé ÿaa has naḵúx̱, yú has alʼóon nuch eechxʼ x̱oo. //
	\gla	{} Wáa \rlap{nanée} @ {} @ {} {} \rlap{sáyú} @ {} @ {}
			{} \rlap{aadé} @ {} {} ÿaa @ has @ \rlap{naḵúx̱} @ {} @ {} +
			{} yú {} has @ \rlap{alʼóon} @ {} @ {} \•nuch {} {}
				\rlap{eechxʼ} @ {} x̱oo {} //
	\glb	{} wáa n- \rt[¹]{niʰ} -μμH {} s= á- yú
			{} á -dé {} ÿaa= has= n- \rt[¹]{ḵux̱} -μH
			{} yú {} has= a- \rt[²]{lʼuʼn} -μμH =nuch {} {}
				eech -xʼ x̱oo {} //
	\glc	{}[\pr{CP} how \xx{ncnj}- \rt[¹]{occur} -\xx{var} {}] \xx{q}= \xx{foc} -\xx{mdst}
			{}[\pr{PP} \xx{3n} -\xx{all} {}] along= \xx{plh}= \xx{ncnj}- \rt[¹]{go·boat} -\xx{var}
			{}[\pr{DP} \xx{dist} {}[\pr{CP} \xx{plh}= \xx{xpl}- \rt[²]{hunt} -\xx{var} =\xx{hab·aux} \·\xx{rel} {}]
				 reef -\xx{pl} among {}] //
	\gld	{} how \rlap{\xx{rlzn}.happen} {} {} {} {} \rlap{it.is} {}
			{} \rlap{there-to} {} {} along they \rlap{\xx{prog}.go·by·boat} {} {}
			{} those {} they \rlap{\xx{impfv}.hunt} {} {} \•always \·where {} \rlap{reefs} {} among {} //
	\glft	‘At some point they are going along there, among those reefs where they always hunt.’
		//
\endgl
\xe

The long phrase \fm{yú has alʼóon nuch eechxʼ x̱oo} ‘those reefs among which they always hunt’ in (\lastx) is a DP modified by a relative clause.
The relative clause \fm{has alʼóon nuch} ‘(that) they always hunt’ predictably lacks the relative clause suffix \fm{-i} because the stative \fm{i-} classifier prefix is absent.
The avoidance term \fm{atnatée} ‘hunting’ cannot be used here because it is an idiomatic fixed expression; the equivalent relative clause with this verb would be something like \fm{at natée nuch} ‘thing which has finally always come to exist’.

\ex\label{ex:106-5-got-there-among-reefs}%
\exmn{369.3}%
\begingl
	\glpreamble	Atxa′yu āt hᴀs uwaqo′x yuī′tcq! xō //
	\glpreamble	Atx̱ áyú át has uwaḵúx̱, yú eechxʼ x̱oo. //
	\gla	{} \rlap{Atx̱} @ {} {} \rlap{áyú} @ {}
			{} \rlap{át} @ {} {} has @ \rlap{uwaḵúx̱,} @ {} @ {} @ {} 
			{} yú \rlap{eechxʼ} @ {} x̱oo. {} //
	\glb	{} á -dáx̱ {} á -yú
			{} á -t {} has= u- i- \rt[¹]{ḵux̱} -μH
			{} yú eech -xʼ x̱oo {} //
	\glc	{}[\pr{PP} \xx{3n} -\xx{abl} {}] \xx{foc} -\xx{dist}
			{}[\pr{PP} \xx{3n} -\xx{pnct} {}] \xx{plh}= \xx{zpfv}- \xx{stv}- \rt[¹]{go·boat} -\xx{var}
			{}[\pr{DP}\rlap{\ix{i}} \xx{dist} reef -\xx{pl} among {}] //
	\gld	{} that -after {} \rlap{it.is} {}
			{} \rlap{there-to} {} {} they \rlap{\xx{pfv}.go·by·boat} {} {} {}
			{} those \rlap{reefs} {} among {} //
	\glft	‘So then they got there, among those reefs.’
		//
\endgl
\xe

\citeauthor{swanton:1909} runs the sentence in (\lastx) together with the sentence in (\nextx).
But each sentence has an initial focused phrase which strongly suggests that they are separate syntactic units, each with its own left periphery.
Complicating this is that the phrase \fm{yú eechxʼ x̱oo} ‘among those reefs’ in (\lastx) is coreferential with the pronoun \fm{á} in \fm{át} ‘to there’ before the verb in (\lastx) but it is also coreferential with the pronoun \fm{á} in \fm{áxʼ} ‘at there’ in (\nextx).
The PP \fm{át} ‘to there’ in is part of the motion derivation \vbderiv{NP-\{t,x̱,dé\}}{∅}{\fm{-μ} repetitive}{arriving at NP} whereas the PP \fm{áxʼ} ‘at there’ in (\nextx) is an optional adjunct, so if these are divided into two separate sentences it makes more sense for the phrase \fm{yú eechxʼ x̱oo} to be in the right periphery of the first sentence rather than at the beginning of the second.
Thus in (\lastx) the referent of \fm{át} ‘to there’ is bound by the DP \fm{yú eechxʼ x̱oo} ‘among those reefs’ whereas in (\nextx) the referent of \fm{áxʼ} ‘at there’ is discourse dependent.

The form \fm{átx̱} in (\lastx) is unique in this narrative, but it is well known elsewhere.
The phrase \fm{aadáx̱} ‘from there; after that’ has a few different realizations depending on still poorly understood phonological operations.
The canonical form is \fm{aadáx̱} [\ipa{ʔàː.táχ}] where the third nonhuman pronoun \fm{á} /\ipa{ʔá}/ is lengthened and switches from high to low tone.
Although phonologically unusual this is regular and occurs with other postpositions such as \fm{aadé} [\ipa{ʔàː.té}] ‘to there’ with allative \fm{-dé} and \fm{aag̱áa} [\ipa{ʔàː.qáː}] ‘for it’ with adessive \fm{-g̱áa}, as well as with other /\ipa{Cá}/ nouns such as \fm{a kaadáx̱} [\ipa{ʔà kʰàː.táχ}] ‘from the top surface of it’ with \fm{ká} ‘horizontal surface’ and \fm{a shaadáx̱} [\ipa{ʔà ʃàː.táχ}] ‘from the head of it’ with \fm{shá} ‘head’.
There are also at least four other realizations of \fm{á} + \fm{-dáx̱}: \fm{aax̱} [\ipa{ʔàːχ}], \fm{aatx̱} [\ipa{ʔàːtχ}], \fm{adax̱} [\ipa{ʔà.tàχ}], and the \fm{atx̱} [\ipa{ʔàtχ}] seen here.
The form \fm{aatx̱} is presumably from syncope of the \fm{-dáx̱} vowel in \fm{aadáx̱}.
The \fm{aax̱} form may be derived from \fm{aatx̱} by coda cluster simplification.
The form \fm{adax̱} is apparently formed by deletion of high tone; it may be simply concatenated from \fm{á} and \fm{dáx̱} or it may be derived from \fm{aadáx̱} by shortening of the first vowel.
Finally \fm{atx̱} may be derived either from \fm{aadáx̱} via \fm{aatx̱} or it may be derived from \fm{adax̱}.

\ex\label{ex:106-6-what-they-saw}%
\exmn{369.4}%
\begingl
	\glpreamble	da′sayu ᴀq! hᴀs aosîtī′n ᴀtk!ᴀ′tsk!ᵘ. //
	\glpreamble	Daa sáyú áxʼ has awsiteen, atkʼátskʼu. //
	\gla	{} Daa {} \rlap{sáyú} @ {} @ {}
			{} \rlap{áxʼ} @ {} {} has @ \rlap{awsiteen,} @ {} @ {} @ {} @ {} @ {} 
			{} \rlap{atkʼátskʼu.} @ {} @ {} @ {} @ {} {} //
	\glb	{} daa {} s= á -yú
			{} á -xʼ {} has= a- wu- s- i- \rt[²]{tin} -μμL
			{} at= kʼí- ÿáts -kʼw -í {} //
	\glc	{}[\pr{DP} what {}] \xx{q}= \xx{foc} -\xx{dist}
			{}[\pr{PP} \xx{3n} -\xx{loc} {}] \xx{plh}= \xx{arg}- \xx{pfv}- \xx{xtn}- \xx{stv}- \rt[²]{see} -\xx{var}
			{}[\pr{DP} \xx{4n·pss}=  base- child -\xx{dim} -\xx{pss} {}] //
	\gld	{} thing {} {} \rlap{it.is} {} 
			{} \rlap{there-at} {} {} they \rlap{3>3.\xx{pfv}.see} {} {} {} {} {}
			{} \rlap{young.boy} {} {} {} {} {}  //
	\glft	‘There is something that they saw there, a young boy.’
		//
\endgl
\xe

Just like the phrase \fm{wáa nanée sáyú} ‘somehow it has finally happened’ in (\ref{ex:106-3-three-men-went-hunting}) above, the phrase \fm{daa sáyú} in (\lastx) looks like a wh-question with the Q-particle \fm{sá} but it is actually a wh-indefinite.
Thus this sentence is not a question like ‘What is it that they saw there?’, but rather a description of an indefinite ‘They saw something there’.
The phrase \fm{atkʼátskʼu} ‘young boy’ in the right periphery is the clarification, explicitly describing the indefinite referent.

The word \fm{atkʼátskʼu} ‘young boy’ in (\lastx) appears to be constructed from a noun \fm[*]{kʼátskʼw} but this is not actually the case.
Instead, as shown by the segmentation and gloss in (\lastx), it derives from a compound of \fm{kʼí} ‘base of standing object, rump’ \parencite[f04/116]{leer:1973} and the allomorph \fm{ÿáts} [\ipa{ɰáts}] of the noun \fm{ÿát} [\ipa{ɰát}] ‘child’, as attested by the form \fm{at kʼiyátskʼu} \parencite[03/144]{leer:1973} and by \citeauthor{veniaminov:1846}’s \fm{atkʼiÿátskʼu} (аткига́ц҆̕ку \fm{atkigác̓ʼku} “дитя \fm{ditjá}” ‘child’, \cite[49]{veniaminov:1846}).
In addition, the \fm{kʼi-} ‘base’ may have been reanalyzed from an earlier diminutive \fm{-kʼ} on the end of \fm{at}.
This is suggested by the female counterpart compound \fm{shaatkʼiyátskʼu} ‘little girl’ \parencite[03/144]{leer:1973} and by \citeauthor{veniaminov:1846}’s \fm{shaatkʼiÿétskʼu} (шатќeге́цку \fm{šatꝁegécku} “дѣва \fm{děva}” ‘virgin, maiden’, \cite[50]{veniaminov:1846}).
The \fm{shaat} element in this compound is from \fm{shaawát} ‘woman; girl’, itself from \fm{sháa} ‘woman’ + \fm{ÿát} ‘child’, so \fm{shaatkʼiÿátskʼu} could be ultimately from a pleonastic \fm{shaawát-kʼ} + \fm{yáts} + \fm{-kʼw}.

\ex\label{ex:106-7-put-aboard}%
\exmn{369.4}%
\begingl
	\glpreamble	Adᴀ′x yāx hᴀs aosîgū′t; //
	\glpreamble	Aadáx̱ yaax̱ has awsigoot; //
	\gla	{} \rlap{Aadáx̱} @ {} {} yaax̱ @ has @ \rlap{awsigoot;} @ {} @ {} @ {} @ {} @ {} //
	\glb	{} á -dáx̱ {}
			yaax̱= has= a- wu- s- i- \rt[¹]{gut} -μμL //
	\glc	{}[\pr{PP} \xx{3n} -\xx{abl} {}]
			aboard= \xx{plh}= \xx{agr}- \xx{pfv}- \xx{csv}- \xx{stv}- \rt[¹]{go·\xx{sg}} -\xx{var} //
	\gld	{} that -after {} aboard they \rlap{3>3.\xx{pfv}.cause.go·\xx{sg}} {} {} {} {} {} //
	\glft	‘So they made him go aboard;’
		//
\endgl
\xe

The preverb \fm{yaax̱} ‘aboard’ in (\lastx) is synchronically unanalyzable as anything other than a monomorphemic unit.
But diachronically it derives from \fm{yaakw} ‘boat, canoe’ and the pertingent postposition \fm{-x̱} ‘of, contacting’.
It is one of a handful of \fm{g̱}-conjugation motion derivations, specifically \vbderiv{yaax̱}{g̱}{\fm{-ch} repetitive}{aboard, embarking, into vehicle}.
This is ultimately derived from the motion derivation \fm{NP-x̱}{g̱}{\fm{-ch} repetitive}{down along NP}; this also gives rise to \fm{héen-x̱}{g̱}{\fm{-ch} repetitive}{down into water} to which we can compare \fm{yaakw-x̱} > \fm{yaax̱}.

\ex\label{ex:106-8-strange}%
\exmn{369.5}%
\begingl
	\glpreamble	tc!a g̣o′na-ᴀt hᴀs uwadjī′. //
	\glpreamble	chʼa g̱una.át has oowajée. //
	\gla	chʼa {} \rlap{g̱una.át} @ {} {} has @ \rlap{oowajée.} @ {} @ {} @ {} @ {} //
	\glb	chʼa {} g̱una- át {} has= a- u- i- \rt[²]{jiʰ} -μμH //
	\glc	just {}[\pr{DP} different- thing {}] \xx{plh}= \xx{arg}- \xx{irr}- \xx{stv}- \rt[²]{think} -\xx{var} //
	\gld	just {} \rlap{strange} {} {} they \rlap{3>3.\xx{impfv}.be.think} {} {} {} {} //
	\glft	‘they think it’s strange.’
		//
\endgl
\xe

The phrase in (\lastx) is most literally translated as ‘they think a different thing’, but there is probably a copular equation implied here so that it is more like ‘they think it is a different thing’.
Compare the English expression \fm{Huh, that’s different} when encountering an unusual or unexpected situation.

\ex\label{ex:106-9-speak-in-vain}%
\exmn{369.5}%
\begingl
	\glpreamble	Adᴀ′xayu djag̣ê′g̣a ᴀx hᴀs q!ᴀtā′n. //
	\glpreamble	Aadáx̱ áyú chʼa g̱ég̱aa áx̱ has x̱ʼataan. //
	\gla	{} \rlap{Aadáx̱} @ {} {} \rlap{áyú} @ {} chʼa g̱ég̱aa {} \rlap{áx̱} @ {} {} has @ \rlap{x̱ʼataan.} @ {} @ {} //
	\glb	{} á -dáx̱ {} á -yú chʼa g̱ég̱aa {} á -x̱ {} has= x̱ʼe- \rt[²]{tan} -μμL //
	\glc	{}[\pr{PP} \xx{3n} -\xx{abl} {}] \xx{foc} -\xx{dist} just vainly {}[\pr{PP} \xx{3n} -\xx{pert} {}]
			\xx{plh}= mouth- \rt[²]{hdl·w/e} -\xx{var} //
	\gld	{} \rlap{that-after} {} {} \rlap{it.is} {} just vainly {} him -to {} they \rlap{\xx{impfv}.speak.\xx{rep}} {} {} //
	\glft	‘So then they’re speaking to him in vain.’
		//
\endgl
\xe

The verb phrase \fm{áx̱ has x̱ʼataan} ‘they repeatedly speak to him’ in (\lastx) is a repetitive imperfective form that lacks an overt repetitive suffix.
This is a consequence of the motion derivation \vblex{NP-\{t,x̱,dé\}}{∅}{\fm{-μ} repetitive}{arriving at NP} which is applied to this verb because \fm{\rt[²]{tan}} ‘handle wooden, empty container’ is a motion root.
The pertingent \fm{-x̱} ‘of, contacting’ postposition is specifically required by this motion derivation for the repetitive imperfective aspect, alternating regularly with the punctual \fm{-t} ‘to/around a point’ (in e.g.\ perfective aspect) and the allative \fm{-dé} ‘toward’ (in e.g.\ progressive aspect).

\ex\label{ex:106-10-no-talk}%
\exmn{369.6}%
\begingl
	\glpreamble	ʟēł hᴀsduī′x qē′cgu. //
	\glpreamble	Tléil hasdu éexʼ ḵʼeishgú. //
	\gla	Tléil {} \rlap{hasdu} @ {} \rlap{éexʼ} @ {} {}
			\rlap{ḵʼeishgú.} @ {} @ {} @ {} @ {} //
	\glb	tléil {} has= du ee -xʼ {} 
			ḵʼe- u- sh- \rt[⁰]{gu} -μH //
	\glc	\xx{neg} {}[\pr{PP} \xx{plh}= \xx{3h·pss} \xx{base} -\xx{loc} {}]
			mouth- \xx{irr}- \xx{pej}- \rt[⁰]{enjoy} -\xx{var} //
	\gld	not {} \rlap{them} {} {} -to {}
			\rlap{mouth.\xx{impfv}.be.enjoyable} {} {} {} {} //
	\glft	‘He doesn’t want to talk to them.’
		//
\endgl
\xe

The translation in (\lastx) is not literal.
The root \fm{\rt[⁰]{gu}} ‘enjoy’ is best known from the phrase \fm{ax̱ tuwáa sigóo} ‘I want it’ that is literally ‘it is enjoyable to the face of my mind’.
Without the PP \fm{ax̱ tuwáa} ‘to the face of my mind’ the verb \fm{sigóo} means ‘it is enjoyable, fun, pleasant’.
The addition of \fm{ḵʼe-} ‘mouth’ (a lexically specified allomorph of \fm{x̱ʼe-} ‘mouth’) makes this about speech or possibly other oral phenomena.
The negative form of \fm{sigóo} usually has the pejorative \fm{sh-} replace \fm{s-} \parencite{cable:2017c}, e.g.\ \fm{tléil ushgú}.
Thus the negated verb in (\lastx) literally means something like ‘it is not mouth-enjoyable’.
The adjunct PP \fm{hasdu eexʼ} ‘to them’ could be interpreted so that the lack of mouth-enjoyment is experienced by the hunters, but it makes more sense in context to interpret this PP as referring to speech directed at the hunters so that the lack of mouth-enjoyment is experienced by the boy.
Literally then (\lastx) is something like ‘he does not mouth-enjoy to them’.

\ex\label{ex:106-11-went-home}%
\exmn{369.6}%
\begingl
	\glpreamble	Adᴀ′x nēł hᴀs uwaqo′x. //
	\glpreamble	Aadáx̱ neil has uwaḵúx̱. //
	\gla	{} \rlap{Aadáx̱} @ {} {} {} \rlap{neil} @ {} {} has= @ \rlap{uwaḵúx̱.} @ {} @ {} @ {} //
	\glb	{} á -dáx̱ {} {} neil -t {} has= u- i- \rt[¹]{ḵux̱} -μH //
	\glc	{}[\pr{PP} \xx{3n} -\xx{abl} {}] {}[\pr{PP} home -\xx{pnct} {}]
			\xx{plh}= \xx{zpfv}- \xx{stv}- \rt[¹]{go·boat} -\xx{var} //
	\gld	{} \rlap{that-after} {} {} {} home -to {} they \rlap{\xx{pfv}.go·boat} {} {} {} //
	\glft	‘After that they boated home.’
		//
\endgl
\xe

The verb phrase \fm{neil has uwaḵúx̱} ‘they boated home’ in (\lastx) is another instances of the same motion derivation \vblex{NP-\{t,x̱,dé\}}{∅}{\fm{-μ} repetitive}{arriving at NP}.
The noun \fm{neil} ‘home; inside of building’ often appears without an overt punctual \fm{-t} postposition where it would be expected with any other noun.
Some speakers do say \fm{neilt} [\ipa{nèːɬt}] with an overt \fm{-t} instead of just \fm{neil} [\ipa{nèːɬ}], but this is less common.

\ex\label{ex:106-12-rescued}%
\exmn{369.6}%
\begingl
	\glpreamble	Hᴀsduxō′nî sᴀkᵘ hᴀs aosînē′x. //
	\glpreamble	Hasdu x̱ooní sákw has awsineix̱. //
	\gla	{} \rlap{hasdu} @ {} \rlap{x̱ooní} @ {} sákw {}
		has @ \rlap{awsineex̱.} @ {} @ {} @ {} @ {} @ {}  //
	\glb	{} has= du x̱oon -í sákw {} 
		has= a- wu- s- i- \rt[¹]{nix̱} -μμL //
	\glc	{}[\pr{PP} \xx{plh}= \xx{3h·pss} family -\xx{pss} \xx{fut}\pr{P} {}]
		\xx{plh}= \xx{arg}- \xx{pfv}- \xx{csv}- \xx{stv}- \rt[¹]{save} -\xx{var} //
	\gld	{} \rlap{their} {} \rlap{family} {} for {}
		they \rlap{3>3.\xx{pfv}.make.safe} {} {} {} {} {} //
	\glft	‘They rescued him to be their future family.’
		//
\endgl
\xe

The word \fm{x̱oon} is translated as ‘family’ in (\lastx), but it can also be translated as ‘friend’ which is how it was represented by \citeauthor{swanton:1909}.
The ambiguity of \fm{x̱oon} as ‘family’ or ‘friend’ is current in modern Tlingit as well as in Ḵaadishaan’s era.
The older meaning is ‘family’, or more precisely a matrilineal relative who is not otherwise specified by a kin term.
The meaning ‘friend’ probably developed through European contact in the 19th century when non-Tlingits were no longer necessarily incorporated into the Tlingit kinship system.
Before this era all resident foreigners would be adopted.
This narrative seems to predate European contact, so ‘family’ is used here.
The noun \fm{x̱oon} is also syntactically interesting because it appears to be alienable since it always occurs with a possessive suffix \fm{-í}, but it acts like an inalienable noun because it never occurs without a possessor.
Such nouns can be called ‘pseudo-inalienable’.
The similar noun \fm{x̱ein} ‘addition, equivalent, matching amount’ is related and is sometimes replaced by \fm{x̱oon} as in \fm{a x̱oonídáx̱ yéi daax̱ané} ‘I am taking some away’ \parencite[f02/12]{leer:1973}.
Also compare the PP \fm{a x̱eináx̱} ‘catching up to it’ with \fm{x̱ein} but without \fm{-í}.
\textcite[70]{leer:1978b} suggests a connection between \fm{x̱oon} ‘family; friend’ and Eyak \fm{dəx̣ųh} ‘person, Eyak’ \parencite[186]{krauss:1970} but does not offer any reflexes in Dene languages.

The postnominal modifier \fm{sákw} ‘to-be, future, expected, intended’ is a future tense indicator for a noun.
It has a past tense counterpart \fm{ÿéeÿi} ‘ex-, past, former, late’; compare \fm{ax̱ x̱úx̱ sákw} ‘my husband-to-be’ and \fm{ax̱ x̱úx̱ yéeyi} ‘my ex-husband’.
Both of these usually act like intersective modifiers, adding a restriction on the interpretation of the noun without changing its syntactic status; compare adjectives like \fm{shaan} ‘old’.
But in (\lastx) the phrase \fm{hasdu x̱ooní sákw} ‘their future family’ does not seem to be either the subject or object of the verb, raising the possibility that \fm{sákw} is doing something syntactically unusual.
The subject is certainly the three hunters established in (\ref{ex:106-3-three-men-went-hunting}).
The object of rescue must be the boy.
So the question is whether \fm{hasdu x̱ooní sákw} is the same as the object in this sentence, or if instead \fm{hasdu x̱ooní sákw} has some other syntactic status in this sentence.
If it is the object then the translation should be something like ‘they rescued their future family’.
But this does not seem to be the intended meaning judging from \citeauthor{swanton:1909}’s gloss “Their friend for they saved him”, nor by his translation “They kept him as their friend”.
His gloss “for” under \fm{sákw} is suggestive; it may be the case that \fm{sákw} sometimes acts as a postposition meaning roughly ‘for the future’.
This means that the phrase \fm{hasdu x̱ooní sákw} is actually an adjunct PP rather than an object.
This postposition-like use of \fm{sákw} is also attested from speakers today but has yet to be investigated.

\section{Paragraph 2}\label{sec:106-para-2}

This paragraph division is not present in the original text from \citeauthor{swanton:1909}, who ran the preceding paragraph and this one together.
It is added here because the original paragraph runs very long and because this is a natural breakpoint in the narrative.

\ex\label{ex:106-13-give-him-to-eat}%
\exmn{369.7}%
\begingl
	\glpreamble	Adᴀ′xayu ᴀq!ē′x hᴀs ᴀt tē′xnutc. //
	\glpreamble	Aadáx̱ áyú a x̱ʼéixʼ has at téex̱ nuch. //
	\gla	{} \rlap{Aadáx̱} @ {} {} \rlap{áyú} @ {}
			{} a \rlap{x̱ʼéixʼ} @ {} {}
			has @ at @ \rlap{téex̱} @ {} @ {} @ \•nuch. //
	\glb	{} á -dáx̱ {} á -yú
			{} a x̱ʼéi -xʼ {}
			has= at= \rt[¹]{ti} -μμH -x̱ =nuch //
	\glc	{}[\pr{PP} \xx{3n} -\xx{abl} {}] \xx{foc} -\xx{dist}
			{}[\pr{PP} \xx{3n·pss} mouth -\xx{loc} {}]
			\xx{plh}= \xx{4n·o}= \rt[¹]{handle} -\xx{var} -\xx{rep} =\xx{hab·aux} //
	\gld	{} \rlap{that-after} {} {}
			\rlap{it.is} {}
			{} his mouth -near {}
			they thing \rlap{\xx{impfv}.handle.\xx{rep}} {} {} \•always //
	\glft	‘So then they are always giving him things to eat.’
		//
\endgl
\xe

The phrase \fm{a x̱ʼéixʼ has at téex̱ nuch} in (\lastx) is not literally translated.
The verb root \fm{\rt[²]{ti}} ‘handle’ is the generic handling root so this is a motion verb since all handling verbs are motion verbs.
Being a motion verb entails the presence of a motion derivation.
The presence of the \fm{-x̱} repetitive suffix and the absence of an aspect prefix means that this is a repetitive imperfective.
The \fm{-x̱} repetitive imperfective implies a \fm{∅}-conjugation motion derivation because no other motion derivations have this repetitive imperfective form, and the \fm{-xʼ} locative postposition verifies that this is the derivation \vbderiv{NP-xʼ}{∅}{\fm{-x̱} repetitive}{nearing, approaching NP}.
The literal meaning of this phrase is therefore ‘they are always repeatedly handling things near his mouth’.
The ‘near’ implies that he refuses to eat them, and this is confirmed in (\nextx) and (\anextx).

Since the verb root in (\lastx) is \fm{\rt[²]{ti}} ‘handle’ and not \fm[*]{\rt{te}}, \citeauthor{swanton:1909}’s transcription \orth{tē′x} for \fm{téex̱} [\ipa{tʰíːχ}] highlights the regular phonological phenomenon of uvular lowering in Ḵaadishaan’s variety of the Transitional Northern Tlingit dialect.
The presence of a uvular consonant before or after a high vowel semi-regularly causes the vowel to lower from /\ipa{i}/ to [\ipa{e}] or from /\ipa{u}/ to [\ipa{o}].
This can be seen in (\ref{ex:106-11-went-home}) where the verb root \fm{\rt[¹]{ḵux̱}} /\ipa{qʰʷuχʷ}/ ‘go by boat’ is transcribed as \orth{qo′x} i.e.\ [\ipa{qʰʷóχʷ}] and in (\ref{ex:106-12-rescued}) where the verb root \fm{\rt[¹]{nix̱}} /\ipa{niχ}/ ‘safe, rescued’ is transcribed as \orth{nē′x} i.e.\ [\ipa{nèːχ}].
This is a surface phenomenon in Transitional Northern Tlingit as verified with modern speakers, and by the root \fm{\rt[²]{ti}} /\ipa{tʰi}/ as [\ipa{tʰíː}] when unsuffixed as in e.g.\ (\ref{ex:106-3-three-men-went-hunting}).
It is optional as seen in (\ref{ex:106-33-loses-things}) where the root \fm{\rt[²]{g̱ixʼ}} /\ipa{qixʼ}/ ‘throw singular inanimate’ appears as \orth{g̣î′q!} i.e.\ [\ipa{qíːxʼ}].
Most other Northern varieties of Tlingit have partially lexicalized uvular lowering so that the high front vowel /\ipa{i}/ is instead /\ipa{e}/ – e.g.\ \fm{\rt[¹]{nex̱}} /\ipa{neχ}/ instead of \fm{\rt[¹]{nix̱}} /\ipa{niχ}/ ‘safe, rescued’ and \fm{g̱eiy} /\ipa{qèːj}/ instead of \fm{g̱eey} /\ipa{qìːj}/ ‘bay’ – but in most varieties of Transitional and Inland Northern Tlingit this lexicalization has not taken place.
The uvular lowering of /\ipa{u}/ to [\ipa{o}] is not well documented in most other Tlingit varieties, but it is well known among speakers of Inland Northern varieties from Teslin.

\ex\label{ex:106-14-doesnt-like-things}%
\exmn{369.7}%
\begingl
	\glpreamble	ʟēł hᴀ′sdudjīq! ᴀt kūctᴀ′n. //
	\glpreamble	Tléil hasdu jeexʼ at kooshtán. //
	\gla	Tléil {} \rlap{hasdu} @ {} \rlap{jeexʼ} @ {} {} at @ \rlap{kooshtán.} @ {} @ {} @ {} @ {} //
	\glb	tléil {} has= du jee -xʼ {} at= k- u- sh- \rt[²]{tan} -μH //
	\glc	\xx{neg} {}[\pr{PP} \xx{plh}= \xx{3h·pss} poss’n -\xx{loc} {}]
			\xx{4n·o}= \xx{qual}- \xx{irr}- \xx{pej}- \rt[²]{habit} -\xx{var} //
	\gld	not {} \rlap{their} {} poss’n -in {} thing \rlap{\xx{impfv}.be.accustomed} {} {} {} {} //
	\glft	‘He is not accustomed to things that they have.’
		//
\endgl
\xe

The verb \vblex{akwshitán}{g}{\fm{-H} state}{s/he has habit of (doing) it} is typically interpreted as describing a person having a habit or inclination for something as in (\ref{ex:106-2-fond-of-hunting}).
But in (\lastx) it has an alternative meaning of being accustomed or habituated to something.
Other examples of this alternative meaning include (Tongass) \fm{akushi̥tan} ‘s/he is habituated to it’ \parencite[06/57]{leer:1973} and \fm{séew akwshitán} ‘s/he is accustomed to the rain’ \parencite[372]{leer:1976}.

The fourth person nonhuman object \fm{at=} ‘something; stuff’ in (\lastx) is especially interesting because it is apparently modified by the PP \fm{hasdu jee-xʼ} ‘in their possession’.
If this was actually an independent noun \fm{át} ‘thing’ then the verb would be expected to have the \fm{a-} argument prefix indicating a third person object and subject.
The object must therefore be the object proclitic pronoun \fm{at=} and not the noun \fm{át}.
But the appearance of a PP modifier together with an object pronoun is surprising, suggesting that the syntax of object pronouns is more complex than is currently understood.

\ex\label{ex:106-15-secretly-eating}%
\exmn{369.8}%
\begingl
	\glpreamble	Adᴀ′x tsa′ts!ᴀ ᴀng̣axē′q!un tsa hᴀ′sduyat!ē′q! ᴀt xā′nutc. //
	\glpreamble	Aadáx̱, tsá chʼa ang̱ax̱éixʼun, tsá hasdu yatʼéixʼ atx̱áa nuch. //
	\gla	{} \rlap{Aadáx̱,} @ {} {} 
			{} tsá chʼa \rlap{ang̱ax̱éixʼun} @ {} @ {} @ {} @ {} @ {} {} +
			tsá {} \rlap{hasdu} @ {} \rlap{yatʼéixʼ} @ {} @ {} {}
			at @ x̱áa @ {} @ \•nuch. //
	\glb	{} á -dáx̱ {}
			{} tsá chʼa a- n- g̱- \rt[¹]{x̱exʼw} -μμH -ín {}
			tsá {} has= du ÿá- tʼéiᵏ -xʼ {}
			at= \rt[²]{x̱a} -μμH =nuch //
	\glc	{}[\pr{PP} \xx{3n} -\xx{abl} {}]
			{}[\pr{CP} then just \xx{4h·s}- \xx{ncnj}- \xx{mod}- \rt[¹]{sleep·\xx{pl}} -\xx{var} -\xx{ctng} {}]
			then {}[\pr{PP} \xx{plh}= \xx{3h·pss} face- behind -\xx{loc} {}]
			\xx{4n·o}= \rt[²]{eat} -\xx{var} =\xx{hab·aux} //
	\gld	{} \rlap{that-after} {} {}
			{} then just \rlap{\xx{ctng}.sleep·\xx{pl}} {} {} {} {} {} {}
			then {} \rlap{their} {} face- behind -at {}
			thing \rlap{\xx{impfv}.eat} {} \•always //
	\glft	‘After that, whenever they are sleeping, he is always eating things behind their backs.’
		//
\endgl
\xe

The verb \fm{ang̱ax̱éixʼun} ‘whenever they are sleeping’ in (\lastx) is an example of the contingent aspect expressed with the \fm{n-} conjugation prefix, the \fm{g̱-} modal prefix, and the \fm{-ín} suffix.\footnote{The \fm{-ín} suffix here is glossed as ‘contingent’ but it could be composed of subordinate \fm{-í} and instrumental \fm{-n}, or perhaps identified with past tense \fm{-ín}.
The analysis of contingent aspect is still in its infancy.} This form is always an adjunct clause that describes a regularly occurring situation that precedes or makes possible the situation in the main clause; the usual English translation has the complementizer ‘whenever’ with either past or present tense depending on the main clause.
The \fm{a-} prefix indicates a fourth person human subject; \fm{du-} is the norm for this but all unergative intransitive motion verbs and the unergatives based on \fm{\rt[¹]{taʰ}} ‘sg.\ sleep’, \fm{\rt[¹]{x̱exʼw}} ‘pl.\ sleep’, and \fm{\rt[¹]{g̱ax̱}} ‘cry’ always have \fm{a-} instead of \fm{du-}.

\ex\label{ex:106-16-touch-spill}%
\exmn{369.9}%
\begingl
	\glpreamble	Adᴀ′x tc!a-dā′sᴀ ᴀt ūwacī′ tc!uʟe′ yᴀx kᴀcxê′nx. //
	\glpreamble	Aadáx̱, chʼa daa sá át uwashée, chʼu tle yax̱ kashx̱ínx̱. //
	\gla	{} \rlap{Aadáx̱,} @ {} {}
			{} {} chʼa daa sá {}
				{} \rlap{át} @ {} {}
				\rlap{uwashée,} @ {} @ {} @ {} {} +
		chʼu tle yax̱ @ \rlap{kashx̱ínx̱} @ {} @ {} @ {} @ {} @ {} //
	\glb	{} á -dáx̱ {}
			{} {} chʼa daa sá {}
			{} á -t {}
			u- i- \rt[¹]{shiʰ} -μμH {}
		chʼu tle ÿáx̱= k- d- sh- \rt[¹]{x̱in} -μH -x̱ //
	\glc	{}[\pr{PP} \xx{3n} -\xx{abl} {}]
			{}[\pr{CP} {}[\pr{QP} just what\ix{i} \xx{q} {}]
			{}[\pr{PP} \xx{3n}\ix{i} -\xx{pnct} {}]
			\xx{zpfv}- \xx{stv}- \rt[¹]{reach} -\xx{var} {}]
		just then facing= \xx{qual}- \xx{mid}- \xx{pej}- \rt[¹]{fall} -\xx{var} -\xx{rep} //
	\gld	{} \rlap{that-after} {} {}
			{} {} just \rlap{whatever} {} {}
			{} it -to {}
			\rlap{\xx{pfv}.reach} {} {} {} {}
		just then over \rlap{\xx{impfv}.container·fall.\xx{rep}} {} {} {} {} {} //
	\glft	‘And then, whatever he has touched, it spills.’
		//
\endgl
\xe

The motion verb \fm{yax̱ kashx̱ínx̱} ‘it (container) repeatedly falls’ in (\lastx) deserves some comment.
The root \fm{\rt[¹]{x̱in}} ‘fall’ applies to either sticklike objects or to substances in containers \parencite[f02/28–30]{leer:1973}.
The particular verb here is \vblex{kawjix̱een}{}{motion}{it (fluid in container) fell} \parencite[796]{leer:1976} which can be interpreted either as describing the fall of a container with fluid in it or the fall of fluid from out of a container.
Since this is a motion verb, there is also a motion derivation which is specifically \vblex{ÿax̱}{∅}{\fm{-x̱} repetitive}{turning over}, confirmed by the \fm{ÿáx̱=} ‘facing’ preverb and the \fm{-x̱} repetitive suffix.
The ‘turning over’ meaning of the motion derivation and the ‘fluid in container fall’ meaning of the verb thus combine to give the meaning ‘spill’.

\ex\label{ex:106-17-sulking}%
\exmn{369.9}%
\begingl
	\glpreamble	Adᴀ′x tuū′s ᴀkucitᴀ′n //
	\glpreamble	Aadáx̱ da.oos aku̬shitán; //
	\gla	{} \rlap{Aadáx̱} @ {} {}
			{} \rlap{da.oos} @ {} @ {} {}
			\rlap{aku̬shitán;} @ {} @ {} @ {} @ {} @ {} @ {} //
	\glb	{} á -dáx̱ {}
			{} d- \rt[¹]{.us} -μμL {}
			a- k- u- sh- i- \rt[²]{tan} -μH //
	\glc	{}[\pr{PP} \xx{3n} -\xx{abl} {}]
			{}[\pr{DP} \xx{mid}- \rt[¹]{sulk} -\xx{var} {}]
			\xx{arg}- \xx{qual}- \xx{irr}- \xx{pej}- \xx{stv}- \rt[²]{habit} -\xx{var} //
	\gld	 {} \rlap{that-after} {} {}
			{} \rlap{sulking} {} {} {} 
			\rlap{3>3.\xx{impfv}.be.habit·of} {} {} {} {} {} {} //
	\glft	‘He has a habit of sulking;’
		//
\endgl
\xe

The verb form \fm{aku̬shitán} [\ipa{ʔà.ˌkʰʷù.ʃ\!ì.ˈtʰán}] ‘s/he has a habit of it’ in (\lastx) is unexpected; the usual form  is \fm{akwshitán} [\ipa{ˌʔàkʷ.ʃ\!ì.ˈtʰán}] but compare the Tongass form \fm{akushi̥tan} [\ipa{ʔà.ˈkʰʷuʃ.tʰan}].
This pronunciation could be another instance of Ḵaadishaan adopting a more conservative style of production for storytelling.
Alternatively, \citeauthor{swanton:1909} could have originally written something like \orth{ᴀkᵘcitᴀ′n} with a superscript \orth{ᵘ} for labialization which was later misread by the printer as \orth{u}.

The noun \fm{da.oos} ‘sulking’ in (\lastx) is a nominalization of the verb \vblex{wudi.ús}{∅}{achievement}{s/he has become cranky}.
The root \fm{\rt[²]{.us}} ‘sulk, pout’ is variously used to describe pouting, sulking, being cranky, and being quietly angry \parencites[02/325]{leer:1973}[156]{leer:1976}.
It is closely related to the root \fm{\rt[¹]{.ush}} ‘mope, sulk’ \parencites[02/333]{leer:1973}[160]{leer:1976}, but seems to be distinct from the homophonous \fm{\rt[⁰]{.us}} ‘lively, playful, hyperactive’ \parencites[02/327]{leer:1973}.
The nominalization \fm{da.oos} ‘sulking’ is not otherwise attested, but it is straightforwardly derived from the verb.

\ex\label{ex:106-18-unhappy}%
\exmn{369.9}%
\begingl
	\glpreamble	ʟᴀkᵘ ʟēł tūcqē′nutc. //
	\glpreamble	tlákw tléil tooshkʼéi nuch. //
	\gla	tlákw tléil \rlap{tooshkʼéi} @ {} @ {} @ {} @ {} @ \•nuch. //
	\glb	tlákw tléil tu- u- sh- \rt[¹]{kʼe} -μμH =nuch //
	\glc	always \xx{neg} mind- \xx{irr}- \xx{pej}- \rt[¹]{good} -\xx{var} =\xx{hab·aux} //
	\gld	always not \rlap{mind.\xx{impfv}.good} {} {} {} {} \•always //
	\glft	‘he is always unhappy.’
		//
\endgl
\xe

\ex\label{ex:106-19-cant-direct}%
\exmn{369.10}%
\begingl
	\glpreamble	ʟēł yūkduᴀ′qukᵘ. //
	\glpreamble	Tléil yoo kdu.áḵuk. //
	\gla	Tléil yoo @ \rlap{kdu.áḵuk.} @ {} @ {} @ {} @ {} //
	\glb	tléil yoo= k- du- \rt[²]{.aḵw} -μH -k //
	\glc	\xx{neg} \xx{alt}= \xx{qual}- \xx{4h·s}- \rt[²]{direct} -\xx{var} -\xx{rep} //
	\gld	not to·fro \rlap{\xx{impfv}.people.direct.\xx{rep}} {} {} {} {} //
	\glft	‘People cannot get him to do anything.’
		//
\endgl
\xe

The translation of \fm{tléil yoo kdu.áḵuk} in (\lastx) is not literal.
The root \fm{\rt[²]{.aḵw}} has a variety of related meanings.
Language learners probably best know it from the verb \vblex{akoo.aaḵw}{n}{\fm{-μ} activity}{s/he tries, attempts it}, and in this context the root is often glossed as ‘try’.
But it also appears in several other verbs that involve planning, control, and directing people.
This sense of the root is reflected in the nominalizations \fm{atkuna.áag̱u} ‘commandment, ruling, order, direction’ and \fm{át atkawu.aag̱ú} ‘commander, director, planner’.
Taking into account the alternating repetitive imperfective state with \fm{yoo=i-…-k} (the \fm{i-} is suppressed by negation), the literal translation of (\lastx) is something more like ‘people repeatedly do not direct him’.
This is somewhat misleading though because it could imply that nobody bothers to give him directions whereas the apparently intended meaning is that he does not follow any directions given to him.
The lack of success rather than the lack of attempts is suggested by \citeauthor{swanton:1909}’s gloss “Not they could do anything with him” as well as his translation “they could do nothing with him”.

\ex\label{ex:106-20-lazy}%
\exmn{369.10}%
\begingl
	\glpreamble	Udzikā′. //
	\glpreamble	Oodzikaa. //
	\gla	\rlap{Oodzikaa.} @ {} @ {} @ {} @ {} @ {} @ {} //
	\glb	a- u- d- s- i- \rt[²]{ka} -μμL //
	\glc	\xx{xpl}- \xx{irr}- \xx{mid}- \xx{xtn}- \xx{stv}- \rt[²]{lazy} -\xx{var} //
	\gld	\rlap{\xx{impfv}.be.lazy} {} {} {} {} {} {} //
	\glft	‘He is lazy.’
		//
\endgl
\xe

\ex\label{ex:106-21-break-axe}%
\exmn{369.11}%
\begingl
	\glpreamble	Adᴀ′x gᴀn axō′t! g̣anūgu′n, tc!uʟe′ taỵī′s yūỵaʟ!ī′q!k. //
	\glpreamble	Aadáx̱ gán ax̱óotʼ g̱anóogún, chʼu tle taÿees yoo aÿalʼéexʼk. //
	\gla	{} \rlap{Aadáx̱} @ {} {}
			{} {} gán {}
			\rlap{ax̱óotʼ} @ {} @ {} @ \•g̱anóogún {} +
			chʼu tle {} \rlap{taÿees} @ {} {}
			yoo @ \rlap{aÿalʼéexʼk.} @ {} @ {} @ {} //
	\glb	{} á -dáx̱ {}
			{} {} gán {}
			a- \rt[²]{x̱utʼ} -μμH =g̱anóogún {}
			chʼu tle {} té- ÿees {}
			yoo= a- i- \rt[²]{lʼixʼ} -μμH -k //
	\glc	{}[\pr{PP} \xx{3n} -\xx{abl} {}]
			{}[\pr{CP} {}[\pr{DP} firewood {}]
			\xx{arg}- \rt[²]{adze} -\xx{var} =\xx{ctng·aux} {}]
			just then {}[\pr{DP} stone- wedge {}]
			\xx{alt}= \xx{arg}- \xx{stv}- \rt[²]{break} -\xx{var} //
	\gld	{} \rlap{that-after} {} {}
			{} {} firewood {}
			\rlap{3>3.\xx{impfv}.adze} {} {} =whenever {}
			just then {} \rlap{stone·axe} {} {}
			to·fro \rlap{3>3.\xx{impfv}.be.break.\xx{rep}} {} {} {} //
	\glft	‘And then whenever he is splitting firewood he just breaks the stone axes.’
		//
\endgl
\xe

The verb \fm{yoo aÿalʼéexʼk} ‘he repeatedly breaks it/them’ in (\lastx) is glossed by \citeauthor{swanton:1909} as “he broke all” which implies that it is transitive with a subject and an object.
But \citeauthor{swanton:1909}’s transcription \orth{yūỵaʟ!ī′q!k} suggests that it is intransitive with only an object because there is no \orth{ᴀ} between the \orth{yū} and \orth{ỵaʟ!ī′q!k} that would reflect the argument marking prefix \fm{a-} which indicates a third person object and subject.
Normally the root is lexically specified for a particular valency that determines transitivity, so this ambiguity would be solved by the lexical documentation.
But this root \fm{\rt{lʼixʼ}} ‘break’ is unusual because there are both unaccusative intransitive verbs like \vblex{woolʼéexʼ}{n}{achievement}{it broke} and transitive verbs like \vblex{aawalʼéexʼ}{n}{achievement}{s/he broke it} with the same root and without the addition of a causative \fm{l-} in the transitive.
This implies that the root is at once both monovalent \fm{\rt[¹]{lʼixʼ}} and bivalent \fm{\rt[²]{lʼixʼ}}.
The alternative analysis is a clause \fm{chʼu tle taÿees yoo ÿalʼéexʼk} ‘the stone axes repeatedly break’.
The syntax of (\lastx) does not distinguish the two possibilities because there is only the overt DP object \fm{taÿees} ‘stone axe(s)’; if there is a subject it must be covert.
But \citeauthor{swanton:1909} unambiguously has \fm{a-} in the \orth{yū′aỵaʟīq!k} \fm{yoo aÿalʼéexʼk} of (\ref{ex:106-26-knife-break}), so \fm{a-} is analyzed as being present in (\lastx) for consistency.

Generally a \fm{\rt{CVC}} root is expected to have a short high tone \fm{-H} stem with a consonant suffix like \fm{-k}.
But the stem \fm{lʼéexʼ} of the repetitive imperfective \fm{yoo aÿalʼéexʼk} has a long vowel because this is a lexicalized invariable stem.
The root \fm{\rt[²]{lʼixʼ}} ‘break’ mostly occurs in such stems with a long vowel and high tone, but \textcite[08/201]{leer:1973} gives three exceptional forms with a short vowel: \fm{x̱waalʼíxʼ} ‘I broke it (pole)’, \fm{kax̱waalʼíxʼ} ‘I broke it (pole) in pieces’, and \fm{kawdilʼíxʼ} ‘it (pole) broke’.
He also notes the obscure noun \fm{a lʼéexʼi} ‘broken piece of it’ with the same root.

\ex\label{ex:106-22-axe-expensive}%
\exmn{369.11}%
\begingl
	\glpreamble	Yū′taỵīs q!ᴀłitsī′n. //
	\glpreamble	Yú taÿees x̱ʼalitseen. //
	\gla	{} Yú \rlap{taÿees} @ {} {} \rlap{x̱ʼalitseen.} @ {} @ {} @ {} @ {} //
	\glb	{} yú té- ÿees {} x̱ʼe- l- i- \rt[¹]{tsin} -μμL //
	\glc	{}[\pr{DP} \xx{dist} stone- wedge {}] mouth- \xx{xtn}- \xx{stv}- \rt[¹]{animate} -\xx{var} //
	\gld	{} those \rlap{stone·axe} {} {} \rlap{mouth.\xx{impfv}.be.strong} {} {} {} {} //
	\glft	‘Those stone axes are valuable.’
		//
\endgl
\xe

The noun \fm{taÿees} in (\blastx) and (\lastx) is translated as ‘stone axe’ following \citeauthor{swanton:1909}’s lead.
The \fm{ta-} part of the noun is from \fm{té} ‘stone’, but the \fm{ÿees} part is generally identified as ‘wedge’.
\textcite[03/210]{leer:1973} gives translations of \fm{taÿees} as ‘rock chisel’ and ‘stone adze with handle?’.
The identification is unclear because there are two different tools which could be described by this noun.
One is the stone equivalent of a modern splitting maul where a stone head with an edge is mounted on a handle.
The other is the stone equivalent of a modern splitting wedge which has no handle.
The splitting maul is implied by the root \fm{\rt[²]{x̱utʼ}} ‘adze; chop’, but this is not conclusive.

\ex\label{ex:106-23-upset}%
\exmn{370.1}%
\begingl
	\glpreamble	Adᴀ′x yu-ᴀ′c-wusînē′xe-qoū′ wā′sᴀ hᴀsdutū′wu nī′knutc. //
	\glpreamble	Aadáx̱ yú ash wusineex̱i ḵu.oo, wáa sá hasdu toowú unéekw nuch. //
	\gla	{} \rlap{Aadáx̱} @ {} {} 
			{} yú {} ash @ \rlap{wusineex̱i} @ {} @ {} @ {} @ {} @ {} {} 
				{} \rlap{ḵu.oo} @ {} @ {} {} {} +
			{} wáa sá {} {} \rlap{hasdu} @ {} \rlap{toowú} @ {} {}
			\rlap{unéekw} @ {} @ {} @ \•nuch. //
	\glb	{} á -dáx̱ {}
			{} yú {} ash= wu- s- i- \rt[¹]{nix̱} -μμL -i {} 
				{} ḵu- \rt[²]{.u} -μμL {} {}
			{} wáa sá {} {} has= du tú -í {}
			u- \rt[¹]{nikw} -μμH =nuch //
	\glc	{}[\pr{PP} \xx{3n} -\xx{abl} {}]
			{}[\pr{DP} \xx{dist} {}[\pr{CP} \xx{3prx·o}= \xx{pfv}- \xx{csv}- \xx{stv}-
					\rt[¹]{safe} -\xx{var} -\xx{rel} {}]
				{}[\pr{NP} \xx{areal}- \rt[²]{own} -\xx{var} {}] {}]
			{}[\pr{QP} how \xx{q} {}] {}[\pr{DP} \xx{plh}= \xx{3h·pss} mind -\xx{pss} {}]
			\xx{irr}- \rt[¹]{sick} -\xx{var} =\xx{hab·aux} //
	\gld	{} \rlap{that-after} {} {}
			{} those {} him \rlap{\xx{pfv}.make.safe.\xx{rel}} {} {} {} {} {} {}
				{} \rlap{dwellers} {} {} {} {}
			{} how {} {} {} \rlap{their} {} \rlap{minds} {} {}
			\rlap{\xx{impfv}.be.sick} {} {} \•always //
	\glft	‘After that, the people who had saved him, how upset they always were.’
		//
\endgl
\xe

\ex\label{ex:106-24-tear-kids}%
\exmn{370.2}%
\begingl
	\glpreamble	Adᴀ′x yu-ᴀt-ỵᴀ′tq!î tîn ᴀckułỵᴀ′dî tc!uʟe′ yuqoỵᴀlis!ê′ʟ!k. //
	\glpreamble	Aadáx̱ yú atÿátxʼi tin ash koolÿádi chʼu tle yoo ḵuÿalisʼélʼk. //
	\gla	{} \rlap{Aadáx̱} @ {} {}
			{} {} {} yú {} \rlap{atÿátxʼi} @ {} @ {} @ {} {} {} tin {} +
				ash @ \rlap{koolÿádi} @ {} @ {} @ {} @ {} @ {} @ {} {}
			chʼu tle yoo @ \rlap{ḵuÿalisʼélʼk.} @ {} @ {} @ {} @ {} @ {} @ {} //
	\glb	{} á -dáx̱ {} 
			{} {} {} yú {} at= ÿát -xʼ -í {} {} tin {}
				ash= k- u- d- l- \rt[¹]{ÿat} -μH -í {}
			chʼu tle yoo= ḵu- ÿ- l- i- \rt[²]{sʼelʼ} -μH -k //
	\glc	{}[\pr{PP} \xx{3n} -\xx{abl} {}]
			{}[\pr{CP} {}[\pr{PP} {}[\pr{DP} \xx{dist}
								{}[\pr{NP} \xx{4n·pss}= child -\xx{pl} \xx{pss} {}] {}] \xx{instr} {}]
				\xx{rflx·o}= \xx{qual}- \xx{irr}- \xx{mid}- \xx{csv}- \rt[¹]{child} -\xx{var} -\xx{sub} {}]
			just then \xx{alt}= \xx{4h·o}- face- \xx{xtn}- \xx{stv}- \rt[²]{tear} -\xx{var} -\xx{rep} //
	\gld	{} \rlap{that-after} {} {}
			{} {} {} those {} \rlap{children} {} {} {} {} {} with {}
				self= \rlap{\xx{impfv}.play} {} {} {} {} {} {} {}
			just then to·fro \rlap{their.face.\xx{impfv}.be.tear.\xx{rep}} {} {} {} {} {} {} //
	\glft	‘Then when he plays with the children, he tears their faces.’
		//
\endgl
\xe

The grisly scene in (\lastx) is unusual in its choice of verb.
At first glance it seems that the boy scratches the faces of other children.
But there are two other verb roots which would make more sense if this was meant: \fm{\rt[²]{xʼutʼ}} ‘scratch with instrument’ \parencites[f04/64–65]{leer:1973}[754–756]{leer:1976} and \fm{\rt[²]{dlakw}} ‘scratch with fingers or claws’ \parencites[08/86]{leer:1973}[457]{leer:1976}.
There is also \fm{\rt[¹]{xen}} ‘scab; scratch wound’ \parencites[f03/39–43]{leer:1973}[616]{leer:1976} but this is only used for scratching oneself.
The root \fm{\rt[²]{sʼelʼ}} ‘tear, rip’ \parencites[09/217–220]{leer:1973}[518–519]{leer:1976} is more typically used for tearing or ripping rather than scratching, hence its gloss.
The image produced here by \fm{\rt[²]{sʼelʼ}} ‘tear, rip’ is thus not simply scratching skin, but ripping away or tearing off strips of skin.

\ex\label{ex:106-25-pay-for-injuries}%
\exmn{370.2}%
\begingl
	\glpreamble	Adᴀ′x yu-ᴀ′c-wusînē′xe-qoū′tc tc!uʟe′ koỵᴀsᴀgē′x. //
	\glpreamble	Aadáx̱ yú ash wusineex̱i ḵu.ooch chʼu tle ḵuÿasagéÿxʼ. //
	\gla	{} \rlap{Aadáx̱} @ {} {}
		{} yú {} ash @ \rlap{wusineex̱i} @ {} @ {} @ {} @ {} @ {} {} \rlap{ḵu.ooch} @ {} @ {} @ {} {} +
		chʼu tle \rlap{ḵuÿasagéÿxʼ.} @ {} @ {} @ {} @ {} @ {} //
	\glb	{} á -dáx̱ {}
		{} yú {} ash= wu- s- i- \rt[¹]{nix̱} -μμL -i {} ḵu- \rt[²]{.u} -μμL -ch {}
		chʼu tle ḵu- ÿ- s- \rt[²]{geʼÿ} -μH -xʼ //
	\glc	{}[\pr{PP} \xx{3n} -\xx{abl} {}]
		{}[\pr{DP} \xx{dist} {}[\pr{CP} \xx{3prx·o}= \xx{pfv}- \xx{csv}- \xx{stv}- \rt[¹]{safe} -\xx{var} -\xx{rel} {}]
			\xx{areal}- \rt[²]{own} -\xx{var} -\xx{erg} {}]
		just then \xx{4h·o}- \xx{qual}- \xx{xtn}- \rt[²]{pay·debt} -\xx{var} -\xx{pl} //
	\gld	{} \rlap{that-after} {} {}
		{} those {} him \rlap{\xx{pfv}.make.safe.\xx{rel}} {} {} {} {} {} {} \rlap{people} {} {} -\xx{erg} {}
		just then \rlap{people.\xx{impfv}.pay·debt.\xx{pl}} {} {} {} {} {} //
	\glft	‘And then those people who saved him just pay back people for the injuries.’
		//
\endgl
\xe

The root \fm{\rt[²]{geʼÿ}} ‘pay debt’ \parencites[f05/72–74]{leer:1973}[655]{leer:1976} in (\lastx) specifically describes payment in compensation for misbehaviour such as injuring or publicly insulting someone.
It is easily confused with the root \fm{\rt[²]{ḵe}} ‘pay’ \parencites[f05/65–66]{leer:1973}[868]{leer:1976} because they have similar meanings and sounds.
But \fm{\rt[²]{geʼÿ}} has velar \fm{g} /\ipa{k}/ rather than uvular \fm{ḵ} /\ipa{qʰ}/, a coda consonant \fm{ÿ} /\ipa{ɰ} \~\ \ipa{j}/, and entails some wrong to recompense.
The root \fm{\rt[²]{ge}} ‘stingy, ungenerous’ \parencites[f05/69–70]{leer:1973}[654]{leer:1976} is probably etymologically related to \fm{\rt[²]{geʼÿ}}, as is the derived stem \fm{–geiÿáḵw} ‘claim as payment’ with the deprivative suffix \fm{-ḵ} \~\ \fm{-áḵw}.
The phrase ‘for the injuries’ in the English translation is not present in the Tlingit sentence; it is added to clarify the meaning of ‘pay back’.

\ex\label{ex:106-26-knife-break}%
\exmn{370.3}%
\begingl
	\glpreamble	Tc!ū łîtā′ ān ᴀt łayē′xe tc!uʟe′ yū′aỵaʟīq!k. //
	\glpreamble	Chʼu lítaa aan at layéix̱i chʼu tle yoo aÿalʼéexʼk. //
	\gla	{} Chʼu {} \rlap{lítaa} @ {} @ {} {} {} \rlap{aan} @ {} {}
			at @ \rlap{layéix̱i} @ {} @ {} @ {} {} +
		chʼu tle yoo @ \rlap{aÿalʼéexʼk.} @ {} @ {} @ {} @ {} //
	\glb	{} chʼu {} \rt[²]{lit} -μH -aa {} {} á -n {}
			at= l- \rt[²]{yex̱} -μμH -í {} 
		chʼu tle yoo= a- i- \rt[²]{lʼixʼ} -μμH -k //
	\glc	{}[\pr{CP} just {}[\pr{NP} \rt[²]{slide} -\xx{var} -\xx{inmz} {}] {}[\pr{PP} \xx{3n} -\xx{instr} {}]
			\xx{4n·o}= \xx{xtn}- \rt[²]{make} -\xx{var} -\xx{sub} {}]
		just then \xx{alt}= \xx{arg}- \xx{stv}- \rt[²]{break} -\xx{var} -\xx{rep} //
	\gld	{} just {} \rlap{knife} {} {} {} {} it -with {}
			thing \rlap{\xx{impfv}.make} {} {} \·when {}
		just then to·fro \rlap{3>3.\xx{impfv}.be.break.\xx{rep}} {} {} {} {} //
	\glft	‘When he makes things with knives, he just breaks them.’
		//
\endgl
\xe

The sentence in (\lastx) is ambiguous and \citeauthor{swanton:1909}’s translation “If he made something with a knife he would break it” does not help clarify the ambiguity.
Specifically, the verb \fm{yoo aÿalʼéexʼk} ‘s/he repeatedly breaks it’ could apply to either the fourth person nonhuman object \fm{at=} in \fm{at layéix̱i} ‘when he makes things’ or to \fm{lítaa} ‘knife’.
Today knives are generally made of steel and so are not easily broken, but the explicit reference to \fm{atdoogú kʼoodásʼ} ‘skin shirt’ in (\nextx) implies that this story takes place before European contact or at latest in the early 19th century when Tlingit people were still generally not wearing European clothing.
Before European contact knives were most often made of stone and hence easily chipped or shattered.
Thus in this context the boy could be breaking knives in the same way as he breaks stone axes in (\ref{ex:106-21-break-axe}).
The English translation is constructed to reflect this ambiguity: the antecedent of the pronoun \fm{them} could be either \fm{things} or \fm{knives}.

\ex\label{ex:106-27-shirt-tear}%
\exmn{370.4}%
\begingl
	\glpreamble	Adᴀ′x tc!aye′su dunā′q! yen duē′tc ᴀt dūgu′ k!udᴀ′s! tc!uʟe′ ᴀ′qg̣as!ê′ʟ!tc. //
	\glpreamble	Aadáx̱ chʼa yeisú du náaxʼ yéi ndu.eich atdoogú kʼoodásʼ chʼu tle akg̱asʼélʼch. //
	\gla	{} \rlap{Aadáx̱} @ {} {}
		{} {} chʼa yeisú {} du \rlap{náaxʼ} @ {} {}
			yéi @ \rlap{ndu.eich} @ {} @ {} @ {} @ {} @ {} {}
			\rlap{atdoogú} @ {} @ {} kʼoodásʼ {}
		chʼu tle \rlap{akg̱asʼélʼch.} @ {} @ {} @ {} @ {} @ {} //
	\glb	{} á -dáx̱ {}
		{} {} chʼa yeisú {} du náa -xʼ {}
			yéi= n- du- \rt[²]{.uʰ} -eμL -ch {} {}
			at= dook -ú kʼoodásʼ {}
		chʼu tle a- k- g̱- \rt[²]{sʼelʼ} -μH -ch //
	\glc	{}[\pr{PP} \xx{3n} -\xx{abl} {}]
		{}[\pr{DP} {}[\pr{CP} just still {}[\pr{PP} \xx{3h·pss} body -\xx{loc} {}]
			thus= \xx{ncnj}- \xx{4h·s}- \rt[²]{put} -\xx{var} -\xx{rep} \·\xx{rel} {}]
			\xx{4n·pss}= skin -\xx{pss} shirt {}]
		just then \xx{arg}- \xx{hsfc}- \xx{g̱cnj}- \rt[²]{tear} -\xx{var} -\xx{rep} //
	\gld	{} \rlap{that-after} {} {}
		{} {} just still {} his body -on {}
			thus \rlap{\xx{hab}.people.put.\xx{rel}} {} {} {} {} {} {}
			\rlap{skin} {} {} shirt {}
		just then \rlap{3>3.\xx{hab}.tear} {} {} {} {} {} //
	\glft	‘And then the skin shirts that people have just put on him, he just always tears them.’
		%\exvblex{du náa-xʼ yéi aawa.oo}{n}{achievement, \fm{-x̱} repetitive}{s/he put it (clothing) on him/her}
		//
\endgl
\xe

\ex\label{ex:106-28-shoe-tear}%
\exmn{370.5}%
\begingl
	\glpreamble	Adᴀ′x tīł duq!ō′sî yen du.ē′tc tc!uʟe′ ᴀ′qg̣as!ê′ʟ!tc. //
	\glpreamble	Aadáx̱ téel du x̱ʼoosí yéi ndu.eich chʼu tle akg̱asʼélʼch. //
	\gla	{} \rlap{Aadáx̱} @ {} {}
		{} {} téel {} {} du \rlap{x̱ʼoosí} @ {} {}
			yéi @ \rlap{ndu.eich} @ {} @ {} @ {} @ {} {} +
		chʼu tle \rlap{akg̱asʼélʼch.} @ {} @ {} @ {} @ {} @ {} //
	\glb	{} á -dáx̱ {}
		{} {} téel {} {} du x̱ʼoos -í {}
			yéi= n- du- \rt[²]{.uʰ} -eμL -ch {}
		chʼu tle a- k- g̱- \rt[²]{sʼelʼ} -μH -ch //
	\glc	{}[\pr{PP} \xx{3n} -\xx{abl} {}]
		{}[\pr{CP} {}[\pr{DP} shoe {}] {}[\pr{PP} \xx{3h·pss} foot -\xx{loc} {}]
			thus= \xx{ncnj}- \xx{4h·s}- \rt[²]{put} -\xx{var} -\xx{rep} {}]
		just then \xx{arg}- \xx{hsfc}- \xx{g̱cnj}- \rt[²]{tear} -\xx{var} -\xx{rep} //
	\gld	{} \rlap{that-after} {} {}
		{} {} shoe {} {} his foot -on {}
			thus \rlap{\xx{hab}.people.put} {} {} {} {} {}
		just then \rlap{3>3.\xx{hab}.tear} {} {} {} {} {} //
	\glft	‘And when people put shoes on his feet, he just always tears them.’
		//
\endgl
\xe

\ex\label{ex:106-29-water-proclivity}%
\exmn{370.6}%
\begingl
	\glpreamble	Hīn ᴀ′łits!êx. //
	\glpreamble	Héen alitsʼéx̱. //
	\gla	{} Héen {} \rlap{alitsʼéx̱.} @ {} @ {} @ {} @ {} //
	\glb	{} héen {} a- l- i- \rt[²]{tsʼex̱} -μH //
	\glc	{}[\pr{DP} water {}] \xx{arg}- \xx{xtn}- \xx{stv}- \rt[²]{crave} -\xx{var} //
	\gld	{} water {} \rlap{3>3.\xx{impfv}.predilection} {} {} {} {} //
	\glft	‘He has a thirst for water.’
		//
\endgl
\xe

\ex\label{ex:106-30-gluttonous}%
\exmn{370.6}%
\begingl
	\glpreamble	Qa ʟᴀx yaʟᴀ′qᵘku. //
	\glpreamble	Ḵa tláx̱ yatléḵwk. //
	\gla	Ḵa tláx̱ \rlap{yatléḵwk} @ {} @ {} @ {} //
	\glb	ḵa tláx̱ i- \rt[¹]{tleḵw} -μH -k //
	\glc	and very \xx{stv}- \rt[¹]{berry} -\xx{var} -\xx{rep} //
	\gld	and very \rlap{\xx{impfv}.gluttonous.\xx{rep}} {} {} {} //
	\glft	‘And he is very gluttonous.’
		//
\endgl
\xe

\ex\label{ex:106-31-dirty-clothing}%
\exmn{370.6}%
\begingl
	\glpreamble	Dunā′ ᴀt łīˈtc!êqᵘkᵘ. //
	\glpreamble	Du náa.at lichʼéx̱ʼwk. //
	\gla	{} Du \rlap{naa.át} @ {} {} \rlap{lichʼéx̱ʼwk.} @ {} @ {} @ {} @ {} //
	\glb	{} du náa- át {} l- i- \rt[¹]{chʼex̱ʼw} -μH -k //
	\glc	{}[\pr{DP} \xx{3h·pss} draping- thing {}] \xx{xtn}- \xx{stv}- \rt[¹]{dirt} -\xx{var} -\xx{rep} //
	\gld	{} his \rlap{clothing} {} {} \rlap{\xx{impfv}.be.dirty.\xx{rep}} {} {} {} {} //
	\glft	‘His clothing is repeatedly dirty.’
		//
\endgl
\xe

\ex\label{ex:106-32-constantly-wailing}%
\exmn{370.7}%
\begingl
	\glpreamble	Kᴀdîgᴀ′xkᵘ. //
	\glpreamble	Kadig̱áx̱kw. //
	\gla	\rlap{Kadig̱áx̱kw} @ {} @ {} @ {} @ {} @ {} //
	\glb	k- d- i- \rt[¹]{g̱ax̱} -μH -kw //
	\glc	\xx{qual}- \xx{mid}- \xx{stv}- \rt[¹]{cry} -\xx{var} -\xx{rep} //
	\gld	\rlap{\xx{impfv}.\xx{mid}.cry.\xx{rep}} {} {} {} {} {} //
	\glft	‘He is constantly wailing.’
		//
\endgl
\xe

\ex\label{ex:106-33-loses-things}%
\exmn{370.7}%
\begingl
	\glpreamble	Yaᴀ′nᴀtîn ᴀt tc!uʟe′ qot ke ᴀg̣î′q!tc. //
	\glpreamble	Yaa anatéen át chʼu tle ḵut kei ag̱íxʼch. //
	\gla	{} {} Yaa @ \rlap{anatéen} @ {} @ {} @ {} @ {} @ {} {} át {}
			chʼu tle ḵut @ kei @ \rlap{ag̱íxʼch.} @ {} @ {} @ {} //
	\glb	{} {} ÿaa= a- n- \rt[²]{ti} -μμH -n {} {} át {}
			chʼu tle ḵut= kei= a- \rt[²]{g̱ixʼ} -μH -ch //
	\glc	{}[\pr{DP} {}[\pr{CP} along= \xx{arg}- \xx{ncnj}- \rt[²]{handle} -\xx{var} -\xx{nsfx} \·\xx{rel} {}] thing {}]
			just then \xx{err}= up= \xx{arg}- \rt[²]{toss} -\xx{var} -\xx{rep} //
	\gld	{} {} along \rlap{3>3.\xx{prog}.handle.\xx{rel}} {} {} {} {} {} {} thing {}
			just then lost up \rlap{3>3.\xx{impfv}.toss.\xx{rep}} {} {} {} //
	\glft	‘He keeps losing things that they are giving him.’
		//
\endgl
\xe

The main verb \fm{ḵut kei ag̱íxʼch} in (\lastx) is somewhat unusual.
It is based on the handling root \fm{\rt[²]{g̱ixʼ}} ‘toss, throw, pitch’.
To this is added a motion derivation \vbderiv{ḵut=}{g}{no repetitive}{lost, astray}, thus giving a literal meaning of ‘toss and lose’ or more loosely ‘lose carelessly as by tossing’.
The unusual factor is that the form \fm{ḵut kei ag̱íxʼch} appears to be a repetitive imperfective as signalled by the preverb \fm{kei=} ‘up’ (reflecting \fm{g}-conjugation class) and the repetitive suffix \fm{-ch} as well as the lack of an overt aspectual prefix.
As documented by \textcite[220]{leer:1991}, this motion derivation does not have a repetitive imperfective form, so a repetitive imperfective with \fm{kei=…-ch} is unexpected.
Although \fm{-ch} also appears in habituals, this is not a habitual because the \fm{g-} conjugation prefix is absent and the stem is short (\citeauthor{swanton:1909}’s \orth{î} implies [\ipa{i}] whereas his \orth{ī} is [\ipa{iː}].) This repetitive imperfective needs to be verified with modern speakers to confirm whether it is still available.

\ex\label{ex:106-34-cause-trouble}%
\exmn{370.7}%
\begingl
	\glpreamble	Adᴀ′x kaxī′ʟ! qadji′ ye aỵaū′. //
	\glpreamble	Aadáx̱ kaxéelʼ ḵaa jée yéi aÿa.óo. //
	\gla	{} \rlap{Aadáx̱} @ {} {}
		{} \rlap{kaxéelʼ} @ {} @ {} {}
		{} ḵaa \rlap{jée} @ {} {}
		yéi @ \rlap{aÿa.óo.} @ {} @ {} @ {} //
	\glb	{} á -dáx̱ {}
		{} k- \rt[²]{xilʼ} -μμH {}
		{} ḵaa jee -H {}
		yéi= a- i- \rt[²]{.uʰ} -μμH //
	\glc	{}[\pr{PP} \xx{3n} -\xx{abl} {}]
		{}[\pr{DP} \xx{qual}- \rt[¹]{rub} -\xx{var} {}]
		{}[\pr{PP} \xx{4h·pss} poss’n -\xx{loc} {}]
		thus= \xx{arg}- \xx{stv}- \rt[²]{own} -\xx{var} //
	\gld	{} \rlap{that-after} {} {}
		{} \rlap{trouble} {} {} {}
		{} ppl’s poss’n -in {}
		thus \rlap{3>3.\xx{impfv}.own} {} {} {} //
	\glft	‘So he causes a lot of trouble for people.’
		//
\endgl
\xe

The phrase in (\lastx) is an idiom and is not literally translated.
The verb \fm{yéi aya.óo} usually means ‘s/he owns it’ \parencite[147]{leer:1976}.
The addition of the noun \fm{kaxéelʼ} ‘trouble’ then gives ‘s/he owns trouble’.
Then the PP \fm{ḵaa jée} ‘in someone’s/people’s possession’ with this gives a literal meaning of ‘s/he owns trouble in (for) people’s possession’.

\section{Paragraph 3}\label{sec:106-para-3}

As mentioned earlier, this paragraph break corresponds to the second paragraph in \citeauthor{swanton:1909}’s original text.
An additional paragraph break was inserted between the first paragraph and this one to reduce the length of the first paragraph.

\ex\label{ex:106-35-call-him-reefman}%
\exmn{370.9}%
\begingl
	\glpreamble	Adᴀ′x ye dū′wasa, “Itckᴀqā′wo.” //
	\glpreamble	Aadáx̱ yéi wduwasaa «\!Eechká Ḵáawu\!». //
	\gla	{} \rlap{Aadáx̱} @ {}
		{} yéi @ \rlap{wduwasaa} @ {} @ {} @ {} @ {} 
			«\!\rlap{Eechká} @ {} \rlap{Ḵáawu\!».} @ {} //
	\glb	{} á -dáx̱ {}
		yéi= wu- du- i- \rt[²]{sa} -μμL
			\pqp{}eech- ká ḵáaʷ -í //
	\glc	{}[\pr{PP} \xx{3n} -\xx{abl} {}]
		thus= \xx{pfv}- \xx{4h·s}- \xx{stv}- \rt[²]{name} -\xx{var}
			\pqp{}reef- \xx{hsfc} man -\xx{pss} //
	\gld	{} \rlap{that-after} {} {}
		thus \rlap{\xx{pfv}.people.call} {} {} {} {} 
			\pqp{}reef- top man -of //
	\glft	‘After that they called him “Man Of The Reef”.’
		//
\endgl
\xe

\citeauthor{swanton:1909}’s translation of \fm{Eechká Ḵáawu} in (\lastx) as the phrase “He is really a man of the rocks” is misleading.
He seems to have taken the verb \fm{yéi wduwasaa} ‘they called/named him’ as a kind of speech verb given his gloss “they said of him”.
This interpretation presumably then led him to see the phrase \fm{Eechká Ḵáawu} as a comment rather than a name.
But this phrase has no predicative structure; it is just a noun rather than a sentence.
\citeauthor{swanton:1909}’s translation of (\lastx) implies that it is some sort of explanation for the character’s behaviour.
But if \fm{Eechká Ḵáawu} is a proper name then the sentence in (\lastx) is just a background comment in the narrative.
He is called this simply because he was found on a reef, not because he is clumsy or because he causes trouble.

The translation of \fm{Eechká Ḵáawu} as ‘Man Of The Reef’ is not precise.
Literally the noun compound \fm{eechká} refers to the horizontal surface of the reef, i.e.\ the top of the reef that is at or above the water line.
With the possessed noun \fm{káawu} ‘man of’ the whole phrase means literally ‘man of the horizontal surface of the reef’.
But the imprecision of the simpler English translation ‘man of the reef’ is not very misleading, and it has the advantage of being four syllables just like the Tlingit phrase.

\ex\label{ex:106-36-take-him-back}%
\exmn{370.9}%
\begingl
	\glpreamble	Adᴀ′x djîłdakᴀ′t yū′āntqenî yē q!aỵaqa′ ā′qox yêx duxa′. //
	\glpreamble	Aadáx̱ chʼa ldakát yú aantḵeení yéi x̱ʼaÿaḵá áa ḵux̱ yax̱dux̱aa. //
	\gla	{} \rlap{Aadáx̱} @ {} {}
		{} chʼa ldakát yú {} \rlap{aantḵeení} @ {} @ {} @ {} @ {} @ {} {} {}
		yéi @ \rlap{x̱ʼaÿaḵá} @ {} @ {} @ {}
		{} {} \rlap{áa} @ {} {} ḵux̱ @ \rlap{yax̱dux̱aa.} @ {} @ {} @ {} @ {} @ {} {} //
	\glb	{} á -dáx̱ {}
		{} chʼa ldakát yú {} aan- d- \rt[¹]{ḵi} -μμL -n -í {} {}
		yéi= x̱ʼe- ÿ- \rt[¹]{ḵa} -μH
		{} {} á -μ {} ḵúx̱=
			ÿ- {} g̱- du- \rt[²]{x̱a} -μμL {} //
	\glc	{}[\pr{PP} \xx{3n} -\xx{abl} {}]
		{}[\pr{DP} just all \xx{dist} {}[\pr{NP} town- \xx{mid}- \rt[¹]{sit·\xx{pl}} -\xx{var} -\xx{nsfx} -\xx{agt} {}] {}]
		thus= mouth- \xx{qual}- \rt[¹]{say} -\xx{var}
		{}[\pr{CP} {}[\pr{PP} \xx{3n} -\xx{loc} {}] \xx{rev}=
			\xx{qual}- \xx{zcnj}\· \xx{mod}- \xx{4h·s}- \rt[²]{paddle} -\xx{var} {}] //
	\gld	{} \rlap{that-after} {} {}
		{} just all those {} \rlap{townspeople} {} {} {} {} {} {} {}
		thus \rlap{\xx{impfv}.say} {} {} {}
		{} {} there -at {} back\• \rlap{\xx{hort}.ppl.take·by·boat} {} {} {} {} {} {} //
	\glft	‘So then all of the townspeople are saying that someone should take him back there.’
		//
\endgl
\xe

\citeauthor{swanton:1909}’s translation of (\lastx) is “All the town people agreed to take him back to the place where he had been found” which is somewhat misleading.
The actual Tlingit sentence has nothing about agreement; the townspeople are saying ‘someone should take him back there’, not ‘we agree that he should be taken back there’ or ‘we agree that we will take him back there’.
Also, the phrase “to the place where he had been found” is entirely absent in the Tlingit original and the PP \fm{áa} ‘at there’ only implicitly refers to this.

\ex\label{ex:106-37-storm-and-rain}%
\exmn{370.10}%
\begingl
	\glpreamble	Tc!u wudusnē′xe dᴀx tcaʟᴀ′kᵘ īłdja′ qᴀsī′wu yē′ỵati. //
	\glpreamble	Chʼu wudusneex̱í dáx̱ chʼa tlákw kʼeeljáa ḵa séew áa yéi ÿatee. //
	\gla	Chʼu {} {} \rlap{wudusneex̱í} @ {} @ {} @ {} @ {} @ {} @ {} {} \•dáx̱ {} +
		chʼa tlákw {} \rlap{kʼeeljáa} @ {} @ {} @ {} {} ḵa {} séew {}
		{} \rlap{áa} @ {} {} yéi @ \rlap{ÿatee.} @ {} @ {} //
	\glb	chʼu {} {} wu- du- d- s- \rt[¹]{nix̱} -μμL -í {} =dáx̱ {}
		chʼa tlákw {} \rt{kʼil} -μμL -ch -áa {} ḵa {} séew {}
		{} á -μ {} yéi= i- \rt[¹]{tiʰ} -μμL //
	\glc	just {}[\pr{PP} {}[\pr{CP} \xx{pfv}- \xx{4h·s}- \xx{mid}- \xx{stv}- \rt[¹]{safe} -\xx{var} -\xx{sub} {}]
			=\xx{abl} {}]
		just always 
		{}[\pr{NP} \rt{storm?} -\xx{var} -\xx{rep} -\xx{inst} {}] and {}[\pr{NP} rain {}]
		{}[\pr{PP} \xx{3n} -\xx{loc} {}] thus= \xx{stv}- \rt[¹]{be} -\xx{var} //
	\gld	just {} {} \rlap{\xx{pfv}.they.make.safe} {} {} {} {} {} {} {} \•after {}
		just always {} \rlap{storm} {} {} {} {} and {} rain {}
		{} there -at {} thus \rlap{\xx{impfv}.be} {} {} //
	\glft	‘Ever since he was rescued there are always storms and rain.’
		//
\endgl
\xe

The sentence in (\lastx) seems to be out of place because it interrupts the sequence of events between (\blastx) and (\nextx).
This is probably not a transcription or editing mistake however.
Instead, it is likely an anticipatory insertion by Ḵaadishaan who, in the process of telling the story, remembered that the ending includes descriptions of the improved weather as seen in (\ref{ex:106-42-world-calm}) and (\ref{ex:106-43-rain-let-up}).
It probably would have been better to mention the storms and rain earlier, perhaps just before or after the litany of \fm{Eechká Ḵáawu}’s troublesome behaviours in (\ref{ex:106-16-touch-spill})–(\ref{ex:106-33-loses-things}).

\ex\label{ex:106-38-take-him-back}%
\exmn{370.11}%
\begingl
	\glpreamble	Adᴀ′x yū′qoū ᴀc wusînē′xe djîłdakᴀ′t hᴀs ts!u yū′yakᵘỵîkx hᴀs wuā′t ā′qox hᴀs aỵāwaxa. //
	\glpreamble	Aadáx̱ yú ḵu.oo, ash wusineex̱i chʼa ldakát hás, tsu yú yaakw ÿíkxʼ has woo.aat
				áa ḵux̱ has aÿaawax̱áa. //
	\gla	{} \rlap{Aadáx̱} @ {} {}
		{} yú {} \rlap{ḵu.oo,} @ {} @ {} {} {} +
		{} {} ash @ \rlap{wusineex̱i} @ {} @ {} @ {} @ {} @ {} {}
			chʼa ldakát hás, {} +
		tsu
		{} yú yaakw \rlap{ÿíkx̱} @ {} {}
		has @ \rlap{woo.aat} @ {} @ {} @ {} +
		{} {} \rlap{áa} @ {} {}
			ḵux̱ @ has @ \rlap{aÿaawax̱áa.} @ {} @ {} @ {} @ {} @ {} {} //
	\glb	{} á -dáx̱ {} 
		{} yú {} ḵu- \rt[²]{.uʰ} -μμL {} {}
		{} {} ash= wu- s- i- \rt[¹]{nix̱} -μμL -i {}
			chʼa ldakát hás {}
		tsu
		{} yú yaakw ÿík -x̱ {} 
		has= wu- i- \rt[¹]{.at} -μμL
		{} {} á -μ {}
			ḵúx̱= has= a- ÿ- wu- i- \rt[²]{x̱a} -μμH {} //
	\glc	{}[\pr{PP} \xx{3n} -\xx{abl} {}]
		{}[\pr{DP} \xx{dist} {}[\pr{NP} \xx{areal}- \rt[²]{own} -\xx{var} {}] {}]
		{}[\pr{DP}	{}[\pr{CP} \xx{3prx·o}= \xx{pfv}- \xx{csv}- \xx{stv}- \rt[¹]{safe} -\xx{var} -\xx{rel} {}]
			just all \xx{3h·pl} {}]
		again
		{}[\pr{PP} \xx{dist} boat within -\xx{pert} {}]
		\xx{plh}= \xx{pfv}- \xx{stv}- \rt[¹]{go·\xx{pl}} -\xx{var}
		{}[\pr{CP} {}[\pr{PP} \xx{3n} -\xx{loc} {}]
			\xx{rev}= \xx{plh}= \xx{arg}- \xx{qual}- \xx{pfv}- \xx{stv}- \rt[²]{paddle} -\xx{var} {}] //	
	\gld	{} \rlap{that-after} {} {}
		{} those {} \rlap{people} {} {} {} {} 
		{} {} him \rlap{\xx{pfv}.make.safe.\xx{rel}} {} {} {} {} {} {} 
			just all them {}
		again
		{} those boat within -in {}
		they \rlap{\xx{pfv}.go·\xx{pl}} {} {} {}
		{} {} there -at {}
			back they \rlap{3>3.\xx{pfv}.take·by·boat} {} {} {} {} {} {} //
	\glft	‘So then those people, all of them who rescued him, again went into their boats
		to take him back there.’
		//
\endgl
\xe

The sentences in (\lastx) and (\nextx) are some of the longest and most syntactically complex in this entire narrative.
Both sentences feature two separate phrases in the left periphery that represent the discourse topic: \fm{yú ḵu.oo} ‘those people’ and \fm{ash wusineex̱i chʼa ldakát hás} ‘all of them who rescued him’ in (\lastx) and \fm{yú eech} ‘that reef’ and \fm{tsá ḵúnáx̱ a kaax̱ has awusnoogu eech} ‘just the very reef which they had rescued him from’ in (\nextx).
The two phrases in each sentence are unique but they refer to the same thing, and so they are repetitions of the same two topics.
The first of each phrase is a determiner and a noun but the second of each phrase is a complex DP with a relative clause and modifiers.
This structural parallelism between sentences is a stylistic technique that is common in Tlingit narrative performance.

Like several other sentences in this narrative the main verb in (\lastx) is accompanied by an unmarked purpose clause \fm{áa ḵux̱ has aÿaawax̱áa} ‘they took him back there by boat’.
This purpose clause is embedded in the main clause but it is not marked for subordination nor is it accompanied by a postposition.
Embedded clauses need not be morphologically different from main clauses despite claims otherwise in the literature.

\ex\label{ex:106-39-put-him-back}%
\exmn{370.13}%
\begingl
	\glpreamble	Yūī′tc ts!aqō′nᴀx ᴀkā′x hᴀs ā′wusnuguītc ts!u ᴀka′ yên hᴀs aosinu′k. //
	\glpreamble	Yú eech, tsá ḵúnáx̱ a kaax̱ has awusnoogu eech, tsu a káa yan has awsinook. //
	\gla	{} Yú eech, {} +
		{} tsá ḵúnáx̱ {} {} a \rlap{kaax̱} @ {} {}
		has @ \rlap{awusnoogu} @ {} @ {} @ {} @ {} @ {} {} eech, {} +
		tsu
		{} a \rlap{káa} @ {} {}
		yan @ has @ \rlap{awsinook.} @ {} @ {} @ {} @ {} @ {} //
	\glb	{} yú eech {}
		{} tsá ḵúnáx̱ {} {} a ká -dáx̱ {}
		has= a- wu- s- \rt[¹]{nuk} -μμL -i {} eech {}
		tsu
		{} a ká -μ {}
		ÿán= has= a- wu- s- i- \rt[¹]{nuk} -μμL //
	\glc	{}[\pr{DP} \xx{dist} reef {}]
		{}[\pr{DP} just very {}[\pr{CP} {}[\pr{PP} \xx{3n·pss} \xx{hsfc} -\xx{abl} {}]
		\xx{plh}= \xx{arg}- \xx{pfv}- \xx{csv}- \rt[¹]{sit·\xx{sg}} -\xx{var} -\xx{rel} {}] reef {}]
		again
		{}[\pr{PP} \xx{3n·pss} \xx{hsfc} -\xx{loc} {}]
		\xx{term}= \xx{plh}= \xx{arg}- \xx{pfv}- \xx{csv}- \xx{stv}- \rt[¹]{sit·\xx{sg}} -\xx{var} //
	\gld	{} that reef {}
		{} just very {} {} it top -from {}
		they \rlap{3>3.\xx{pfv}.make.safe} {} {} {} {} \·which {} reef {}
		again
		{} its top -on {}
		done they \rlap{3>3.\xx{pfv}.make.sit·\xx{sg}} {} {} {} {} {} //
	\glft	‘That reef, just the very reef which they had rescued him from, they put him back down on it again.’
		//
\endgl
\xe

\ex\label{ex:106-40-leave-behind}%
\exmn{370.14}%
\begingl
	\glpreamble	Adᴀ′x ᴀnᴀ′q qox hᴀs wudiqo′x. //
	\glpreamble	Aadáx̱ a náḵ ḵux̱ has wudiḵoox̱. //
	\gla	{} \rlap{Aadáx̱} @ {} {}
		{} a náḵ {}
		ḵux̱ @ has @ \rlap{wudiḵoox̱} @ {} @ {} @ {} @ {} //
	\glb	{} á -dáx̱ {} 
		{} a náḵ {} 
		ḵúx̱= has= wu- d- i- \rt[¹]{ḵux̱} -μμL //
	\glc	{}[\pr{PP} \xx{3n} -\xx{abl} {}]
		{}[\pr{PP} \xx{3n} \xx{elat} {}]
		\xx{rev}= \xx{plh}= \xx{pfv}- \xx{mid}- \xx{stv}- \rt[¹]{go·boat} -\xx{var} //
	\gld {} \rlap{that-after} {} {}
		{} him away·from {}
		back they \rlap{\xx{pfv}.go·by·boat} {} {} {} {} //
	\glft	‘After that they went back away from him.’
		//
\endgl
\xe

The sentence in (\lastx) is a nice example of the relatively uncommon elative postposition \fm{náḵ} ‘away from, leaving behind’.
Here it is part of the motion derivation \vbderiv{NP náḵ}{n}{\fm{yoo=i-…-k} repetitive}{away from, leaving behind NP}.
Although \fm{náḵ} [\ipa{náq}] is phonologically similar to the perlative \fm{-náx̱} [\ipa{náχ}] ‘through, via, by way of, along, across’, the two are semantically and morphophonologically distinct.
In particular, the combination of the pronoun \fm{á} ‘it; there’ and perlative \fm{-náx̱} is like other suffixes in that the canonical form is \fm{aanáx̱} [\ipa{ʔàː.náχ}] with lengthening and loss of high tone on the first syllable.
Contrastively, the combination of \fm{á} and elative \fm{náḵ} leaves the pronoun unchanged as \fm{a náḵ} [\ipa{ʔá náq}] as seen in (\lastx); note that \citeauthor{swanton:1909} has \orth{ᴀnᴀ′q} and not \orth{ānᴀ′q}.

\ex\label{ex:106-41-went-home}%
\exmn{370.15}%
\begingl
	\glpreamble	Adᴀ′x nēł hᴀs uwaqo′x. //
	\glpreamble	Aadáx̱ neil has uwaḵúx̱. //
	\gla	{} \rlap{Aadáx̱} @ {} {}
		{} neil {} {} 
		has @ \rlap{uwaḵúx̱.} @ {} @ {} @ {} //
	\glb	{} á -dáx̱ {}
		{} neil -t {} 
		has= u- i- \rt[¹]{ḵux̱} -μH //
	\glc	{}[\pr{PP} \xx{3n} -\xx{abl} {}]
		{}[\pr{PP} home -\xx{pnct} {}]
		\xx{plh}= \xx{zpfv}- \xx{stv}- \rt[¹]{go·\xx{sg}} -\xx{var} //
	\gld	{} \rlap{that-after} {} {}
		{} home -to {}
		they \rlap{\xx{pfv}.go·by·boat} {} {} {} //
	\glft	‘And then they went home.’
		//
\endgl
\xe

\ex\label{ex:106-42-world-calm}%
\exmn{370.15}%
\begingl
	\glpreamble	Yū′līngîtanî kanduwayē′ʟ!. //
	\glpreamble	Yú leengít aaní kanduwayéilʼ. //
	\gla	{} Yú leengít \rlap{aaní} @ {} {}
		\rlap{kanduwayéilʼ.} @ {} @ {} @ {} @ {} @ {} //
	\glb	{} yú leengít aan -í {}
		k- n- du- i- \rt[²]{yelʼ} -μμH //
	\glc	{}[\pr{DP} \xx{dist} person land -\xx{pss} {}]
		\xx{qual}- \xx{ncnj}- \xx{xpl}- \xx{stv}- \rt[²]{calm} -\xx{var} //
	\gld	{} the person land -of {}
		\rlap{\xx{rlzn}.calm} {} {} {} {} {} //
	\glft	‘At last the world became calm.’
		//
\endgl
\xe

The word \fm{leengít} in (\lastx) is unusual from a modern perspective because the initial syllable is long as [\ipa{ɬìːn.kít}] rather than the usual \fm{lingít} [\ipa{ɬìn.kít}] with a short vowel.
\citeauthor{swanton:1909} clearly has \orth{ī} here rather than \orth{î}, although of course this could be a mistranscription or misreading.
But \citeauthor{leer:1973} records both \fm{leèngit} [\ipa{ɬiʰn.kit}] and \fm{liǹgit} [\ipa{ɬin̤.kit}] in Tongass Tlingit \parencite[f05/96]{leer:1973} where the fading [\ipa{iʰ}] should regularly correspond to long low tone [\ipa{ìː}] in both Southern and Northern Tlingit, and furthermore notes \orth{łēni gitāni} in an unidentified Haida song text from \citeauthor{swanton:1909} \parencite[f05/96]{leer:1973}.\footnote{Elsewhere \citeauthor{leer:2007} argues that the conservative form \fm{leengít} and its Tongass counterpart \fm{leèngit} reflect an earlier \fm[*]{łinagit} where syncope of the second vowel is accompanied (more or less regularly) by the introduction of fading on the preceding vowel \parencite[3]{leer:2007}.
The \fm[*]{łina} portion is then cognate with Eyak \fm{łilaːˀ} ‘man’ and hence also Proto-Dene \fm[*]{dəneː} ‘person’.
The \fm[*]{łi-} vs.\ \fm[*]{də-} is unexplained but both classifier-like prefixes are reconstructed in various nouns across the family \parencite[97]{leer:1990a}.} Thus \citeauthor{swanton:1909}’s \orth{łīngît} is probably not a mistake.
The conservative form \fm{leengít} in (\lastx) might either be used by Ḵaadishaan because his speech is generally conservative or because he deliberately employs conservative forms in traditional narrative performance.

The phrase \fm{leengít aaní} ‘people’s land’ in (\lastx) is conventionally interpreted as ‘world’.
Some speakers have irregularly contracted it to a single word \fm{lingidaaní} [\ipa{ɬìn.kì.ˈtàː.ní}] or \fm{lingitʼaaní} [\ipa{ɬìn.kì.ˈtʼàː.ní}], the latter of which is sometimes further contracted to \fm{ltʼaaní} [\ipa{ɬtʼàː.ní}] \parencite[\textsc{t}·50]{leer:2001}.
\citeauthor{swanton:1909}’s transcription suggests that it may be smushed into a single word.

The verb in (\lastx) is a good example of a realizational aspect form in a main clause context.
Realizationals are formed with the conjugation class prefix (here \fm{n-}), the stative \fm{i-} prefix, and a long high tone \fm{-μH} stem.
Realizational aspect describes a situation that has come to pass after a period of anticipation, hence the English translation with ‘at last’.

\ex\label{ex:106-43-rain-let-up}%
\exmn{371.1}%
\begingl
	\glpreamble	Yū′siu ts!u kāwatā′n. //
	\glpreamble	Yú séew tsú akawaataan. //
	\gla	{} Yú séew {} tsú \rlap{akaawataan.} @ {} @ {} @ {} @ {} @ {} //
	\glb	{} yú séew {} tsú a- k- wu- i- \rt[²]{tan} -μμL //
	\glc	{}[\pr{DP} \xx{dist} rain {}] also \xx{xpl}- \xx{qual}- \xx{pfv}- \xx{stv}- \rt[²]{handle·w/e} -\xx{var} //
	\gld	{} the rain {} also \rlap{\xx{pfv}.precip·stop} {} {} {} {} {} //
	\glft	‘The rain also let up.’
		//
\endgl
\xe

\ex\label{ex:106-44-talk-about-it}%
\exmn{371.1}%
\begingl
	\glpreamble	Adᴀ′x ada′ yuq!ᴀ′duʟ̣iᴀtk yū′an qoū′wutc. //
	\glpreamble	Aadáx̱ a daa yoo x̱ʼadudli.átk, yú aan ḵu.oowúch. //
	\gla	{} \rlap{Aadáx̱} @ {} {}
		{} a \rlap{daa} @ {} {}
		yoo @ \rlap{x̱ʼadudli.átk,} @ {} @ {} @ {} @ {} @ {} @ {} @ {} +
		{} yú aan \rlap{ḵu.oowúch.} @ {} @ {} @ {} @ {} {} //
	\glb	{} á -dáx̱ {}
		{} a daa {} {}
		yoo= x̱ʼe- du- d- l- i- \rt[¹]{.at} -μH -k
		{} yú aan ḵu- \rt[²]{.uʰ} -μμL -í -ch {} //
	\glc	{}[\pr{PP} \xx{3n} -\xx{abl} {}]
		{}[\pr{PP} \xx{3n·pss} around \·\xx{loc} {}]
		\xx{alt}= mouth- \xx{4h·s}- \xx{mid}- \xx{xtn}- \xx{stv}- \rt[¹]{go·\xx{pl}} -\xx{var} -\xx{rep}
		{}[\pr{DP} \xx{dist} town- \xx{areal}- \rt[²]{own} -\xx{var} -\xx{pss} -\xx{erg} {}] //
	\gld	{} \rlap{that-after} {} {}
		{} its about -on {}
		to·fro \rlap{mouth.\xx{impfv}.people.handle·\xx{pl}.\xx{rep}} {} {} {} {} {} {} {}
		{} those town \rlap{dwellers} {} {} \·of {} {} //
	\glft	‘After that people are talking about it, those townspeople.’
		//
\endgl
\xe

\ex\label{ex:106-45-what-was-he}%
\exmn{371.2}%
\begingl
	\glpreamble	Ye qoq!ā′ỵaqa, “Dasa′yu,”. //
	\glpreamble	Yéi ḵux̱ʼaÿaḵá «\!Daa sáyú?\!»; //
	\gla	Yéi @ \rlap{ḵux̱ʼaÿaḵá} @ {} @ {} @ {} @ {} 
		{} \llap{«\!}Daa \rlap{sáyú?\!».} @ {} @ {} {} //
	\glb	yéi= ḵu- x̱ʼe- ÿ- \rt[¹]{ḵa} -μH 
		{} daa s= á -yú {} //
	\glc	thus= \xx{4h·s}- mouth- \xx{qual}- \rt[¹]{say} -\xx{var}
		{}[\pr{CP} what \xx{q}= \xx{cpl} -\xx{dist} {}] //
	\gld	thus \rlap{people.\xx{impfv}.say} {} {} {} {}
		{} what {} \rlap{it.is} {} {} //
	\glft	‘People are saying “What was he?”;’
		//
\endgl
\xe

The verb \fm{yéi ḵux̱ʼaÿaḵá} ‘thus people say’ in (\lastx) is irregular, with the fourth person human object prefix \fm{ḵu-} instead of the usual fourth person human subject \fm{du-} ‘someone; people’.
This unique irregularity is stable across all dialects of Tlingit and is insensitive to aspect and perhaps also to derivation.
It is probably a conservative feature dating back to Proto-Na-Dene because the exact same irregularity also occurs Dene languages that have no historical contact with Tlingit \parencite[77]{leer:1990a}.

\ex\label{ex:106-46-dont-know}%
\exmn{371.2}%
\begingl
	\glpreamble	ʟēł wudusku′. //
	\glpreamble	tléil wuduskú. //
	\gla	tléil \rlap{wuduskú.} @ {} @ {} @ {} @ {} @ {} //
	\glb	tléil wu- du- d- s- \rt[²]{kuʰ} -μH //
	\glc	\xx{neg} \xx{pfv}- \xx{4h·s}- \xx{mid}- \xx{xtn}- \rt[²]{know} -\xx{var} //
	\gld	not \rlap{\xx{pfv}.people.know} {} {} {} {} {} //
	\glft	‘they don’t know.’
		//
\endgl
\xe

\ex\label{ex:106-47-then-townspeople-say}%
\exmn{371.3}%
\begingl
	\glpreamble	Adᴀ′x yuan-qoū′wu ye hᴀs q!ā′ỵaqa, //
	\glpreamble	Aadáx̱ yú aan ḵu.oowú yéi has x̱ʼaÿaḵá //
	\gla	{} \rlap{Aadáx̱} @ {} {}
		{} yú aan \rlap{ḵu.oowú} @ {} @ {} @ {} {}
		yéi @ has @ \rlap{x̱ʼaÿaḵá} @ {} @ {} @ {} //
	\glb	{} á -dáx̱ {}
		{} yú aan ḵu- \rt[²]{.uʰ} -μμL -í {}
		yéi= has= x̱ʼe- ÿ- \rt[¹]{ḵa} -μH //
	\glc	{}[\pr{PP} \xx{3n} -\xx{abl} {}]
		{}[\pr{DP} \xx{dist} town \xx{areal}- \rt[²]{own} -\xx{var} -\xx{pss} {}]
		thus= \xx{plh}= mouth- \xx{qual}- \rt[¹]{say} -\xx{var} //
	\gld	{} \rlap{that-after} {} {}
		{} those town \rlap{dwellers} {} {} -of {}
		thus they \rlap{\xx{impfv}.say} {} {} {} //
	\glft	‘So then the townspeople say’
		//
\endgl
\xe


\ex\label{ex:106-48-doncha-know}%
\exmn{371.3}%
\begingl
	\glpreamble	“ʟē′gîł ỵī-sᴀku îtckᴀqā′wu ỵᴀ′di ayu′.” //
	\glpreamble	«\!Tleigíl ÿisakú Eechká Ḵáawu ÿádi áyú.\!» //
	\gla	«\!\rlap{Tleigíl} @ {} @ {}
		\rlap{ÿisakú} @ {} @ {} @ {} @ {}
		{} \rlap{Eechká} @ {} \rlap{Ḵáawu} @ {} \rlap{ÿádi} @ {} \rlap{áyú.\!»} @ {} {} //
	\glb	\pqp{}tle= gí =l 
		ÿu- i- s- \rt[²]{kuʰ} -μH
		{} eech- ká ḵáaʷ -í ÿát -í á -yú {} //
	\glc	\pqp{}then= \xx{yn} =\xx{neg}
		\xx{pfv}- \xx{2sg·s}- \xx{xtn}- \rt[²]{know} -\xx{var}
		{}[\pr{CP} reef- \xx{hsfc} man -\xx{pss} child -\xx{pss} \xx{cpl} -\xx{dist} {}] //
	\gld	\pqp{}then \rlap{don’t} {}
		\rlap{\xx{pfv}.you·\xx{sg}.know} {} {} {} {}
		{} reef- top man -of child -of \rlap{it.is} {} {} //
	\glft	‘“Well don’t you know, it’s a child of Man Of The Reef.”’
		//
\endgl
\xe

The clitic particle cluster \fm{tleigíl} in (\lastx) is remarkable.
It has a relatively straightforward internal structure with the particle \fm{tle} ‘then, just, simply’ \parencite[32]{leer:1978b}, the polar yes/no question particle \fm{gí}, and the negative clitic \fm{l} ‘not’.
But despite being easily decomposed and interpreted, this particular combination is not otherwise known in Tlingit.
The typical form in modern Tlingit would be \fm{Tléil gí yisakú?} ‘Don’t you know?’ with the polar yes/no question particle \fm{gí} appearing after the negative particle \fm{tléil} like all other second position particles in the language.
It provides some support for the hypothesis that the negative particle \fm{tléil} is derived from \fm{tle} and \fm{l} since it indicates that they can occur together with another element in between them.
\citeauthor{swanton:1909}’s transcription \orth{ʟē′gîł} does not indicate whether the first syllable has low or high tone; his \orth{′} only reliably reflects stress.
The representation with low tone as \fm{tleigíl} [\ipa{ˈtɬʰèː.kíɬ}] is predicted from the Tongass allomorph \fm{tleì} [\ipa{tɬʰeʰ}] with a fading vowel which regularly corresponds to a long low tone vowel in Northern Tlingit.
But if the connection with \fm{tléil} is appropriate then the form \fm{tléigíl} [\ipa{ˈtɬʰéː.kíɬ}] is also plausible, though it would require a Tongass counterpart \fm[*]{tleigil} rather than \fm[*]{tleìgil}.
