%!TEX root = ../swanton-texts.tex
%%
%% 91. Future Little Slave (267–279)
%%

\resetexcnt
\chapter{Goox̱kʼ Sákw: Future Little Slave and Láx̱ʼ Yéet: Heron’s Son}\label{ch:91-future-little-slave}

This narrative was told to \citeauthor{swanton:1909} by \fm{Deikeenaakʼw} in Sitka in 1904.
In the original publication it is number 91, running from page 267 to 279 and totalling 164 lines of glossed transcription.
\citeauthor{swanton:1909}’s original title was “The shaman who went into the fire, and the heron’s son”.
As implied by the title, this narrative encompasses two stories, and \citeauthor{swanton:1909} indicates the separation with a break in between them.
It is clear from the transcription that the stories were delivered as a single continuous narrative performance: the second story starts with the particle \fm{x̱ách} ‘actually’ that presupposes some prior information.
Because of this the two stories have been maintained as a single unit with a division between them just as \citeauthor{swanton:1909} did in his original presentation.
The first story is well known in Tlingit by the name \fm{Goox̱kʼ Sákw} ‘Future Little Slave’ which is one name of the main character; this story also goes by the English name “One Horned Mountain Goat” for unclear reasons (Dauenhauer p.c.\ 2010).
The second story has not been documented anywhere else in the Tlingit literature.
Its name is given as \fm{Láx̱ʼ Yéet} ‘Heron’s Son’ following \citeauthor{swanton:1909}.

The \fm{Goox̱kʼ Sákw} story was recorded in English in the late 19th century by \citeauthor{emmons:1907} from an unnamed elderly man of Hoonah \parencite[333–334]{emmons:1907}.
In this narrative \citeauthor{emmons:1907} gives the name as \orth{Kokesᴀk} which can be identified as an attempt to transcribe \fm{Goox̱kʼ Sákw}.
This story was also recorded in Tlingit from \fm{Kaatyé} David Ka\-da\-shan of Hoonah by \fm{Ḵeixwnéi} Nora Marks Dauenhauer.
There is an untranslated 27-page typescript \parencite{dauenhauer:1973} which was based on an audio recording that has since been lost (Dauenhauer p.c.\ 2010).
One narrative recorded in English by Viola Garfield from an unknown consultant in the Saxman area in the 1930s seems to reflect the same story although the ending and several details are different.
This story was given as a description of a lower figure on the \fm{Kaatsʼ} totem pole in Saxman, and \fm{Goox̱kʼ Sákw} is said in this case to be a descendant of \fm{Kaatsʼ} \parencite[33–35]{garfield-forrest:1948}.

\FIXME{Discuss the second story.
The title \fm{Láx̱ʼ Yeet} ‘Heron’s Son’ is used because the mythological father of the protagonist is explicitly identified as a \fm{láx̱ʼ} ‘heron’, but he has only a limited role in the story and might replaced by some other entity or be entirely be absent in other versions of the story.
The narrative also features a creature called \fm{héen taaÿéeshi} which also appears in the story called \fm{Héen Taaÿéeshi} (\citeauthor{swanton:1909}’s “The Hīn-Taỵī′cî”) told by \fm{Léekʼ} the mother of \fm{Ḵaadashaan} in Wrangell \parencite[217–219]{swanton:1909}.
This other story seems to be entirely unrelated except for the creature.
Another instance of the same creature is \citeauthor{olson:1967}’s “hĭntagi·eci′h” recorded from “JW” (still unidentified, ask Langdon), but this appearance is only as a background painting in the sea lion’s house in the \fm{Naatsilanéi} story.}

\FIXME{So far no other similar stories have turned up in the Tlingit texts, nor those by Southern Tutchone and Tagish narrators recorded by \textcites{mcclellan-cruikshank:2007a}{mcclellan-cruikshank:2007b}.
There are no obviously similar stories featuring a heron in \cite{boas:2002}, but there may be similar plots that have not yet been noted.
Still need to check Tahltan, Haida, Nisg̱aʼa, and Coast Tsimshian collections.
If nothing shows up there then look further afield.}

\FIXME{End of narrative has several parts which can be identified as separate stories, in particular Mosquito.}

\clearpage
\begin{pairs}
\begin{Leftside}
\beginnumbering
\pstart
\noindent
\snum{1}Du shagóoni ḵutx̱ shoowaxeex, yú atkʼátskʼu.
\snum{2}Du káak ḵu.aa at sʼaatéex̱ sitee.
\snum{3}Chʼa goot sá nagútch yú atkʼátskʼu chooneit tin, yú dáaḵ.
\snum{4}Deisgwách tlʼag̱áa ligéi.
\snum{5}Aan wuligáasʼ.
\snum{6}Ḵul\-yéis, at wujaaḵ shóot nagútch du káak.
\snum{7}Shaa ÿadaadáx̱ ÿeiḵ at koojeilch aatlein.
\pend
%2
\pstart
\snum{8}Wáa nanée sáwé woogoot tsu at lʼóon.
\snum{9}Yú neil ḵu.aa at kageidéech shawlihík yú kax̱ÿee.
\snum{10}Ha wáa sdagaa ashikʼáan du x̱úx̱ kéilkʼátskʼu.
\snum{11}Gánde nagóot áwé du káak shát, yá kax̱ÿeix̱ dixwásʼi at kageidí shóotx̱ aawalʼéexʼ, yú taayákʼw, chʼa du wásh tóog̱aa.
\snum{12}«\!Ha goosú tlax̱ yéi ÿakoo\-géiÿi át?\!»
\snum{13}Át awdlig̱én du káak shátch.
\snum{14}Tle gwáayá l a shoowú aa.
\snum{15}«\!Wa.éich gwáawégé yéi yisinee?\!»
\snum{16}tle yóo aÿawsiḵaa du x̱úx̱ kéil\-kʼátskʼu.
\snum{17}Chʼa aadé g̱ax̱yéide áwé yéi aÿawsiḵaa «\!Tléikʼ\!».
\snum{19}Chʼu tle a wásh tóode wooshee du x̱úx̱ kéilkʼ.
\snum{20}A washtú akadlaakw.
\snum{21}Du x̱ʼéináx̱ shí yéi koowóox̱ʼ.
\snum{22}Chʼa aadé g̱ax̱yéide áwé du káak ḵóogu tóot awsi.ín.
\snum{23}A ÿíkdáx̱ kei aawatée du káak ÿaÿéinakʼu.
\snum{24}Du káak ḵwa uyéx̱.
\pend
%3
\pstart
\snum{25}Tle g̱unayéi uwagút atgutóode.
\snum{26}Héen yaax̱í daaḵ uwagút;
\snum{27}wé ÿaÿéinaa, xákw káa aan daaḵ uwagút.
\snum{28}Akatʼéix̱ʼ x̱áat yáx̱.
\snum{29}Kút aawasʼít yú héen yaax̱xʼ.
\snum{30}A káa uwax̱éi yú kút.
\snum{31}Du jooní áyú yéi ÿatee.
\snum{32}Yéi daaÿaduḵá «\!Héen ÿíx̱ nasxʼaak\!».
\snum{33}X̱ách du tuyéigi áséyú yéi ash eet toowditán.
\pend
%4
\pstart
\snum{34}Du eetéexʼ ÿeiḵ uwagút du káak.
\snum{35}Ax̱ʼeiwóosʼ du shát «\!Goosú hú, ax̱ kéilkʼ?\!».
\snum{36}«\!Wéi\-de áwé chooneit awli.aat.\!»
\pend
%5
\pstart
\snum{37}Naanaaxʼ kei góot áwé tsu aa uwasʼít, wé kút sákw.
\snum{38}«\!Goox̱kʼ Sákw\!» yóo duwasáakw yú ḵáa.
\snum{39}Nasʼgadooshú aawasʼít yu kút.\linebreak
\snum{40}Hóochʼi aaÿí káxʼ áwé óox̱ yéik uwatsʼáḵ.
\snum{41}Cha chʼa aag̱áa áwé héen ÿee yaa aawatee du ÿaÿéinaÿi.
\snum{42}Héen ÿíkt wudlitsís.
\snum{43}Chʼu tle l sh daax̱ awu̬danoogú̥ áwé yú g̱ílʼ ÿát wudzigít.
\snum{44}Áx̱ wulixáatʼ yú g̱ílʼ yá.
\snum{45}Áxʼ áwé du x̱ʼétx̱ at wuskwéiÿ̃ch ldakát át, kéi\-dladi, chʼáakʼ.
\snum{46}Ḵutx̱ shunaxíxch du eetx̱ ḵuyéik g̱a.ádín, yú chʼáakʼ, kéidladi, ldakát á.
\pend
%6
\pstart
\snum{47}Du eeg̱áa ḵoowashee du káak.
\snum{48}Nasʼga\-dooshú ux̱éi, a káx̱ ḵoowashee yú kút du kéilkʼ a ji.eetí yú héen yaax̱xʼ.
\snum{49}Chʼu tle ldakát aa wusiteen yú kútxʼ, du kéilkʼ áx̱ kei unax̱éini ÿé.
\snum{50}Yú héen ÿíkt awdlig̱én du káak.
\snum{51}A ÿíx̱ uwaxʼák yú x̱áat.
\snum{52}Áa uwax̱éi yú kút taÿee.
\snum{53}Atx̱ áwé ḵul.áx̱sʼ.
\snum{54}Tsʼootaat áwé aawa.áx̱ x̱éjaa kayéik.
\snum{55}Chʼa yú g̱ílʼ yakatʼóode áwé aawa.áx̱.
\snum{56}Chʼu tle a kʼiÿeet uwagút.
\snum{57}Chʼu l ash utéenx̱ ash wujeeÿí áyú ash eet x̱ʼeiwatán.
\snum{58}«\!Ḵwáaḵ ÿéináx̱ ax̱ taÿeet eeÿagút, káak.\!»
\snum{59}Tlax̱ wáa sá aÿaawag̱aax̱ du kéilkʼ, hóoch.
\snum{60}«\!He xʼwán gukshínáx̱ kei gú.\!»
\snum{61}Goox̱kʼ Ságuch yéi uwasáa.
\snum{62}Du káakch x̱ʼeiwawóosʼ
\snum{63}
\snum{64}
\snum{65}
\snum{66}yóo aÿawsiḵaa du káak.
\snum{67}Aatlein tatóok áyú hít ÿáanáx̱ koogéi.
\pend
%7
\pstart
\snum{68}Chʼu tle ḵuyéik woo.aat du káak tin.
\snum{69}A ÿee neil has .áat ash x̱ʼax̱éich du káak.
\snum{70}Yan ashukaawajaa du káak
\snum{71}«\!Ganaltáa x̱áan gug̱a\-shéex yú yéik Neex̱á, líl tlax̱ yéi x̱at kugánjiḵ.
\snum{72}Chʼu yéi x̱at kugéikʼi xʼwán léetʼ tóot áyú x̱at \{nag̱eiÿag̱éxʼt\}.\!»
\snum{73}A yáx̱ áwé awsinei;
\snum{74}ash een ganaltáat wu\-shee\-xí léetʼ tóode ash woog̱éixʼ.
\snum{75}Tle áxʼ áwé ḵux̱ udagútch, yú léetʼ tóoxʼ.
\snum{76}Aatlein ḵáax̱ nasteech.
\pend
%8
\pstart
\snum{77}Aadáx̱ xáanaa áwé xʼéig̱aa yú áxʼ ax̱ʼa\-kaawanáa du káak.
\snum{78}«\!Haadé yú atshéeÿi.\!»
\snum{79}Tléil áwé át akawdagaan yú g̱ílʼ, jánu̬wu yú tatóok ÿeedé adultíni sákw.
\snum{80}Yan ḵéi áwé du sáḵsi yóo awsinee;
\snum{81}tle ḵútx̱ shoowaxeex yú jánu̬wu.
\snum{82}Yú tatóok ÿee, tle shaawahík.
\snum{83}Du káak áwé áa ajikaawaḵaa
\snum{84}«\!Daatx̱ keeda\-sʼélʼ\!».
\snum{85}Aÿashí Nee\-x̱á Tlein.
\snum{86}Du tóotx̱ kei ishéex du káak yan yoo aÿasiḵéik
\snum{87}«\!X̱áan ganaltáat ishíxni xʼwán a kát tse iseixʼáaḵw aax̱ daaḵ x̱at isháadi léetʼ tóode.\!»
\snum{88}Ash nashátch.
\pend
%9
\pstart
\snum{89}Yan kadulgáa yú tatóok ÿeexʼ yú jánwu kageidí, du shátg̱aa ash kaawaḵaa.
\snum{90}Aan kei uwa.át du shát, tatóok taÿeexʼ.
\snum{91}«\!Wé a kát at gatus.ée ḵákw taÿee xʼwán tsú gataan.
\snum{92}A yík ádi katʼéx̱ʼ\!» du káak yéi aÿawsiḵaa, «\!i shát x̱ʼéis\!».
\snum{93}A kʼóolʼi aax̱ aawatee yú shaawát du yakg̱wahéiÿág̱u.
\snum{94}A yáx̱ áwé tléil yéi oonéx̱ch yú shaawát, a taaÿí ax̱á yú jánwu.
\snum{95}Tléil x̱ʼéi akootán nooch yú taay.
\snum{96}A xʼaaxʼ, du washtú akawudlaagóoch áwé yéi akwsayéin.
\pend
%10
\pstart
\snum{97}Yéi adaaÿaḵá du káak
\snum{98}«\!I toowú xʼwán shatʼíxʼ Neex̱á neil gútni.\!»
\snum{99}Xáanaa áwé yóoxʼ áxʼ akaawanáa du káak.
\snum{100}Tléil áwé át akawdagaan yú g̱ílʼ.
\snum{101}Xóots hít yát uwa.át, yú tatóok x̱ʼawool.
\snum{102}Tle yú kínde sixát.
\snum{103}Kei akaawashée yú yéik, du káakch.
\snum{104}Neildé naa\-.áat.
\snum{105}Chʼa a katʼóot káxʼ áwé yéi ash yawsiḵaa
\snum{106}«\!Aag̱áa ḵunatéesʼ.
\snum{107}Neildé ÿaa ax̱ʼakg̱wasáa.\!»
\snum{108}Wáa nanée sáwé neil ÿaw\-dzi.áa yú xóots.
\snum{109}Du gúk káxʼ x̱ʼwáalʼ wudu\-wadúxʼu aa yáx̱ áwé ÿatee.
\snum{110}Aanáx̱ áwé awditlʼékw du shát.
\snum{111}Tle wóoshdáx̱ woolʼéexʼ.
\snum{112}Yú \{taaÿí x̱ʼáax̱ʼ\} du washtú akaawudlaa\-gúch áwé yéi awliyéx̱.
\snum{113}Galtáat ash een wujixeex du yéigi tsú.
\snum{114}Áwé chʼa á áyú;
\snum{115}xóots jée akoolx̱éitlʼ áwé ganaltáax̱ kaawagaan Goox̱kʼ Sákw.
\pend
%11
\pstart
\snum{116}Chʼu tle áwé wushikélʼ yú tatóok,
\snum{117}chʼu yé ḵáa áwé du doogú tóode wujixeexi yé.
\snum{118}Yú dusxoogu át áyú yéi ḵoowanook.
\snum{119}Yú íx̱tʼ kaawugaaních áyú yéi wudzigeet.
\snum{120}Hóochʼ áyú, tle kaawagaan yú íx̱tʼ ḵa du káak.
\pend
\endnumbering
\end{Leftside}
%%
%% Column break.
%%
\begin{Rightside}
\beginnumbering
\pstart
\noindent
\snum{1}His family became extinct, that young boy.
\snum{2}His uncle however was an expert hunter.
\snum{3}He always went around wherever, that boy, with his bow and arrow, up inland.
\snum{4}Eventually he is tall.
\snum{5}He relocated with him.
\snum{6}Having become autumn, he would go around in order to kill things, his uncle.
\snum{7}From around the face of the mountains he would carry lots of things to the beach.
\pend
%2
\pstart
\snum{8}At some point he again went hunting things.
\snum{9}Inside however, sides of meat filled the rafters.
\snum{10}Well how indeed does she hate her husband’s young nephew.
\snum{11}Her having gone out, his uncle’s wife, he broke off from the end of the side of meat that hangs on the rafters a little piece of fat, just for the inside of his cheek.
\snum{12}\qqk{}“Well where is this thing that is so very big?”
\snum{13}She looked around, his uncle’s wife.
\snum{14}Then really there isn’t some of the end of it.
\snum{15}\qqk{}“Was it really you who did this?”
\snum{16}so she said to her husband’s young nephew.
\snum{17}Just that way, it was crying that he said to her “no”.
\snum{18}Just then she felt inside his cheek, her husband’s nephew.
\snum{19}\qqk{}“How have you not gone around on the face of the mountain?”
\snum{20}She scratches the inside of his cheek.
\snum{21}Blood is wide along his mouth.
\snum{22}\!Just that way, it was crying that he pulled his uncle’s box to himself.
\snum{23}From within it he picked it up, his uncle’s little whetstone.
\snum{24}His uncle however is absent.
\pend
%3
\pstart
\snum{25}So he began going to the forest.
\snum{26}He went inland on a riverbank;
\snum{27}the whetstone, he went out on top of a sandbar with it.
\snum{28}He hammers it like a salmon.
\snum{29}He made a nest by that riverbank.
\snum{30}He spent the night on it, that nest.
\snum{31}It is his dream that is thus.
\snum{32}Someone says to him “Make it swim within the river”.
\snum{33}Actually it is apparently his inner spirit that thought this to him.
\pend
%4
\pstart
\snum{34}In his absence he came to the beach, his uncle.
\snum{35}He asks his wife “Where is my nephew?”.
\snum{36}\qqk{}“It is that way that he made arrows go.”
\pend
%5
\pstart
\snum{37}Having gone upstream, he again made some of them, those future nests.
\snum{38}People call him “Future Little Slave”, that man.
\snum{39}He made eight, those nests.
\snum{40}It was on top of the last one that a spirit came to him.
\snum{41}So it was just after that that he was putting it below the water, his whetstone.
\snum{42}It floated within the water.
\snum{43}Just then it was without feeling himself that he fell to the face of the cliff.
\snum{44}He was suspended on it, that cliff face.
\snum{45}It was there that everything would learn things from his mouth, seagulls, eagles.
\snum{46}They died off whenever those spirits went away from him, those eagles, seagulls, all of them.
\pend
%6
\pstart
\snum{47}He searched for him, his uncle.
\snum{48}Having overnighted eight times, he discovered it, that nest, his nephew’s handiwork, by that riverside.
\snum{49}Then he saw all of them, those nests, the places where his nephew was spending the night.
\snum{50}He looked within that river, his uncle.
\snum{51}It swam there, that salmon.
\snum{52}He spent the night there, underneath that nest.
\snum{53}After that he listens.
\snum{54}It was in the morning that he heard it, the sound of beaters.
\snum{55}It was partway up the face of the cliff that he heard it.
\snum{56}He went below the base of it.
\snum{57}It was having thought of him that he could not see him that he spoke to him.
\snum{58}\qqk{}“You came below me through the wrong way, uncle.”
\snum{59}How very much he pitied his nephew, he did.
\snum{60}\qqk{}“Well then come up through the corner.”
\snum{61}Future Little Slave named him thus.
\snum{62}His uncle asked him
\snum{63}
\snum{64}He did not tell him more than the scratching of the inside of his cheek.
\snum{65}
\snum{66}so he said to his maternal uncle.
\snum{67}It is a big cave that is bigger than a house.
\pend
%6
\pstart
\snum{68}So then animal spirits went with his uncle.
\snum{69}Within it, them having gone inside, he is beating rhythm for him, his uncle.
\snum{70}He instructed his uncle
\snum{71}\qqk{}“When it runs with me in the middle of the fire, that spirit Neex̱á, don’t let me be very burnt.
\snum{72}While I am getting small \{throw\} me into a basket.”
\snum{73}It was like that that he did so;
\snum{74}when it ran into the middle of the fire with him he threw him into the basket.
\snum{75}Then it was there that he would come back, inside that basket.
\snum{76}He would become a big man.
\pend
%8
\pstart
\snum{77}After that it was in the evening that he directly ordered him out at that place to speak, his uncle.
\snum{78}\qqk{}“Come here, those singers.”
\snum{79}No light shone there, that cliff, for mountain goats watching below that cave.
\snum{80}Them having sat down, he did his bow and arrow;
\snum{81}then they all died off, those mountain goats.
\snum{82}At the bottom of that cave, then it was full.
\snum{83}It was his uncle that he instructed there
\snum{84}\qqk{}“Tear it off from around yourself”.
\snum{85}He sings Great Neex̱á.
\snum{86}Having run up from inside him, he repeatedly says to his uncle
\snum{87}\qqk{}“If it runs into the middle of the fire with me, beware lest you forget to grab me off out of it into the basket.”
\snum{88}He repeatedly grabbed him.
\pend
%9
\pstart
\snum{89}Having finished putting them away in the bottom of that cave, those mountain goat sides of meat, he told him to go for his wife.
\snum{90}He went up with her, his wife, beneath the cave.
\snum{91}\qqk{}“Also pick up that below-basket that we cook things for ourselves on.
\snum{92}Pound the things within it” he said to his uncle “for your wife to eat”.
\snum{93}Her tailbone, it took it from there, that woman, his spirit.
\snum{94}It was as if she was never satiated, that woman, eating its fat, that mountain goat.
\snum{95}She was never satisfied by the taste of it, that fat.
\snum{96}On the point of it, it was because she had scratched the inside of his cheek that he makes it happen.
\pend
%10
\pstart
\snum{97}So he says to his uncle
\snum{98}\qqk{}“Steady your spirit if \fm{Neex̱á} comes inside.”
\snum{99}It was in the evening that he sent him out there, his uncle.
\snum{100}No light shone there, that cliff.
\snum{101}Brown bears came up to the front of the house, that cave entrance.
\snum{102}Then they just extend out upwards.
\snum{103}He sang up that spirit, his uncle.
\snum{104}At last they came inside.
\snum{105}It was just as it was partway up there that he said to him
\snum{106}\qqk{}“Look out for it.
\snum{107}It will be calling them inside.”
\snum{108}At some point it sticks its face inside, that brown bear.
\snum{109}It was like one that eagle down had been tied on his ears.
\snum{110}It was during this that she recoiled from it, his wife.
\snum{111}Then she broke apart.
\snum{112}That \{unidentified\}, he made it because she scratched the inside of his mouth.
\snum{113}Into the fire it ran with him, his spirit, also.
\snum{114}So that was it;
\snum{115}it was while he feared possession of the bear that he burned in the middle of the fire, Future Little Slave.
\pend
%11
\pstart
\snum{116}It was then that it turned to ash, that cave;
\snum{117}it was an ordinary man that way that he ran inside his skin.
\snum{118}It is those things that people are drying that acted thus.
\snum{119}It was because that shaman was burned that it happened so.
\snum{120}That’s it, they burned, that shaman and his uncle.
\pend
\endnumbering
\end{Rightside}
\end{pairs}
\Columns

% This break reflects the one in Swanton’s original publication.
% The \fancybreak command is from memoir.cls, see memman.pdf §6.7 Fancy anonymous breaks.
\fancybreak{\rule{16em}{0.125pt}}

\begin{pairs}
\begin{Leftside}
\beginnumbering
\setcounter{pstartL}{12}
%12
\pstart
\noindent
\snum{121}X̱ách yú aax̱ yéi s ḵuwanoogu yé aan ḵu.aa ásiyú tléil kei wudaḵáat.
\snum{122}Déix̱ ÿatee yú gáng̱aa dei.
\snum{123}Yáatʼaa ÿíde awusheexí tléil ÿeiḵ ugootch.
\snum{124}Leengít ḵa yáatʼaa ÿíde awugoodí tléil yei dustínch;
\snum{125}ḵa awuḵoox̱ú tléil yei dustínch.
\snum{126}Tléil aanx̱ uḵoox̱.
\snum{127}Tsu tléixʼ táakw áwé tléil wudaḵáat wé aan.
\snum{128}Chʼu tle dáx̱anáx̱ áwé áa sh wudzineix̱, wé shaawát yú aanxʼ, yú shaawát du sée tin.
\snum{129}Chʼa l sh daa yankáx̱ has toondatáan áwé aan woo.aat, du sée.
\snum{130}«\!Aadóo sgí ḵaa sée ag̱ashaa\!» yóo áwé x̱ʼayaḵá.
\snum{131}Yú tʼéexʼ shukát daaḵ nahéin láx̱ʼ;
\snum{132}á áwé hasdu eet x̱ʼeiwatán.
\snum{133}«\!Wáa sá x̱at ÿatee, x̱át?\!»
\snum{134}«\!Ha daat een sá?\!» yóo ash yawsiḵaa yú shaawát.
\snum{135}«\!Kanéiḵ x̱áan daaḵ akg̱ajóoxun a tóo yan x̱ahánch;
\snum{136}aan haa áwé.\!»
\snum{137}«\!Ha neildé haa een na.ádi\!»
\snum{138}yóo ash yawsiḵaa yú shaawátch.
\snum{139}Tle ash uwasháa yú láx̱ʼch.
\snum{140}Tle du eet yáts jeewaháa.
\snum{141}Chʼu tle ḵoowdzitee.
\snum{142}Ḵáa áyú ḵoowdzitee.
\snum{143}Deisgwách ÿaa nalgéin.
\snum{144}Yéi adaaÿaḵá du shát yú láx̱ʼch
\snum{145}«\!Wáa sá woonee i x̱oonxʼí?\!»
\snum{146}«\!Tle gáng̱aa awugoodí tléil yeiḵ ugootch.\!»
\pend
%13
\pstart
\snum{147}Chʼu tle yéi kawulgéiÿi áwé taashukáade aksanúkch.
\snum{148}Tʼéexʼ tóoxʼ áwé ashóoch nuch.
\snum{149}Deisgwách at tʼúkt yú atkʼátskʼu.
\snum{150}At\-choo\-n\-eit anal.átch.
\snum{151}Oog̱aajaag̱i átg̱aa ḵut woogoot, hasdu éesh, yú atkʼátskʼu:
\snum{152}«\!De chʼa x̱át áwé ax̱ ÿéetkʼ.\!»
\snum{153}Yóo aÿawsiḵaa du shát
\snum{154}«\!De ÿináḵ kuḵagóot.\!»
\snum{155}Deisgwách héenx̱ yei ishxíxch yú atkʼátskʼu.
\snum{156}Chʼa yéi daaḵ g̱asheech ash g̱ajág̱ín.
\pend
%14
\pstart
\snum{157}Wáa nanée sáwé chooneit tin woogoot.
\snum{158}Eiḵt wugoodí áwé áakʼw kát wuxʼaagí héen taaÿéeshi.
\snum{159}Aax̱ aawasháat.
\snum{160}Yá du jín áwé tle wóoshdáx̱ aawakʼúts a wánich.
\snum{161}Áakʼw káxʼ áwé ÿaa anaswát.
\snum{162}A x̱ʼéix̱ at téexʼ nuch chooneit aag̱áa ala.át g̱anúk.
\pend
%15
\pstart
\snum{163}Wáa nanée sáwé yéi aÿawsiḵaa du tláa
\snum{164}«\!Gáng̱aa na.aadí.\!»
\snum{165}«\!Ÿee káak hás tléil ÿeiḵ aa ugútch.\!»
\snum{166}Tsʼootaat áwé tʼáa kát wujixeex.
\snum{167}Taÿees aawasháat.
\snum{168}Chʼu tle yá tléixʼ yateeÿi aa ÿík áwé daaḵ anashíx.
\snum{169}Yá dei ÿíknáx̱ áwé nḁlishóo ḵaa tlʼeiḵ.
\snum{170}«\!Haan\-dé i tlʼeiḵ\!»
\snum{171}yóo ash yawsiḵaa.
\snum{172}Tle aanáx̱ tlʼeḵwulitsaag̱i yé áwé aax̱ kei aawax̱útʼ.
\snum{173}Té kát áwé ashawlix̱útʼ.
\snum{174}Yú aax̱ kei aawax̱útʼi yé áwé ḵaa xáagi yéi wudzig̱áat aadé ḵu.eeni ÿé.
\snum{175}Chʼu tle a kaax̱ aseiwasʼóow yú sʼoow taÿeesch.
\snum{176}Íḵde aawatee du tláaxʼ.
\snum{177}Anáx̱ áwé neil aawag̱íxʼ.
\snum{178}Ḵa du léelkʼw, tle has aÿadaakaháan.
\snum{179}Kwás tin has aÿatʼóosʼ gan\-altáaxʼ.
\snum{180}Aadé has danóogu yé yáx̱ áwé asataan áa.
\snum{181}Wáa nanée sáwé tsu chooneit awli.aat yú ÿaa anaswát héen taÿéeshi x̱ánde.
\snum{182}Tsáa chʼas duháni yáanáx̱ kulagéiÿi áwé aawatʼúk a shakéenáx̱, yú ÿaa anaswát héen taÿéeshi.
\snum{183}Shunaaÿeit daaḵ aawatée.
\snum{184}Chʼu tle atgoowú náax̱ aawatee.
\snum{185}A wán tlax̱ wáa sá yakʼátsʼ.
\pend
%16
\pstart
\snum{186}Aant góot áwé tsu tʼáa kát wujixeex.\linebreak
\snum{187}Yáade wooshoowu aa ÿíkde taÿees aawa\-sháat.
\snum{188}Yóot dáag̱i daaḵ ishéex áwé awsiteen ḵaa shaaÿí dei ÿíknáx̱ shanaashóo.
\snum{189}«\!Kínde i waaḵ, Kushaḵʼéitʼkw.\!»
\snum{190}Yú chʼa gáxde wuduwalʼéexʼi yé yáx̱ áwé woonee, yú ḵaa shaaÿí.
\snum{191}Tle kaax̱ aseiwasʼóow, gáxwde yoo anaskítʼgi.
\snum{192}Ḵaa xáagi áyú yei wudzig̱áat.
\snum{193}Du x̱án aawasháat a shaaÿí.
\snum{194}Íḵde tsu, neil aawag̱íxʼ.
\snum{195}Háatlʼi tin áwé ayaawatʼúsʼ.
\snum{196}Aadé a jeeÿeet has sh danoogu ÿé yáx̱ áyú has adaané.
\snum{197}Atx̱ áwé chʼu tle chooneit kei al.átch.
\snum{198}Ldakát át du tláa hás x̱ʼéis neil\-dé ÿaa aka̬gajélch.
\snum{199}Láx̱ʼ du ÿéet áwé yéi ḵu\-wanóok, aag̱áa woosoo yú sháa.
\snum{200}Wáa nanée sá ax̱ʼeiwawóosʼ du tláa
\snum{201}«\!Goonáx̱ aadé wooḵoox̱u ÿé sá ax̱ káak hás tléil x̱áax̱ awuḵoox̱?\!»
\snum{202}«\!Wéide wuḵoox̱ú áwé, ÿítkʼi\!»\linebreak
\snum{203}yóo aÿawsiḵaa.
\snum{204}A niyaadé áwé woogoot chooneit tin.
\snum{205}A kináa daak uwagút a waḵx̱ʼawoolí, náaḵw.
\snum{206}Kínde xʼisʼtóo \{n\} áwé \{a\} tliyéi a xʼisʼtóot awdlig̱én.
\snum{207}Aag̱áa ḵux̱ wujixeex yú awlisíni kʼoodásʼ, héen taa\-ÿéeshi kʼoodásʼ.
\snum{208}Áa ḵux̱ wudagoodí áwé tle té aadé daaḵ aawag̱íxʼ, a xʼisʼshantóode.
\snum{209}Káx̱ awdig̱éixʼ yú héen taaÿéeshi kʼoodásʼ.
\snum{210}Chʼu tle a xʼisʼshantóo wujixeex.
\snum{211}A tóot áwé yaawakʼút tlʼaadéin ḵa héide yú náaḵw xʼisʼshantóot.
\snum{212}Chʼa yéi kwdayáatlʼ áwé yei akanaxásh du wánch.
\snum{213}Wáa nanée sáwé a xʼisʼshantukaxwéix̱i, a ká akawlixaash.
\snum{214}Ajáaḵ áwé a waḵx̱ʼawoolínáx̱ daaḵ uwaxʼák.
\snum{215}A tʼikát áwé sh wudlihaash.
\snum{216}Xákw káa woogoot.
\snum{217}Kaax̱ kei awu̬ditée.
\snum{218}Atgoowú náax̱ aÿaawax̱ích.
\snum{219}Tsu tsʼas du tláa x̱ánt uwagút.
\snum{220}Hasdu eeg̱ayáat awsigúḵ yú at tlʼeig̱í tlénxʼ.
\snum{221}Yéi kwdayáatlʼ yáx̱ ÿawdudlixásh.
\snum{222}De chʼa á áyú yú aantḵeiní ḵutx̱ ashuwlixeex.
\pend
%17
\pstart
\snum{223}Át áyú tsu chooneit awli.aat.
\snum{224}A káx̱ woogoot kutsʼeen a waḵx̱ʼawoolí.
\snum{225}A lʼeet aanáx̱ kei awlitsáḵ.
\snum{226}Chʼakʼát aant uwagút.
\snum{227}Chʼu tsʼootaat l yéil du.áx̱ji áwé aadé g̱unéi uwagút.
\snum{228}Aawa.aax̱ du kʼoodásʼi, héen taa\-ÿéeshi kʼoodásʼ.
\snum{229}A yatʼéit góot áwé káx̱ awditee;
\snum{230}aÿalaneisʼ áwé a wán.
\snum{231}A tóoxʼ nagóot áwé áa kei uwagút yú waḵx̱ʼawool.
\snum{232}Chʼu tle té áwé a dix̱ʼkát aawagúḵ;
\snum{233}áa kaawasʼúnk, yú shaa wáa sá wóoshdáx̱ g̱aax̱\-dag̱ádín.
\snum{234}Anáx̱ yóode gug̱axʼáagich áyú chʼa yá té yát áwé sh wudlihaash a yaÿeexʼ, anáx̱ daak xʼákni ÿís.
\snum{235}Anáx̱ daak xʼáak áwé du eet ḵʼaloowatsáḵ.
\snum{236}Ash niyaanáx̱ yaa\-wa\-xʼaak.
\snum{237}Du lʼeet ash káa ayanax̱lax̱óotʼ.
\snum{238}Á\-sí\-yú tle kindewáneen sh wudlihaash, du lʼeet yaÿeexʼ.
\snum{239}Ash káa ayanax̱lax̱óotʼ.
\snum{240}Wooshx̱ kandulxáshi aa yáx̱ áyú yéide du lʼeet ash káa yan yoo alx̱útʼgu.
\snum{241}A kagoowúx̱ nastée áwé a ḵatlyát wudlitsís, kadagáax̱ ash jeeÿeet.
\snum{242}Ldakát yei ash kanalxásh.
\snum{243}Atx̱ áwé yan u\-waxʼák.
\snum{244}Tsu atgoowú náax̱ aawatee.
\snum{245}Ḵei\-na.áa áwé hasdu eig̱ayáa wulihaash a shaaÿí.
\snum{246}Yei has akanaxásh.
\pend
%18
\pstart
\snum{247}Áyá déix̱ ux̱éi áwé yaakw ÿeiḵ aawashát.
\snum{248}Gug̱aḵóox̱ áyú.
\snum{249}Gaḵóox̱ sáwé a eig̱ayáat uwaḵúx̱.
\snum{250}Háʼ, shaawát gwáayú neilt áa.
\snum{251}Chʼa tléixʼ ÿatee du waaḵ.
\snum{252}«\!Daaḵ gú, ax̱ kéilkʼ.
\snum{253}Kʼínkʼ áwé x̱aa.óo, ax̱ kéilkʼ.\!»
\snum{254}tle yóo ash yawsiḵaa.
\snum{255}X̱áchdei chʼa hú, ḵúnáx̱ du jeedé, yaakw naháaÿi ásíwé a eig̱ayáat uwaḵúx̱.
\snum{256}X̱ách ḵaa shaaÿí áyú kʼínkʼix̱ awli\-yéx̱.
\snum{257}Tléil awux̱á.
\snum{258}Awsiteen áx̱ siteeÿi át.
\snum{259}«\!Kaháakw tsú x̱aa.óo.\!»
\snum{260}X̱ách ḵaa waag̱í ásíyú tléil awux̱á.
\snum{261}Át tsú ganchʼóokʼ yax̱ akawsixaa.
\snum{262}Du x̱úx̱ ḵu.aa áwé uyéx̱.
\snum{263}De chʼa leengítg̱aa áyá ḵushée.
\snum{264}Hóochʼaaÿí sákw áwé \{awdihaan\} leengít sʼóog̱u.
\snum{265}Cha chʼa á ḵu.aa hél oox̱áa áwé ash tóon wootee.
\snum{266}Ách ḵusa.een yéesʼ ash ÿeetéet aawag̱éxʼ.
\snum{267}Áa ÿetx̱ kei wujikʼén.
\snum{268}Tle aax̱ aawasháat.
\snum{269}Chush yáaxʼ ách ashawlidzóo.
\snum{270}Tle wóosh\-dáx̱ woolʼéexʼ yú shaawát;
\snum{271}x̱ách ḵusax̱á ḵáa shát áyú.
\snum{272}Ajáaḵ áwé chʼu tle ganaltáa dáagi aawax̱útʼ.
\snum{273}A kélʼtʼi ḵu.aa áwé chʼu tle awu̬li\-.óoxu aa wé táaxʼaax̱ wusitee.
\snum{274}Ách áwé ḵusax̱á táaxʼaa.
\snum{275}Ajáaḵ áwé wooḵoox̱.
\snum{276}Ash géit uwaḵúx̱ wé ḵusax̱á ḵwáan.
\snum{277}Aawajáḵ, ash géit ḵóox̱.
\snum{278}A shaaÿí aax̱ aawalʼéexʼ.
\snum{279}Aandé aÿaawax̱áa du tláa hás jeedé.
\snum{280}Aÿa\-kaawahán.
\snum{281}Háatlʼ tin has aÿaawatʼúsʼ.
\pend
%19
\pstart
\snum{282}Tlaagú yan kax̱dulnígín yéi ḵuÿanaḵéich
\snum{283}«\!Hóochʼ x̱á alḵáx̱\!».
\pend
\endnumbering
\end{Leftside}
%%
%% Column break.
%%
\begin{Rightside}
\beginnumbering
\setcounter{pstartR}{12}
%12
\pstart
\noindent
\snum{121}Actually the town place from where they acted so, however, apparently there is nobody there.
\snum{122}They are two, those roads for firewood.
\snum{123}Wh\-en someone would run in this one they would not come down to the beach.
\snum{124}When people go in this one they are not seen;
\snum{125}and when people go by boat they are not seen.
\snum{126}They do not get to town.
\snum{127}It was (after) one more year that nobody was there, that town.
\snum{128}Then it was only two that saved themselves there, that woman in that town, with that woman’s daughter.
\snum{129}Having not made up their minds about themselves, she went with her, her daughter.
\snum{130}\qqk{}“Who perhaps should marry one’s daughter” is what she says.
\snum{131}It is wading ashore on the end of the ice, a heron;
\snum{132}it is that which spoke to them.
\snum{133}\qqk{}“How am I, me?”
\snum{134}\qqk{}“Well what with?” she said to him, that woman.
\snum{135}\qqk{}“Whenever the slush rolls ashore with me, I stand inside it;
\snum{136}it’s with that.”
\snum{137}“Well come home with us”
\snum{138}she said, that woman.
\snum{139}Then he married her, that heron did.
\snum{140}Then a child was sent to her.
\snum{141}Then it was born.
\snum{142}It was a man that was born.
\snum{143}Gradually he gets big.
\snum{144}So he says to his wife, that heron
\snum{145}\qqk{}“What happened to your relatives?”
\snum{146}\qqk{}“When someone went for firewood they did not come down to the beach.”
\pend
%13
\pstart
\snum{147}Then it was when he had gotten big that he would always sit him on the flats.
\snum{148}It was in the ice that he would always bathe him.
\snum{149}Eventually he is shooting things, that boy.
\snum{150}He always took around a bow and arrows.
\snum{151}He went off for something that he could kill, (and) their father, (about) that boy:
\snum{152}\qqk{}“He is already me, my little son.”
\snum{153}He said to his wife
\snum{154}\qqk{}“Now I am leaving you.”
\snum{155}Eventually he is running down into the water, that boy.
\snum{156}He always just reaches for shore whenever it (tries to) kill him.
\pend
%14
\pstart
\snum{157}At some point he went with a bow and arrows.
\snum{158}It was while he was going around on the beach that it was swimming around in a pond, a \fm{héen taaÿéeshi}.
\snum{159}He picked it up from there.
\snum{160}It was these hands of his that were broken apart by its edge.
\snum{161}It is in a pond that he is raising it.
\snum{162}He would always be giving it things to eat whenever he took his bow and arrows for it.
\pend
%15
\pstart
\snum{163}At some point he said to his mother
\snum{164}\qqk{}“Going for firewood.”
\snum{165}\qqk{}“Your uncles never came down to the beach.”
\snum{166}It was in the morning that he ran around on the floor.
\snum{167}He picked up a stone adze.
\snum{168}Just then it was within this one that is one that he was running inland.
\snum{169}It was along within this road that it was sticking out, someone’s finger.
\snum{170}\qqk{}“Here, your finger”
\snum{171}so he said to him.
\snum{172}Then that place through where it poked the finger, he pulled it up from there.
\snum{173}It was onto a stone that he dragged it.
\snum{174}It was that place where he dragged it up from, people’s skeletons were scattered there, the place where it kills people.
\snum{175}The he cut it off of it at the neck with that greenstone adze.
\snum{176}He took it to the beach, to his mother.
\snum{177}Along there it was that he threw it inside.
\snum{178}And his grandmother, then they are cutting the skin of its face in strips.
\snum{179}They are toasting its face with aged urine in the middle of the fire.
\snum{180}It was like the way that they felt that they handled it, there.
\snum{181}At some point again he made his arrows go there, near that \fm{héen taaÿéeshi} that he was raising.
\snum{182}Then it was maybe as it was as tall as a person standing that he shot it through the top of the head, that \fm{héen taaÿéeshi} that he was raising.
\snum{183}He took off the \fm{shunaaÿeit}.
\snum{184}Then he hung it over a stump.
\snum{185}Its edge is so very sharp.
\pend
%16
\pstart
\snum{186}Having arrived at town, again he ran around on the floor.
\snum{187}To within the one which extended this way he took up the stone adze.
\snum{188}Having run off up inland, he saw someone’s head sticking out through the road.
\snum{189}\qqk{}“Eyes up, Kushaḵʼéitʼkw.”
\snum{190}It was like that way that someone had broken it backward that it happened, that head.
\snum{191}Then he hacked it off, as he was prying it backward back and forth.
\snum{192}It was people’s skeletons that were scattered down.
\snum{193}He picked it up near him, its head.
\snum{194}To the beach again, he threw it inside.
\snum{195}It was with crap that they toasted its face.
\snum{196}It is like the way that they feel themselves under its burden that they are doing to it.
\snum{197}After that, then he keeps making his arrows go up.
\snum{198}He was always bringing everything home for his mothers to eat.
\snum{199}It is the heron’s son that is acting so, having supernaturally helped that woman.
\snum{200}At some point he asked his mother
\snum{201}\qqk{}“Through which way did they go by boat that my uncles did not come to me?”
\snum{202}\qqk{}“It is that they went that way, little son”
\snum{203}she said to him.
\snum{204}It was in its direction that he went with bow and arrows.
\snum{205}He went out above it, its eye hole, a devilfish.
\snum{206}
\snum{207}He ran back for it, that shirt that he hid, the \fm{héen taaÿéeshi} shirt.
\snum{209}He put it on, the \fm{héen taaÿéeshi} shirt.
\snum{210}Then he ran inside its tangle.
\snum{211}It was inside it that he bounced, this way and that, in that devilfish’s tangle.
\snum{212}It is just into short pieces that it is cutting it up, his edge.
\snum{213}At some point, its ink sac, he cut the top of it.
\snum{214}Having killed it, he swam out through its eye hole.
\snum{215}He had himself floating around seaward of it.
\snum{216}He went atop a sandbar.
\snum{217}He took it off of himself.
\snum{218}He hung it over a stump.
\snum{219}Again he just goes near to his mother.
\snum{220}Currents floated them to the beach below them, those giant tentacles.
\snum{221}They were cut up like small pieces.
\snum{222}Now that was it that extinguished those townspeople.
\pend
%17
\pstart
\snum{223}It was around there again that he made arrows go.
\snum{224}He came upon it, a mouse, its hole.
\snum{225}It poked its tail up through it.
\snum{226}He went to town immediately.
\snum{227}Just in the morning, it was when people were not hearing ravens that he started going there.
\snum{228}He took his shirt, the \fm{héen taaÿeeshi} shirt.
\snum{229}Having gone behind it he put it on himself;
\snum{230}he repeatedly rubs it, its edge.
\snum{231}Having gone into it, he went up to it, that hole.
\snum{232}Just then it was a rock that he pushed on top of its back;
\snum{233}there it crackled, however a mountain does whenever it falls apart.
\snum{234}Because it is going to swim off through there, it was around the face of this rock that he had himself float, in wait of it, for if it swims in through there.
\snum{235}Having swum out through there, it poked him with its upper lip.
\snum{236}It swam past him.
\snum{237}Its tail tries to split him.
\snum{238}Apparently he just floated himself edge up, awaiting its tail.
\snum{239}It tries to split him.
\snum{240}It was like the way people are cutting it apart that its tail chops up and down on him.
\snum{241}Having become a stump it bobbed up to his flank, crying under its burden.
\snum{242}He was cutting it all up.
\snum{243}After that he swam ashore.
\snum{244}Again he hung it over a stump.
\snum{245}Having dawned, it had floated to the beach below them, its head.
\snum{246}They are cutting it up.
\pend
%18
\pstart
\snum{247}So two nights having passed he took a canoe down the beach.
\snum{248}He is going to go by boat.
\snum{249}Apparently having left, he came to the beach below it.
\snum{250}Hey, there seems to be a woman sitting inside.
\snum{251}She has only one eye.
\snum{252}\qqk{}“Come up, my nephew.
\snum{253}It is fermented fish heads that I have, my nephew.”
\snum{254}so she says to him then.
\snum{255}Actually it is just her, really into her possession, that the canoe travellers apparently came to the beach below her.
\snum{256}It was actually human heads that she had made into fermented fish heads.
\snum{257}He did not eat them.
\snum{258}He saw the things that they were of.
\snum{259}\qqk{}“I also have fish eggs.”
\snum{260}Actually it was apparently human eyes that he did not eat.
\snum{261}Those things too he dumped out by the fire.
\snum{262}Her husband however is absent.
\snum{263}Now it is just for humans that he is searching.
\snum{264}It is for the last thing that she put out human ribs.
\snum{265}So then it however, him not having eaten it, she was upset.
\snum{266}She threw a large mussel shell with which she killed people at the place where he was.
\snum{267}He jumped up off there.
\snum{268}Then he grabbed it from there.
\snum{269}In return he hit her with it.
\snum{270}Then she broke apart, that woman;
\snum{271}actually she is the wife of the cannibal man.
\snum{272}Having killed her, he then dragged her onto the middle of the fire.
\snum{273}Its ashes however, the ones that he blew on became mosquitoes.
\snum{274}Because of that they eat people, mosquitoes.
\snum{275}Having killed her he went by boat.
\snum{276}It came against him, the cannibal.
\snum{277}He killed it, it having come against him.
\snum{278}He snapped off its head.
\snum{279}He paddled it to town into his mothers’ possession.
\snum{280}They shredded its face.
\snum{281}They toasted its face with crap.
\pend
%19
\pstart
\snum{282}Whenever they are done telling an old story people would say
\snum{283}\qqk{}“Done coughing”.
\pend
\endnumbering
\end{Rightside}
\end{pairs}
\Columns

\vspace{1\baselineskip}

\section{Swanton’s abstract}\label{sec:91-swanton-abstract}

A little boy was so badly treated by his uncle’s wife that he went off into the woods, made eight nests, like those of the salmon, along the edge of a stream, and spent as many nights in them.
So he became a shaman and could bring himself to destroy all kinds of animals by means of his songs.
By and by his uncle searched for him and found him.
A spirit called Nīx̣â′ came to him and took him into the fire, and he burned down to a very small size, but his uncle, obeying his directions, took him out, put him into a basket, and so restored him.
Afterward he had his uncle send for his wife, but he took the bottom part of her away so that what she ate did her no good.
By and by a spirit showed itself in the form of a bear, after the shaman had been carried into the fire, scaring his uncle’s wife so that she died, while the uncle forgot to take his nephew out of the fire and let him burn up.
At once all of the animals that had been killed came to life and ran away.

All the people of the town to which this shaman had belonged disappeared except a woman and her daughter.
The woman called for something to marry her daughter and was answered by the heron, by whom the daughter had a son very fond of hunting.
One time he found a fish called hīn-taỵī′cî swimming in a pool, reared it, and, when it became as large as himself, killed it and made use of its skin.
After a while he went up on one of the two trails on which his uncles had disappeared, saw a finger sticking up there, pulled up the being to which it belonged, and killed it.
Then he went along in the other trail, saw a head, and killed the being to which it belonged.
Next he went along the beach, came upon a monster devilfish, and killed it by means of his hīn-taỵī′cî coat.
He killed an enormous rat in the same manner.
Then he came to a cannibal woman who offered him human flesh to eat.
When he refused it she threw a mussel shell at him to kill him, but he jumped aside, threw the shell back, and destroyed her.
He put her body into the fire and she became mosquitoes.
Then he met and killed her cannibal husband.

\section{Swanton’s translation}\label{sec:91-swanton-translation}

\snum{1}A little boy’s friends were all gone.
\snum{2}His uncle was a great hunter, \snum{3}and the little boy was always going around far up in the woods with bow and arrows.
\snum{4}He was growing bigger.
\snum{5}He also went out with his uncle.
\snum{6}His uncle went about everywhere to kill things.
\snum{7}He always brought plenty of game down from the mountains.

\snum{8}One time he again went hunting.
\snum{9}At that time the inside of the house was full of the sides of mountain sheep, on racks.
\snum{10}His uncle’s wife hated her husband’s little nephew very much.
\snum{11}When she went outside for a moment, he broke off a little piece of fat from the sides of mountain sheep hanging on the rack, to put inside of his cheek.
\snum{12}Although there was so much he broke off only so much.
\snum{13}Then his uncle’s wife looked all around.
\snum{14}The end piece was not there.
\snum{15}“Is it you that has done this?”
\snum{16}she said to her husband’s little nephew.
\snum{17}He cried and said, “No.”
\snum{18}Then she put her hand inside of his cheek.
\snum{19}“Why don’t you go up on the mountain?” [she said].
\snum{20}She scratched the inside of his cheek.
\snum{21}Blood ran out of his mouth.
\snum{22}While crying he pulled his uncle’s box toward him.
\snum{23}He took his uncle’s whetstone out of it.
\snum{24}Meanwhile his uncle was far away.

\snum{25}Then he started off into the woods,
\snum{27}carrying the whetstone,
\snum{26}and came out to a creek.
\snum{27}He came out on a sandy bank,
\snum{28}pounded (or scooped) it out like a salmon,
\snum{29}and made a nest beside the water.
\snum{30}He stayed upon it overnight.
\snum{31}His dream was like this.
\snum{32}He was told, “Let it swim down into the water.”
\snum{33}It was his spirit that told him to do this.

\snum{34}When his uncle came down he missed him.
\snum{35}He asked his wife, “Where is my nephew?”
\snum{36}⸢She answered,⸣ “He went up that way with his bow and arrows.”

\snum{37}When [the boy] got up farther he made another nest.
\snum{38}This man was named “For-little-slave”.
\snum{39}He made eight nests.
\snum{40}Now his spirit helper began to come to him on the last.
\snum{41}At that time he took his whetstone down into the creek,
\snum{42}and it swam up in it.
\snum{43}Then he lost his senses and went right up against the cliff.
\snum{44}He stayed up there against the cliff.
\snum{45}Everything came to hear him there – sea gulls, eagles, etc.
\snum{46}When his spirits left him they would always be destroyed – the eagles, sea gulls, all of them.

\snum{47}Now, his uncle hunted for him.
\snum{48}After he had been out for eight days he discovered the nest his nephew had made by the creek.
\snum{49}He saw all the nests his nephew had camped in.
\snum{50}His uncle looked into the creek.
\snum{51}The salmon was swimming there,
\snum{52}and he camped under the nest.
\snum{53}Afterward he listened.
\snum{54}In the morning he heard the beating made by shaman’s sticks.
\snum{55}He heard it just in the middle of the cliff.
\snum{56}Then he came up underneath it.
\snum{57}Before he thought that [his nephew] had seen him, his nephew spoke to him:
\snum{58}\qqk{}“You came under me the wrong way, uncle.”
\snum{59}The uncle pitied his nephew very much.
\snum{60}\qqk{}“Come up by this corner,” ⸢said his nephew.⸣
\snum{61}Ever afterward he was named, “For-little-slave.”
\snum{62}Then his uncle asked him, 
\snum{63}\qqk{}“What caused you to do this?”
\snum{64}He did not say that his uncle’s wife had scratched the inside of his cheek.
\snum{66}Instead he said to his uncle:
\snum{65}“Cave spirits told me to come here.”
\snum{67}This was a big cave, bigger than a house.

\snum{68}Then his spirits came to him while his uncle was with him.
\snum{69}They went inside, and his uncle beat time for him.
\snum{70}Then he told his uncle to remember this:
\snum{71}\qqk{}“When the spirit Nix̣â′ runs into the fire with me, do not let me burn up.
\snum{72}While I am getting small throw me into a basket.” That was the way he did with him.
\snum{74}It ran into the fire with him, and he threw him into the basket.
\snum{75}Then he always came to life inside of the basket.
\snum{76}He became a big man again.

\snum{77}That same evening he sent out his uncle to call,
\snum{78}\qqk{}“This way those that can sing.”
\snum{79}Then the cliff could hardly be seen for the mountain sheep that came down to look into the cave.
\snum{80}When they were seated there, he whirled about his bow and arrows
\snum{81}and all the mountain sheep were destroyed.
\snum{82}The inside of the cave was full of them.
\snum{83}Now, he said to his uncle:
\snum{84}\qqk{}“Take off the hides.”
\snum{85}He was singing for great Nix̣â′.
\snum{86}When the spirit came out of him he reminded his uncle,
\snum{87}“When it runs into the fire with me, don’t forget to take me out and put me into the basket.”

\snum{89}After all of the sheeps’ sides were covered up he sent him for his wife.
\snum{90}He came up with his wife into the cave.
\snum{92}Then he said to his uncle:
\snum{91}Take the half-basket in which we cook.
\snum{92}\qqk{}“Mash up the inside fat for your wife.”
\snum{93}His spirits took out the woman’s bottom part from her.
\snum{94}For this reason the woman never got full eating the mountain-sheep fat.
\snum{95}She could not taste the fat.
\snum{96}He put her in this condition because she scratched the inside of his cheek.

\snum{97}By and by he said to his uncle:
\snum{98}\qqk{}“Make your mind courageous when Nix̣â′ comes in.”
\snum{99}In the evening he told his ucnle to go out and call.
\snum{100}The cliffs could hardly be seen.
\snum{101}Grizzly bears came in front of the house to the door of the cave.
\snum{102}They extended far up in lines.
\snum{103}Then his uncle started the song for the spirit.
\snum{104}They kept coming inside.
\snum{108}Suddenly a grizzly bear came in.
\snum{109}It was as if eagle down were tied around its ears.
\snum{110}At that [the uncle’s] wife became scared
\snum{111}and broke in two.
\snum{112}He did this to her because she had scratched on the inside of his cheek on account of the fat.
\snum{113}His spirit also ran into the fire with him.
\snum{115}While his uncle stood in fear of the grizzly bear, For-little-slave burned up in the fire.

\snum{116}At that the cave creaked, and every animal ran into its skin.
\snum{118}The things they were drying did so.
\snum{119}They did so because the shaman had been burned up.
\snum{120}So the shaman and his uncle also were finally burned up.

% The \fancybreak command is from memoir.cls, see memman.pdf §6.7 Fancy anonymous breaks.
\fancybreak{\rule{16em}{0.125pt}}

\snum{121}Now people were disappearing from the town they had left.
\snum{122}There were two wood roads.
\snum{123}When anybody went out on one of these roads he never came back,
\snum{124}and a person who went out on the other also, never came back.
\snum{125}When one went away by canoe, he, too, was never seen again.
\snum{126}He did not come home.
\snum{127}In a single year there was no one left in that town
\snum{128}except two, a woman and her daughter.
\snum{129}After she had thought over their condition, this woman took her daughter away.
\snum{130}She said, “Who will marry my daughter?”
\snum{131}A heron that was walking upon the shore ice
\snum{132}spoke to them,
\snum{133}“How am I?”
\snum{134}“What can you do?” said the woman.
\snum{135}“I can stand upon the ice when it comes up.”
\snum{137}“Come home with us,”
\snum{138}said the woman.
\snum{139}So the heron married [the girl],
\snum{140}and she became pregnant.
\snum{141}She brought forth.
\snum{142}She bore a son.
\snum{143}It began to grow large.
\snum{144}The heron said to his wife,
\snum{145}\qqk{}“What is the matter with your friends?”
\snum{146}⸢and she answered⸣ “When they went after wood they never came back.”

\snum{147}After the child had become large he kept taking it to the beach.
\snum{148}He would bathe it amid the ice.
\snum{149}Then the little boy began shooting with arrows.
\snum{150}He always took his bow and arrows around.
\snum{151}When he killed anything his father would say of the little boy,
\snum{152}\qqk{}“My little son is just like me.”
\snum{153}⸢By and by⸣ he said to his wife,
\snum{154}\qqk{}“I am going away.”
\snum{155}After that the little boy began to go into the water.
\snum{156}He crawled up, when he was almost killed by it.

\snum{157}Once he started off with his bow and arrows.
\snum{158}When he was walking along the beach [he saw] a hīn-taỵī′cî swimming in a little pond of sea water.
\snum{159}He took it up.
\snum{160}It cut his hands with its sharp sides.
\snum{161}He reared it in the little pond.
\snum{162}As he was going along with his bow and arrows he would feed it.

\snum{163}One time he said to his mother,
\snum{164}\qqk{}“I am going after firewood.”
\snum{165}\qqk{}“But your uncles never came down,” [she said].
\snum{166}In the morning he jumped quickly out on the floor.
\snum{167}He took a stone ax
\snum{168}and ran up in one of the roads.
\snum{169}In it there was a finger sticking up,
\snum{171}which said to him,
\snum{170}\qqk{}“This way with your finger.”
\snum{172}He took hold of it and pulled up the being which was there.
\snum{173}He threw it down on a stone.
\snum{174}In the place from which he took it bones were left where it had been killing.
\snum{175}Then he cut off its head with his stone ax.
\snum{176}He took it down to his mother.
\snum{177}He threw it into the house to her
\snum{178}and to his grandmother, and they cut the face all up.
\snum{179}They burned its face in the fire along with urine.
\snum{180}They treated it just as they felt like doing.
\snum{181}By and by the boy went up to the hīn-taỵī′cî he was raising.
\snum{182}Before it got longer than himself he shot it in the head.
\snum{183}He took off its skin.
\snum{184}Then he put [the skin] on a stump.
\snum{185}How sharp were its edges!

\snum{186}When he got home again he jumped quickly out on the floor in the morning.
\snum{187}He took his stone ax along in the next road.
\snum{188}When he got far up he saw a head sticking up in the road.
\snum{189}⸢He said,⸣ “Up with your eyes, Kucaqē′!tkᵘ.”
\snum{190}The head was bent far backward.
\snum{191}After he had moved its head backward he cut it off.
\snum{192}The place where he took up this head was all full of bones.
\snum{194}He threw that down into the house.
\snum{195}They rubbed its face with dung.
\snum{196}They did to it as they felt toward it.
\snum{197}After that he kept taking his bow and arrows up.
\snum{198}He brought all kinds of things into the house for his mothers (i.e., his mother and grandmother).
\snum{199}The son of the heron who came to help the woman was doing this.
\snum{200}By and by he asked his mother,
\snum{201}\qqk{}“In which direction did my uncles go who went out by sea and never came home?” \snum{203}She said to him,
\snum{202}\qqk{}“They would go this way, little son.”
\snum{204}He went in that direction with his bow and arrow,
\snum{205}and came out above the hole of a devilfish.
\snum{206}As he was sitting there ready for action he looked right down into it.
\snum{207}Then he went back for the hīn-taỵī′cî coat he had hidden.
\snum{208}When he returned he threw a stone down upon the devilfish.
\snum{209}He put on the hīn-taỵī′cî coat
⸢in order to jump into the midst of the devilfish’s arms.⸣
\snum{210}Then he went right into them very quickly.
\snum{211}He moved backward and forward inside of the devilfish’s arms,
\snum{212}and cut them all up into fine pieces with his side.
\snum{213}By and by he cut its color sac in the midst of its arms,
\snum{214}and afterward he swam out of the hole.
\snum{215}He was floating outside,
\snum{216}and he came ashore 
\snum{217}and took off his coat.
\snum{218}Then he put it on the stump,
\snum{219}and came again to his mother.
\snum{220}The large tentacles floated up below them.
\snum{221}He had cut them up into small pieces.
\snum{222}It was that which had destroyed the people.

\snum{223}Again he took his bow and arrows.
\snum{224}He came across a rat hole.
\snum{225}The rat’s tail was hanging out.
\snum{226}He came directly home
\snum{227}and, early in the morning before the raven called, he set out for it.
\snum{228}He took his hīntaỵī′cî shirt.
\snum{229}When he got back he started to put [the shirt] on
\snum{230}after he had sharpened its edges.
\snum{231}After he had gotten into it he went up to the [rat] hole.
\snum{232}Then he threw a stone down upon it,
\snum{233}making it give forth a peeping sound, as if the mountain were cracking in two.
\snum{234}He swam round a stone, waiting for it to swim out.
\snum{235}When it swam out it ran its nose against him.
\snum{236}It swam past him.
\snum{237}It wanted to drop its tail down upon him.
\snum{240}When it dropped its tail down upon him it was cut up into small pieces.
\snum{241}Then it swam up to his side, crying on account of what he had done.
\snum{242}He cut it all up.
\snum{243}Afterward he swam ashore.
\snum{244}He put his skin back on the stump.
\snum{245}In the morning its head floated in front of them.
\snum{246}They cut it up.

\snum{247}After two days he pulled down his canoe.
\snum{248}Going along for a while,
\snum{249}he came up to the beach in front of
\snum{250}a woman sitting in a house.
\snum{251}She had only one eye. 
\snum{252}\qqk{}“Come up, my nephew.
\snum{253}I have stale salmon heads, my nephew,”
\snum{254}she said to him.
\snum{255}This person in front of whom he had come was the real one who had destroyed the canoes.
\snum{256}Those were human heads that she spoke of as stale heads.
\snum{257}He did not eat them.
\snum{258}He saw what they were.
\snum{259}\qqk{}“I have also fish eggs,” [she said].
\snum{260}Those were human eyes, and he did not eat them.
\snum{261}He emptied them by the fire.
\snum{262}The woman’s husband, however, was away
\snum{263}hunting for human beings.
\snum{264}Lastly she got human ribs,
\snum{265}and when he would not eat those she became angry about it.
\snum{266}She threw a shell at him with which she used to kill human beings, but missed him,
\snum{267}for he jumped away quickly.
\snum{268}Then he took it up.
\snum{269}He hit her with it in return,
\snum{271}and the cannibal wife
\snum{270}broke in two.
\snum{272}After he had killed her he pulled her over on the fire.
\snum{273}When he blew upon her ashes, however, they became mosquitoes.
\snum{274}This is why mosquitoes eat people.
\snum{275}After he had killed her he went away
\snum{276}and met the cannibal man.
\snum{277}When he met him he killed him.
\snum{278}He cut off his head
\snum{279}and took it to his mother’s home.
\snum{280}There they cut his face all up.
\snum{281}They burned his face with dung.

\snum{282}In olden times when a person finished a story he said,
\snum{283}\qqk{}“It’s up to you.”

\clearpage
\section{Paragraph 1}\label{sec:91-para-1}

\ex\label{ex:91-1-relatives-died-out}%
\exmn{267.1}%
\begingl
	\glpreamble	Ducᴀgū′nî qotx cūˈwax̣īx̣ yū-ᴀt-k-!ᴀ′tsk!ᵒ. //
	\glpreamble	Du shagóoni ḵutx̱ shoowaxeex, yú atkʼátskʼu. //
	\gla	{} Du \rlap{shagóoni} @ {} @ {} {}
		{} \rlap{ḵutx̱} @ {} {} \rlap{shoowaxeex} @ {} @ {} @ {} @ {} +
		{} yú \rlap{atkʼátskʼu.} @ {} @ {} @ {} @ {} {} //
	\glb	{} du shá- góon -í {}
		{} ḵú -dáx̱ {} shu- wu- i- \rt[¹]{xix} -μμL
		{} yú at= kʼí- ÿáts -kʼʷ -í {} //
	\glc	{}[\pr{DP} \xx{3h·pss} head- isthmus -\xx{pss} {}]
		{}[\pr{PP} \xx{areal} -\xx{abl} {}] end- \xx{pfv}- \xx{stv}- \rt[¹]{fall} -\xx{var}
		{}[\pr{DP} \xx{dist} \xx{4n·pss}= base- child -\xx{dim} -\xx{pss} {}] //
	\gld	{} his\ix{i} \rlap{ancestor} {} {} {} 
		{} \rlap{lost} {} {} \rlap{end.\xx{ncnj}.\xx{pfv}.use·up} {} {} {} {} 
		{} that \rlap{boy\ix{i}} {} {} {} {} {} //
	\glft	‘His family became extinct, that young boy.’
		//
\endgl
\xe

\ex\label{ex:91-2-uncle-expert-hunter}%
\exmn{267.1}%
\begingl
	\glpreamble	Dukā′k qo′a ᴀt s!atē′x sîtî. //
	\glpreamble	Du káak ḵu.aa at sʼaatéex̱ sitee. //
	\gla	{} Du káak {} ḵu.aa
		{} at \rlap{sʼaatéex̱} @ {} {}
		\rlap{sitee.} @ {} @ {} @ {} //
	\glb	{} du káak {} ḵu.aa
		{} at sʼaatí -x̱ {}
		s- i- \rt[¹]{tiʰ} -μμL //
	\glc	{}[\pr{DP} \xx{3h·pss} mat·uncle {}] \xx{contr}
		{}[\pr{PP} \xx{4n·pss} master -\xx{pert} {}]
		\xx{appl}- \xx{stv}- \rt[¹]{be} -\xx{var} //
	\gld	{} his mat·uncle {} however
		{} thing’s master -of {} 
		\rlap{\xx{ncnj}.\xx{stv}·\xx{impfv}.be.as} {} {} {} //
	\glft	‘His uncle however was an expert hunter.’
		//
\endgl
\xe

The term \fm{at sʼaatí} in (\lastx) is a avoidance term for an expert hunter.
It literally means ‘master of something’, based on the inalienable noun \fm{sʼaatí} ‘master’.
An explicit replacement of \fm{at} in this context would be \fm{alʼóon} ‘hunting’ giving \fm{alʼóon sʼaatí} ‘master of hunting’.
As well as its literal use in e.g.
\fm{goox̱ sʼaatí} ‘master of a slave’ and \fm{aan sʼaatí} ‘mayor, lit.\ master of a town’, the noun \fm{sʼaatí} is used in a variety of conventional terms for people who are either experts at or have a pronounced tendency for something.
Examples of this include \fm{dáanaa sʼaatí} ‘rich man, lit.\ money master’, \fm{naaw sʼaatí} ‘drunkard, lit.\ liquor master’, and \fm{atgóok sʼaatí} ‘wise man, lit.\ knowhow master’ \parencite[09/191]{leer:1973}.
For another avoidance term for hunting see the discussion of \fm{atnatée} ‘existing’ instead of \fm{alʼóon} ‘hunting’ in the story “\fm{Eechká Ḵáawu:} Man of the Reef” (ch.\ \ref{ch:106-low-caste-name} p.\ \pageref{ex:106-2-fond-of-hunting}).

\ex\label{ex:91-3-go-around-bownarrow}%
\exmn{267.2}%
\begingl
	\glpreamble	Tc!agu′tsᴀ nagu′ttc yu-ᴀt-k!ᴀ′tsk!ᵘ tcū′net tîn yudā′q. //
	\glpreamble	Chʼa goot sá nagútch yú atkʼátskʼu chooneit tin, yú dáaḵ. //
	\gla	{} Chʼa {} \rlap{goot} @ {} {} sá {}
		\rlap{nagútch} @ {} @ {} @ {} 
		{} yú \rlap{atkʼátskʼu} @ {} @ {} @ {} @ {} {} +
		{} {} {} \rlap{chooneit} @ {} @ {} {} {} {} tin, {}
		{} yú dáaḵ. {} //
	\glb	{} chʼa {} goo -t {} sá {}
		n- \rt[¹]{gut} -μH -ch
		{} yú at= kʼí- ÿáts -kʼʷ -í {}
		{} {} {} \rt[¹]{chun} -μμL -i {} át {} tin {}
		{} yú dáaḵ {} //
	\glc	{}[\pr{QP} just {}[\pr{PP} where -\xx{pnct} {}] \xx{q} {}]
		\xx{ncnj}- \rt[¹]{go·\xx{sg}} -\xx{var} -\xx{rep}
		{}[\pr{DP} \xx{dist} \xx{4n·pss}= base- child -\xx{dim} -\xx{pss} {}]
		{}[\pr{PP} {}[\pr{NP} {}[\pr{CP} \rt[¹]{wound} -\xx{var} -\xx{rel} {}] thing {}] \xx{instr} {}]
		{}[\pr{DP} \xx{dist} inland {}] //
	\gld	{} just {} where -around {} ever {}
		\rlap{\xx{hab}.go·\xx{sg}} {} {} {}
		{} that \rlap{boy} {} {} {} {} {}
		{} {} {} \rlap{bow·and·arrow} {} {} {} {} {} with {}
		{} that inland {} //
	\glft	‘He always went around wherever, that boy, with his bow and arrow, up inland.’
		//
\endgl
\xe

The noun \fm{chooneit} ‘bow and arrow’ in (\lastx) is generally considered to be a monomorphemic noun but it can be decomposed into a complex noun phrase as shown.
The \fm{…eit} [\ipa{èːt}] ending of a noun generally indicates that the word is originally from some sequence of a suffix \fm{-i} or \fm{-í} followed by the noun \fm{át} [\ipa{ʔát}] ‘thing’.
The noun \fm{chooneit} is thus expected to come from something like \fm{chooni} + \fm{át}.
The \fm{chooni} portion can be identified as a relative clause based on the root \fm{\rt[¹]{chun}} ‘wound’ with the relative clause suffix \fm{-i} on the end.
This root is attested by a few verbs such as in \fm{ax̱ tláa wudichún} ‘my mother was wounded’, \fm{gandaasʼaají x̱at wulichún} ‘a bee wounded me’ \parencite[both from][250]{story-naish:1973}, and \fm{yú neek a.aax̱ ash wulchóonin} ‘hearing of the news wounded him’ \parencite[10/214]{leer:1973}.
It is also attested once as a noun in \fm{choon kasʼéetg̱aa áyá i x̱ánt x̱waagút} ‘it is for the payment for death (of my kinsman) that I came to you’ \parencite[10/213]{leer:1973}.
The root \fm{\rt[¹]{chun}} ‘wound’ may be etymologically related to the directional noun \fm{dachóon} ‘direct, straight ahead’ but their connection is still unclear.
The noun \fm{chooneit} is often translated as just ‘arrow’ and \citeauthor{leer:1973} specifically gives it as “barbed arrow for wounding” \parencite[10/213]{leer:1973}, but it is also frequently translated as ‘bow and arrow’ by synecdoche (metonymy).
In this narrative the synecdochic interpretation seems to be intended since there is no independent reference to a bow (\fm{sáḵs}), so the translation uses ‘bow and arrow’.

\ex\label{ex:91-4-gradually-is-big}%
\exmn{267.3}%
\begingl
	\glpreamble	Desgwᴀ′tc ʟ!agā′łigê. //
	\glpreamble	Deisgwách tlʼag̱áa ligéi. //
	\gla	Deisgwách tlʼag̱áa \rlap{ligéi.} @ {} @ {} @ {} //
	\glb	deisgwách tlʼag̱áa l- i- \rt[¹]{ge} -μμH //
	\glc	eventually lots \xx{xtn}- \xx{stv}- \rt[¹]{big} -\xx{var} //
	\gld	eventually lots \rlap{\xx{gcnj}.\xx{stv}·\xx{impfv}.big} {} {} {} //
	\glft	‘Eventually he is tall.’
		//
\endgl
\xe

The sentence in (\lastx) is nearly identical to (\ref{ex:89-112-gradually-get-big}) in \fm{Eiḵ Shagóon} (ch.\ \ref{ch:89-origin-of-copper} p.\ \pageref{ex:89-112-gradually-get-big}).
The two adverbs \fm{deisgwách} ‘after a while, pretty soon, eventually’ and \fm{tlʼag̱áa} ‘lots, much, more’ are identical in both sentences; see the discussion of (\ref{ex:89-112-gradually-get-big}) for details on their etymology and analysis as well as lexicographic references.
The significant difference is that here the verb form is a state imperfective where in the other narrative the verb is a progressive form.
This is linguistically important because it implies that the temporal adverb \fm{deisgwách} is compatible with imperfective states (i.e.\ not result states like the perfective aspect).
For this reason \fm{deisgwách} is translated here as ‘eventually’ rather than ‘gradually’.

\ex\label{ex:91-5-with-him-relocated}%
\exmn{267.3}%
\begingl
	\glpreamble	Ān wułîgā′s!. //
	\glpreamble	Aan wuligáasʼ. //
	\gla	{} \rlap{Aan} @ {} {} \rlap{wuligáasʼ.} @ {} @ {} @ {} @ {} //
	\glb	{} á -n {} wu- l- i- \rt[¹]{gasʼ} -μμH //
	\glc	{}[\pr{PP} \xx{3n} -\xx{instr} {}] \xx{pfv}- \xx{xtn}- \xx{stv}- \rt[¹]{extend} -\xx{var} //
	\gld	{} him -with {} \rlap{\xx{ncnj}.\xx{pfv}.relocate} {} {} {} {}  //
	\glft	‘He relocated with him.’
		//
\endgl
\xe

\citeauthor{swanton:1909}’s translation of (\lastx) as “He also went out with his uncle” is misleading.
The Tlingit sentence does not explicitly identify either of the two third person referents.
The use of third person nonhuman \fm{á} ‘it’ is probably obviative or backgrounding, referring to the uncle who is not explicitly mentioned until the next sentence in (\ref{ex:91-6-autumn-uncle-go-out-to-kill}).
Also the Tlingit in (\lastx) has the verb \fm{wuligáasʼ} ‘he relocated, moved house’ which can be found for example in sentences (\ref{ex:90-8-food-out-moved-away}) and (\ref{ex:90-45-carry-suphelp-thing}) of \fm{Jiwduwanág̱i Ḵáa} (ch.\ \ref{ch:90-man-abandoned} pp.\ \pageref{ex:90-8-food-out-moved-away} \&\ \pageref{ex:90-45-carry-suphelp-thing}).
This verb describes a change in residence, moving one’s home to a different location \parencite[cf.\ e.g.][137.1843, 185.2550]{story-naish:1973}; the related noun \fm{gáasʼ} ‘housepost’ describes the four houseposts within a traditional Tlingit house.
In this particular context this verb seems to refer the traditional practice of a maternal uncle taking a young man away from the village to live in some remote location for private training and education \parencites[cf.][479]{de-laguna:1972}[51]{kamenskii-kan:1985}[27]{emmons:1991}[167]{kan:2016}.

\ex\label{ex:91-6-autumn-uncle-go-out-to-kill}%
\exmn{267.3}%
\begingl
	\glpreamble	Qolyē′s ᴀt wudjā′q cūt nagu′ttc du kā′k //
	\glpreamble	Ḵulyéis, at wujaaḵ shóot nagútch du káak. //
	\gla	{} \rlap{Ḵulyéis,} @ {} @ {} @ {} @ {} @ {} @ {} {}
		{} {} at @ \rlap{wujaaḵ} @ {} @ {} @ {} {} \rlap{shóot} @ {} {}
		\rlap{nagútch} @ {} @ {} @ {}
		{} du káak. {} //
	\glb	{} ḵu- {} d- l- \rt[¹]{yes} -μμH {} {}
		{} {} at= wu- \rt[²]{jaḵ} -μμL {} {} shú -t {}
		n- \rt[¹]{gut} -μH -ch
		{} du káak {} //
	\glc	{}[\pr{CP} \xx{areal}- \xx{zcnj}\· \xx{mid}- \xx{xtn}- \rt[¹]{autumn} -\xx{var} \·\xx{sub} {}]
		{}[\pr{PP} {}[\pr{NP} \xx{4n·o}= \xx{pfv}- \rt[²]{kill} -\xx{var} \·\xx{nmz} {}] end -\xx{pnct} {}]
		\xx{ncnj}- \rt[¹]{go·\xx{sg}} -\xx{var} -\xx{rep}
		{}[\pr{DP} \xx{3h·pss} mat·uncle {}] //
	\gld	{} \rlap{\xx{csec}.autumn} {} {} {} {} {} {} {}
		{} {} sth\• \rlap{\xx{zcnj}.\xx{pfv}.kill} {} {} -ing {} end -around {}
		\rlap{\xx{hab}.go·\xx{sg}} {} {} {}
		{} his mat·uncle {} //
	\glft	‘Having become autumn, he would go around in order to kill things, his uncle.’
		//
\endgl
\xe

The verb \fm{ḵulyéis} in (\lastx) is a hapax legomenon: it is not attested anywhere else.
Although \citeauthor{swanton:1909} glosses it as “for a long time”, it is actually appears to be based on the root \fm{\rt{yes}} which is otherwise only documented in the noun \fm{yeis} ‘autumn, fall’ (Tongass \fm{yeìs} [\ipa{jeʰs}]).
The form appears to be an instance of the consecutive aspect, and thus the verb is \fm{∅}-conjugation.
Consecutive aspect forms normally have a long high tone vowel (\fm{-μμH}) in the verb stem.
Since \citeauthor{swanton:1909}’s transcription does not include information about tone, we cannot say for sure if the form \orth{Qolyē′s} reflects the predicted high tone or if instead this is irregular with low tone as e.g.\ \fm[?]{ḵulyeis} [\ipa{qʰʷùɬ.ˈjèːs}], but a long vowel is supported by his symbol \orth{ē′}.
The retranscription assumes that the stem is regular and so gives a long vowel with high tone.

\FIXME{Discuss \fm{XP shóo-t} structure.
In this context it seems to be an alternative to the purposive clause with hortative + \fm{-t}, but it’s not described anywhere and needs some looking at.
The form seems to be a nominalization; since \fm{aawajáḵ} ‘s/he killed it’ is an achievement the imperfective aspect is prohibited so the perfective is expected.
The \fm{-μμL} stem might be unusual, but we don’t know much about nominalized verb stem variation.}

\ex\label{ex:91-7-mountains-from-lug-lots-beach}%
\exmn{267.4}%
\begingl
	\glpreamble	Cā′ỵadadᴀx ỵēq ᴀt kūdjē′łtc aʟē′n. //
	\glpreamble	Shaa ÿadaadáx̱ ÿeiḵ at koojeilch aatlein. //
	\gla	{} Shaa \rlap{ÿadaadáx̱} @ {} @ {} {}
		ÿeiḵ @ at @ \rlap{koojeilch} @ {} @ {} @ {} @ {} 
		\rlap{aatlein.} @ {} //
	\glb	{} shaa ÿá- daa -dáx̱ {}
		ÿeiḵ= at= k- u- \rt[²]{jel} -μμL -ch
		aa =tlein //
	\glc	{}[\pr{PP} mountain face- around -\xx{abl} {}]
		beach= \xx{4n·o}= \xx{qual}- \xx{zpfv}- \rt[²]{lug} -\xx{var} -\xx{rep}
		\xx{part} =big //
	\gld	{} mountain face- around -from {}
		beach\• sth\• \rlap{\xx{hab}.lug} {} {} {} {}
		\rlap{lots} {} //
	\glft	‘From around the face of the mountains he would carry lots of things to the beach.’
		//
\endgl
\xe

\section{Paragraph 2}\label{sec:91-para-2}

\ex\label{ex:91-8-went-again-hunting}%
\exmn{267.5}%
\begingl
	\glpreamble	Wānanī′sawe wugū′t ts!u ᴀt łūn. //
	\glpreamble	Wáa nanée sáwé woogoot tsu at lʼóon. //
	\gla	{} Wáa \rlap{nanée} @ {} @ {} @ {} {}
		\rlap{sáwé} @ {} @ {}
		\rlap{woogoot} @ {} @ {} @ {} tsu +
		{} at @ \rlap{lʼóon.} @ {} {} {} //
	\glb	{} wáa n- \rt[¹]{ni} -μμH {} {} 
		s= á -wé
		wu- i- \rt[¹]{gut} -μμL tsu
		{} at= \rt[²]{lʼuʼn} -μμH {} {} //
	\glc	{}[\pr{CP} how \xx{ncnj}- \rt[¹]{happen} -\xx{var} \·\xx{sub} {}]
		\xx{q}= \xx{foc} -\xx{mdst}
		\xx{pfv}- \xx{stv}- \rt[¹]{go·\xx{sg}} -\xx{var} again
		{}[\pr{NP} \xx{4n·o}= \rt[²]{hunt} -\xx{var} \hspace{1.5em}\·\xx{nmz} {}] //
	\gld	{} how \rlap{\xx{csec}.happen} {} {} \·while {}
		ever\· \rlap{it.is} {}
		\rlap{\xx{ncnj}.\xx{pfv}.go·\xx{sg}} {} {} {} again
		{} sth\• \rlap{\xx{zcnj}.\xx{impfv}.hunt} {} \hspace{1.5em}-ing {} //
	\glft	‘At some point he went again hunting things.’
		//
\endgl
\xe

\FIXME{Is \fm{at lʼóon} a noun or is it still a verbal clause?
Hard to say because they look identical.
The answer would lie in whether it could still have a D pronoun subject, e.g.\ \fm{at x̱alʼóon} versus \fm{at x̱alʼóoni}. If not then it’s a noun.}

\ex\label{ex:91-9-went-again-hunting}%
\exmn{267.5}%
\begingl
	\glpreamble	Yū′nēł qo′a ᴀt-kᴀgedī′tc coałihî′k yū′kᴀxỵî. //
	\glpreamble	Yú neil ḵu.aa at kageidéech shawlihík yú kax̱ÿee. //
	\gla	{} Yú \rlap{neil} @ {} {} ḵu.aa
		{} at \rlap{kageidéech} @ {} {}
		\rlap{shawlihík} @ {} @ {} @ {} @ {} @ {} @ {}
		{} yú kax̱ÿee. {} //
	\glb	{} yú neil {} {} ḵu.aa
		{} at kageidí -ch {}
		ⱥ- sha- wu- l- i- \rt[¹]{hik} -μH
		{} yú kax̱ÿee {} //
	\glc	{}[\pr{PP} \xx{dist} inside \·\xx{loc} {}] \xx{contr}
		{}[\pr{DP} \xx{4n·pss} side·meat -\xx{erg} {}]
		\xx{arg}- head- \xx{pfv}- \xx{csv}- \xx{stv}- \rt[¹]{full} -\xx{var}
		{}[\pr{DP} \xx{dist} rafter {}] //
	\gld	{} that inside \·at {} however
		{} sth’s side·meat {} {}
		\rlap{3>3.\xx{zcnj}.\xx{pfv}.make.full} {} {} {} {} {} {}
		{} those rafters {} //
	\glft	‘Inside however, sides of meat filled the rafters’
		//
\endgl
\xe

The sentence in (\lastx) contains two nouns which are treated as monolithic units but which have identifiable internal morphology, namely \fm{at kageidí} ‘something’s side of meat’ and \fm{kax̱ÿee} ‘rafters’.
The noun \fm{at kageidí} describes a side of meat from an animal, with \fm{at} ‘something’s’ being an indefinite possessive pronoun; see sentence (\ref{ex:91-89-done-put-up-meat-sent-for-wife}) for \fm{jánwu kageidí} with the noun \fm{jánwu} ‘mountain goat’ as a possessor.
This appears to be formed with \fm{ká} ‘horizontal surface’ and the possessive suffix \fm{-í}, but the middle portion \fm{geit} is unidentified and the whole noun cannot occur unpossessed (i.e.\ \fm[*]{kageit}).
The noun \fm{géi} ‘against, opposing’ might be related, as might the root \fm{\rt{ge}} ‘big, large’, but the final \fm{t} is unaccounted for and the low tone is also problematic.
The root \fm{\rt[¹]{get}} ‘walk softly, tiptoe, stalk’ is probably unrelated given its meaning and there is also no obvious connection to the well known contraction \fm{…eit} < \fm{-i át}.

The noun \fm{kax̱ÿee} ‘rafters’ also appears to be decomposable at first glance but it resists a coherent analysis.
The \fm{ÿee} portion can be identified with the noun \fm{ÿee} ‘below a convex shape, inside a building’ given that rafters sit below the roof of a building.
But the initial \fm{kax̱} is unidentified; \citeauthor{leer:1973} finds “\fm{kax̣}” only in this noun \parencite[f06/49]{leer:1973} and lists it as unrelated to \fm{káx̱} ‘cambium’ \parencite[f06/48]{leer:1973} and to the otherwise unidentified \fm{-káx̱} in \fm{lʼag̱akáx̱} ‘west wind’ \parencite[f06/50]{leer:1973}; compare also \fm{tsʼéekáx̱kʼw} ‘alpine blueberry (\species{Vaccinium}{uliginosum}[L.])’ \parencite[09/326]{leer:1973}. 

\ex\label{ex:91-10-hate-husbands-nephew}%
\exmn{267.6}%
\begingl
	\glpreamble	Hawā′staga ᴀcik!ā′n doxo′x qeł k!ᴀ′tsk!ᵒ. //
	\glpreamble	Ha wáa sdagaa ashikʼáan du x̱úx̱ kéilkʼátskʼu. //
	\gla	Ha wáa \rlap{sdagaa} @ {}
		\rlap{ashikʼáan} @ {} @ {} @ {} @ {}
		{} du x̱úx̱ \rlap{kéilkʼátskʼu.} @ {} @ {} @ {} @ {} {} //
	\glb	ha wáa s= dagaa
		a- sh- i- \rt[²]{kʼan} -μμH
		{} dú x̱úx̱ kéilkʼ- kʼí- ÿáts -kʼʷ -í {} //
	\glc	well how \xx{q}= \xx{mir}
		\xx{arg}- \xx{pej}- \xx{stv}- \rt[²]{hate} -\xx{var}
		{}[\pr{DP} \xx{3h·pss} husband sor·neph- base- child -\xx{dim} -\xx{pss} {}] //
	\gld	well how \rlap{indeed} {}
		\rlap{3>3.\xx{gcnj}.\xx{stv}·\xx{impfv}.hate} {} {} {} {}
		{} her husband’s nephew- \rlap{young·boy} {} {} {} {} //
	\glft	‘Well how indeed does she hate her husband’s young nephew.’
		//
\endgl
\xe

What \citeauthor{swanton:1909} transcribes as \orth{qeł k!ᴀ′tsk!ᵒ} in (\lastx) appears to be a combination of the noun \fm{kéilkʼ} ‘sororal nephew (of man), sororal niece (of woman)’ and the noun \fm{kʼátskʼu} ‘young boy’.
This compound is not attested anywhere else.
\citeauthor{swanton:1909}’s transcription suggests a form \fm{kéilkʼátskʼu} and not \fm{kéilkʼkʼátskʼu} so that \fm{kéilkʼ} is missing its final ejective velar stop [\ipa{kʼ}].
Several kinship terms include a diminutive suffix \fm{-kʼ} which is not optional \parencite[cf.][]{krauss:1977b}: \fm{kéilkʼ}, \fm{kéekʼ} ‘younger sibling (same gender)’, \fm{éekʼ} ‘woman’s brother’, \fm{dlaakʼ} ‘sister’, \fm{tláakʼw} ‘maternal aunt’, \fm{káalkʼw} ‘fraternal niece/nephew’, \fm{léelkʼw} ‘grandparent’.
A few kinship terms are typically found with a diminutive suffix but can occur without it: \fm{sée-kʼ} ‘daughter’, \fm{yéet-kʼ} ‘son’, \fm{dachx̱án-kʼ} ‘grandchild’.
If \citeauthor{swanton:1909}’s transcription is not mistaken then the occurrence of \fm{kéilkʼ} without \fm{-kʼ} in (\lastx) suggests that an optional diminutive suffix used to be more common.
An alternative analysis is that \fm{kéilkʼátskʼu} is composed of \fm{kéilkʼ} and \fm{ÿátskʼu} so that there is no \fm{kʼí}, but \fm{ÿátskʼu} is not otherwise attested.

\ex\label{ex:91-11-break-off-fatlet-for-cheek}%
\exmn{267.6}%
\begingl
	\glpreamble	Gᴀ′nde nagū′tawe dukā′k cᴀt yā′kᴀxỵēx dîx̣wᴀ′ts!î ᴀtkage′dî cūtx
			awaʟī′q! yutayᴀ′k! tc!a dū′wᴀc tū′g̣ā. //
	\glpreamble	Gánde nagóot áwé du káak shát, yá kax̱ÿeix̱ dixwásʼi at kageidí shóotx̱
			aawalʼéexʼ, yú taayákʼw, chʼa du wásh tóog̱aa. //
	\gla	{} {} \rlap{Gánde} @ {} {} \rlap{nagóot} @ {} @ {} @ {} {}
		\rlap{áwé} @ {}
		{} du káak shát, {} +
		{} yá {} {} \rlap{kax̱ÿeix̱} @ {} {}
				\rlap{dixwásʼi} @ {} @ {} @ {} @ {} {}
			at kageidí \rlap{shóotx̱} @ {} {}
		\rlap{aawalʼéexʼ,} @ {} @ {} @ {} @ {}
		{} yú \rlap{taayákʼw,} @ {} {}
		{} chʼa du wásh \rlap{tóog̱aa.} @ {} {} //
	\glb	{} {} gáan -dé {} n- \rt[¹]{gut} -μμH {} {}
		á -wé
		{} du káak shát {}
		{} yá {} {} kax̱ÿee -x̱ {}
				d- i- \rt[¹]{xwasʼ} -μH -i {}
			at kageidí shú -dáx̱ {}
		a- wu- i- \rt[²]{lʼixʼ} -μμH
		{} yú taaÿ -kʼw {}
		{} chʼa du wásh tú -g̱áa {} //
	\glc	{}[\pr{CP} {}[\pr{PP} outside -\xx{all} {}]
			\xx{ncnj}- \rt[¹]{go·\xx{sg}} -\xx{var} \·\xx{sub} {}]
		\xx{foc} -\xx{mdst}
		{}[\pr{DP} \xx{3h·pss} mat·uncle wife {}]
		{}[\pr{PP} \xx{prox} {}[\pr{CP} {}[\pr{PP} rafter -\xx{pert} {}]
				\xx{mid}- \xx{stv}- \rt[¹]{hang·\xx{pl}} -\xx{var} -\xx{rel} {}]
			\xx{4n·pss} side·meat end -\xx{abl} {}]
		\xx{arg}- \xx{pfv}- \xx{stv}- \rt[²]{break} -\xx{var}
		{}[\pr{DP} \xx{dist} fat -\xx{dim} {}]
		{}[\pr{PP} just \xx{3h·pss} cheek inside -\xx{ades} {}] //
	\gld	{} {} outside -to {} \rlap{\xx{csec}.go·\xx{sg}} {} {} {} {}
		\rlap{it.is} {}
		{} his mat·uncle’s wife {}
		{} this {} {} rafter -on {}
			\rlap{\xx{zcnj}.\xx{stv}·\xx{impfv}.hang·\xx{pl}} {} {} {} \•that {}
			sth’s side·meat end -from {}
		\rlap{3>3.\xx{pfv}.break} {} {} {} {}
		{} that fat -little {}
		{} just his cheek inside -for {} //
	\glft	‘Her having gone out, his uncle’s wife, he broke off from the end of
		the side of meat that hangs on the rafters a little piece of fat,
		just for the inside of his cheek.’
		//
\endgl
\xe

\ex\label{ex:91-12-where-is-this-big-thing}%
\exmn{267.8}%
\begingl
	\glpreamble	Hagū′sa ʟᴀx yē′ỵakugā′ỵi ᴀt. //
	\glpreamble	«\!Ha goosú tlax̱ yéi ÿakoogéiÿi át?\!» //
	\gla	«\!Ha \rlap{goosú} @ {} @ {} {} {} tlax̱ yéi @ \rlap{ÿakoogéiÿi} @ {} @ {} @ {} @ {} @ {} @ {} {} át?\!» {} //
	\glb	\pqp{}ha goo =s =ú {} {} tlax̱ yéi= ÿ- k- u- i- \rt[¹]{ge} -μμH -i {} át {} //
	\glc	\pqp{}well where =\xx{q} =\xx{locp} {}[\pr{DP} {}[\pr{CP} very thus= \xx{qual}- \xx{cmpv}-
			\xx{irr}- \xx{stv}- \rt[¹]{big} -\xx{var} -\xx{rel} {}] thing {}] //
	\gld	\pqp{}well where \•\xx{q} \•is.at {} {} very thus
			\rlap{\xx{gcnj}.\xx{stv}·\xx{impfv}.big} {} {} {} {} {} -that {} thing {} //
	\glft	‘“Well where is this thing that is so very big?”’
		//
\endgl
\xe

\citeauthor{swanton:1909}’s gloss and translation of (\lastx) is quite wide of the mark and suggests that he had a problem understanding the interpretation given by his consultant.
The sentence is clearly a question with the phrase \fm{goosú} ‘where is it’ and the relative clause is nothing like \citeauthor{swanton:1909}’s “yet he only broke off so much”.
Instead (\lastx) appears to be a question uttered (or thought) by the uncle’s wife: she has noticed the small amount of fat missing from (\ref{ex:91-11-break-off-fatlet-for-cheek}) and is talking about it like it is a large piece.
Alternatively, this could be a rhetorical question posed by the narrator which exaggerates the amount of fat to reflect the attitude of the uncle’s wife.
In either case, (\lastx) is a question not a statement and it describes the amount as large rather than moderate or small.

\ex\label{ex:91-13-she-looked-around-uncles-wife}%
\exmn{267.9}%
\begingl
	\glpreamble	ᴀt aoʟ̣igê′n dokā′k cᴀttc. //
	\glpreamble	Át awdlig̱én du káak shátch. //
	\gla	{} \rlap{Át} @ {} {} \rlap{awdlig̱én} @ {} @ {} @ {} @ {} @ {} @ {}
		{} du káak \rlap{shátch.} @ {} {} //
	\glb	{} á -t {} a- wu- d- l- i- \rt[²]{g̱eͥn} -\xx{var}
		{} du káak shát -ch {} //
	\glc	{}[\pr{PP} \xx{3n} -\xx{pnct} {}] \xx{xpl}- \xx{pfv}- \xx{mid}- \xx{xtn}- \xx{stv}- \rt[²]{look} -\xx{var}
		{}[\pr{DP} \xx{3h·pss} mat·uncle wife -\xx{erg} {}] //
	\gld	{} there -around {} \rlap{\xx{zcnj}.\xx{pfv}.look} {} {} {} {} {} {}
		{} his mat·uncle’s wife {} {} //
	\glft	‘She looked around, his uncle’s wife.’
		//
\endgl
\xe

\ex\label{ex:91-14-end-of-it-isnt}%
\exmn{267.9}%
\begingl
	\glpreamble	ʟe gwâyᴀ′ł acū′wua. //
	\glpreamble	Tle gwáayá l a shoowú aa. //
	\gla	Tle @ \rlap{gwáayá} @ {} @ {} @ \•l {} {} a \rlap{shoowú} @ {} {} aa. {} //
	\glb	tle= gwá= á -yá =l {} {} a shú -í {} aa {} //
	\glc	then= \xx{mir}= \xx{foc} -\xx{prox} =\xx{neg}
		{}[\pr{DP} {}[\pr{DP} \xx{3n·pss} end -\xx{pss} {}] \xx{part} {}] //
	\gld	then really\• \rlap{it.is} {} \•not {} {} its end -of {} some {} //
	\glft	‘Then really there isn’t some of the end of it.’
		//
\endgl
\xe

The sequence \fm{tle gwáayá l} in (\lastx) can be interpreted as an instance of negative \fm{tléil} ‘not’ interrupted by the mirative focus particle \fm{gwáayá}.
This is suggested in the segmentation and gloss by giving \fm{tle} and \fm{l} as clitics.
The orthographic representation maintains \fm{tle} and \fm{l} as separate words, but if the analysis is correct then a more accurate representation would be \fm{tléigwáayál}.
Compare the \fm{tléigíl} in chapter \ref{ch:106-low-caste-name} sentence \ref{ex:106-48-doncha-know} (p.\ \pageref{ex:106-48-doncha-know}) which has a similar interruption of negative \fm{tléil} with the polar yes/no question particle \fm{gí}. 

\ex\label{ex:91-15-was-it-you}%
\exmn{267.9}%
\begingl
	\glpreamble	“Waē′tc gâwê′ ge yē′sînî,” //
	\glpreamble	«\!Wa.éich gwáawégé yéi yisinee?\!» //
	\gla	{} \llap{«\!}\rlap{Wa.éich} @ {} {} \rlap{gwáawégé} @ {} @ {} @ {}
		yéi @ \rlap{yisinee?\!»} @ {} @ {} @ {} @ {} @ {} //
	\glb	{} wa.é -ch {} gwá= á -wé =gé
		yéi= wu- i- s- i- \rt[¹]{niͤʰ} -μμL //
	\glc	{}[\pr{DP} \xx{2sg} -\xx{erg} {}] \xx{mir}= \xx{foc} -\xx{mdst} =\xx{yn}
		thus= \xx{pfv}- \xx{2sg·s}- \xx{csv}- \xx{stv}- \rt[¹]{occur} -\xx{var} //
	\gld	{} \rlap{you·\xx{sg}} {} {} really\• \rlap{it.is} {} \•yes/no?
		thus\• \rlap{\xx{ncnj}.\xx{pfv}.you·\xx{sg}.make.happen} {} {} {} {} {}  //
	\glft	‘“Was it really you who did this?”’
		//
\endgl
\xe

\ex\label{ex:91-16-so-she-said-to-him}%
\exmn{267.10}%
\begingl
	\glpreamble	ʟe yū′aỵaosîqa duxo′x qēł k!ᴀtsk!ᵘ. //
	\glpreamble	tle yóo aÿawsiḵaa du x̱úx̱ kéilkʼátskʼu. //
	\gla	tle yóo @ \rlap{aÿawsiḵaa} @ {} @ {} @ {} @ {} @ {} @ {} 
		{} du x̱úx̱ \rlap{kéilkʼátskʼu.} @ {} @ {} @ {} @ {} {} //
	\glb	tle yóo= a- ÿ- wu- s- i- \rt[¹]{ḵa} -μμL
		{} du x̱úx̱ kéilkʼ- kʼí- ÿáts -kʼʷ -í {} //
	\glc	then \xx{quot}= \xx{arg}- \xx{qual}- \xx{pfv}- \xx{csv}- \xx{stv}- \rt[¹]{say} -\xx{var}
		{}[\pr{DP} \xx{3h·pss} husband sor·neph- base- child -\xx{dim} -\xx{pss} {}] //
	\gld	then so \rlap{3>3.\xx{ncnj}.\xx{pfv}.say·to} {} {} {} {} {} {}
		{} her husband’s nephew- \rlap{young·boy} {} {} {} {} //
	\glft	‘so she said to her husband’s young nephew.’
		//
\endgl
\xe

\ex\label{ex:91-17-crying-he-said-no}%
\exmn{267.10}%
\begingl
	\glpreamble	Tc!a adē′ g̣axyē′dê awe′ ye aỵa′osîqa, “ʟēk!.” //
	\glpreamble	Chʼa aadé g̱ax̱yéide áwé yéi aÿawsiḵaa «\!Tléikʼ\!». //
	\gla	Chʼa {} \rlap{aadé} @ {} {}
		{} \rlap{g̱ax̱yéide} @ {} @ {} {} \rlap{áwé} @ {}
		yéi @ \rlap{aÿawsiḵaa} @ {} @ {} @ {} @ {} @ {} @ {}
		«\!Tléíkʼ\!». //
	\glb	chʼa {} á -dé {}
		{} g̱áax̱- yé -dé {} á -wé
		yéi= a- ÿ- wu- s- i- \rt[¹]{ḵa} -μμL
		\pqp{}tléikʼ //
	\glc	just {}[\pr{PP} \xx{3n} -\xx{all} {}]
		{}[\pr{PP} crying- way -\xx{all} {}] \xx{foc} -\xx{mdst}
		thus= \xx{arg}- \xx{qual}- \xx{pfv}- \xx{csv}- \xx{stv}- \rt[¹]{say} -\xx{var}
		\pqp{}no //
	\gld	just {} that -way {}
		{} crying way -to {} \rlap{it.is} {}
		thus\• \rlap{3>3.\xx{ncnj}.\xx{pfv}.say·to} {} {} {} {} {} {}
		\pqp{}no //
	\glft	‘Just that way, it was crying that he said to her “no”.’
		//
\endgl
\xe

The phrase \fm{g̱ax̱yéide} in (\lastx) is not attested elsewhere.
Its structure is unusual but its interpretation is straightforward.
It can be compared to the well known but more or less frozen \fm{chʼa g̱unayéide} ‘differently, in a different direction/manner’ and \fm{woosh diyéide} ‘differently from each other, variously’.
The phrase \fm{g̱ax̱yéide} could be an innovation based on the same pattern, but it plausibly reflects a now uncommon mechanism for forming manner adverbs.

\ex\label{ex:91-18-she-felt-inside-his-cheek}%
\exmn{268.1}%
\begingl
	\glpreamble	Tc!uʟe′ a′wᴀc tū′dî wūcî′ doxo′x qēłk!. //
	\glpreamble	Chʼu tle a wásh tóode wooshee du x̱úx̱ kéilkʼ. //
	\gla	Chʼu tle {} a wásh \rlap{tóode} @ {} {}
		\rlap{wooshee} @ {} @ {} @ {}
		{} du x̱úx̱ kéilkʼ. {} //
	\glb	chʼu tle {} a wásh tú -dé {}
		wu- i- \rt[¹]{shiʰ} -μμL
		{} du x̱úx̱ kéilkʼ {} //
	\glc	just then {}[\pr{PP} \xx{3n·pss} cheek inside -\xx{all} {}]
		\xx{pfv}- \xx{stv}- \rt[¹]{reach·for} -\xx{var}
		{}[\pr{DP} \xx{3h·pss} husband sor·neph {}] //
	\gld	just then {} his cheek inside -to {}
		\rlap{\xx{ncnj}.\xx{pfv}.feel} {} {} {}
		{} her husband’s nephew {} //
	\glft	‘Just then she felt inside his cheek, her husband’s nephew.’
		//
\endgl
\xe

\ex\label{ex:91-19-why-not-go-around-on-mountain}%
\exmn{268.2}%
\begingl
	\glpreamble	“Wāsᴀł cāỵadat igu′t?” //
	\glpreamble	«\!Wáa sá l shaa ÿadaat ÿeegoot?\!» //
	\gla	{} \llap{«\!}Wáa sá {}
		l {} shaa \rlap{ÿadaat} @ {} @ {} {}
		\rlap{ÿeegoot?\!»} @ {} @ {} @ {} @ {} //
	\glb	{} wáa sá {}
		l {} shaa ÿá- daa -t {}
		wu- i- i- \rt[¹]{gut} -μμL //
	\glc	{}[\pr{QP} how \xx{q} {}]
		\xx{neg} {}[\pr{PP} mountain face- around -\xx{pnct} {}]
		\xx{pfv}- \xx{2sg·s}- \xx{stv}- \rt[¹]{go·\xx{sg}} -\xx{var} //
	\gld	{} how \xx{q} {}
		not {} mountain face- around -on {}
		\rlap{\xx{ncnj}.\xx{pfv}.you·\xx{sg}.go·\xx{sg}} {} {} {} {} //
	\glft	‘“How have you not gone around on the face of the mountain?”’
		//
\endgl
\xe

\ex\label{ex:91-20-scratch-inside-cheek}%
\exmn{268.2}%
\begingl
	\glpreamble	Awᴀctu′ akᴀʟ̣ā′k. //
	\glpreamble	A washtú akadlaakw. //
	\gla	{} A \rlap{washtú} @ {} {}
		\rlap{akadlaakw.} @ {} @ {} @ {} //
	\glb	{} a wásh- tú {}
		a- k- \rt[²]{dlakw} -μμL //
	\glc	{}[\pr{DP} \xx{3n·pss} cheek- inside {}]
		\xx{arg}- \xx{qual}- \rt[²]{scratch} -\xx{var} //
	\gld	{} his cheek- inside {}
		\rlap{3>3.\xx{zcnj}.\xx{impfv}.scratch} {} {} {} //
	\glft	‘She scratches the inside of his cheek.’
		//
\endgl
\xe

\ex\label{ex:91-21-blood-wide-along-mouth}%
\exmn{268.2}%
\begingl
	\glpreamble	Doq!ē′nᴀx cî ỵē′kuwūq. //
	\glpreamble	Du x̱ʼéináx̱ shí yéi koowóox̱ʼ. //
	\gla	{} Du \rlap{x̱ʼéináx̱} @ {} {}
		{} shí {}
		yéi @ \rlap{koowóox̱ʼ.} @ {} @ {} @ {} @ {} //
	\glb	{} du x̱ʼé -náx̱ {}
		{} shí {}
		yéi= k- u- i- \rt[¹]{wux̱ʼ} -μμH //
	\glc	{}[\pr{PP} \xx{3h·pss} mouth- \xx{perl} {}]
		{}[\pr{DP} blood {}]
		thus= \xx{cmpv}- \xx{irr}- \xx{stv}- \rt[¹]{wide} -\xx{var} //
	\gld	{} his mouth -along {}
		{} blood {}
		thus\• \rlap{\xx{cmpv}.\xx{gcnj}.\xx{stv}·\xx{impfv}.wide} {} {} {} {} //
	\glft	‘Blood is wide along his mouth.’
		//
\endgl
\xe

\ex\label{ex:91-22-pulled-box-to-self}%
\exmn{268.3}%
\begingl
	\glpreamble	Tc!a adî′ g̣ᴀxỵē′de awe′ dukā′k qō′gu tūt aosīî′n. //
	\glpreamble	Chʼa aadé g̱ax̱yéide áwé du káak ḵóogu tóot awsi.ín. //
	\gla	Chʼa {} \rlap{aadé} @ {} {}
		{} \rlap{g̱ax̱yéide} @ {} @ {} {} \rlap{áwé} @ {}
		{} du káak \rlap{ḵóogu} @ {} {} +
		{} {} \rlap{tóot} @ {} {}
		\rlap{awsi.ín.} @ {} @ {} @ {} @ {} @ {} //
	\glb	chʼa {} á -dé {}
		{} g̱áax̱- yé -dé {} á -wé
		{} du káak ḵóok -í {}
		{} {} tú -t {}
		a- wu- s- i- \rt[²]{.in} -μH //
	\glc	just {}[\pr{PP} \xx{3n} -\xx{all} {}]
		{}[\pr{PP} crying- way -\xx{all} {}] \xx{foc} -\xx{mdst}
		{}[\pr{DP} \xx{3h·pss} mat·uncle box -\xx{pss} {}]
		{}[\pr{PP} \xx{rflx·pss} inside -\xx{pnct} {}]
		\xx{arg}- \xx{pfv}- \xx{xtn}- \xx{stv}- \rt[²]{gather} -\xx{var} //
	\gld	just {} that -way {}
		{} crying- way -to {} \rlap{it.is} {}
		{} his mat·uncle’s box {} {}
		{} self’s inside -to {}
		\rlap{3>3.\xx{zcnj}.\xx{pfv}.handle·filled} {} {} {} {} {} //
	\glft	‘Just that way, it was crying that he pulled his uncle’s box to himself.’
		//
\endgl
\xe

\FIXME{Comment on \fm{\rt{.in}} ‘gather’ versus ‘handle filled container’.
This may have discussion in one of the other narratives; if that is after this one then move that discussion here and leave a pointer there.}

\FIXME{Comment on apparent lack of \fm{d-}.
We would normally expect this since the reflexive possessor is covert.
Could just be a transcription error since the forms \fm{awdzi.ín} [\ipa{ʔàw.tsì.ˈʔín}] and \fm{awsi.ín}  [\ipa{ʔàw.sì.ˈʔín}] would easily be confused by \citeauthor{swanton:1909}.}

\FIXME{Comment on \fm{tóo-t}.
This is an unusual use of \fm{tú} ‘inside of (hollow object)’, we would expect something more like \fm{jee-t} ‘to the possession of’.
This would be \orth{djīt} in \citeauthor{swanton:1909}’s transcription but \orth{tūt} is hard to relate as a misreading of \orth{djīt}.
Are there any other plausible inalienable nouns in this context?}

\ex\label{ex:91-23-from-within-pick-up-whetstone}%
\exmn{268.3}%
\begingl
	\glpreamble	Aỵîˈkdᴀx ke ā′watī dokā′k ỵaỵī′nᴀk!o. //
	\glpreamble	A ÿíkdáx̱ kei aawatée du káak ÿaÿéinakʼu. //
	\gla	{} A \rlap{ÿíkdáx̱} @ {} {}
		kei @ \rlap{aawatée} @ {} @ {} @ {} @ {}
		{} du káak \rlap{ÿaÿéinakʼu.} @ {} @ {} @ {} @ {} @ {} @ {} {} //
	\glb	{} a ÿíᵏ -dáx̱ {}
		kei= a- wu- i- \rt[²]{ti} -μμH
		{} du káak ÿa- \rt{ÿa} -eH -n -aa⁽ʷ⁾ -kʼ -í {} //
	\glc	{}[\pr{PP} \xx{3n·pss} within -\xx{abl} {}]
		up= \xx{arg}- \xx{pfv}- \xx{stv}- \rt[²]{handle} -\xx{var}
		{}[\pr{DP} \xx{3h·pss} mat·uncle face- \rt[²]{spread}
			-\xx{var} -\xx{nsfx} -\xx{inst} -\xx{dim} -\xx{pss} {}] //
	\gld	{} its within -from {}
		up \rlap{3>3.\xx{zcnj}.\xx{pfv}.handle} {} {} {} {}
		{} his mat·uncle’s \rlap{whetstone} {} {} {} {} -little {} {} //
	\glft	‘From within it he picked it up, his uncle’s little whetstone.’
		//
\endgl
\xe

\FIXME{Discuss \fm{ÿaÿéinaa}.
The root appears to be \fm{\rt[²]{ÿa}} ‘spread out, lower’ and is so identified by \textcites[03/110]{leer:1973}[12]{leer:1978b}, but this is semantically puzzling.
The \fm{-aa} suffix forms instrument nouns; it is generally not seen with occult labialization but this seems to be the case here.
Check for other diminutives of \fm{-aa} to see if they also labialize.}

\ex\label{ex:91-24-uncle-absent}%
\exmn{268.4}%
\begingl
	\glpreamble	Dukā′k ko uyê′x. //
	\glpreamble	Du káak ḵwa uyéx̱. //
	\gla	{} Du káak {} ḵwa
		\rlap{uyéx̱.} @ {} @ {} //
	\glb	{} du káak {} ḵu.aa
		u- \rt[¹]{yex̱} -μH //
	\glc	{}[\pr{DP} \xx{3h·pss} mat·uncle {}] \xx{contr}
		\xx{irr}- \rt[¹]{absent} -\xx{var} //
	\gld	{} his mat·uncle {} however
		\rlap{\xx{irr}.\xx{ncnj}.\xx{impfv}.absent} //
	\glft	‘His uncle however is absent.’
		//
\endgl
\xe

\FIXME{Comment on how \fm{uyéx̱} apparently is an activity rather than a state since it’s not \fm[*]{ooyéx̱} \parencites[03/77]{leer:1973}[179]{leer:1976}.
The irrealis is probably pejorative in function, for which see dissertation 6§4.2.4.}

\section{Paragraph 3}\label{sec:91-para-3}

\ex\label{ex:91-25-began-go-to-forest}%
\exmn{268.5}%
\begingl
	\glpreamble	ʟe g̣onaye′ uwagu′t ᴀtgotū′dî. //
	\glpreamble	Tle g̱unayéi uwagút atgutóode. //
	\gla	Tle g̱unayéi @ \rlap{uwagút} @ {} @ {} @ {}
		{} \rlap{atgutóode.} @ {} @ {} @ {} {} //
	\glb	tle g̱unayéi= u- i- \rt[¹]{gut} -μH
		{} at= gú- tú -dé {} //
	\glc	then \xx{incep}= \xx{zpfv}- \xx{stv}- \rt[¹]{go·\xx{sg}} -\xx{var}
		{}[\pr{PP} \xx{3n·pss}= base- inside -\xx{all} {}] //
	\gld	then begin\• \rlap{\xx{pfv}.go·\xx{sg}} {} {} {}
		{} \rlap{forest} {} {} -to {} //
	\glft	‘So he began going to the forest.’
		//
\endgl
\xe

\ex\label{ex:91-26-riverbank-at-went-inland}%
\exmn{268.5}%
\begingl
	\glpreamble	Hīn yax hî′taq uwagu′t //
	\glpreamble	Héen yaax̱í daaḵ uwagút; //
	\gla	{} Héen \rlap{yaax̱í} @ {} {} daaḵ @ \rlap{uwagút;} @ {} @ {} @ {} //
	\glb	{} héen yaax̱ -í {} daaḵ= u- i- \rt[¹]{gut} -μH //
	\glc	{}[\pr{PP} water side -\xx{loc} {}] inland= \xx{zpfv}- \xx{stv}- \rt[¹]{go·\xx{sg}} -\xx{var}  //
	\gld	{} river bank -at {} inland\• \rlap{\xx{pfv}.go·\xx{sg}} {} {} {} //
	\glft	‘He went inland on a riverbank;’
		//
\endgl
\xe

\ex\label{ex:91-27-with-whetstone-went-out-sandbar}%
\exmn{268.5}%
\begingl
	\glpreamble	weyaỵī′na x̣ᴀkᵘ ka an dak uwagu′t. //
	\glpreamble	wé ÿaÿéinaa, xákw káa aan daaḵ uwagút. //
	\gla	{} wé \rlap{ÿaÿéinaa} @ {} @ {} @ {} @ {} {}
		{} xákw \rlap{káa} @ {} {}
		{} \rlap{aan} @ {} {}
		daak @ \rlap{uwagút.} @ {} @ {} @ {} //
	\glb	{} wé ÿa- \rt{ÿa} -eH -n -aa {}
		{} xákw ká -μμL {}
		{} á -n {}
		daak= u- i- \rt[¹]{gut} -μH //
	\glc	{}[\pr{DP} \xx{mdst} face- \rt[²]{spread} -\xx{var} -\xx{nsfx} -\xx{inst} {}]
		{}[\pr{PP} sandbar \xx{hsfc} -\xx{loc} {}]
		{}[\pr{PP} \xx{3n} -\xx{instr} {}]
		seaward= \xx{zpfv}- \xx{stv}- \rt[¹]{go·\xx{sg}} -\xx{var} //
	\gld	{} that \rlap{whetstone} {} {} {} {} {}
		{} sandbar atop -on {}
		{} it -with {}
		seaward\• \rlap{\xx{pfv}.go·\xx{sg}} {} {} {} //
	\glft	‘the whetstone, he went out on top of a sandbar with it.’
		//
\endgl
\xe

\citeauthor{swanton:1909} runs sentences (\ref{ex:91-26-riverbank-at-went-inland}) and (\ref{ex:91-27-with-whetstone-went-out-sandbar}) together as a single utterance in his transcription.
This is grammatically implausible because both sentences contain the main verb form \fm{uwagút} ‘s/he/it went’.
Following \citeauthor{leer:1977}’s lead \parencite[17]{leer:1977}, they are given here as separate paratactically linked sentences with the semicolon at the end of (\ref{ex:91-26-riverbank-at-went-inland}) reflecting this structure.

\ex\label{ex:91-28-hammers-like-salmon}%
\exmn{268.6}%
\begingl
	\glpreamble	Akᴀt!ē′q! xāt yᴀx. //
	\glpreamble	Akatʼéix̱ʼ x̱áat yáx̱. //
	\gla	\rlap{Akatʼéix̱ʼ} @ {} @ {} @ {}
		{} x̱áat yáx̱. {} //
	\glb	a- k- \rt[²]{tʼeͥx̱ʼ} -μμH
		{} x̱áat yáx̱ {} //
	\glc	\xx{arg}- \xx{hsfc}- \rt[²]{pound} -\xx{var}
		{}[\pr{PP} salmon \xx{sim} {}] //
	\gld	\rlap{3>3.\xx{zcnj}.\xx{impfv}.pound} {} {} {}
		{} salmon like {} //
	\glft	‘He hammers it like a salmon.’
		//
\endgl
\xe

\ex\label{ex:91-29-made-nest}%
\exmn{268.6}%
\begingl
	\glpreamble	Kut awas!î′t yuhī′n yāxq!. //
	\glpreamble	Kút aawasʼít yú héen yaax̱xʼ.  //
	\gla	{} Kút {} \rlap{aawasʼít} @ {} @ {} @ {} @ {}
		{} yú héen \rlap{yaax̱xʼ.} @ {} {}//
	\glb	{} kút {} a- wu- i- \rt[²]{sʼit} -μH
		{} yú héen yaax̱ -xʼ {} //
	\glc	{}[\pr{DP} nest {}] \xx{arg}- \xx{pfv}- \xx{stv}- \rt[²]{bind} -\xx{var}
		{}[\pr{PP} \xx{dist} water side -\xx{loc} {}] //
	\gld	{} nest {} \rlap{3>3.\xx{zcnj}.\xx{pfv}.bind} {} {} {} {}
		{} that river side -at {} //
	\glft	‘He made a nest by that riverbank.’
		//
\endgl
\xe

The precise interpretation of the verb in (\lastx) is difficult.
\citeauthor{swanton:1909}’s transcription \orth{awas!î′t} suggets \fm{aawasʼít} which is exactly what \citeauthor{leer:1977} transcribes \parencite[17]{leer:1977}.
\citeauthor{swanton:1909}’s gloss is “he made” which is congruent with his translation.
But the root \fm{\rt[²]{sʼit}} is not attested anywhere with any kind of association with nests, salmon or otherwise.
Normally \fm{\rt[²]{sʼit}} is glossed as ‘bind’ or ‘screw’, appearing in forms like \fm{du jín kax̱waasʼít} ‘I bound up his hand’ \parencite[30.217]{story-naish:1973}, \fm{x̱aanásʼ wududlisʼít} ‘people tied a raft’ \parencite[230.3262]{story-naish:1973}, \fm{shakawdisʼít} ‘s/he has hair in a bun’ \parencite[09/232]{leer:1973}, and \fm{kasʼéet} ‘screw (fastener)’ \parencite[\textsc{t}·32]{leer:2001}.
If the noun \fm{kút} is taken to refer to a bird’s nest then the use of \fm{\rt[²]{sʼit}} ‘bind’ might be interpretable as a reference to the weaving together of sticks, but this is usually expressed with the root \fm{\rt[²]{.ak}} ‘weave’.
Furthermore, the implicit referent here is salmon, not birds, and the nest of a salmon (a redd) is a shallow hollow in gravel made by waving the tail back and forth.
If \orth{s!î′t} is a mistranscription for something else then some phonologically possible roots are: \fm{\rt{siʼt}} ‘braid’, \fm{\rt{shitʼ}} ‘crowd, shove’, \fm{\rt{sʼatʼ}} ‘left (handed)’, \fm{\rt{tsʼetʼ}} ‘flense’, \fm{\rt{tsʼitʼ}} ‘full of liquid’, \fm{\rt{tsʼuʼt}} ‘close tightly, seal’, \fm{\rt{tsʼutsʼ}} ‘(fish) jerk line’, \fm{\rt{lit}} ‘slide, slit, dorsal fin’, \fm{\rt{lʼit}} ‘tail’, \fm{\rt{tlʼit}} ‘throw away, pitch, shovel’.
None of these is particularly compelling, either because their semantics does not fit with salmon nests or because they are only nominal and not attested in verb forms.
Further investigation is needed to make sense of \orth{awas!î′t}.
The current English translation follows \citeauthor{swanton:1909} with the English verb ‘make’.

\ex\label{ex:91-30-overnight-on-nest}%
\exmn{268.7}%
\begingl
	\glpreamble	Aka′ uwaxe′ yuku′t.′ //
	\glpreamble	A káa uwax̱éi yú kút. //
	\gla	{} A \rlap{káa} @ {} {}
		\rlap{uwax̱éi} @ {} @ {} @ {}
		{} yú kút. {} //
	\glb	{} a ká -μμL {}
		u- i- \rt[¹]{x̱eͥ} -μμH
		{} yú kút {} //
	\glc	{}[\pr{PP} \xx{3n·pss} \xx{hsfc} -\xx{loc} {}]
		\xx{zpfv}- \xx{stv}- \rt[¹]{overnight} -\xx{var}
		{}[\pr{DP} \xx{dist} nest {}] //
	\gld	{} its\ix{i} atop -at {}
		\rlap{\xx{pfv}.overnight} {} {} {}
		{} that nest\ix{i} {} //
	\glft	‘He spent the night on it, that nest.’
		//
\endgl
\xe

\ex\label{ex:91-31-dream-thus}%
\exmn{268.7}%
\begingl
	\glpreamble	Dutcū′nî ayu′ yē ỵatî′. //
	\glpreamble	Du jooní áyú yéi ÿatee. //
	\gla	{} Du \rlap{jooní} @ {} {} \rlap{áyú} @ {}
		yéi @ \rlap{ÿatee.} {} {} //
	\glb	{} du joon -í {} á -yú
		yéi= i- \rt[¹]{tiʰ} -μμL //
	\glc	{}[\pr{DP} \xx{3h·pss} dream -\xx{pss} {}] \xx{foc} -\xx{dist}
		thus= \xx{stv}- \rt[¹]{be} -\xx{var} //
	\gld	{} his dream {} {} \rlap{it.is} {}
		thus \rlap{\xx{ncnj}.\xx{stv}·\xx{impfv}.be} {} {} //
	\glft	‘It is his dream that is thus.’
		//
\endgl
\xe

\ex\label{ex:91-32-say-make-swim-in-river}%
\exmn{268.7}%
\begingl
	\glpreamble	Ye daỵa′doqa, “Hīnỵî′x nᴀsq!ā′q.” //
	\glpreamble	Yéi daaÿaduḵá «\!Héen ÿíx̱ nasxʼaak\!». //
	\gla	Yéi @ \rlap{daaÿaduḵá} @ {} @ {} @ {} @ {} @ {} +
		{} {} \llap{«\!}Héen \rlap{ÿíx̱} @ {} {}
			\rlap{nasxʼaak.\!»} @ {} @ {} @ {} @ {} {} //
	\glb	yéi= daa- ÿ- du- d- \rt[¹]{ḵa} -μH
		{} {} héen ÿíᵏ -x̱ {}
			n- {} s- \rt[¹]{xʼak} -μμL {} //
	\glc	thus= around- \xx{qual}- \xx{4h·s}- \xx{mid}- \rt[¹]{say} -\xx{var}
		{}[\pr{CP} {}[\pr{PP} water within -\xx{pert} {}]
			\xx{ncnj}- \xx{2sg·s}\· \xx{csv}- \rt[¹]{fish·swim} -\xx{var} {}] //
	\gld	thus \rlap{\xx{ncnj}.\xx{impfv}.one.say·to} {} {} {} {} {}
		{} {} river within -at {}
			\rlap{\xx{imp}.you·\xx{sg}.make.fish·swim} {} {} {} {} {} {} //
	\glft	‘Someone says to him “Make it swim within the river”.’
		//
\endgl
\xe

\ex\label{ex:91-33-inner-spirit-thought-to-him}%
\exmn{268.8}%
\begingl
	\glpreamble	Xᴀtc duyuyē′k ᴀseyu′ ye acī′t tū′ditᴀn. //
	\glpreamble	X̱ách du tuyéigi áséyú yéi ash eet toowditán. //
	\gla	X̱ách {} du \rlap{tuyéigi} @ {} @ {} {} \rlap{áséyú} @ {} @ {} +
		yéi {} ash \rlap{eet} @ {} {} \rlap{toowditán.} @ {} @ {} @ {} @ {} @ {} //
	\glb	x̱áju {} du tú- yéik -í {} á -sí -yú
		yéi= {} ash ee -t {} tu- wu- d- i- \rt[²]{tan} -μH //
	\glc	actually {}[\pr{DP} \xx{3h·pss} inside- spirit -\xx{pss} {}] \xx{foc} -\xx{dub} -\xx{dist}
		thus {}[\pr{PP} \xx{3prx} \xx{base} -\xx{pnct} {}]
			mind- \xx{pfv}- \xx{mid}- \xx{stv}- \rt[²]{hdl·w/e} -\xx{var} //
	\gld	actually {} his \rlap{inner.spirit} {} {} {} \rlap{apparently.it.is} {} {}
		thus= {} him {} -to {} \rlap{\xx{zcnj}.\xx{pfv}.think} {} {} {} {} {} //
	\glft	‘Actually it is apparently his inner spirit that thought this to him.’
		//
\endgl
\xe

\FIXME{Discuss interpretations of \orth{duyuyē′k}.}

\section{Paragraph 4}\label{sec:91-para-4}

\ex\label{ex:91-34-after-beach-uncle}%
\exmn{268.9}%
\begingl
	\glpreamble	Duitī′q! ỵēq uwagu′t dukā′k. //
	\glpreamble	Du eetéexʼ ÿeiḵ uwagút du káak. //
	\gla	{} Du \rlap{eetéexʼ} @ {} {}
		ÿeiḵ @ \rlap{uwagút} @ {} @ {} @ {}
		{} du káak. {} //
	\glb	{} du eetí -xʼ {}
		ÿeiḵ= u- i- \rt[¹]{gut} -μH
		{} du káak {} //
	\glc	{}[\pr{PP} \xx{3h·pss} remains -\xx{loc} {}]
		beach= \xx{zpfv}- \xx{stv}- \rt[¹]{go·\xx{sg}} -\xx{var}
		{}[\pr{DP} \xx{3h·pss} mat·uncle {}] //
	\gld	{} his absence -at {}
		beach \rlap{\xx{pfv}.go·\xx{sg}} {} {} {}
		{} his mat·uncle {} //
	\glft	‘In his absence he came to the beach, his uncle.’
		//
\endgl
\xe

\ex\label{ex:91-35-asks-wife-where-nephew}%
\exmn{268.9}%
\begingl
	\glpreamble	Aq!ewū′s! ducᴀ′t “Gūsu′ ho ᴀxqē′łk!.” //
	\glpreamble	Ax̱ʼeiwóosʼ du shát «\!Goosú hú, ax̱ kéilkʼ?\!». //
	\gla	\rlap{Ax̱ʼeiwóosʼ} @ {} @ {} @ {}
		{} du shát {}
		{} \llap{«\!}\rlap{Goosú} @ {} @ {} {} hú, {}
			{} ax̱ kéilkʼ?\!». {} {} //
	\glb	a- x̱ʼe- \rt[¹]{wuͣsʼ} -μμH
		{} du shát {}
		{} goo =s =ú {} hú {}
			{} ax̱ kéilkʼ {} //
	\glc	\xx{arg}- mouth- \rt[²]{ask} -\xx{var}
		{}[\pr{DP} \xx{3h·pss} wife {}]
		{}[\pr{CP} where =\xx{q} =\xx{locp} {}[\pr{DP} \xx{3h} {}]
			{}[\pr{DP} \xx{1sg·pss} sor·neph {}] {}] //
	\gld	\rlap{3>3.\xx{ncnj}.\xx{impfv}.ask} {} {} {}
		{} his wife {}
		{} where \•\xx{q} \•is.at {} him {}
			{} my nephew {} {} //
	\glft	‘He asks his wife “Where is my nephew?”.’
		//
\endgl
\xe

\ex\label{ex:91-36-went-thattaway}%
\exmn{268.10}%
\begingl
	\glpreamble	“Wé′de awe′ tcū′nēt a′ołi.āt.” //
	\glpreamble	«\!Wéide áwé chooneit awli.aat.\!» //
	\gla	{} \llap{«\!}\rlap{Wéide} @ {} {} \rlap{áwé} @ {}
		{} {} \rlap{chooneit} @ {} @ {} {} {} {}
		\rlap{awli.aat.\!»} @ {} @ {} @ {} @ {} @ {} //
	\glb	{} wé -dé {} á -wé
		{} {} \rt[¹]{chun} -μμL -i {} át {}
		a- wu- l- i- \rt[¹]{.at} -μμL //
	\glc	{}[\pr{PP} \xx{mdst} -\xx{all} {}] \xx{foc} -\xx{mdst}
		{}[\pr{DP} {}[\pr{CP} \rt[¹]{wound} -\xx{var} -\xx{rel} {}] thing {}]
		\xx{arg}- \xx{pfv}- \xx{csv}- \xx{stv}- \rt[¹]{go·\xx{pl}} -\xx{var} //
	\gld	{} that -way {} \rlap{it.is} {}
		{} {} \rlap{arrow} {} {} {} {} {}
		\rlap{3>3.\xx{ncnj}.\xx{pfv}.make.go·\xx{pl}} {} {} {} {} {} //
	\glft	‘“It is that way that he made arrows go.”’
		//
\endgl
\xe

\citeauthor{swanton:1909} translates the sentence in (\lastx) with a comitative expression “with his bow and arrows”.
The Tlingit has no \fm{tin}, \fm{teen}, or \fm{een} ‘with’ following \fm{chooneit} so unless the instrumental postposition was either missed or not spoken, this is not actually a comitative structure.
The monovalent verb root \fm{\rt[¹]{.at}} ‘plural go’, the three-on-three argument prefix \fm{a-}, and the \fm{l-} prefix together suggest that the verb is actually causative.
The \fm{chooneit} then is the object of the verb and the subject is implicitly the protagonist.

\section{Paragraph 5}\label{sec:91-para-5}

\ex\label{ex:91-37-upstream-made-more-nests}%
\exmn{268.11}%
\begingl
	\glpreamble	Nā′nāq! ke gū′tawe ts!u′a uwas!ît weku′t sᴀkᵘ. //
	\glpreamble	Naanaaxʼ kei góot áwé tsu aa uwasʼít, wé kút sákw. //
	\gla	{} {} \rlap{Naanaaxʼ} @ {} @ {} {}
			kei @ \rlap{góot} @ {} @ {} @ {} {} \rlap{áwé} @ {}
		tsu aa @ \rlap{uwasʼít,} @ {} @ {} @ {}
		{} wé kút sákw. {} //
	\glb	{} {} naa- niÿaa -xʼ {}
			kei= {} \rt[¹]{gut} -μμH {} {} á -wé
		tsu aa= u- i- \rt[²]{sʼit} -μH
		{} wé kút sákw {} //
	\glc	{}[\pr{CP} {}[\pr{PP} upstream- dir’n -\xx{loc} {}]
			up= \xx{zcnj}\· \rt[¹]{go·\xx{sg}} -\xx{var} \·\xx{sub} {}] \xx{foc} -\xx{mdst}
		again \xx{part·o}= \xx{zpfv}- \xx{stv}- \rt[²]{bind} -\xx{var}
		{}[\pr{DP} \xx{mdst} nest \xx{fut} {}] //
	\gld	{} {} upstream- dir’n -at {}
			up \rlap{\xx{pfv}.go·\xx{sg}} {} {} {} {} \rlap{it.is} {}
		again some \rlap{\xx{pfv}.bind} {} {} {}
		{} those nests future {} //
	\glft	‘Having gone upstream, he again made some of them, those future nests.’
		//
\endgl
\xe

\ex\label{ex:91-38-call-future-little-slave}%
\exmn{268.11}%
\begingl
	\glpreamble	“Gu′xk!ᵘsᴀkᵘ,” yūˈdowasakᵘ yuqā′. //
	\glpreamble	«\!Goox̱kʼ Sákw\!» yóo duwasáakw yú ḵáa. //
	\gla	{} \llap{«\!}\rlap{Goox̱kʼ} @ {} Sákw {}
		yóo @ \rlap{duwasáakw} @ {} @ {} @ {} @ {}
		{} yú ḵáa. {} //
	\glb	{} goox̱ -kʼ sákw {}
		yóo= du- i- \rt[²]{sa} -μμH -kw
		{} yú ḵáa {} //
	\glc	{}[\pr{DP} slave -\xx{dim} \xx{fut} {}]
		\xx{quot}= \xx{4h·s}- \xx{stv}- \rt[²]{name} -\xx{var} -\xx{rep}
		{}[\pr{DP} \xx{dist} man {}] //
	\gld	{} slave -little future {}
		so \rlap{\xx{zcnj}.\xx{stv}·\xx{impfv}.ppl.call.\xx{rep}} {} {} {} {}
		{} that man {} //
	\glft	‘People call him “Future Little Slave”, that man.’
		//
\endgl
\xe

\ex\label{ex:91-39-made-eight-nests}%
\exmn{268.12}%
\begingl
	\glpreamble	Nᴀs!gaducu′ awas!î′t yuku′t. //
	\glpreamble	Nasʼgadooshú aawasʼít yu kút. //
	\gla	Nasʼgadooshú \rlap{aawasʼít} @ {} @ {} @ {} @ {}
		{} yú kút. {} //
	\glb	nasʼgadooshú a- wu- i- \rt[²]{sʼit} -μH
		{} yú kút {} //
	\glc	eight \xx{arg}- \xx{pfv}- \xx{stv}- \rt[²]{bind} -\xx{var}
		{}[\pr{DP} \xx{dist} nest {}] //
	\gld	eight \rlap{3>3.\xx{zcnj}.\xx{pfv}.bind} {} {} {} {}
		{} those nests {} //
	\glft	‘He made eight, those nests.’
		//
\endgl
\xe

The numeral \fm{nasʼgadooshú} ‘eight’ is one of three polysyllabic numerals which are composed from a monosyllabic numeral and an opaque suffix.
These three numerals are \fm{tleidooshú} [\ipa{tɬʰèː.ˈtùː.ʃú}] ‘six’ from \fm{tléixʼ} [\ipa{tɬʰéːxʼ}] ‘one’, \fm{dax̱.adooshú} [\ipa{tàχ.ʔà.ˈtùː.ʃú}] ‘seven’ from \fm{déix̱} [\ipa{téːχ}] ‘two’, and \fm{nasʼgadooshú} [\ipa{nàsʼ.kà.ˈtùː.ʃú}] ‘eight’ from \fm{násʼk} [\ipa{násʼk}] ‘three’.
These forms all imply a suffix \fm{-₍.₎adooshú} [\ipa{(ʔ)à.ˈtùː.ʃú}].
Etymologically this suffix is from a verb phrase \fm{át uwashóo} ‘it extends, ends there’ as in \fm{dei atgutóot uwashóo} ‘the road extends to the forest’ \parencite[84.1034]{story-naish:1973}.
Various Southern and Tongass Tlingit forms of these numbers illustrate the connection more clearly: Sanya \fm{dax̱.atwooshú} [\ipa{tàχ.ʔàt.ˈwùː.ʃú}] ‘seven’, Tongass \fm{tleìtwoòshu} [\ipa{tɬʰeʰt.ˈwuʰ.ʃu}] ‘six’ and \fm{dax̱.atwoòshu} [\ipa{taχ.ʔat.ˈwuʰ.ʃu}] ‘seven’ \parencite[10.101]{leer:1973}.
From a purely synchronic perspective the suffix \fm{-₍.₎adooshú} is not decomposable and cannot be interpreted as anything more than ‘+ 5’.
In general, numerals below ten are most often left as unsegmented monoliths.

\ex\label{ex:91-40-atop-last-spirit-came}%
\exmn{268.12}%
\begingl
	\glpreamble	Hūtc!î aỵe′ kᴀ′q!awe uxyē′k uwats!ᴀ′q. //
	\glpreamble	Hóochʼi aaÿí káxʼ áwé óox̱ yéik uwatsʼáḵ. //
	\gla	{} \rlap{Hóochʼi} @ {} \rlap{aaÿí} @ {} \rlap{káxʼ} @ {} {} \rlap{áwé} @ {}
		{} \rlap{óox̱} @ {} {} 
		{} yéik {}
		\rlap{uwatsʼáḵ.} @ {} @ {} @ {} //
	\glb	{} hóochʼ -í aa -í ká -xʼ {} á -wé
		{} ú -x̱ {}
		{} yéik {}
		u- i- \rt[¹]{tsʼaḵ} -μH //
	\glc	{}[\pr{PP} last -\xx{pss} \xx{part} -\xx{pss} \xx{hsfc} -\xx{loc} {}] \xx{foc} -\xx{mdst}
		{}[\pr{PP} \xx{3h} -\xx{pert} {}]
		{}[\pr{DP} spirit {}]
		\xx{zpfv}- \xx{stv}- \rt[¹]{\xx{unkn}} -\xx{var} //
	\gld	{} last {} one {} atop -on {} \rlap{it.is} {}
		{} him -at {}
		{} spirit {}
		\rlap{\xx{pfv}.??} {} {} {} //
	\glft	‘It was on top of the last one that a spirit came to him.’
		//
\endgl
\xe

The verb in (\lastx) that \citeauthor{swanton:1909} transcribes as \orth{uwats!ᴀ′q} is still unclear.
This appears to be \fm{uwatsʼáḵ} [\ipa{ʔù.wà.ˈtsʼáq}] as has been given in the retranscription, but this entails an unknown root \fm{\rt{tsʼaḵ}}.
The only semblance of such a root is the noun \fm{tsʼáḵl} [\ipa{tsʼáqɬ}] “real black paint made from \fm{nexɩntɛ́} (graphite)” which is listed by \citeauthor{leer:1973} as a Yakutat dialect word \parencite[09/305]{leer:1973} borrowed from Eyak \fm{tsʼəɢł} < \fm{tsʼəɢ} ‘black’ \parencite[713]{krauss:1970}.\footnote{The Tlingit form in the Eyak dictionary is cited as “cf.\ Tlingit \fm{cʼαɢł} ‘real black paint, made from \fm{nexɩntɛ́}, form present in Yakutat Tlingit but not attested elsewhere in Tlingit, Constance Naish, personal communication, 18 April 1967” \parencite[713]{krauss:1970}.
A note written in \citeauthor{krauss:1970}’s hand in the Eyak dictionary reads “Amos Wallace, of Juneau, tells me, 3 June 1969, that \fm{cʼαɢł} is known and used in Juneau Tlingit, ‘black’, pigment made from \fm{nexɩntɛ́}.” \citeauthor{leer:1973}’s documentation is also from Naish like \citeauthor{krauss:1970}’s \parencite[09/305]{leer:1973}, but \citeauthor{leer:1973} does not refer to \citeauthor{krauss:1970}’s later handwritten note.
Regardless of its use south of Yakutat, \fm{tsʼáḵl} is most likely from Eyak since Tlingit lacks a form without \fm{-l} [\ipa{ɬ}] which is a somewhat productive instrumental suffix in Eyak but is at best frozen in Tlingit.
Compare \fm{táḵl} ‘hammer’ (\fm{\rt{taḵ}} ‘poke’), \fm{áatʼl} ‘cooling pit’ (\fm{\rt{.atʼ}} ‘cold’), \fm{chʼúḵʼl} ‘sand lance (\species{Ammodytes}{hexapterus}[Pallas 1814])’ (\fm{\rt{chʼuḵʼ}} unknown), \fm{x̱ákwl} ‘eagle claw hook’ (\fm{\rt{x̱aʼkw}} ‘nail, claw’), and \fm{gukÿikdínl} ‘hard of hearing’ and \fm{shantudínl} ‘idiot’ (\fm{\rt{din}} unknown but cf.\ \fm{\rt{diʼn}} ‘bother, upset, concern’).} There is no straightforward association with blackness in this narrative context and the use of the borrowed noun without its \fm{-l} would be unusual.

If the root in (\lastx) is not actually \fm{\rt{tsʼaḵ}} then there are several other possible candidates given \citeauthor{swanton:1909}’s transcription.
To start with, the form \orth{uwats!ᴀ′q} clearly indicates that the verb is a \fm{∅}-conjugation perfective with the characteristic \fm{u-}, and that it is an intransitive with a third person argument (presumably \fm{yéik}).
Any candidate root must support verbs with these constraints.

\begin{multicols}{2}
\begin{itemize}[leftmargin=0em]
\item	initial \fm{tsʼ}:
	\begin{itemize}
	\item	\fm{\rt{tsʼakw}} in \fm{kaneiltsʼákw} ‘black currants’
	\item	\fm{l-\rt{tsʼex̱}} ‘indulge craving’
	\item	\fm{l-\rt{tsʼix}} ‘damp’
	\item	\fm{l-\rt{tsʼik}} ‘whine, fuss’
	\item	\fm{\rt{tsʼikʼ}} in \fm{tsʼikʼtsʼíkʼ} ‘cart, wheelbarrow’ (< CJ)
	\item	\fm{\rt{tsʼikʼw}} ‘pinch’
	\item	\fm{\rt{tsʼikʼw}} in \fm{tsʼíkʼwti} ‘bivalve adductor muscle’
	\item	\fm{\rt{tsʼux̱}} \~\ \fm{\rt{sʼux̱}} ‘move slightly, budge’
	\end{itemize}
\item	initial \fm{sʼ}
	\begin{itemize}
	\item	\fm{\rt{sʼaḵ}} in \fm{sʼaaḵ} ‘bone’
	\item	\fm{\rt{sʼax̱}} in \fm{sʼaax̱} ‘groundhog’
	\item	\fm{\rt{sʼax̱ʼ}} in \fm{sʼáax̱ʼ} ‘gray cod’
	\item	\fm{\rt{sʼax}} in \fm{sʼáxtʼ} ‘devilsclub’
	\item	\fm{\rt{sʼax}} in \fm{sʼáx} ‘starfish’
	\item	\fm{\rt{sʼaxw}} in \fm{sʼáaxw} ‘hat’
	\item	\fm{\rt{sʼekw}} \~\ \fm{\rt{tlʼekw}} ‘duck inside’
	\item	\fm{\rt{sʼex̱}} in \fm{sʼéx̱} ‘western water hemlock’
	\item	\fm{\rt{sʼex̱}} in \fm{sʼéix̱wani} ‘wolf moss’
	\item	\fm{\rt{sʼix}} in \fm{sʼeex} ‘dirt’
	\item	\fm{\rt{sʼixʼ}} in \fm{sʼíxʼg̱aa} ‘moss’
	\item	\fm{\rt{sʼixʼ}} in \fm{sʼíxʼ} ‘dish’
	\item	\fm{\rt{sʼik}} in \fm{sʼeek} ‘black bear’
	\item	\fm{\rt{sʼiḵ}} in \fm{sʼeeḵ} \~\ \fm{sʼeiḵ} ‘smoke’
	\item	\fm{\rt{sʼix̱ʼ}} in \fm{sʼéex̱ʼ} \~\ \fm{sʼéix̱ʼ} ‘watery diarrhea’
	\item	\fm{\rt{sʼuk}} in \fm{sʼook} ‘barnacle’
	\item	\fm{\rt{sʼuḵ}} in \fm{sʼóoḵ} ‘rib’
	\item	\fm{\rt{sʼux̱ʼ}} in \fm{sʼóox̱ʼ} ‘goop, mud’
	\end{itemize}
\end{itemize}
\end{multicols}

\FIXME{Finish evaluation and discussion.}

\ex\label{ex:91-41-put-below-water-whetstone}%
\exmn{268.12}%
\begingl
	\glpreamble	Tc!ᴀ′tc!a ag̣ā′awe hīn ỵī yaawatî′ du yaỵī′naỵî. //
	\glpreamble	Cha chʼa aag̱áa áwé héen ÿee yaa aawatee du ÿaÿéinaÿi. //
	\gla	Cha chʼa {} \rlap{aag̱áa} @ {} {} \rlap{áwé} @ {}
		{} héen \rlap{ÿee} @ {} {}
		ÿaa @ \rlap{aawatee} @ {} @ {} @ {} @ {} +
		{} du \rlap{ÿaÿéinaÿi.} @ {} @ {} @ {} @ {} @ {} {} //
	\glb	cha chʼa {} á -g̱áa {} á -wé
		{} héen ÿee {} {}
		ÿaa= a- wu- i- \rt[²]{ti} -μμL
		{} du ÿa- \rt{ÿa} -eH -n -aa -í {} //
	\glc	so just {}[\pr{PP} \xx{3n} -\xx{ades} {}] \xx{foc} -\xx{mdst}
		{}[\pr{PP} water below \·\xx{loc} {}]
		along= \xx{arg}- \xx{pfv}- \xx{stv}- \rt[²]{handle} -\xx{var}
		{}[\pr{DP} \xx{3h·pss} face- \rt[²]{spread} -\xx{var} -\xx{nsfx} -\xx{inst} -\xx{pss} {}] //
	\gld	so just {} that -after {} \rlap{it.is} {}
		{} water below \·at {}
		along \rlap{3>3.\xx{ncnj}.\xx{pfv}.handle} {} {} {} {}
		{} his \rlap{whetstone} {} {} {} {} {} {} //
	\glft	‘So it was just after that that he was putting it below the water, his whetstone.’
		//
\endgl
\xe

\FIXME{Could also be \fm{ayaawatee} instead of \fm{yaa aawatee}.
What would it mean?}

\ex\label{ex:91-42-it-floated}%
\exmn{269.1}%
\begingl
	\glpreamble	Hīn ÿîkt wuʟ̣îtsî′s. //
	\glpreamble	Héen ÿíkt wudlitsís. //
	\gla	{} Héen \rlap{ÿíkt} @ {} {}
		\rlap{wudlitsís.} @ {} @ {} @ {} @ {} @ {} //
	\glb	{} héen ÿíᵏ -t {}
		wu- d- l- i- \rt[¹]{tsis} -μH //
	\glc	{}[\pr{PP} water within -\xx{pnct} {}]
		\xx{pfv}- \xx{pasv}- \xx{csv}- \rt[¹]{float} -\xx{var} //
	\gld	{} water within -at {}
		\rlap{\xx{zcnj}.\xx{pfv}.float} {} {} {} {} {} //
	\glft	‘It floated within the water.’
		//
\endgl
\xe

\FIXME{Why \fm{-μH} therefore \fm{∅}-conj.\ and not \fm{-μμL} therefore \fm{n}-conj.?}

\FIXME{Is \fm{d-} passive or middle in this verb?
Does it float itself or is it floated by something?}

\ex\label{ex:91-43-unfeeling-face-of-cliff}%
\exmn{269.1}%
\begingl
	\glpreamble	Tc!uʟe′ łstā′x awudᴀnū′kᵘ awe′ yug̣ā′ʟ! ỵêt wudzigî′t. //
	\glpreamble	Chʼu tle l sh daax̱ awu̬danoogú̥ áwé yú g̱ílʼ ÿát wudzigít. //
	\gla	{} Chʼu tle l {} sh \rlap{daax̱} @ {} {}
			\rlap{awu̬danoogú̥} @ {} @ {} @ {} @ {} @ {} {}
		\rlap{áwé} @ {} +
		{} yú g̱ílʼ \rlap{ÿát} @ {} {}
		\rlap{wudzigít.} @ {} @ {} @ {} @ {} @ {} //
	\glb	{} chʼu tle l {} sh daa -x̱ {}
			a- wu- d- \rt[²]{nuͥk} -μμL -í {}
		á -wé
		{} yú g̱ílʼ ÿá -t {}
		wu- d- s- i- \rt[¹]{git} -μH //
	\glc	{}[\pr{CP} just then \xx{neg} {}[\pr{PP} \xx{rflx·pss} around -\xx{pert} {}]
			\xx{arg}- \xx{pfv}- \xx{mid}- \rt[²]{feel} -\xx{var} -\xx{sub} {}]
		\xx{foc} -\xx{mdst}
		{}[\pr{PP} \xx{dist} cliff face -\xx{pnct} {}]
		\xx{pfv}- \xx{mid}- \xx{xtn}- \xx{stv}- \rt[¹]{fall·anim} -\xx{var} //
	\gld	{} just then not {} self’s around -of {}
			\rlap{3>3.\xx{zcnj}.\xx{pfv}.feel} {} {} {} {} {} {}
		\rlap{it.is} {}
		{} that cliff face -to {}
		\rlap{\xx{pfv}.\xx{zcnj}.fall·anim} {} {} {} {} {} //
	\glft	‘Just then it was without feeling himself that he fell to the face of the cliff.’
		//
\endgl
\xe

\citeauthor{swanton:1909}’s transcription \orth{g̣ā′ʟ!} in (\lastx) and (\nextx) etc.\ is surprising.
He glosses and translates it as “cliff”, but this is only ever \fm{g̱ílʼ} [\ipa{qíɬʼ}] ‘cliff’ in modern Tlingit (Tongass \fm{g̱ilʼ} [\ipa{qiɬʼ}]) whereas his transcription implies \fm[*]{g̱aalʼ} [\ipa{qàːɬʼ}] or \fm[*]{g̱áalʼ} [\ipa{qáːɬʼ}].
There is no attested root \fm[*]{\rt{g̱alʼ}} nor \fm[*]{\rt{g̱atlʼ}} so his transcription cannot be easily interpreted as referring to some other noun, and certainly his gloss and transcription point only to \fm{g̱ílʼ} ‘cliff’.
A plausible explanation here is a misreading of a loop-shaped handwritten \fm{i} or \fm{e} for handwritten \fm{a}.
This hypothesis is supported by \citeauthor{swanton:1909}’s later transcription of the same word as \orth{g̣eʟ!} in (\ref{ex:91-55-heard-it-partway-up-cliff}).

\ex\label{ex:91-44-suspended-on-cliff-face}%
\exmn{269.2}%
\begingl
	\glpreamble	ᴀx̣ wułix̣ā′t! yū g̣āʟ! ya. //
	\glpreamble	Áx̱ wulixáatʼ yú g̱ílʼ yá. //
	\gla	{} \rlap{Áx̱} @ {} {} \rlap{wulixáatʼ} @ {} @ {} @ {} @ {}
		{} yú g̱ílʼ yá. {} //
	\glb	{} á -x̱ {} wu- l- i- \rt[¹]{xatʼ} -μμH
		{} yú g̱ílʼ yá {} //
	\glc	{}[\pr{PP} \xx{3n} -\xx{pert} {}] \xx{pfv}- \xx{xtn}- \xx{stv}- \rt[¹]{hang} -\xx{var}
		{}[\pr{DP} \xx{dist} cliff face {}] //
	\gld	{} it -on {} \rlap{\xx{g̱cnj}.\xx{pfv}.suspend} {} {} {} {}
		{} that cliff face {} //
	\glft	‘He was suspended on it, that cliff face.’
		//
\endgl
\xe

\ex\label{ex:91-45-learned-from-him-seagull-eagle}%
\exmn{269.2}%
\begingl
	\glpreamble	ᴀ′qawe doq!ᴀ′kᴀt wuskē′ntc łdakᴀ′t-ᴀt, kē′ʟ̣adî, tcāk!. //
	\glpreamble	Áxʼ áwé du x̱ʼétx̱ at wuskwéiÿ̃ch ldakát át, kéidladi, chʼáakʼ. //
	\gla	{} \rlap{Áxʼ} @ {} {} \rlap{áwé} @ {}
		{} du \rlap{x̱ʼétx̱} @ {} {} 
		at @ \rlap{wuskwéiÿ̃ch} @ {} @ {} @ {} @ {} @ {}
		{} ldakát át, {}
		{} kéidladi, {} 
		{} chʼáakʼ. {} //
	\glb	{} á -xʼ {} á -wé
		{} du x̱ʼá k {}
		at= wu- s- \rt[²]{kuʰ} -eH -ÿ -ch
		{} ldakát át {}
		{} kéidladi {}
		{} chʼáakʼ {} //
	\glc	{}[\pr{PP} \xx{3n} -\xx{loc} {}] \xx{foc} -\xx{mdst}
		{}[\pr{PP} \xx{3h·pss} mouth -\xx{abl} {}]
		\xx{4n·o}= \xx{pfv}- \xx{xtn}- \rt[²]{know} -\xx{var} -\xx{ÿsfx} -\xx{rep}
		{}[\pr{DP} \xx{univ} thing {}]
		{}[\pr{DP} seagull {}]
		{}[\pr{DP} eagle {}] //
	\gld	{} there -at {} \rlap{it.is} {}
		{} his mouth -from {}
		sth\• \rlap{\xx{zcnj}.\xx{hab}.know} {} {} {} {} {}
		{} every thing {}
		{} seagull {}
		{} eagle {} //
	\glft	‘It was there that everything would learn things from his mouth, seagulls, eagles.’
		//
\endgl
\xe

\FIXME{Discuss \fm{at wuskwéiÿ̃ch}.
Habitual or repetitive perfective?
If hab, why not \fm{at u…}?
The \orth{n} is probably nasalized \fm{ÿ} < \fm[*]{ŋ}.}

\ex\label{ex:91-46-spirits-die-off}%
\exmn{269.3}%
\begingl
	\glpreamble	Qotx cū′nax̣îx̣tc duī′tx qeyē′k g̣aᴀ′tîn, yutcā′k!, kē′ʟ̣adî, łdakᴀ′t a. //
	\glpreamble	Ḵutx̱ shunaxíxch du eetx̱ ḵuyéik g̱a.ádín, yú chʼáakʼ, kéidladi, ldakát á. //
	\gla	{} \rlap{Ḵutx̱} @ {} {} \rlap{shunaxíxch} @ {} @ {} @ {} @ {}
		{} {} du \rlap{eetx̱} @ {} {}
			{} \rlap{ḵuyéik} @ {} {}
			\rlap{g̱a.ádín} @ {} @ {} @ {} @ {} {}
		{} yú chʼáakʼ, {}
		{} kéidladi, {}
		{} ldakát á. {} //
	\glb	{} ḵú -dáx̱ {} shu- n- \rt[¹]{xix} -μH -ch
		{} {} du ee -dáx̱ {}
			{} ḵu- yéik {}
			{} g̱- \rt[¹]{.at} -μH -ín {}
		{} yú chʼáakʼ {}
		{} kéidladi {}
		{} ldakát á {} //
	\glc	{}[\pr{PP} \xx{areal} -\xx{abl} {}] end- \xx{ncnj}- \rt[¹]{fall} -\xx{var} -\xx{rep}
		{}[\pr{CP} {}[\pr{PP} \xx{3h·pss} \xx{base} -\xx{abl} {}]
			{}[\pr{DP} \xx{areal}- spirit {}]
			\xx{zcnj}\· \xx{mod}- \rt[¹]{go·\xx{pl}} -\xx{var} -\xx{ctng} {}]
		{}[\pr{DP} \xx{dist} eagle {}]
		{}[\pr{DP} seagull {}]
		{}[\pr{DP} \xx{univ} \xx{3n} {}] //
	\gld	{} \rlap{lost} {} {} \rlap{end.\xx{hab}.use·up} {} {} {} {}
		{} {} him {} -from {}
			{} \rlap{animal·spirit} {} {}
			\rlap{\xx{ctng}.go·\xx{pl}} {} {} {} {} {}
		{} that eagle {}
		{} seagull {}
		{} all it {} //
	\glft	‘They died off whenever those spirits went away from him, those eagles, seagulls, all of them.’
		//
\endgl
\xe

\section{Paragraph 6}\label{sec:91-para-6}

\ex\label{ex:91-47-searched-for-him-uncle}%
\exmn{269.5}%
\begingl
	\glpreamble	Duīg̣ā′ quwacî′ dukā′k. //
	\glpreamble	Du eeg̱áa ḵoowashee du káak. //
	\gla	{} Du \rlap{eeg̱áa} @ {} {}
		\rlap{ḵoowashee} @ {} @ {} @ {} @ {}
		{} du káak. {} //
	\glb	{} du ee -g̱áa {}
		ḵu- wu- i- \rt[²]{shiʰ} -μμL
		{} du káak {} //
	\glc	{}[\pr{PP} \xx{3h} \xx{base} -\xx{ades} {}]
		\xx{areal}- \xx{pfv}- \xx{stv}- \rt[²]{reach·for} -\xx{var}
		{}[\pr{DP} \xx{3h·pss} mat·uncle {}] //
	\gld	{} him {} for {}
		\rlap{\xx{ncnj}.\xx{pfv}.search} {} {} {} {}
		{} his mat·uncle {} //
	\glft	‘He searched for him, his uncle.’
		//
\endgl
\xe

\ex\label{ex:91-48-discovered-nest-handiwork-riverside}%
\exmn{269.5}%
\begingl
	\glpreamble	Nᴀs!gaducu′ uxe′ aqᴀ′x quuwacī′ yuku′t duqē′łk! a′djî ite′ yuhī′n yāxq!. //
	\glpreamble	Nasʼgadooshú ux̱éi, a káx̱ ḵoowashee yú kút du kéilkʼ a ji.eetí yú héen yaax̱xʼ. //
	\gla	{} Nasʼgadooshú \rlap{ux̱éi} @ {} @ {} @ {} @ {} {}
		{} a \rlap{káx̱} @ {} {}
		\rlap{ḵoowashee} @ {} @ {} @ {} @ {}
		{} yú kút, {}
		{} du kéilkʼ {}
		{} a \rlap{ji.eetí,} @ {} {}
		{} yú héen \rlap{yaax̱xʼ.} @ {} {} //
	\glb	{} nasʼgadooshú {} u- \rt[¹]{x̱e} -μμH {} {}
		{} a ká -x̱ {}
		ḵu- wu- i- \rt[²]{shiʰ} -μμL
		{} yú kút {}
		{} du kéilkʼ {}
		{} a jín- eetí {}
		{} yú héen yaax̱ -xʼ {} //
	\glc	{}[\pr{CP} eight \xx{zcnj}\· \xx{irr}- \rt[¹]{overnight} -\xx{var} \·\xx{sub} {}]
		{}[\pr{PP} \xx{3n·pss} \xx{hsfc} -\xx{pert} {}]
		\xx{areal}- \xx{pfv}- \xx{stv}- \rt[²]{reach·for} -\xx{var}
		{}[\pr{DP} \xx{dist} nest {}]
		{}[\pr{DP} \xx{3h·pss} sor·nephew {}]
		{}[\pr{DP} \xx{3n·pss} hand- remains {}]
		{}[\pr{PP} \xx{dist} water side -\xx{loc} {}] //
	\gld	{} eight \rlap{\xx{csec}.overnight} {} {} {} {} {}
		{} its\ix{i} atop -on {}
		\rlap{\xx{ncnj}.\xx{pfv}.search} {} {} {} {}
		{} that nest\ix{i} {}
		{} his sor·nephew\ix{j} {}
		{} his\ix{j} \rlap{handiwork\ix{i}} {} {}
		{} that river side -at {} //
	\glft	‘Having overnighted eight times, he discovered it, that nest, his nephew’s handiwork, by that riverside.’
		//
\endgl
\xe

\ex\label{ex:91-49-saw-all-those-nests}%
\exmn{269.6}%
\begingl
	\glpreamble	Tc!uʟ′e łdakᴀ′t ā′wusitīn yuku′tq! duqē′łk! ᴀx kēnaxē′nîỵa. //
	\glpreamble	Chʼu tle ldakát aa wusiteen yú kútxʼ, du kéilkʼ áx̱ kei unax̱éini ÿé. //
	\gla	Chʼu tle {} ldakát aa {}
		\rlap{wusiteen} @ {} @ {} @ {} @ {}
		{} yú \rlap{kútxʼ,} @ {} {} +
		{} {} {} du kéilkʼ {}
			{} \rlap{áx̱} @ {} {}
			kei @ \rlap{unax̱éini} @ {} @ {} @ {} @ {} @ {} {} ÿé. //
	\glb	chʼu tle {} ldakát aa {}
		wu- s- i- \rt[²]{tin} -μμL
		{} yú kút -xʼ {} 
		{} {} {} du kéilkʼ {}
			{} á -x̱ {}
			kei= u- n- \rt[¹]{x̱eͥ} -μμH -n -i {} ÿé {} //
	\glc	just then {}[\pr{DP} \xx{univ} \xx{part} {}]
		\xx{pfv}- \xx{xtn}- \xx{stv}- \rt[²]{see} -\xx{var}
		{}[\pr{DP} \xx{dist} nest -\xx{pl} {}]
		{}[\pr{DP} {}[\pr{CP} {}[\pr{DP} \xx{3h·pss} sor·nephew {}]
			{}[\pr{PP} \xx{3n} -\xx{pert} {}]
			up= \xx{irr}- \xx{ncnj}- \rt[¹]{overnight} -\xx{var} -\xx{nsfx} -\xx{rel} {}] place {}] //
	\gld	just then {} all some {} 
		\rlap{\xx{pfv}.see} {} {} {} {}
		{} those nest -s {}
		{} {} {} his sor·nephew {}
			{} there -at {}
			up \rlap{\xx{prog}.overnight} {} {} {} {} -where {} place {} //
	\glft	‘Then he saw all of them, those nests, the places where his nephew was spending the night.’
		//
\endgl
\xe

\FIXME{Unsure about \fm{ldakát aa}.
\citeauthor{leer:1977} gives \fm{ldakát awsiteen} instead, but this misses \orth{ā′} and seems weird without a complement of \fm{ldakát}.
The \fm{aa} could be the partitive object pronominal clitic \fm{aa=} or the independent partitive \fm{aa} pronoun; the latter is suggested by \orth{wusitīn} \fm{wusiteen} rather than *\orth{wsitīn} \fm{wsiteen}.
More remotely, could be \fm{át}, or even \fm{at=}.
The former would presumably be coreferential with \fm{yú kútxʼ} but the latter would be strange since \fm{at=} does not normally occur with a definite interpretation.}

\FIXME{Why is \fm{kei=} introduced for ‘overnight’?
The verb is normally \fm{∅}-conjugation, but \fm{kei=} with the progressive implies a switch to \fm{g}-conjugation.
Alternatively it’s the \fm{∅}-conjugation derivation with \fm{kei=} and here we can’t tell the difference because the progressive would look the same (unless \fm{ÿaa=kei=} is possible?).
Either way, the semantics is presumably ‘movement up the river’.}

\ex\label{ex:91-50-look-within-river}%
\exmn{269.7}%
\begingl
	\glpreamble	Yuhī′n ỵîkt aoʟ̣îg̣ê′n dukā′k. //
	\glpreamble	Yú héen ÿíkt awdlig̱én du káak. //
	\gla	{} Yú héen \rlap{ÿíkt} @ {} {}
		\rlap{awdlig̱én} @ {} @ {} @ {} @ {} @ {} @ {}
		{} du káak. {} //
	\glb	{} yú héen ÿíᵏ -t {}
		a- wu- d- l- i- \rt[²]{g̱eͥn} -μH
		{} du káak {} //
	\glc	{}[\pr{PP} \xx{dist} water within -\xx{pnct} {}]
		\xx{xpl}- \xx{pfv}- \xx{mid}- \xx{xtn}- \xx{stv}- \rt[²]{look} -\xx{var}
		{}[\pr{DP} \xx{3h·pss} mat·uncle {}] //
	\gld	{} that river within -to {}
		\rlap{\xx{zcnj}.\xx{pfv}.look} {} {} {} {} {} {}
		{} his mat·uncle {} //
	\glft	‘He looked within that river, his uncle.’
		//
\endgl
\xe

\ex\label{ex:91-51-it-swam-there-salmon}%
\exmn{269.7}%
\begingl
	\glpreamble	Aỵî′x uwaq!ᴀ′q yuxā′t. //
	\glpreamble	A ÿíx̱ uwaxʼák yú x̱áat. //
	\gla	{} A \rlap{ÿíx̱} @ {} {}
		\rlap{uwaxʼák} @ {} @ {} @ {}
		{} yú x̱áat. {} //
	\glb	{} a ÿíᵏ -x̱ {}
		u- i- \rt[¹]{xʼak} -μH
		{} yú x̱áat {} //
	\glc	{}[\pr{PP} \xx{3n·pss} within -\xx{pert} {}]
		\xx{zcnj}- \xx{stv}- \rt[¹]{fish·swim} -\xx{var}
		{}[\pr{DP} \xx{dist} salmon {}] //
	\gld	{} its within -at {}
		\rlap{\xx{pfv}.fish·swim} {} {} {}
		{} that salmon {} //
	\glft	‘It swam there, that salmon.’
		//
\endgl
\xe

The verb form in (\lastx) is singular although this is not explicitly indicated in its morphology.
There are small handful of verbs which have a paradigmatic difference between singular and plural where the plural occurs with an additional \fm{k-} and \fm{du-}.
One example is the root \fm{\rt[¹]{kʼeʼn}} ‘jump’ which has a singular form like \fm{kei x̱wajikʼén} ‘I jumped up’ with a plural counterpart \fm{kei haa kawduwakʼén} ‘we jumped up’.
The  root \fm{\rt[¹]{xʼak}} ‘fish swim’ is similar, with singular forms like \fm{kei uwaxʼák} ‘it (fish) swam up’ and plurals like \fm{kei kawduwaxʼák} ‘they (fish) swam up’.

\FIXME{Translating \fm{x̱áat} as ‘salmon’ here rather than ‘fish’.
Partly because that’s what \citeauthor{swanton:1909} does, but also because other fish are almost certainly not intended.}

\ex\label{ex:91-52-spend-night-underneath}%
\exmn{269.8}%
\begingl
	\glpreamble	Ā′uwaxe yukut taỵe′. //
	\glpreamble	Áa uwax̱éi yú kút taÿee. //
	\gla	{} \rlap{Áa} {} {}
		\rlap{uwax̱éi} @ {} @ {} @ {}
		{} yú kút taÿee. {} //
	\glb	{} á -μμL {}
		u- i- \rt[¹]{x̱e} -μμH
		{} yú kút taÿee {} //
	\glc	{}[\pr{PP} \xx{3n} -\xx{loc} {}]
		\xx{zpfv}- \xx{stv}- \rt[¹]{overnight} -\xx{var}
		{}[\pr{DP} \xx{dist} nest underneath {}] //
	\gld	{} there -at {}
		\rlap{\xx{pfv}.overnight} {} {} {}
		{} that nest underneath {} //
	\glft	‘He spent the night there, underneath that nest.’
		//
\endgl
\xe

\ex\label{ex:91-53-after-listens}%
\exmn{269.8}%
\begingl
	\glpreamble	ᴀtxā′we qołᴀ′xs!. //
	\glpreamble	Atx̱ áwé ḵul.áx̱sʼ. //
	\gla	{} \rlap{Átx̱} @ {} {} \rlap{áwé} @ {}
		\rlap{ḵul.áx̱sʼ.} @ {} @ {} @ {} @ {} @ {} //
	\glb	{} á -dáx̱ {} á -wé
		ḵu- d- s- \rt[²]{.ax̱} -μH -sʼ //
	\glc	{}[\pr{PP} \xx{3n} -\xx{abl} {}] \xx{foc} -\xx{mdst}
		\xx{areal}- \xx{mid}- \xx{xtn}- \rt[²]{hear} -\xx{var} -\xx{rep} //
	\gld	{} then -from {} \rlap{it.is} {}
		\rlap{\xx{ncnj}.\xx{impfv}.listen.\xx{rep}} {} {} {} {} {} //
	\glft	‘After that he listens.’
		//
\endgl
\xe

The verb form \fm{ḵul.áx̱sʼ} in (\lastx) is a repetitive imperfective with the specialized repetitive suffix \fm{-sʼ}.
The exact meaning of this suffix is still unclear, but it is often associated with serial events along a path such as multiple stitches, a marine animal surfacing multiple times, or baiting a line of hooks.
The English translation here is a simple imperfective activity with present tense, but the Tlingit form is actually a repetitive imperfective with unspecified tense.
Other possible translations of the form in (\lastx) include ‘he repeatedly listens’, ‘he was repeatedly listening’, and ‘he kept listening’, depending on how the aspect and tense properties might be rendered in English.

\ex\label{ex:91-54-morning-heard-sound-of-beaters}%
\exmn{269.9}%
\begingl
	\glpreamble	Ts!utā′tawe ā′waᴀx xē′tca kayē′k. //
	\glpreamble	Tsʼootaat áwé aawa.áx̱ x̱éjaa kayéik. //
	\gla	{} Tsʼootaat {} \rlap{áwé} @ {}
		\rlap{aawa.áx̱} @ {} @ {} @ {} @ {}
		{} \rlap{x̱éjaa} @ {} @ {} \rlap{kayéik.} @ {} {} //
	\glb	{} tsʼootaat {} á -wé
		a- wu- i- \rt[²]{.ax̱} -μH
		{} \rt[²]{x̱eͥch} -μH -aa ká- yéik {} //
	\glc	{}[\pr{NP} morning {}] \xx{foc} -\xx{mdst}
		\xx{arg}- \xx{pfv}- \xx{stv}- \rt[²]{hear} -\xx{var}
		{}[\pr{DP} \rt[²]{beat} -\xx{var} -\xx{inst} \xx{hsfc}- spirit {}] //
	\gld	{} morning {} \rlap{it.is} {}
		\rlap{3>3.\xx{zcnj}.\xx{pfv}.hear} {} {} {} {}
		{} \rlap{beat} {} -er \rlap{sound:\xx{inal}} {} {} //
	\glft	‘It was in the morning that he heard it, the sound of beaters.’
		//
\endgl
\xe

\citeauthor{swanton:1909}’s gloss in (\lastx) of \orth{xē′tca} is “beating of shaman’s sticks” and of \orth{kayē′k} is “for spirits”.
In fact the noun \fm{kayéik} simply means ‘sound, noise of’ as in \fm{washéen kayéik} ‘sound of a machine or motor’ or in the sentence \fm{teet kayéik ḵaa daa yaa ḵusagátch} ‘the noise of the waves confuses people’ \parencite[54.602]{story-naish:1973}.
This noun is based on the root \fm{\rt[¹]{yek}} ‘animated, alert; rapid, (too) fast; spirit’ just as is \fm{yéik} ‘spirit’ and verbs like \fm{yaa ayanashyék} ‘s/he is getting too fast (in drumming)’ \parencite[27.160]{story-naish:1973}.
Like most of the verbs based on \fm{\rt[¹]{yek}} but unlike the noun \fm{yéik}, \fm{kayéik} does not necessarily refer to spiritual phenomena.

The noun \fm{x̱íjaa} \~\ \fm{x̱éjaa} ‘beater’ that \citeauthor{swanton:1909} transcribes as \orth{xē′tca} in (\lastx) is a relatively unknown term for sticks that are used to beat out a rapid rhythm against each other or against a wooden plank, floor, or other hard surface.
This noun is derived transparently from the root \fm{\rt[²]{x̱ich}} \~\ \fm{\rt[²]{x̱ech}} ‘beat, club; throw anim.\ or filled container’ (cf.\ \fm{\rt[²]{x̱ish}-t} ‘beat, club’), with the instrument noun suffix \fm{-aa} ‘thing used for’.
\citeauthor{emmons:1991} mentions the use of “beating sticks” in a few places such as in the context of music \parencite[292]{emmons:1991} and gambling \parencite[419]{emmons:1991} as well as in shamanic and witchcraft practices \parencite[376, 383ff, 407ff]{emmons:1991}.
He describes them as “about fifteen inches long, sometimes with otter heads carved on the ends” \parencite[381]{emmons:1991}.
\citeauthor{de-laguna:1972} mentions them as “tapping sticks” only in the context of shamanic practices \parencite[697]{de-laguna:1972} which might imply that by the 20th century they had fallen out of use in other situations, or it could be that their wider use described earlier by \citeauthor{emmons:1991} was an occasional extension of their shamanic use.
\citeauthor{de-laguna:1972} provides a detailed description of \fm{x̱íjaa} and their use:

\begin{quote}\small
The shaman’s assistants, or rather all the men of his sib who were present, beat time with tapping sticks (x̣ítc̓ᴀ̀, x̣ítcᴀ̀) These were described as plain undecorated wooden rods, about 12 inches long.
A man held one in each hand and struck them together crosswise.
This was apparently the way they were used at Yakutat.
The shaman kept the sticks until they were needed, then handed them out to his assistants (CW).

Professor Libbey obtained a set of 24 plain wooden tapping sticks from the shaman’s grave near Yakutat which correspond to the description of my informant (pl.\ 171).
The x̣̓at̓kᴀ′ayi shaman’s outfit obtained by Emmons contained one of bone, carved to represent a land otter’s head (pl.\ 205).
This suggests that some of the longer bone and ivory charms in the Yakutat grave may have been used as tapping sticks.

Moreover, a Dry Bay informant said that the tapping sticks would be carved to represent the shaman’s spirits. “When they got that kucda yek [land otter spirits], they’re going to use kucda faces.
Sometimes they use it for something like this – to make a noise [demonstrating by tapping with a pencil]… All those people sitting around, that Indian doctor go around over here.
All those people hitting on the floor [with sticks].”

“And sometimes they got something like this:” She described a low flat rectangular sounding board, about 24 inches long, 6 inches wide, and “just flat,” or about 3 inches high.
It was called ‘mouth tapping-stick place’ (x̣̓a xítcᴀ̀ yet).
It was hollowed out below, “just like a box, you know… just like a dish, so it can loud,” and it was decorated with yek designs. “They got kucda under there.
And all that things, they’re going to use kucda or bear designs in it.” This would imply that the designs were on the hollow under surface, although my impression at the time was that they were carved on the top of the sounding box. “That’s the one they using it on the floor.
They hitting it.” Each man used only one stick, not two.

It is interesting that when Frank Italio sang some of G̣utcda’s spirit songs for the tape recorder, he pushed aside the tambourine drum provided him, and instead beat time with a pencil on a cigar box.
\sourceatright{\parencite[697–698]{de-laguna:1972}}
\end{quote}

\citeauthor{de-laguna:1972}’s form \orth{x̣ítc̓ᴀ̀} would be \fm{x̱íchʼaa} [\ipa{χí.tʃʼàː}], her form \orth{x̣ítcᴀ̀} would be \fm{x̱íchaa} [\ipa{χítʃʰàː}], and her form \orth{x̣̓a xítcᴀ̀ yet} would be \fm{x̱ʼaxíchaa yeit} [\ipa{χʼà.xí.tʃʰàː jèːt}].
These are all clearly attempts to transcribe the same noun which is \fm{x̱íjaa} based on \fm{\rt[²]{x̱ich}} with the typical lack of lexicalized uvular lowering in Yakutat.
\citeauthor{de-laguna:1972}’s longer phrase is \fm{x̱ʼax̱íjaa yeit} ‘mouth-beater below-thing’ (with \fm{x̱ʼa-} ‘mouth’ and \fm{yeit} a contraction of \fm{ÿee át} ‘below thing’) which is fairly close to what she gives as a translation.

The English translation of \fm{x̱éjaa} in (\lastx) is plural: ‘the sound of beaters’.
This is because plurals are regularly used for generic reference in this kind of context in English.
The Tlingit form is, like the vast majority of nouns, not specified for plurality; Tlingit nouns are number neutral rather than singular by default.
In this sentence it is impossible to say for sure whether the noun should be interpreted as singular or plural since there are no other cues – no singular or plural verb roots, no singular or plural pronouns, no plural markers – that distinguish singular and plural number in this sentence.
Thus the Tlingit utterance is interpretable as either the sound of one beater or the sound of many beaters, with neither interpretation being preferred.

\ex\label{ex:91-55-heard-it-partway-up-cliff}%
\exmn{269.9}%
\begingl
	\glpreamble	Tc!a yū′g̣eʟ! yakᴀtū′de awe′ ā′waᴀx. //
	\glpreamble	Chʼa yú g̱ílʼ yakatʼóode áwé aawa.áx̱. //
	\gla	Chʼa {} yú g̱ílʼ \rlap{yakatʼóode} @ {} @ {} {} \rlap{áwé} @ {}
		\rlap{aawa.áx̱.} @ {} @ {} @ {} @ {} //
	\glb	chʼa {} yú g̱ílʼ ÿá- katʼóoᵗ -dé {} á -wé
		a- wu- i- \rt[²]{.ax̱} -μH //
	\glc	just {}[\pr{DP} \xx{dist} cliff face- partway -\xx{all} {}] \xx{foc} -\xx{mdst}
		\xx{arg}- \xx{pfv}- \xx{stv}- \rt[²]{hear} -\xx{var} //
	\gld	just {} that cliff face- partway -to {} \rlap{it.is} {}
		\rlap{3>3.\xx{zcnj}.\xx{pfv}.hear} {} {} {} {} //
	\glft	‘It was partway up the face of the cliff that he heard it.’
		//
\endgl
\xe

The noun \fm{katʼóot} ‘partway’ in (\lastx) looks like it should be morphologically decomposable into something like \fm{ká} + \fm{tʼóo} + \fm{-t} but synchronically this is not the case because there is no independently identifiable noun \fm{tʼóo} or \fm{tʼú}.
There are exactly two words attested with this \fm{tʼóo}: \fm{katʼóot} ‘partway; waist’ and \fm{ḵʼeishtʼóo} ‘pitch ball; worm mark in caribou hide’ \parencites[07/151–152]{leer:1973}.
These suggest that historically there may have been a root \fm[*]{\rt{tʼu}}, especially since \fm{ḵʼeishtʼóo} has the look of a verb like \fm[*]{ḵʼe-sh-\rt{tʼu}-μμH}, but in the modern language this hypothetical root has no definable meaning.
The analysis of \fm{katʼóot} gives the underlying form \fm{katʼóoᵗ} with a final superscript \fm{t} because \citeauthor{swanton:1909}’s transcription \orth{yakᴀtū′de} suggests that this \fm{t} is missing.
If this is correct it could indicate that this final \fm{t} is a fossilized \fm{-t} suffix similar to the obsolete locative postposition \fm{-k} which occurs at the end of relational nouns like \fm{ÿík} ‘within concavity’, \fm{táak} ‘bottom of concavity’, \fm{sháak} ‘head of water body’, \fm{x̱ʼáak} ‘interstice; space between’, etc.
Alternatively, it is possible that \citeauthor{swanton:1909}’s transcription is in error and that the speaker actually said \fm{katʼóotde}, but there is no way to confirm this.

\ex\label{ex:91-56-went-to-base}%
\exmn{269.10}%
\begingl
	\glpreamble	Tc!uʟe′ ᴀk!eỵī′t ū′wagut. //
	\glpreamble	Chʼu tle a kʼiÿeet uwagút. //
	\gla	Chʼu tle {} a \rlap{kʼiÿeet} @ {} @ {} {}
		\rlap{uwagút.} @ {} @ {} @ {} //
	\glb	chʼu tle {} a kʼí- ÿee -t {}
		u- i- \rt[¹]{gut} -μH //
	\glc	just then {}[\pr{PP} \xx{3n·pss} base- below -\xx{pnct} {}]
		\xx{zpfv}- \xx{stv}- \rt[¹]{go·\xx{sg}} -\xx{var} //
	\gld	just then {} its base- below -to {}
		\rlap{\xx{pfv}.go·\xx{sg}} {} {} {} //
	\glft	‘He went below the base of it.’
		//
\endgl
\xe

\ex\label{ex:91-57-thinking-not-seen-spoke}%
\exmn{269.10}%
\begingl
	\glpreamble	Tc!uł ᴀc utē′nx ᴀc wudjîỵī′ayu ᴀcī′t q!ē′watᴀn. //
	\glpreamble	Chʼu l ash utéenx̱ ash wujeeÿí áyú ash eet x̱ʼeiwatán. //
	\gla	{} {} {} Chʼu l ash @ \rlap{utéenx̱} @ {} @ {} @ {} {} {} {}
			ash \rlap{wujeeÿí} @ {} @ {} @ {} {} \rlap{áyú} @ {}
		{} ash \rlap{eet} @ {} {}
		\rlap{x̱ʼeiwatán.} @ {} @ {} @ {} @ {} //
	\glb	{} {} {} chʼu l ash= u- \rt[²]{tin} -μμH {} {} -x̱ {}
			ash= wu- \rt[²]{jiʰ} -μμL -í {} á -yú
		{} ash ee -t {}
		x̱ʼe- wu- i- \rt[²]{tan} -μH //
	\glc	{}[\pr{CP} {}[\pr{PP} {}[\pr{CP} just \xx{neg}
			\xx{3prx·o}= \xx{irr}- \rt[²]{see} -\xx{var} \·\xx{sub} {}] -\xx{pert} {}]
			\xx{3prx·o}= \xx{pfv}- \rt[²]{think} -\xx{var} -\xx{sub} {}] \xx{foc} -\xx{dist}
		{}[\pr{PP} \xx{3prx} \xx{base} -\xx{pnct} {}]
		mouth- \xx{pfv}- \xx{stv}- \rt[²]{hdl·w/e} -\xx{var} //
	\gld	{} {} {} just not him\ix{i} \rlap{\xx{gcnj}.\xx{impfv}.see} {} {} {} {} -as {}
		him\ix{j} \rlap{\xx{ncnj}.\xx{pfv}.think} {} {} {} {} \rlap{it.is} {}
		{} him\ix{i} {} -to {}
		\rlap{mouth.\xx{zcnj}.\xx{pfv}.handle} {} {} {} {} //
	\glft	‘It was having thought of him that he could not see him that he spoke to him.’
		//
\endgl
\xe

The sentence in (\lastx) is difficult to render precisely in English because of the complex subordinate clause that appears before the focus particle \fm{áyú} and because of the complex interaction between two third person human referents which is resolved in Tlingit via alternation of third person \fm{ash} pronouns.
The clause \fm{chʼu l ash\ix{i} utéenx̱ ash\ix{j} uwajée} means ‘he\ix{i} thought of him\ix{j} that he\ix{j} could not see him\ix{i}’, where the index \sv{i} refers to the maternal uncle and the index \sv{j} refers to the sororal nephew and protagonist.
This is then marked as a subordinate clause with \fm{-í} and focused by the following \fm{áyú}.
The matrix clause \fm{ash\ix{i} eet x̱ʼeiwatán} ‘he\ix{j} spoke to him\ix{i}’ then 
has the sororal nephew (\sv{i}) speak to the maternal uncle (\!\!\sv{j}). 

\ex\label{ex:91-58-wrong-way-below-me-uncle}%
\exmn{269.11}%
\begingl
	\glpreamble	“Qâq ỵē′nᴀx ᴀxtaỵī′t ī′ỵagut kāk.” //
	\glpreamble	«\!Ḵwáaḵ ÿéináx̱ ax̱ taÿeet eeÿagút, káak.\!» //
	\gla	{} \llap{«\!}Ḵwáaḵ \rlap{ÿéináx̱} @ {} {}
		{} ax̱ \rlap{taÿeet} @ {} {}
		\rlap{eeÿagút,} @ {} @ {} @ {} @ {} 
		{} káak.\!» {} //
	\glb	{} ḵwáaḵ ÿé -náx̱ {}
		{} ax̱ taÿee -t {}
		u- i- i- \rt[¹]{gut} -μH
		{} káak {} //
	\glc	{}[\pr{PP} wrong way -\xx{perl} {}]
		{}[\pr{PP} \xx{1sg·pss} below -\xx{pnct} {}]
		\xx{zpfv}- \xx{2sg·s}- \xx{stv}- \rt[¹]{go·\xx{pl}} -\xx{var}
		{}[\pr{DP} mat·uncle.\xx{voc} {}] //
	\gld	{} wrong way -thru {}
		{} my below -to {}
		\rlap{\xx{pfv}.you·\xx{sg}.go·\xx{sg}} {} {} {} {}
		{} mat·uncle {} //
	\glft	‘You came below me through the wrong way, uncle.’
		//
\endgl
\xe

\ex\label{ex:91-59-pitied-nephew}%
\exmn{269.11}%
\begingl
	\glpreamble	ʟᴀx wâ′sa awug̣ā′x duqē′łk! hūtc. //
	\glpreamble	Tlax̱ wáa sá aÿaawag̱aax̱ du kéilkʼ, hóoch. //
	\gla	Tlax̱ {} wáa sá {}
		\rlap{aÿaawag̱aax̱} @ {} @ {} @ {} @ {} @ {}
		{} du kéilkʼ, {}
		{} \rlap{hóoch.} @ {} {} //
	\glb	tlax̱ {} wáa sá {}
		a- ÿ- wu- i- \rt[²]{g̱ax̱} -μμL
		{} du kéilkʼ {}
		{} hú -ch {} //
	\glc	very {}[ how \xx{q} {}]
		\xx{arg}- \xx{qual}- \xx{pfv}- \xx{stv}- \rt[²]{pity} -\xx{var}
		{}[\pr{DP} \xx{3h·pss} sor·nephew {}]
		{}[\pr{DP} \xx{3h} -\xx{erg} {}] //
	\gld	very {} how {} {}
		\rlap{3>3.\xx{gcnj}.\xx{pfv}.pity} {} {} {} {} {}
		{} his sor·nephew {}
		{} he {} {} //
	\glft	‘How very much he pitied his nephew, he did.’
		//
\endgl
\xe

The verb in (\lastx) that \citeauthor{swanton:1909} transcribes as \orth{awug̣ā′x} must be \fm{aÿaawag̱aax̱} [\ipa{ʔà.ɰàː.wà.ˈqàːχ}] ‘s/he pitied him/her; consoled his/her grief’ with the sequence \fm{aÿaawa} [\ipa{ʔà.ɰàː.wà}] probably contracted to something like [\ipa{ʔɑ̀ɑ̀ːwə̀}] and so misheard by \citeauthor{swanton:1909} as [\ipa{ʔà.wù}].
Although perfective here, it has an imperfective activity form \fm{aÿag̱áax̱} \parencites[f02/130]{leer:1973}[831]{leer:1976}.
This verb is syntactically interesting because it appears to be based on the monovalent root \fm{\rt[¹]{g̱ax̱}} ‘cry’ as in the unergative intransitive \fm{woog̱aax̱} ‘s/he cried’, but \fm{aÿaawag̱aax̱} is clearly transitive without overt causativization (no \fm{s-} or \fm{l-}) which implies that the root is bivalent.
Pending a better understanding of the valency patterns here, this particular verb is analyzed with a homophonous root \fm{\rt[²]{g̱ax̱}} ‘pity, console’ which is separate from the monovalent \fm{\rt[¹]{g̱ax̱}} ‘cry’.

\ex\label{ex:91-60-go-up-thru-corner}%
\exmn{269.12}%
\begingl
	\glpreamble	“He q!êngu′kcî nᴀ′axo ke gu′.” //
	\glpreamble	«\!He xʼwán gukshínáx̱ kei gú.\!» //
	\gla	«\!Hé xʼwán {} \rlap{gukshínáx̱} @ {} {}
		kei @ \rlap{gú.} @ {} @ {} @ {} //
	\glb	\pqp{}hé xʼwán {} gukshí -náx̱ {}
		kei= {} {} \rt[¹]{gut} -⊗ //
	\glc	\pqp{}well \xx{imp} {}[\pr{PP} corner -\xx{perl} {}]
		up= \xx{zcnj}\· \xx{2sg·s}\· \rt[¹]{go·\xx{sg}} -\xx{var} //
	\gld	\pqp{}well \xx{imp} {} corner -thru {}
		up \rlap{\xx{imp}.you·\xx{sg}.go·\xx{sg}} {} {} {} //
	\glft	‘Well then come up through the corner.’
		//
\endgl
\xe

The noun \fm{gukshí} ‘corner’ in (\lastx) is lexically interesting.
It is documented with a surprising variety of forms: \fm{gúksh}, \fm{gukshú}, \fm{gukshtú}, \fm{gukshitú}, Tongass \fm{gukshtu} \parencite[f05/170]{leer:1973}, and \fm{gukshatú} \parencite[\textsc{t}·18]{leer:2001}.
\citeauthor{swanton:1909}’s transcription \orth{gu′kcî} suggests the form \fm{gukshí} which is not attested elsewhere, but which could be taken to imply that the attested \fm{gukshitú} is decomposable into \fm{gukshí-tú} with \fm{tú} ‘inside of hollow’.
The noun \fm{gukshí} is not further analyzable and its etymology is unclear, but it is probably related to \fm{gangook} (Tongass \fm{gangoòk} [\ipa{kan.ˈkʷuʰkʷ}]) ‘around the fire’ with \fm{gán} ‘fire’ (cf.\ \fm{\rt[¹]{gan}} ‘burn’) and possibly also to \fm{gúk} ‘ear’; the similar \fm{\rt[²]{guk}} ‘know how’, \fm{\rt[²]{guʼk}} ‘peck’, and the interjection \fm{góok!} ‘go!’ are unrelated.

\ex\label{ex:91-61-named-him}%
\exmn{269.12}%
\begingl
	\glpreamble	Gūxk!ᵒ sagu′tc yē uwasa′. //
	\glpreamble	Goox̱kʼ Ságuch yéi uwasáa. //
	\gla	{} \rlap{Goox̱kʼ} @ {} \rlap{Ságuch} @ {} {}
		yéi @ \rlap{uwasáa.} @ {} @ {} @ {} @ {} //
	\glb	{} goox̱ -kʼ sákw -ch {}
		yéi= ⱥ- u- i- \rt[²]{sa} -μμH //
	\glc	{}[\pr{DP} slave -\xx{dim} \xx{fut} -\xx{erg} {}]
		thus= \xx{arg}- \xx{zpfv}- \xx{stv}- \rt[²]{name} -\xx{var} //
	\gld	{} slave -little future {} {}
		thus \rlap{3>3.\xx{pfv}.name} {} {} {} {} //
	\glft	‘Future Little Slave named him thus.’
		//
\endgl
\xe

\citeauthor{swanton:1909}’s gloss and translation of (\lastx) are effectively backward.
The name \fm{Goox̱kʼ Sákw} here appears with the ergative suffix \fm{-ch}, indicating that this is the subject of the verb \fm{yéi aawasáa} ‘call, name’.
The \fm{-ch} cannot be an instrument applicative because the verb lacks applicative \fm{s-} or \fm{l-} and because only the ergative \fm{-ch} causes the disappearance of \fm{a-}: the verb form is \fm{yéi uwasáa} and not \fm{yéi aawasáa}.
Presumably the implicit object of the verb here is the maternal uncle.
A consequent puzzle is the name which has not been explicitly given in this narrative.
One possibility is that the speaker had in mind a relatively well known name, perhaps associated with the command in (\ref{ex:91-60-go-up-thru-corner}), but left this name unsaid.
Another possibility is that the name was given as an aside which \citeauthor{swanton:1909} did not record for some reason.
A third possibility is that the name is actually unknown: the nephew called out his uncle’s name, but the speaker did not know it or had forgotten it.

\ex\label{ex:91-62-uncle-asked-him}%
\exmn{269.13}%
\begingl
	\glpreamble	Dukā′ktc q!ē′wawus, //
	\glpreamble	Du káakch x̱ʼeiwawóosʼ //
	\gla	{} Du \rlap{káakch} @ {} {}
		\rlap{x̱ʼeiwawóosʼ} @ {} @ {} @ {} @ {} @ {} //
	\glb	{} du káak -ch {}
		ⱥ- x̱ʼe- wu- i- \rt[²]{wuͣsʼ} -μμH //
	\glc	{}[\pr{DP} \xx{3h·pss} mat·uncle -\xx{erg} {}]
		\xx{arg}- mouth- \xx{pfv}- \xx{stv}- \rt[²]{ask} -\xx{var} //
	\gld	{} his mat·uncle {} {}
		\rlap{3>3.\xx{ncnj}.\xx{pfv}.ask} {} {} {} {} {} //
	\glft	‘His uncle asked him’
		//
\endgl
\xe

\ex\label{ex:91-63-}%
\exmn{269.13}%
\begingl
	\glpreamble	“Dātkułā′nsaỵa yē qēỵanū′kᵘ.” //
	\glpreamble	«\!Daat \{kulaan\} sáÿá yéi ḵeeÿanook?\!» //
	\gla	{} \llap{«\!}Daat \{kulaan\} \rlap{sáÿá} {} {} @ {}
		yéi @ \rlap{ḵeeÿanook?\!»} @ {} @ {} @ {} @ {} @ {} //
	\glb	{} daat \{kulaan\} s= {} á -ÿá
		yéi= ḵu- wu- i- i- \rt[²]{nuͥk} -μμL //
	\glc	{}[\pr{QP} what {} \xx{q}= {}] \xx{foc} -\xx{prox}
		thus= \xx{areal}- \xx{pfv}- \xx{2sg·s}- \xx{stv}- \rt[²]{act} -\xx{var} //
	\gld	{} what ?? {} {} \rlap{it.is} {}
		thus \rlap{\xx{ncnj}.\xx{pfv}.you·\xx{sg}.act} //
	\glft	‘“What ?? so that you behaved thus?”’
		//
\endgl
\xe

\FIXME{What is \orth{kułā′n}?}

\ex\label{ex:91-64-not-tell-more-than-cheek-scratch}%
\exmn{270.1}%
\begingl
	\glpreamble	ʟēł ān akā′wunīk duwᴀ′ctu ka′oduʟ̣akᵘ yā′nᴀx. //
	\glpreamble	Tléil aan akawuneek du washtú kawdudlaakw yáanáx̱. //
	\gla	Tléil {} \rlap{aan} @ {} {}
		\rlap{akawuneek} @ {} @ {} @ {} @ {} @ {}
		{} {} {} du \rlap{washtú} @ {} {}
				\rlap{kawdudlaak} @ {} @ {} @ {} @ {} @ {} {}
			ÿáanáx̱. {} //
	\glb	tléil {} á -n {}
		a- k- u- wu- \rt[²]{nik} -μμL
		{} {} {} du wásh- tú {}
				k- wu- du- \rt[²]{dlak} -μμL {} {}
			ÿáanáx̱ {} //
	\glc	\xx{neg} {}[\pr{PP} \xx{3n} -\xx{instr} {}]
		\xx{arg}- \xx{qual}- \xx{irr}- \xx{pfv}- \rt[²]{tell} -\xx{var}
		{}[\pr{PP} {}[\pr{CP} {}[\pr{DP} \xx{3h·pss} cheek- inside {}]
				\xx{qual}- \xx{pfv}- \xx{4h·s}- \rt[²]{scratch} -\xx{var} \·\xx{nmz} {}]
			\xx{sup} {}] //
	\gld	not {} him -to {}
		\rlap{3>3.\xx{ncnj}.\xx{pfv}.tell} {} {} {} {} {}
		{} {} {} his cheek- inside {}
				\rlap{\xx{zcnj}.\xx{pfv}.one.scratch} {} {} {} {} -ing {}
			more·than {} //
	\glft	‘He did not tell him more than the scratching of the inside of his cheek.’
		//
\endgl
\xe

\FIXME{Discuss problem of translating \fm{du-} as ‘her’ versus implicit.}

\ex\label{ex:91-65-spirits-of-cave-there-order-me}%
\exmn{270.2}%
\begingl
	\glpreamble	“Ā′wu tatū′k yēkq!î′tc adê′ xᴀt kuna′,” //
	\glpreamble	«\!Áwu tatóok yéikxʼich aadé x̱at koonáa.\!» //
	\gla	{} {} {} \llap{«\!}\rlap{Áwu} @ {} {} tatóok {} \rlap{yéikxʼich} @ {} @ {} @ {} {}
		{} \rlap{aadé} @ {} {}
		x̱at @ \rlap{koonáa.\!»} @ {} @ {} @ {} //
	\glb	{} {} {} á -ú {} tatóok {} yéik -xʼ -í -ch {}
		{} á -dé {}
		x̱at= k- u- \rt[²]{na(ʼÿ)} -μμH //
	\glc	{}[\pr{DP} {}[\pr{NP} {}[\pr{CP} \xx{3n} -\xx{locp} {}] cave {}] spirit -\xx{pl} -\xx{pss} -\xx{erg} {}]
		{}[\pr{PP} \xx{3n} -\xx{all} {}]
		\xx{1sg·o}= \xx{qual}- \xx{irr}- \rt[²]{order} -\xx{var} //
	\gld	{} {} {} there -it.is {} cave {} spirit -s -of {} {}
		{} there -to {}
		me= \rlap{\xx{ncnj}.\xx{impfv}.order·go} {} {} {}  //
	\glft	‘The spirits of the cave there tell me to go there.’
		//
\endgl
\xe

The DP \fm{áwu tatóok yéikxʼich} in (\lastx) notably contains a locative predicate clause \fm{áwu} ‘it is there’ which is used like a relative clause as a modifier of the noun \fm{tatóok} ‘cave’.
This highlights the parallelism between the verbless predication of locative predicates and the verbal predication of other clauses like relative clauses.

\ex\label{ex:91-66-he-said-to-his-uncle}%
\exmn{270.2}%
\begingl
	\glpreamble	yū′aỵaosîqa du kā′k //
	\glpreamble	yóo aÿawsiḵaa du káak. //
	\gla	yóo @ \rlap{aÿawsiḵaa} @ {} @ {} @ {} @ {} @ {} @ {}
		{} du káak. {} //
	\glb	yóo= a- ÿ- wu- s- i- \rt[¹]{ḵa} -μμL
		{} du káak {} //
	\glc	\xx{quot}= \xx{arg}- \xx{qual}- \xx{pfv}- \xx{csv}- \xx{stv}- \rt[¹]{say} -\xx{var}
		{}[\pr{DP} \xx{3h·pss} mat·uncle {}] //
	\gld	so \rlap{3>3.\xx{ncnj}.\xx{pfv}.say·to} {} {} {} {} {} {}
		{} his mat·uncle {} //
	\glft	‘so he said to his maternal uncle.’
		//
\endgl
\xe

\ex\label{ex:91-67-bigger-than-a-house}%
\exmn{270.2}%
\begingl
	\glpreamble	aʟēˈn tatū′kâyu hît ỵā′nᴀx kug̣e′. //
	\glpreamble	Aatlein tatóok áyú hít ÿáanáx̱ koogéi. //
	\gla	{} \rlap{Aatlein} @ {} tatóok {} \rlap{áyú} @ {}
		{} hít ÿáanáx̱ {}
		\rlap{koogéi.} @ {} @ {} @ {} @ {} //
	\glb	{} aa= tlein tatóok {} á -yú
		{} hít ÿáanáx̱ {}
		k- u- i- \rt[¹]{ge} -μμH //
	\glc	{}[\pr{DP} \xx{part}= big cave {}] \xx{foc} -\xx{mdst}
		{}[\pr{PP} house \xx{sup} {}]
		\xx{cmpv}- \xx{irr}- \xx{stv}- \rt[¹]{big} -\xx{var} //
	\gld	{} \rlap{lots} {} cave {} \rlap{it.is} {}
		{} house more·than {}
		\rlap{\xx{gcnj}.\xx{cmpv}.\xx{stv}·\xx{impfv}.big} {} {} {} {} //
	\glft	‘It is a big cave that is bigger than a house.’
		//
\endgl
\xe

\section{Paragraph 7}\label{sec:91-para-7}

\ex\label{ex:91-68-spirits-went-with-uncle}%
\exmn{270.4}%
\begingl
	\glpreamble	Tc!uʟe′ kāyē′k wūā′t dukā′k tîn. //
	\glpreamble	Chʼu tle ḵuyéik woo.aat du káak tin. //
	\gla	Chʼu tle {} \rlap{ḵuyéik} @ {} {}
		\rlap{woo.aat} @ {} @ {} @ {}
		{} du káak tin. {} //
	\glb	chʼu tle {} ḵu- yéik {}
		wu- i- \rt[¹]{.at} -μμL
		{} du káak tin {} //
	\glc	just then {}[\pr{DP} \xx{areal}- spirit {}]
		\xx{pfv}- \xx{stv}- \rt[¹]{go·\xx{pl}} -\xx{var}
		{}[\pr{PP} \xx{3h·pss} mat·uncle \xx{instr} {}] //
	\gld	just then {} \rlap{animal·spirit} {} {}
		\rlap{\xx{ncnj}.\xx{pfv}.go·\xx{pl}} {} {} {}
		{} his mat·uncle with {} //
	\glft	‘So then animal spirits went with his uncle.’
		//
\endgl
\xe

\ex\label{ex:91-69-inside-uncle-beat-rhythm}%
\exmn{270.4}%
\begingl
	\glpreamble	A′ỵi nēł hᴀs āt ᴀc q!axē′tc dukā′k. //
	\glpreamble	A ÿee neil has .áat ash x̱ʼax̱éich du káak. //
	\gla	{} A ÿee @ {} {}
		{} {} neil @ {} {} has @ \rlap{.áat} @ {} @ {} @ {} {} +
		ash @ \rlap{x̱ʼax̱éich} @ {} @ {}
		{} du káak. {} //
	\glb	{} a ÿee {} {}
		{} {} neil -t {} has= {} \rt[¹]{.at} -μμH {} {}
		ash= x̱ʼe- \rt[¹]{x̱ech} -μμH
		{} du káak {} //
	\glc	{}[\pr{PP} \xx{3n·pss} below \·\xx{loc} {}]
		{}[\pr{CP} {}[\pr{PP} inside -\xx{pnct} {}] \xx{plh}= \xx{zcnj}\· \rt[¹]{go·\xx{pl}} -\xx{var} \·\xx{sub} {}]
		\xx{3prx·o}= mouth- \rt[²]{beat} -\xx{var}
		{}[\pr{DP} \xx{3h·pss} mat·uncle {}] //
	\gld	{} its inside -at {}
		{} {} inside -to {} they \rlap{\xx{csec}.go·\xx{pl}} {} {} {} {}
		him \rlap{mouth.\xx{ncnj}.\xx{impfv}.beat} {} {}
		{} his mat·uncle {} //
	\glft	‘Within it, them having gone inside, he is beating rhythm for him, his uncle.’
		//
\endgl
\xe

\ex\label{ex:91-70-instruct-uncle}%
\exmn{270.5}%
\begingl
	\glpreamble	Yên ᴀcukā′wadja dukā′k. //
	\glpreamble	Yan ashukaawajaa du káak //
	\gla	Yan @ \rlap{ashukaawajaa} @ {} @ {} @ {} @ {} @ {} @ {}
		{} du káak {} //
	\glb	ÿán= a- shu- k- wu- i- \rt[²]{jaʰ} -μμL
		{} du káak {} //
	\glc	\xx{term}= \xx{arg}- end- \xx{qual}- \xx{pfv}- \xx{stv}- \rt[²]{advise} -\xx{var}
		{}[\pr{DP} \xx{3h·pss} mat·uncle {}] //
	\gld	done \rlap{3>3.\xx{ncnj}.\xx{pfv}.instruct} {} {} {} {} {} {} //
	\glft	‘He instructed his uncle’
		//
\endgl
\xe

\ex\label{ex:91-71-when-run-in-fire-dont-let-burn}%
\exmn{270.5}%
\begingl
	\glpreamble	“Gᴀnᴀłtā′ xān gu g̣acīx̣ yuyē′k Nīx̣â′, łîł ʟax ye xᴀt kugᴀ′ndjîq. //
	\glpreamble	«\!Ganaltáa x̱áan gug̱ashéex yú yéik Neex̱á, líl tlax̱ yéi x̱at kugánjiḵ. //
	\gla	{} {} \llap{«\!}\rlap{Ganaltáa} @ {} @ {} {}
			{} \rlap{x̱áan} @ {} {}
			\rlap{gug̱ashéex} @ {} @ {} @ {} @ {} @ {} @ {} @ {} +
			{} yú yéik Neex̱á, {} {}
		líl tlax̱ yéi @ x̱at @ \rlap{kugánjiḵ.} @ {} @ {} @ {} @ {} @ {} //
	\glb	{} {} \rt[¹]{gan}- ltáaᵏ {} {}
			{} x̱á -n {}
			w- g- g̱- d- sh- \rt[¹]{xix} -μμH {}
			{} yú yéik Neex̱á, {} {}
		líl tlax̱ yéi= x̱at= k- u- \rt[¹]{gan} -μH -ch -ḵ //
	\glc	{}[\pr{CP} {}[\pr{PP} \rt[¹]{burn}- middle \·\xx{loc} {}]
			{}[\pr{PP} \xx{1sg·s} -\xx{instr} {}]
			\xx{irr}- \xx{gcnj}- \xx{mod}- \xx{mid}- \xx{pej}- \rt[¹]{fall} -\xx{var} \·\xx{sub}
			{}[\pr{DP} \xx{dist} spirit \xx{name} {}] {}]
		\xx{phib} very thus= \xx{1sg·o}= \xx{qual}- \xx{zpfv}- \rt[¹]{burn} -\xx{var} -\xx{rep} -\xx{phib} //
	\gld	{} {} fire- middle -in {}
			{} me -with {}
			\rlap{\xx{ncnj}.\xx{prsp}.run·\xx{sg}} {} {} {} {} {} {} {}
			{} that spirit \xx{name} {} {}
		don’t very thus me \rlap{\xx{hab}.burn} {} {} {} {} {} //
	\glft	‘“When it runs with me in the middle of the fire, that spirit Neex̱á, don’t let me be very burnt.’
		//
\endgl
\xe

The noun \fm{ganaltáa} \~\ \fm{ganaltáak} ‘middle of the fire’ in (\lastx) is one of a small number that include the otherwise unattested noun stem \fm{–ltáak} (Tongass \fm{–ltaak} [\ipa{ɬtaːk}]) which apparently means ‘(in the) middle of, centre of’.
Current documentation lists exactly two nouns with \fm{–ltáak}: \fm{ganaltáak} ‘middle of the fire’ with \fm{\rt[¹]{gan}} ‘burn’ \parencite[f05/17]{leer:1973} and \fm{waḵltáak} ‘pupil of eye’ with \fm{waaḵ} ‘eye’ \parencite[03/278–279]{leer:1973}.
The stem \fm{–ltáak} appears to be based on the relational noun \fm{táak} ‘bottom of concavity’ which includes the obsolete locative postposition \fm{-k}, and this is reinforced by the form in (\lastx) where the \fm{-k} is absent.
The function of the \fm{l-} in this form is unknown.

The name that \citeauthor{swanton:1909} transcribes as \orth{Nīx̣â′} in (\lastx) does not appear in any other sources.
A straightforward reading of this gives the modern \fm{Neex̱á} which is used here.
This appears to be based on the root \fm{\rt[¹]{nix̱}} \~\ \fm{\rt[¹]{nex̱}} ‘safe, rescued, healed, preserved’ as found in examples like \fm{yei nanéx̱} ‘she is recovering’ \parencite[107.1383]{story-naish:1973}, \fm{x̱at g̱asneix̱!}\ ‘save me!’\ \parencite[179.2472]{story-naish:1973}, \fm{héen káast káxʼ awlineix̱} ‘he saved water in a barrel’ \parencite[179.2474]{story-naish:1973}, \fm{g̱aneex̱} ‘saviour’ \parencite[04/189]{leer:1973}, \fm{woòshde x̱ʼakux̱daneex̱} ‘it (wound) will heal closed’  \parencite[04/189]{leer:1973}, and \fm{gút akawsineix̱ yéi jinéidáx̱} ‘he saved a dime from work’ \parencite[04/190]{leer:1973}.
The final \fm{-á} in \fm{Neex̱á} is still unidentified, but if it is a third person nonhuman pronoun ‘it’ then the whole name could possibly be a relative clause based on an activity imperfective and thus something like [\pr{DP} [\pr{CP} \fm{neex̱}] \fm{á}] ‘it which is (being) saved’.
If instead the final vowel is actually long instead of short then the resulting form \fm{Neex̱áa} could plausibly have the instrument nominalization suffix \fm{-áa} and the name might then mean something like ‘thing for rescuing, saving’.

\ex\label{ex:91-72-}%
\exmn{270.6}%
\begingl
	\glpreamble	Tcuyē′ xᴀt k!ugē′k!î q!wᴀn łīt! tū′dayu x̣ᴀt nag̣ē′ỵagîq!t.” //
	\glpreamble	Chʼu yéi x̱at kugéikʼi xʼwán léetʼ tóot áyú x̱at \{nag̱eiÿag̱éxʼt\}.\!» //
	\gla	 //
	\glb	 //
	\glc	 //
	\gld	 //
	\glft	‘’\newline
		“While I am getting small (imp.)\ basket into it is me throw.”\newline
		“While I am getting small throw me into a basket.”
		//
\endgl
\xe

\FIXME{Make sense of \fm{nag̱eiÿag̱éxʼt}.}

\FIXME{Discuss \orth{łīt!}.
This appears to be \fm{léetʼ}, but the gloss “basket” is problematic.
\citeauthor{leer:1973} has \fm{léetʼ} (Tongass \fm{leetʼ} [\ipa{ɬiːtʼ}]) “roots or vines used in basket decoration” and a compound \fm{shaa yaléetʼi} “rushes used for basket weaving” \parencite[08/50]{leer:1973}.
Doesn’t appear in \cite{leer:2001}.
Same definition as \citeauthor{leer:1973} in \cite[64]{naish-story:1976}.}

\ex\label{ex:91-73-so-he-did}%
\exmn{270.7}%
\begingl
	\glpreamble	Ayᴀ′xawe aosî′ne //
	\glpreamble	A yáx̱ áwé awsinei; //
	\gla	{} Á yáx̱ {} \rlap{áwé} @ {}
		\rlap{awsinei;} @ {} @ {} @ {} @ {} @ {}//
	\glb	{} á yáx̱ {} á -wé
		a- wu- s- i- \rt[¹]{neͥʰ} -μμL //
	\glc	{}[\pr{PP} \xx{3n} \xx{sim} {}]
		\xx{arg}- \xx{pfv}- \xx{csv}- \xx{stv}- \rt[¹]{occur} -\xx{var} //
	\gld	{} it like {}
		\rlap{3>3.\xx{ncnj}.\xx{pfv}.make.happen} {} {} {} {} {} //
	\glft	‘It was like that that he did so;’
		//
\endgl
\xe

\ex\label{ex:91-74-ran-into-fire-threw-into-basket}%
\exmn{270.7}%
\begingl
	\glpreamble	ᴀcī′n gᴀnᴀłtā′ dîcî′x̣ łīt! tū′dî ᴀc wug̣ē′q!. //
	\glpreamble	ash een ganaltáat wusheexí léetʼ tóode ash woog̱éixʼ. //
	\gla	{} {} ash \rlap{een} @ {} {}
			{} \rlap{ganaltáat} @ {} @ {} {}
			\rlap{wusheexí} @ {} @ {} @ {} @ {} @ {} {}
		{} léetʼ \rlap{tóode} {} {}
		ash @ \rlap{woog̱éixʼ.} @ {} @ {} @ {}  //
	\glb	{} {} ash ee -n {}
			{} \rt[¹]{gan}- ltáaᵏ -t {}
			wu- d- sh- \rt[¹]{xix} -μμL -í {}
		{} léetʼ tú -dé {}
		ash= wu- i- \rt[²]{g̱eͥxʼ} -μμH //
	\glc	{}[\pr{CP} {}[\pr{PP} \xx{3prx} \xx{base} -\xx{instr} {}]
			{}[\pr{PP} \rt[¹]{burn}- middle -\xx{pnct} {}]
			\xx{pfv}- \xx{mid}- \xx{pej}- \rt[¹]{fall} -\xx{var} -\xx{sub} {}]
		{}[\pr{PP} basket? inside -\xx{all} {}]
		\xx{3prx·o}= \xx{pfv}- \xx{stv}- \rt[²]{throw·\xx{inan}} -\xx{var} //
	\gld	{} {} him {} -with {}
			{} fire- middle -to {}
			\rlap{\xx{ncnj}.\xx{pfv}.run·\xx{sg}} {} {} {} {} -when {}
		{} basket? inside -to {}
		him \rlap{\xx{ncnj}.\xx{pfv}.throw·\xx{sg}} {} {} {} //
	\glft	‘when it ran into the middle of the fire with him he threw him into the basket.’
		//
\endgl
\xe

The form that \citeauthor{swanton:1909} transcribes as \orth{gᴀnᴀłtā′ dîcî′x̣} in (\lastx) is somewhat garbled.  His gloss “into the fire” suggests the same \fm{ganaltáaᵏ} ‘middle of the fire’ in (\ref{ex:91-71-when-run-in-fire-dont-let-burn}).
His gloss “it ran” for \orth{dîcî′x̣} suggests the verb \fm{d-sh-\rt[¹]{xix}} ‘run’ based on \fm{\rt[¹]{xix}} ‘sg.\ fall, move through space; deplete; run’.
But contrary to \citeauthor{swanton:1909}’s transcription, the \orth{dî} element cannot be analyzed as part of the verb because the resulting form \fm{dishíx} or \fm{disheex} implies a verb \fm[*]{d-\rt{shix}} which does not exist: there is no root \fm[*]{\rt{shix}} and there is no suffix \fm[*]{-x} to attach to a root like \fm{\rt[²]{shiʰ}} ‘reach for, touch, stroke; search; help’ or \fm{\rt[²]{shi}} ‘hope’.
The \orth{dî} could alternatively be analyzed as the allative postposition \fm{-dé} ‘toward’ attached to the preceding noun, giving us \fm{ganaltáade} ‘toward the middle of the fire’.
We are then left with \orth{cî′x̣} alone as a verb word which implies either \fm{shíx}, \fm{shéex}, or \fm{sheex}, but all three of these forms are ungrammatical.
If instead we take \orth{dî} as the punctual postposition \fm{-t} followed by a vowel then we would have \fm{ganaltáa-t ishíx} which is grammatical, but this is an imperative meaning ‘run into the middle of the fire!’ that makes no sense in this context.
Working from \citeauthor{swanton:1909}’s gloss “it ran” and \fm{d-sh-\rt[¹]{xix}} ‘run’, if we assume a subordinate clause suffix \fm{-í} and interpret the outstanding \orth{î} as the residue of some prefix syllable reduced down to schwa then we have a plausible form \fm{wusheexí} as given in (\lastx).

\ex\label{ex:91-75-come-back-in-basket}%
\exmn{270.8}%
\begingl
	\glpreamble	ʟe ᴀ′q!awe qo′xodaguttc yułī′t! tūq!. //
	\glpreamble	Tle áxʼ áwé ḵux̱ udagútch, yú léetʼ tóoxʼ. //
	\gla	Tle {} \rlap{áxʼ} @ {} {} \rlap{áwé} @ {}
		ḵux̱ @ \rlap{udagútch,} @ {} @ {} @ {} @ {}
		{} yú léetʼ \rlap{tóoxʼ.} @ {} {} //
	\glb	tle {} á -xʼ {} á -wé
		ḵúx̱= u- d- \rt[¹]{gut} -μH -ch
		{} yú léetʼ tú -xʼ {} //
	\glc	then {}[\pr{PP} \xx{3n} -\xx{loc} {}] \xx{foc} -\xx{mdst}
		\xx{rev}= \xx{zpfv}- \xx{mid}- \rt[¹]{go·\xx{sg}} -\xx{var} -\xx{rep}
		{}[\pr{PP} \xx{dist} basket? inside -\xx{loc} {}] //
	\gld	then {} there -at {} \rlap{it.is} {}
		back\• \rlap{\xx{hab}.go·\xx{sg}} {} {} {} {}
		{} that basket? inside -at {}  //
	\glft	‘Then it was there that he would come back, inside that basket.’
		//
\endgl
\xe

The sentence in (\lastx) is notable for its use of \fm{ḵux̱=d-\rt[¹]{gut}} ‘sg.\ go back (along earlier path)’.
This verb usually has only a literal interpretation which describes a singular entity travelling in the opposite direction along some previously taken path in space.
Here it has an additional metaphorical interpretation where it describes reviving by taking death as a path through space that can be followed in reverse.
This metaphor is unremarkable in English – ‘she \emph{came back} to life’ – but it is relatively uncommon in Tlingit, particularly without explicit mention of the destination \fm{naná} ‘death’.

\ex\label{ex:91-76-become-big-man}%
\exmn{270.9}%
\begingl
	\glpreamble	Aʟē′n qāx nᴀstī′tc. //
	\glpreamble	Aatlein ḵáax̱ nasteech. //
	\gla	{} \rlap{Aatlein} @ {} \rlap{ḵáax̱} @ {} {} \rlap{nasteech.} @ {} @ {} @ {} @ {} //
	\glb	{} aa =tlein ḵáa -x̱ {} n- s- \rt[¹]{tiʰ} -μμL -ch //
	\glc	{}[\pr{PP} \xx{part} =big man -\xx{pert} {}] \xx{ncnj}- \xx{appl}- \rt[¹]{be} -\xx{var} -\xx{rep} //
	\gld	{} \rlap{big} {} man -of {} \rlap{\xx{hab}.be} {} {} {} {} //
	\glft	‘He would become a big man.’
		//
\endgl
\xe

\section{Paragraph 8}\label{sec:91-para-8}

\ex\label{ex:91-77-order-to-speak-outside}%
\exmn{270.10}%
\begingl
	\glpreamble	Adᴀ′x x̣ā′na-awe q!ē′g̣a yū′ᴀq! aq!ā′wana dukā′k. //
	\glpreamble	Aadáx̱ xáanaa áwé xʼéig̱aa yú áxʼ ax̱ʼakaawanáa du káak. //
	\gla	{} \rlap{Aadáx̱} @ {} {} {} xáanaa {} \rlap{áwé} @ {}
		xʼéig̱aa {} yú \rlap{áxʼ} @ {} {}
		\rlap{ax̱ʼakaawanaa} @ {} @ {} @ {} @ {} @ {} @ {}
		{} du káak. {} //
	\glb	{} á -dáx̱ {} {} xáanaa {} á -wé
		xʼéig̱aa {} yú á -xʼ {}
		a- x̱ʼe- k- wu- i- \rt[²]{na(ʼÿ)} -μμH
		{} du káak {}  //
	\glc	{}[\pr{PP} \xx{3n} -\xx{abl} {}] {}[\pr{NP} evening {}] \xx{foc} -\xx{mdst}
		directly {}[\pr{PP} \xx{dist} \xx{3n} -\xx{loc} {}]
		\xx{arg}- mouth- \xx{qual}- \xx{pfv}- \xx{stv}- \rt[²]{order} -\xx{var}
		{}[\pr{DP} \xx{3h·pss} mat·uncle {}] //
	\gld	{} that -after {} {} evening {} \rlap{it.is} {}
		directly {} that there -at {}
		\rlap{3>3.mouth.\xx{ncnj}.\xx{pfv}.order} {} {} {} {} {} {}
		{} his uncle {}  //
	\glft	‘After that it was in the evening that he directly ordered him out at that place to speak, his uncle.’
		//
\endgl
\xe

The verb that \citeauthor{swanton:1909} transcribes as \orth{aq!ā′wana} in (\lastx) is not attested elsewhere.
His gloss “he sent out” implies the root \fm{\rt[²]{na(ʼÿ)}} ‘order, send to do’ as seen earlier in (\ref{ex:91-65-spirits-of-cave-there-order-me}).
His transcription suggests a form like \fm{ax̱ʼaawanáa} but this would be ungrammatical because the prefix \fm{x̱ʼe-} ‘mouth’ with the perfective \fm{wu-} and stative \fm{i-} should instead give a form like \fm{ax̱ʼeiwanáa}.
This could be excused as a misreading of \orth{ē} for \orth{ā}, but it is possibile that \citeauthor{swanton:1909}’s \orth{ā} is correct.
There would need to be an untranscribed prefix between \fm{x̱ʼe-} and the perfective \fm{wu-} to block the regular lengthening of the vowel in \fm{x̱ʼe-} so that it surfaces as \fm{a} (short /\ipa{e}/ is not allowed in the conjunct prefix domain; cf.
\fm{se-} ‘voice’).
The root \fm{\rt[²]{na(ʼÿ)}} is only attested with \fm{k-} \parencites[04/23–25]{leer:1973}[249–250]{leer:1976}[16]{leer:1978b}, so it is likely that \citeauthor{swanton:1909}’s transcription \orth{aq!ā′wana} reflects \fm{ax̱ʼakaawanáa} [\ipa{ʔà.χʼà.kʰàː.wà.ˈnáː}] or perhaps \fm{ax̱ʼkaawanáa} [\ipa{ʔàχʼ.kʰàː.wà.ˈnáː}] with irregular syncope of the vowel of \fm{x̱ʼe-}.

The interpretation of the \fm{x̱ʼe-} ‘mouth’ in \fm{ax̱ʼakaawanáa} must be metaphorical since presumably the uncle’s mouth cannot be ordered to do anything separately from the uncle himself.
The \fm{x̱ʼe-} prefix usually refers to speech in such metaphorical cases, so the verb can be interpreted as ‘order to go speak’.
This is parallel to an attested verb \fm{ji-k-\rt[²]{na(ʼÿ)}} ‘order to work’ with \fm{ji-} ‘hand’ that is metaphorically used to refer to work (cf.\ \fm{yéi jiné} ‘work’), as in \fm{yax̱ has jikawduwanáa} ‘people sent them to work’ \parencite[143.1936]{story-naish:1973}.

\ex\label{ex:91-78-cmere-singers}%
\exmn{270.10}%
\begingl
	\glpreamble	“Hāde′wᴀt ᴀt-cī′ỵî.” //
	\glpreamble	«\!Haadé yú atshéeÿi.\!» //
	\gla	{} \llap{«\!}\rlap{Haadé,} @ {} {}
		{} yú {} \rlap{atshéeÿi} @ {} @ {} @ {} {} {} //
	\glb	{} haaⁿ -dé {}
		{} yú {} at= \rt[²]{shiʰ} -μμH -í {} {} //
	\glc	{}[\pr{PP} \xx{cis} -\xx{all} {}]
		{}[\pr{DP} \xx{dist} {}[\pr{NP} \xx{4n·o}= \rt[²]{sing} -\xx{var} -\xx{nmz} {}] {}] //
	\gld	{} here -to {}
		{} those {} sth\• \rlap{sing} {} -er {} {} //
	\glft	‘“Come here, those singers.”’
		//
\endgl
\xe

\citeauthor{swanton:1909}’s transcription \orth{Hāde′wᴀt ᴀt-cī′ỵî} in (\lastx) suggests something like \fm{Haadéw at atshéeÿi} but this is nonsensical.
The \orth{w} probably conceals the distal determiner \fm{yú} ‘that, those’.
This leaves an apparently duplicate \orth{ᴀt} given that \orth{ᴀt-cī′ỵî.} is \fm{atshéeÿi} ‘singer’ matching \citeauthor{swanton:1909}’s gloss “those that can sing”.
The simplest explanation for this duplication is just a transcription error.
An alternative is \fm{at atshéeÿi} meaning something like ‘singers of things’, but this would require that the object D pronoun \fm{at=} ‘something, things’ in \fm{atshéeÿi} is uninterpreted which is not the case in modern Tlingit.

\ex\label{ex:91-79-no-light-mtn-goats-watching}%
\exmn{270.11}%
\begingl
	\glpreamble	ʟē′ławe ᴀt aka′odag̣ān yug̣ē′ʟ! djᴀ′nuwu yū′tatūk ỵidê′ adołtīnī′ sᴀkᵘ. //
	\glpreamble	Tléil áwé át akawdagaan yú g̱ílʼ, jánu̬wu yú tatóok ÿeedé adultíni sákw. //
	\gla	Tléil \rlap{áwé} @ {} {} \rlap{át} @ {} {}
		\rlap{akawdagaan} @ {} @ {} @ {} @ {} @ {}
		{} yú g̱ílʼ, {} +
		{} {} {} jánu̬wu {} {} yú tatóok \rlap{ÿeedé} @ {} {}
			\rlap{adultíni} @ {} @ {} @ {} @ {} @ {} @ {} {} sákw {} //
	\glb	tléil á -wé {} á -t {}
		a- k- wu- d- \rt[¹]{gan} -μμL
		{} yú g̱ílʼ {}
		{} {} {} jánwu {} {} yú tatóok ÿee -dé {}
			a- du- d- l- \rt[²]{tin} -μH -í {} sákw {} //
	\glc	\xx{neg} \xx{foc} -\xx{mdst} {}[\pr{PP} \xx{3n} -\xx{pnct} {}]
		\xx{xpl}- \xx{hsfc}- \xx{pfv}- \xx{mid}- \rt[¹]{burn} -\xx{var}
		{}[\pr{DP} \xx{dist} cliff {}]
		{}[\pr{PP} {}[\pr{CP} {}[\pr{DP} mtn·goat {}] {}[\pr{PP} \xx{dist} cave below -\xx{all} {}]
			\xx{xpl}- \xx{4h·s}- \xx{mid}- \xx{xtn}- \rt[²]{see} -\xx{var} -\xx{sub} {}] \xx{fut} {}] //
	\gld	not \rlap{it.is} {} {} there\ix{i} -at {}
		\rlap{\xx{zcnj}.\xx{pfv}.light·shine} {} {} {} {} {}
		{} that cliff\ix{i} {}
		{} {} {} mtn·goats\ix{j} {} {} that cave below -to {}
			\rlap{\xx{ncnj}.\xx{impfv}.ppl\ix{j}.watch} {} {} {} {} {} {} {} for {} //
	\glft	‘No light shone there, that cliff, for mountain goats watching below that cave.’
		//
\endgl
\xe

\ex\label{ex:91-80-they-sat-did-bow}%
\exmn{270.12}%
\begingl
	\glpreamble	Yên qē′awe dusᴀ′ksî yū′aosînî, //
	\glpreamble	Yan ḵéi áwé du sáḵsi yóo awsinee; //
	\gla	{} Yan @ \rlap{ḵéi} @ {} @ {} @ {} {} \rlap{áwé} @ {}
		{} du \rlap{sáḵsi} @ {} {}
		yóo @ \rlap{awsinee;} @ {} @ {} @ {} @ {} @ {} //
	\glb	{} ÿán= {} \rt[¹]{ḵe} -μμH {} {} á -wé
		{} du sáḵs -í {}
		yóo= a- wu- s- i- \rt[¹]{niͤʰ} -μμL //
	\glc	{}[\pr{CP} \xx{term}= \xx{zcnj}\· \rt[¹]{sit·pl} -\xx{var} \·\xx{sub} {}] \xx{foc} -\xx{mdst}
		{}[\pr{DP} \xx{3h·pss} bow -\xx{pss} {}]
		thus= \xx{arg}- \xx{pfv}- \xx{csv}- \xx{stv}- \rt[¹]{occur} -\xx{var} //
	\gld	{} done\• \rlap{\xx{csec}.sit·\xx{pl}} {} {} {} {} \rlap{it.is} {}
		{} his bow {} {}
		thus \rlap{3>3.\xx{ncnj}.\xx{pfv}.make.happen} {} {} {} {} {} //
	\glft	‘Them having sat down, he did his bow and arrow;’
		//
\endgl
\xe

The consecutive aspect clause \fm{yan ḵéi} ‘them having sat down’ in (\lastx) is a good example of a plural verb root used without overt plural marking.
This root is typically found with the plural human argument proclitic \fm{has=} as in \fm{yan has ḵéi} ‘they sat down’.
The \fm{has=} is not used here because the mountain goats introduced in (\ref{ex:91-79-no-light-mtn-goats-watching}) are not considered to be human in this context but \fm{has=} requires its pluralization target to be human.

\citeauthor{swanton:1909} translates \fm{du sáḵsi yóo awsinee} in (\lastx) as “he whirled about his bow and arrows” but “whirl” is nowhere present in the Tlingit form.
The verb \fm{awsinee} ‘s/he made it happen’ is the closest Tlingit approximation to English \fm{do}, with no explicit denotation for the event caused by the agent.
The preverb \fm{yóo=} here is also unusual; it is normally a quotative adverbial which can be used instead of \fm{yéi=} ‘thus’ with speech verbs and is related to the particle \fm{yú.á} ‘they say; it is said’.
Alternatively, it is possible that \citeauthor{swanton:1909}’s \orth{yū′} is not \fm{yóo=} but is instead the alternating preverb \fm{yoo=} ‘back and forth, to and fro, up and down, on and off’.
This \fm{yoo=} would entail a \fm{∅}-conjugation motion derivation and thus the verb stem \fm{–née} with \fm{-μμH} rather than \fm{–nee} with \fm{-μμL}.
This could possibly account for \citeauthor{swanton:1909}’s “whirled” because the event could perhaps be spread out in space, but there is no other evidence to support this hypothesis.

\ex\label{ex:91-81-mtn-goats-died-off}%
\exmn{270.12}%
\begingl
	\glpreamble	ʟa qotx cū′wax̣ix̣ yudjê′nuwa. //
	\glpreamble	tle ḵútx̱ shoowaxeex yú jánu̬wu. //
	\gla	tle {} \rlap{ḵútx̱} @ {} {} \rlap{shoowaxeex} @ {} @ {} @ {} @ {}
		{} yú jánu̬wu. {} //
	\glb	tle {} ḵú -dáx̱= {} shu- wu- i- \rt[¹]{xix} -μμL
		{} yú jánwu {} //
	\glc	then {}[\pr{PP} \xx{areal} -\xx{abl}= {}] end- \xx{pfv}- \xx{stv}- \rt[¹]{fall} -\xx{var}
		{}[\pr{DP} \xx{dist} mtn·goat {}] //
	\gld	then {} \rlap{too·much} {} {} \rlap{end.\xx{ncnj}.\xx{pfv}.use·up} {} {} {} {} 
		{} those mtn·goats {} //
	\glft	‘then they all died off, those mountain goats.’
		//
\endgl
\xe

\ex\label{ex:91-82-bottom-of-cave-full}%
\exmn{270.13}%
\begingl
	\glpreamble	Yutatū′kỵî ʟe cā′wahîk. //
	\glpreamble	Yú tatóok ÿee, tle shaawahík. //
	\gla	{} Yú tatóok \rlap{ÿee,} @ {} {} tle
		\rlap{shaawahík.} @ {} @ {} @ {} @ {} //
	\glb	{} yú tatóok ÿee {} {} tle
		sha- wu- i- \rt[¹]{hik} -μH //
	\glc	{}[\pr{PP} \xx{dist} cave below \·\xx{loc} {}] then
		head- \xx{pfv}- \xx{stv}- \rt[¹]{full} -\xx{var} //
	\gld	{} that cave below -at {} then
		\rlap{\xx{zcnj}.\xx{pfv}.full} {} {} {} {} //
	\glft	‘At the bottom of that cave, then it was full.’
		//
\endgl
\xe

\ex\label{ex:91-83-he-instructed-his-uncle}%
\exmn{270.13}%
\begingl
	\glpreamble	Dukā′kawe ā′adjî kā′waqa, //
	\glpreamble	Du káak áwé áa ajikaawaḵaa //
	\gla	{} Du káak {} \rlap{áwé} @ {} 
		{} \rlap{áa} @ {} {}
		\rlap{ajikaawaḵaa} @ {} @ {} @ {} @ {} @ {} @ {} //
	\glb	{} du káak {} á -wé
		{} á -μμL {}
		a- ji- k- wu- i- \rt[²]{ḵa} -μμL //
	\glc	{}[\pr{DP} \xx{3h·pss} mat·uncle {}] \xx{foc} -\xx{mdst}
		{}[\pr{PP} \xx{3n} -\xx{loc} {}]
		\xx{arg}- hand- \xx{qual}- \xx{stv}- \rt[²]{say} -\xx{var} //
	\gld	{} his mat·uncle {} \rlap{it.is} {}
		{} there -at {}
		\rlap{3>3.hand.\xx{ncnj}.\xx{pfv}.say} {} {} {} {} {} {} //
	\glft	‘It was his uncle that he instructed there’
		//
\endgl
\xe

\ex\label{ex:91-84-tear-off-around}%
\exmn{270.14}%
\begingl
	\glpreamble	“Dātx kī′das!îʟ.” //
	\glpreamble	«\!Daatx̱ keedasʼélʼ.\!» //
	\gla	{} {} \llap{«\!}\rlap{Daatx̱} @ {} {}
		\rlap{keedasʼélʼ.\!»} @ {} @ {} @ {} @ {} @ {} //
	\glb	{} {} daa -dáx̱ {}
		k- {} i- d- \rt[²]{sʼelʼ} -μH //
	\glc	{}[\pr{PP} \xx{rflx·pss} around -\xx{abl} {}]
		\xx{hsfc}- \xx{zcnj}\· \xx{2sg·s}- \xx{mid}- \rt[²]{tear} -\xx{var} //
	\gld	{} self’s around -from {}
		\rlap{\xx{imp}.you·\xx{sg}.rip} {} {} {} {} {}  //
	\glft	‘“Tear it off from around yourself”.’
		//
\endgl
\xe

\FIXME{The combination of a covert possessor for inalienable \fm{daa} ‘around’ and the \fm{d-} middle voice prefix indicate that this covert pronoun is a reflexive possessor.
But why?
\citeauthor{swanton:1909}’s translation is just “Take off the hides” without any reflexiveness.
He glosses \orth{Dātx} as “from there” but “there” doesn’t make sense.}

\ex\label{ex:91-85-sings-the-spirit}%
\exmn{270.14}%
\begingl
	\glpreamble	Aỵacî Nix̣â′ ʟēn. //
	\glpreamble	Aÿashí Neex̱á Tlein. //
	\gla	\rlap{Aÿashí} @ {} @ {} @ {}
		{} Neex̱á Tlein. {} //
	\glb	a- ÿ- \rt[²]{shiʰ} -μH
		{} Neex̱á tlein {} //
	\glc	\xx{arg}- face- \rt[²]{sing} -\xx{var}
		{}[\pr{DP} \xx{name} big {}] //
	\gld	\rlap{3>3.face.\xx{gcnj}.\xx{impfv}.sing} {} {} {} //
	\glft	‘He sings Great Neex̱á.’
		//
\endgl
\xe

The name \fm{Neex̱á} in (\lastx) was discussed earlier in the context of (\ref{ex:91-71-when-run-in-fire-dont-let-burn}).
\citeauthor{swanton:1909} transcribes it here as \orth{Nix̣â′} instead of \orth{Nīx̣â′} as in (\ref{ex:91-71-when-run-in-fire-dont-let-burn}) which implies that the first vowel here is short rather than long.
This variation has been ignored in the retranscription here because it is probably a mistranscription or misreading of his original notes.

The verb in (\lastx) is not very well attested.
It is clearly based on the root \fm{\rt[²]{shiʰ}} ‘sing’ and there are a variety of other verbs based on this well known root, e.g.\ \fm{ashí} ‘s/he sings it’, \fm{kei akaawashée} ‘s/he started, broke into singing it (song)’, and \fm{akawlishee} ‘s/he composed it’ \parencites[10/54–57]{leer:1973}.
This particular verb is distinct with its use of the \fm{ÿ-} qualifier which here is probably best identified as incorporated \fm{ÿá} ‘face’.
There are two attestations of verbs based on \fm{\rt[²]{shiʰ}} ‘sing’ which include \fm{ÿ-}, one meaning ‘sing about relatives’ in \fm{ax̱ káak hás daat yoo kooteek kei yakx̱waashée} ‘I sang the doings about my maternal uncles’ \parencite[191.2654]{story-naish:1973} and another meaning ‘sing back to humanity’ in \fm{ḵúx̱de has yadushí} ‘people sing them back to human form, back to life’ \parencites[10/57]{leer:1973}[542]{leer:1976}.
The verb in (\lastx) likely reflects the same basis form as in \citeauthor{leer:1973}’s \fm{ḵúx̱de has yadushí}, so it could plausibly be interpreted as ‘sing alive’ or ‘sing into realization’.
The English translation of (\lastx) is deliberately vague about the particulars, using ‘sing’ without further elaboration.

\ex\label{ex:91-86-run-up-from-inside-repeatedly-say-to-uncle}%
\exmn{270.14}%
\begingl
	\glpreamble	Dutū′tx ke yîcī′x̣ dukā′k yên yuaỵᴀsîqē′k. //
	\glpreamble	Du tóotx̱ kei ishéex du káak yan yoo aÿasiḵéik //
	\gla	{} {} Du \rlap{tóotx̱} @ {} {}
			kei @ \rlap{ishéex} @ {} @ {} @ {} @ {} @ {} {}
		{} du káak {}
		yan @ yoo @ \rlap{aÿasiḵéik.} @ {} @ {} @ {} @ {} @ {} @ {} //
	\glb	{} {} du tú -dáx̱ {}
			kei= {} d- sh- \rt[¹]{xix} -μμH {} {}
		{} du káak {}
		ÿán= yoo= a- ÿ- s- i- \rt[¹]{ḵa} -eH -k //
	\glc	{}[\pr{CP} {}[\pr{PP} \xx{3h·pss} inside -\xx{abl} {}]
			up= \xx{zcnj}\· \xx{mid}- \xx{pej}- \rt[¹]{fall} -\xx{var} \·\xx{sub} {}]
		{}[\pr{DP} \xx{3h·pss} mat·uncle {}]
		\xx{term}= \xx{alt}= \xx{arg}- \xx{qual}- \xx{csv}- \xx{stv}- \rt[¹]{say} -\xx{var} -\xx{rep} //
	\gld	{} {} his inside -from {}
			up \rlap{\xx{csec}.run·\xx{sg}} {} {} {} {} {} {}
		{} his mat·uncle {}
		done\• \xx{alt} \rlap{3>3.\xx{stv}·\xx{impfv}.say·to.\xx{rep}} {} {} {} {} {} //
	\glft	‘Having run up from inside him, he repeatedly says to his uncle’
		//
\endgl
\xe

The verb form \citeauthor{swanton:1909} transcribes as \orth{yîcī′x̣} in (\lastx) is \fm{ishéex} which is a consecutive aspect form.
The consecutive is derived from the realizational aspect with \fm{\xx{cnj}-…i-…-μμH} but the consecutive appears only in adjunct clauses and consequently lacks the stative \fm{i-}.
Since the verb \fm{d-sh-\rt[¹]{xix}} ‘sg.\ run’ has both \fm{d-} and \fm{sh-} and this form lacks \fm{i-}, the Classifier domain spells out as just a single fricative \fm{sh} [\ipa{ʃ}].
The result should look like \fm[*]{shxéex} [\ipa{ʃxíːx}] but this has a complex consonant cluster which is prohibited so an epenthetic syllable \fm{i} [\ipa{ʔì}] is inserted to rescue it giving \fm[*]{ishxéex} [\ipa{ʔìʃ.ˈxíːx}].
But the root \fm{\rt[¹]{xix}} is unique in that its onset /\ipa{x}/ irregularly disappears when immediately preceded by /\ipa{ʃ}/ so that \fm[*]{ishxéex} [\ipa{ʔì.ˈʃíːx}] → \fm{ishéex} [\ipa{ʔì.ˈʃíːx}].
\citeauthor{swanton:1909}’s initial \orth{y} in \orth{yîcī′x̣} is either a mistranscription or reflects the speaker replacing the usual prothetic glottal stop /\ipa{ʔ}/ with a palatal approximant /\ipa{j}/ because of the immediately preceding proclitic syllable \fm{kei} [\ipa{kʰèː}].

\ex\label{ex:91-87-lest-forget-run-in-grab-out}%
\exmn{271.1}%
\begingl
	\glpreamble	“Xān gᴀnᴀłtā′t îcî′x̣ni q!wᴀn akᴀ′ttse îsaq!ā′kᵘ āx dāq xᴀt īcā′dê łīt! tū′de.” //
	\glpreamble	«\!X̱áan ganaltáat ishíxni xʼwán a kát tse iseixʼáaḵw aax̱ daaḵ x̱at isháadi léetʼ tóode.\!» //
	\gla	{} {} \llap{«\!}\rlap{X̱áan} @ {} {} 
			{} \rlap{ganaltáat} @ {} @ {} {}
			\rlap{ishíxni,} @ {} @ {} @ {} @ {} @ {} @ {} {}
		xʼwán
		{} a \rlap{kát} @ {} {}
		tsé
		\rlap{iseixʼáaḵw} @ {} @ {} @ {} @ {}
		{} {} \rlap{aax̱} @ {} {}
			daaḵ @ x̱at @ \rlap{isháadi} @ {} @ {} @ {}
			{} léetʼ \rlap{tóode.\!»} @ {} {} {} //
	\glb	{} {} x̱á -n {}
			{} \rt[¹]{gan}- ltáaᵏ -t {}
			{} d- sh- \rt[¹]{xix} -μH -n -í {}
		xʼwán
		{} a ká -t {}
		tsé
		i- se- u- \rt[¹]{xʼaḵw} -μμH
		{} {} á -dáx̱ {}
			daaḵ= x̱at= i- \rt[²]{shaʼt} -μμH -í
			{} léetʼ tú -dé {} {} //
	\glc	{}[\pr{CP} {}[\pr{PP} \xx{1sg} -\xx{instr} {}]
			{}[\pr{PP} \rt[¹]{burn}- middle -\xx{pnct} {}]
			\xx{zcnj}\· \xx{mid}- \xx{pej}- \rt[¹]{fall} -\xx{var} -\xx{nsfx} -\xx{sub} {}]
		\xx{imp}
		{}[\pr{PP} \xx{3n·pss} \xx{hsfc} -\xx{pnct} {}]
		\xx{admon}
		\xx{2sg·o}- voice- \xx{irr}- \rt[¹]{die·off} -\xx{var}
		{}[\pr{CP} {}[\pr{PP} \xx{3n} -\xx{abl} {}]
			inland= \xx{1sg·o}= \xx{2sg·s}- \rt[²]{grab} -\xx{var} -\xx{sub}
			{}[\pr{PP} basket? inside -\xx{all} {}] {}] //
	\gld	{} {} me -with {}
		{} fire- middle -to {}
		\rlap{\xx{cond}.run} {} {} {} {} {} -if {}
		\xx{imp}
		{} it atop -at {}
		lest
		\rlap{you·\xx{sg}.\xx{admon}.forget} {} {} {} {}
		{} {} it -from {}
			off·fire\• me\• \rlap{\xx{impfv}.grab} {} {} {}
			{} basket inside -to {} {} //
	\glft	‘“If it runs into the middle of the fire with me, beware lest you forget to grab me off out of it into the basket.”’
		//
\endgl
\xe

\ex\label{ex:91-88-he grabbed-him}%
\exmn{271.3}%
\begingl
	\glpreamble	ᴀc nacᴀ′ttc //
	\glpreamble	Ash nashátch. //
	\gla	Ash @ \rlap{nashátch.} @ {} @ {} @ {} //
	\glb	ash= n- \rt[²]{shaʼt} -μH -ch //
	\glc	\xx{3prx·o}= \xx{ncnj}- \rt[²]{grab} -\xx{var} -\xx{rep} //
	\gld	him \rlap{\xx{hab}.grab} {} {} {}  //
	\glft	‘He repeatedly grabbed him.’
		//
\endgl
\xe

\citeauthor{swanton:1909} places the material in (\lastx) at the beginning of a long sentence including (\nextx) at the start of the next paragraph.
But this does not make sense given that \fm{ash} refers to the protagonist since in (\nextx) \fm{ash} refers to the uncle.
Furthermore, \citeauthor{swanton:1909}’s translation actually leaves out (\lastx) as though it does not exist.
Instead, (\lastx) makes the most sense as the last sentence of this paragraph.

\section{Paragraph 9}\label{sec:91-para-9}

\ex\label{ex:91-89-done-put-up-meat-sent-for-wife}%
\exmn{271.3}%
\begingl
	\glpreamble	yên kudag̣a′ yū′tatūk ỵīq! yudjê′nwu kagē′dî ducᴀ′t g̣a ᴀc kā′waqa. //
	\glpreamble	Yan kadulgáa yú tatóok ÿeexʼ yú jánwu kageidí, du shátg̱aa ash kaawaḵaa. //
	\gla	{} Yan @ \rlap{kadulgáa} @ {} @ {} @ {} @ {} @ {} @ {} @ {}
			{} yú tatóok \rlap{ÿeexʼ} @ {} {} +
			{} yú jánwu kageidí, {} {}
		{} du \rlap{shátg̱aa} @ {} {}
		ash @ \rlap{kaawaḵaa.} @ {} @ {} @ {} @ {} //
	\glb	{} ÿán= k- {} du- d- l- \rt[²]{ga} -μμH {} 
			{} yú tatóok ÿee -xʼ {}
			{} yú jánwu kageidí {} {}
		{} du shát -g̱áa {}
		ash= k- wu- i- \rt[²]{ḵa} -μμL //
	\glc	{}[\pr{CP} \xx{term}= \xx{qual}- \xx{zcnj}\· \xx{4h·s}- \xx{mid}- \xx{xtn}-
			\rt[²]{distribute} -\xx{var} \·\xx{sub}
			{}[\pr{PP} \xx{dist} cave below -\xx{loc} {}]
			{}[\pr{DP} \xx{dist} mtn·goat side·meat {}] {}]
		{}[\pr{PP} \xx{3h·pss} wife -\xx{ades} {}]
		\xx{3prx·o}= \xx{qual}- \xx{pfv}- \xx{stv}- \rt[²]{say} -\xx{var} //
	\gld	{} done \rlap{\xx{csec}.put·up} {} {} {} {} {} {} {}
			{} that cave below -at {}
			{} that mtn·goat side·meat {} {}
		{} his wife -for {}
		him \rlap{\xx{pfv}.say·to} {} {} {} {} //
	\glft	‘Having finished putting them away in the bottom of that cave, those mountain goat sides of meat, he told him to go for his wife.’
		//
\endgl
\xe

\citeauthor{swanton:1909}’s transcription \orth{yên kudag̣a′} in (\lastx) naively suggests a form like \fm{yan kudag̱áa} but this seems to be ungrammatical.
The root \fm{\rt[²]{g̱a}} means ‘bashful, shy, afraid to ask’ which is nonsensical in this context.
The root \fm{\rt[²]{ga}} ‘distribute; put away for storage’ is probably intended here, so that \citeauthor{swanton:1909} mistook [\ipa{k}] for [\ipa{q}] as he often does.
The form would then be \fm{yan kudagáa} but this is also problematic because there is only one instance of a verb with \fm{d-\rt[²]{ga}} which is \fm{i daax̱ yakawdigáa} ‘they are scattered around you’ and a related noun \fm{wooshx̱ dagaa} ‘food box with a separator in the middle’ \parencite[f05/5]{leer:1973}.
All other verbs based on \fm{\rt[²]{ga}} all have \fm{l-} like in the examples \fm{woosh g̱uwanáade atx̱á yan kadulgáayjín} ‘people used to put up all kinds of foods’ and \fm{ax̱ yátxʼi x̱ʼéis atx̱á yánde kakḵwalagáa} ‘I’m going to put up food for my children’ \parencite[164.2249–2250]{story-naish:1973}.
A sensible reinterpretation is that \citeauthor{swanton:1909} misheard \fm{yan kadulgáa}, accidentally exchanging the two vowels \fm{a} and \fm{u} and missing the \fm{l}.
The resulting consecutive aspect form with a fourth person subject is suitable for the context, particularly since this reflects a jump forward in the progression of the narrative.

There are two other interesting properties of (\lastx) that deserve mention.
One is the appearance of modern disyllabic \orth{djê′nwu} \fm{jánwu} rather than the apparently archaic trisyllabic \orth{djᴀ′nuwu} \fm{jánu̬wu} found earlier in (\ref{ex:91-79-no-light-mtn-goats-watching}) and (\ref{ex:91-81-mtn-goats-died-off}).
The other is the appearance of two separate postverbal phrases, a PP and a DP, in the consecutive aspect clause based on the verb \fm{yan kadulgáa}.

\ex\label{ex:91-90-went-up-wife-beneath-cave}%
\exmn{271.4}%
\begingl
	\glpreamble	Ān ke ū′waᴀt ducᴀ′t tatū′k taỵī′q!. //
	\glpreamble	Aan kei uwa.át du shát, tatóok taÿeexʼ. //
	\gla	{} \rlap{Aan} @ {} {} kei @ \rlap{uwa.át} @ {} @ {} @ {}
		{} du shát, {}
		{} tatóok \rlap{taÿeexʼ.} @ {} {} //
	\glb	{} á -n {} kei= u- i- \rt[¹]{.at} -μH
		{} du shát {}
		{} tatóok taÿee -xʼ {} //
	\glc	{}[\pr{PP} \xx{3n} -\xx{instr} {}] up= \xx{zpfv}- \xx{stv}- \rt[¹]{go·\xx{pl}} -\xx{var}
		{}[\pr{DP} \xx{3h·pss} wife {}]
		{}[\pr{PP} cave beneath -\xx{loc} {}] //
	\gld	{} her -with {} up \rlap{\xx{pfv}.go·\xx{pl}} {} {} {}
		{} his wife {}
		{} cave beneath -at {} //
	\glft	‘He went up with her, his wife, beneath the cave.’
		//
\endgl
\xe

\ex\label{ex:91-91-pick-up-cooking-basket}%
\exmn{271.5}%
\begingl
	\glpreamble	We-akᴀ′t-ᴀt gatū′sî kᴀgᵘ-tā′ỵî q!wᴀn ts!ū gatā′n. //
	\glpreamble	«\!Wé a kát at gatus.ée ḵákw taÿee xʼwán tsú gataan. //
	\gla	{} \llap{«\!}Wé {} {} a \rlap{kát} @ {} {}
				at @ \rlap{gatus.ée} @ {} @ {} @ {} @ {} @ {} @ {} {} 
			ḵákw taÿee {} xʼwán tsú 
		\rlap{gataan.} @ {} @ {} @ {} //
	\glb	{} wé {} {} a ká -t {}
				at= g- tu- d- s- \rt[¹]{.i} -μμH {} {}
			ḵákw taÿee {} xʼwán tsú
		g- {} \rt[²]{tan} -μμL //
	\glc	{}[\pr{PP} \xx{mdst} {}[\pr{CP} {}[\pr{PP} \xx{3n·pss} \xx{hsfc} -\xx{pnct} {}]
				\xx{4n·o}= \xx{sben}- \xx{1pl·s}- \xx{mid}- \xx{csv}- \rt[¹]{cook} -\xx{var} \·\xx{rel} {}]
			basket below {}] \xx{imp} also
		\xx{gcnj}- \xx{2sg·s}\· \rt[²]{hdl·w/e} -\xx{var} //
	\gld	{} that {} {} its atop -on {}
				sth\• \rlap{for·self.\xx{impfv}.we.make.cooked} {} {} {} {} {} \·that {}
			basket below {} \xx{imp} too
		\rlap{\xx{imp}.you·\xx{sg}.pick·up} {} {} {} //
	\glft	‘“Also pick up that below-basket that we cook things for ourselves on.’
		//
\endgl
\xe

\ex\label{ex:91-92-pound-within-for-wife-mouth}%
\exmn{271.5}%
\begingl
	\glpreamble	“Ayî′k-ᴀ′dî kat!ᴀ′q!” dukā′k ye aỵa′osîqa, “îcᴀ′t q!ēs.” //
	\glpreamble	A yík ádi katʼéx̱ʼ\!» du káak yéi aÿawsiḵaa, «\!i shát x̱ʼéis\!». //
	\gla	{} \llap{«\!}A yík \rlap{ádi} @ {} {} \rlap{katʼéx̱ʼ\!»} @ {} @ {} @ {} @ {}
		{} du káak {} yéi @ \rlap{aÿawsiḵaa} @ {} @ {} @ {} @ {} @ {} @ {}
		{} \llap{«\!}i shát \rlap{x̱'éis\!».} @ {} {} //
	\glb	{} a yík át -í {} k- {} {} \rt[²]{tʼeͥx̱ʼ} -μH
		{} du káak {} yéi= a- ÿ- wu- s- i- \rt[¹]{ḵa} -μμL
		{} i shát x̱ʼé =ÿís {} //
	\glc	{}[\pr{DP} \xx{3n·pss} within thing -\xx{pss} {}] \xx{sro}- \xx{zcnj}\· \xx{2sg·s}\· \rt[²]{pound} -\xx{var}
		{}[\pr{DP} \xx{3h·pss} mat·uncle {}] thus= \xx{arg}- \xx{qual}- \xx{pfv}- \xx{csv}- \xx{stv}-
			\rt[¹]{say} -\xx{var}
		{}[\pr{PP} \xx{2sg·pss} wife mouth =\xx{ben} {}] //
	\gld	{} its within thing -of {} \rlap{\xx{imp}.you·\xx{sg}.pound} {} {} {} {}
		{} his mat·uncle {} thus \rlap{3>3.\xx{pfv}.say·to} {} {} {} {} {} {}
		{} your·\xx{sg} wife mouth =for {} //
	\glft	‘Pound the things within it” he said to his uncle “for your wife to eat”.’
		//
\endgl
\xe

The sentence in (\ref{ex:91-92-pound-within-for-wife-mouth}) confirms that it is possible to break a quotative clause into multiple parts.
What appears to have happened here is the PP \fm{i shát x̱ʼéis} “for your wife’s mouth” has been dislocated from the quotative clause \fm{a yík ádi katʼéx̱ʼ} “pound the things within it” to the right periphery of the matrix clause with the speech verb \fm{yéi aÿawsiḵaa} ‘so he said to him’.

\ex\label{ex:91-93-tailbone-took-from-woman-spirit}%
\exmn{271.6}%
\begingl
	\glpreamble	Ak!ū′łî āx ā′watê yū′cāwᴀt yuqg̣wahē′ỵᴀkᵘ. //
	\glpreamble	A kʼóolʼi aax̱ aawatee yú shaawát du yakg̱wahéiÿág̱u. //
	\gla	{} A \rlap{kʼóolʼi} @ {} {} {} \rlap{aax̱} @ {} {}
		\rlap{aawatee} @ {} @ {} @ {} @ {} +
		{} yú \rlap{shaawát} @ {} {}
		{} du \rlap{yakg̱wahéiÿág̱u.} @ {} @ {} @ {} @ {} @ {} @ {} @ {} {} //
	\glb	{} a kʼóolʼ -í {} {} á -dáx̱ {}
		a- wu- i- \rt[²]{ti} -μμL
		{} yú sháaʷ- ÿát {}
		{} du ÿ- k- w- g̱- \rt[¹]{ha} -eH -áḵw -í {} //
	\glc	{}[\pr{DP} \xx{3n·pss} tailbone -\xx{pss} {}] {}[\pr{PP} \xx{3n} -\xx{abl} {}]
		\xx{arg}- \xx{pfv}- \xx{stv}- \rt[²]{hdl} -\xx{var}
		{}[\pr{DP} \xx{dist} woman- child {}]
		{}[\pr{DP} \xx{3h·pss} \xx{qual}- \xx{qual}- \xx{irr}- \xx{mod}-
			\rt[¹]{mv·invis} -\xx{var} -\xx{dprv} -\xx{pss} {}] //
	\gld	{} her tailbone {} {} {} it -from {}
		\rlap{3>3.\xx{pfv}.handle} {} {} {} {}
		{} that \rlap{woman} {} {}
		{} his \rlap{spirit} {} {} {} {} {} {} {} {} //
	\glft	‘Her tailbone, it took it from there, that woman, his spirit.’
		//
\endgl
\xe

The interpretation of (\lastx) is difficult because this sentence involves several ambiguous third person referents and because it describes a shamanic phenomenon that is poorly understood today.
There are seven possible third person references in (\lastx), some of which probably have the same referent in the discourse:
\begin{inlineenum}
\item	the possessor in \fm{a kʼóolʼi} ‘its tailbone’
\item	the location in \fm{aax̱} ‘from it’
\item	the subject of \fm{aawatee}
\item	the object of \fm{aawatee}
\item	the woman in \fm{yú shaawát} ‘that woman’
\item	the spirit labelled by \fm{yakg̱wahéiyáḵw}
\item	the possessor (if any) of that spirit.
\end{inlineenum}
The larger discourse context suggests that \fm{yú shaawát} is the uncle’s wife mentioned in (\ref{ex:91-89-done-put-up-meat-sent-for-wife}), (\ref{ex:91-90-went-up-wife-beneath-cave}), and (\ref{ex:91-92-pound-within-for-wife-mouth}).
This same woman is likely to be the possessor of the tailbone given that the nonhuman possessive \fm{a} ‘its’ is used for backgrounding instead of human \fm{du} ‘his, hers’.
Since the possessor in \fm{a kʼóolʼi} is not reflexive it is also likely that the subject of \fm{aawatee} is not the woman.
The remainder of the third person references in (\lastx) are unclear.
For example, we cannot confidently say whether the tailbone is the object of \fm{aawatee} or if instead it is the location in \fm{aax̱}.

A related issue with (\lastx) is the interpretation of the noun \fm{yakg̱wahéiÿáḵw} with respect to its syntactic context.
The noun \fm{yakg̱wahéiÿáḵw} is a complex deverbal nominalization.
It is usually translated as just ‘spirit’ \parencites[e.g.][462]{swanton:1908}[\textsc{t}·96]{leer:2001}, but its denotation is more complicated as discussed below.
It is derived from a relatively poorly documented subset of verbs based on the root \fm{\rt[¹]{ha}} ‘move invisibly; appear’ \parencite[1]{leer:1978b} with the deprivative suffix \fm{-áḵw} \~\ \fm{-ḵ} ‘without, lacking’.
The only four attestations of these verbs are from \citeauthor{leer:1973}:
\begin{inlineenum}
\item	perfective \fm{áx̱ kawdihéiyáḵw} “it dwindled to nothingness, vanished”
\item	perfective \fm{áx̱ yakawdzihéiyáḵw} ‘id.’
\item	subordinate habitual \fm{áx̱ yakoog̱as.héiyáḵwji} “before”\footnote{Presumably this gloss is intended to mean something like ‘before it had dwindled to nothing’.}
\item	and finally progressive \fm{yax̱ yaa yanas.héiyáḵw} “they are vanishing”
\end{inlineenum}
\parencite[01/19]{leer:1973}.
The original sources of \citeauthor{leer:1973}’s data are unknown.
The nominalization is typically encountered with the fourth person human possessor \fm{ḵaa} ‘(some)one’s’ as in \fm{ḵaa yakg̱wahéiyág̱u} ‘one’s spirit’ and it is part of the set phrase \fm{L.ulitóogu Ḵaa Yakg̱wahéiyág̱u} ‘Holy Spirit’ (lit.\ ‘one’s spirit that is unsullied’).
It is only occasionally found with other possessors; one example is \fm{a yakg̱wahéiyagu yáx̱ áwé ash tuwáa yatee} ‘it looked to him like her ghost’ \parencite[162.187]{dauenhauer:1987}.

\citeauthor{swanton:1909}’s transcription \orth{yuqg̣wahē′ỵᴀkᵘ} in (\lastx) suggests \fm{yakg̱wahéiÿáḵw} which notably lacks a possessive suffix \fm{-í}, unlike all other attested instances of this noun.
This then implies that it is not possessed which is unknown but certainly plausible for this noun.
But \citeauthor{swanton:1909} also gives two pieces of evidence against the lack of possession here: he glosses and translates the noun as “his spirits” and also gives an alternative form \orth{duyē′gî} which would be \fm{du yéigi} ‘his/her spirit’ with the noun \fm{yéik} ‘spirit’.
This implies that \fm{yakg̱wahéiÿáḵw} should actually be possessed, but \citeauthor{swanton:1909}’s transcription has no hint of a possessive pronoun.

There are three plausible analyses for the problem with \fm{yakg̱wahéiÿáḵw} in (\lastx):
\begin{inlineenum}
\item	\citeauthor{swanton:1909}’s translation is incorrect so the noun \fm{yakg̱wahéiÿáḵw} is not possessed
\item	\citeauthor{swanton:1909}’s transcription is incorrect and the possessor is a missing \fm{du} which refers to the protagonist
\item	\citeauthor{swanton:1909}’s transcription is (almost) correct and the possessor is the preceding \fm{yú shaawát}.
\end{inlineenum}
The analysis adopted here assumes that \citeauthor{swanton:1909}’s translation and his alternative \fm{du yéigi} are correct, so that there is a missing \fm{du}.
\citeauthor{swanton:1909} could have easily missed [\ipa{tù}] following the final [\ipa{t}] in \fm{shaawát} [\ipa{ʃàː.ˈwát}], particularly if the speaker did not fully release that final [\ipa{t}].
Following \citeauthor{swanton:1909}’s translation, this \fm{du} is taken to refer to the protagonist and not to \fm{yú shaawát} nor to anyone else.
The English translation given for (\lastx) is relatively literal to convey the remaining ambiguity of third person reference. 

As noted above, the meaning of the noun \fm{ÿakg̱wahéiÿáḵw} is complicated and unclear.
The gloss in (\lastx) is simply ‘spirit’ but this obscures both the morphological complexity of the noun and the fact that there are several other nouns that are often given the same translation.
Several ethnographers have discussed the variety of spirits in 19th century Tlingit culture, and a few have explicitly mentioned this noun in contrast with others.
\citeauthor{veniaminov:1984} includes a few nouns for spirits: \fm{yéik} and its derivatives like \fm{ḵaa kinaayéigi} and \fm{daag̱i yéigi}, \fm{keewaaḵáawu}, \fm{sʼag̱iḵáawu} \parencite[396–398]{veniaminov:1984}.
In his \textit{Social condition, beliefs, and linguistic relationship of the Tlingit Indians}, \citeauthor{swanton:1908} gives \fm{yéik} \parencite[451]{swanton:1908} as the most common label for any spirit; he also gives several proper names for particular spirits \parencite[453, 454, 460, 461, 465, 467–469]{swanton:1908}, and has the noun in question as \orth{kayūkg̣wahē′ỵakᵘ} \fm{ḵaa yakg̱wahéiÿág̱u} \parencite[460, 462]{swanton:1908}. 

\FIXME{Finish review of the names of spirits.}

\begin{quote}\small
De Laguna’s (1972: 765–766) older consultants in the 1940s–50s and some of my own in the early 1980s still occasionally used the terms (\fm{ḵaa}) \fm{yakg̱wahéiyág̱u} and \fm{ḵaa yahaayí}, but they were somewhat uncertain about the difference between these two terms.
Nevertheless, they still distinguished between an entity that left the body at death and remained forever in the land of the dead (or the Christian heaven) and one that eventually returned and was reincarnated.
Their views were not unlike those of their late nineteenth-century ancestors, summarized by George Emmons (1991: 288) as follows “The Tlingit recognize three entities in man: the material body; \emph{the spirit}, a vital central force thorugh which the body functions during life and which, leaving the body, causes death; and \emph{the soul}, a spiritual element that has no mechanical connection with the body and is eternal, dwelling in the spirit land or returning from time to time to live in different bodies” (italics mine).
\sourceatright{\parencite[57–58]{kan:2016}}
\end{quote}

\ex\label{ex:91-94-never-saved-woman-eating-goat-fat}%
\exmn{271.7}%
\begingl
	\glpreamble	Ayᴀ′xawe ʟēł ye unᴀ′xtc yucā′wᴀt, ᴀtaỵî′ ᴀxa′ yudjê′nwu. //
	\glpreamble	A yáx̱ áwé tléil yéi oonéx̱ch yú shaawát, a taaÿí ax̱á yú jánwu. //
	\gla	{} A yáx̱ {} \rlap{áwé} @ {}
		tléil yéi @ \rlap{oonéx̱ch} @ {} @ {} @ {} @ {}
		{} yú \rlap{shaawát} @ {} {} +
		{} {} a \rlap{taaÿí} @ {} {}
			\rlap{ax̱á} @ {} @ {}
			{} yú jánwu. {} {}  //
	\glb	{} a yáx̱ {} á -wé
		tléil yéi= u- u- \rt[¹]{neͥx̱} -μH -ch
		{} yú sháaʷ- ÿát {}
		{} {} a taaÿ -í {}
			a- \rt[²]{x̱a} -μH
			{} yú jánwu {} {} //
	\glc	{}[\pr{PP} \xx{3n} \xx{sim} {}] \xx{foc} -\xx{mdst}
		\xx{neg} thus= \xx{irr}- \xx{zpfv}- \rt[¹]{save} -\xx{var} -\xx{rep}
		{}[\pr{DP} \xx{dist} woman- child {}]
		{}[\pr{CP} {}[\pr{DP} \xx{3n·pss} fat -\xx{pss} {}]
			\xx{arg}- \rt[²]{eat} -\xx{var}
			{}[\pr{DP} \xx{dist} mtn·goat {}] {}] //
	\gld	{} it like {} \rlap{it.is} {}
		not thus \rlap{\xx{hab}.safe} {} {} {} {}
		{} that \rlap{woman} {} {}
		{} {} its fat {} {}
			\rlap{3>3.\xx{impfv}.eat} {} {}
			{} that mtn·goat {} //
	\glft	‘It was as if she was never satiated, that woman, eating its fat, that mountain goat.’
		//
\endgl
\xe

The phrase \fm{tléil yéi oonéx̱ch} in (\lastx) literally refers to being saved or healed \parencite[cf.][107.1384]{story-naish:1973}.
In this context the intended interpretation is probably something like ‘satiated’ given the reference to eating and \citeauthor{swanton:1909}’s gloss and translation “never got full”.
This reflects a metaphoric use of the root \fm{\rt[¹]{nix̱}} \~\ \fm{\rt[¹]{nex̱}} ‘safe, healed’ to describe the satiation of hunger.
For example, after a satisfying meal an aristocratic guest might say \fm{x̱at wooneix̱} ‘I am saved; I am rescued’.
Some speakers do not recognize this usage today, suggesting that this may have been used primarily in higher registers such as in public oratory.

\ex\label{ex:91-95-never-satisfied-taste-fat}%
\exmn{271.8}%
\begingl
	\glpreamble	ʟēł q!eakutᴀ′nūk yū′tāî. //
	\glpreamble	Tléil x̱ʼéi akootán nooch yú taay. //
	\gla	Tléil {} {} \rlap{x̱ʼéi} @ {} {}
		\rlap{akootán} @ {} @ {} @ {} @ {} @ \•nooch
		{} yú taay. {} //
	\glb	Tléil {} {} x̱ʼé -μμL {}
		a- k- u- \rt[²]{tan} -μH =nooch
		{} yú taaÿ {} //
	\glc	\xx{neg} {}[\pr{PP} \xx{rflx·pss} mouth -\xx{loc} {}]
		\xx{arg}- \xx{qual}- \xx{irr}- \rt[²]{habit} -\xx{var} =\xx{hab·aux}
		{}[\pr{DP} \xx{dist} fat {}] //
	\gld	not {} self’s mouth -to {}
		\rlap{3>3.\xx{impfv}.accustomed} {} {} {} {} \•never
		{} that fat {} //
	\glft	‘She was never satisfied by the taste of it, that fat.’
		//
\endgl
\xe

The transcription \orth{nūk} in (\lastx) suggests the consecutive auxiliary \fm{=nóok}.
But this is both ungrammatical and infelicitous.
Grammatically the consecutive always occurs as an adjunct clause because it describes an eventuality that precedes the one described by the matrix clause.
There is however no matrix clause here, and the next sentence in (\ref{ex:91-96-because-scratch-made-happen}) is incoherent as a consequence of this clause.
Instead \citeauthor{swanton:1909}’s form seems to be a misreading \orth{k} for what was probably \orth{tc} in his manuscript notes.
His original transcription would then be \orth{nūtc} and thus the habitual auxiliary \fm{=nooch} which fits both grammatically and semantically in this context.

\FIXME{Note \fm{taay} not \fm{taaÿ}.}

\ex\label{ex:91-96-because-scratch-made-happen}%
\exmn{271.9}%
\begingl
	\glpreamble	ᴀq!ā′q! duwᴀctu′ akawuʟ̣āgū′tcawe yē ᴀqsayī′n. //
	\glpreamble	A xʼaaxʼ, du washtú akawudlaagóoch áwé yéi akwsayéin. //
	\gla	{} A \rlap{xʼaaxʼ,} @ {} {}
		{} {} {} du \rlap{washtú} @ {} {}
			\rlap{akawudlaagóoch} @ {} @ {} @ {} @ {} @ {} {} {} {} \rlap{áwé} @ {}
		yéi @ \rlap{akwsayéin.} @ {} @ {} @ {} @ {} @ {} @ {} //
	\glb	{} a xʼaa -xʼ {}
		{} {} {} du wásh- tú {} 
			a- k- wu- \rt[²]{dlakw} -μμL -í {} -ch {} á -wé
		 yéi= a- k- u- s- \rt[¹]{ÿaʰ} -eH -n //
	\glc	{}[\pr{PP} \xx{3n·pss} point -\xx{loc} {}]
		{}[\pr{PP} {}[\pr{CP} {}[\pr{DP} \xx{3h·pss} cheek- inside {}]
			\xx{arg}- \xx{qual}- \xx{pfv}- \rt[²]{scratch} -\xx{var} -\xx{sub} {}] -\xx{erg} {}] \xx{foc} -\xx{mdst}
		thus= \xx{arg}- \xx{qual}- \xx{irr}- \xx{csv}- \rt[¹]{happen} -\xx{var} -\xx{nsfx} //
	\gld	{} its point -at {}
		{} {} {} his cheek- inside {}
			\rlap{3>3.\xx{pfv}.scratch} {} {} {} {} {} {} -cause {} \rlap{it.is} {}
		thus \rlap{3>3.\xx{impfv}.make.happen} {} {} {} {} {} {} //
	\glft	‘On the point of it, it was because she had scratched the inside of his cheek that he makes it happen.’
		//
\endgl
\xe

The verb \fm{akwsayéin} in (\lastx) is a causativized counterpart of forms like \fm{yéi kwdayéin} ‘it happens thus; it is like that’ \parencite[03/97]{leer:1973}.
There is a perfective form \fm{yéi akawsiÿaa} in \citeauthor{leer:1976}’s verb catalogue with a description from Johnny Marks “to make plans change, change order of things” \parencite[187]{leer:1976}.
The root \fm{\rt[¹]{ÿaʰ}} ‘move uncertainly; happen, go’ notably has an irregular stem \fm{−ÿéin} with high tone and the \fm{-n} suffix in the imperfective aspect as well as another irregular stem \fm{–ÿein} with low tone and the \fm{-n} suffix in the progressive aspect.

\FIXME{\fm{a xʼaaxʼ} is a bit odd.
Look for any other similar phrases that could match \orth{ᴀq!ā′q!}}

\section{Paragraph 10}\label{sec:91-para-10}

\ex\label{ex:91-97-says-to-uncle}%
\exmn{271.10}%
\begingl
	\glpreamble	Yē ada′ỵaqa do kā′k, //
	\glpreamble	Yéi adaaÿaḵá du káak //
	\gla	Yéi @ \rlap{adaaÿaḵá} @ {} @ {} @ {} @ {}
		{} du káak {} //
	\glb	yéi= a- daa- ÿ- \rt[²]{ḵa} -μH
		{} du káak {} //
	\glc	thus= \xx{arg}- around- \xx{qual}- \rt[²]{say} -\xx{var}
		{}[\pr{DP} \xx{3h·pss} mat·uncle {}] //
	\gld	thus \rlap{3>3.\xx{ncnj}.\xx{impfv}.say·to} {} {} {} {}
		{} his mat·uncle {} //
	\glft	‘So he says to his uncle’
		//
\endgl
\xe

\ex\label{ex:91-98-steady-spirit-neexha-comes}%
\exmn{271.10}%
\begingl
	\glpreamble	“Îtuwū′ q!wᴀn cᴀt!î′q!
Nîx̣â′ neł gu′tni.” //
	\glpreamble	«\!I toowú xʼwán shatʼíxʼ Neex̱á neil gútni.\!» //
	\gla	{} \llap{«\!}I \rlap{toowú} @ {} {} xʼwán
		\rlap{shatʼíxʼ} @ {} @ {} @ {} @ {} +
		{} Neex̱á {} neil @ {} {}
			\rlap{gútni} @ {} @ {} @ {} @ {} {} //
	\glb	{} i tú -í {} xʼwán
		{} {} sh- \rt[¹]{tʼixʼ} -μH
		{} Neex̱á {} neil -t {}
			{} \rt[¹]{gut} -μH -n -í {}  //
	\glc	{}[\pr{DP} \xx{2sg·pss} inside -\xx{pss} {}] \xx{imp}
		\xx{zcnj}\· \xx{2sg·s}\· \xx{csv}- \rt[¹]{hard} -\xx{var}
		{}[\pr{CP} \xx{name} {}[\pr{PP} inside -\xx{pnct} {}]
			\xx{zcnj}- \rt[¹]{go·\xx{sg}} -\xx{var} -\xx{nsfx} -\xx{sub} {}] //
	\gld	{} your \rlap{spirit} {} {} \xx{imp}
		\rlap{\xx{imp}.you·\xx{sg}.make.steady} {} {} {} {}
		{} \xx{name} {} inside -to {}
			\rlap{\xx{cond}.go·\xx{sg}} {} {} {} {} {} //
	\glft	‘“Steady your spirit if \fm{Neex̱á} comes inside.”’
		//
\endgl
\xe

\ex\label{ex:91-99-evening-ordered-uncle-out}%
\exmn{271.11}%
\begingl
	\glpreamble	X̣ā′naawe yoxᴀ′q akā′wana dokā′k. //
	\glpreamble	Xáanaa áwé yóoxʼ áxʼ akaawanáa du káak. //
	\gla	{} Xáanaa {} \rlap{áwé} @ {}
		{} \rlap{yóoxʼ} @ {} {} {} \rlap{áxʼ} @ {} {}
		\rlap{akaawanáa} @ {} @ {} @ {} @ {} @ {} +
		{} du káak. {} //
	\glb	{} xáanaa {} á -wé
		{} yú -xʼ {} {} á -xʼ {}
		a- k- wu- i- \rt[²]{na(ʼÿ)} -μμH
		{} du káak {} //
	\glc	{}[\pr{NP} evening {}] \xx{foc} -\xx{mdst}
		{}[\pr{PP} \xx{dist} -\xx{loc} {}] {}[\pr{PP} \xx{3n} -\xx{loc} {}]
		\xx{arg}- \xx{qual}- \xx{pfv}- \xx{stv}- \rt[²]{send} -\xx{var}
		{}[\pr{DP} \xx{3h·pss} mat·uncle {}] //
	\gld	{} evening {} \rlap{it.is} {}
		{} off -at {} {} there -at {}
		\rlap{3>3.\xx{zpfv}.\xx{pfv}.send} {} {} {} {} {}
		{} his mat·uncle {} //
	\glft	‘It was in the evening that he sent him out there, his uncle.’
		//
\endgl
\xe

\ex\label{ex:91-100-no-light-shone-there-that-cliff}%
\exmn{271.11}%
\begingl
	\glpreamble	ʟēł awe′ ᴀt ak′odagan yugē′ʟ!. //
	\glpreamble	Tléil áwé át akawdagaan yú g̱ílʼ. //
	\gla	Tléil \rlap{áwé} @ {}
		{} \rlap{át} @ {} {}
		\rlap{akawdagaan} @ {} @ {} @ {} @ {} @ {}
		{} yú g̱ílʼ. {}  //
	\glb	tléil á -wé
		{} á -t {}
		a- k- wu- d- \rt[¹]{gan} -μμL
		{} yú g̱ílʼ {} //
	\glc	\xx{neg} \xx{foc} -\xx{mdst}
		{}[\pr{PP} \xx{3n} -\xx{pnct} {}]
		\xx{xpl}- \xx{hsfc}- \xx{pfv}- \xx{mid}- \rt[¹]{burn} -\xx{var}
		{}[\pr{DP} \xx{dist} cliff {}] //
	\gld	not \rlap{it.is} {}
		{} there -at {}
		\rlap{\xx{zcnj}.\xx{pfv}.light·shine} {} {} {} {} {}
		{} that cliff {} //
	\glft	‘No light shone there, that cliff.’
		//
\endgl
\xe

\ex\label{ex:91-101-brown-bears-come}%
\exmn{271.12}%
\begingl
	\glpreamble	X̣ūts! hît yᴀt uwaᴀ′t yutatū′k q!awu′ł. //
	\glpreamble	Xóots hít yát uwa.át, yú tatóok x̱ʼawool. //
	\gla	{} Xóots {} {} hít \rlap{yát} @ {} {}
		\rlap{uwa.át,} @ {} @ {} @ {}
		{} yú tatóok \rlap{x̱ʼawool.} @ {} {} //
	\glb	{} xóots {} {} hít ÿá -t {}
		u- i- \rt[¹]{.at} -μH
		{} yú tatóok x̱ʼé- wool {} //
	\glc	{}[\pr{DP} br·bear {}] {}[\pr{PP} house face -\xx{pnct} {}]
		\xx{zpfv}- \xx{stv}- \rt[¹]{go·\xx{pl}} -\xx{var}
		{}[\pr{DP} \xx{dist} cave mouth- hole {}] //
	\gld	{} br·bear {} {} house face -to {}
		\rlap{\xx{pfv}.go·\xx{pl}} {} {} {}
		{} that cave \rlap{entrance} {} {} //
	\glft	‘Brown bears came up to the front of the house, that cave entrance.’
		//
\endgl
\xe

\ex\label{ex:91-102-extend-out-upwards}%
\exmn{271.13}%
\begingl
	\glpreamble	ʟe yukî′nde sîx̣ᴀ′t //
	\glpreamble	Tle yú kínde sixát. //
	\gla	Tle {} yú \rlap{kínde} @ {} {}
		\rlap{sixát.} @ {} @ {} @ {} //
	\glb	tle {} yú kín -dé {}
		s- i- \rt[¹]{xat} -μH //
	\glc	then {}[\pr{PP} \xx{dist} up -\xx{all} {}]
		\xx{xtn}- \xx{stv}- \rt[¹]{stick·out} -\xx{var} //
	\gld	then {} that up -to {}
		\rlap{\xx{gcnj}.\xx{stv}·\xx{impfv}.extend} {} {} {} //
	\glft	‘Then they just extend out upwards.’
		//
\endgl
\xe

\ex\label{ex:91-103-sing-up-spirit-uncle}%
\exmn{271.13}%
\begingl
	\glpreamble	ke akā′wacī yuyē′k dukā′ktc. //
	\glpreamble	Kei akaawashée yú yéik, du káakch. //
	\gla	Kei @ \rlap{akaawashée} @ {} @ {} @ {} @ {} @ {}
		{} yú yéik, {}
		{} du \rlap{káakch.} @ {} {} //
	\glb	kei= a- k- wu- i- \rt[²]{shiʰ} -μμH
		{} yú yéik {}
		{} du káak -ch {} //
	\glc	up= \xx{arg}- \xx{qual}- \xx{pfv}- \xx{stv}- \rt[²]{sing} -\xx{var}
		{}[\pr{DP} \xx{dist} spirit {}]
		{}[\pr{DP} \xx{3h·pss} mat·uncle -\xx{erg} {}] //
	\gld	up \rlap{3>3.\xx{zcnj}.\xx{pfv}.sing} {} {} {} {} {}
		{} that spirit {}
		{} his mat·uncle {} {} //
	\glft	‘He sang up that spirit, his uncle.’
		//
\endgl
\xe

\citeauthor{swanton:1909} erroneously combines the clauses in (\blastx) and (\lastx) into a single sentence.
Both have main clause verb forms and neither clause has any adjunct or complement relationship to the other.
Furthermore, as a state imperfective (\blastx) elaborates on the description of the scene in (\ref{ex:91-101-brown-bears-come}) whereas (\lastx) describes a new event and so moves the narrative time forward.
Thus by both syntactic and discourse-semantic considerations the two sentences in (\blastx) and (\lastx) are independent.

\ex\label{ex:91-104-finally-came-inside}%
\exmn{271.13}%
\begingl
	\glpreamble	Nełdê′ naā′t. //
	\glpreamble	Neildé naa.áat. //
	\gla	{} \rlap{Neildé} @ {} {}
		\rlap{naa.áat.} @ {} @ {} @ {} //
	\glb	{} neil -dé {}
		n- i- \rt[¹]{.at} -μμH //
	\glc	{}[\pr{PP} inside -\xx{all} {}]
		\xx{ncnj}- \xx{stv}- \rt[¹]{go·\xx{pl}} -\xx{var} //
	\gld	{} inside -to {}
		\rlap{\xx{rlzn}.go·\xx{pl}} {} {} {} //
	\glft	‘At last they came inside.’
		//
\endgl
\xe

The form in (\lastx) looks like realizational aspect with \fm{n-} + \fm{i-} + \fm{-μμH}.
\citeauthor{swanton:1909}’s gloss is “they kept coming” which suggests a progressive, but that would be \fm{neildé ÿaa na.át} with a preverb and a short vowel stem.
The preverb could be easily missed in something like [\ipa{ˈnèːɬ.té‿àː.nà.ˈʔát}], but it would be odd for \citeauthor{swanton:1909} to mistake a short vowel stem as a long vowel.

\ex\label{ex:91-105-partway-up-said-to-him}%
\exmn{271.14}%
\begingl
	\glpreamble	Tc!a akᴀ′t!ut kᴀ′q!awe ye ᴀcia′osîqa //
	\glpreamble	Chʼa a katʼóot káxʼ áwé yéi ash yawsiḵaa //
	\gla	Chʼa {} a katʼóot \rlap{káxʼ} @ {} {} \rlap{áwé} @ {}
		yéi @ ash @ \rlap{yawsiḵaa} @ {} @ {} @ {} @ {} @ {} //
	\glb	chʼa {} a katʼóot ká -xʼ {} á -wé
		yéi= ash= ÿ- wu- s- i- \rt[¹]{ḵa} -μμL //
	\glc	just {}[\pr{PP} \xx{3n·pss} partway \xx{hsfc} -\xx{loc} {}] \xx{foc} -\xx{mdst}
		thus= \xx{3prx·o}= \xx{qual}- \xx{pfv}- \xx{csv}- \xx{stv}- \rt[¹]{say} -\xx{var} //
	\gld	just {} its partway atop -at {} \rlap{it.is} {}
		thus\• him\• \rlap{\xx{ncnj}.\xx{pfv}.say·to} {} {} {} {} {} //
	\glft	‘It was just as it was partway up there that he said to him’
		//
\endgl
\xe

The sentences from (\ref{ex:91-105-partway-up-said-to-him}) through (\ref{ex:91-107-come-inside-calling}) are missing in \citeauthor{swanton:1909}’s English translation.
\citeauthor{swanton:1909} occasionally fails to translate an occasional sentence that is present and glossed in his transcription, but usually this is only a single sentence and not three in a row.
The complete absence of a few sentences could be interpreted as censorship, but elsewhere when \citeauthor{swanton:1909} censors something he does so by switching from English to Latin, and furthermore there is no obvious reason why he would have wanted to censor these particular sentences.
The lacuna here is therefore most likely an error.
It is unclear if this is an error on \citeauthor{swanton:1909}’s part in preparing the English translation or if instead this gap arose during the typesetting process.

\ex\label{ex:91-106-look-out-for-it}%
\exmn{271.14}%
\begingl
	\glpreamble	ag̣a′ qonatī′s. //
	\glpreamble	«\!Aag̱áa ḵunatéesʼ. //
	\gla	{} \llap{«\!}\rlap{Aag̱áa} @ {} {} \rlap{ḵunatéesʼ.} @ {} @ {} @ {} @ {} {} //
	\glb	{} á -g̱áa {} ḵu- n- {} \rt[²]{tisʼ} -μμH //
	\glc	{}[\pr{PP} \xx{3n} -\xx{ades} {}]
		\xx{areal}- \xx{ncnj}- \xx{2sg·s}\· \rt[²]{look·out} -\xx{var} //
	\gld	{} it -for {} \rlap{\xx{imp}.you·\xx{sg}.look·out} {} {} {} {} //
	\glft	‘“Look out for it.’
		//
\endgl
\xe

\citeauthor{swanton:1909} ran together the sentences in (\ref{ex:91-105-partway-up-said-to-him}) and (\ref{ex:91-106-look-out-for-it}) and gave them as separate from (\ref{ex:91-107-come-inside-calling}).
The imperative mood in (\ref{ex:91-106-look-out-for-it}) strongly implies that it is part of the quoted speech together with (\ref{ex:91-107-come-inside-calling}) which \citeauthor{swanton:1909} gives explicitly in quotes.
Also \citeauthor{swanton:1909} glosses (\ref{ex:91-107-come-inside-calling}) but omits it in his translation.

\ex\label{ex:91-107-come-inside-calling}%
\exmn{272.1}%
\begingl
	\glpreamble	“Nełdê′ ỵākugwâsa′.” //
	\glpreamble	Neildé ÿaa ax̱ʼakg̱wasáa.\!» //
	\gla	{} \rlap{Neildé} @ {} {}
		ÿaa @ \rlap{ax̱ʼakg̱wasáa.\!»} @ {} @ {} @ {} @ {} @ {} @ {} //
	\glb	{} neil -dé {}
		ÿaa= a- x̱ʼe- w- g- g̱- \rt[²]{sa} -μμH //
	\glc	{}[\pr{PP} inside -\xx{all} {}]
		along= \xx{arg}- mouth- \xx{irr}- \xx{gcnj}- \xx{mod}- \rt[²]{call} -\xx{var} //
	\gld	{} inside -to {}
		along \rlap{3>3.\xx{zcnj}.\xx{prsp}.call·spirits} {} {} {} {} {} {} //
	\glft	‘It will be calling them inside.”’
		//
\endgl
\xe

The verb that \citeauthor{swanton:1909} transcribes as \orth{ỵākugwâsa′} in (\ref{ex:91-107-come-inside-calling}) is difficult to interpret.
His gloss for it is  “he is going to come” but this is certainly not any kind of verb based on the roots \fm{\rt[¹]{gut}} ‘sg.\ go’ or \fm{\rt[¹]{.at}} ‘pl.\ go’.
The final \orth{sa′} looks like it reflects a root \fm{\rt{sa}} which could be any of \fm{\rt[¹]{saʰ}} ‘narrow’, \fm{\rt[¹]{sa}} ‘breathe; rest’, or \fm{\rt[²]{sa}} ‘call; name’.
The initial \orth{ỵā} appears to be the preverb \fm{ÿaa=} ‘along’.
\citeauthor{swanton:1909}’s gloss “going to” suggests prospective aspect in the remaining portion \orth{kugwâ}.
The verb \fm{ax̱ʼayasáakw} ‘s/he calls it (spirit)’ is a plausible match here, taking \citeauthor{swanton:1909}’s \orth{ku} for \fm{x̱ʼa} and \orth{gwâ} for prospective \fm{kg̱wa}.
All together we have something like \fm{ax̱ʼakg̱wasáa} [\ipa{ʔà.ˌχʼàk.qʷà.ˈsáː}] ‘s/he will call it (spirit)’ preceded by the preverb \fm{ÿaa=} ‘along’, with the third person argument prefix \fm{a-} absorbed into the preverb like [\ipa{ˌɰàːa.ˌχʼàk.qʷà.ˈsáː}].

A problem for this analysis is that the verb \fm{ax̱ʼayasáakw} ‘s/he calls it (spirit)’ is exceedingly rare.
It is only documented by two examples from \citeauthor{story-naish:1973}: \fm{yéik ágé ax̱ʼayasáakw?} ‘is he calling on the spirits?’ and \fm{yéik ax̱ʼeiyasáakw} ‘he calls on the spirits’ \parencite[40.395–396]{story-naish:1973}.
Both of these forms are varieties of the repetitive state imperfective aspect with the \fm{-kw} suffix.
We cannot tell from these forms what conjugation class this verb belongs to, but it is likely \fm{∅}-conjugation.
This is because the related verb \fm{yéi ayasáakw} ‘s/he calls it thus’ (perfective \fm{yéi aawasáa}) is \fm{∅}-conjugation \parencites[09/8]{leer:1973}[487]{leer:1976}, but the lack of other aspect forms of \fm{ax̱ʼayasáakw} means that we cannot exclude various ‘invisible’ derivational processes that could influence conjugation class membership.

Given that this verb is \fm{∅}-conjugation, we can predict the possibility of perfective forms like \fm{ax̱ʼeiwasáa} ‘s/he called it (spirit)’ and prospective forms like \fm{ax̱ʼakg̱wasáa} ‘s/he will call it (spirit)’.
\citeauthor{swanton:1909}’s transcription \orth{ỵākugwâsa′} fits with a prospective aspect form as detailed above, but the \fm{ÿaa=} preverb remains to be explained since it is not normally required for prospective aspect forms.
The \fm{ÿaa=} in this context probably reflects the layering of progressive aspect on top of prospective aspect, hence the translation ‘will be calling’ instead of ‘will call’.
\citeauthor{leer:1991} calls this phenomenon “epiaspect” \parencite[216–217]{leer:1991}; compare “superaspect” in Dene languages \parencites{kari:1979}[118, 130]{kari:1992}[88ff.]{axelrod:1993}.
The use of \fm{ÿaa=} ‘along’ rather than \fm{yei=} ‘down’ or \fm{kei=} ‘up’ is compatible with the assumption that this verb belongs to the \fm{∅}-conjugation class.

\ex\label{ex:91-108-stick-face-inside}%
\exmn{272.1}%
\begingl
	\glpreamble	Wananī′sawe nēł ỵa′odzia yux̣ū′ts!. //
	\glpreamble	Wáa nanée sáwé neil ÿawdzi.áa yú xóots. //
	\gla	{} Wáa \rlap{nanée} @ {} @ {} @ {} {}
		\rlap{sáwé} @ {} @ {}
		{} neil @ {} {} \rlap{ÿawdzi.áa} @ {} @ {} @ {} @ {} @ {} @ {}
		{} yú xóots. {} //
	\glb	{} wáa n- \rt[¹]{niʰ} -μμH {} {} s= á -wé
		{} neil -t {} ÿ- wu- d- s- i- \rt[¹]{.a} -μμH
		{} yú xóots {} //
	\glc	{}[\pr{CP} how \xx{ncnj}- \rt[¹]{happen} -\xx{var} \·\xx{sub} {}]
		\xx{q}= \xx{foc} -\xx{mdst}
		{}[\pr{PP} inside -\xx{pnct} {}] face- \xx{pfv}- \xx{mid}- \xx{csv}- \xx{stv}- \rt[¹]{end·mv} -\xx{var}
		{}[\pr{DP} \xx{dist} br·bear {}] //
	\gld	{} how \rlap{\xx{csec}.happen} {} {} \·while {}
		ever\• \rlap{it.is} {}
		{} inside -to {} \rlap{face.\xx{zcnj}.\xx{pfv}.make.end·move} {} {} {} {} {} {}
		{} that br·bear {} //
	\glft	‘At some point it sticks its face inside, that brown bear.’
		//
\endgl
\xe

\ex\label{ex:91-109-like-down-tied-on-ear}%
\exmn{272.2}%
\begingl
	\glpreamble	Dogu′k kᴀq! q!âʟ! wudu′waduq!wa yᴀ′xawe ỵati′. //
	\glpreamble	Du gúk káxʼ x̱ʼwáalʼ wuduwadúxʼu aa yáx̱ áwé ÿatee. //
	\gla	{} {} {} {} Du gúk \rlap{káxʼ} @ {} {}
			{} x̱ʼwáalʼ @ {}
			\rlap{wuduwadúxʼu} @ {} @ {} @ {} @ {} @ {} @ {}
			aa {} + 
			yáx̱ {} \rlap{áwé} @ {}
		\rlap{ÿatee} @ {} @ {} //
	\glb	{} {} {} {} du gúk ká -xʼ {}
			{} x̱ʼwáalʼ {}
			wu- du- i- \rt[²]{duxʼ} -μH -í {}
			aa {} yáx̱ {} á -wé
		i- \rt[¹]{tiʰ} -μμL //
	\glc	{}[\pr{PP} {}[\pr{DP} {}[\pr{CP} {}[\pr{PP} \xx{3h·pss} ear \xx{hsfc} -\xx{loc} {}]
			{}[\pr{DP} down {}]
			\xx{pfv}- \xx{4h·s}- \xx{stv}- \rt[²]{tie·knot} -\xx{var} -\xx{rel} {}]
			\xx{part} {}] \xx{sim} {}] \xx{foc} -\xx{mdst}
		\xx{stv}- \rt[¹]{be} -\xx{var} //
	\gld	{} {} {} {} his ear atop -on {}
			{} down {}
			\rlap{\xx{zcnj}.\xx{pfv}.ppl.tie·knot} {} {} {} {} -that {}
			one {} like {} \rlap{it.is} {}
		\rlap{\xx{ncnj}.\xx{stv}·\xx{impfv}.be} {} {} //
	\glft	‘It was like one that eagle down had been tied on his ears.’
		//
\endgl
\xe

\ex\label{ex:91-110-wife-recoiled-from}%
\exmn{272.2}%
\begingl
	\glpreamble	Anᴀ′qawe aodiʟ!ᴀ′kᵘ ducᴀ′t. //
	\glpreamble	Aanáx̱ áwé awditlʼékw du shát. //
	\gla	{} \rlap{Aanáx̱} @ {} {} \rlap{áwé} @ {}
		\rlap{awditlʼékw} @ {} @ {} @ {} @ {} @ {}
		{} du shát. {} //
	\glb	{} á -náx̱ {} á -wé
		a- wu- d- i- \rt[²]{tlʼeʼkw} -μH
		{} du shát {} //
	\glc	{}[\pr{PP} \xx{3n} -\xx{perl} {}] \xx{foc} -\xx{mdst}
		\xx{arg}- \xx{pfv}- \xx{mid}- \xx{stv}- \rt[²]{dodge} -\xx{var}
		{}[\pr{DP} \xx{3h·pss} wife {}] //
	\gld	{} that -thru {} \rlap{it.is} {}
		\rlap{3>3.\xx{zcnj}.\xx{pfv}.recoil·from} {} {} {} {} {}
		{} his wife {} //
	\glft	‘It was during this that she recoiled from it, his wife.’
		//
\endgl
\xe

\ex\label{ex:91-111-she-broke-apart}%
\exmn{272.3}%
\begingl
	\glpreamble	ʟe wū′cdᴀx wuʟ!ī′q!. //
	\glpreamble	Tle wóoshdáx̱ woolʼéexʼ. //
	\gla	Tle {} \rlap{wóoshdáx̱} @ {} {}
		\rlap{woolʼéexʼ.} @ {} @ {} @ {} //
	\glb	tle {} wóosh =dáx̱ {}
		wu- i- \rt[¹]{lʼixʼ} -μμH //
	\glc	then {}[\pr{PP} \xx{recip} =\xx{abl} {}]
		\xx{pfv}- \xx{stv}- \rt[¹]{break} -\xx{var} //
	\gld	then {} ea·oth \•from {}
		\rlap{\xx{ncnj}.\xx{pfv}.break} {} {} {} //
	\glft	‘Then she broke apart.’
		//
\endgl
\xe

\ex\label{ex:91-112-made-it-because-scratched-his-mouth}%
\exmn{272.3}%
\begingl
	\glpreamble	Yū′taỵîq!āq! duwᴀctu′ aka′ wuʟ̣agū′tcawe ye aołiyᴀ′x. //
	\glpreamble	Yú taaÿí x̱ʼáax̱ʼ du washtú akaawudlaagúch áwé yéi awliyéx̱. //
	\gla	{} {} {} Yú \{taaÿí x̱ʼáax̱ʼ\} {}
			{} du \rlap{washtú} @ {} {}
			\rlap{akaawudlaagúch} @ {} @ {} @ {} @ {} @ {} {} {} {} \rlap{áwé} @ {}
		yéi @ \rlap{awliyéx̱.} @ {} @ {} @ {} @ {} @ {} //
	\glb	{} {} {} yú \xx{unid} \xx{unid} {}
			{} du wásh- tú {}
			a- k- wu- \rt[²]{dlakw} -μμL -í {}
				-ch {} á -wé
		yéi= a- wu- l- i- \rt[²]{yex̱} -μH //
	\glc	{}[\pr{PP} {}[\pr{CP} {}[\pr{DP} \xx{dist} \xx{unid} \xx{unid} {}]
			{}[\pr{DP} \xx{3h·pss} cheek- inside {}]
			\xx{arg}- \xx{qual}- \xx{pfv}- \rt[²]{scratch} -\xx{var} -\xx{sub} {}]
				-\xx{erg} {}] \xx{foc} -\xx{mdst}
		thus= \xx{arg}- \xx{pfv}- \xx{xtn}- \xx{stv}- \rt[²]{make} -\xx{var} //
	\gld	{} {} {} that \xx{unid} \xx{unid} {}
			{} his cheek- inside {}
			\rlap{3>3.\xx{pfv}.scratch} {} {} {} {} {} {} \·cause {} \rlap{it.is} {}
		thus \rlap{3>3.\xx{pfv}.make} {} {} {} {} {} //
	\glft	‘That \{unidentified\}, he made it because she scratched the inside of his mouth.’
		//
\endgl
\xe

\FIXME{Figure out what \orth{taỵîq!āq!} means.}

\ex\label{ex:91-113-also-ran-into-fire-with-him}%
\exmn{272.4}%
\begingl
	\glpreamble	Gᴀłtā′t ᴀcī′n wudjîx̣ī′x̣ duyē′gî ts!u. //
	\glpreamble	Galtáat ash een wujixeex du yéigi tsú. //
	\gla	{} \rlap{Galtáat} @ {} @ {} {}
		{} ash \rlap{een} @ {} {}
		\rlap{wujixeex} @ {} @ {} @ {} @ {} @ {} +
		{} du \rlap{yéigi} @ {} {} tsú. //
	\glb	{} \rt[¹]{gan}- ltáaᵏ -t {}
		{} ash ee -n {}
		wu- d- sh- i- \rt[¹]{xix} -μμL
		{} du yéik -í {} tsu //
	\glc	{}[\pr{PP} \rt[¹]{burn}- middle -\xx{pnct} {}]
		{}[\pr{PP} \xx{3prx} \xx{base} -\xx{instr} {}]
		\xx{pfv}- \xx{mid}- \xx{pej}- \xx{stv}- \rt[¹]{fall} -\xx{var}
		{}[\pr{DP} \xx{3h·pss} spirit -\xx{pss} {}] also //
	\gld	{} fire- middle -to {}
		{} him \rlap{with} {} {}
		\rlap{\xx{ncnj}.\xx{pfv}.run·\xx{sg}} {} {} {} {} {}
		{} his \rlap{spirit} {} {} also //
	\glft	‘Into the fire it ran with him, his spirit, also.’
		//
\endgl
\xe

The form \fm{galtáa} in (\lastx) is a contraction of \fm{ganaltáa} ‘middle of the fire’.
This noun was discussed in the context of (\ref{ex:91-71-when-run-in-fire-dont-let-burn}).

\ex\label{ex:91-114-so-thats-it}%
\exmn{272.5}%
\begingl
	\glpreamble	Awē′tch!ayu //
	\glpreamble	Áwé chʼa á áyú; //
	\gla	\rlap{Áwé} @ {} chʼa á \rlap{áyú;} @ {} //
	\glb	á -wé chʼa á á -yú //
	\glc	\xx{foc} -\xx{mdst} just \xx{3n} \xx{foc} -\xx{dist} //
	\gld	\rlap{so} {} just it \rlap{it.is} {} //
	\glft	‘So that was it;’
		//
\endgl
\xe

Sentence (\ref{ex:91-114-so-thats-it}) is missing in \citeauthor{swanton:1909}’s English translation although it is present in his transcription and gloss.
In this case it is possible that \citeauthor{swanton:1909} omitted this sentence because it is a comment from the narrator.

\ex\label{ex:91-115-fear-bear-burn-up}%
\exmn{272.5}%
\begingl
	\glpreamble	x̣ūts! djiakułxē′ʟ!awe gᴀnᴀłtā′x kāwagā′n Gu′xk!ᵘsᴀkʷ. //
	\glpreamble	xóots jée akoolx̱éitlʼ áwé ganaltáax̱ kaawagaan Goox̱kʼ Sákw. //
	\gla	{} {} xóots \rlap{jée} @ {} {}
			\rlap{akoolx̱éitlʼ} @ {} @ {} @ {} @ {} @ {} @ {} @ {} {}
		\rlap{áwé} @ {} +
		{} \rlap{ganaltáax̱} @ {} @ {} {}
		\rlap{kaawagaan} @ {} @ {} @ {} @ {}
		{} \rlap{Goox̱kʼ} @ {} Sákw. {} //
	\glb	{} {} xóots jee -H {}
			a- k- u- d- l- \rt[²]{x̱itlʼ} -μμH {} {}
		á -wé
		{} \rt[¹]{gan}- ltáaᵏ -x̱ {}
		k- wu- i- \rt[¹]{gan} -μμL
		{} goox̱ -kʼ sákw {} //
	\glc	{}[\pr{CP} {}[\pr{PP} br·bear poss’n -\xx{loc} {}]
			\xx{arg}- \xx{qual}- \xx{irr}- \xx{mid}- \xx{xtn}-
				\rt[²]{fear} -\xx{var} \·\xx{sub} {}]
		\xx{foc} -\xx{mdst}
		{}[\pr{PP} fire- middle -in {}]
		\xx{qual}- \xx{pfv}- \xx{stv}- \rt[¹]{burn} -\xx{var}
		{}[\pr{DP} slave -\xx{dim} \xx{fut} {}] //
	\gld	{} {} br·bear poss’n -in {}
		\rlap{\xx{gcnj}.\xx{stv}·\xx{impfv}.fear} {} {} {} {} {} {} {} {}
		\rlap{it.is} {}
		{} fire- middle -at {}
		\rlap{\xx{g̱cnj}.\xx{pfv}.burn} {} {} {} {}
		{} slave -little future {} //
	\glft	‘it was while he feared possession of the bear
		that he burned in the middle of the fire, Future Little Slave.’
		//
\endgl
\xe

\section{Paragraph 11}\label{sec:91-para-11}

\ex\label{ex:91-116-cave-turn-to-ash}%
\exmn{272.7}%
\begingl
	\glpreamble	Tc!uʟe′ awe′ wucik!ᴀ′ʟ! yutatū′k, //
	\glpreamble	Chʼu tle áwé wushikélʼ yú tatóok, //
	\gla	Chʼu tle \rlap{áwé} @ {}
		\rlap{wushikélʼ} @ {} @ {} @ {} @ {}
		{} yú tatóok, {} //
	\glb	chʼu tle á -wé
		wu- sh- i- \rt[¹]{kelʼ} -μH
		{} yú tatóok {} //
	\glc	just then \xx{foc} -\xx{mdst}
		\xx{pfv}- \xx{pej}- \xx{stv}- \rt[¹]{ash} -\xx{var}
		{}[\pr{DP} \xx{dist} cave {}] //
	\gld	just then \rlap{it.is} {}
		\rlap{\xx{zcnj}.\xx{pfv}.ash} {} {} {} {}
		{} that cave {} //
	\glft	‘It was then that it turned to ash, that cave,’
		//
\endgl
\xe

The verb that \citeauthor{swanton:1909} transcribes as \orth{wucik!ᴀ′ʟ!} in (\lastx) could be interpreted as \fm{wushikʼátlʼ} ‘s/he/it became silent’ following his transcription exactly.
\citeauthor{swanton:1909} glosses this as ‘creaked’, but that goes against the meaning of the verb root \fm{\rt[¹]{kʼatlʼ}} ‘silent, taciturn, say nothing’.
If \citeauthor{swanton:1909}’s \orth{k!} /\ipa{kʼ}/ is a mistranscription for \fm{k} /\ipa{kʰ}/ then this could be based on one of the three homophonous roots \fm{\rt[²]{kelʼ}} ‘undo, take apart; track’, \fm{\rt[¹]{kelʼ}} ‘pl.\ run away’, and \fm{\rt[¹]{kelʼ}} ‘ash’ \parencite[61]{leer:1978b}.
Of these three, only \fm{\rt[¹]{kelʼ}} ‘ash’ is attested with the \fm{sh-} classifier prefix that is implied by \citeauthor{swanton:1909}’s \orth{c} /\ipa{ʃ}/ with forms like \fm{awshikélʼ} ‘he burned it to ashes’ \parencite[38.366]{story-naish:1973}.
The form in (\lastx) is then apparently \fm{wushikélʼ} which would be an intransitive counterpart of \fm{awshikélʼ}, though this is not otherwise attested.
The meaning ‘turn to ash’ is plausible in this context given that the fire mentioned in (\ref{ex:91-113-also-ran-into-fire-with-him}) somehow grows to encompass everything within the cave.

\ex\label{ex:91-117-man-run-into-skin}%
\exmn{272.7}%
\begingl
	\glpreamble	tc!ūye qā′awe dudugu′ tū′de wudjix̣ī′x̣iyᴀ. //
	\glpreamble	chʼu yé ḵáa áwé du doogú tóode wujixeexi yé. //
	\gla	chʼu yé ḵáa \rlap{áwé} @ {}
		{} {} du \rlap{doogú} @ {} \rlap{tóode} @ {} {} +
			\rlap{wujixeexi} @ {} @ {} @ {} @ {} @ {} @ {} {} yé. //
	\glb	chʼu yé ḵáa á -wé
		{} {} du dook -í tú -dé {}
			wu- d- sh- i- \rt[¹]{xix} -μμL -i {} yé //
	\glc	just way man \xx{foc} -\xx{mdst}
		{}[\pr{CP} {}[\pr{PP} \xx{3h·pss} skin -\xx{pss} inside -\xx{all} {}]
			\xx{pfv}- \xx{mid}- \xx{pej}- \xx{stv}-
				\rt[¹]{fall·\xx{sg}} -\xx{var} -\xx{rel} {}] way //
	\gld	\rlap{ordinary} {} man \rlap{it.is} {}
		{} {} his skin {} inside -to {}
			\rlap{\xx{ncnj}.\xx{pfv}.run·\xx{sg}} {} {} {} {} {} -that {} way //
	\glft	‘it was an ordinary man that way that he ran inside his skin.’
		//
\endgl
\xe

\FIXME{Sentence (\lastx) is missing in \citeauthor{swanton:1909}’s English translation.}

\FIXME{What is \fm{chʼu yé} here?
How is \fm{ḵáa} related to \fm{du doogú} and the subject of \fm{wujixeex}?}

\ex\label{ex:91-118-things-ppl-dry-act-thus}%
\exmn{272.8}%
\begingl
	\glpreamble	Yudusx̣ū′gu-ᴀ′t ayu′ yē kawanū′kᵘ. //
	\glpreamble	Yú dusxoogu át áyú yéi ḵoowanook. //
	\gla	{} Yú {} \rlap{dusxoogu} @ {} @ {} @ {} @ {} @ {} {} át {} \rlap{áyú} @ {}
		yéi @ \rlap{ḵoowanook.} @ {} @ {} @ {} @ {} //
	\glb	{} yú {} du- d- s- \rt[¹]{xuk} -μμL -i {} át {} á -yú
		yéi= ḵu- wu- i- \rt[¹]{nuk} -μμL //
	\glc	{}[\pr{DP} \xx{dist}
			{}[\pr{CP} \xx{4h·s}- \xx{mid}- \xx{csv}- 
				\rt[¹]{dry} -\xx{var} -\xx{rel} {}] thing {}]
		\xx{foc} -\xx{dist}
		thus= \xx{areal}- \xx{pfv}- \xx{stv}- \rt[¹]{feel} -\xx{var} //
	\gld	{} those {} \rlap{\xx{impfv}.ppl.make.dry} {} {} {} {} {} {}
			thing {} \xx{it.is} {}
		thus \rlap{\xx{ncnj}.\xx{pfv}.act} {} {} {} {} //
	\glft	‘It is those things that people are drying that acted thus.’
		//
\endgl
\xe

\FIXME{Apparently \orth{yē kawanū′kᵘ} is \fm{yéi ḵoowanook} given the gloss “did it”.}

\ex\label{ex:91-119-because-burn-happen}%
\exmn{272.8}%
\begingl
	\glpreamble	Yuî′xt! kawagān′etcayu yē wudzigē′t. //
	\glpreamble	Yú íx̱tʼ kaawugaaních áyú yéi wudzigeet. //
	\gla	{} {} {} Yú íx̱tʼ {} \rlap{kaawugaaních} @ {} @ {} @ {} @ {} {} {} {} 
		\rlap{áyú} @ {}
		yéi @ \rlap{wudzigeet.} @ {} @ {} @ {} @ {} @ {} //
	\glb	{} {} {} yú íx̱tʼ {} k- wu- \rt[¹]{gan} -μμL -í {} -ch {}
		á -yú
		yéi= wu- d- s- i- \rt[¹]{git} -μμL //
	\glc	{}[\pr{PP} {}[\pr{CP} {}[\pr{DP} \xx{dist} shaman {}]
			\xx{qual}- \xx{pfv}- \rt[¹]{burn} -\xx{var} -\xx{sub} {}] -\xx{erg} {}]
		\xx{foc} -\xx{dist}
		thus= \xx{pfv}- \xx{mid}- \xx{xtn}- \xx{stv}- \rt[¹]{fall·\xx{sg}} -\xx{var} //
	\gld	{} {} {} that shaman {} 
			\rlap{\xx{g̱cnj}.\xx{pfv}.burn} {} {} {} {} {} -cause {}
		\xx{it.is} {}
		thus \rlap{\xx{ncnj?}.\xx{pfv}.happen} {} {} {} {} {} //
	\glft	‘It was because that shaman was burned that it happened so.’
		//
\endgl
\xe

\ex\label{ex:91-120-burned-up-shaman-and-uncle}%
\exmn{272.9}%
\begingl
	\glpreamble	Hū′tc!ayu ʟe kā′wagān yuî′xt! qᴀ dukā′k. //
	\glpreamble	Hóochʼ áyú, tle kaawagaan yú íx̱tʼ ḵa du káak. //
	\gla	Hóochʼ \rlap{áyú,} @ {}
		tle \rlap{kaawagaan} @ {} @ {} @ {} @ {}
		{} yú íx̱tʼ {} ḵa {} du káak. {} //
	\glb	hóochʼ á -yú
		tle k- wu- i- \rt[¹]{gan} -μμL
		{} yú íx̱tʼ {} ḵa {} du káak {} //
	\glc	finished \xx{foc} -\xx{dist}
		then \xx{qual}- \xx{pfv}- \xx{stv}- \rt[¹]{burn} -\xx{var}
		{}[\pr{DP} \xx{dist} shaman {}] and {}[\pr{DP} \xx{3h·pss} mat·uncle {}] //
	\gld	finished \rlap{it.is} {}
		then \rlap{\xx{g̱cnj}.\xx{pfv}.burn} {} {} {} {}
		{} that shaman {} and {} his uncle {} //
	\glft	‘That’s it, they burned, that shaman and his uncle.’
		//
\endgl
\xe

\section{Paragraph 12}\label{sec:91-para-12}

\ex\label{ex:91-121-burned-up-shaman-and-uncle}%
\exmn{272.11}%
\begingl
	\glpreamble	Xᴀtc yu′-āx yēs qowanū′guỵa ān qoʼa ᴀsiyu′ ʟeł ke wudaqā′t. //
	\glpreamble	X̱ách yú aax̱ yéi s ḵuwanoogu yé aan ḵu.aa ásiyú tléil kei wudaḵáat. //
	\gla	X̱ách {} yú {} {} \rlap{aax̱} @ {} {} 
			yéi @ s @ \rlap{ḵoowanoogu} @ {} @ {} @ {} @ {} @ {} {} yé aan {}
		ḵu.aa \rlap{ásíyú} @ {} @ {}
		tléil kei @ \rlap{wudaḵáat.} @ {} @ {} @ {} @ {} //
	\glb	x̱áju {} yú {} {} á -dáx̱ {}
			yéi= has= ḵu- wu- i- \rt[¹]{nuk} -μμL -i {} yé aan {}
		ḵu.aa á -sí -yú
		tléil kei= u- wu- d- \rt[¹]{ḵaʼt} -μμH //
	\glc	actually {}[\pr{DP} \xx{dist} {}[\pr{CP} {}[\pr{PP} \xx{3n} -\xx{abl} {}]
			thus= \xx{plh}= \xx{areal}- \xx{pfv}- \xx{stv}-
				\rt[¹]{feel} -\xx{var} -\xx{rel} {}] way town {}]
		however \xx{foc} -\xx{dub} -\xx{dist}
		\xx{neg} up= \xx{irr}- \xx{pfv}- \xx{mid}- \rt[¹]{split} -\xx{var} //
	\gld	actually {} those {} {} there -from {}
			thus they \rlap{\xx{ncnj}.\xx{pfv}.act} {} {} {} {} \·that {}
			place town {}
		however \rlap{it.is.maybe} {} {}
		not up \rlap{\xx{g̱cnj}?.\xx{pfv}.exist?} {} {} {} {} //
	\glft	‘Actually the town place from where they acted so, however, apparently
		there is nobody there.’
		//
\endgl
\xe

\FIXME{The phrase \orth{ʟeł ke wudaqā′t} suggests \fm{tléil kei wudaḵáat} but \fm{\rt[¹]{ḵaʼt}} is documented only as \fm{tléíl ḵoodaḵáat} \parencite[883]{leer:1976} or \fm{tléil koodaḵáat} \parencite[74]{leer:1978b}.
The root is homophonous and plausibly identical with \fm{\rt[²]{ḵaʼt}} ‘split’ except that their valencies differ; \textcite[74]{leer:1978b} implies that they are the same root.
Glossing this verb is difficult because it is not only idiomatic but also because the literal meaning is unclear.
The same verb is repeated in (\ref{ex:91-127-another-year-nobody}) without \fm{kei=}.}

\ex\label{ex:91-122-two-roads-for-firewood}%
\exmn{272.12}%
\begingl
	\glpreamble	Dēx ỵᴀtî′ yugᴀ′ng̣adê. //
	\glpreamble	Déix̱ ÿatee yú gáng̱aa dei. //
	\gla	Déix̱ \rlap{ÿatee} @ {} @ {}
		{} yú {} \rlap{gáng̱aa} @ {} {} dei. {} //
	\glb	déix̱ i- \rt[¹]{tiʰ} -μμL
		{} yú {} gán -g̱áa {} dei {} //
	\glc	two \xx{stv}- \rt[¹]{be} -\xx{var}
		{}[\pr{DP} \xx{dist} {}[\pr{PP} firewood -\xx{ades} {}] path {}] //
	\gld	two \rlap{\xx{ncnj}.\xx{impfv}.be} {} {}
		{} those {} firewood -for {} road {} //
	\glft	‘They are two, those roads for firewood.’
		//
\endgl
\xe

\ex\label{ex:91-123-this-one-run-not-back-beach}%
\exmn{272.12}%
\begingl
	\glpreamble	Yā′t!aỵîtê awucix̣î′ ʟēł ỵek ugu′ttc. //
	\glpreamble	Yáatʼaa ÿíde awusheexí tléil ÿeiḵ ugootch. //
	\gla	{} {} \rlap{Yáatʼaa} @ {} @ {} \rlap{ÿíde} @ {} {}
			\rlap{awusheexí} @ {} @ {} @ {} @ {} @ {} @ {} {} +
		tléil ÿeiḵ \rlap{ugootch.} @ {} @ {} @ {} @ {} //
	\glb	{} {} yá -tʼ- aa ÿíᵏ -dé {} a- wu- d- sh- \rt[¹]{xix} -μμL -í {}
		tléil ÿeiḵ= u- u- \rt[¹]{gut} -μμL -ch //
	\glc	{}[\pr{CP} {}[\pr{PP} \xx{prox} -\xx{link}- \xx{part} within -\xx{all} {}]
			\xx{4h·s}- \xx{pfv}- \xx{mid}- \xx{pej}-
				\rt[¹]{fall} -\xx{var} -\xx{sub} {}]
		\xx{neg} beach= \xx{irr}- \xx{zpfv}- \rt[¹]{go·\xx{sg}} -\xx{var} -\xx{rep} //
	\gld	{} {} \rlap{this.one} {} {} within -to {}
			\rlap{ppl.\xx{ncnj}.\xx{pfv}.run·\xx{sg}} {} {} {} {} {} \·when {}
		not beach \rlap{\xx{zcnj}.\xx{hab}.go·\xx{sg}} {} {} {} {} //
	\glft	‘When someone would run in this one they would not come down to the beach.’
		//
\endgl
\xe

\ex\label{ex:91-124-ppl-go-not-seen}%
\exmn{272.13}%
\begingl
	\glpreamble	Łīngî′t qᴀ yā′t!aỵîte awugū′de ʟēł ye dustī′nto; //
	\glpreamble	Leengít ḵa yáatʼaa ÿíde awugoodí tléil yei dustínch; //
	\gla	{} {} Leengít {} ḵa {} \rlap{yáatʼaa} @ {} @ {} \rlap{ÿíde} @ {} {}
			\rlap{awugoodí} @ {} @ {} @ {} @ {} {} +
		tléil yei @ \rlap{dustínch;} @ {} @ {} @ {} @ {} @ {} @ {} @ {} //
	\glb	{} {} leengit {} ḵa {} yá -tʼ- aa ÿíᵏ -de {}
			a- wu- \rt[¹]{gut} -μμL -í {}
		tléil yei= u- du- d- s- \rt[²]{tin} -μH -ch //
	\glc	{}[\pr{CP} {}[\pr{DP} person {}] and
			{}[\pr{PP} \xx{prox} -\xx{link}- \xx{part} within -\xx{all} {}]
			\xx{4h·s}- \xx{pfv}- \rt[¹]{go·\xx{sg}} -\xx{var} -\xx{sub} {}]
		\xx{neg} down= \xx{irr}- \xx{4h·s}- \xx{mid}- \xx{xtn}-
			\rt[²]{see} -\xx{var} -\xx{rep} //
	\gld	{} {} person {} and {} \rlap{this.one} {} {} within -to {}
			\rlap{ppl.\xx{ncnj}.\xx{pfv}.go·\xx{sg}} {} {} {} \·when {}
		not down \rlap{\xx{g̱cnj}.\xx{impfv}.ppl.see.\xx{rep}} {} {} {} {} {} {} {} //
	\glft	‘When people go in this one they are not seen;’
		//
\endgl
\xe

\FIXME{Swanton ran these together as a single sentence, but they’re separate main clauses in coordination with the \fm{ḵa}.}

\FIXME{Discuss possible \fm{ḵa … ḵa} construction. Alternatively, first \fm{ḵa} could be \fm{ḵwa}.
If it should then the translation of (\lastx) should start with ‘but’.}

\ex\label{ex:91-125-ppl-boat-not-seen}%
\exmn{272.13}%
\begingl
	\glpreamble	qa awuqō′xo ʟēł ye dustî′ntc. //
	\glpreamble	ḵa awuḵoox̱ú tléil yei dustínch. //
	\gla	ḵa {} \rlap{awuḵoox̱ú} @ {} @ {} @ {} @ {} {}
		tléil yei @ \rlap{dustínch;} @ {} @ {} @ {} @ {} @ {} @ {} @ {} //
	\glb	ḵa {} a- wu- \rt[¹]{ḵux̱} -μμL -í {}
		tléil yei= u- du- d- s- \rt[²]{tin} -μH -ch //
	\glc	and {}[\pr{CP} \xx{4h·s}- \xx{pfv}- \rt[¹]{go·boat} -\xx{var} -\xx{sub} {}]
		\xx{neg} down= \xx{irr}- \xx{4h·s}- \xx{mid}- \xx{xtn}-
			\rt[²]{see} -\xx{var} -\xx{rep} //
	\gld	and {} \rlap{ppl.\xx{ncnj}.\xx{pfv}.go·boat} {} {} {} \·when {}
		not down \rlap{\xx{g̱cnj}.\xx{impfv}.ppl.see.\xx{rep}} {} {} {} {} {} {} {} //
	\glft	‘and when people go by boat they are not seen.’
		//
\endgl
\xe

\ex\label{ex:91-126-dont-get-to-town}%
\exmn{272.14}%
\begingl
	\glpreamble	ʟēł ānx uqō′x. //
	\glpreamble	Tléil aanx̱ uḵoox̱. //
	\gla	Tléil {} \rlap{aanx̱} @ {} {} \rlap{uḵoox̱.} @ {} @ {} //
	\glb	tléil {} aan -x̱ {} u- \rt[¹]{ḵux̱} -μ //
	\glc	\xx{neg} {}[\pr{PP} town -\xx{var} {}] \xx{irr}- \rt[¹]{go·boat} -\xx{var} //
	\gld	not {} town -to {} \rlap{\xx{zcnj}.\xx{impfv}.go·boat.\xx{rep}} {} {} //
	\glft	‘They do not get to town.’
		//
\endgl
\xe

\FIXME{Interpretation of \orth{uqō′x} as \fm{uḵoox̱} with a third person subject versus \fm{ooḵoox̱} with a fourth person subject \fm{a-}.
Both are plausible.
The fourth person would copy the adjunct clause in the preceding sentence.
The third person would be a bound variable anaphor so that for every indefinite/nonspecific referent in the preceding sentence, that referent is the third person in this sentence.}

\ex\label{ex:91-127-another-year-nobody}%
\exmn{272.14}%
\begingl
	\glpreamble	Ts!u ʟēq! tā′gawe ʟēł wudaqā′t we′ān. //
	\glpreamble	Tsu tléixʼ táakw áwé tléil wudaḵáat wé aan. //
	\gla	Tsu {} tléixʼ táakw {} \rlap{áwé} @ {}
		tléil \rlap{wudaḵáat} @ {} @ {} @ {} @ {} 
		{} wé aan. {} //
	\glb	tsu {} tléixʼ táakw {} á -wé
		tléil u- wu- d- \rt[¹]{ḵaʼt} -μμH
		{} wé aan {} //
	\glc	again {}[\pr{DP} one winter {}] \xx{foc} -\xx{mdst}
		\xx{neg} \xx{irr}- \xx{pfv}- \xx{mid}- \rt[¹]{split} -\xx{var}
		{}[\pr{DP} \xx{mdst} town {}] //
	\gld	again {} one year {} \rlap{it.is} {}
		not \rlap{\xx{g̱cnj}.\xx{pfv}.exist} {} {} {} {} //
	\glft	‘It was (after) one more year that nobody was there, that town.’
		//
\endgl
\xe

\FIXME{See (\ref{ex:91-121-burned-up-shaman-and-uncle}) for the same verb with \fm{kei=}.}

\ex\label{ex:91-128-two-saved-self-woman-daughter}%
\exmn{272.15}%
\begingl
	\glpreamble	Tc!uʟe′ dᴀxanᴀ′xawe ā cwudzinē′x wecā′wᴀt yū′ānq! yūcā′wᴀt dusī′ tîn. //
	\glpreamble	Chʼu tle dáx̱anáx̱ áwé áa sh wudzineix̱, wé shaawát yú aanxʼ, yú shaawát du sée tin. //
	\gla	Chʼu tle \rlap{dáx̱anáx̱} @ {} \rlap{áwé} @ {}
		{} \rlap{áa} @ {} {} sh @ \rlap{wudzineix̱,} @ {} @ {} @ {} @ {} @ {} +
		{} wé \rlap{shaawát} @ {} {}
		{} yú \rlap{aanxʼ,} @ {} {}
		{} yú \rlap{shaawát} @ {} du sée tin. {}  //
	\glb	chʼu tle déix̱ -náx̱ á -wé
		{} á -μ {} sh= wu- d- s- i- \rt[¹]{neͥx̱} -μμL
		{} wé sháaʷ- ÿát {}
		{} yú aan -xʼ {}
		{} yú sháaʷ- ÿát du sée tin {} //
	\glc	just then two -\xx{hum} \xx{foc} -\xx{mdst}
		{}[\pr{PP} \xx{3n} -\xx{loc} {}]
		\xx{rflx·o}= \xx{pfv}- \xx{mid}- \xx{csv}- \xx{stv}- \rt[¹]{safe} -\xx{var}
		{}[\pr{DP} \xx{mdst} woman- child {}]
		{}[\pr{PP} \xx{dist} town -\xx{loc} {}]
		{}[\pr{PP} \xx{dist} woman- child \xx{3h·pss} daughter \xx{instr} {}] //
	\gld	just then \rlap{two} {} \rlap{it.is} {}
		{} there -at {} self= \rlap{\xx{g̱cnj}.\xx{pfv}.make.safe} {} {} {} {} {}
		{} that \rlap{girl} {} {}
		{} that town -at {}
		{} that \rlap{girl} {} her daughter with {} //
	\glft	‘Then it was only two that saved themselves there, that woman in that town, with that woman’s daughter.’
		//
\endgl
\xe

\ex\label{ex:91-129-undecided-leave-with-daughter}%
\exmn{273.1}%
\begingl
	\glpreamble	Tcᴀł cta′yênkᴀx hᴀs tūndatā′nawe ān wuā′t dusī′. //
	\glpreamble	Chʼa l sh daa yankáx̱ has toondatáan áwé aan woo.aat, du sée. //
	\gla	{} Chʼa l {} {} {} sh daa @ {} {} \rlap{yankáx̱} @ {} {} {} {}
			has @ \rlap{toondatáan} @ {} @ {} @ {} @ {} @ {} @ {} {}
			\rlap{áwé} @ {} +
		{} \rlap{aan} @ {} {} \rlap{woo.aat,} @ {} @ {} @ {}
		{} du sée. {} //
	\glb	{} chʼa l {} {} {} sh daa {} {} ÿán- ká {} -x̱ {}
			has= tu- u- n- d- \rt[²]{tan} -μμH {} {} á -wé
		{} á -n {} wu- i- \rt[¹]{.at} -μμL
		{} du sée {} //
	\glc	{}[\pr{CP} just \xx{neg}
			{}[\pr{PP} {}[\pr{DP} {}[\pr{PP} \xx{rflx·pss} around -\xx{loc} {}]
				shore- \xx{hsfc} {}] -\xx{pert} {}]
			\xx{plh}= inside- \xx{irr}- \xx{ncnj}- \xx{mid}-
				\rt[²]{hdl·w/e} -\xx{var} -\xx{sub} {}] \xx{foc} -\xx{mdst}
		{}[\pr{PP} \xx{3n}\ix{i} -\xx{instr} {}]
		\xx{pfv}- \xx{stv}- \rt[¹]{go·\xx{pl}} -\xx{var}
		{}[\pr{DP} \xx{3h·pss} daughter\ix{i} {}] //
	\gld	{} just not {} {} {} self’s around -at {} shore- atop {} -at {}
			they \rlap{\xx{irr}.\xx{ncnj}.\xx{csec}.think} {} {} {} {} {} {} {}
			\rlap{it.is} {}
		{} her\ix{i} -with {} \rlap{\xx{ncnj}.\xx{pfv}.go·\xx{pl}} {} {} {}
		{} her daughter\ix{i} {} //
	\glft	‘Having not made up their minds about themselves, she went with her, her daughter.’
		//
\endgl
\xe

The phrase \fm{chʼa l sh daa yankáx̱} is an uncommon idiom in combination with the verb \fm{has toondatáan}.
A similar phrase occurs twice in \fm{Shaadaaxʼ} Robert Zuboff’s “Mosquito” \parencite[78 lines 118 \&\ 120]{dauenhauer:1987}: \fm{tléil a daaxʼ yankáx̱, toodashátx̱} “when he couldn’t make up his mind, he thought” and \fm{Áyá chʼa l a daa yankáx̱, toodashátx̱i áwé awli.óox,} “And while he couldn’t make up his mind, he blew on it”, though these use a verb based on \fm{\rt[²]{shaʼt}} ‘grab, hold’ rather than \fm{\rt[²]{tan}} ‘handle long, wooden, or empty container’.
The \fm{yanká} compound noun elsewhere appears in \fm{yankáx̱ sh yawdigút} “he stopped walking after wandering around”, \fm{yankáxʼ N} “real N”, and \fm{yankát uwadáa} “tide is high” from \textcite[f06/8]{leer:1973} from an unknown souce.
\citeauthor{leer:1976} seems to consider \fm{yanká} to include \fm{ÿán} ‘shore’ since he also has \fm{yankáxʼ N} “real N” and \fm{yankát uwadáa} “tide is high” listed under \fm{ÿ} \parencite[03/139]{leer:1973}, although he does not list the \fm{yankáx̱} form there which could be significant.
The negative \fm{l} in the phrase \fm{chʼa l sh daa yankáx̱} is clausally active because \citeauthor{swanton:1909} has \orth{tūndatā′n} with \orth{ū} rather than \fm{tundatā′n} with \orth{u}, where this long \fm{oo} results from the combination of \fm{tu-} ‘inside; mind’ and the irrealis prefix \fm{u-}.
The same long vowel can be seen in both instances of \fm{toodashátx̱} from \fm{Shaadaaxʼ} Robert Zuboff.
These appearances of the irrealis prefix \fm{u-} are predicted if the negative particle \fm{l} has its normal behaviour of triggering irrealis marking in the verb of its clause.

\ex\label{ex:91-130-who-should-marry-she-says}%
\exmn{273.2}%
\begingl
	\glpreamble	“Adū′s gî qasī′ g̣aca′ yū′awe q!ayaqa′. //
	\glpreamble	«\!Aadóo sgí ḵaa sée ag̱ashaa\!» yóo áwé x̱ʼayaḵá.  //
	\gla	{} \llap{«\!}Aadóo \rlap{sgí} @ {} {} ḵaa sée {}
			\rlap{ag̱ashaa} @ {} @ {} @ {} @ {} {} +
		yóo \rlap{áwé} @ {} \rlap{x̱ʼayaḵá.} @ {} @ {} @ {} //
	\glb	{} aadóo s= gí {} ḵaa sée {} a- {} g̱- \rt[²]{shaʷ} -μμL {}
		yóo á -wé x̱ʼe- ÿ- \rt[¹]{ḵa} -μH //
	\glc	{}[\pr{CP} aadóo \xx{q}= \xx{yn} {}[\pr{DP} \xx{4h·pss} daughter {}]
			\xx{arg}- \xx{zcnj}\· \xx{mod}- \rt[²]{woman} -\xx{var} {}]
		\xx{quot} \xx{foc} -\xx{mdst} mouth- \xx{qual}- \rt[¹]{say} -\xx{var} //
	\gld	{} who \rlap{maybe} {} {} one’s daughter {}
			\rlap{\xx{3>3}.\xx{hort}.marry} {} {} {} {} {}
		thus \rlap{it.is} {} \rlap{\xx{ncnj}.\xx{impfv}.say} {} {} {} //
	\glft	‘“Who perhaps should marry one’s daughter” is what she says.’
		//
\endgl
\xe

\ex\label{ex:91-131-ice-end-wading-ashore-heron}%
\exmn{273.3}%
\begingl
	\glpreamble	Yut!ī′q! cukᴀ′t dāq nahē′n łᴀq! //
	\glpreamble	Yú tʼéexʼ shukát daaḵ nahéin láx̱ʼ; //
	\gla	{} Yú tʼéexʼ \rlap{shukát} @ {} @ {} {}
		daaḵ @ \rlap{nahéin} @ {} @ {} @ {}
		{} láx̱ʼ. {} //
	\glb	{} yú tʼéexʼ shú- ká -t {}
		dáaḵ= n- \rt[¹]{hu} -eμH -n
		{} láx̱ʼ {} //
	\glc	{}[\pr{PP} \xx{dist} ice end- \xx{hsfc} -\xx{pnct} {}]
		inland= \xx{ncnj}- \rt[¹]{wade·\xx{sg}} -\xx{var} -\xx{nsfx}
		{}[\pr{DP} heron {}] //
	\gld	{} that ice end- atop -around {}
		ashore= \rlap{\xx{zcnj}.\xx{prog}.wade·\xx{sg}} {} {} {}
		{} heron {} //
	\glft	‘It is wading ashore on the end of the ice, a heron;’
		//
\endgl
\xe

\ex\label{ex:91-132-thats-what-spoke-to-them}%
\exmn{273.3}%
\begingl
	\glpreamble	a′awe hᴀsduī′t q!ewatᴀ′n. //
	\glpreamble	á áwé hasdu eet x̱ʼeiwatán. //
	\gla	{} á {} \rlap{áwé} @ {} {} \rlap{hasdu} @ {} \rlap{eet} @ {} {} 
		\rlap{x̱ʼeiwatán.} @ {} @ {} @ {} @ {} //
	\glb	{} á {} á -wé {} has= du ee -t {}
		x̱ʼe- wu- i- \rt[²]{tan} -μH //
	\glc	{}[\pr{DP} \xx{3n} {}] \xx{foc} -\xx{mdst}
		{}[\pr{PP} \xx{plh}= \xx{3h} \xx{base} -\xx{pnct} {}]
		mouth- \xx{pfv}- \xx{stv}- \rt[²]{hdl·w/e} -\xx{var} //
	\gld	{} it {} \rlap{it.is} {} {} \rlap{them} {} {} -to {} 
		\rlap{\xx{zcnj}.\xx{pfv}.speak} {} {} {} {} //
	\glft	‘it is that which spoke to them.’
		//
\endgl
\xe

Sentences (\ref{ex:91-131-ice-end-wading-ashore-heron}) and (\ref{ex:91-132-thats-what-spoke-to-them}) were run together as a single sentence by \citeauthor{swanton:1909}.
They are separated here because the two verbs are main clause forms and because (\ref{ex:91-132-thats-what-spoke-to-them}) includes an initial focus structure which would be unusual in an embedded clause.

\ex\label{ex:91-133-how-am-i-me}%
\exmn{273.4}%
\begingl
	\glpreamble	“Wâ′sᴀ xᴀt ỵate′, xᴀt.” //
	\glpreamble	«\!Wáa sá x̱at ÿatee, x̱át?\!» //
	\gla	{} \llap{«\!}Wáa sá {} x̱at @ \rlap{ÿatee,} @ {} @ {} {} x̱át?\!» {} //
	\glb	{} wáa sá {} x̱at= i- \rt[¹]{tiʰ} -μμL {} x̱át {} //
	\glc	{}[\pr{QP} how \xx{q} {}] \xx{1sg·o}= \xx{stv}- \rt[¹]{be} -\xx{var}
		{}[\pr{DP} \xx{1sg} {}] //
	\gld	{} how \xx{q} {} me= \rlap{\xx{ncnj}.\xx{impfv}.be} {} {}
		{} me {} //
	\glft	‘“How am I, me?”’
		//
\endgl
\xe

A more idiomatic and less literal translation of (\lastx) would be “What about me, me?” using the English wh-question PP ‘what about’ rather than the English wh-question adverb ‘how’.
Another alternative is ‘how about’.
The more literal translation has been used to avoid obscuring the distinct structure of the Tlingit expression.
\citeauthor{swanton:1909} may have had a similar line of thought behind his English translation of “How am I?”.

\ex\label{ex:91-134-what-with-she-said-to-him}%
\exmn{273.4}%
\begingl
	\glpreamble	“Hᴀdā′tîn sᴀ,” yū′aciaosiqa yucā′wᴀt. //
	\glpreamble	«\!Ha daat een sá?\!» yóo ash yawsiḵaa yú shaawát. //
	\gla	{} \llap{«\!}Ha daat \rlap{een} @ {} sá?\!» {} 
		yóo @ ash @ \rlap{yawsiḵaa} @ {} @ {} @ {} @ {} @ {} +
		{} yú \rlap{shaawát.} @ {} {}  //
	\glb	{} ha daat ee -n sá {}
		yóo= ash= ÿ- wu- s- i- \rt[¹]{ḵa} -μμL
		{} yú sháaʷ- ÿát {} //
	\glc	{}[\pr{CP} well what \xx{base} -\xx{instr} \xx{q} {}]
		\xx{quot}= \xx{3prx·o}= \xx{qual}- \xx{pfv}- \xx{csv}- \xx{stv}-
			\rt[¹]{say} -\xx{var}
		{}[\pr{DP} \xx{dist} woman- child {}] //
	\gld	{} well what {} -with ? {}
		so him \rlap{\xx{ncnj}.\xx{pfv}.say·to} {} {} {} {} {}
		{} that \rlap{woman} {} {} //
	\glft	‘“Well what with?” she said to him, that woman.’
		//
\endgl
\xe

\ex\label{ex:91-135-slush-rolls-ashore-stand-inside}%
\exmn{273.5}%
\begingl
	\glpreamble	“Kᴀnē′q xān dāq ᴀqg̣atcū′kun ᴀtū′ yên xâhantc //
	\glpreamble	«\!Kanéiḵ x̱áan daaḵ akg̱ajóoxun a tóo yan x̱ahánch; //
	\gla	{} {} \llap{«\!}\rlap{Kanéiḵ} @ {} @ {} {} {} \rlap{x̱áan} @ {} {}
			daaḵ @ \rlap{akg̱ajóoxun} @ {} @ {} @ {} @ {} @ {} @ {} {} +
		{} a \rlap{tóo} @ {} {}
		yan @ \rlap{x̱ahánch;} @ {} @ {} @ {} //
	\glb	{} {} ká- \rt[¹]{neʼḵ} -μμH {} {} x̱á -n {}
			dáaḵ= a- k- {} g̱- \rt[¹]{juͥx} -μμH -ín {}
		{} a tú -μH {}
		ÿán= x̱- \rt[¹]{han} -μH -ch //
	\glc	{}[\pr{CP} {}[\pr{DP} \xx{hsfc}- \rt[¹]{slush} -\xx{var} {}]
			{}[\pr{PP} \xx{1sg} -\xx{instr} {}]
			inland= \xx{arg?}- \xx{hsfc}- \xx{zcnj}\· \xx{mod}-
				\rt[¹]{roll} -\xx{var} -\xx{ctng} {}]
		{}[\pr{PP} \xx{3n·pss} inside -\xx{loc} {}]
		\xx{term}= \xx{1sg·s}- \rt[¹]{stand·\xx{sg}} -\xx{var} -\xx{rep} //
	\gld	{} {} \rlap{slush} {} {} {} {} me -with {}
			inland \rlap{\xx{zcnj}.\xx{ctng}.roll} {} {} {} {} {} {} {}
		{} its inside -at {}
		done \rlap{\xx{impfv}.I.stand·\xx{sg}.\xx{rep}} {} {} {} //
	\glft	‘“Whenever the slush rolls ashore with me, I stand inside it;’
		//
\endgl
\xe

\FIXME{Discuss interpretation as ‘tolerate, withstand’, per \fm{a yáa yan uwahán} “he withstood it” \parencite[38]{leer:1976} and \fm{tléil aadé a yáa yan ḵwahani yé} “I can’t stand it” \parencite[01/59]{leer:1973}.}

\FIXME{What is the \fm{a-} doing here?}

\ex\label{ex:91-136-thats-what-with}%
\exmn{273.5}%
\begingl
	\glpreamble	ān hā′awe.” //
	\glpreamble	aan haa áwé.\!» //
	\gla	{} \rlap{aan} @ {} {} haa \rlap{áwé.\!»} @ {} //
	\glb	{} á -n {} haa á -wé //
	\glc	{}[\pr{PP} \xx{3n} -\xx{instr} {}] ?? á -wé //
	\gld	{} it -with {} \xx{unkn} \xx{foc} -\xx{mdst} //
	\glft	‘it’s with that.”’
		//
\endgl
\xe

\FIXME{Can’t identify \fm{haa} there.
Could possibly be a repetition of the \fm{ha} in (\ref{ex:91-134-what-with-she-said-to-him}) but no evidence of that anywhere else.
Does not make either semantic or morphological sense as any of the verb roots with the shape \fm{\rt{ha}}. Also nonsense as \fm{haa} ‘our’.
Any possible mistranscriptions?}

\ex\label{ex:91-137-come-home-with-us}%
\exmn{273.5}%
\begingl
	\glpreamble	“Hᴀ nē′łdê hā′īn naᴀ′dî,” //
	\glpreamble	«\!Ha neildé haa een na.ádi\!» //
	\gla	«\!Ha {} \rlap{neildé} @ {} {}
		{} haa \rlap{een} @ {} {}
		\rlap{na.ádi\!»} @ {} @ {} @ {} @ {} //
	\glb	\pqp{}ha {} neil -dé {}
		{} haa ee -n {}
		n- {} \rt[¹]{.at} -μH -í //
	\glc	\pqp{}well {}[\pr{PP} inside -\xx{all} {}]
		{}[\pr{PP} \xx{1pl} \xx{base} -\xx{instr} {}]
		\xx{ncnj}- \xx{2sg·s}\· \rt[¹]{go·\xx{pl}} -\xx{var} -\xx{sub} //
	\gld	\pqp{}well {} home -to {}
		{} us {} with {}
		\rlap{\xx{ncnj}.\xx{imp}.you·\xx{sg}.go·\xx{pl}} {} {} {} {} //
	\glft	‘“Well come home with us”’
		//
\endgl
\xe

\FIXME{Note imperative of plural root with covert singular subject.
Also note unusual subordinate clause, comparable with the more common case of subordinate hortatives.
Alternative interpretation of \orth{naᴀ′dî} as \fm{naa.ádi} is ungrammatical.
Interpretation as progressive is weird in this context and the expected \fm{ÿaa=} is missing.}

\ex\label{ex:91-138-she-said-that-woman}%
\exmn{273.6}%
\begingl
	\glpreamble	yū′aciaosîqa yucā′wᴀttc. //
	\glpreamble	yóo ash yawsiḵaa yú shaawátch. //
	\gla	yóo @ ash @ \rlap{yawsiḵaa} @ {} @ {} @ {} @ {} @ {}
		{} yú \rlap{shaawátch.} @ {} @ {} {} //
	\glb	yóo= ash= ÿ- wu- s- i- \rt[¹]{ḵa} -μμL
		{} yú sháaʷ- ÿát -ch {} //
	\glc	\xx{quot}= \xx{3prx·o}= \xx{qual}- \xx{pfv}- \xx{csv}- \xx{stv}-
			\rt[¹]{say} -\xx{var}
		{}[\pr{DP} \xx{dist} woman- child -\xx{erg} {}] //
	\gld	so him \rlap{\xx{ncnj}.\xx{pfv}.say·to} {} {} {} {} {}
		{} that \rlap{woman} {} {} {} //
	\glft	‘she said, that woman.’
		//
\endgl
\xe

\ex\label{ex:91-139-then-married-her}%
\exmn{273.6}%
\begingl
	\glpreamble	ʟe ᴀc uwaca′ yułᴀ′q!tc. //
	\glpreamble	Tle ash uwasháa yú láx̱ʼch. //
	\gla	Tle ash @ \rlap{uwasháa} @ {} @ {} @ {}
		{} yú \rlap{láx̱ʼch.} @ {} {} //
	\glb	tle ash= u- i- \rt[²]{shaʷ} -μμH
		{} yú láx̱ʼ -ch {} //
	\glc	then \xx{3prx·o}= \xx{zpfv}- \xx{stv}- \rt[²]{woman} -\xx{var}
		{}[\pr{DP} \xx{dist} heron -\xx{erg} {}] //
	\gld	then her \rlap{\xx{zcnj}.\xx{pfv}.marry} {} {} {}
		{} that heron {} {} //
	\glft	‘Then he married her, that heron did.’
		//
\endgl
\xe

\ex\label{ex:91-140-child-sent-to-her}%
\exmn{273.7}%
\begingl
	\glpreamble	ʟe duī′t yêts!djîwaha′. //
	\glpreamble	Tle du eet yáts jeewaháa. //
	\gla	Tle {} du \rlap{eet} @ {} {} {} yáts {}
		\rlap{jeewaháa.} @ {} @ {} @ {} @ {} //
	\glb	tle {} du ee -t {} {} ÿáts {}
		ji- wu- i- \rt[¹]{ha} -μμH //
	\glc	then {}[\pr{PP} \xx{3h·pss} \xx{base} -\xx{pnct} {}] {}[\pr{DP} child {}]
		hand- \xx{pfv}- \xx{stv}- \rt[¹]{appear} -\xx{var} //
	\gld	then {} her {} -to {} {} child {}
		\rlap{\xx{zcnj}.\xx{pfv}.be·sent} {} {} {} {} //
	\glft	‘Then a child was sent to her.’
		//
\endgl
\xe

Although translated with passive, the sentence in (\lastx) is not actually passive in Tlingit.
The intransitive verb \fm{jeewahaa} (motion) ‘it\pr{obj} was sent’ is based on the monovalent root \fm{\rt[¹]{ha}} ‘move imperceptibly; appear’ and has an object as its only core argument.
The transitive equivalent is the causative \fm{ajiwlihaa} (motion) ‘s/he sent him/her/it’ \parencite[10]{leer:1976}.
The sense of agency is implied by the presence of the \fm{ji-} ‘hand’ incorporated noun, but this does not introduce any additional arguments.
A more literal translation could be ‘a child appeared to her by hand’.
The well known noun \fm{jinaháa} ‘misfortune, accident, bad luck’ is derived from the same verb and is more literally ‘something that appears by hand’.

The noun \fm{yáts} in (\lastx) is a variant form of \fm{yát} ‘child’ that has become obsolete.
It is frozen in this expression although many people today have replaced it with the regular \fm{yát}.
It can also be identified etymologically in the noun \fm{atkʼátskʼu} ‘young boy’ that contains the possessive pronoun \fm{at} ‘something’s and the compound \fm{kʼiyátskʼu} from \fm{kʼí} ‘base’, \fm{ÿáts} ‘child’, diminutive \fm{-kʼʷ}, and the possessive suffix \fm{-í} which was once literally ’something’s base-child’.

\ex\label{ex:91-141-it-was-born}%
\exmn{273.7}%
\begingl
	\glpreamble	Tc!uʟe′ ʟe ka′odzîte. //
	\glpreamble	Chʼu tle ḵoowdzitee. //
	\gla	Chʼu tle \rlap{ḵoowdzitee.} @ {} @ {} @ {} @ {} @ {} @ {} //
	\glb	chʼu tle ḵu- wu- d- s- i- \rt[¹]{tiʰ} -μμL //
	\glc	just then \xx{areal}- \xx{pfv}- \xx{mid}- \xx{xtn}- \xx{stv}-
			\rt[¹]{be} -\xx{var} //
	\gld	just then \rlap{\xx{g̱cnj}.\xx{pfv}.born} {} {} {} {} {} {} //
	\glft	‘Then it was born.’
		//
\endgl
\xe

The verb in (\lastx) and (\nextx) is somewhat puzzling. \citeauthor{swanton:1909}’s transcription of \orth{ka′odzîte} strongly suggests something like \fm{kawdzitee}, but his gloss “(it) came to be born” instead strongly suggests \fm{ḵoowdzitee}.
The verb \fm{kawdzitee} is very rare, found only in \fm{Kéet Yanaayí} Willie Marks’s telling of \fm{Naatsilanéi} as the proverb \fm{óodáx̱ kát kawdziteeyi yáx̱ woonei} “he was like the man who had a spear removed” \parencite[114.95]{dauenhauer:1987}.
This is based on the root \fm{\rt[²]{ti}} ‘handle’ with passive \fm{d-} and the qualifier \fm{k-}.
Nothing like this makes sense in the context of (\lastx) and (\nextx) so \citeauthor{swanton:1909}’s gloss is favoured over his transcription.
The reasoning behind \citeauthor{swanton:1909}’s transcription is still unclear.

\ex\label{ex:91-142-a-man-was-born}%
\exmn{273.7}%
\begingl
	\glpreamble	Qā ayu′ ka′odzîte. //
	\glpreamble	Ḵáa áyú ḵoowdzitee. //
	\gla	{} Ḵáa {} \rlap{áyú} @ {}
		\rlap{ḵoowdzitee.} @ {} @ {} @ {} @ {} @ {} @ {} //
	\glb	{} ḵáa {} á -yú
		ḵu- wu- d- s- i- \rt[¹]{tiʰ} -μμL //
	\glc	{}[\pr{DP} man {}] \xx{foc} -\xx{dist}
		\xx{areal}- \xx{pfv}- \xx{mid}- \xx{xtn}- \xx{stv}- \rt[¹]{be} -\xx{var} //
	\gld	{} man {} \rlap{it.is} {}
		\rlap{\xx{g̱cnj}.\xx{pfv}.born} {} {} {} {} {} {} //
	\glft	‘It was a man that was born.’
		//
\endgl
\xe

\ex\label{ex:91-143-gradually-big}%
\exmn{273.8}%
\begingl
	\glpreamble	Desgwᴀ′tc ỵānᴀłg̣ē′n. //
	\glpreamble	Deisgwách ÿaa nalgéin. //
	\gla	Deisgwách ÿaa @ \rlap{nalgéin.} @ {} @ {} @ {} @ {} //
	\glb	deisgwách ÿaa= n- l- \rt[¹]{ge} -μμH -n //
	\glc	gradually along= \xx{ncnj}- \xx{xtn}- \rt[¹]{big} -\xx{var} -\xx{nsfx} //
	\gld	gradually along \rlap{\xx{ncnj}.\xx{prog}.big} {} {} {} {} //
	\glft	‘Gradually he gets big.’
		//
\endgl
\xe

\FIXME{Xref to (\ref{ex:91-4-gradually-is-big}) and to ch.\ \ref{ch:89-origin-of-copper} (\ref{ex:89-112-gradually-get-big}) for discussion of \fm{deisgwách} and its interpretation as ‘gradually’ or ‘eventually’.}

\ex\label{ex:91-144-says-to-wife-heron}%
\exmn{273.8}%
\begingl
	\glpreamble	Ye ada′ỵaqa ducᴀ′t yułᴀ′q!tc, //
	\glpreamble	Yéi adaaÿaḵá du shát yú láx̱ʼch //
	\gla	Yéi @ \rlap{adaaÿaḵá} @ {} @ {} @ {} @ {} 
		{} du shát {}
		{} yú \rlap{láx̱ʼch} @ {} {} //
	\glb	yéi= a- daa- ÿ- \rt[²]{ḵa} -μH
		{} du shát {}
		{} yú láx̱ʼ -ch {}  //
	\glc	thus= \xx{arg}- around- \xx{qual}- \rt[²]{say} -\xx{var}
		{}[\pr{DP} \xx{3h·pss} wife {}]
		{}[\pr{DP} \xx{dist} heron -\xx{erg} {}] //
	\gld	thus \rlap{3>3.\xx{ncnj}.\xx{impfv}.say·to} {} {} {} {}
		{} his wife {}
		{} that heron {} {} //
	\glft	‘So he says to his wife, that heron’
		//
\endgl
\xe

\ex\label{ex:91-145-what-happen-relatives}%
\exmn{273.9}%
\begingl
	\glpreamble	“Wâ′sa wū′nî îxō′nq!î.” //
	\glpreamble	«\!Wáa sá woonee i x̱oonxʼí?\!» //
	\gla	{} \llap{«\!}Wáa sá {} \rlap{woonee} @ {} @ {} @ {}
		{} i \rlap{x̱oonxʼí?\!»} @ {} @ {} {} //
	\glb	{} wáa sá {} wu- i- \rt[¹]{niʰ} -μμL
		{} i x̱oon -xʼ -í {} //
	\glc	{}[\pr{QP} how \xx{q} {}] \xx{pfv}- \xx{stv}- \rt[¹]{happen} -\xx{var}
		{}[\pr{DP} \xx{2sg·pss} relative -\xx{pl} -\xx{pss} {}] //
	\gld	{} how ? {} \rlap{\xx{ncnj}.\xx{pfv}.happen} {} {} {}
		{} your \rlap{relatives} {} {} {} //
	\glft	‘“What happened to your relatives?”’
		//
\endgl
\xe

\ex\label{ex:91-146-when-go-for-wood-not-go-beach}%
\exmn{273.9}%
\begingl
	\glpreamble	“ʟe g̣ᴀ′ng̣a awugudî′ ʟēł yēq ugu′ttc.” //
	\glpreamble	«\!Tle gáng̱aa awugoodí tléil yeiḵ ugootch.\!» //
	\gla	{} \llap{«\!}Tle {} \rlap{gáng̱aa} @ {} {}
			\rlap{awugoodí} @ {} @ {} @ {} @ {} {}
		tléil yeiḵ @ \rlap{ugútch.\!»} @ {} @ {} @ {} @ {} //
	\glb	{} tle {} gán -g̱áa {}
			a- wu- \rt[¹]{gut} -μμL -í {}
		tléil ÿeiḵ= u- u- \rt[¹]{gut} -μμL -ch //
	\glc	{}[\pr{CP} then {}[\pr{PP} firewood -\xx{ades} {}]
			\xx{4h·s}- \xx{pfv}- \rt[¹]{go·\xx{sg}} -\xx{var} -\xx{sub} {}]
		\xx{neg} beach= \xx{irr}- \xx{zpfv}- \rt[¹]{go·\xx{sg}} -\xx{var} -\xx{rep} //
	\gld	{} then {} firewood -for {}
			\rlap{one.\xx{pfv}.go·\xx{sg}} {} {} {} -when {}
		not beach \rlap{\xx{zcnj}.\xx{hab}.go·\xx{sg}} {} {} {} {} //
	\glft	‘“When someone went for firewood they did not come down to the beach.”’
		//
\endgl
\xe

\section{Paragraph 13}\label{sec:91-para-13}

\ex\label{ex:91-147-gotten-big-sit-on-flats}%
\exmn{273.11}%
\begingl
	\glpreamble	T!cuʟe′ ye kawu′łgeỵî awe′ tacukā′dî aksanu′ktc. //
	\glpreamble	Chʼu tle yéi kawulgéiÿi áwé taashukáade aksanúkch. //
	\gla	{} Chʼu tle yéi @ \rlap{kawulgéiÿi} @ {} @ {} @ {} @ {} @ {} {}
		\rlap{áwé} @ {}
		{} \rlap{taashukáade} @ {} @ {} @ {} {}
		\rlap{aksanúkch.} @ {} @ {} @ {} @ {} @ {} //
	\glb	{} chʼu tle yéi= k- wu- l- \rt[¹]{ge} -μμH -í {}
		á -wé
		{} táaᵏ- shú- ká -dé {}
		a- k- s- \rt[¹]{nuk} -μH -ch //
	\glc	{}[\pr{CP} just then
			thus= \xx{cmpv}- \xx{pfv}- \xx{xtn}- \rt[¹]{big} -\xx{var} -\xx{sub} {}]
		\xx{foc} -\xx{mdst}
		{}[\pr{PP} bottom- end- \xx{hsfc} -\xx{all} {}]
		\xx{arg}- \xx{qual}- \xx{csv}- \rt[¹]{sit·\xx{sg}} -\xx{var} //
	\gld	{} just then thus \rlap{\xx{cmpv}.\xx{pfv}.big} {} {} {} {} {} {}
		\rlap{it.is} {}
		{} \rlap{river.flats} {} {} -to {}
		\rlap{3>3.\xx{hab}.make.sit·\xx{sg}} {} {} {} {} {} //
	\glft	‘Then it was when he had gotten big that he would always sit him on the flats.’
		//
\endgl
\xe

\FIXME{\fm{taashuká} (T.\ \fm{tashkwa} [\ipa{tʰaʃ.ˈkʰʷa}]) “river flats” \parencite[f06/7]{leer:1973} clearly has \fm{–shú} ‘end’ and \fm{–ká} ‘horizontal surface’.
Similar \fm{taashuyee} “river flats” reported from Atlin and Carcross \parencite[\textsc{t}·89]{leer:2001}.
The \fm{taa-} portion is probably \fm{–táaᵏ} ‘interior bottom’ as in \fm{héen táak} ‘bottom of body of water’, but the compound is probably frozen since there is no \fm{héen} in it.}

\ex\label{ex:91-148-in-ice-always-bathe}%
\exmn{273.11}%
\begingl
	\glpreamble	T!īq! tū′q!awe ᴀ′cutcnuttc. //
	\glpreamble	Tʼéexʼ tóoxʼ áwé ashóoch nuch. //
	\gla	{} Tʼéexʼ \rlap{tóoxʼ} @ {} {} \rlap{áwé} @ {}
		\rlap{ashóoch} @ {} @ {} @ \•nuch. //
	\glb	{} tʼéexʼ tú -xʼ {} á -wé
		a- \rt[²]{shuch} -μμH =nuch //
	\glc	{}[\pr{PP} ice inside -\xx{loc} {}] \xx{foc} -\xx{mdst}
		\xx{arg}- \rt[²]{bathe} -\xx{var} =\xx{hab·aux} //
	\gld	{} ice \rlap{within} {} {} \rlap{it.is} {}
		\rlap{3>3.\xx{zcnj}.\xx{impfv}.bathe} {} {} \•always //
	\glft	‘It was in the ice that he would always bathe him.’
		//
\endgl
\xe

\ex\label{ex:91-149-eventually-shooting-things}%
\exmn{273.12}%
\begingl
	\glpreamble	Desgwᴀ′tc ᴀt t!ukt yuᴀtk!ᴀ′tsk!ᵒ. //
	\glpreamble	Deisgwách at tʼúkt yú atkʼátskʼu. //
	\gla	Deisgwách at @ \rlap{tʼúkt} @ {} @ {}
		{} yú \rlap{atkʼátskʼu.} @ {} @ {} @ {} @ {} {} //
	\glb	deisgwách at= \rt[²]{tʼuk} -μH -t
		{} yú at= kʼí- ÿáts -kʼʷ -í {} //
	\glc	eventually \xx{4n·o}= \rt[²]{shoot·bow} -\xx{var} -\xx{ict}
		{}[\pr{DP} \xx{dist} \xx{4n·pss}= base- child -\xx{dim} -\xx{pss} {}] //
	\gld	eventually things\• \rlap{\xx{impfv}.shoot·bow.\xx{rep}} {} {}
		{} that \rlap{boy\ix{i}} {} {} {} {} {} //
	\glft	‘Eventually he is shooting things, that boy.’
		//
\endgl
\xe

\ex\label{ex:91-150-took-around-bow-and-arrow}%
\exmn{273.12}%
\begingl
	\glpreamble	ᴀttcū′net ᴀna′łᴀttc //
	\glpreamble	Atchooneit anal.átch. //
	\gla	{} {} \rlap{Atchooneit} @ {} @ {} @ {} {} {} {}
		\rlap{anal.átch.} @ {} @ {} @ {} @ {} @ {} //
	\glb	{} {} at= \rt[¹]{chun} -μμL -i {} át {}
		a- n- l- \rt[¹]{.at} -μH -ch //
	\glc	{}[\pr{DP} {}[\pr{CP} \xx{4n·o}= \rt[¹]{wound} -\xx{var} -\xx{rel} {}] thing {}]
		\xx{arg}- \xx{ncnj}- \xx{csv}- \rt[¹]{go·\xx{pl}} -\xx{var} -\xx{rep} //
	\gld	{} {} \rlap{bow·and·arrow} {} {} {} {} {} {}
		\rlap{3>3.\xx{hab}.handle·\xx{pl}} {} {} {} {} {} //
	\glft	‘He always took around a bow and arrows.’
		//
\endgl
\xe

\FIXME{Discuss noun \fm{atchooneit}. Note discussion of \fm{chooneit} at (\ref{ex:91-3-go-around-bownarrow}).
Appearance of \fm{at=} confirms its verbal 
But it suggests that \fm{\rt[¹]{chun}} should be bivalent rather than monovalent.}

\ex\label{ex:91-151-off-for-something-to-kill}%
\exmn{273.13}%
\begingl
	\glpreamble	Wug̣ādjā′g̣e ᴀ′tg̣a qot wugū′t hᴀsduī′c yuᴀtk!ᴀ′tsk!ᵒ. //
	\glpreamble	Oog̱aajaag̱i átg̱aa ḵut woogoot, hasdu éesh yú atkʼátskʼu: //
	\gla	{} {} \rlap{Oog̱aajaag̱i} @ {} @ {} @ {} @ {} @ {} @ {} @ {} {}
			\rlap{átg̱aa} @ {} {}
		ḵut @ \rlap{woogoot,} @ {} @ {} @ {} +
		{} \rlap{hasdu} @ {} éesh {}
		{} yú \rlap{atkʼátskʼu:} @ {} @ {} @ {} @ {} {} //
	\glb	{} {} a- u- {} g̱- i- \rt[²]{jaḵ} -μμL -i {} át -g̱áa {}
		ḵut= wu- i- \rt[¹]{gut} -μμL
		{} has= du éesh {}
		{} yú at= kʼí- ÿáts -kʼʷ -í {} //
	\glc	{}[\pr{PP} {}[\pr{CP} \xx{arg}- \xx{irr}- \xx{zcnj}\· \xx{mod}- \xx{stv}-
				\rt[²]{kill} -\xx{var} -\xx{rel} {}]
			thing -\xx{ades} {}]
		away= \xx{pfv}- \xx{stv}- \rt[¹]{go·\xx{sg}} -\xx{var}
		{}[\pr{DP} \xx{plh}= \xx{3h·pss} father {}]
		{}[\pr{DP} \xx{dist} \xx{4n·pss}= base- child -\xx{dim} -\xx{pss} {}] //
	\gld	{} {} \rlap{3>3.\xx{zcnj}.\xx{pot}.kill} {} {} {} {} {} {} -that {}
			thing -for {}
		away \rlap{\xx{pfv}.go·\xx{sg}} {} {} {}
		{} \rlap{their} {} father {}
		{} that \rlap{boy} {} {} {} {} {} //
	\glft	‘He went off for something that he could kill, (and) their father (about) that boy:’
		//
\endgl
\xe

\FIXME{This verblessness in (\lastx) is somewhat odd, but there’s no sign of a missing verb nor does the first part of the sentence look like an adjunct clause that could imply an elided main verb.}

\FIXME{The context provided by (\ref{ex:91-149-eventually-shooting-things}) and (\ref{ex:91-150-took-around-bow-and-arrow}) leads us to infer that the subject of \fm{oog̱aajaag̱i} and \fm{woogoot} in (\lastx) is the boy, not the father.}

\FIXME{The use of \fm{hasdu} rather than just \fm{du} in (\lastx) is curious.
There are only four possible characters in this context: the woman, her daughter, the heron who married the daughter, and the boy.
The \fm{hasdu éesh} ‘their father’ implies that the heron is father to more than one person, but the only person for whom the heron could be a true father is the boy.
We have to take the meaning of \fm{hasdu} loosely to include the two women, perhaps as something like ‘father’s clan’.
It is also possible that the speaker made a mistake and should have said just \fm{du}.}

\ex\label{ex:91-152-already-me}%
\exmn{273.14}%
\begingl
	\glpreamble	“Detc!a′ xᴀ′tawe ᴀxỵī′tk!.” //
	\glpreamble	«\!De chʼa x̱át áwé ax̱ ÿéetkʼ.\!» //
	\gla	«\!De chʼa {} x̱át {} \rlap{áwé} @ {}
		{} ax̱ \rlap{ÿéetkʼ.\!»} @ {} {} //
	\glb	\pqp{}de chʼa {} x̱át {} á -wé
		{} ax̱ ÿéet -kʼ {} //
	\glc	\pqp{}already just {}[\pr{DP} \xx{1sg} {}] \xx{foc} -\xx{mdst}
		{}[\pr{DP} \xx{1sg·pss} son -\xx{dim} {}] //
	\gld	\pqp{}already just {} me {} \rlap{he.is} {}
		{} my son -little {} //
	\glft	‘“He is already me, my little son.”’
		//
\endgl
\xe

\ex\label{ex:91-153-he-said-to-his-wife}%
\exmn{273.14}%
\begingl
	\glpreamble	Yū′aỵaosîqa ducᴀ′t //
	\glpreamble	Yóo aÿawsiḵaa du shát //
	\gla	Yóo @ \rlap{aÿawsiḵaa} @ {} @ {} @ {} @ {} @ {} @ {}
		{} du shát {} //
	\glb	yóo= a- ÿ- wu- s- i- \rt[¹]{ḵa} -μμL
		{} du shát {} //
	\glc	\xx{quot}= \xx{arg}- \xx{qual}- \xx{pfv}- \xx{csv}- \xx{stv}-
			\rt[¹]{say} -\xx{var}
		{}[\pr{DP} \xx{3h·pss} wife {}] //
	\gld	thus \rlap{3>3.\xx{ncnj}.\xx{pfv}.say·to} {} {} {} {} {} {}
		{} his wife {} //
	\glft	‘He said to his wife’
		//
\endgl
\xe

\ex\label{ex:91-154-now-leaving}%
\exmn{273.14}%
\begingl
	\glpreamble	 “Deỵînᴀ′q koqāgu′t.”//
	\glpreamble	 «\!De ÿináḵ kuḵagóot.\!» //
	\gla	«\!De {} \rlap{ÿináḵ} @ {} {}
		\rlap{kuḵagóot.} @ {} @ {} @ {} @ {} @ {} //
	\glb	\pqp{}de {} ÿee -náḵ {}
		w- g- g̱- x̱- \rt[¹]{gut} -μμH //
	\glc	\pqp{}now {}[\pr{PP} \xx{2pl} -\xx{abes} {}]
		\xx{irr}- \xx{gcnj}- \xx{mod}- \xx{1sg·s}- \rt[¹]{go·\xx{sg}} -\xx{var} //
	\gld	\pqp{}now {} you·\xx{pl} -away·from {}
		\rlap{\xx{ncnj}.\xx{prsp}.I.go·\xx{sg}} {} {} {} {} {} //
	\glft	‘“Now I am leaving you.”’
		//
\endgl
\xe

\ex\label{ex:91-155-running-down-into-water}%
\exmn{274.1}%
\begingl
	\glpreamble	Dêsgwᴀ′tc hīnx ye îcx̣îx̣tc yuᴀtk!ᴀ′tsk!ᵒ. //
	\glpreamble	Deisgwách héenx̱ yei ishxíxch yú atkʼátskʼu. //
	\gla	Deisgwách {} \rlap{héenx̱} @ {} {}
		yei @ \rlap{ishxíxch} @ {} @ {} @ {} @ {}
		{} yú \rlap{atkʼátskʼu:} @ {} @ {} @ {} @ {} @ {} //
	\glb	deisgwách {} héen -x̱ {}
		yei= d- sh- \rt[¹]{xix} -μH -ch
		{} yú at= kʼí- ÿáts -kʼʷ -í {} //
	\glc	eventually {}[\pr{PP} water -\xx{pert} {}]
		down= \xx{mid}- \xx{pej}- \rt[¹]{fall·\xx{sg}} -\xx{var} -\xx{rep}
		{}[\pr{DP} \xx{dist} \xx{4n·pss}= base- child -\xx{dim} -\xx{pss} {}] //
	\gld	eventually {} water -into {}
		down \rlap{\xx{g̱cnj}.\xx{impfv}.run·\xx{sg}.\xx{rep}} {} {} {} {}
		{} that \rlap{boy} {} {} {} {} {} //
	\glft	‘Eventually he is running down into the water, that boy.’
		//
\endgl
\xe

\ex\label{ex:91-156-reach-shore-whenever-kill}%
\exmn{274.1}%
\begingl
	\glpreamble	Tc!aye′ dāq g̣acī′tc ᴀc g̣adjᴀ′qên. //
	\glpreamble	Chʼa yéi daaḵ g̱asheech ash g̱ajág̱ín. //
	\gla	Chʼa yéi daaḵ \rlap{g̱asheech} @ {} @ {} @ {}
		{} ash @ \rlap{g̱ajág̱ín.} @ {} @ {} @ {} @ {} {} //
	\glb	chʼa yéi dáaḵ= g̱- \rt[¹]{shi} -μμL -ch
		{} ash= {} g̱- \rt[²]{jaḵ} -μH -ín {} //
	\glc	just thus inland= \xx{g̱cnj}- \rt[¹]{reach} -\xx{var} -\xx{rep}
		{}[\pr{CP} \xx{3prx·o}= \xx{zcnj}\· \xx{mod}-
			\rt[²]{kill} -\xx{var} -\xx{ctng} {}] //
	\gld	just thus inland \rlap{\xx{g̱cnj}.\xx{hab}.reach} {} {} {}
		{} him \rlap{\xx{zcnj}.\xx{ctng}.kill} {} {} {} {} {} //
	\glft	‘He always just reaches for shore whenever it (tries to) kill him.’
		//
\endgl
\xe

\FIXME{The root \fm{\rt[¹]{shi}} ‘touch, reach for, search, help’ acts either like it is lexically specified for the \fm{n}-conjugation class or like it is a motion root and so requires motion derivations to supply a conjugation class.
The difficulty of (\lastx) is that it has \fm{dáaḵ=} suggesting a \fm{∅}-conjugation class motion derivation but the form of the verb seems to be a \fm{g̱}-conjugation class habitual with the \fm{g̱-} prefix that \citeauthor{swanton:1909} clearly transcribes as \orth{g̣}.
There don’t seem to be any \fm{\rt[¹]{shi}} with lexically specified \fm{g̱}-conjugation and none of the \fm{g̱}-conjugation motion derivations applies here.}

\section{Paragraph 14}\label{sec:91-para-14}

\ex\label{ex:91-157-went-with-bow-and-arrows}%
\exmn{274.3}%
\begingl
	\glpreamble	Wananī′sawe tcū′net tîn wugū′t. //
	\glpreamble	Wáa nanée sáwé chooneit tin woogoot. //
	\gla	{} Wáa \rlap{nanée} @ {} @ {} @ {} {}
		\rlap{sáwé} @ {} @ {}
		{} {} {} \rlap{chooneit} @ {} @ {} {} {} {} tin {}
		\rlap{woogoot.} @ {} @ {} @ {} //
	\glb	{} wáa n- \rt[¹]{ni} -μμH {} {}
		s= á -wé
		{} {} {} \rt[¹]{chun} -μμL -i {} át {} tin {}
		wu- i- \rt[¹]{gut} -μμL //
	\glc	{}[\pr{CP} how \xx{ncnj}- \rt[¹]{happen} -\xx{var} \·\xx{sub} {}]
		\xx{q}= \xx{foc} -\xx{mdst}
		{}[\pr{PP} {}[\pr{NP} {}[\pr{CP} \rt[¹]{wound} -\xx{var} -\xx{rel} {}]
			thing {}] \xx{instr} {}]
		\xx{pfv}- \xx{1sg·s}- \rt[¹]{go·\xx{sg}} -\xx{var} //
	\gld	{} how \rlap{\xx{csec}.happen} {} {} \·while {}
		ever\• \rlap{it.is} {}
		{} {} {} \rlap{bow·and·arrow} {} {} {} {} {} with {}
		\rlap{\xx{ncnj}.\xx{pfv}.go·\xx{sg}} {} {} {} //
	\glft	‘At some point he went with a bow and arrows.’
		//
\endgl
\xe

\ex\label{ex:91-158-on-beach-in-puddle-swim-hintaayeeshi}%
\exmn{274.3}%
\begingl
	\glpreamble	Eq dugūde′awe āk!ᵘ kᴀt wuq!ā′gî hīntāỵī′cî. //
	\glpreamble	Eiḵt wugoodí áwé áakʼw kát wuxʼaagí héen taaÿéeshi. //
	\gla	{} {} \rlap{Eiḵt} @ {} {} \rlap{wugoodí} @ {} @ {} @ {} {}
		\rlap{áwé} @ {}
		{} \rlap{áakʼw} @ {} \rlap{kát} @ {} {}
		\rlap{wuxʼaagí} @ {} @ {} @ {} 
		{} héen \rlap{taaÿéeshi.} @ {} @ {} @ {} {} //
	\glb	{} {} eeͥḵ -t {} wu- \rt[¹]{gut} -μμL -í {}
		á -wé
		{} áaʷ -kʼ ká -t {}
		wu- \rt[¹]{xʼak} -μμL -í
		{} héen táaᵏ= \rt[²]{ÿish} -μμH -i {} //
	\glc	{}[\pr{CP} {}[\pr{PP} beach -\xx{pnct} {}]
			\xx{pfv}- \rt[¹]{go·\xx{sg}} -\xx{var} -\xx{sub} {}]
		\xx{foc} -\xx{mdst}
		{}[\pr{PP} lake -\xx{dim} \xx{hsfc} -\xx{pnct} {}]
		\xx{pfv}- \rt[¹]{fish·swim} -\xx{var} -\xx{sub}
		{}[\pr{DP} water bottom= \rt[²]{pull} -\xx{var} -\xx{pss} {}] //
	\gld	{} {} beach -around {} \rlap{\xx{ncnj}.\xx{pfv}.go·\xx{sg}} {} {} -while {}
		\rlap{it.is} {} 
		{} \rlap{pond} {} top -around {}
		\rlap{\xx{ncnj}.\xx{pfv}.fish·swim} {} {} -while
		{} water bottom \rlap{puller} {} -of {} //
	\glft	‘It was while he was going around on the beach that it was swimming around in a pond, a héen taÿéeshi.’
		//
\endgl
\xe

The noun \fm{héen taaÿéeshi} in (\lastx) is very rare and is difficult to interpret.
\citeauthor{swanton:1909} gives a definition of it at the beginning of story number 60 “The Hīn-Taỵī′cî” from \fm{Léekʼ} the mother of Ḵaadashaan: “There is a fish, called hīn-taỵī′cî, which is shaped like a halibut but has very many ‘legs’.” \parencite[217]{swanton:1909}.
\citeauthor{olson:1967} has “hĭntagi·eci′h” from “JW” which he describes as “an animal with a sharp shell?” \parencite[39 col.\ 2]{olson:1967}, echoing the sharp edge mentioned later in (\ref{ex:91-160-hands-broke-apart-from-edge}).
Emmons has “tar yish” glossed as “starfish” in a list of decorated gambling sticks  \parencite[459 \#60]{emmons:1991}.
This is obviously not \fm{sʼáx} ‘starfish’, but it is likely something like \fm{taayéesh} or \fm{taaÿéesh}.

\citeauthor{leer:1973} interprets \citeauthor{swanton:1909}’s transcription as \fm{hintaÿéeshi} “fish shaped like halibut but with many ‘legs’” and adds a subscript note “type of flounder (J)” \parencite[03/218]{leer:1973}.
The attribution “(J)” could plausibly refer to \fm{Kʼóox} Johnny Marks, but \citeauthor{leer:1973} usually cites him as “JM” so the meaning of “(J)” is uncertain.
\citeauthor{leer:1978b} later gives a form \fm{hintu-ÿiʼshi} “flounder sp.”\ \parencite[13]{leer:1978b} with \fm{–tú} ‘inside’ rather than \fm{–táaᵏ} ‘inner bottom’ but this has no source and likely a misreading of \fm{hintaÿéeshi} with \fm{taa} reduced to \fm{ta}.

The analysis of \fm{héen taaÿéeshi} in (\lastx) assumes that it is a possessive compound of the phrase \fm{héen táak} ‘bottom of water’ and the noun \fm{ÿéesh}.
\citeauthor{olson:1967}’s transcription “hĭntagi·eci′h” can be plausibly read as [\ipa{hìn.tàkⁱ.ɰéː.ʃì}] and so probably represents \fm{hintakÿéeshi} with the final \fm{k} of \fm{táak} preserved.
\citeauthor{leer:1973} identifies the \fm{ÿéesh} as ‘leech, bloodsucker’ \parencite[03/218]{leer:1973}; the Tongass form \fm{ÿeésh} [\ipa{ɰiˀʃ}] suggests a root \fm{\rt{ÿiʼsh}} which is otherwise unknown.
The alternative identification followed in (\lastx) is that it has the root \fm{\rt[²]{ÿish}} ‘pull’.%
\footnote{The noun \fm{ÿéesh} (T.\ \fm{ÿeésh} [\ipa{ɰiˀʃ}]) ‘leech, bloodsucker’ is possibly derivable from \fm{\rt[²]{ÿish}} ‘pull’ but the glottalized vowel in the Tongass form would need an explanation for its origin since the root is not \fm[*]{\rt{ÿiʼsh}}.}
Taken literally the compound \fm{héen táak ÿéeshi} should then mean something like ‘puller of the water bottom’ or more succinctly ‘underwater puller’.

\citeauthor{swanton:1909}’s description of something shaped like a halibut – round and flat – but with many legs, combined with Emmons’s identification as a ‘starfish’, suggests that \fm{héen taaÿéeshi} refers to something like the sunflower sea star (\species{Pycnopodia}{helianthoides}[Brandt 1835]).
The sharp edges described in (\ref{ex:91-160-hands-broke-apart-from-edge}) and (\ref{ex:91-185-so-sharp}) are then plausibly a reference to the calcareous ossicles common among sea stars. The suckers present on many sea stars could be the referent of \fm{\rt[²]{ÿish}} ‘pull’.

Another plausible interpretation is that \fm{héen taaÿéeshi} refers to something like the Pacific Dover sole (\species{Microstomus}{pacificus}[Lockington 1879]).
The fins of the sole have many spiny rays which could be the many ‘legs’ of \citeauthor{swanton:1909}’s description.
Sole, halibut, and flounder sit on the bottom (\fm{héen táak} ‘water bottom’) and when they fight a hook they generally swim straight down (\fm{\rt[²]{ÿish}} ‘pull’) rather than horizontally like most other fish (\fm{Shaag̱aw Éesh} D.\ Anderstrom, p.c.\ 2020).
Since the \fm{héen taaÿéeshi} or \fm{hintakÿéeshi} cannot yet be conclusively identified it has been left untranslated in the English translation.

\ex\label{ex:91-159-grabbed-from-there}%
\exmn{274.4}%
\begingl
	\glpreamble	Āx awacā′t. //
	\glpreamble	Aax̱ aawasháat. //
	\gla	{} \rlap{Aax̱} @ {} {} \rlap{aawasháat.} @ {} @ {} @ {} @ {} //
	\glb	{} á -dáx̱ {} a- wu- i- \rt[²]{shaʼt} -μμH //
	\glc	{}[\pr{PP} \xx{3n} -\xx{abl} {}]
		\xx{arg}- \xx{pfv}- \xx{stv}- \rt[²]{grab} -\xx{var} //
	\gld	{} there -from {} \rlap{3>3.\xx{gcnj}.\xx{pfv}.grab} //
	\glft	‘He picked it up from there.’
		//
\endgl
\xe

\ex\label{ex:91-160-hands-broke-apart-from-edge}%
\exmn{274.4}%
\begingl
	\glpreamble	Yadudjînawe ʟe wū′ctᴀx uwak!u′ts awᴀ′nîte. //
	\glpreamble	Yá du jín áwé tle wóoshdáx̱ aawakʼúts a wánich. //
	\gla	{} Yá du jín {} \rlap{áwé} @ {}
		{} \rlap{wóoshdáx̱} @ {} {}
		\rlap{aawakʼúts} @ {} @ {} @ {} @ {} +
		{} a \rlap{wánich.} @ {} {} //
	\glb	{} yá du jín {} á -wé
		{} wóosh -dáx̱ {}
		a- wu- i- \rt[²]{kʼuts} -μH
		{} a wán -ch {} //
	\glc	{}[\pr{DP} \xx{prox} \xx{3h·pss} hand {}] \xx{foc} -\xx{mdst}
		{}[\pr{PP} \xx{recip} -\xx{abl} {}]
		\xx{arg}- \xx{pfv}- \xx{stv}- \rt[²]{break} -\xx{var}
		{}[\pr{DP} \xx{3n·pss} edge -\xx{erg} {}] //
	\gld	{} this his hand {} \rlap{it.is} {}
		{} ea·oth -from {}
		\rlap{3>3.\xx{zcnj}.\xx{pfv}.break} {} {} {} {}
		{} its edge -by {} //
	\glft	‘It was these hands of his that were broken apart by its edge.’
		//
\endgl
\xe

The decipherment of \citeauthor{swanton:1909}’s transcription of the sentence in (\lastx) is complicated by a couple of grammatical problems, namely the presence or absence of \fm{a-} in the verb \fm{aawakʼúts} and the presence or absence of \fm{-í} in the phrase \fm{a wánich} ‘its edge’.
The analysis given in (\lastx) assumes a verb \fm{aawakʼúts} with \fm{a-} but \citeauthor{swanton:1909}’s transcription \orth{uwak!u′ts} suggests a form \fm{uwakʼúts} instead.
The root \fm{\rt{kʼuts}} ‘break (long, flexible)’ is documented with two valencies: a monovalent root \fm{\rt[¹]{kʼuts}} giving intransitive verbs like \fm{wookʼoots} (\fm{n}; achievement) ‘it (long, flexible object) broke’ and a bivalent root \fm{\rt[²]{kʼuts}} giving transitive verbs like \fm{aadáx̱ akʼoots} (\fm{∅}/\fm{n}; \fm{-μμL} activity) ‘s/he breaks it (long, flexible object) off of it’ \parencites[f04/136–137]{leer:1973}[780]{leer:1976}.
If \fm{uwakʼúts} is correct then the verb would be intransitive with the sole argument being \fm{yá du jín} ‘these his hands’.
But this does not fit with the phrase \fm{a wánich} ‘by its edge’ which has the ergative \fm{-ch} marking a subject since an intransitive verb cannot have both an object and a subject.
The one context where \fm{uwakʼúts} would be plausible is if the phrase with ergative \fm{-ch} immediately preceded the verb, in which case the \fm{-ch} regularly triggers the disappearance of the \fm{a-} argument marking prefix.
But the phrase \fm{a wánich} does not immediately precede the verb; rather, it follows the verb and is thus not in a position to trigger the disappearance of \fm{a-}.

It is possible that the speaker did actually utter the intransitive \fm{uwakʼúts} in (\lastx) but changed his mind and added the phrase with ergative \fm{-ch} as an afterthought.
This is occasionally found elsewhere, but native speakers generally consider it to be a dispreferred error.
An additional possibility is that \citeauthor{swanton:1909} actually did transcribe the predicted \fm{aawakʼúts} as \orth{awak!u′ts}, and that the initial \orth{a} was misread as \orth{u} during manuscript preparation or typesetting.
This possibility is weakly supported by the mistaken \orth{…te} instead of \orth{…tc} in the phrase \fm{a wánich}.

Another alternative analysis takes the transcription \orth{awᴀ′nîte} in (\lastx) to mean \fm{a wánidé} with the allative \fm{-dé} rather than the ergative \fm{-ch}.
Compare the transcription \orth{yā′t!aỵîte} for \fm{yáatʼaa ÿíde} in (\ref{ex:91-124-ppl-go-not-seen}) where \citeauthor{swanton:1909} similarly gives \orth{te} for a final allative \fm{-dé}.
This would make the intransitive \fm{uwakʼúts} more plausible, but it would require an interpretation like “it was these hands of his that broke apart toward its edge”.
This runs counter to \citeauthor{swanton:1909}’s gloss “from the sharp sides of it” which implies that the phrase denotes an instrument and not a destination.

The phrase \fm{a wánich} in (\lastx) presents its own problem.
\citeauthor{swanton:1909}’s transcription \orth{awᴀ′nîte} suggests a form \fm{a wánich} [\ipa{ʔà ˈwá.nìtʃ}] with a vowel \fm{i} interceding between the noun \fm{–wán} ‘edge’ and the suffix \fm{-ch}.
The noun \fm{–wán} is an inalienable part noun or spatial relation noun and so will not typically occur with the possessive suffix \fm{-í}.
When such nouns occur with \fm{-í} this indicates that they have been alienated, i.e.\ separated somehow from their natural possessor.
The context here does not support an interpreation where the edge of the \fm{héen taaÿéeshi} is separated from the body of the creature, so the possessive suffix does not make sense.
The vowel \fm{i} here is instead analyzed as a semantically meaningless epenthetic vowel inserted to avoid to a complex coda.
The problem with this analysis is that the preceding consonant \fm{n} is a sonorant and Tlingit phonology does not usually prohibit complex codas of a sonorant followed by an affricate like \fm{ch}.
There are some more or less irregular cases like \fm{héen} ‘fresh water, river’ + \fm{-kʼ} ‘diminutive’ → \fm{héenákʼw} [\ipa{ˈhí:.nákʼʷ}] rather than  \fm[*]{héenkʼ} [\ipa{híːnkʼ}], but these are generally considered to be lexicalized and they involve an epenthetic vowel \fm{a} [\ipa{a}] rather than \fm{i} [\ipa{i}].
Further research on the phonological resolution of morphologically derived complex codas should clarify this issue.

\ex\label{ex:91-161-in-pond-raising-it}%
\exmn{274.5}%
\begingl
	\glpreamble	Āk!ᵘ kᴀ′q!awe ỵā′ᴀnᴀswᴀt. //
	\glpreamble	Áakʼw káxʼ áwé ÿaa anaswát. //
	\gla	{} \rlap{Áakʼw} @ {} \rlap{káxʼ} @ {} {} \rlap{áwé} @ {}
		ÿaa @ \rlap{anaswát.} @ {} @ {} @ {} @ {} //
	\glb	{} áaʷ -kʼ ká -xʼ {} á -wé
		ÿaa= a- n- s- \rt[¹]{waʼt} -μH //
	\glc	{}[\pr{PP} lake -\xx{dim} \xx{hsfc} -\xx{loc} {}] \xx{foc} -\xx{mdst}
		along= \xx{arg}- \xx{ncnj}- \xx{csv}- \rt[¹]{adult} -\xx{var} //
	\gld	{} lake -little atop -at {} \rlap{it.is} {}
		along \rlap{3>3.\xx{zcnj}.\xx{prog}.make.grow·up} {} {} {} {} //
	\glft	‘It is in a pond that he is raising it.’
		//
\endgl
\xe

\ex\label{ex:91-162-always-feed-whenever-take-bow-and-arrows}%
\exmn{274.5}%
\begingl
	\glpreamble	Aq!ē′x ᴀt tī′q!nutc tcūnē′t ag̣a′łaᴀt g̣anu′kᵘ. //
	\glpreamble	A x̱ʼéix̱ at téexʼ nuch chooneit aag̱áa ala.át g̱anúk. //
	\gla	{} A \rlap{x̱ʼéix̱} @ {} {}
		at @ \rlap{téexʼ} @ {} @ {} @ \•nuch +
		{} {} {} \rlap{chooneit} @ {} @ {} {} {} {}
			{} \rlap{aag̱áa} @ {} {}
			\rlap{ala.át} @ {} @ {} @ {} @ \•g̱anúk. {} //
	\glb	{} a x̱ʼé -x̱ {}
		at= \rt[²]{ti} -μμH -xʼ =nuch
		{} {} {} \rt[¹]{chun} -μμL -i {} át {}
			{} á -g̱áa {}
			a- l- \rt[¹]{.at} -μH =g̱anúk {} //
	\glc	{}[\pr{PP} \xx{3n·pss} mouth -\xx{pert} {}]
		\xx{4n·o}= \rt[²]{handle} -\xx{var} -\xx{pl}\hspace{1ex} =\xx{hab·aux}
		{}[\pr{CP} {}[\pr{NP} {}[\pr{CP} \rt[¹]{wound} -\xx{var} -\xx{rel} {}] thing {}]
			{}[\pr{PP} \xx{3n} -\xx{ades} {}]
			\xx{arg}- \xx{csv}- \rt[¹]{go·\xx{pl}} -\xx{var} =\xx{ctng·aux} {}] //
	\gld	{} its\ix{i} mouth -at {}
		things\• \rlap{\xx{zcnj}.\xx{impfv}.hdl.\xx{rep}} {} {} \•always
		{} {} {} \rlap{bow·and·arrow} {} {} {} {} {}
			{} it\ix{j} -for {}
			\rlap{3>3.\xx{impfv}.handle·\xx{pl}} {} {} {} \•whenever {} //
	\glft	‘He would always be giving it things to eat whenever he took his bow and arrows for it.’
		//
\endgl
\xe

The English translation of (\lastx) has two third person inanimate \fm{it} pronouns that refer to separate things.
This accurately reflects the Tlingit third person nonhuman pronouns in \fm{a x̱ʼéix̱} ‘at its mouth’ and \fm{aag̱áa} ‘for (obtaining) it’, each of which refers to a separate entity.
The referent of \fm{a x̱ʼéix̱} ‘at its mouth’ is the \fm{héen taaÿéeshi} that the protagonist is said to be raising in (\ref{ex:91-161-in-pond-raising-it}).
The referent of \fm{aag̱áa} ‘for (obtaining) it’ is unspecified and is probably meant to by any arbitrary thing that the protagonist might hunt.

The contingent adjunct clause in (\lastx) has nearly the same structure as the sentence in (\ref{ex:91-150-took-around-bow-and-arrow}).
Once again this is a kind of circumlocution that describes hunting as handling a bow and arrows. Here the postposition phrase \fm{aag̱áa} ‘for (obtaining) it’ adds an unspecified goal not present in (\ref{ex:91-150-took-around-bow-and-arrow}).
It is not clear why this additional phrase is needed given that it has no specific referent, but there is no obvious alternative analysis that would give it a specific interpretation.

The contingent auxiliary \fm{g̱anúk} in (\lastx) is not otherwise reported but it is predictable from other attested forms of this auxiliary.
\textcite[155–156]{leer:1991} reports the forms \fm{g̱anikw} in Tongass Tlingit, \fm{g̱aníkw} in Southern Tlingit, \fm{g̱anígún} in Inland Northern Tlingit, and \fm{g̱anúgún} in the rest of Northern Tlingit.
The form \fm{g̱anúk} can be derived either by clipping of \fm{g̱anúgún} or by anticipatory rounding of the second vowel in the Southern form \fm{g̱aníkw}.
Both of these processes are plausible given the other forms.
Alternatively, it is possible that \citeauthor{swanton:1909} missed transcribing the final syllable \fm{…ún}.
But given the presence of \fm{g̱aníkw} elsewhere, the form \fm{g̱anúk} is entirely reasonable as an accurate transcription.

\section{Paragraph 15}\label{sec:91-para-15}

\ex\label{ex:91-163-}%
\exmn{274.7}%
\begingl
	\glpreamble	Wananīsawe ye aỵa′osîqa duʟā′, //
	\glpreamble	Wáa nanée sáwé yéi aÿawsiḵaa du tláa //
	\gla	{} Wáa \rlap{nanée} @ {} @ {} @ {} {}
		\rlap{sáwé} @ {} @ {}
		yéi @ \rlap{aÿawsiḵaa} @ {} @ {} @ {} @ {} @ {} @ {} +
		{} du tláa {} //
	\glb	{} wáa n- \rt[¹]{ni} -μμH {} {} 
		s= á -wé
		yéi= a- ÿ- wu- s- i- \rt[¹]{ḵa} -μμL
		{} du tláa {} //
	\glc	{}[\pr{CP} how \xx{ncnj}- \rt[¹]{happen} -\xx{var} \·\xx{sub} {}]
		\xx{q}= \xx{foc} -\xx{mdst}
		thus= \xx{arg}- \xx{qual}- \xx{pfv}- \xx{csv}- \xx{stv}- \rt[¹]{say} -\xx{var}
		{}[\pr{DP} \xx{3h·pss} mother {}] //
	\gld	{} how \rlap{\xx{csec}.happen} {} {} \·while {}
		ever= \rlap{it.is} {}
		thus\• \rlap{3>3.\xx{ncnj}.\xx{pfv}.say·to} {} {} {} {} {} {}
		{} his mother {} //
	\glft	‘At some point he said to his mother’
		//
\endgl
\xe

\ex\label{ex:91-164-going-for-firewood}%
\exmn{274.7}%
\begingl
	\glpreamble	“G̣a′ng̣ā naādê′.” //
	\glpreamble	«\!Gáng̱aa na.aadí.\!» //
	\gla	{} \llap{«\!}\rlap{Gáng̱aa} @ {} {}
		\rlap{na.aadí\!»} @ {} @ {} @ {}  //
	\glb	{} gán -g̱áa {}
		n- \rt[¹]{.at} -μμL -í //
	\glc	{}[\pr{PP} firewood -\xx{ades} {}]
		\xx{ncnj}- \rt[¹]{go·\xx{pl}} -\xx{var} -\xx{nmz} //
	\gld	{} firewood -for {}
		\rlap{\xx{ncnj}.\xx{prog}.go·\xx{pl}} {} {} -ing //
	\glft	‘“Going for firewood.”’
		//
\endgl
\xe

\ex\label{ex:91-165-your-uncles-never-come-beach}%
\exmn{274.7}%
\begingl
	\glpreamble	“Ỵīkā′k-hᴀs ʟēł ỵî g̣ā′uguttc.” //
	\glpreamble	«\!Ÿee káak hás tléil ÿeiḵ aa ugútch.\!» //
	\gla	{} \llap{«\!}Ÿee káak @ \•hás {}
		tléil ÿeiḵ @ aa @ \rlap{ugútch.\!»} @ {} @ {} @ {} //
	\glb	{} ÿee káak \•hás {}
		tléil ÿeiḵ= aa= u- \rt[¹]{gut} -μH -ch //
	\glc	{}[\pr{DP} \xx{2pl·pss} mat·uncle =\xx{plh} {}]
		\xx{neg} beach= \xx{part}= \xx{zpfv}- \rt[¹]{go·\xx{sg}} -\xx{var} -\xx{rep} //
	\gld	{} your·\xx{pl} uncle \•s {}
		not beach\• some\• \rlap{\xx{zcnj}.\xx{hab}.go·\xx{sg}} //
	\glft	‘“Your uncles never came down to the beach.”’
		//
\endgl
\xe

The sentences in (\ref{ex:91-164-going-for-firewood}) and (\ref{ex:91-165-your-uncles-never-come-beach}) have some unusual uses of plurality.
In (\ref{ex:91-164-going-for-firewood}) \citeauthor{swanton:1909}’s transcription \orth{naādê′} unambiguously points to a verb with \fm{\rt[¹]{.at}} ‘plural go’ even though his gloss “I am going” clearly describes the singular protagonist.
In (\ref{ex:91-165-your-uncles-never-come-beach}) \citeauthor{swanton:1909}’s transcription \orth{Ỵī} unambiguously points to the second person plural possessive pronoun \fm{ÿee} ‘your (pl.)’ even though again the protagonist is singular.
Furthermore, the phrase \fm{ÿee káak hás} in (\ref{ex:91-165-your-uncles-never-come-beach}) denotes a plurality of maternal uncles, but the verb transcribed as \orth{ā′uguttc} can only reasonably be the root \fm{\rt[¹]{gut}} ‘singular go’.
Any one of these divergences from the usual singular/plural distinction could be dismissed as an accident, but all three suggest that the speaker intentionally used these forms.
At present it is not clear what these unusual uses of singular and plural are meant to convey.

\ex\label{ex:91-166-morning-run-around-floor}%
\exmn{274.8}%
\begingl
	\glpreamble	Ts!utā′tawe t!akᴀ′t wudjîx̣ī′x̣. //
	\glpreamble	Tsʼootaat áwé tʼáa kát wujixeex. //
	\gla	Tsʼootaat \rlap{áwé} @ {}
		{} tʼáa \rlap{kát} @ {} {}
		\rlap{wujixeex.} @ {} @ {} @ {} @ {} @ {} //
	\glb	tsʼootaat á -wé
		{} tʼáa ká -t {} 
		wu- d- sh- i- \rt[¹]{xix} -μμL //
	\glc	morning \xx{foc} -\xx{mdst}
		{}[\pr{PP} board \xx{hsfc} -\xx{pnct} {}]
		\xx{pfv}- \xx{mid}- \xx{pej}- \xx{stv}- \rt[¹]{fall·\xx{sg}} -\xx{var} //
	\gld	morning \rlap{it.is} {}
		{} board atop -around {}
		\rlap{\xx{ncnj}.\xx{pfv}.run·\xx{sg}} {} {} {} {} {} //
	\glft	‘It was in the morning that he ran around on the floor.’
		//
\endgl
\xe

\ex\label{ex:91-167-grabbed-stone-adze}%
\exmn{274.8}%
\begingl
	\glpreamble	Tāỵī′s ā′wacāt. //
	\glpreamble	Taÿees aawasháat. //
	\gla	{} \rlap{Taÿees} @ {} {} \rlap{aawasháat.} @ {} @ {} @ {} @ {} //
	\glb	{} té- ÿees {} a- wu- i- \rt[²]{shaʼt} -μμH //
	\glc	{}[\pr{DP} stone- wedge {}]
		\xx{arg}- \xx{pfv}- \xx{stv}- \rt[²]{grab} -\xx{var} //
	\gld	{} \rlap{stone·adze} {} {} \rlap{3>3.\xx{gcnj}.\xx{pfv}.grab} {} {} {} {} //
	\glft	‘He picked up a stone adze.’
		//
\endgl
\xe

\FIXME{Discuss noun \fm{taÿees} ‘stone chisel, stone adze’ \parencite[03/210]{leer:1973} and noun \fm{ÿees} ‘wedge’ as well as \fm{yees} \~\ \fm{yeis} \~\ \fm{ÿees} ‘horse clam’.}

\ex\label{ex:91-168-running-inland-within-this-one}%
\exmn{274.9}%
\begingl
	\glpreamble	Tc!uʟe′ ya ʟēq! yatī′ỵīỵa ỵîk ā′we dāq anacî′k. //
	\glpreamble	Chʼu tle yá tléixʼ yateeÿi aa ÿík áwé daaḵ anashíx. //
	\gla	Chʼu tle {} yá {} tléixʼ \rlap{yateeÿi} @ {} @ {} @ {} {} aa ÿík {}
		\rlap{áwé} @ {}
		daaḵ @ \rlap{anashíx.} @ {} @ {} @ {} @ {} @ {} //
	\glb	chʼu tle {} yá {} tléixʼ i- \rt[¹]{tiʰ} -μμL -i {} aa ÿíᵏ {}
		á -wé
		dáaḵ= a- n- d- sh- \rt[¹]{xix} -μH //
	\glc	just then {}[\pr{DP} \xx{prox}
			{}[\pr{CP} one \xx{stv}- \rt[¹]{be} -\xx{var} -\xx{rel} {}]
			one within {}]
		\xx{foc} -\xx{mdst}
		inland= \xx{4h·s}- \xx{ncnj}- \xx{mid}- \xx{pej}- \rt[¹]{fall·\xx{sg}} -\xx{var} //
	\gld	just then {} this {} one \rlap{\xx{ncnj}.\xx{impfv}.be} {} {} -that {}
			one within {}
		\rlap{it.is} {}
		inland \rlap{one.\xx{zcnj}.\xx{prog}.run·\xx{sg}} {} {} {} {} {} //
	\glft	‘Just then it was within this one that is one that he was running inland.’
		//
\endgl
\xe

\FIXME{Discuss the use of fourth person human subject \fm{a-} here.
Strange since it’s referential and the referent is not obviously a background character.}

\FIXME{The phrase \fm{yá tléixʼ yateeÿi aa} apparently refers to one of the paths for firewood.
Why is it ‘alone’?}

\ex\label{ex:91-169-this-one-within-finger-sticks-out}%
\exmn{274.9}%
\begingl
	\glpreamble	Yā′de ỵîk nᴀxawê′n łîcu′ qāʟ!ē′q. //
	\glpreamble	Yá dei ÿíknáx̱ áwé nḁlishóo ḵaa tlʼeiḵ. //
	\gla	{} Yá dei \rlap{ÿíknáx̱} @ {} {} \rlap{áwé} @ {}
		\rlap{nḁlishóo} @ {} @ {} @ {} @ {}
		{} ḵaa tlʼeiḵ. {} //
	\glb	{} yá dei ÿíᵏ -náx̱ {} á -wé
		n- l- i- \rt[¹]{shu} -μμH
		{} ḵaa tlʼeeͥḵ {} //
	\glc	{}[\pr{PP} \xx{prox} path within -\xx{perl} {}]
		\xx{foc} -\xx{mdst}
		\xx{ncnj}- \xx{xtn}- \xx{stv}- \rt[¹]{end} -\xx{var}
		{}[\pr{DP} \xx{4h·pss} finger {}] //
	\gld	{} this road within -along {} \rlap{it.is} {}
		\rlap{\xx{ncnj}.\xx{ext}·\xx{impfv}.extend} {} {} {} {}
		{} someone’s finger {} //
	\glft	‘It was along within this road that it was sticking out, someone’s finger.’
		//
\endgl
\xe

\ex\label{ex:91-170-gimme-your-finger}%
\exmn{274.10}%
\begingl
	\glpreamble	“Hāndê′ īʟ!ē′q,” //
	\glpreamble	«\!Haandé i tlʼeiḵ\!» //
	\gla	{} \llap{«\!}\rlap{Haandé} @ {} {} {} i tlʼeiḵ\!» {} //
	\glb	{} haaⁿ -dé {} {} i tlʼeeͥḵ {} //
	\glc	{}[\pr{PP} \xx{cis} -\xx{all} {}] {}[\pr{DP} \xx{2sg·pss} finger {}] //
	\gld	{} here -to {} {} your·\xx{sg} finger {} //
	\glft	‘“Here, your finger”’
		//
\endgl
\xe

\ex\label{ex:91-171-so-he-said}%
\exmn{274.10}%
\begingl
	\glpreamble	yū′aciaosîqa. //
	\glpreamble	yóo ash yawsiḵaa. //
	\gla	yóo @ ash @ \rlap{yawsiḵaa.} @ {} @ {} @ {} @ {} @ {} //
	\glb	yóo= ash= ÿ- wu- s- i- \rt[¹]{ḵa} -μμL //
	\glc	\xx{quot}= \xx{3prx·o}= \xx{qual}- \xx{pfv}- \xx{csv}- \xx{stv}- \rt[¹]{say} -\xx{var} //
	\gld	thus him \rlap{\xx{ncnj}.\xx{pfv}.say·to} {} {} {} {} {} //
	\glft	‘so he said to him.’
		//
\endgl
\xe

\ex\label{ex:91-172-pull-finger-where-poke-thru}%
\exmn{274.11}%
\begingl
	\glpreamble	ʟe anᴀ′x ʟ!akołi′tsag̣ē′awe āx ke ā′waxot!. //
	\glpreamble	Tle aanáx̱ tlʼeḵwulitsaag̱i yé áwé aax̱ kei aawax̱útʼ. //
	\gla	Tle {} {} {} \rlap{aanáx̱} @ {} {}
			\rlap{tlʼeḵwulitsaag̱i} @ {} @ {} @ {} @ {} @ {} @ {} {} yé {}
		\rlap{áwé} @ {} +
		{} \rlap{aax̱} @ {} {}
		kei @ \rlap{aawax̱útʼ.} @ {} @ {} @ {} @ {} //
	\glb	tle {} {} {} á -náx̱ {}
			tlʼeͥḵ- wu- l- i- \rt[¹]{tsaḵ} -μμH -i {} yé {}
		á -wé
		{} á -dáx̱ {}
		kei= a- wu- i- \rt[²]{x̱utʼ} -μH //
	\glc	tle {}[\pr{DP} {}[\pr{CP} {}[\pr{PP} \xx{3n} -\xx{perl} {}]
			finger- \xx{pfv}- \xx{xtn}- \xx{stv}-
				\rt[²]{poke} -\xx{var} -\xx{rel} {}] place {}]
		\xx{foc} -\xx{mdst}
		{}[\pr{PP} \xx{3n} -\xx{abl} {}]
		up= \xx{arg}- \xx{pfv}- \xx{stv}- \rt[²]{pull} -\xx{var} //
	\gld	then {} {} {} it -thru {}
			\rlap{finger.\xx{ncnj}.\xx{pfv}.poke} {} {} {} {} {} -where {} place {}
		\rlap{it.is} {}
		{} it -from {}
		up \rlap{3>3.\xx{zcnj}.\xx{pfv}.pull} {} {} {} {} //
	\glft	‘Then that place through where it poked the finger, he pulled it up from there.’
		//
\endgl
\xe

\ex\label{ex:91-173-onto-stone-drag}%
\exmn{274.11}%
\begingl
	\glpreamble	Tᴀkᴀ′tawe aca′ołîxot!. //
	\glpreamble	Té kát áwé ashawlix̱útʼ. //
	\gla	{} Té \rlap{kát} @ {} {} \rlap{áwé} @ {}
		\rlap{ashawlix̱útʼ.} @ {} @ {} @ {} @ {} @ {} @ {} //
	\glb	{} té ká -t {} á -wé
		a- sha- wu- l- i- \rt[²]{x̱utʼ} -μH //
	\glc	{}[\pr{PP} stone \xx{hsfc} -\xx{pnct} {}] \xx{foc} -\xx{mdst}
		\xx{arg}- head- \xx{pfv}- \xx{xtn}- \xx{stv}- \rt[²]{pull} -\xx{var} //
	\gld	{} stone atop -to {} \rlap{it.is} {}
		\rlap{3>3.head.\xx{zcnj}.\xx{pfv}.pull} {} {} {} {} {} {} //
	\glft	‘It was onto a stone that he dragged it.’
		//
\endgl
\xe

\FIXME{\fm{ashawlix̱óot} looks like the verb for ‘fish with a pole’ but that doesn’t make sense.
It’s probably literal ‘drag its head’, but then we’ve switched from finger to head.}

\ex\label{ex:91-174-skeletons-scattered}%
\exmn{274.12}%
\begingl
	\glpreamble	Yuˈ-āx-kē-awaxo′t!eya awe′ qāxagîˈ ye wudzîg̣ā′t ade′ qoī′n-iỵa. //
	\glpreamble	Yú aax̱ kei aawax̱útʼi yé áwé ḵaa xáagi yéi wudzig̱áat aadé ḵu.eeni ÿé. //
	\gla	{} Yú {} {} \rlap{aax̱} @ {} {}
			kei @ \rlap{aawax̱útʼi} @ {} @ {} @ {} @ {} @ {} {}
			yé {} \rlap{áwé} @ {} +
		{} ḵaa \rlap{xáagi} @ {} {}
		yéi @ \rlap{wudzig̱áat,} @ {} @ {} @ {} @ {} @ {} +
		{} {} {} \rlap{aadé} @ {} {}
			\rlap{ḵu.eeni} @ {} @ {} @ {} {} yé. {} //
	\glb	{} yú {} {} á -dáx̱ {}
			kei= a- wu- i- \rt[²]{x̱utʼ} -μH -i {}
			yé {} á -wé
		{} ḵaa xáak -í {}
		yéi= wu- d- s- i- \rt[¹]{g̱aʼt} -μμH
		{} {} {} á -dé {}
			ḵu- \rt[²]{.in} -μμL\hspace{1.25em} -i {} yé {} //
	\glc	{}[\pr{DP} \xx{dist} {}[\pr{CP} {}[\pr{PP} \xx{3n} -\xx{abl} {}]
			up= \xx{arg}- \xx{pfv}- \xx{stv}- \rt[²]{pull} -\xx{var} -\xx{rel} {}]
			place {}] \xx{foc} -\xx{mdst}
		{}[\pr{DP} \xx{4h·pss} skeleton -\xx{pss} {}]
		thus \xx{pfv}- \xx{pasv}- \xx{csv}- \xx{stv}- \rt[¹]{scatter} -\xx{var}
		{}[\pr{DP} {}[\pr{CP} {}[\pr{PP} \xx{3n} -\xx{all} {}]
			\xx{4h·o}- \rt[²]{kill·\xx{pl}} -\xx{var} -\xx{rel} {}] place {}] //
	\gld	{} that {} {} it -from {}
			up \rlap{3>3.\xx{zcnj}.\xx{pfv}.pull} {} {} {} {} -where {}
			place {} \rlap{it.is} {}
		{} ppl’s \rlap{skeletons} {} {}
		thus \rlap{\xx{g̱cnj}.\xx{pfv}.be.make.scatter} {} {} {} {} {}
		{} {} {} it -to {}
			\rlap{ppl.\xx{ncnj}.\xx{impfv}.kill·\xx{pl}} {} {} -where {} place {} //
	\glft	‘It was that place where he dragged it up from, people’s skeletons were scattered there, the place where it kills people.’
		//
\endgl
\xe

\ex\label{ex:91-175-cut-at-neck-with-greenstone-adze}%
\exmn{274.13}%
\begingl
	\glpreamble	Tc!uʟe′ qaxᴀse′ was!ū′ yū′s!u taỵī′stc. //
	\glpreamble	Chʼu tle a kaax̱ aseiwasʼóow yú sʼoow taÿeesch. //
	\gla	Chʼu tle {} a \rlap{kaax̱} @ {} {}
		\rlap{aseiwasʼóow} @ {} @ {} @ {} @ {} @ {} +
		{} yú sʼoow \rlap{taÿeesch.} @ {} @ {} {} //
	\glb	chʼu tle {} a ká -dáx̱ {}
		a- se- wu- i- \rt[²]{sʼuʼw} -μμH
		{} yú sʼoow té- ÿees -ch {} //
	\glc	just then {}[\pr{PP} \xx{3n·pss} \xx{hsfc} -\xx{abl} {}]
		\xx{arg}- voice- \xx{pfv}- \xx{stv}- \rt[²]{chop} -\xx{var}
		{}[\pr{PP} \xx{dist} greenstone stone- wedge -\xx{erg} {}] //
	\gld	just then {} its atop -from {}
		\rlap{3>3.neck.\xx{ncnj}.\xx{pfv}.chop} {} {} {} {} {}
		{} that greenstone \rlap{stone·adze} {} -with {} //
	\glft	‘The he cut it off of it at the neck with that greenstone adze.’
		//
\endgl
\xe

\FIXME{Discuss use of \fm{se-} ‘voice’ and interpretation as neck. The head is implicit. \citeauthor{swanton:1909}’s gloss “Then his neck he cut off [with] his stone ax” shows that he understood this.}

\FIXME{Note the meaning of \fm{sʼoow} < \fm{sʼoom} < \fm[*]{sʼuʰm}.}

\ex\label{ex:91-176-to-beach-took-to-mother}%
\exmn{274.13}%
\begingl
	\glpreamble	Yî′qdê awate′ duʟā′x. //
	\glpreamble	Íḵde aawatee du tláaxʼ. //
	\gla	{} \rlap{Íḵde} @ {} {}
		\rlap{aawatee} @ {} @ {} @ {} @ {}
		{} du \rlap{tláaxʼ.} @ {} {} //
	\glb	{} éeͥḵ -dé {} a- wu- i- \rt[²]{ti} -μμL
		{} du tláa xʼ {} //
	\glc	{}[\pr{PP}  beach -\xx{all} {}]
		\xx{arg}- \xx{pfv}- \xx{stv}- \rt[²]{handle} -\xx{var}
		{}[\pr{PP} \xx{3h·pss} mother -\xx{loc} {}] //
	\gld	{} beach -to {}
		\rlap{3>3.\xx{ncnj}.\xx{pfv}.handle} {} {} {} {}
		{} his mother -at {} //
	\glft	‘He took it to the beach, to his mother.’
		//
\endgl
\xe

\ex\label{ex:91-177-threw-inside}%
\exmn{274.14}%
\begingl
	\glpreamble	ᴀnᴀ′xawe nēł ā′wagîq! //
	\glpreamble	Anáx̱ áwé neil aawag̱íxʼ. //
	\gla	{} \rlap{Anáx̱} @ {} {} \rlap{áwé} @ {}
		{} neil @ {} {}
		\rlap{aawag̱íxʼ.} @ {} @ {} @ {} @ {} //
	\glb	{} á -náx̱ {} á -wé
		{} neil -t {}
		a- wu- i- \rt[²]{g̱ixʼ} -μH //
	\glc	{}[\pr{PP} \xx{3n} -\xx{perl} {}] \xx{foc} -\xx{mdst}
		{}[\pr{PP} inside -\xx{pnct} {}]
		\xx{arg}- \xx{pfv}- \xx{stv}- \rt[²]{throw·inan} -\xx{var} //
	\gld	{} it -thru {} \xx{it.is} {}
		{} inside -to {}
		\rlap{3>3.\xx{zcnj}.\xx{pfv}.throw·inan} {} {} {} {} //
	\glft	‘Along there it was that he threw it inside.’
		//
\endgl
\xe

The sentence in (\ref{ex:91-177-threw-inside}) was transcribed by \citeauthor{swanton:1909} with the phrase \orth{qᴀ dułī′łk!} \fm{ḵa du léelkʼw} at the end.
This is not grammatical in context because it coordinates two unlike phrases, namely a main clause with a DP.
As the beginning of the sentence in (\ref{ex:91-178-cut-face-into-strips}) it can be interpreted as a topic expansion adding the grandmother to the mother mentioned in (\ref{ex:91-176-to-beach-took-to-mother}).
This is still grammatically unusual, but it is more semantically coherent since the mother is implicitly a subject of the verb \fm{has aÿadaakaháan} and the grandmother can be sensibly added to the subject to satisfy the plurality required by \fm{has=}.

\ex\label{ex:91-178-cut-face-into-strips}%
\exmn{274.14}%
\begingl
	\glpreamble	qᴀ dułī′łk!. ʟe hᴀs aỵada′kahān. //
	\glpreamble	Ḵa du léelkʼw, tle has aÿadaakaháan. //
	\gla	Ḵa {} du léelkʼw, {} tle has @ \rlap{aÿadaakaháan.} @ {} @ {} @ {} @ {} @ {} //
	\glb	ḵa {} du léelkʼw {}
		tle has= a- ÿa- daa- k- \rt[²]{han} -μμH //
	\glc	ḵa {}[\pr{DP} \xx{3h·pss} grandparent {}]
		then \xx{plh}=
			\xx{arg}- face- around- \xx{qual}- \rt[²]{cut·strips} -\xx{var} //
	\gld	and {} his grandmother {}
		then they
			\rlap{3>3.face.around.\xx{g̱cnj}.\xx{impfv}.cut·strips} {} {} {} {} {} //
	\glft	‘And his grandmother, then they are cutting the skin of its face in strips.’
		//
\endgl
\xe

The verb in (\lastx) is unusual in that it contains the sequence \fm{ÿa-daa-} where the order of these two prefixes is otherwise only \fm{daa-ÿa-}.
This reflects a phenomenon of noun phrase incorporation that is very uncommon in modern Tlingit but which apparently used to be more frequent in the past.
The sequence \fm{a-ÿa-daa-} is congruent with the noun phrase \fm{a ÿa-daa} ‘its face-around’, i.e.\ ‘its face skin’.
The \fm{a-} prefix thus seems to do double duty in (\lastx) as the three-on-three argument marker and as the possessor of the incorporated noun phrase.
If this phrase was not incorporated we would expect a sequence like \fm{tle a yadaa has akaháan} which cannot be reasonably inferred from \citeauthor{swanton:1909}’s transcription.
The base verb to which \fm{a yadaa} has been incorporated is \fm{akaháan} (\fm{g̱}; \fm{-μμH} activity, \fm{-t} repetitive) ‘s/he cuts it into strips’ which is an uncommon variant of the more usual \fm{aksaháan} (\fm{g̱}; \fm{-μμH} activity, \fm{-t} repetitive) ‘s/he cuts it into strips’ \parencite[01/65]{leer:1973} that includes the \fm{s-} classifier prefix presumably indicating spatial extension.

\ex\label{ex:91-179-roast-with-urine}%
\exmn{275.1}%
\begingl
	\glpreamble	Kwᴀs tîn hᴀs aỵat!ū′s! gᴀ′nᴀłtāq!. //
	\glpreamble	Kwás tin has aÿatʼóosʼ ganaltáaxʼ. //
	\gla	{} Kwás tin {} has @ \rlap{aÿatʼóosʼ} @ {} @ {} @ {}
		{} \rlap{ganaltáak.} @ {} @ {} {}  //
	\glb	{} kwás tin {} has= a- ÿ- \rt[²]{tʼusʼ} -μμH
		{} \rt[¹]{gan}- ltáaᵏ {} {} //
	\glc	{}[\pr{PP} old·urine \xx{instr} {}]
		\xx{plh}= \xx{arg}- face- \rt[²]{toast} -\xx{var}
		{}[\pr{PP} \rt[¹]{burn}- middle \·\xx{loc} {}] //
	\gld	{} old·urine with {} they \rlap{3>3.face.\xx{zcnj}.\xx{impfv}.toast} {} {} {}
		{} fire- middle -in {} //
	\glft	‘They are toasting its face with aged urine in the middle of the fire.’
		//
\endgl
\xe

\ex\label{ex:91-180-like-how-felt}%
\exmn{275.1}%
\begingl
	\glpreamble	Ade′ hᴀs dᴀnū′guya yᴀ′xawe sᴀtānâ′. //
	\glpreamble	Aadé has danóogu yé yáx̱ áwé asataan áa. //
	\gla	{} {} {} {} \rlap{Aadé} @ {} {}
			has @ \rlap{danoogu} @ {} @ {} @ {} {} yé {}
		yáx̱ {} \rlap{áwé} @ {} +
		\rlap{asataan} @ {} @ {} @ {}
		{} \rlap{áa.} @ {} {} //
	\glb	{} {} {} {} á -dé {}
			has= d- \rt[²]{nuͥk} -μμH -i {} yé {}
		yáx̱ {} á -wé
		a- s- \rt[²]{tan} -μμL
		{} á -μ {} //
	\glc	{}[\pr{PP} {}[\pr{DP} {}[\pr{CP} {}[\pr{PP} \xx{3n} -\xx{all} {}]
			\xx{plh}= \xx{mid}- \rt[²]{feel} -\xx{var} -\xx{rel} {}] way {}]
		\xx{sim} {}] \xx{foc} -\xx{mdst}
		\xx{arg}- \xx{xtn}- \rt[²]{handle·w/e} -\xx{var}
		{}[\pr{PP} \xx{3n} -\xx{loc} {}] //
	\gld	{} {} {} {} it -to {}
			they \rlap{\xx{zcnj}.\xx{impfv}.feel} {} {} -that {} way {}
		like {} \rlap{it.is} {}
		\rlap{3>3.\xx{zcnj}.\xx{impfv}.hdl·w/e.\xx{rep}} {} {} {}
		{} there -at {} //
	\glft	‘It was like the way that they felt that they handled it, there.’
		//
\endgl
\xe

\ex\label{ex:91-181-made-arrows-go-near}%
\exmn{275.2}%
\begingl
	\glpreamble	Wananī′sawe ts!u tcū′net a′ołīāt yū′ỵaanᴀ′swᴀt hīntaỵī′cî xᴀ′ndî. //
	\glpreamble	Wáa nanée sáwé tsu chooneit awli.aat yú ÿaa anaswát héen taaÿéeshi x̱ánde. //
	\gla	{} Wáa \rlap{nanée} @ {} @ {} @ {} {}
		\rlap{sáwé} @ {} @ {} tsu
		{} {} \rlap{chooneit} @ {} @ {} {} {} {}
		\rlap{awli.aat} @ {} @ {} @ {} @ {} @ {}
		{} {} yú {} ÿaa @ \rlap{anaswát} @ {} @ {} @ {} @ {} {} {} +
			{} héen \rlap{taaÿéeshi} @ {} @ {} @ {} {} {}
			\rlap{x̱ánde} @ {} {} //
	\glb	{} wáa n- \rt[¹]{ni} -μμH {} {} 
		s= á -wé tsu
		{} {} \rt[¹]{chun} -μμL -i {} át {}
		a- wu- l- i- \rt[¹]{.at} -μμL
		{} {} yú {} ÿaa= a- n- s- \rt[¹]{waʼt} -μH {} {}
			{} héen táaᵏ= \rt[²]{ÿish} -μμH -i {} {}
			x̱án -dé {} //
	\glc	{}[\pr{CP} how \xx{ncnj}- \rt[¹]{happen} -\xx{var} \·\xx{sub} {}]
		\xx{q}= \xx{foc} -\xx{mdst} also
		{}[\pr{DP} {}[\pr{CP} \rt[¹]{wound} -\xx{var} -\xx{rel} {}] thing {}]
		\xx{arg}- \xx{pfv}- \xx{csv}- \xx{stv}- \rt[¹]{go·\xx{pl}} -\xx{var}
		{}[\pr{PP} {}[\pr{DP} \xx{dist}
				{}[\pr{CP} along= \xx{arg}- \xx{ncnj}- \xx{csv}-
					\rt[¹]{adult} -\xx{var} \·\xx{rel} {}]
			{}[\pr{NP} water bottom= \rt[²]{pull} -\xx{var} -\xx{pss} {}] {}]
			near -\xx{all} {}] //
	\gld	{} how \rlap{\xx{csec}.happen} {} {} \·while {}
		ever\• \rlap{it.is} {} again
		{} {} \rlap{arrow} {} {} {} {} {}
		\rlap{3>3.\xx{ncnj}.\xx{pfv}.make.go·\xx{pl}} {} {} {} {} {}
		{} {} that {} along \rlap{3>3.\xx{zcnj}.\xx{prog}.make.grow·up} {} {} {} {} 
					\·that {}
			{} water bottom \rlap{puller} {} -of {} {}
			near -to {} //
	\glft	‘At some point again he made his arrows go there, near that \fm{héen taaÿéeshi} that he was raising.’
		//
\endgl
\xe

The phrase \fm{chooneit awli.aat} ‘he made his bow and arrows go’ in (\lastx) could be interpreted to mean that he shot the \fm{héen taaÿéeshi}.
There are two arguments against this interpretation.
One is that the next sentence in (\nextx) explicitly describes his shooting of the \fm{héen taaÿéeshi}, so this sentence would then be redundant.
The second argument is that earlier in sentences (\ref{ex:91-36-went-thattaway}), (\ref{ex:91-150-took-around-bow-and-arrow}), and (\ref{ex:91-162-always-feed-whenever-take-bow-and-arrows}) a similar phrase does not describe the act of shooting.
Instead this expression describes going out with a bow and arrows, either for practice or for hunting.
Here its use in (\lastx) sets up the context for the protagonist to shoot the \fm{héen taaÿéeshi} in (\nextx) and there is no redundancy.

\ex\label{ex:91-182-shot-it-above-head}%
\exmn{275.3}%
\begingl
	\glpreamble	Tsā′tc!ᴀs dohᴀ′nî yā′nᴀx kołagē′ỵîawe āwat!u′k cakī′nᴀx yū′ỵaanᴀ′swᴀt hīntaỵī′cî. //
	\glpreamble	Tsáa chʼas duháni yáanáx̱ kulagéiÿi áwé aawatʼúk a shakéenáx̱, yú ÿaa anaswát héen taaÿéeshi. //
	\gla	Tsáa \rlap{chʼas} @ {}
		{} {} {} \rlap{duháni} @ {} @ {} @ {} {} \rlap{yáanáx̱} {} 
			\rlap{kulagéiÿi} @ {} @ {} @ {} @ {} @ {} @ {}
		\rlap{áwé} @ {}
		\rlap{aawatʼúk} @ {} @ {} @ {} @ {}
		{} a \rlap{shakéenáx̱,} @ {} @ {} {} +
		{} yú {} ÿaa @ \rlap{anaswát} @ {} @ {} @ {} @ {} {} {}
			{} héen \rlap{taaÿéeshi} @ {} @ {} @ {} {} {} //
	\glb	tsáa chʼa =s
		{} {} {} du- \rt[¹]{han} -μH -í {} yáanáx̱ {}
			k- u- l- \rt[¹]{ge} -μμH -í {}
		á -wé
		a- wu- i- \rt[²]{tʼuk} -μH
		{} a shá- kée -náx̱ {}
		{} yú {} ÿaa= a- n- s- \rt[¹]{waʼt} -μH {} {}
			{} héen táaᵏ= \rt[²]{ÿish} -μμH -i {} {} //
	\glc	then just =\xx{dub}
		{}[\pr{CP} {}[\pr{PP} {}[\pr{NP}
					\xx{4h·s}- \rt[¹]{stand·\xx{sg}} -\xx{var} -\xx{nmz} {}]
				\xx{sup} {}]
			\xx{cmpv}- \xx{irr}- \xx{xtn}- \rt[¹]{big} -\xx{var} -\xx{sub} {}]
		\xx{foc} -\xx{mdst}
		\xx{arg}- \xx{pfv}- \xx{stv}- \rt[²]{shoot·bow} -\xx{var}
		{}[\pr{PP} \xx{3n·pss} head- above -\xx{perl} {}]
		{}[\pr{DP} \xx{dist} 
				{}[\pr{CP} along= \xx{arg}- \xx{ncnj}- \xx{csv}-
					\rt[¹]{adult} -\xx{var} \·\xx{rel} {}]
			{}[\pr{NP} water bottom= \rt[²]{pull} -\xx{var} -\xx{pss} {}] {}] //
	\gld	then just \•may\rlap{be}
		{} {} {} \llap{\xx{zcnj}.}\rlap{\xx{impfv}.one.stand·\xx{sg}} {} {} -ing {}
				more·than {}
			\rlap{\xx{gcnj}.\xx{cmpv}.\xx{impfv}.tall} {} {} {} {} -when {}
		\rlap{it.is} {}
		\rlap{3>3.\xx{zcnj}.\xx{pfv}.shoot·bow} {} {} {} {}
		{} its head- above -thru {}
		{} that {} along \rlap{3>3.\xx{zcnj}.\xx{prog}.make.grow·up} {} {} {} {}
					\·that {}
			{} water bottom \rlap{puller} {} -of {} {} //
	\glft	‘Then it was maybe as it was as tall as a person standing that he shot it through the top of the head, that \fm{héen taaÿéeshi} that he was raising.’
		//
\endgl
\xe

\ex\label{ex:91-183-}%
\exmn{275.4}%
\begingl
	\glpreamble	Cunāỵ′êt dāq ā′wate. //
	\glpreamble	Shunaaÿeit daaḵ aawatée. //
	\gla	{} \rlap{Shunaaÿeit} @ {} @ {} @ {} {}
		daaḵ @ \rlap{aawatée.} @ {} @ {} @ {} @ {} //
	\glb	{} shú- niÿaa- ÿee- át {}
		dáaḵ= a- wu- i- \rt[²]{ti} -μμH //
	\glc	{}[\pr{DP} end- dir’n- below- thing {}]
		inland= \xx{arg}- \xx{pfv}- \xx{stv}- \rt[²]{handle} -\xx{var} //
	\gld	{} \rlap{??} {} {} {} {}
		off= \rlap{3>3.\xx{zcnj}.\xx{pfv}.handle} {} {} {} {} //
	\glft	‘He took off the \fm{shunaaÿeit}.’
		//
\endgl
\xe

\FIXME{What is \fm{shunaaÿeit}?}

\ex\label{ex:91-184-hung-on-stump}%
\exmn{275.4}%
\begingl
	\glpreamble	Tc!uʟe′ ᴀtgūwu′n ax awate′. //
	\glpreamble	Chʼu tle atgoowú náax̱ aawatee. //
	\gla	Chʼu tle {} \rlap{atgoowú} @ {} @ {} \rlap{náax̱} @ {} {}
		\rlap{aawatee.} @ {} @ {} @ {} @ {} //
	\glb	chʼu tle {} at= gú -í náa -x̱ {}
		a- wu- i- \rt[²]{ti} -μμL //
	\glc	just then {}[\pr{PP} \xx{4n·pss} base -\xx{pss} covering -\xx{pert} {}]
		\xx{arg}- \xx{pfv}- \xx{stv}- \rt[²]{handle} -\xx{var} //
	\gld	just then {} sth’s \rlap{base} {} covering -on {}
		\rlap{3>3.\xx{ncnj}.\xx{pfv}.handle} {} {} {} {} //
	\glft	‘Then he hung it over a stump.’
		//
\endgl
\xe

\ex\label{ex:91-185-so-sharp}%
\exmn{275.5}%
\begingl
	\glpreamble	Awᴀ′n ʟᴀx wâsa yak!u′ts!. //
	\glpreamble	A wán tlax̱ wáa sá yakʼátsʼ. //
	\gla	{} A wán {} {} tlax̱ wáa sá {}
		\rlap{yakʼátsʼ.} @ {} @ {} //
	\glb	{} a wán {} {} tlax̱ wáa sá {}
		i- \rt[¹]{kʼatsʼ} -μμH //
	\glc	{}[\pr{DP} \xx{3n·pss} edge {}] {}[\pr{QP} very how \xx{q} {}]
		\xx{stv}- \rt[¹]{sharp} -\xx{var} //
	\gld	{} its edge {} {} very how \xx{q} {}
		\rlap{\xx{gcnj}.\xx{impfv}.be.sharp} {} {} //
	\glft	‘Its edge is so very sharp.’
		//
\endgl
\xe

\section{Paragraph 16}\label{sec:91-para-16}

\ex\label{ex:91-186-arrive-town-run-around-on-floor}%
\exmn{275.6}%
\begingl
	\glpreamble	Ānt gū′dawe ts!u t!ākᴀt wudjîx̣ī′x. //
	\glpreamble	Aant góot áwé tsu tʼáa kát wujixeex. //
	\gla	{} {} \rlap{Aant} @ {} {} \rlap{góot} @ {} @ {} @ {} {}
		\rlap{áwé} @ {}
		tsu {} tʼáa \rlap{kát} @ {} {}
		\rlap{wujixeex.} @ {} @ {} @ {} @ {} @ {} //
	\glb	{} {} aan -t {} {} \rt[¹]{gut} -μμH {} {}
		á -wé
		tsu {} tʼáa ká -t {}
		wu- d- sh- i- \rt[¹]{xix} -μμL //
	\glc	{}[\pr{CP} {}[\pr{PP} town -\xx{pnct} {}]
			\xx{zcnj}- \rt[¹]{go·\xx{pl}} -\xx{var} \·\xx{sub} {}]
		\xx{foc} -\xx{mdst}
		also {}[\pr{PP} board \xx{hsfc} -\xx{pnct} {}]
		\xx{pfv}- \xx{mid}- \xx{pej}- \xx{stv}- \rt[¹]{fall} -\xx{var} //
	\gld	{} {} town -to {} \rlap{\xx{zcnj}.\xx{csec}.go·\xx{sg}} {} {} {} {}
		\rlap{it.is} {}
		again {} board atop -around {}
		\rlap{\xx{ncnj}.\xx{pfv}.run·\xx{sg}} {} {} {} {} {} //
	\glft	‘Having arrived at town, again he ran around on the floor.’
		//
\endgl
\xe

\ex\label{ex:91-187-pick-up-stone-adze-within-one-extended}%
\exmn{275.6}%
\begingl
	\glpreamble	Yā′de wucū′wu aỵî′kdê tā′ỵīs ā′wacāt. //
	\glpreamble	Yáade wooshoowu aa ÿíkde taÿees aawasháat. //
	\gla	{} {} {} \rlap{Yáade} @ {} {}
				\rlap{wooshoowu} @ {} @ {} @ {} @ {} {}
			aa \rlap{ÿíkde} @ {} {} 
		{} \rlap{taÿees} @ {} {}
		\rlap{aawasháat.} @ {} @ {} @ {} @ {} //
	\glb	{} {} {} yá -dé {}
				wu- i- \rt[¹]{shu} -μμL -i {}
			 aa ÿíᵏ -dé {}
		{} té- ÿees {}
		a- wu- i- \rt[²]{shaʼt} -μμH //
	\glc	{}[\pr{PP} {}[\pr{CP} {}[\pr{PP} \xx{prox} -\xx{all} {}]
				\xx{pfv}- \xx{stv}- \rt[¹]{extend} -\xx{var} -\xx{rel} {}]
			\xx{part} within -\xx{all} {}]
		{}[\pr{DP} stone- wedge {}]
		\xx{arg}- \xx{pfv}- \xx{stv}- \rt[²]{grab} -\xx{var} //
	\gld	{} {} {} this -way {} \rlap{\xx{ncnj}.\xx{pfv}.extend} {} {} {} -which {}
			one within -to {}
		{} \rlap{stone·adze} {} {}
		\rlap{3>3.\xx{gcnj}.grab} {} {} {} {} //
	\glft	‘To within the one which extended this way he took up the stone adze.’
		//
\endgl
\xe

\ex\label{ex:91-188-run-off-inland-saw-head-sticking-out}%
\exmn{275.7}%
\begingl
	\glpreamble	Yudā′qedaqe cī′x̣awe aosiē′n qācaỵî′ deỵîknᴀx cᴀnacū′• //
	\glpreamble	Yóot dáag̱i daaḵ ishéex áwé awsiteen ḵaa shaaÿí dei ÿíknáx̱ shanaashóo. //
	\gla	{} \rlap{Yóot} @ {} \rlap{dáag̱i} @ {} @ daaḵ @
			\rlap{ishéex} @ {} @ {} @ {} @ {} @ {} {}
		\rlap{áwé} @ {}
		\rlap{awsiteen} @ {} @ {} @ {} @ {} @ {}
		{} {} ḵaa \rlap{shaaÿí} @ {} {}
			{} dei \rlap{ÿíknáx̱} @ {} {}
			\rlap{shanaashóo.} @ {} @ {} @ {} @ {} @ {} {} //
	\glb	{} yú -t= dáaḵ -í= dáaḵ= {} d- sh- \rt[¹]{xix} -μμH {} {}
		á -wé
		a- wu- s- i- \rt[²]{tin} -μμL
		{} {} ḵaa shá -í {}
			{} dei ÿíᵏ -náx̱ {}
			sha- n- i- \rt[¹]{shu} -μμH {} {} //
	\glc	{}[\pr{CP} \xx{dist} -\xx{pnct}= inland -\xx{loc}= inland=
			\xx{zcnj}\· \xx{mid}- \xx{pej}- \rt[¹]{fall} -\xx{var} \·\xx{sub} {}]
		\xx{foc} -\xx{mdst}
		\xx{arg}- \xx{pfv}- \xx{xtn}- \xx{stv}- \rt[²]{see} -\xx{var}
		{}[\pr{CP} {}[\pr{DP} \xx{4h·pss} head -\xx{pss} {}]
			{}[\pr{PP} path within -\xx{perl} {}]
			head- \xx{ncnj}- \xx{stv}- \rt[¹]{extend} -\xx{var} \·\xx{sub} {}] //
	\gld	{} off -to inland -at inland
			\rlap{\xx{zcnj}.\xx{csec}.run·\xx{sg}} {} {} {} {} {} {}
		\rlap{it.is} {}
		\rlap{3>3.\xx{g̱cnj}.\xx{pfv}.see} {} {} {} {} {}
		{} {} one’s head {} {}
			{} road within -thru {}
			\rlap{head.\xx{ncnj}.\xx{ext}·\xx{impfv}.extend} {} {} {} {} {} {} //
	\glft	‘Having run off up inland, he saw someone’s head sticking out through the road.’
		//
\endgl
\xe

The identification of \orth{Yudā′qedaqe cī′x̣awe} “far up when he got” in (\lastx) as \fm{yóot dáag̱i daaḵ ishéex áwé} starts with taking \orth{cī′x̣awe} as reflecting a consecutive aspect form followed by the usual focus particle \fm{áwé}.
The verb \fm{wujixeex} (motion) ‘s/he/it ran’ regularly shows coalescence of the \fm{d-sh-} and the stem onset as \fm{sh} when no stative \fm{i-} prefix is present.
The sequence \fm{d-sh-} without \fm{i-} requires the preceding epenthetic ‘peg vowel’ \fm{i} when \fm{d-sh-} ends up being word initial, so the form has to be \fm{ishéex} and thus \citeauthor{swanton:1909}’s \orth{cī′x̣} must include the preceding \orth{e}.
This leaves \orth{yudā′qedaq} remaining.
There is a motion derivation \fm{dáag̱i daaḵ} (\fm{∅}; \fm{-ch} repetitive) ‘further up from shore’ which has the directional \fm{dáaḵ} ‘inland’ repeated twice, the first time with the locative postposition allomorph \fm{-í}.
Then \orth{yu} is the last element to be interpreted.
This could be the \fm{yoo=} ‘alternating’ preverb but that does not make much sense in this context and would be out of order since it should follow \fm{daaḵ=}.
It could also be the \fm{yú} \~\ \fm{yóo} distal determiner, but there is no noun phrase nor any postposition to go along with it.
There is the motion derivation \fm{yóot} \~\ \fm{yóox̱} \~\ \fm{yóode} (\fm{∅}; \fm{-μμL} repetitive) ‘off somewhere’ which is semantically plausible, where the preverb is formed from the \fm{yú} determiner and the punctual postposition \fm{-t}, the pertingent postposition \fm{-x̱}, or the allative postposition \fm{-dé}.
\citeauthor{swanton:1909}’s \orth{yu} would then be a mistranscription of \fm{yóot} which is reasonable given the immediately following \fm{d} in \fm{dáag̱i}, so that the speaker said [\ipa{júːt.táː.qì}] which \citeauthor{swanton:1909} misheard as [\ipa{júː.táː.qì}].

The remainder of the sentence in (\lastx) is a good example of an unmarked complement clause.
The verb \fm{awsiteen} (\fm{g̱}; achievement) ‘s/he/it saw him/her/it’ here takes a full clause as its complement rather than a DP.
A number of verbs take complement clauses as arguments when the argument is an eventuality rather than an entity.
Here the protagonist does not simply see a person’s head; rather, he sees the state of a person’s head sticking out of the ground in the middle of the road.
Descriptions by \textcite{naish:1966} and \textcite{leer:1991} say that complement clauses are supposed to be marked like adjunct clauses using overt subordination with the \fm{-í} clause type suffix and suppression of the \fm{i-} stative prefix.
Elicitation research with fluent speakers and investigation of textual materials has shown that complement clauses can also be unmarked without \fm{-í} and suppression of \fm{i-}.
This is the case in (\lastx) with the verb \fm{shanaashóo} which is identifiably not \fm{shanashoowú} in \citeauthor{swanton:1909}’s transcription \orth{cᴀnacū′}.

\ex\label{ex:91-189-eyes-up}%
\exmn{275.8}%
\begingl
	\glpreamble	“Kî′ndê iwâ′q Kucaq!ē′tkᵘ.” //
	\glpreamble	«\!Kínde i waaḵ, Kushaḵʼéitʼkw.\!» //
	\gla	{} \llap{«\!}\rlap{Kínde} @ {} {}
		{} i waaḵ {}
		\rlap{Kushaḵʼéitʼkw.\!»} @ {} @ {} @ {} @ {} @ {} @ {} //
	\glb	{} kín -dé {}
		{} i waaḵ {}
		k- u- sh- \rt[¹]{ḵʼetʼ} -μμH -kw //
	\glc	{}[\pr{PP} up -\xx{all} {}]
		{}[\pr{DP} \xx{2sg·pss} eye {}]
		\xx{qual}- \xx{irr}- \xx{pej}- \rt[¹]{wobble} -\xx{var} -\xx{rep} //
	\gld	{} up -ward {}
		{} your eye {}
		\rlap{\xx{name}} {} {} {} {} {} {} //
	\glft	‘“Eyes up, Kushaḵʼéitʼkw.”’
		//
\endgl
\xe

The word that \citeauthor{swanton:1909} transcribes as \orth{Kucaq!ē′tkᵘ} is puzzling.
He seems to have considered it to be a name: it is capitalized and there is no translation either in the English translation or in the gloss.
The form \orth{Kucaqē′!tkᵘ} in the translation is slightly different from the form \orth{Kucaq!ē′tkᵘ} in the transcription, but this variance is plausibly a result of copying errors either during manuscript preparation or typesetting.
This name does not currently seem to be used in modern Tlingit so the interpretation given here is tentative.
The stem appears to be \fm{−ḵʼéitʼkw} based on the root \fm{\rt[¹]{ḵʼetʼ}} ‘wobble, topple, tip over’ with the repetitive suffix \fm{-kw}.
The \orth{ca} is taken to be the pejorative \fm{sh-} prefix, and then the preceding \orth{Ku} must be the \fm{k-} followed by \fm{u-}.
The \fm{u-} is likely the irrealis prefix, but the \fm{k-} could be any of
(i) the horizontal surface prefix from \fm{ká} ‘horizontal surface’,
(ii) the small round object qualifier,
(iii) the comparative prefix,
or (iv) the often meaningless \fm{k-} qualifier.

There are several possible alternative analyses of the name in (\lastx).
If it begins with \fm{ḵu} [\ipa{qʰʷù}] rather than \fm{ku} [\ipa{kʰʷù}] then the \fm{sha} could be the incorporated noun \fm{sha-} ‘head’ which would entail the absence of overt classifier prefixes.
The root could alternatively be \fm{\rt[²]{kʼet}} ‘tip out, dump out, empty out container’, but note that this root is bivalent rather than monovalent.
Another less likely root is \fm{\rt{xʼet}} \~\ \fm{\rt{xʼit}} ‘hunting’, but this is very rare and known only from a fixed construction \fm{asxʼeitdé woogoot} ‘s/he went hunting’ \parencites[f04/36]{leer:1973}[64]{leer:1978b}.
Other possible roots include \fm{\rt{xʼit}} ‘uproot’ (attested only with \fm{s-}), \fm{\rt[²]{kʼitʼ}} ‘eat up; pick berry’, \fm{\rt{ḵʼit}} ‘twist’ (cf.\ \fm{\rt{tiqʼ}}), \fm{\rt[²]{ḵʼit}} ‘shave’, \fm{\rt[²]{x̱ʼit}} ‘gnaw’, and \fm{\rt{ḵʼetlʼ}} ‘cut open, gut’.
The name is simply given untranslated as \fm{Kushaḵʼéitʼkw} for now pending further investigation.

\ex\label{ex:91-190-broke-back-that-head}%
\exmn{275.8}%
\begingl
	\glpreamble	Yutc!agᴀ′x̣dê wūdūwaʟ!ī′xe yᴀ′xawe wūnî′ yūqācā′ỵî. //
	\glpreamble	Yú chʼa gáxde wuduwalʼéexʼi yé yáx̱ áwé woonee, yú ḵaa shaaÿí. //
	\gla	{} {} Yú {} chʼa {} \rlap{gáxde} @ {} {}
			\rlap{wuduwalʼéexʼi} @ {} @ {} @ {} @ {} @ {} {}
			yé {} yáx̱ {} \rlap{áwé} @ {}
		\rlap{woonee,} @ {} @ {} @ {}
		{} yú ḵaa \rlap{shaaÿí.} @ {} {} //
	\glb	{} {} yú {} chʼa {} gáx -dé {}
			wu- du- i- \rt[²]{lʼixʼ} -μμH -i {}
			yé {} yáx̱ {} á -wé
		wu- i- \rt[¹]{niʰ} -μμL
		{} yú ḵaa shá -í {} //
	\glc	{}[\pr{PP} {}[\pr{DP} \xx{dist}
			{}[\pr{CP} just {}[\pr{PP} back -\xx{all} {}]
			\xx{pfv}- \xx{4h·s}- \xx{stv}- \rt[²]{break} -\xx{var} \xx{rel} {}]
			way {}] \xx{sim} {}] \xx{foc} -\xx{mdst}
		\xx{pfv}- \xx{stv}- \rt[¹]{happen} -\xx{var}
		{}[\pr{DP} \xx{dist} \xx{4h·pss} head -\xx{pss} {}] //
	\gld	{} {} that {} just {} back -to {}
			\rlap{\xx{ncnj}.\xx{pfv}.one.break} {} {} {} {} -that {}
			way {} like {} \rlap{it.is} {}
		\rlap{\xx{ncnj}.\xx{pfv}.happen} {} {} {}
		{} that one’s head {} {} //
	\glft	‘It was like that way that someone had broken it backward that it happened, that head.’
		//
\endgl
\xe

The phrase \fm{gáxde} ‘backward’ in (\lastx) is fairly uncommon and not well documented.
It also occurs in (\nextx) as \fm{gáxwde}.
It is listed by \citeauthor{leer:1978b} once in his stem list as “\fm{gax-} (w/ \fm{-de}) forward \$ \fm{ga;g} (also (EN) \fm{gaxw-de})” \parencite[56]{leer:1978b}.
It also appears in his noun database (the basis of \cite{leer:2001}) as \fm{a gáaxw} glossed as “back (of a knife or other blade); back edge, spine of it” attributed to Atlin speakers (presumably from \fm{Seidaayaa} Elizabeth Nyman), as \fm{gáxde} (Northern) or \fm{gáxwde} (Atlin) with the gloss “(sticking, protruding) out; out of the ground”, and finally as \fm{gáxwde} (Atlin) glossed “with one’s head back; bent backwards”.
The etymology of \fm{gáx} \~\ \fm{gáxw} \~\ \fm{gáaxw} is uncertain but as \textcite{leer:1978b} suggests it is plausibly related to \fm{gáak} ‘sticking out, protruding’ and \fm{gági} ‘out into the open’ (< \fm{gáak} + \fm{-í} locative).
The two semantic ranges of ‘spine, back edge of blade’ and ‘sticking out’ are not straightforwardly connected.
It is possible that there once were two separate forms \fm[*]{gaːxʷ} ‘spine, back edge’ meaning and \fm[*]{gaːx} ‘protruding’ which were conflated due to their similar pronunciations.
The \fm[*]{gaːx} ‘protruding’ would be tied to \fm{gáak} ‘protruding’ and \fm{gági} ‘out into the open’ from \fm[*]{gaːg} via an irregular alternation between [\ipa{x}] and [\ipa{k}].
The \fm[*]{gaːxʷ} could have alternatively been \fm[*]{gaˀxʷ} with a glottalized vowel which would be lost through tonogenesis thus setting the stage for conflation with \fm[*]{gaːx}, but there is no attestation from Tongass Tlingit and no other data to confirm this hypothesis.

\ex\label{ex:91-191-then-hacked-off}%
\exmn{275.9}%
\begingl
	\glpreamble	ʟe kāx ᴀse′ was!u′, gᴀ′x̣ᵘde yū′nᴀskīt. //
	\glpreamble	Tle kaax̱ aseiwasʼóow, gáxwde yoo anaskítʼgi. //
	\gla	Tle {} \rlap{kaax̱} @ {} {}
		\rlap{aseiwasʼóow,} @ {} @ {} @ {} @ {} @ {} +
		{} {} \rlap{gáxwde} @ {} {} 
			yoo @ \rlap{anaskítʼi.} @ {} @ {} @ {} @ {} @ {} {} //
	\glb	tle {} ká -dáx̱ {}
		a- se- wu- i- \rt[²]{sʼuʼw} -μμH
		{} {} gáxw -dé {}
			yoo= a- n- s- \rt[¹]{kitʼ} -μH -k -í {}  //
	\glc	then {}[\pr{PP} \xx{hsfc} -\xx{abl} {}]
		\xx{arg}- voice- \xx{pfv}- \xx{stv}- \rt[²]{chop} -\xx{var}
		{}[\pr{CP} {}[\pr{PP} back -\xx{all} {}]
			\xx{alt}= \xx{arg}- \xx{ncnj}- \xx{csv}- \rt[¹]{wedge}
				-\xx{var} -\xx{rep} -\xx{sub} {}] //
	\gld	then {} atop -from {}
		\rlap{3>3.\xx{ncnj}.\xx{pfv}.chop} {} {} {} {} {}
		{} {} back -to {}
			to·fro \rlap{3>3.\xx{ncnj}.\xx{prog}.pry} {} {} {} {} {} -while {}  //
	\glft	‘Then he hacked it off, as he was prying it backward back and forth.’
		//
\endgl
\xe

\citeauthor{swanton:1909} transcribes what appear to be two main clauses in (\lastx), one with a main verb \fm{aawasʼóow} ‘s/he chopped it’ and the next with a verb \fm{yoo anaskítʼ} ‘s/he pries it back and forth’.
But \citeauthor{swanton:1909} translates them as a single clause “After he had moved its head backward he cut it off” and his gloss “Then off he cut his head, backward when he moved” shows the same interpretation.
The sequence of events also seems to be out of order, with the chopping first and the prying backward second.
If these were separate main clauses then they would seem to be temporally reversed, but if they are a single clause with an adjunct then the adjunct would naturally be interpreted as preceding the main clause despite their reverse linear order.
This analysis is followed in (\lastx) with the addition of a subordinate clause suffix \fm{-í} on the second verb even though this is not present in \citeauthor{swanton:1909}’s transcription.
That same verb also appears to be missing the expected \fm{-k} repetitive suffix which normally occurs with the alternating preverb \fm{yoo=} ‘back and forth, to and fro’, so it is plausible that \citeauthor{swanton:1909}’s transcription is missing a final syllable and so the verb should be \fm{yoo anaskítʼgi} [\ipa{jùː.ʔà.nàs.kʰítʼ.kì}].

\ex\label{ex:91-192-skeletons-scattered}%
\exmn{275.9}%
\begingl
	\glpreamble	Qāxā′ge ayu′ yē udzîg̣ā′t. //
	\glpreamble	Ḵaa xáagi áyú yei wudzig̱áat. //
	\gla	{} Ḵaa \rlap{xáagi} @ {} {} \rlap{áyú} @ {}
		yei @ \rlap{wudzig̱áat.} @ {} @ {} @ {} @ {} @ {} //
	\glb	{} ḵaa xáak -í {} á -yú
		yei= wu- d- s- i- \rt[¹]{g̱aʼt} -μμH //
	\glc	{}[\pr{DP} \xx{4h·pss} skeleton -\xx{pss} {}] \xx{foc} -\xx{dist}
		down= \xx{pfv}- \xx{pasv}- \xx{csv}- \xx{stv}- \rt[¹]{scatter} -\xx{var} //
	\gld	{} ppl’s \rlap{skeletons} {} {} \rlap{it.is} {}
		down \rlap{\xx{g̱cnj}.\xx{pfv}.\xx{pasv}.make.scatter} {} {} {} {} {} //
	\glft	‘It was people’s skeletons that were scattered down.’
		//
\endgl
\xe

\ex\label{ex:91-193-pick-up-head}%
\exmn{275.10}%
\begingl
	\glpreamble	Doxᴀ′n awacā′t ᴀcaỵî′. //
	\glpreamble	Du x̱án aawasháat a shaaÿí. //
	\gla	{} Du \rlap{x̱án} @ {} {} \rlap{aawasháat} @ {} @ {} @ {} @ {}
		{} a \rlap{shaaÿí.} @ {} {} //
	\glb	{} du x̱án {} {} a- wu- i- \rt[²]{shaʼt} -μμH
		{} a shá -í {} //
	\glc	{}[\pr{PP} \xx{3h·pss} near -\xx{loc} {}]
		\xx{arg}- \xx{pfv}- \xx{stv}- \rt[²]{grab} -\xx{var}
		{}[\pr{DP} \xx{3n·pss} head -\xx{pss} {}] //
	\gld	{} his near -at {} \rlap{3>3.\xx{gcnj}.\xx{pfv}.grab} {} {} {} {}
		{} its \rlap{head} {} {} //
	\glft	‘He picked it up near him, its head.’
		//
\endgl
\xe

\ex\label{ex:91-194-beach-again-throw-inside}%
\exmn{275.10}%
\begingl
	\glpreamble	Î′qdê ts!u nēł ā′wagîq!. //
	\glpreamble	Íḵde tsu, neil aawag̱íxʼ. //
	\gla	{} \rlap{Íḵde} @ {} {} tsu,
		{} \rlap{neil} @ {} {} \rlap{aawag̱íxʼ.} @ {} @ {} @ {} @ {} //
	\glb	{} íḵ -dé {} tsu
		{} neil -t {} a- wu- i- \rt[²]{g̱ixʼ} -μH //
	\glc	{}[\pr{PP} beach -\xx{all} {}] again
		{}[\pr{PP} inside -\xx{pnct} {}]
		\xx{arg}- \xx{pfv}- \xx{stv}- \rt[²]{throw·ina n} -\xx{var} //
	\gld	{} beach -to {} again
		{} inside -to {} \rlap{3>3.\xx{zcnj}.\xx{pfv}.throw·inanimate} {} {} {} {} //
	\glft	‘To the beach again, he threw it inside.’
		//
\endgl
\xe

\ex\label{ex:91-195-toast-with-crap}%
\exmn{275.11}%
\begingl
	\glpreamble	Hā′ʟ!i tî′nawe ayā′wat!us. //
	\glpreamble	Háatlʼi tin áwé ayaawatʼúsʼ. //
	\gla	{} {} \rlap{Háatlʼi} @ {} @ {} {} tin {} \rlap{áwé} @ {}
		\rlap{ayaawatʼúsʼ.} @ {} @ {} @ {} @ {} @ {} //
	\glb	{} {} \rt[²]{hatlʼ} -μμH -í {} tin {} á -wé
		a- ÿ- wu- i- \rt[²]{tʼusʼ} -μH //
	\glc	{}[\pr{PP} {}[\pr{NP} \rt[²]{crap} -\xx{var} -\xx{nmz} {}] \xx{instr} {}]
		\xx{foc} -\xx{mdst}
		\xx{arg}- face- \xx{pfv}- \xx{stv}- \rt[²]{toast} -\xx{var} //
	\gld	{} {} \rlap{crap} {} {} {} with {} \rlap{it.is} {}
		\rlap{3>3.face.\xx{zcnj}.\xx{pfv}.toast} {} {} {} {} {} //
	\glft	‘It was with crap that they toasted its face.’
		//
\endgl
\xe

\ex\label{ex:91-196-do-however-they-feel}%
\exmn{275.11}%
\begingl
	\glpreamble	Āde′ adjī′ỵit hᴀs ctᴀnū′guỵa yᴀx ayū′ hᴀs adā′na. //
	\glpreamble	Aadé a jeeÿeet has sh danoogu ÿé yáx̱ áyú has adaané. //
	\gla	{} {} {} {} \rlap{Aadé} @ {} {}
			{} a \rlap{jeeÿeet} @ {} @ {} {}
			has @ sh @ \rlap{danoogu} @ {} @ {} @ {} {} +
			ÿé {} yáx̱ {} \rlap{áyú} @ {}
		has @ \rlap{adaané.} @ {} @ {} @ {} //
	\glb	{} {} {} {} á -dé {}
			{} a ji- ÿee -t {}
			has= sh= d- \rt[²]{nuͥk} -μμL -i {}
			ÿé {} yáx̱ {} á -yú
		has= a- daa- \rt[²]{ne} -μH //
	\glc	{}[\pr{PP} {}[\pr{DP} {}[\pr{CP} {}[\pr{PP} \xx{3n} -\xx{all} {}]
			{}[\pr{PP} \xx{3n·pss} hand- below -\xx{pnct} {}]
			\xx{plh}= \xx{rflx·o}= \xx{mid}- \rt[²]{feel} -\xx{var} -\xx{rel} {}]
			way {}] \xx{sim} {}] \xx{foc} -\xx{dist}
		\xx{plh}= \xx{arg}- around- \rt[²]{work} -\xx{var} //
	\gld	{} {} {} {} it -to {}
			{} its \rlap{burden} {} -at {}
			they self \rlap{\xx{zcnj}.\xx{impfv}.feel} {} {} -that {}
			way {} like {} \rlap{it.is} {}
		they \rlap{3>3.about.\xx{ncnj}.\xx{impfv}.work} {} {} {} //
	\glft	‘It is like the way that they feel themselves under its burden that they are doing to it.’
		//
\endgl
\xe

\ex\label{ex:91-197-after-make-bownarrow-go-up}%
\exmn{275.12}%
\begingl
	\glpreamble	ᴀtxawe′ tc!uʟe′ tcū′nēτ ke ᴀłᴀ′ttc. //
	\glpreamble	Atx̱ áwé chʼu tle chooneit kei al.átch. //
	\gla	{} \rlap{Atx̱} @ {} {} \rlap{áwé} @ {}
		chʼu tle
		{} {} \rlap{chooneit} @ {} @ {} {} {} {} +
		kei @ \rlap{al.átch.} @ {} @ {} @ {} @ {} //
	\glb	{} á -dáx̱ {} á -wé
		chʼu tle
		{} {} \rt[¹]{chun} -μμL -i {} át {}
		kei= a- l- \rt[¹]{.at} -μH -ch //
	\glc	{}[\pr{PP} \xx{3n} -\xx{abl} {}] \xx{foc} -\xx{mdst}
		just then
		{}[\pr{DP} {}[\pr{CP} \rt[¹]{wound} -\xx{var} -\xx{rel} {}] thing {}]
		up= \xx{arg}- \xx{csv}- \rt[¹]{go·\xx{pl}} -\xx{var} -\xx{rep} //
	\gld	{} then -after {} \rlap{it.is} {}
		just then
		{} {} \rlap{bow·and·arrow} {} {} {} {} {}
		up \rlap{3>3.\xx{zcnj}.\xx{impfv}.handle·\xx{pl}.\xx{rep}} {} {} {} {} //
	\glft	‘After that, then he keeps making his arrows go up.’
		//
\endgl
\xe

\ex\label{ex:91-198-always-bringing-home}%
\exmn{275.12}%
\begingl
	\glpreamble	Łdakᴀ′t-ᴀt duʟa′-hᴀs q!ēs nēłde′ ỵaᴀkāgādjᴀ′łtc. //
	\glpreamble	Ldakát át du tláa hás x̱ʼéis neildé ÿaa aka̬gajélch. //
	\gla	{} Ldakát át {}
		{} du tláa @ \•hás \rlap{x̱ʼéis} @ {} {}
		{} \rlap{neildé} @ {} {}
		ÿaa @ \rlap{aka̬gajélch.} @ {} @ {} @ {} @ {} @ {} //
	\glb	{} ldakát át {}
		{} du tláa =hás x̱ʼé -ÿís {}
		{} neil -dé {}
		ÿaa= a- k- g- \rt[²]{jel} -μH -ch //
	\glc	{}[\pr{DP} all thing {}]
		{}[\pr{PP} \xx{3h·pss} mother =\xx{plh} mouth -\xx{ben} {}]
		{}[\pr{PP} home -\xx{all} {}]
		along= \xx{arg}- \xx{qual}- \xx{gcnj}- \rt[²]{load} -\xx{var} -\xx{rep} //
	\gld	{} every thing {}
		{} his mother =s mouth -for {}
		{} home -to {}
		along \rlap{3>3.\xx{gcnj}.\xx{hab}.carry·load} {} {} {} {} {} //
	\glft	‘He was always bringing everything home for his mothers to eat.’
		//
\endgl
\xe

\ex\label{ex:91-199-heron-son-act-so-super-help}%
\exmn{275.13}%
\begingl
	\glpreamble	Łᴀq! duỵī′dawe yē quwanū′k ag̣ā′ wūsu′ yuca′. //
	\glpreamble	Láx̱ʼ du ÿéet áwé yéi ḵuwanóok, aag̱áa woosoo yú sháa. //
	\gla	{} Láx̱ʼ du ÿéet {} \rlap{áwé} @ {}
		yéi @ \rlap{ḵuwanóok,} @ {} @ {} @ {} +
		{} {} \rlap{aag̱áa} @ {} {}
			\rlap{woosoo} @ {} @ {} @ {} @ {}
			{} yú sháa. {} {} //
	\glb	{} láx̱ʼ du ÿéet {} á -wé
		yéi= ḵu- i- \rt[²]{nuͥk} -μμH
		{} {} á -g̱áa {}
			wu- i- \rt[¹]{suʰ} -μμL {}
			{} yú sháaʷ {} {} //
	\glc	{}[\pr{DP} heron \xx{3h·pss} son {}] \xx{foc} -\xx{mdst}
		thus= \xx{areal}- \xx{stv}- \rt[²]{act} -\xx{var}
		{}[\pr{CP} {}[\pr{PP} \xx{3n} -\xx{ades} {}]
			\xx{pfv}- \xx{stv}- \rt[¹]{sup·help} -\xx{var} \·\xx{sub}
			{}[\pr{DP} \xx{dist} woman {}] {}] //
	\gld	{} heron his son {} \rlap{it.is} {}
		thus \rlap{\xx{ncnj}.\xx{impfv}.be.acting} {} {} {}
		{} {} it -for {}
			\rlap{\xx{ncnj}.\xx{pfv}.super·help} {} {} {} {}
			{} that woman {} {}  //
	\glft	‘It is the heron’s son that is acting so, having supernaturally helped that woman.’
		//
\endgl
\xe

\ex\label{ex:91-200-asked-mother}%
\exmn{275.14}%
\begingl
	\glpreamble	Wananī′sa q!ē′wawūs duʟa′, //
	\glpreamble	Wáa nanée sá ax̱ʼeiwawóosʼ du tláa //
	\gla	{} Wáa \rlap{nanée} @ {} @ {} @ {} {} sá
		\rlap{ax̱ʼeiwawóosʼ} @ {} @ {} @ {} @ {} @ {}
		{} du tláa {} //
	\glb	{} wáa n- \rt[¹]{ni} -μμH {} {} sá
		a- x̱ʼe- wu- i- \rt[²]{wuͣsʼ} -μμH
		{} du tláa {} //
	\glc	{}[\pr{CP} how \xx{ncnj}- \rt[¹]{happen} -\xx{var} \·\xx{sub} {}] \xx{q}
		\xx{arg}- mouth- \xx{pfv}- \xx{stv}- \rt[²]{ask} -\xx{var}
		{}[\pr{DP} \xx{3h·pss} mother {}] //
	\gld	{} how \rlap{\xx{csec}.happen} {} {} \·while {} ever
		\rlap{3>3.\xx{ncnj}.\xx{pfv}.ask} {} {} {} {} {}
		{} his mother {} //
	\glft	‘At some point he asked his mother’
		//
\endgl
\xe

\ex\label{ex:91-201-where-uncles-went}%
\exmn{275.14}%
\begingl
	\glpreamble	“Gū′nᴀx a′de wuqoxō′ sa ᴀxkā′k-hᴀs ʟē′łxax ā′woqox.” //
	\glpreamble	«\!Goonáx̱ aadé wooḵoox̱u ÿé sá ax̱ káak hás tléil x̱áax̱ awuḵoox̱?\!» //
	\gla	{} {} {} {} \llap{«\!}\rlap{Goonáx̱} @ {} {}
			{} \rlap{aadé} @ {} {}
			\rlap{wooḵoox̱u} @ {} @ {} @ {} @ {} {} ÿé {} sá {}
		{} ax̱ káak @ \•hás {}
		tléil {} \rlap{x̱áax̱} @ {} {}
		\rlap{awuḵoox̱?\!»} @ {} @ {} @ {} @ {} //
	\glb	{} {} {} {} goo -náx̱ {}
			{} á -dé {}
			wu- i- \rt[¹]{ḵux̱} -μμL -i {} ÿé {} sá {}
		{} ax̱ káak =hás {}
		tléil {} x̱á -x̱ {}
		a- u- wu- \rt[¹]{ḵux̱} -μμL //
	\glc	{}[\pr{QP} {}[\pr{DP} {}[\pr{CP} {}[\pr{PP} where -\xx{perl} {}]
			{}[\pr{PP} \xx{3n} -\xx{all} {}]
			\xx{pfv}- \xx{stv}- \rt[¹]{go·boat} -\xx{var} -\xx{rel} {}]
			way {}] \xx{q} {}]
		{}[\pr{DP} \xx{1sg·pss} mat·uncle =\xx{plh} {}]
		\xx{neg} {}[\pr{PP} \xx{1sg} -\xx{pert} {}]
		\xx{4h·s}- \xx{irr}- \xx{pfv}- \rt[¹]{go·boat} -\xx{var} //
	\gld	{} {} {} {} where -thru {}
			{} there -to {}
			\rlap{\xx{ncnj}.\xx{pfv}.go·boat} {} {} {} -that {} place {} ? {}
		{} my uncle =s {}
		not {} me -at {}
		\rlap{\xx{4h·s}.\xx{ncnj}.\xx{pfv}.go·boat} {} {} {} {} //
	\glft	‘“Through which way did they go by boat that my uncles did not come to me?”’
		//
\endgl
\xe

\FIXME{Discuss the syntactic difficulties with this wh-question.
The wh-questioned relative clause seems to be an adjunct clause.}

\ex\label{ex:91-202-thattaway}%
\exmn{276.1}%
\begingl
	\glpreamble	“Wē′de wuqoxō′awe ỵî′tk!ī,” //
	\glpreamble	«\!Wéide wuḵoox̱ú áwé, ÿítkʼi\!» //
	\gla	{} \llap{«\!}\rlap{Wéide} @ {} {}
		\rlap{wuḵoox̱ú} @ {} @ {} @ {}
		\rlap{áwé} @ {}
		\rlap{ÿítkʼi\!»} @ {} //
	\glb	{} wé -dé {}
		wu- \rt[¹]{ḵux̱} -μμL -í
		á -wé
		ÿéet -kʼ //
	\glc	{}[\pr{PP} \xx{mdst} -\xx{all} {}]
		\xx{pfv}- \rt[¹]{go·boat} -\xx{var} -\xx{sub}
		\xx{foc} -\xx{mdst}
		son -\xx{dim} //
	\gld	{} there -to {}
		\rlap{\xx{ncnj}.\xx{pfv}.go·boat} {} {} {} //
	\glft	‘“It is that they went that way, little son”’
		//
\endgl
\xe

\FIXME{Note insubordinate clause.}

\ex\label{ex:91-203-she-said}%
\exmn{276.2}%
\begingl
	\glpreamble	yū′aỵaosîqa. //
	\glpreamble	yóo aÿawsiḵaa. //
	\gla	yóo @ \rlap{aÿawsiḵaa.} @ {} @ {} @ {} @ {} @ {} @ {} //
	\glb	yóo= a- ÿ- wu- s- i- \rt[¹]{ḵa} -μμL //
	\glc	\xx{quot}= \xx{arg}- \xx{qual}- \xx{pfv}- \xx{csv}- \xx{stv}-
			\rt[¹]{say} -\xx{var} //
	\gld	so \rlap{3>3.\xx{ncnj}.\xx{pfv}.say·to} {} {} {} {} {} {} //
	\glft	‘she said to him.’
		//
\endgl
\xe

\FIXME{Note \fm{a-} rather than \fm{ash=}.}

\ex\label{ex:91-204-direction-go-with-bownarrows}%
\exmn{276.2}%
\begingl
	\glpreamble	ᴀniyā′deawe wūgū′t tcūnē′t tîn. //
	\glpreamble	A niyaadé áwé woogoot chooneit tin. //
	\gla	{} A \rlap{niyaadé} @ {} {} \rlap{áwé} @ {}
		\rlap{woogoot} @ {} @ {} @ {} +
		{} {} {} \rlap{chooneit} @ {} @ {} {} {} {} tin. {} //
	\glb	{} a niÿaa -dé {} á -wé
		wu- i- \rt[¹]{gut} -μμL
		{} {} {} \rt[¹]{chun} -μμL -i {} át {} tin {} //
	\glc	{}[\pr{PP} \xx{3n·pss} dir’n -\xx{all} {}] \xx{foc} -\xx{mdst}
		\xx{pfv}- \xx{stv}- \rt[¹]{go·\xx{sg}} -\xx{var}
		{}[\pr{PP} {}[\pr{DP} {}[\pr{CP} \rt[¹]{wound} -\xx{var} -\xx{rel} {}] thing {}]
			\xx{instr} {}] //
	\gld	{} its dir’n -to {} \rlap{it.is} {}
		\rlap{\xx{ncnj}.\xx{pfv}.go·\xx{sg}} {} {} {}
		{} {} {} \rlap{bow·and·arrow} {} {} {} {} {} with {} //
	\glft	‘It was in its direction that he went with bow and arrows.’
		//
\endgl
\xe

\ex\label{ex:91-205-out-above-it-eye-mouth-hole}%
\exmn{276.2}%
\begingl
	\glpreamble	Akînā′ dak uwagu′t awᴀ′q-qa′owułī nāq. //
	\glpreamble	A kináa daak uwagút a waḵx̱ʼawoolí, náaḵw. //
	\gla	{} A \rlap{kináa} @ {} {}
		daak @ \rlap{uwagút} @ {} @ {} @ {}
		{} a \rlap{waḵx̱ʼawoolí,} @ {} @ {} @ {} {} +
		{} náaḵw. {} //
	\glb	{} a kináa {} {}
		daak= u- i- \rt[¹]{gut} -μH
		{} a waaḵ- x̱ʼé- wool -í {}
		{} náaḵw {} //
	\glc	{}[\pr{PP} \xx{3n·pss} above -\xx{loc} {}]
		seaward= \xx{zpfv}- \xx{stv}- \rt[¹]{go·\xx{sg}} -\xx{var}
		{}[\pr{DP} \xx{3n·pss} eye- mouth- hole -\xx{pss} {}]
		{}[\pr{DP} octopus {}] //
	\gld	{} its above -at {}
		out \rlap{\xx{zcnj}.\xx{pfv}.go·\xx{sg}} {} {} {}
		{} its eye- \rlap{door} {} {} {}
		{} octopus {} //
	\glft	‘He went out above it, its eye hole, a devilfish.’
		//
\endgl
\xe

\ex\label{ex:91-206-}%
\exmn{276.3}%
\begingl
	\glpreamble	Kî′ndaq!es!tū′nawe a ʟeye′ aq!îs!tū′t aoʟ̣îg̣ê′n. //
	\glpreamble	Kínde xʼisʼtóo \{n\} áwé \{a\} tliyéi a xʼisʼtóot awdlig̱én. //
	\gla	 //
	\glb	 //
	\glc	 //
	\gld	 //
	\glft	‘’\newline
		“When he was sitting ready for action then right into it he looked.”\newline
		“As he was sitting there ready for action he looked right down into it.”
		//
\endgl
\xe

\FIXME{Discuss \fm{xʼéesʼ} and \fm{xʼisʼshantóo}, as well as \fm{shantuxʼéesʼ} and \fm{shantuxʼisʼkaxwéix̱}.
Cf.\ (\ref{ex:91-208-got-back-threw-rock-out}) and (\ref{ex:91-213-cut-ink-sac})}

\ex\label{ex:91-207-ran-back-for-shirt}%
\exmn{276.4}%
\begingl
	\glpreamble	Ag̣ā′ qox wudjîx̣ī′x̣ yū′-aołîsî′nî-k!udᴀ′s! hīntaỵī′cî k!udᴀ′s!. //
	\glpreamble	Aag̱áa ḵux̱ wujixeex yú awlisíni kʼoodásʼ, héen taaÿéeshi kʼoodásʼ. //
	\gla	{} \rlap{Aag̱áa} @ {} {}
		ḵux̱ @ \rlap{wujixeex} @ {} @ {} @ {} @ {} @ {}
		{} yú {} \rlap{awlisíni} @ {} @ {} @ {} @ {} @ {} @ {} @ {} kʼoodásʼ, {}
		{} héen \rlap{taaÿéeshi} @ {} @ {} @ {} kʼoodásʼ. {} //
	\glb	{} á -g̱áa {}
		ḵúx̱= wu- d- sh- i- \rt[¹]{xix} -μμL
		{} yú {} a- wu- l- i- \rt[¹]{sin} -μH -i {} kʼoodásʼ {}
		{} héen táaᵏ= \rt[²]{ÿish} -μμH -i kʼoodásʼ {} //
	\glc	{}[\pr{PP} \xx{3n} -\xx{ades} {}]
		\xx{rev}= \xx{pfv}- \xx{mid}- \xx{pej}- \xx{stv}- \rt[¹]{fall·\xx{sg}} -\xx{var}
		{}[\pr{DP} \xx{dist}
			{}[\pr{CP} \xx{arg}- \xx{pfv}- \xx{csv}- \xx{stv}-
				\rt[¹]{hidden} -\xx{var} -\xx{rel} {}] tunic {}]
		{}[\pr{DP} water bottom= \rt[²]{pull} -\xx{var} -\xx{pss} tunic {}] //
	\gld	{} it -for {}
		back \rlap{\xx{ncnj}.\xx{pfv}.run·\xx{sg}} {} {} {} {} {}
		{} that {} \rlap{3>3.\xx{zcnj}.\xx{pfv}.hide} {} {} {} {} {} -that {} shirt {}
		{} water bottom \rlap{puller} {} -of shirt {} //
	\glft	‘He ran back for it, that shirt that he hid, the \fm{héen taaÿéeshi} shirt.’
		//
\endgl
\xe

\ex\label{ex:91-208-got-back-threw-rock-out}%
\exmn{276.5}%
\begingl
	\glpreamble	Āqo′x wudagude′awe ʟe tᴀ adê′ dāq awagî′q!. Aq!î′ts cᴀntū′dî //
	\glpreamble	Áa ḵux̱ wudagoodí áwé tle té aadé daaḵ aawag̱íxʼ, a xʼisʼshantóode. //
	\gla	{} {} \rlap{Áa} @ {} {}
			ḵux̱ @ \rlap{wudagoodí} @ {} @ {} @ {} @ {} {}
		\rlap{áwé} @ {} +
		tle {} té {}
		{} \rlap{aadé} @ {} {}
		daaḵ @ \rlap{aawag̱íxʼ,} @ {} @ {} @ {} @ {} +
		{} a \rlap{xʼisʼshantóode.} @ {} @ {} @ {} {} //
	\glb	{} {} á -μ {}
			ḵúx̱= wu- d- \rt[¹]{gut} -μμL -í {}
		á -wé
		tle {} té {}
		{} á -dé {}
		dáaḵ= a- wu- i- \rt[²]{g̱ixʼ} -μH
		{} a xʼéesʼ- sháⁿ- tú -dé {} //
	\glc	{}[\pr{CP} {}[\pr{PP} \xx{3n} -\xx{loc} {}]
			\xx{rev}= \xx{pfv}- \xx{mid}-
				\rt[¹]{go·\xx{sg}} -\xx{var} -\xx{sub} {}]
		\xx{foc} -\xx{mdst}
		then {}[\pr{DP} stone {}]
		{}[\pr{PP} \xx{3n} -\xx{all} {}]
		\xx{admar}= \xx{arg}- \xx{pfv}- \xx{stv}- \rt[²]{throw·inan} -\xx{var}
		{}[\pr{PP} \xx{3n·pss} tangle- head- inside -\xx{all} {}] //
	\gld	{} {} there -at {}
			back \rlap{\xx{ncnj}.\xx{pfv}.go·\xx{sg}} {} {} {} -when {}
		\rlap{it.is} {}
		then {} stone {}
		{} it -to {}
		out \rlap{3>3.\xx{zcnj}.\xx{pfv}.throw·inan} {} {} {} {}
		{} its tangle- head- inside -to {} //
	\glft	‘When he got back there then he threw a rock out at it, into its tangle.’
		//
\endgl
\xe

\citeauthor{swanton:1909} put the phrase \orth{Aq!î′ts cᴀntū′dî} \fm{a xʼisʼshantóode} ‘into its tangle’ in (\lastx) as part of the beginning of (\nextx).
But this phrase is ungrammatical in the context of (\nextx) because the main verb describes putting on clothing and not any kind of motion that could be construed as being directed toward the devilfish.

\ex\label{ex:91-209-put-on-shirt}%
\exmn{276.6}%
\begingl
	\glpreamble	kᴀx aodîgē′q! yū′hīntaỵī′cî k!udᴀ′s!. //
	\glpreamble	Káx̱ awdig̱éixʼ yú héen taaÿéeshi kʼoodásʼ. //
	\gla	{} {} \rlap{Káx̱} @ {} {}
		\rlap{awdig̱éixʼ} @ {} @ {} @ {} @ {} @ {} +
		{} yú héen \rlap{taaÿéeshi} @ {} @ {} @ {} kʼoodásʼ. {} //
	\glb	{} {} ká -x̱ {}
		a- wu- d- i- \rt[²]{g̱ixʼ} -μμH
		{} yú héen táaᵏ= \rt[²]{ÿish} -μμH -i kʼoodásʼ {} //
	\glc	{}[\pr{PP} \xx{rflx·pss} \xx{hsfc} -\xx{pert} {}]
		\xx{arg}- \xx{pfv}- \xx{mid}- \xx{stv}- \rt[²]{throw·inan} -\xx{var}
		{}[\pr{DP} \xx{dist} water bottom= \rt[²]{pull} -\xx{var} -\xx{pss} tunic {}] //
	\gld	{} self’s atop -at {}
		\rlap{3>3.\xx{g̱cnj}.\xx{pfv}.put·on} {} {} {} {} {}
		{} that water bottom \rlap{puller} {} -of shirt {} //
	\glft	‘He put it on, the \fm{héen taaÿéeshi} shirt.’
		//
\endgl
\xe

\ex\label{ex:91-210-ran-inside-tangle}%
\exmn{276.6}%
\begingl
	\glpreamble	Tc!uʟe′ aq!îts cᴀntū′ wudîx̣ī′x̣. //
	\glpreamble	Chʼu tle a xʼisʼshantóo wujixeex. //
	\gla	Chʼu tle {} a \rlap{xʼisʼshantóo} @ {} @ {} @ {} {}
		\rlap{wujixeex.} @ {} @ {} @ {} @ {} @ {} //
	\glb	chʼu tle {} a xʼéesʼ- sháⁿ- tú -μ {}
		wu- d- sh- i- \rt[¹]{xix} -μμL //
	\glc	just then {}[\pr{PP} \xx{3n·pss} tangle- head- inside -\xx{loc} {}]
		\xx{pfv}- \xx{mid}- \xx{pej}- \xx{stv}- \rt[¹]{fall·\xx{sg}} -\xx{var} //
	\gld	just then {} its tangle- head- inside -at {}
		\rlap{\xx{ncnj}.\xx{pfv}.run·\xx{sg}} {} {} {} {} {} //
	\glft	‘Then he ran inside its tangle.’
		//
\endgl
\xe

\ex\label{ex:91-211-bounce-around-tangle}%
\exmn{276.7}%
\begingl
	\glpreamble	Atudawe′ yawak!u′t ʟ!adē′n qa hē′de yunā′q q!îts cᴀntū′t. //
	\glpreamble	A tóot áwé yaawakʼút tlʼaadéin ḵa héide yú náaḵw xʼisʼshantóot. //
	\gla	{} A \rlap{tóot} @ {} {} \rlap{áwé} @ {} 
		\rlap{yaawakʼút} @ {} @ {} @ {} @ {} +
		{} \rlap{tlʼaadéin} @ {} {}
		ḵa  {} \rlap{héide} @ {} {}
		{} yú náaḵw \rlap{xʼisʼshantóot.} @ {} @ {} @ {} {} //
	\glb	{} a tú -t {} á -wé
		ÿ- wu- i- \rt[¹]{kʼut} -μH
		{} tlʼaa -déin {}
		ḵa {} hé -dé {}
		{} yú náaḵw xʼéesʼ- sháⁿ- tú -t {} //
	\glc	{}[\pr{PP} \xx{3n·pss} inside -\xx{pnct} {}] \xx{foc} -\xx{mdst}
		\xx{qual}- \xx{pfv}- \xx{stv}- \rt[¹]{bounce} -\xx{var}
		{}[\pr{AdvP} side? -\xx{adv} {}]
		and {}[\pr{PP} \xx{mprx} -\xx{all} {}]
		{}[\pr{PP} \xx{dist} octopus tangle- head- inside -\xx{pnct} {}] //
	\gld	{} its inside -to {} \rlap{it.is} {}
		\rlap{off.\xx{zcnj}.\xx{pfv}.bounce} {} {} {} {}
		{} \rlap{crosswise} {} {}
		and {} here -to {}
		{} that octopus tangle- head- inside -to {} //
	\glft	‘It was inside it that he bounced, this way and that, in that devilfish’s tangle.’
		//
\endgl
\xe

The \fm{tlʼaa} in the adverb \fm{tlʼaadéin} ‘crosswise’ has an unclear meaning; it is glossed ‘side’ here only to suggest the interpretation of the whole adverb.
This \fm{tlʼaa} element is attested only in this adverb and in the adverb \fm{tlʼag̱áa} ‘lots, much, quite a bit’ \parencite[08/217]{leer:1973}.
\citeauthor{leer:1978b} suggests a connection between \fm{tlʼag̱áa} and \fm{chʼa} ‘just, only’ with \fm{-g̱áa} likely the adessive postposition \parencite[35]{leer:1978b}.
He gives \fm{tlʼaadéin} as a separate entry in this context however, so it is also possible that the two are unrelated.
There are no obvious connections to any roots or other elements in the language.

\ex\label{ex:91-212-short-bits-cut-edge}%
\exmn{276.7}%
\begingl
	\glpreamble	Tc!ayē′ kᵘdayā′ʟ!awe ye akᴀ′nax̣ᴀc duwᴀ′ntc. //
	\glpreamble	Chʼa yéi kwdayáatlʼ áwé yei akanaxásh du wánch. //
	\gla	Chʼa {} yéi @ \rlap{kwdayáatlʼ} @ {} @ {} @ {} @ {} @ {} {}
		\rlap{áwé} @ {}
		yei @ \rlap{akanaxásh} @ {} @ {} @ {} @ {}
		{} du \rlap{wánch.} @ {} {} //
	\glb	chʼa {} yéi= k- u- d- \rt[¹]{ÿatlʼ} -μμH {} {}
		á -wé
		yei= a- k- n- \rt[²]{xash} -μH
		{} du wán -ch {} //
	\glc	just {}[\pr{NP} thus= \xx{cmpv}- \xx{irr}- \xx{mid}-
			\rt[¹]{short} -\xx{var} -\xx{nmz} {}]
		\xx{foc} -\xx{mdst}
		down= \xx{arg}- \xx{qual}- \xx{ncnj}- \rt[²]{cut} -\xx{var}
		{}[\pr{DP} \xx{3h·pss} edge -\xx{erg} {}] //
	\gld	just {} thus \rlap{\xx{cmpv}.\xx{impfv}.short·\xx{pl}} {} {} {} {} -ness {}
		\rlap{it.is} {}
		down \rlap{3>3.\xx{g̱cnj}.\xx{prog}.cut} {} {} {} {}
		{} his edge {} {} //
	\glft	‘It is just into short pieces that it is cutting it up, his edge.’
		//
\endgl
\xe

\FIXME{Discuss nominalization.}

\FIXME{Discuss \fm{g̱}-conjugation for \fm{\rt[²]{xash}} which is normally \fm{n}-conjugation.}

\ex\label{ex:91-213-cut-ink-sac}%
\exmn{276.8}%
\begingl
	\glpreamble	Wananī′sawe aq!î′ts cᴀ′ntu kaxwē′xê aka′ ka′ołîx̣ā′c. //
	\glpreamble	Wáa nanée sáwé a xʼisʼshantukaxwéix̱i, a ká akawlixaash. //
	\gla	{} Wáa \rlap{nanée} @ {} @ {} @ {} {}
		\rlap{sáwé} @ {} @ {} +
		{} a \rlap{xʼisshantukaxwéix̱i,} @ {} @ {} {} {} @ {} @ {} @ {} {} {} {} +
		{} a ká {}
		\rlap{akawlixaash.} @ {} @ {} @ {} @ {} @ {} @ {} //
	\glb	{} wáa n- \rt[¹]{ni} -μμH {} {} 
		s= á -wé
		{} a xʼéesʼ- sháⁿ- tú- {} ká- \rt[¹]{xu} -eμH -x̱ {} -í {}
		{} a ká {}
		a- k- wu- l- i- \rt[²]{xash} -μμL //
	\glc	{}[\pr{CP} how \xx{ncnj}- \rt[¹]{happen} -\xx{var} \·\xx{sub} {}]
		\xx{q}= \xx{foc} -\xx{mdst}
		{}[\pr{DP} \xx{3n·pss} tangle- head- inside-
			{}[\pr{NP} \xx{hsfc}- \rt[¹]{steam} -\xx{var} -\xx{rep} {}]
			-\xx{pss} {}]
		{}[\pr{DP} \xx{3n·pss} \xx{hsfc} {}]
		\xx{arg}- \xx{qual}- \xx{pfv}- \xx{xtn}- \xx{stv}- \rt[²]{cut} -\xx{var} //
	\gld	{} how \rlap{\xx{csec}.happen} {} {} \·while {}
		ever\· \rlap{it.is} {}
		{} its tangle- head- inside {} \rlap{cranberry} {} {} {} {} -of {}
		{} its atop {}
		\rlap{3>3.\xx{ncnj}.\xx{pfv}.cut} {} {} {} {} {} {} //
	\glft	‘At some point, its ink sac, he cut the top of it.’
		//
\endgl
\xe

\ex\label{ex:91-214-killed-swam-out}%
\exmn{276.9}%
\begingl
	\glpreamble	Atcā′g̣awe awᴀqa′owu nᴀx dāq uwaq!ᴀ′q. //
	\glpreamble	Ajáaḵ áwé a waḵx̱ʼawoolínáx̱ daaḵ uwaxʼák. //
	\gla	{} \rlap{Ajáaḵ} @ {} @ {} @ {} @ {} {} \rlap{áwé} @ {}
		{} a \rlap{waḵx̱ʼawoolínáx̱} @ {} @ {} @ {} @ {} {}
		daaḵ @ \rlap{uwaxʼák.} @ {} @ {} @ {} //
	\glb	{} a- {} \rt[²]{jaḵ} -μμH {} {} á -wé
		{} a waaḵ- x̱ʼé- wool -í -náx̱ {}
		dáaḵ= u- i- \rt[¹]{xʼak} -μH //
	\glc	{}[\pr{CP} \xx{arg}- \xx{zcnj}\· \rt[²]{kill} -\xx{var} \·\xx{sub} {}]
		\xx{foc} -\xx{mdst}
		{}[\pr{PP} \xx{3n·pss} eye- mouth- hole -\xx{pss} -\xx{perl} {}]
		seaward= \xx{zpfv}- \xx{stv}- \rt[¹]{fish·swim} -\xx{var} //
	\gld	{} \rlap{3>3.\xx{zcnj}.\xx{csec}.kill} {} {} {} {} {} \rlap{it.is} {}
		{} its eye- \rlap{door} {} {} -thru {}
		out \rlap{\xx{zcnj}.\xx{pfv}.fish·swim} {} {} {} //
	\glft	‘Having killed it, he swam out through its eye hole.’
		//
\endgl
\xe

\citeauthor{swanton:1909}’s transcription \orth{awᴀqa′owu nᴀx} in (\lastx) is somewhat opaque but his gloss “the hole” clarifies the form.
The sentence in (\ref{ex:91-205-out-above-it-eye-mouth-hole}) has the noun phrase \orth{awᴀ′q-qa′owułī} \fm{a waḵx̱ʼawoolí} ‘its eye door’.
This is probably the same noun phrase here in (\lastx) and so the form is \fm{a waḵx̱ʼawoolínáx̱} as given.
The speaker may have pronounced a rapid speech form like [\ipa{ʔà wàqχàwùːɬ͉í̥náχ}] or [\ipa{ʔà wàqχàwùːhnáχ}] which misled \citeauthor{swanton:1909}.
Alternatively, \citeauthor{swanton:1909}’s transcription was garbled or an original \orth{łi} in the manuscript was missed.

The root \fm{\rt[¹]{xʼak}} in (\lastx) specifically refers to the swimming of fish or sea mammals (sometimes limited to cetaceans) as opposed to other kinds of swimming by humans, waterfowl, etc.
This would not normally be used to describe a person like the protagonist, so there is an implicit meaning here that has not been literally translated into English.

\ex\label{ex:91-215-float-seaward}%
\exmn{276.10}%
\begingl
	\glpreamble	At!ekᴀtawe′ cwuʟ̣îx̣ā′c. //
	\glpreamble	A tʼikát áwé sh wudlihaash. //
	\gla	{} A \rlap{tʼikát} @ {} {} \rlap{áwé} @ {}
		sh @ \rlap{wudlihaash.} @ {} @ {} @ {} @ {} @ {} //
	\glb	{} a tʼiká -t {} á -wé
		sh= wu- d- l- i- \rt[¹]{hash} -μμL //
	\glc	{}[\pr{PP} \xx{3n·pss} seaward -\xx{pnct} {}] \xx{foc} -\xx{mdst}
		\xx{rflx·o}= \xx{pfv}- \xx{mid}- \xx{xtn}- \xx{stv}- \rt[¹]{float} -\xx{var} //
	\gld	{} its seaward -around {} \rlap{it.is} {}
		self \rlap{\xx{pfv}.\xx{ncnj}.float} {} {} {} {} {} //
	\glft	‘He had himself floating around seaward of it.’
		//
\endgl
\xe

\ex\label{ex:91-216-went-atop-sandbar}%
\exmn{276.10}%
\begingl
	\glpreamble	X̣ᴀkᵘkā′ wugu′t. //
	\glpreamble	Xákw káa woogoot. //
	\gla	{} Xákw \rlap{káa} @ {} {} \rlap{woogoot.} @ {} @ {} @ {} //
	\glb	{} xákw ká -μ {} wu- i- \rt[¹]{gut} -μμL //
	\glc	{}[\pr{PP} sandbar \xx{hsfc} -\xx{loc} {}]
		\xx{pfv}- \xx{stv}- \rt[¹]{go·\xx{pl}} -\xx{var} //
	\gld	{} sandbar atop -at {} \rlap{\xx{ncnj}.\xx{pfv}.go·\xx{sg}} {} {} {} //
	\glft	‘He went atop a sandbar.’
		//
\endgl
\xe

\ex\label{ex:91-217-take-off-self}%
\exmn{276.10}%
\begingl
	\glpreamble	Kāx kē awudîtī′. //
	\glpreamble	Kaax̱ kei awu̬ditée. //
	\gla	{} {} \rlap{Kaax̱} @ {} {}
		kei @ \rlap{awu̬ditée.} @ {} @ {} @ {} @ {} @ {} //
	\glb	{} {} ká -dáx̱ {}
		kei= a- wu- d- i- \rt[²]{ti} -μμH //
	\glc	{}[\pr{PP} \xx{rflx·pss} \xx{hsfc} -\xx{abl} {}]
		up= \xx{arg}- \xx{pfv}- \xx{mid}- \xx{stv}- \rt[²]{handle} -\xx{var} //
	\gld	{} self’s atop -off {}
		up \rlap{3>3.\xx{zcnj}.\xx{pfv}.handle} {} {} {} {} {} //
	\glft	‘He took it off of himself.’
		//
\endgl
\xe

\ex\label{ex:91-218-hang-on-stump}%
\exmn{276.11}%
\begingl
	\glpreamble	ᴀtguwunā′x aỵā′waxetc. //
	\glpreamble	Atgoowú náax̱ aÿaawax̱ích. //
	\gla	{} \rlap{Atgoowú} @ {} @ {} \rlap{náax̱} @ {} {}
		\rlap{aÿaawax̱ích.} @ {} @ {} @ {} @ {} @ {} //
	\glb	{} at= gú -í náa -x̱ {}
		a- ÿ- wu- i- \rt[²]{x̱ich} -μH //
	\glc	{}[\pr{PP} \xx{4n·pss} base -\xx{pss} covering -\xx{pert} {}]
		\xx{arg}- \xx{qual}- \xx{pfv}- \xx{stv}- \rt[²]{throw·inan} -\xx{var} //
	\gld	{} \rlap{stump} {} {} covering -at {}
		\rlap{3>3.\xx{zcnj}.hang} {} {} {} {} {} //
	\glft	‘He hung it over a stump.’
		//
\endgl
\xe

The verb \fm{aÿaawax̱ích} in (\lastx) is poorly documented.
It is attested only once as the phrase \fm{áx̱ yéi aÿax̱íchch} “he’s trying to hang it there” in \citeauthor{leer:1973}’s lexicography notes \parencite[f02/59]{leer:1973} and does not appear in his later materials.
The form given by \citeauthor{leer:1973} is a repetitive imperfective and implies the use of the motion derivation \fm{NP-x̱} + \fm{ÿaa=} \~\ \fm{ÿa-u-} (\fm{∅}; \fm{-ch} repetitive) ‘obliquely, circuitously along NP’.
The \fm{∅}-conjugation class is confirmed by the short high tone perfective stem \fm{-μH} which is identifiable in \citeauthor{swanton:1909}’s transcription \orth{xech}.
The translation as ‘hang’ follows \citeauthor{leer:1973}’s documentation but given the root \fm{\rt[²]{x̱ich}} meaning ‘throw inanimate’ it seems to involve tossing the object which is difficult to render straightforwardly in English.


\ex\label{ex:91-219-again-go-to-mother}%
\exmn{276.11}%
\begingl
	\glpreamble	Ts!u ts!ᴀs duʟa′ xᴀnt ūwagu′t. //
	\glpreamble	Tsu tsʼas du tláa x̱ánt uwagút. //
	\gla	Tsu \rlap{tsʼas} @ {} {} du tláa \rlap{x̱ánt} @ {} {}
		\rlap{uwagút.} @ {} @ {} @ {} //
	\glb	tsu tsʼa =s {} du tláa x̱án -t {}
		u- i- \rt[¹]{gut} -μH //
	\glc	again just =\xx{dub} {}[\pr{PP} \xx{3h·pss} mother near -\xx{pnct} {}]
		\xx{zpfv}- \xx{stv}- \rt[¹]{go·\xx{sg}} -\xx{var} //
	\gld	again just =maybe {} his mother near -to {}
		\rlap{\xx{zcnj}.\xx{pfv}.go·\xx{sg}} {} {} {} //
	\glft	‘Again he just goes near to his mother.’
		//
\endgl
\xe

\ex\label{ex:91-220-big-tentacles-floated-to-beach}%
\exmn{276.11}%
\begingl
	\glpreamble	Hᴀsduīg̣ayā′t asoîgu′q yuatʟ!e′g̣e ʟ!ᴀnq!. //
	\glpreamble	Hasdu eeg̱ayáat awsigúḵ yú at tlʼeig̱í tlénxʼ. //
	\gla	{} \rlap{Hasdu} @ {} \rlap{eeg̱ayáat} @ {} @ {} {}
		\rlap{awsigúḵ} @ {} @ {} @ {} @ {} @ {} +
		{} yú at \rlap{tlʼeig̱í} @ {} \rlap{tlénxʼ.} @ {} {} //
	\glb	{} hás= du eeḵ- ÿáᵏ -t {}
		a- wu- s- i- \rt[²]{guḵ} -μH
		{} yú at tlʼeeͥḵ -í tlein -xʼ {} //
	\glc	{}[\pr{PP} \xx{plh}= \xx{3h·pss} beach- face -\xx{pnct} {}]
		\xx{arg}- \xx{pfv}- \xx{xtn}- \xx{stv}- \rt[²]{poke} -\xx{var}
		{}[\pr{PP} \xx{dist} \xx{4n·pss} finger -\xx{pss} big -\xx{pl} {}] //
	\gld	{} \rlap{their} {} beach- face -to {}
		\rlap{3>3.\xx{zcnj}.\xx{pfv}.water·move} {} {} {} {} {}
		{} those sth’s \rlap{tentacle} {} big -s {} //
	\glft	‘Currents floated them to the beach below them, those giant tentacles.’
		//
\endgl
\xe

\ex\label{ex:91-221-cut-small-pieces}%
\exmn{276.12}%
\begingl
	\glpreamble	Yēkᵘdayā′ʟ! yêx ỵaoduʟ̣îxᴀ′c. //
	\glpreamble	Yéi kwdayáatlʼ yáx̱ ÿawdudlixásh. //
	\gla	{} {} Yéi @ \rlap{kwdayáatlʼ} @ {} @ {} @ {} @ {} @ {} {} yáx̱ {}
		\rlap{ÿawdudlixásh.} @ {} @ {} @ {} @ {} @ {} @ {} @ {} //
	\glb	{} {} yéi= k- u- d- \rt[¹]{ÿatlʼ} -μμH {} {} yáx̱ {}
		ÿ- wu- du- d- l- i- \rt[²]{xash} -μH //
	\glc	{}[\pr{PP} {}[\pr{NP} thus= \xx{cmpv}- \xx{irr}- \xx{mid}-
			\rt[¹]{short} -\xx{var} -\xx{nmz} {}] \xx{sim} {}]
		\xx{qual}- \xx{pfv}- \xx{4h·s}- \xx{mid}- \xx{xtn}- \xx{stv}-
			\rt[²]{cut} -\xx{var} //
	\gld	{} {} thus \rlap{\xx{cmpv}.\xx{impfv}.short·\xx{pl}} {} {} {} {} -ness {}
			like {}
		\rlap{\xx{zcnj}.\xx{pfv}.ppl.cut} {} {} {} {} {} {} {} //
	\glft	‘They were cut up like small pieces.’
		//
\endgl
\xe

The verb with \fm{du-} in (\lastx) could be analyzed two ways, either with \fm{du-} as a pseudopassive so that there is no explicit agent, or with \fm{du-} as an obviate pronoun.
The pseudopassive analysis is given by the translation in (\lastx), but the alternative obviate translation would be like “They cut them up like small pieces.”
If \fm{du-} is taken to be an obviate pronoun then its referent could be either the mother and grandmother or the protagonist.
The argument for the mother and grandmother is based on them being background characters in contrast with the foreground character of the protagonist.
The argument for the protagonist takes him to be backgrounded with the tentacles in the foreground.
\citeauthor{swanton:1909}’s translation is “He had cut them up into small pieces” and his gloss of the verb is “he cut it up”.
This implies the analysis with the protagonist as the backgrounded referent of the obviate pronoun.
The pseudopassive analysis thus may not be entirely accurate, but it has the advantage of avoiding the uncertain referent of \fm{du-}.

\ex\label{ex:91-222-thats-it-extinguished-townspeople}%
\exmn{276.12}%
\begingl
	\glpreamble	Detc!a′ a′ayu yū′antqenî qot acułix̣ī′x̣. //
	\glpreamble	De chʼa á áyú yú aantḵeiní ḵutx̱ ashuwlixeex. //
	\gla	De chʼa {} á {} \rlap{áyú} @ {}
		{} yú \rlap{aantḵeiní} @ {} @ {} @ {} @ {} @ {} {}
		\rlap{ḵutx̱} @ {} @ \rlap{ashuwlixeex.} @ {} @ {} @ {} @ {} @ {} @ {} //
	\glb	de chʼa {} á {} á -yú
		{} yú aan- d- \rt{ḵi} -μμL -n -í {}
		ḵú -dáx̱= a- shu- wu- l- i- \rt[¹]{xix} -μμL //
	\glc	now just {}[\pr{DP} \xx{3n} {}] \xx{foc} -\xx{dist}
		{}[\pr{DP} \xx{dist} town- \xx{mid}- \rt[¹]{sit·\xx{pl}}
			-\xx{var} -\xx{nsfx} -\xx{nmz} {}]
		\xx{areal} -\xx{abl}= \xx{arg}- end- \xx{pfv}- \xx{csv}- \xx{stv}-
			\rt[¹]{fall} -\xx{var} //
	\gld	now just {} it {} \rlap{it.is} {}
		{} those \rlap{townspeople} {} {} {} {} {} {}
		\rlap{off} {} \rlap{3>3.end.\xx{pfv}.make.extinct} {} {} {} {} {} {} //
	\glft	‘Now that was it that extinguished those townspeople.’
		//
\endgl
\xe

\section{Paragraph 17}\label{sec:91-para-17}

\ex\label{ex:91-223-again-make-arrows-go}%
\exmn{277.1}%
\begingl
	\glpreamble	ᴀdayu′ ts!u tcū′net a′ołiāt. //
	\glpreamble	Át áyú tsu chooneit awli.aat. //
	\gla	{} \rlap{Át} @ {} {} \rlap{áyú} @ {} tsu
		{} {} \rlap{chooneit} @ {} @ {} {} {} {}
		\rlap{awli.aat.} @ {} @ {} @ {} @ {} @ {} //
	\glb	{} á -t {} á -wé tsu
		{} {} \rt[¹]{chun} -μμL -i {} át {}
		a- wu- l- i- \rt[¹]{.at} -μμL //
	\glc	{}[\pr{PP} \xx{3n} -\xx{pnct} {}] \xx{foc} -\xx{mdst} again
		{}[\pr{DP} {}[\pr{CP} \rt[¹]{wound} -\xx{var} -\xx{rel} {}] thing {}]
		\xx{arg}- \xx{pfv}- \xx{csv}- \xx{stv}- \rt[¹]{go·\xx{pl}} -\xx{var} //
	\gld	{} there -around {} \rlap{it.is} {} again
		{} {} \rlap{arrow} {} {} {} {} {}
		\rlap{3>3.\xx{ncnj}.\xx{pfv}.make.go·\xx{pl}} {} {} {} {} {} //
	\glft	‘It was around there again that he made arrows go.’
		//
\endgl
\xe

\ex\label{ex:91-224-came-upon-mouse-hole}%
\exmn{277.1}%
\begingl
	\glpreamble	Akᴀ′x wugu′t kuts!ī′n awᴀ′q-qa′owuli. //
	\glpreamble	A káx̱ woogoot kutsʼeen a waḵx̱ʼawoolí. //
	\gla	{} A \rlap{káx̱} @ {} {}
		\rlap{woogoot} @ {} @ {} @ {}
		{} kutsʼeen {}
		{} a \rlap{waḵx̱ʼawoolí.} @ {} @ {} @ {} {} //
	\glb	{} a ká -x̱ {}
		wu- i- \rt[¹]{gut} -μμL
		{} kutsʼeen {}
		{} a waaḵ- x̱ʼé- wool -í {} //
	\glc	{}[\pr{PP} \xx{3n·pss} \xx{hsfc} -\xx{pert} {}]
		\xx{pfv}- \xx{stv}- \rt[¹]{go·\xx{sg}} -\xx{var}
		{}[\pr{DP} mouse {}]
		{}[\pr{DP} \xx{3n·pss} eye- mouth- hole -\xx{pss} {}] //
	\gld	{} its atop -on {}
		\rlap{\xx{ncnj}.\xx{pfv}.go·\xx{sg}} {} {} {} 
		{} mouse {}
		{} its eye- \rlap{door} {} {} {} //
	\glft	‘He came upon it, a mouse, its hole.’
		//
\endgl
\xe

\FIXME{Refer to (\ref{ex:89-33-deermouse-super-help}) in chapter \ref{ch:89-origin-of-copper} for identification of \fm{kutsʼeen}.}

\ex\label{ex:91-225-poke-tail-up-thru}%
\exmn{277.2}%
\begingl
	\glpreamble	Aʟ!ī′t a′nᴀx kē aołitsᴀ′q. //
	\glpreamble	A lʼeet aanáx̱ kei awlitsáḵ. //
	\gla	{} A lʼeet {} {} \rlap{aanáx̱} @ {} {}
		kei @ \rlap{awlitsáḵ.} @ {} @ {} @ {} @ {} @ {} //
	\glb	{} a lʼeet {} {} á -náx̱ {}
		kei= a- wu- l- i- \rt[²]{tsaḵ} -μH //
	\glc	{}[\pr{DP} \xx{3n·pss} tail {}] {}[\pr{PP} \xx{3n} -\xx{perl} {}]
		up= \xx{arg}- \xx{pfv}- \xx{xtn}- \xx{stv}- \rt[²]{poke} -\xx{var} //
	\gld	{} its tail {} {} it -thru {}
		up \rlap{3>3.\xx{zcnj}.\xx{pfv}.poke} {} {} {} {} {} //
	\glft	‘It poked its tail up through it.’
		//
\endgl
\xe

\ex\label{ex:91-226-went-to-town}%
\exmn{277.2}%
\begingl
	\glpreamble	Tc!akᴀ′t ānt uwagu′t. //
	\glpreamble	Chʼakʼát aant uwagút. //
	\gla	\rlap{Chʼakʼát} @ {} {} \rlap{aant} @ {} {}
		\rlap{uwagút.} @ {} @ {} @ {} //
	\glb	chʼa= kʼát {} aan -t {} 
		u- i- \rt[¹]{gut} -μH //
	\glc	just= ?? {}[\pr{PP} town -\xx{pnct} {}]
		\xx{zpfv}- \xx{stv}- \rt[¹]{go·\xx{sg}} -\xx{var} //
	\gld	\rlap{immediately} {} {} town -to {}
		\rlap{\xx{zcnj}.\xx{pfv}.go·\xx{sg}} {} {} {} //
	\glft	‘He went to town immediately.’
		//
\endgl
\xe

The word that \citeauthor{swanton:1909} transcribes as \orth{Tc!akᴀ′t} (\lastx) appears to be something like \fm{chʼakát} but this is unattested.
There is a very close adverb \fm{chʼakʼát} or \fm{chʼa kʼát} that \citeauthor{leer:1973} translates as ‘at least’ \parencites[f04/89]{leer:1973}[66]{leer:1978b}.
The \fm{kʼát} element is not otherwise known, but there is a similar \fm{chʼa kʼeekát} ‘just barely; barely managing (to do a little); just now; a little while ago’ \parencite[\textsc{t}·5]{leer:2001} and further a possibly related \fm{wankʼeenáx̱} ‘quietly; under one’s breath; without telling anybody’ \parencite[\textsc{t}·83]{leer:2001}.
The context and translation in (\lastx) do not support the interpretation of \fm{chʼakʼát} as ‘at least’, but the sense ‘just now’ could match \citeauthor{swanton:1909}’s gloss “right” and his translation “directly”, thus the translation of ‘immediately’ given in (\lastx).
See also the discussion of \fm{chʼayéi} ‘ordinary’ at (\ref{ex:100-40-salmon-people-rescue}) in chapter \ref{ch:100-salmon-boy-wrg}.

\ex\label{ex:91-227-started-going-before-ravens}%
\exmn{277.2}%
\begingl
	\glpreamble	Tc!u ts!ūtā′t łiyē′ł duāxdjī′awe adê′ g̣one′ uwagu′t. //
	\glpreamble	Chʼu tsʼootaat l yéil du.áx̱ji áwé aadé g̱unéi uwagút. //
	\gla	Chʼu {} tsʼootaat @ {} {}
		{} l {} yéil {} \rlap{du.áx̱ji} @ {} @ {} @ {} @ {} @ {} {}
		\rlap{áwé} @ {} +
		{} \rlap{aadé} @ {} {}
		g̱unéi @ \rlap{uwagút.} @ {} @ {} @ {} //
	\glb	chʼu {} tsʼootaat {} {}
		{} l {} yéil {} u- du- \rt[²]{.ax̱} -μH -ch -í {}
		á -wé
		{} á -dé {}
		g̱unaÿéi= u- i- \rt[¹]{gut} -μH //
	\glc	just {}[\pr{PP} morning -\xx{loc} {}]
		{}[\pr{CP} \xx{neg} {}[\pr{DP} raven {}] \xx{irr}- \xx{4h·s}-
			\rt[²]{hear} -\xx{var} -\xx{rep} -\xx{sub} {}]
		\xx{foc} -\xx{mdst}
		{}[\pr{PP} \xx{3n} -\xx{all} {}]
		\xx{incep}= \xx{zpfv}- \xx{stv}- \rt[¹]{go·\xx{sg}} -\xx{var} //
	\gld	just {} morning -at {}
		{} not {} raven {}
			\rlap{\xx{gcnj}.\xx{impfv}.ppl.hear.\xx{rep}} {} {} {} {} -when {}
		\rlap{it.is} {}
		{} there -to {}
		start \rlap{\xx{zcnj}.\xx{pfv}.go·\xx{sg}} {} {} {} //
	\glft	‘Just in the morning, it was when people were not hearing ravens that he started going there.’
		//
\endgl
\xe

\ex\label{ex:91-228-took-his-shirt}%
\exmn{277.3}%
\begingl
	\glpreamble	Awaā′x duk!udᴀ′s!î, hīntaỵī′cî k!udᴀ′s!. //
	\glpreamble	Aawa.aax̱ du kʼoodásʼi, héen taaÿéeshi kʼoodásʼ. //
	\gla	\rlap{Aawa.aax̱} @ {} @ {} @ {} @ {}
		{} du \rlap{kʼoodásʼi,} @ {} {}
		{} héen \rlap{taaÿéeshi} @ {} @ {} @ {} kʼoodásʼ\rlap{.} @ {} //
	\glb	a- wu- i- \rt[²]{.ax̱} -μμL
		{} du kʼoodásʼ -í {}
		{} héen táaᵏ= \rt[²]{ÿish} -μμH -i kʼoodásʼ {} //
	\glc	\xx{arg}- \xx{pfv}- \xx{stv}- \rt[²]{hdl·fab} -\xx{var}
		{}[\pr{DP} \xx{3h·pss} shirt -\xx{pss} {}]
		{}[\pr{DP} water bottom= \rt[²]{pull} -\xx{var} -\xx{pss} tunic {}] //
	\gld	\rlap{3>3.\xx{ncnj}.\xx{pfv}.handle·fabric} {} {} {} {}
		{} his shirt {} {}
		{} water bottom \rlap{puller} {} -of shirt {} //
	\glft	‘He took his shirt, the \fm{héen taaÿeeshi} shirt.’
		//
\endgl
\xe

\ex\label{ex:91-229-going-behind-put-it-on}%
\exmn{277.4}%
\begingl
	\glpreamble	Ayat!ē′t gudawe′ kᴀx aoditi′, //
	\glpreamble	A yatʼéit góot áwé káx̱ awditee; //
	\gla	{} {} A \rlap{yatʼéit} @ {} @ {} {} \rlap{góot} @ {} @ {} @ {} {}
		\rlap{áwé} @ {}
		{} {} \rlap{káx̱} @ {} {}
		\rlap{awditee;} @ {} @ {} @ {} @ {} @ {} //
	\glb	{} {} a ÿá- tʼéiᵏ -t {} {} \rt[¹]{gut} -μμH {} {}
		á -wé
		{} {} ká -x̱ {}
		a- wu- d- i- \rt[²]{ti} -μμL //
	\glc	{}[\pr{CP} {}[\pr{PP} its face- behind -\xx{pnct} {}]
			\xx{zcnj}\· \rt[¹]{go·\xx{sg}} -\xx{var} \·\xx{sub} {}]
		\xx{foc} -\xx{mdst}
		{}[\pr{PP} \xx{rflx·pss} \xx{hsfc} -\xx{pert} {}]
		\xx{arg}- \xx{pfv}- \xx{mid}- \xx{stv}- \rt[²]{handle} -\xx{var} //
	\gld	{} {} its face- behind -to {} \rlap{\xx{zcnj}.\xx{csec}.go·\xx{sg}} {} {} {} {}
		\rlap{it.is} {}
		{} self’s atop -on {}
		\rlap{3>3.\xx{g̱cnj}.\xx{pfv}.handle} {} {} {} {} {} //
	\glft	‘Having gone behind it he put it on himself;’
		//
\endgl
\xe

\ex\label{ex:91-230-repeatedly-rub-edge}%
\exmn{277.4}%
\begingl
	\glpreamble	aỵᴀłanē′s!awe awᴀ′n. //
	\glpreamble	aÿalaneisʼ áwé a wán. //
	\gla	\rlap{aÿalaneisʼ} @ {} @ {} @ {} @ {} @ {} 
		\rlap{áwé} @ {}
		{} a wán. {} //
	\glb	a- ÿ- l- \rt[¹]{naʰ} -eμL -sʼ
		á -wé
		{} a wán {} //
	\glc	\xx{arg}- \xx{qual}- \xx{csv}- \rt[¹]{damp} -\xx{var} -\xx{rep}
		\xx{foc} -\xx{mdst}
		{}[\pr{DP} \xx{3n·pss} edge {}] //
	\gld	\rlap{3>3.\xx{zcnj}.\xx{impfv}.grease.\xx{rep}} {} {} {} {} {}
		\rlap{it.is} {}
		{} its edge {} //
	\glft	‘he repeatedly rubs it, its edge.’
		//
\endgl
\xe

\FIXME{\citeauthor{swanton:1909} glosses “when he had sharpened” and translates “after he had sharpened its edges”.
Discuss problems with verb root \fm{\rt[¹]{naʰ}} ‘damp’ and with the derived verb \fm{l-\rt[¹]{naʰ}} ‘grease’.
Note \fm{sh yawdlináa} ‘s/he greased face’ and \fm{yaneisʼí} ‘face grease’ \parencite[246]{leer:1976}.}

\ex\label{ex:91-231-gotten-in-went-up-to-hole}%
\exmn{277.5}%
\begingl
	\glpreamble	Atū′x nagū′dawe ā′ke uwagu′t yu′wᴀq-qawū′ł. //
	\glpreamble	A tóoxʼ nagóot áwé áa kei uwagút yú waḵx̱ʼawool. //
	\gla	{} {} A \rlap{tóoxʼ} @ {} {}
			\rlap{nagóot} @ {} @ {} @ {} {}
		\rlap{áwé} @ {}
		{} \rlap{áa} @ {} {}
		kei @ \rlap{uwagút} @ {} @ {} @ {}
		{} yú \rlap{waḵx̱ʼawool.} @ {} @ {} {} //
	\glb	{} {} a tú -xʼ {}
			n- \rt[¹]{gut} -μμH {} {}
		á -wé
		{} á -μ {}
		kei= u- i- \rt[¹]{gut} -μH
		{} yú waaḵ- x̱ʼé- wool {} //
	\glc	{}[\pr{CP} {}[\pr{PP} \xx{3n·pss} inside -\xx{loc} {}]
			\xx{ncnj}- \rt[¹]{go·\xx{pl}} -\xx{var} \·\xx{sub} {}]
		\xx{foc} -\xx{mdst}
		{}[\pr{PP} \xx{3n} -\xx{loc} {}]
		up= \xx{zpfv}- \xx{stv}- \rt[¹]{go·\xx{sg}} -\xx{var}
		{}[\pr{DP} \xx{dist} eye- mouth- hole {}] //
	\gld	{} {} its inside -at {}
			\rlap{\xx{ncnj}.\xx{csec}.go·\xx{sg}} {} {} {} {}
		\rlap{it.is} {}
		{} there -at {} 
		up \rlap{\xx{zcnj}.\xx{pfv}.go·\xx{sg}} {} {} {}
		{} that eye- mouth- hole {} //
	\glft	‘Having gone into it, he went up to it, that hole.’
		//
\endgl
\xe

The referent of the PP \fm{a tóoxʼ} ‘at the inside of it’ in (\lastx) is the \fm{héen taaÿéeshi kʼoodásʼ} mentioned in (\ref{ex:91-228-took-his-shirt}) through (\ref{ex:91-230-repeatedly-rub-edge}).
The referent of the PP \fm{áa} ‘at it’ in (\lastx) is in contrast the right dislocated DP \fm{yú waḵx̱ʼawool} or mouse hole introduced in (\ref{ex:91-224-came-upon-mouse-hole}).

\ex\label{ex:91-232-rock-push-atop-back}%
\exmn{277.5}%
\begingl
	\glpreamble	Tc!uʟe′ te′awe ᴀdᴀ′q! kᴀt ā′wᴀguq //
	\glpreamble	Chʼu tle té áwé a dix̱ʼkát aawagúḵ; //
	\gla	Chʼu tle {} té {} \rlap{áwé} @ {}
		{} a \rlap{dix̱ʼkát} @ {} @ {} {}
		\rlap{aawagúḵ;} @ {} @ {} @ {} @ {} //
	\glb	chʼu tle {} té {} á -wé
		{} a dáaḵ- ká -t {}
		a- wu- i- \rt[²]{guḵ} -μH //
	\glc	just then {}[\pr{DP} rock {}] \xx{foc} -\xx{mdst}
		{}[\pr{PP} \xx{3n·pss} spine- \xx{hsfc} -\xx{pnct} {}]
		\xx{arg}- \xx{pfv}- \xx{stv}- \rt[²]{poke} -\xx{var} //
	\gld	just then {} rock {} \rlap{it.is} {}
		{} its spine- atop -to {}
		\rlap{3>3.\xx{zcnj}.\xx{pfv}.push} {} {} {} {} //
	\glft	‘Just then it was a rock that he pushed on top of its back;’
		//
\endgl
\xe

\ex\label{ex:91-233-crackle-like-mountain-fall-apart}%
\exmn{277.6}%
\begingl
	\glpreamble	ākā′was!ūnk yū′ca wᴀ′sî wū′cdᴀx g̣ā′xdag̣ᴀ′dîn. //
	\glpreamble	áa kaawasʼúnk, yú shaa wáa sá wóoshdáx̱ g̱aax̱dag̱ádín. //
	\gla	{} \rlap{áa} @ {} {}
		\rlap{kaawasʼúnk,} @ {} @ {} @ {} @ {} @ {} +
		{} {} yú shaa {} {} wáa sá {} {} \rlap{wóoshdáx̱} @ {} {}
			\rlap{g̱aax̱dag̱ádín.} @ {} @ {} @ {} @ {} @ {} @ {} {} //
	\glb	{} á -μ {}
		k- wu- i- \rt[¹]{sʼuʼn} -μH -k
		{} {} yú shaa {} {} wáa sá {} {} wóosh -dáx̱ {}
			g̱- g̱- d- \rt[¹]{g̱aʼt} -μH -ín {} //
	\glc	{}[\pr{PP} \xx{3n} -\xx{loc} {}]
		\xx{qual}- \xx{pfv}- \xx{stv}- \rt[¹]{kiss·noise} -\xx{var} -\xx{rep}
		{}[\pr{CP} {}[\pr{DP} \xx{dist} mountain {}] {}[\pr{QP} how \xx{q} {}]
			{}[\pr{PP} \xx{recip} -\xx{abl} {}]
			\xx{g̱cnj}- \xx{mod}- \xx{mid}-
				\rt[¹]{scatter} -\xx{var} -\xx{ctng} {}] //
	\gld	{} there -at {}
		\rlap{\xx{ncnj}.\xx{pfv}.kiss·noise.\xx{rep}} {} {} {} {} {}
		{} {} the mountain {} {} how ever {} {} ea·oth -from {}
			\rlap{\xx{g̱cnj}.\xx{ctng}.fall·apart} {} {} {} {} {} {} {} //
	\glft	‘there it crackled, however a mountain does whenever it falls apart.’
		//
\endgl
\xe

\ex\label{ex:91-234-float-around-rock-if-it-swims-out}%
\exmn{277.7}%
\begingl
	\glpreamble	ᴀnᴀ′x yū′de ko′kwaq!aqī′djayu tcaya′ cwuʟ̣ix̣ā′c ayaỵī′q! ᴀ′nᴀx dāq q!ᴀ′qnî ỵîs //
	\glpreamble	Anáx̱ yóode gug̱axʼáagich áyú chʼa yá té yát áwé sh wudlihaash a yaÿeexʼ, anáx̱ daak xʼákni ÿís. //
	\gla	{} {} {} \rlap{Anáx̱} @ {} {} {} \rlap{yóode} @ {} {}
			\rlap{kuḵaxʼáagich} @ {} @ {} @ {} @ {} @ {} {} {} {} 
		\rlap{áyú} @ {}
		chʼa {} yá té \rlap{yát} @ {} {} \rlap{áwé} @ {}
		sh @ \rlap{wudlihaash} @ {} @ {} @ {} @ {} @ {}
		{} a \rlap{yaÿeexʼ,} @ {} @ {} {}
		{} {} {} \rlap{anáx̱} @ {} {}
			daak @ \rlap{xʼákni} @ {} @ {} @ {} @ {} {} ÿís {} //
	\glb	{} {} {} á -náx̱ {} {} yú -dé {}
			u- g- g̱- \rt[¹]{xʼak} -μμH -í {} -ch {}
		á -yú
		chʼa {} yá té ÿá -t {} á -wé
		sh= wu- d- l- i- \rt[¹]{hash} -μμL
		{} a ÿá- ÿee -xʼ {}
		{} {} {} á -náx̱ {}
			dáak= {} \rt[¹]{xʼak} -μH -n -í {} ÿís {} //
	\glc	{}[\pr{PP} {}[\pr{CP} {}[\pr{PP} \xx{3n} -\xx{perl} {}]
			{}[\pr{PP} \xx{dist} -\xx{all} {}]
			\xx{irr}- \xx{gcnj}- \xx{mod}- \rt[¹]{fish·swim} -\xx{var} -\xx{sub} {}]
			-\xx{erg} {}]
		\xx{foc} -\xx{dist}
		just {}[\pr{DP} \xx{prox} rock face -\xx{pnct} {}] \xx{foc} -\xx{mdst}
		\xx{rflx·o}= \xx{pfv}- \xx{mid}- \xx{xtn}- \xx{stv}- \rt[¹]{float} -\xx{var}
		{}[\pr{PP} \xx{3n·pss} face- below -\xx{loc} {}]
		{}[\pr{PP} {}[\pr{CP} {}[\pr{PP} \xx{3n} -\xx{perl} {}]
			seaward= \xx{zcnj}\· \rt[¹]{fish·swim}
				-\xx{var} -\xx{nsfx} -\xx{sub} {}] \xx{ben} {}] //
	\gld	{} {} {} there -thru {} {} off -to {}
			\rlap{\xx{ncnj}.\xx{prsp}.fish·swim} {} {} {} {} {} {} -cause {}
		\rlap{it.is} {}
		just {} this rock face -around {} \rlap{it.is} {}
		self \rlap{\xx{pfv}.\xx{ncnj}.float} {} {} {} {} {}
		{} its \rlap{await} {} -at {}
		{} {} {} there -thru {}
			out \rlap{\xx{zcnj}.\xx{cond}.fish·swim} {} {} {} {} {} for {} //
	\glft	‘Because it is going to swim off through there, it was around the face of this rock that he had himself float, in wait of it, for if it swims in through there.’
		//
\endgl
\xe

\citeauthor{swanton:1909}’s \orth{dāq} in (\lastx) could be either \fm{daaḵ} ‘inland, up from water’s edge’ or \fm{daak} ‘seaward, out from water’s edge’ because he is notoriously unreliable in distinguishing velar and uvular sounds.
In this context it is difficult to tell from meaning alone which of the two is the right interpretation.
\citeauthor{swanton:1909} glosses \orth{dāq} as “out” and his translation has “swim out” which both suggest that the preverb is \fm{daak=} ‘seaward’ since this is often translated as ‘out to sea’.

\ex\label{ex:91-235-swam-out-poke-upper-lip}%
\exmn{277.8}%
\begingl
	\glpreamble	ᴀ′nᴀx dāk q!ā′g̣awe duī′t k!ᴀłū′wᴀts!ᴀq. //
	\glpreamble	Anáx̱ daak xʼáak áwé du eet ḵʼaloowatsáḵ. //
	\gla	{} {} \rlap{Anáx̱} @ {} {}
			daak @ \rlap{xʼáak} @ {} @ {} @ {} {}
		\rlap{áwé} @ {}
		{} du \rlap{eet} @ {} {}
		\rlap{ḵʼaloowatsáḵ.} @ {} @ {} @ {} @ {} @ {} //
	\glb	{} {} á -náx̱ {}
			dáak= {} \rt[¹]{xʼak} -μμH {} {}
		á -wé
		{} du ee -t {}
		ḵʼe- lu- wu- i- \rt[²]{tsaḵ} -μH //
	\glc	{}[\pr{CP} {}[\pr{PP} \xx{3n} -\xx{perl} {}]
			seaward= \xx{zcnj}\· \rt[¹]{fish·swim} -\xx{var} \·\xx{sub} {}]
		\xx{foc} -\xx{mdst}
		{}[\pr{PP} \xx{3h} \xx{base} -\xx{pnct} {}]
		mouth- nose- \xx{pfv}- \xx{stv}- \rt[²]{poke} -\xx{var} //
	\gld	{} {} there -thru {}
			out \rlap{\xx{zcnj}.\xx{csec}.fish·swim} {} {} {} {}
		\rlap{it.is} {}
		{} him {} -to {}
		\rlap{upper.lip.\xx{zcnj}.\xx{pfv}.poke} {} {} {} {} {}  //
	\glft	‘Having swum out through there, it poked him with its upper lip.’
		//
\endgl
\xe

\FIXME{Discuss \fm{ḵʼalú}.}

\ex\label{ex:91-236-swam-past}%
\exmn{277.9}%
\begingl
	\glpreamble	ᴀciyā′nᴀx yā′waq!aq. //
	\glpreamble	Ash niyaanáx̱ yaawaxʼaak. //
	\gla	{} Ash \rlap{niyaanáx̱} @ {} {}
		\rlap{yaawaxʼaak.} @ {} @ {} @ {} @ {} //
	\glb	{} ash niÿaa -náx̱ {}
		ÿ- wu- i- \rt[¹]{xʼak} -μμL //
	\glc	{}[\pr{PP} \xx{3prx·pss} dir’n -\xx{perl} {}]
		\xx{qual}- \xx{pfv}- \xx{stv}- \rt[¹]{fish·swim} -\xx{var} //
	\gld	{} his dir’n -thru {}
		\rlap{off.\xx{ncnj}.\xx{pfv}.fish·swim} {} {} {} {} //
	\glft	‘It swam past him.’
		//
\endgl
\xe

\ex\label{ex:91-237-tail-split-him}%
\exmn{277.9}%
\begingl
	\glpreamble	Duʟ!ī′t ᴀckā′yᴀnᴀx łaxō′t!. //
	\glpreamble	Du lʼeet ash káa ayanax̱lax̱óotʼ. //
	\gla	{} Du lʼeet {} {} ash \rlap{káa} @ {} {}
		\rlap{ayanax̱lax̱óotʼ.} @ {} @ {} @ {} @ {} @ {} //
	\glb	{} du lʼeet {} {} ash ká -μ {}
		a- ÿ- n- g̱- l- \rt[²]{x̱utʼ} -μμH //
	\glc	{}[\pr{DP} \xx{3h·pss} tail {}] {}[\pr{DP} \xx{3prx·pss} \xx{hsfc} -\xx{loc} {}]
		\xx{arg}- \xx{qual}- \xx{ncnj}- \xx{mod}- \xx{xtn}- \rt[²]{adze} -\xx{var} //
	\gld	{} its tail {} {} his atop -on {}
		\rlap{3>3.\xx{ncnj}.\xx{hort}.split} {} {} {} {} {} //
	\glft	‘Its tail tries to split him.’
		//
\endgl
\xe

The verb in (\lastx) is difficult to decipher.
\citeauthor{swanton:1909} identifies the verb alone as \orth{łaxō′t!} which suggests \fm{lax̱óotʼ} but this is ungrammatical if the root \fm{\rt[²]{x̱utʼ}} ‘adze, hew; chop, split’ is correct.
There is a candidate verb phrase \fm{NP-náx̱ yaa awlix̱óotʼ} (\fm{g̱}; achievement, \fm{-ch} repetitive) ‘s/he split him/her/it down through NP’ which is attested in the example sentence \fm{du sháanáx̱ yaa wdudlix̱óotʼ} “they split the head open (with a descending blow)” \parencite[205.2864]{story-naish:1973} and which includes the motion derivations \fm{NP-náx̱} (\fm{g̱}; \fm{-ch} repetitive) ‘down by way of, through NP’ and \fm{ÿaa=} (\fm{g̱}; \fm{-ch} repetitive) ‘downward’.
There is another candidate verb \fm{ayawlix̱óotʼ} (\fm{g̱}? \fm{n}?; achievement) ‘s/he gave him/her/it a chop’ that is cited from an unknown “Stk” (perhaps “Stikine”, i.e.\ Wrangell) source by \textcite[f02/785]{leer:1973}.
Coincidentally this entry is pasted over what seems to be a quote of (\lastx) judging by the crossed out “…tʼ (Sw.\ 227-9)”.
Both of these were apparently missed in \citeauthor{leer:1976}’s later entry for \fm{\rt[²]{x̱utʼ}} ‘adze, hew; chop, split’ \parencite[818]{leer:1976}.
The analysis in (\lastx) identifies the verb as \fm{ayanax̱lax̱óotʼ}, assuming the second candidate verb as \fm{n}-conjugation class with hortative mood \fm{\xx{cnj}-g̱-…-μμH}.
\FIXME{Discuss the propblem with hortative here.
\citeauthor{swanton:1909} glosses “it wanted to drop”, suggesting conativity.
But the hortative is not normally used for conation.}


\ex\label{ex:91-238-float-edge-up}%
\exmn{277.10}%
\begingl
	\glpreamble	ᴀsiyu′ ʟa kî′ndawᴀnīn cwuʟ̣ixā′c. Duʟ!ī′t yaỵī′q! //
	\glpreamble	Ásíyú tle kindewáneen sh wudlihaash, du lʼeet yaÿeexʼ. //
	\gla	\rlap{Ásíyú} @ {} @ {} tle
		\rlap{kindewáneen} @ {} @ {} @ {} 
		sh @ \rlap{wudlihaash,} @ {} @ {} @ {} @ {} @ {} +
		{} du lʼeet \rlap{yaÿeexʼ} @ {} @ {} {} //
	\glb	á -sí -yú tle
		kín -dé= wán -een
		sh= wu- d- l- i- \rt[¹]{hash} -μμL
		{} du lʼeet ÿá- ÿee -xʼ {} //
	\glc	\xx{foc} -\xx{dub} -\xx{dist} then
		up -\xx{all}= edge -\xx{adv}
		\xx{rflx·o}= \xx{pfv}- \xx{mid}- \xx{xtn}- \xx{stv}- \rt[¹]{float} -\xx{var}
		{}[\pr{PP} \xx{3h·pss} tail face- below -\xx{loc} {}] //
	\gld	\rlap{it.is.apparently} {} {} then
		\rlap{edge.upward} {} {} {}
		self \rlap{\xx{pfv}.\xx{ncnj}.float} {} {} {} {} {}
		{} his tail \rlap{await} {} -at {} //
	\glft	‘Apparently he just floated himself edge up, awaiting its tail.’
		//
\endgl
\xe

The word that \citeauthor{swanton:1909} transcribes as \orth{kî′ndawᴀnīn} appears to be an adverb that is otherwise unattested.
The elements \fm{kínde} ‘upward’ and \fm{wán} ‘edge’ are both readily identifiable.
The suffix \fm{-een} might be identified in a number of other adverbs and relational nouns: \fm{seig̱ánin} ‘tomorrow’, \fm{dziyáagin} ‘after a while, later on’, \fm{a dakádin} ‘opposite of it; other direction from it’, \fm{du daséixʼín} ‘swapping places with him/her’, \fm{a dayéen} ‘facing it’, \fm{a kujéen} ‘on account of, motivated by it’, \fm{wushduwag̱íg̱ín} ‘in circles’.
At least in some cases it could be plausibly derived from \fm{een} ‘with’ which is composed of the meaningless attachment base \fm{ee} and the instrumental postposition \fm{-n}.
This would connect it to a few conventionalized adverbial PPs based on the instrumental postposition allomorph \fm{teen}: \fm{tleikax̱ʼustín} ‘on one foot’, \fm{deix̱kax̱ʼustín} ‘on both feet’, \fm{tleikajíntin} ‘one-handed’, \fm{deix̱kajíntin} ‘two-handed’, \fm{kagoochkʼítin} ‘with all one’s strength’.

\ex\label{ex:91-239-split-him}%
\exmn{277.10}%
\begingl
	\glpreamble	ᴀckā′yanᴀx łaxō′t!. //
	\glpreamble	Ash káa ayanax̱lax̱óotʼ. //
	\gla	{} Ash \rlap{káa} @ {} {}
		\rlap{ayanax̱lax̱óotʼ.} @ {} @ {} @ {} @ {} @ {} //
	\glb	{} ash ká -μ {}
		a- ÿ- n- g̱- l- \rt[²]{x̱utʼ} -μμH //
	\glc	{}[\pr{DP} \xx{3prx·pss} \xx{hsfc} -\xx{loc} {}]
		\xx{arg}- \xx{qual}- \xx{ncnj}- \xx{mod}- \xx{xtn}- \rt[²]{adze} -\xx{var} //
	\gld	{} his atop -on {}
		\rlap{3>3.\xx{ncnj}.\xx{hort}.split} {} {} {} {} {} //
	\glft	‘It tries to split him.’
		//
\endgl
\xe

\ex\label{ex:91-240-like-cut-apart-tail-repeatedly-chops}%
\exmn{277.11}%
\begingl
	\glpreamble	Wū′cxken du′łx̣ᴀce ayê′xayu yᴀ′te duʟ!ī′t ᴀckā′yᴀn yuałxō′t!kuᵘ. //
	\glpreamble	Wooshx̱ kandulxáshi aa yáx̱ áyú yéide du lʼeet ash káa yan yoo alx̱útʼgu. //
	\gla	{} {} {} {} \rlap{Wóoshx̱} @ {} {}
			\rlap{kandulxáshi} @ {} @ {} @ {} @ {} @ {} @ {} @ {} {}
			aa {} yáx̱ {} \rlap{áyú} @ {}
		{} \rlap{yáade} @ {} {} 
		{} du lʼeet {} {} ash \rlap{káa} @ {} {}
		yan @ yoo @ \rlap{alx̱útʼgu.} @ {} @ {} @ {} @ {} @ {} //
	\glb	{} {} {} {} wóosh -x̱ {}
			k- n- du- d- l- \rt[²]{xash} -μH -i {}
			aa {} yáx̱ {} á -yú
		{} yéi -dé {}
		{} du lʼeet {} {} ash ká -μ {}
		ÿán= yoo= a- l- \rt[²]{x̱utʼ} -μH -k -í //
	\glc	{}[\pr{PP} {}[\pr{DP} {}[\pr{CP} {}[\pr{PP} \xx{recip} -\xx{pert} {}]
			\xx{qual}- \xx{ncnj}- \xx{4h·s}- \xx{mid}- \xx{xtn}-
				\rt[²]{cut} -\xx{var} -\xx{rel} {}]
			\xx{part} {}] \xx{sim} {}] \xx{foc} -\xx{mdst}
		{}[\pr{PP} thus -\xx{all} {}]
		{}[\pr{DP} \xx{3h·pss} tail {}]
		{}[\pr{PP} \xx{3prx·pss} \xx{hsfc} -\xx{loc} {}]
		ground= \xx{alt}= \xx{arg}- \xx{xtn}-
			\rt[²]{adze} -\xx{var} -\xx{rep} -\xx{sub} //
	\gld	{} {} {} {} ea·oth -of {}
			\rlap{\xx{ncnj}.\xx{prog}.ppl.cut} {} {} {} {} {} {} -that {}
			one {} like {} \rlap{it.is} {}
		{} thus -way {}
		{} his tail {} {} his atop -on {}
		down \xx{alt} \rlap{3>3.\xx{ncnj}.\xx{impfv}.split.\xx{rep}} {} {} {} {} {} //
	\glft	‘It was like the way people are cutting it apart that its tail chops up and down on him.’
		//
\endgl
\xe

\ex\label{ex:91-241-become-stump-bobbed}%
\exmn{277.12}%
\begingl
	\glpreamble	ᴀkᴀguwu′ nasti′awe a′kᴀʟyêt wuʟ̣itsî′s, kᴀdag̣ā′x ᴀcdjī′ỵīt. //
	\glpreamble	A kagoowúx̱ nastée áwé a ḵatlyát wudlitsís, kadagáax̱ ash jeeÿeet. //
	\gla	{} {} A \rlap{kagoowúx̱} @ {} @ {} @ {} {}
			\rlap{nastée} @ {} @ {} @ {} @ {} {} \rlap{áwé} @ {} +
		{} a \rlap{ḵatlyát} @ {} @ {} {}
		\rlap{wudlitsís,} @ {} @ {} @ {} @ {} @ {} +
		{} \rlap{kadag̱áax̱} @ {} @ {} @ {} @ {}
			{} ash \rlap{jeeÿeet.} @ {} @ {} {} {} //
	\glb	{} {} a ká- gú -í -x̱ {}
			n- s- \rt[¹]{tiʰ} -μμH {} {} á -wé
		{} a ḵaatl- ÿá -t {}
		wu- d- l- i- \rt[¹]{tsis} -μH
		{} k- d- \rt[¹]{g̱ax̱} -μμH {}
			{} ash jee- ÿee -t {} {} //
	\glc	{}[\pr{CP} {}[\pr{PP} \xx{3n·pss} \xx{hsfc}- base -\xx{pss} -\xx{pert} {}]
			\xx{ncnj}- \xx{appl}- \rt[¹]{be} -\xx{var} \·\xx{sub} {}]
		\xx{foc} -\xx{mdst}
		{}[\pr{PP} \xx{3n·pss} flank- face -\xx{pnct} {}]
		\xx{pfv}- \xx{pasv}- \xx{csv}- \xx{stv}- \rt[¹]{bob} -\xx{var}
		{}[\pr{CP} \xx{qual}- \xx{mid}- \rt[¹]{cry} -\xx{var} \·\xx{sub}
			{}[\pr{PP} \xx{3prx·pss} poss‘n- below -\xx{pnct} {}] {}] //
	\gld	{} {} its \rlap{stump} {} {} -of {}
			\rlap{\xx{ncnj}.\xx{csec}.become} {} {} {} {} {} \rlap{it.is} {}
		{} his flank- face -to {}
		\rlap{\xx{zcnj}.\xx{pfv}.float} {} {} {} {} {}
		{} \rlap{\xx{gcnj}.\xx{impfv}.cry} {} {} {} {}
			{} its \rlap{burden} {} -at {} {} //
	\glft	‘Having become a stump it bobbed up to his flank, crying under its burden.’
		//
\endgl
\xe

\FIXME{What is up with the referent of \fm{ash} switching here from the boy to the monster?}

\ex\label{ex:91-242-cutting-all-up}%
\exmn{277.12}%
\begingl
	\glpreamble	Łdakᴀ′t ye ᴀckᴀ′nałx̣ᴀc. //
	\glpreamble	Ldakát yei ash kanalxásh. //
	\gla	Ldakát yei @ ash @ \rlap{kanalxásh.} @ {} @ {} @ {} @ {} //
	\glb	ldakát yei= ash= k- n- l- \rt[²]{xash} -μH //
	\glc	all down= \xx{3prx·o}= \xx{qual}- \xx{ncnj}- \xx{xtn}- \rt[²]{cut} -\xx{var} //
	\gld	all down it \rlap{3>3.\xx{g̱cnj}.\xx{prog}.cut} {} {} {} {} //
	\glft	‘He was cutting it all up.’
		//
\endgl
\xe

\ex\label{ex:91-243-swam-ashore}%
\exmn{277.13}%
\begingl
	\glpreamble	ᴀtxawe′ yên uwaq!ᴀ′q. //
	\glpreamble	Atx̱ áwé yan uwaxʼák. //
	\gla	{} \rlap{Atx̱} @ {} {} \rlap{áwé} @ {}
		yan @ \rlap{uwaxʼák.} @ {} @ {} @ {} //
	\glb	{} á -dáx̱ {} á -wé
		ÿán= u- i- \rt[¹]{xʼak} -μH //
	\glc	{}[\pr{PP} \xx{3n} -\xx{abl} {}] \xx{foc} -\xx{mdst}
		ashore= \xx{zpfv}- \xx{stv}- \rt[¹]{fish·swim} -\xx{var} //
	\gld	{} then -from {} \rlap{it.is} {}
		ashore \rlap{\xx{zcnj}.\xx{pfv}.fish·swim} {} {} {} //
	\glft	‘After that he swam ashore.’
		//
\endgl
\xe

\ex\label{ex:91-244-again-hung-over-stump}%
\exmn{277.13}%
\begingl
	\glpreamble	Ts!u atguwu′ nāx awate′. //
	\glpreamble	Tsu atgoowú náax̱ aawatee. //
	\gla	Tsu {} \rlap{atgoowú} @ {} @ {} \rlap{náax̱} @ {} {}
		\rlap{aawatee.} @ {} @ {} @ {} @ {} //
	\glb	tsu {} at= gú -í náa -x̱ {}
		a- wu- i- \rt[²]{ti} -μμL //
	\glc	again {}[\pr{PP} \xx{4n·pss} base -\xx{pss} covering -\xx{pert} {}]
		\xx{arg}- \xx{pfv}- \xx{stv}- \rt[²]{handle} -\xx{var} //
	\gld	again {} \rlap{stump} {} {} covering -on {}
		\rlap{3>3.\xx{ncnj}.\xx{pfv}.handle} {} {} {} {} //
	\glft	‘Again he hung it over a stump.’
		//
\endgl
\xe

\ex\label{ex:91-245-dawned-float-beach-head}%
\exmn{277.14}%
\begingl
	\glpreamble	Qē′na a′awe hᴀ′sduēg̣aya′ wułîx̣ā′c ᴀcā′ỵî. //
	\glpreamble	Ḵeina.áa áwé hasdu eig̱ayáa wulihaash a shaaÿí. //
	\gla	{} \rlap{Ḵeina.áa} @ {} @ {} @ {} @ {} {} \rlap{áwé} @ {}
		{} \rlap{hasdu} @ {} \rlap{eig̱ayáa} @ {} @ {} {}
		\rlap{wulihaash} @ {} @ {} @ {} @ {}
		{} a \rlap{shaaÿí.} @ {} {} //
	\glb	{} ḵei- n- \rt[¹]{.a} -μμH {} {} á -wé
		{} has= du eeͥḵ- ÿá -μ {}
		wu- l- i- \rt[¹]{hash} -μμL
		{} a shá -í {} //
	\glc	{}[\pr{CP} dawn- \xx{ncnj}- \rt[¹]{end·mv} -\xx{var} \·\xx{sub} {}]
		\xx{foc} -\xx{mdst}
		{}[\pr{PP} \xx{plh}= \xx{3h·pss} beach- face -\xx{loc} {}]
		\xx{pfv}- \xx{xtn}- \xx{stv}- \rt[¹]{float} -\xx{var}
		{}[\pr{DP} \xx{3n·pss} head -\xx{pss} {}] //
	\gld	{} \rlap{dawn.\xx{csec}.move} {} {} {} {} {} \rlap{it.is} {}
		{} \rlap{their} {} beach- face -at {}
		\rlap{\xx{ncnj}.\xx{pfv}.float} {} {} {} {}
		{} its head {} {} //
	\glft	‘Having dawned, it had floated to the beach below them, its head.’
		//
\endgl
\xe

\ex\label{ex:91-246-they-cut-it-up}%
\exmn{277.14}%
\begingl
	\glpreamble	Yē has ᴀkᴀ′nᴀx̣ᴀc. //
	\glpreamble	Yei has akanaxásh. //
	\gla	Yei @ has @ \rlap{akanaxásh.} @ {} @ {} @ {} @ {} //
	\glb	yei= has= a- k- n- \rt[²]{xash} -μμH //
	\glc	down= \xx{plh}= \xx{arg}- \xx{qual}- \xx{ncnj}- \rt[²]{cut} -\xx{var} //
	\gld	down they \rlap{3>3.\xx{ncnj}.\xx{prog}.cut} {} {} {} {} //
	\glft	‘They are cutting it up.’
		//
\endgl
\xe

\section{Paragraph 18}\label{sec:91-para-18}

\ex\label{ex:91-247-two-nights-took-canoe-beach}%
\exmn{278.1}%
\begingl
	\glpreamble	Aya′ dēx uxē′awe yākᵘ ỵēk ā′wacᴀt. //
	\glpreamble	Áyá déix̱ ux̱éi áwé yaakw ÿeiḵ aawashát. //
	\gla	\rlap{Áyá} @ {}
		{} déix̱ \rlap{ux̱éi} @ {} @ {} @ {} {}
		\rlap{áwé} @ {}
		{} yaakw {}
		ÿeiḵ @ \rlap{aawashát.} @ {} @ {} @ {} @ {} //
	\glb	á -yá
		{} déix̱ u- \rt[¹]{x̱eͥ} -μμH {} {}
		á -wé
		{} yaakw {}
		ÿeiḵ= a- wu- i- \rt[²]{shaʼt} -μH //
	\glc	\xx{foc} -\xx{prox}
		{}[\pr{CP} two \xx{irr}- \rt[¹]{overnight} -\xx{var} \·\xx{sub} {}]
		\xx{foc} -\xx{mdst}
		{}[\pr{DP} canoe {}]
		beach= \xx{arg}- \xx{pfv}- \xx{stv}- \rt[²]{grab} -\xx{var} //
	\gld	\rlap{it.is} {}
		{} two \rlap{\xx{zcnj}.\xx{csec}.overnight} {} {} {} {}
		\rlap{it.is} {}
		{} canoe {}
		beach \rlap{3>3.\xx{zcnj}.\xx{pfv}.grab} {} {} {} {} //
	\glft	‘So two nights having passed he took a canoe down the beach.’
		//
\endgl
\xe

\ex\label{ex:91-248-going-to-go-by-boat}%
\exmn{278.1}%
\begingl
	\glpreamble	Gog̣aqō′xwayu //
	\glpreamble	Gug̱aḵóox̱ áyú. //
	\gla	\rlap{Gug̱aḵóox̱} @ {} @ {} @ {} @ {} \rlap{áyú.} @ {}  //
	\glb	w- g- g̱- \rt[¹]{ḵux̱} -μμH á -yú //
	\glc	\xx{irr}- \xx{gcnj}- \xx{mod}- \rt[¹]{go·boat} -\xx{var} \xx{foc} -\xx{dist} //
	\gld	\rlap{\xx{prosp}.go·boat} {} {} {} {} \rlap{it.is} {} //
	\glft	‘He is going to go by boat.’
		//
\endgl
\xe

\ex\label{ex:91-249-having-left-came-to-beach}%
\exmn{278.1}%
\begingl
	\glpreamble	koqō′xsawe aēg̣ayā′t uwaqo′x, //
	\glpreamble	Gaḵóox̱ sáwé a eig̱ayáat uwaḵúx̱. //
	\gla	{} \rlap{Gaḵóox̱} @ {} @ {} @ {} {} 
		\rlap{sáwé} @ {} @ {}
		{} a \rlap{eig̱ayáat} @ {} @ {} {}
		\rlap{uwaḵúx̱.} @ {} @ {} @ {} //
	\glb	{} g- \rt[¹]{ḵux̱} -μμH {} {}
		s= á -wé
		{} a eeͥḵ- ÿá -t {}
		u- i- \rt[¹]{ḵux̱} -μH //
	\glc	{}[\pr{CP} \xx{gcnj}- \rt[¹]{go·boat} -\xx{var} \·\xx{sub} {}]
		\xx{q}= \xx{foc} -\xx{mdst}
		{}[\pr{PP} \xx{3n·pss} beach- face -\xx{pnct} {}]
		\xx{zpfv}- \xx{stv}- \rt[¹]{go·boat} -\xx{var} //
	\gld	{} \rlap{\xx{gcnj}.\xx{csec}.go·boat} {} {} {} {}
		maybe= \rlap{it.is} {}
		{} its beach- face -to {}
		\rlap{\xx{zcnj}.\xx{pfv}.go·boat} {} {} {} //
	\glft	‘Apparently having left, he came to the beach below it.’
		//
\endgl
\xe

The phrase that \citeauthor{swanton:1909} transcribes as \orth{koqō′xsawe} in (\lastx) seems to be a consecutive adjunct clause based on the root \fm{\rt[¹]{ḵux̱}} ‘go by boat’.
Given that this is a consecutive aspect form \fm{\xx{cnj}-…-μμH (áwé)}, the initial \orth{ko} is either one of the conjugation prefixes \fm{g-} or \fm{g̱-}, and there is no preceding postposition phrase.
The only two motion derivations that fit these requirements are (\fm{g̱-}; \fm{-ch} repetitive) ‘falling, downward’ and (\fm{g-}; \fm{-ch} repetitive) ‘starting off, picking up, upward’.
The \fm{g̱}-conjugation motion derivation is strange in this context, but the \fm{g}-conjugation derivation fits with its meaning ‘starting off’.

The \orth{s} of \orth{sawe} in (\lastx) cannot be analyzed as any kind of verb suffix, so it must be part of the focus particle \fm{sáwé}.
This \orth{s} would then be the wh-question particle \fm{sá}, here probably used as a dubitative.
The narrator thus implies that the event of departure and subsequent travel is not known and so must be inferred without any details given.
This deployment of a dubitative can be seen as a kind of rhetorical technique where the narrator jumps forward in time, advancing over events that are uninteresting by implying ignorance of them.

The phrase \fm{a eig̱ayáat} in (\lastx) has a nonhuman possessive pronoun \fm{a} ‘its’ with no obvious referent.
Given the sentence in (\nextx), this pronoun presumably refers to the house in which the woman is sitting.
It could alternatively be the woman herself which is implied by \citeauthor{swanton:1909}’s translation that runs together (\lastx) and (\nextx) as “he came up to the beach in front of a woman sitting in a house”.
The English translation for (\lastx) uses “it” which implies the house and not the woman.

\ex\label{ex:91-250-woman-sitting-inside}%
\exmn{278.2}%
\begingl
	\glpreamble	hᴀt cāwᴀ′t gâ′yu nełta′. //
	\glpreamble	Háʼ, shaawát gwáayú neilt áa. //
	\gla	Háʼ, {} \rlap{shaawát} @ {} {} \rlap{gwáayú} @ {} @ {}
		{} \rlap{neilt} @ {} {} \rlap{áa.} @ {} //
	\glb	háʼ {} sháaʷ- ÿát {} gwá= á -yú
		{} neil -t {} \rt[¹]{.a} -μμH //
	\glc	\xx{interj} {}[\pr{DP} woman- child {}] \xx{mir}= \xx{foc} -\xx{dist}
		{}[\pr{PP} home -\xx{pnct} {}] \rt[¹]{sit·\xx{sg}} -\xx{var} //
	\gld	hey {} \rlap{woman} {} {} really\• \rlap{it.is} {}
		{} inside -at {} \rlap{\xx{pos}·\xx{impfv}.sit·\xx{sg}} {} //
	\glft	‘Hey, there seems to be a woman sitting inside.’
		//
\endgl
\xe

\ex\label{ex:91-251-only-one-eye}%
\exmn{278.2}%
\begingl
	\glpreamble	Tca ʟēq! ỵᴀtî′ duwā′q. //
	\glpreamble	Chʼa tléixʼ ÿatee du waaḵ. //
	\gla	Chʼa tléixʼ \rlap{ÿatee} @ {} @ {}
		{} du waaḵ. {} //
	\glb	chʼa tléixʼ i- \rt[¹]{tiʰ} -μμL
		{} du waaḵ {} //
	\glc	just one \xx{stv}- \rt[¹]{be} -\xx{var}
		{}[\pr{DP} \xx{3h·pss} eye {}] //
	\gld	just one \rlap{\xx{ncnj}.\xx{impfv}.be} {} {}
		{} her eye {} //
	\glft	‘She has only one eye.’
		//
\endgl
\xe

\ex\label{ex:91-252-come-up-my-nephew}%
\exmn{278.3}%
\begingl
	\glpreamble	“Dāq gu ᴀxqē′łk!. //
	\glpreamble	«\!Daaḵ gú, ax̱ kéilkʼ. //
	\gla	«\!Daaḵ @ \rlap{gú,} @ {} @ {} @ {}
		{} ax̱ kéilkʼ. {} //
	\glb	\pqp{}dáaḵ= {} {} \rt[¹]{gut} -⊗
		{} ax̱ kéilkʼ {} //
	\glc	\pqp{}inland= \xx{zcnj}\· \xx{2sg·s}\· \rt[¹]{go·\xx{sg}} -\xx{var}
		{}[\pr{DP} \xx{1sg·pss} sor·neph {}] //
	\gld	\pqp{}up \rlap{\xx{zcnj}.\xx{imp}.you·\xx{sg}.go·\xx{sg}} {} {} {}
		{} my nephew {} //
	\glft	‘“Come up, my nephew.’
		//
\endgl
\xe

\ex\label{ex:91-253-I-have-kink}%
\exmn{278.3}%
\begingl
	\glpreamble	K!înk!awe xāu′, ᴀxqē′łk!,” //
	\glpreamble	Kʼínkʼ áwé x̱aa.óo, ax̱ kéilkʼ.\!» //
	\gla	{} Kʼínkʼ {} \rlap{áwé} @ {}
		\rlap{x̱aa.óo,} @ {} @ {} @ {}
		{} ax̱ kéilkʼ.\!» {} //
	\glb	{} kʼínkʼ {} á -wé
		x̱- i- \rt[²]{.u} -μμH
		{} ax̱ kéilkʼ {} //
	\glc	{}[\pr{DP} ferm·head {}] \xx{foc} -\xx{mdst}
		\xx{1sg·s}- \xx{stv}- \rt[²]{own} -\xx{var}
		{}[\pr{DP} \xx{1sg·pss} sor·neph {}] //
	\gld	{} ferm·head {} \rlap{it.is} {}
		\rlap{\xx{ncnj}.\xx{impfv}.I.own} {} {} {}
		{} my nephew {} //
	\glft	‘It is fermented fish heads that I have, my nephew.”’
		//
\endgl
\xe

\ex\label{ex:91-254-she-says-to-him}%
\exmn{278.3}%
\begingl
	\glpreamble	ʟe yu′acia′osîqa. //
	\glpreamble	tle yóo ash yawsiḵaa. //
	\gla	tle yóo @ ash @ \rlap{yawsiḵaa.} @ {} @ {} @ {} @ {} @ {} //
	\glb	tle yóo= ash= ÿ- wu- s- i- \rt[¹]{ḵa} -μμL //
	\glc	then \xx{quot}= \xx{3prx·o}= \xx{qual}- \xx{pfv}- \xx{csv}- \xx{stv}-
			\rt[¹]{say} -\xx{var} //
	\gld	then thus him \rlap{\xx{ncnj}.\xx{pfv}.say·to} {} {} {} {} {} //
	\glft	‘so she says to him then.’
		//
\endgl
\xe

\ex\label{ex:91-255-actually-her-into-her-possession}%
\exmn{278.4}%
\begingl
	\glpreamble	Xᴀtc dēca′ ho qo′nᴀx djīde′ yākᵘ nahā′ỵî, ᴀsiwé′ ayeg̣ayā′t uwaqo′x. //
	\glpreamble	X̱áchdei chʼa hú, ḵúnáx̱ du jeedé, yaakw naháaÿi ásíwé a eig̱ayáat uwaḵúx̱. //
	\gla	\rlap{X̱áchdei} @ {}
		chʼa {} hú, {}
		\rlap{ḵúnáx̱} @ {} {} du \rlap{jeedé,} @ {} {}
		{} yaakw \rlap{naháaÿi} @ {} @ {} @ {} {}
		\rlap{ásíwé} @ {} @ {}
		{} a \rlap{eig̱ayáat} @ {} @ {} {}
		\rlap{uwaḵúx̱.} @ {} @ {} @ {} //
	\glb	x̱ách= dei
		chʼa {} hú {}
		ḵú -náx̱ {} du jee -dé {}
		{} yaakw n- \rt[¹]{ha} -μμH -í {}
		á -sí -wé
		{} a eeͥḵ- ÿá -t {}
		u- i- \rt[¹]{ḵux̱} -μH //
	\glc	actually= now
		just {}[\pr{DP} \xx{3h} {}]
		\xx{areal} -\xx{perl} {}[\pr{PP} \xx{3h·pss} poss’n -\xx{all} {}]
		{}[\pr{DP} canoe \xx{ncnj}- \rt[¹]{mv·invis} -\xx{var} -\xx{nmz} {}]
		\xx{foc} -\xx{dub} -\xx{mdst}
		{}[\pr{PP} \xx{3n·pss} beach- face -\xx{pnct} {}]
		\xx{zpfv}- \xx{stv}- \rt[¹]{go·boat} -\xx{var} //
	\gld	actually now
		just {} her {} \rlap{really} {}
		{} her poss’n -to {}
		{} canoe \rlap{traveller} {} {} {} {}
		\rlap{it.is.apparently} {} {}
		{} its beach- face -to {}
		\rlap{\xx{zcnj}.\xx{pfv}.go·boat} {} {} {} //
	\glft	‘Actually it is just her, really into her possession, that the canoe travellers apparently came to the beach below her.’
		//
\endgl
\xe

\FIXME{The structure of the left periphery in (\lastx) is complex and hard to translate.
The phrase \fm{du jeedé} is lacking an overt possessor in \citeauthor{swanton:1909}’s transcription.
This could actually be some kind of weird topic left dislocation of \fm{hú} out from the possessor.}

\FIXME{The \fm{x̱áchdei} is interesting. It’s documented as \fm{x̱áshdei} ‘I supposed’ \parencite[05/84]{leer:1973} or \fm{ḵáshde} ‘I thought…’ \parencite[32]{leer:1991}.
The \fm{x̱ách} form suggests a connection to \fm{x̱ách} \~\ \fm{ḵách} \~\ \fm{x̱áju} \~\ \fm{ḵáju} ‘actually, in fact’, noted earlier but never expanded on by \textcite[21]{leer:1978b}.
This suggests either that the \fm{-u} is an optional suffix or that it is regularly syncopated, and that \fm{x̱áshdei} is constructed from \fm{x̱ách} \~\ \fm{x̱áju}.
The \fm{dei} could be either \fm{dei} ‘already, by now’ \parencite[32]{leer:1991} or \fm{déi} ‘now, this time’ \parencite[31]{leer:1991}, both of which \citeauthor{leer:1973} has repeatedly analyzed as the same basic lexical item \parencites[05/83]{leer:1973}[21]{leer:1978b}.
The dubitative particle \fm{shákdéi} ‘perhaps, probably’ \parencites[05/85]{leer:1973}[21]{leer:1978b}[30]{leer:1991} may also contain the same \fm{dei}.}

\ex\label{ex:91-256-human-heads-made-into-kink}%
\exmn{278.5}%
\begingl
	\glpreamble	Xᴀtc qācāỵē′ ayu′ k!î′nk!îx aołiyᴀ′x. //
	\glpreamble	X̱ách ḵaa shaaÿí áyú kʼínkʼix̱ awliyéx̱. //
	\gla	X̱ách {} ḵaa \rlap{shaaÿí} @ {} {} \rlap{áyú} @ {}
		{} \rlap{kʼínkʼix̱} @ {} {}
		\rlap{awliyéx̱.} @ {} @ {} @ {} @ {} @ {} //
	\glb	x̱ách {} ḵaa shá -í {} á -yú
		{} kʼinkʼ -x̱ {}
		a- wu- l- i- \rt[²]{yex̱} -μH //
	\glc	actually {}[\pr{DP} \xx{4h·pss} head -\xx{pss} {}] \xx{foc} -\xx{dist}
		{}[\pr{PP} ferm·head -\xx{pert} {}]
		\xx{arg}- \xx{pfv}- \xx{xtn}- \xx{stv}- \rt[²]{make} -\xx{var} //
	\gld	actually {} ppl’s \rlap{heads} {} {} \rlap{it.is} {}
		{} ferm·head -into {}
		\rlap{3>3.\xx{zcnj}.\xx{pfv}.make} {} {} {} {} {} //
	\glft	‘It was actually human heads that she had made into fermented fish heads.’
		//
\endgl
\xe

\ex\label{ex:91-257-didnt-eat}%
\exmn{278.5}%
\begingl
	\glpreamble	ʟēł awuxa′. //
	\glpreamble	Tléil awux̱á. //
	\gla	Tléil \rlap{awux̱á.} @ {} @ {} @ {} @ {} //
	\glb	tléil a- u- wu- \rt[²]{x̱a} -μH //
	\glc	\xx{neg} \xx{arg}- \xx{irr}- \xx{pfv}- \rt[²]{eat} -\xx{var} //
	\gld	not \rlap{3>3.\xx{zcnj}.\xx{pfv}.eat} {} {} {} {} //
	\glft	‘He did not eat them.’
		//
\endgl
\xe

\ex\label{ex:91-258-saw-what-made-of}%
\exmn{278.5}%
\begingl
	\glpreamble	Aosîtī′n ᴀ′xsîteỵe ᴀt. //
	\glpreamble	Awsiteen áx̱ siteeÿi át. //
	\gla	\rlap{Awsiteen} @ {} @ {} @ {} @ {} @ {}
		{} {} {} \rlap{áx̱} @ {} {} \rlap{siteeÿi} @ {} @ {} @ {} @ {} {} át. {} //
	\glb	a- wu- s- i- \rt[²]{tin} -μμL
		{} {} {} á -x̱ {} s- i- \rt[¹]{tiʰ} -μμL -i {} át {} //
	\glc	\xx{arg}- \xx{pfv}- \xx{xtn}- \xx{stv}- \rt[²]{see} -\xx{var}
		{}[\pr{DP} {}[\pr{CP} {}[\pr{PP} \xx{3n} -\xx{pert} {}]
			\xx{appl}- \xx{stv}- \rt[¹]{be} -\xx{var} -\xx{rel} {}] thing {}] //
	\gld	\rlap{3>3.\xx{g̱cnj}.\xx{pfv}.see} {} {} {} {} {}
		{} {} {} it -of {} \rlap{\xx{ncnj}.\xx{impfv}.be} {} {} {} -that {} thing {} //
	\glft	‘He saw the things that they were of.’
		//
\endgl
\xe

\ex\label{ex:91-259-I-also-have-fish-eggs}%
\exmn{278.6}%
\begingl
	\glpreamble	“Qᴀhā′kᵘ ts!u xā-u.” //
	\glpreamble	«\!Kaháakw tsú x̱aa.óo.\!» //
	\gla	{} \llap{«\!}\rlap{Kaháakw} @ {} @ {} {} tsú
		\rlap{x̱aa.óo.\!»} @ {} @ {} @ {} //
	\glb	{} k- \rt{hakw} -μμH {} tsú
		x̱- i- \rt[²]{.u} -μμH //
	\glc	{}[\pr{DP} \xx{qual}- \rt{??} -\xx{var} {}] also
		\xx{1sg·s}- \xx{stv}- \rt[²]{own} -\xx{var} //
	\gld	{} \rlap{fish·eggs} {} {} {} also
		\rlap{\xx{ncnj}.\xx{impfv}.I.own} {} {} {} //
	\glft	‘“I also have fish eggs.”’
		//
\endgl
\xe

\ex\label{ex:91-260-didnt-eat-human-eyes}%
\exmn{278.6}%
\begingl
	\glpreamble	Xᴀtc qā′wag̣ē ᴀsîyu′ ʟēł awuxa′. //
	\glpreamble	X̱ách ḵaa waag̱í ásíyú tléil awux̱á. //
	\gla	X̱ách {} ḵaa \rlap{waag̱í} @ {} {} \rlap{ásíyú} @ {} @ {}
		tléil \rlap{awux̱á.} @ {} @ {} @ {} @ {} //
	\glb	x̱ách {} ḵaa waaḵ -í {} á -sí -yú
		tléil a- u- wu- \rt[²]{x̱a} -μH //
	\glc	actually {}[\pr{DP} \xx{4h·pss} eye -\xx{pss} {}] \xx{foc} -\xx{dub} -\xx{dist}
		\xx{neg} \xx{arg}- \xx{irr}- \xx{pfv}- \rt[²]{eat} -\xx{var} //
	\gld	actually {} ppl’s \rlap{eyes} {} {} \rlap{it.is.apparently} {} {}
		not \rlap{3>3.\xx{zcnj}.\xx{pfv}.eat} {} {} {} {} //
	\glft	‘Actually it was apparently human eyes that he did not eat.’
		//
\endgl
\xe

\ex\label{ex:91-261-pour-out-by-fire}%
\exmn{278.6}%
\begingl
	\glpreamble	ᴀt ts!u gᴀntc!u′k! yêx aka′osîx̣a. //
	\glpreamble	Át tsú ganchʼóokʼ yax̱ akawsixaa. //
	\gla	{} Át {} tsú {} \rlap{ganchʼóokʼ} @ {} @ {} {}
		yax̱ @ \rlap{akawsixáa.} @ {} @ {} @ {} @ {} @ {} @ {} //
	\glb	{} át {} tsú {} gán- chʼóokʼ {} {}
		ÿáx̱= a- k- wu- s- i- \rt[²]{xa} -μμL //
	\glc	{}[\pr{DP} thing {}] also {}[\pr{PP} fire- corner -\xx{loc} {}]
		\xx{exh}= \xx{arg}- \xx{qual}- \xx{pfv}- \xx{xtn}- \xx{stv}-
			\rt[²]{pour} -\xx{var} //
	\gld	{} thing {} also {} fire- corner -at {}
		all \rlap{3>3.\xx{g̱cnj}.\xx{pfv}.pour} {} {} {} {} {} {} //
	\glft	‘Those things too he dumped out by the fire.’
		//
\endgl
\xe

\ex\label{ex:91-262-husband-absent}%
\exmn{278.7}%
\begingl
	\glpreamble	Doxo′x qo′a awe′ wuyê′x. //
	\glpreamble	Du x̱úx̱ ḵu.aa áwé uyéx̱. //
	\gla	{} Du x̱úx̱ {} ḵu.aa \rlap{áwé} @ {} \rlap{uyéx̱.} @ {} @ {} //
	\glb	{} du x̱úx̱ {} ḵu.aa á -wé u- \rt[¹]{yex̱} -μH //
	\glc	{}[\pr{DP} \xx{3h·pss} husband {}] \xx{contr} \xx{foc} -\xx{mdst}
		\xx{irr}- \rt[¹]{absent} -\xx{var} //
	\gld	{} her husband {} however \rlap{it.is} {}
		\rlap{\xx{ncnj}.\xx{impfv}.absent} {} {} //
	\glft	‘Her husband however is absent.’
		//
\endgl
\xe

\ex\label{ex:91-263-search-for-humans}%
\exmn{278.7}%
\begingl
	\glpreamble	Detc!a′ łīngî′t g̣a ā′ya qocī′. //
	\glpreamble	De chʼa leengítg̱aa áyá ḵushée. //
	\gla	De chʼa {} \rlap{leengítg̱aa} @ {} {} \rlap{áyá} @ {}
		\rlap{ḵushée.} @ {} @ {} //
	\glb	de chʼa {} leengít -g̱áa {} á -yá
		ḵu- \rt[¹]{shiʰ} -μμH //
	\glc	now just {}[\pr{PP} person -\xx{ades} {}] \xx{foc} -\xx{mdst}
		\xx{areal}- \rt[¹]{reach} -\xx{var} //
	\gld	now just {} person -for {} \rlap{it.is} {}
		\rlap{\xx{ncnj}.\xx{impfv}.search} {} {} //
	\glft	‘Now it is just for humans that he is searching.’
		//
\endgl
\xe

\ex\label{ex:91-264-last-put-out-human-ribs}%
\exmn{278.8}%
\begingl
	\glpreamble	Hūtc!aỵî′ sᴀ′kawe a′odihān łīngî′t s!ū′g̣o. //
	\glpreamble	Hóochʼaaÿí sákw áwé \{awdihaan\} leengít sʼóog̱u. //
	\gla	{} \rlap{Hóochʼaaÿí} @ {} @ {} sákw {} \rlap{áwé} @ {}
		\{awdihaan\}
		{} leengít \rlap{sʼóog̱u.} @ {} {} //
	\glb	{} hóochʼ= aa -í sákw {} á -wé
		a-w-d-i-\rt[¹]{han}-μμL
		{} leengít sʼóoḵ -í {} //
	\glc	{}[\pr{PP} finished= \xx{part} -\xx{pss} \xx{fut} {}] \xx{foc} -\xx{mdst}
		…-\rt[¹]{stand·\xx{sg}}-…
		{}[\pr{DP} person rib -\xx{pss} {}] //
	\gld	{} \rlap{last.one} {} {} for {} \rlap{it.is} {}
		\{??\}
		{} person \rlap{ribs} {} {} //
	\glft	‘It is for the last thing that she put out human ribs.’
		//
\endgl
\xe

\ex\label{ex:91-265-not-having-eaten-upset}%
\exmn{278.9}%
\begingl
	\glpreamble	Tc′atc!a ᴀ′qoa ᴀłuxā′awe ᴀctū′n wute′. //
	\glpreamble	Cha chʼa á ḵu.aa hél oox̱áa áwé ash tóon wootee. //
	\gla	Cha chʼa {} á {} ḵu.aa
		{} hél \rlap{oox̱áa} @ {} @ {} @ {} @ {} @ {} {} \rlap{áwé} @ {} +
		{} ash \rlap{tóon} @ {} {}
		\rlap{wootee.} @ {} @ {} @ {} //
	\glb	cha chʼa {} á {} ḵu.aa
		{} hél a- u- {} \rt[²]{x̱a} -μμH {} {} á -wé
		{} ash tú -n {}
		wu- i- \rt[¹]{tiʰ} -μμL //
	\glc	\xx{interj} just {}[\pr{DP} \xx{3n} {}] \xx{contr}
		{}[\pr{CP} \xx{neg} \xx{arg}- \xx{irr}- \xx{zcnj}\·
			\rt[²]{eat} -\xx{var} \·\xx{sub} {}] \xx{foc} -\xx{mdst}
		{}[\pr{PP} \xx{3prx·pss} inside -\xx{instr} {}]
		\xx{pfv}- \xx{stv}- \rt[¹]{be} -\xx{var} //
	\gld	so then {} it {} however
		{} not \rlap{3>3.\xx{zcnj}.\xx{csec}.eat} {} {} {} {} {} {} \rlap{it.is} {}
		{} her mind -with {}
		\rlap{\xx{ncnj}.\xx{pfv}.be} {} {} {} //
	\glft	‘So then it however, him not having eaten it, she was upset.’
		//
\endgl
\xe

\FIXME{Discuss idiom \fm{du tóon wootee}.}

\ex\label{ex:91-266-threw-shell-at-him}%
\exmn{278.9}%
\begingl
	\glpreamble	ᴀ′cqosa īn yīs! ᴀc ỵītī′t āwag̣ē′q!. //
	\glpreamble	Ách ḵusa.een yéesʼ ash ÿeetéet aawag̱éxʼ. //
	\gla	{} {} {} \rlap{Ách} @ {} {} \rlap{ḵusa.een} @ {} @ {} @ {} @ {} {} yéesʼ {} +
		{} ash \rlap{ÿeetéet} @ {} {}
		\rlap{aawag̱éxʼ.} @ {} @ {} @ {} @ {} //
	\glb	{} {} {} á -ch {} ḵu- s- \rt[²]{.in} -μμL {} {} yéesʼ {}
		{} ash ÿeetí -t {}
		a- wu- i- \rt[²]{g̱eͥxʼ} -μH //
	\glc	{}[\pr{DP} {}[\pr{CP} {}[\pr{PP} \xx{3n} -\xx{instr} {}]
			\xx{areal}- \xx{appl}- \rt[²]{kill·\xx{pl}} -\xx{var} \·\xx{rel} {}]
			lg·mus·shell {}]
		{}[\pr{PP} \xx{3prx·pss} remains -\xx{pnct} {}]
		\xx{arg}- \xx{pfv}- \xx{stv}- \rt[²]{throw·inan} -\xx{var} //
	\gld	{} {} {} it -with {}
			\rlap{ppl.\xx{ncnj}.\xx{impfv}.\xx{appl}.kill·\xx{pl}} {} {} {} {} {}
			lg·mus·shell {}
		{} his place -to {}
		\rlap{3>3.\xx{zcnj}.\xx{pfv}.throw·inan} {} {} {} {} //
	\glft	‘She threw a large mussel shell with which she killed people at the place where he was.’
		//
\endgl
\xe

\FIXME{Discuss species identification of \fm{yéesʼ}.}

\ex\label{ex:91-267-he-jumped-up-off}%
\exmn{278.10}%
\begingl
	\glpreamble	Ā′ỵet kē wudjîg̣ᴀ′n. //
	\glpreamble	Áa ÿetx̱ kei wujikʼén. //
	\gla	{} \rlap{Áa} @ {} {} \rlap{ÿetx̱} @ {} @ kei @
		\rlap{wujikʼén.} @ {} @ {} @ {} @ {} @ {} //
	\glb	{} á -μ {} ÿé -dáx̱= kei= wu- d- sh- i- \rt[¹]{kʼeʼn} -μH //
	\glc	{}[\pr{PP} \xx{3n} -\xx{loc} {}] place -\xx{abl}= up=
		\xx{pfv}- \xx{mid}- \xx{pej}- \xx{stv}- \rt[¹]{jump} -\xx{var} //
	\gld	{} there -at {} \rlap{off} {} up \rlap{\xx{zcnj}.\xx{pfv}.jump} {} {} {} {} {} //
	\glft	‘He jumped up off there.’
		//
\endgl
\xe

\ex\label{ex:91-268-grabbed-it}%
\exmn{278.10}%
\begingl
	\glpreamble	ʟe āx āwacā′t. //
	\glpreamble	Tle aax̱ aawasháat. //
	\gla	Tle {} \rlap{aax̱} @ {} {} \rlap{aawasháat.} @ {} @ {} @ {} @ {} //
	\glb	tle {} á -dáx̱ {} a- wu- i- \rt[²]{shaʼt} -μμH //
	\glc	then {}[\pr{PP} \xx{3n} -\xx{abl} {}]
		\xx{arg}- \xx{pfv}- \xx{stv}- \rt[²]{grab} -\xx{var} //
	\gld	then {} it -from {} \rlap{3>3.\xx{gcnj}.\xx{pfv}.grab} {} {} {} {} //
	\glft	‘Then he grabbed it from there.’
		//
\endgl
\xe

\ex\label{ex:91-269-hit-her-with-it}%
\exmn{278.10}%
\begingl
	\glpreamble	Tcucyā′q! ᴀc āca′oʟ̣itsu. //
	\glpreamble	Chush yáaxʼ ách ashawlidzóo. //
	\gla	{} Chush \rlap{yáaxʼ} @ {} {} {} \rlap{ách} @ {} {}
		\rlap{ashawlidzóo.} @ {} @ {} @ {} @ {} @ {} @ {} //
	\glb	{} chush ÿá -xʼ {} {} á -ch {}
		a- sha- wu- l- i- \rt[²]{dzuʰ} -μμH //
	\glc	{}[\pr{PP} \xx{rflx·pss} face -\xx{loc} {}] {}[\pr{PP} \xx{3n} -\xx{instr} {}]
		\xx{arg}- head- \xx{pfv}- \xx{appl}- \xx{stv}- \rt[²]{throw} -\xx{var} //
	\gld	{} self’s face -at {} {} it -with {}
		\rlap{3>3.head.\xx{zcnj}.\xx{pfv}.throw·missile} {} {} {} {} {} {} //
	\glft	‘In return he hit her with it.’
		//
\endgl
\xe

\FIXME{Discuss \fm{chush yáaxʼ} idiom.
Also appears in \fm{chush yáa ash wudzinei} “retaliated against him” \parencite[277]{leer:1976}.
Perhaps similar to \fm{chush daadé asawdihaa} “he wanted to be with him” \parencite[36]{leer:1976}?}

\ex\label{ex:91-270-she-broke-apart}%
\exmn{278.11}%
\begingl
	\glpreamble	ʟa wū′cdᴀq wuʟ!ī′k yucā′wᴀt //
	\glpreamble	Tle wóoshdáx̱ woolʼéexʼ yú shaawát; //
	\gla	Tle {} \rlap{wóoshdáx̱} @ {} {}
		\rlap{woolʼéexʼ,} @ {} @ {} @ {}
		{} yú \rlap{shaawát;} @ {} {} //
	\glb	tle {} wóosh =dáx̱ {}
		wu- i- \rt[¹]{lʼixʼ} -μμH
		{} yú sháaʷ- ÿát {} //
	\glc	then {}[\pr{PP} \xx{recip} =\xx{abl} {}]
		\xx{pfv}- \xx{stv}- \rt[¹]{break} -\xx{var}
		{}[\pr{DP} \xx{dist} woman- child {}] //
	\gld	then {} ea·oth \•from {}
		\rlap{\xx{ncnj}.\xx{pfv}.break} {} {} {}
		{} that \rlap{woman} {} {} //
	\glft	‘Then she broke apart, that woman;’
		//
\endgl
\xe

\FIXME{Note same use of \fm{\rt[¹]{lʼixʼ}} ‘break, snap’ in (\ref{ex:91-111-she-broke-apart}).}

\ex\label{ex:91-271-actually-cannibal-wife}%
\exmn{278.11}%
\begingl
	\glpreamble	xᴀtc qo′sa xakā′ cᴀdayu′. //
	\glpreamble	x̱ách ḵusax̱á ḵáa shát áyú. //
	\gla	x̱ách {} {} \rlap{ḵusax̱á} @ {} @ {} @ {} @ {} {}
			ḵáa shát {} \rlap{áyu.} @ {} //
	\glb	x̱ách {} {} ḵu- s- \rt[²]{x̱a} -μH {} {}
			ḵáa shát {} á -yú //
	\glc	actually {}[\pr{DP} {}[\pr{CP} \xx{4h·o}- \xx{xtn}-
			\rt[²]{eat} -\xx{var} \·\xx{rel} {}]
			man wife {}] \xx{foc} -\xx{dist} //
	\gld	actually {} {} \rlap{ppl.\xx{zcnj}.\xx{impfv}.eat} {} {} {} {} {} 
			man wife {} \rlap{it.is} {} //
	\glft	‘actually she is the wife of the cannibal man.’
		//
\endgl
\xe

\ex\label{ex:91-272-dragged-onto-fire}%
\exmn{278.12}%
\begingl
	\glpreamble	Adjā′g̣awe tc!uʟe′ gᴀłqādā′ga awaxo′t!. //
	\glpreamble	Ajáaḵ áwé chʼu tle ganaltáa dáagi aawax̱útʼ. //
	\gla	{} \rlap{Ajáaḵ} @ {} @ {} @ {} @ {} {} \rlap{áwé} @ {}
		chʼu tle {} \rlap{ganaltáa} @ {} @ {} {}
		\rlap{dáagi} @ {} @ \rlap{aawax̱útʼ.} @ {} @ {} @ {} @ {} //
	\glb	{} a- {} \rt[²]{jaḵ} -μμH {} {} á -wé
		chʼu tle {} \rt[¹]{gan}- ltáaᵏ {} {}
		dáak -í= a- wu- i- \rt[²]{x̱utʼ} -μH //
	\glc	{}[\pr{CP} \xx{arg}- \xx{zcnj}\· \rt[²]{kill} -\xx{var} \·\xx{sub} {}]
		\xx{foc} -\xx{mdst}
		just then {}[\pr{PP} \rt[¹]{burn}- middle \·\xx{loc} {}]
		inland -\xx{loc}= \xx{arg}- \xx{pfv}- \xx{stv}- \rt[²]{pull} -\xx{var} //
	\gld	{} \rlap{3>3.\xx{zcnj}.\xx{csec}.kill} {} {} {} {} {} \rlap{it.is} {}
		just then {} fire- middle -in {}
		\rlap{onto·fire} {} \rlap{3>3.\xx{zcnj}.\xx{pfv}.pull} {} {} {} {} //
	\glft	‘Having killed her, he then dragged her onto the middle of the fire.’
		//
\endgl
\xe

\FIXME{Discuss interpretation of \orth{gᴀłqādā′ga} which looks like \fm{galḵaadáagi} but that is nonsense.}

\ex\label{ex:91-273-ashes-became-mosquitoes}%
\exmn{278.12}%
\begingl
	\glpreamble	Akᴀ′ʟ!t!ê qo′a awe′ tc!uʟe′ awułiū′x̣ ā′we tā′q!ax osîte′. //
	\glpreamble	A kélʼtʼi ḵu.aa áwé chʼu tle awu̬li.óoxu aa wé táaxʼaax̱ wusitee. //
	\gla	{} A \rlap{kélʼtʼi} @ {} @ {} @ {} {} ḵu.aa \rlap{áwé} @ {} +
		chʼu tle {} {} \rlap{awu̬li.óoxu} @ {} @ {} @ {} @ {} @ {} @ {} {} aa {} +
		{} {} wé \rlap{táaxʼaax̱} @ {} @ {} {} {} {}
		\rlap{wusitee.} @ {} @ {} @ {} @ {} //
	\glb	{} a \rt[¹]{kelʼ} -μH -tʼ -í {} ḵu.aa á -wé
		chʼu tle {} {} a- wu- l- i- \rt[²]{.uʼx} -μμH -i {} aa {}
		{} {} wé \rt[²]{taxʼ} -μμH -áa {} -x̱ {}
		wu- s- i- \rt[¹]{tiʰ} -μμL //
	\glc	{}[\pr{DP} \xx{3n·pss} \rt[¹]{ash} -\xx{var} -\xx{rep} -\xx{pss} {}]
		\xx{contr} \xx{foc} -\xx{mdst}
		just then {}[\pr{DP} {}[\pr{CP} \xx{arg}- \xx{pfv}- \xx{xtn}- \xx{stv}-
				\rt[²]{blow} -\xx{var} -\xx{rel} {}] \xx{part} {}]
		{}[\pr{PP} {}[\pr{DP} \xx{dist} \rt[²]{bite} -\xx{var} -\xx{nmz} {}]
			-\xx{pert} {}]
		\xx{pfv}- \xx{appl}- \xx{stv}- \rt[¹]{be} -\xx{var} //
	\gld	{} its \rlap{ashes} {} {} {} {} however \rlap{it.is} {}
		just then {} {} \rlap{3>3.\xx{ncnj}.\xx{pfv}.blow·on} {} {} {} {} {} -that {}
			one {}
		{} {} that \rlap{mosquito} {} {} {} -of {}
		\rlap{\xx{ncnj}.\xx{pfv}.be} {} {} {} {} //
	\glft	‘Its ashes however, the ones that he blew on became mosquitoes.’
		//
\endgl
\xe

\ex\label{ex:91-274-eat-people-mosquitoes}%
\exmn{278.13}%
\begingl
	\glpreamble	ᴀtcawe qo′saxa tā′q!a. //
	\glpreamble	Ách áwé ḵusax̱á táaxʼaa. //
	\gla	{} \rlap{Ách} @ {} {} \rlap{áwé} @ {}
		\rlap{ḵusax̱á} @ {} @ {} @ {}
		{} \rlap{táaxʼaa.} @ {} @ {} {} //
	\glb	{} á -ch {} á -wé
		ḵu- s- \rt[²]{x̱a} -μH
		{} \rt[²]{taxʼ} -μμH -áa {} //
	\glc	{}[\pr{PP} \xx{3n} -\xx{erg} {}] \xx{foc} -\xx{mdst}
		\xx{4h·o}- \xx{xtn}- \rt[²]{eat} -\xx{var}
		{}[\pr{DP} \rt[²]{bite} -\xx{var} -\xx{nmz} {}] //
	\gld	{} it -because {} \rlap{it.is} {}
		\rlap{ppl.\xx{zcnj}.\xx{impfv}.eat} {} {} {}
		{} \rlap{mosquito} {} {} {} //
	\glft	‘Because of that they eat people, mosquitoes.’
		//
\endgl
\xe

\ex\label{ex:91-275-killed-went}%
\exmn{278.13}%
\begingl
	\glpreamble	ᴀtcā′g̣awe wuq′ox. //
	\glpreamble	Ajáaḵ áwé wooḵoox̱. //
	\gla	{} \rlap{Ajáaḵ} @ {} @ {} @ {} @ {} {} \rlap{áwé} @ {}
		\rlap{wooḵoox̱.} @ {} @ {} @ {}  //
	\glb	{} a- {} \rt[²]{jaḵ} -μμH {} {} á -wé
		wu- i- \rt[¹]{ḵux̱} -μμL //
	\glc	{}[\pr{CP} \xx{arg}- \xx{zcnj}\· \rt[²]{kill} -\xx{var} \·\xx{sub} {}]
		\xx{foc} -\xx{mdst}
		\xx{pfv}- \xx{stv}- \rt[¹]{go·boat} -\xx{var} //
	\gld	{} \rlap{3>3.\xx{zcnj}.\xx{csec}.kill} {} {} {} {} {} \rlap{it.is} {}
		\rlap{\xx{ncnj}.\xx{pfv}.go·boat} {} {} {} //
	\glft	‘Having killed her he went by boat.’
		//
\endgl
\xe

\ex\label{ex:91-276-came-against-cannibal}%
\exmn{279.1}%
\begingl
	\glpreamble	ᴀcgē′t uwaqo′x weqo′saxa qoan. //
	\glpreamble	Ash géit uwaḵúx̱ wé ḵusax̱á ḵwáan. //
	\gla	{} Ash \rlap{géit} @ {} {}
		\rlap{uwaḵúx̱} @ {} @ {} @ {} +
		{} wé {} \rlap{ḵusax̱á} @ {} @ {} @ {} @ {} {} ḵwáan. {} //
	\glb	{} ash géi -t {}
		u- i- \rt[¹]{ḵux̱} -μH
		{} wé {} ḵu- s- \rt[²]{x̱a} -μH {} {} ḵwáan {} //
	\glc	{}[\pr{PP} \xx{3prx·pss} against -\xx{pnct} {}]
		\xx{zpfv}- \xx{stv}- \rt[¹]{go·boat} -\xx{var}
		{}[\pr{DP} \xx{mdst} {}[\pr{CP} \xx{4h·o}- \xx{xtn}-
			\rt[²]{eat} -\xx{var} \·\xx{rel} {}] people {}] //
	\gld	{} him against -to {}
		\rlap{\xx{zcnj}.\xx{pfv}.go·boat} {} {} {} 
		{} that {} \rlap{ppl.\xx{zcnj}.\xx{impfv}.eat} {} {} {} -that {} person {} //
	\glft	‘It came against him, the cannibal.’
		//
\endgl
\xe

\ex\label{ex:91-277-he-killed-it}%
\exmn{279.1}%
\begingl
	\glpreamble	Ā′wadjᴀq ᴀcgē′t qōx. //
	\glpreamble	Aawajáḵ, ash géit ḵóox̱. //
	\gla	\rlap{Aawajáḵ,} @ {} @ {} @ {} @ {}
		{} {} ash \rlap{géit} @ {} {} \rlap{ḵóox̱.} @ {} @ {} @ {} @ {} {} //
	\glb	a- wu- i- \rt[²]{jaḵ} -μH
		{} {} ash géi -t {} {} \rt[¹]{ḵux̱} -μμH {} {} //
	\glc	\xx{arg}- \xx{pfv}- \xx{stv}- \rt[²]{kill} -\xx{var}
		{}[\pr{CP} {}[\pr{PP} \xx{3prx·pss} against -\xx{pnct} {}]
			\xx{zcnj}- \rt[¹]{go·boat} -\xx{var} \·\xx{sub} {}] //
	\gld	\rlap{3>3.\xx{zcnj}.\xx{pfv}.kill} {} {} {} {}
		{} {} his against -to {} \rlap{\xx{zcnj}.\xx{csec}.go·boat} {} {} {} {} {} //
	\glft	‘He killed it, it having come against him.’
		//
\endgl
\xe

\FIXME{This is a remarkable instance of a consecutive coming after the main verb. It can’t be the start of the following sentence because the temporal ordering sequence is ‘come against’ and then ‘kill’, where the next sentence entails ‘kill’ for ‘break off head’, but ‘kill’ entails ‘come against’, i.e.\ ‘come against’ > ‘kill’ > ‘break off head’ and not *‘kill’ > ‘come against’ > ‘break off head’.}

\ex\label{ex:91-278-break-off-head}%
\exmn{279.1}%
\begingl
	\glpreamble	ᴀcaỵī′ āx āwaʟī′q!. //
	\glpreamble	A shaaÿí aax̱ aawalʼéexʼ. //
	\gla	{} A \rlap{shaaÿí} @ {} {} {} \rlap{aax̱} @ {} {}
		\rlap{aawalʼéexʼ.} @ {} @ {} @ {} @ {} //
	\glb	{} a shá -í {} {} á -dáx̱ {}
		a- wu- i- \rt[²]{lʼixʼ} -μμH //
	\glc	{}[\pr{DP} \xx{3n·pss} head -\xx{pss} {}] {}[\pr{PP} \xx{3n} -\xx{abl} {}]
		\xx{arg}- \xx{pfv}- \xx{stv}- \rt[²]{break} -\xx{var} //
	\gld	{} its head {} {} {} it -off {}
		\rlap{3>3.\xx{ncnj}.\xx{pfv}.break} {} {} {} {} //
	\glft	‘He snapped off its head.’
		//
\endgl
\xe

\ex\label{ex:91-279-paddle-to-town-into-possession}%
\exmn{279.2}%
\begingl
	\glpreamble	Āndê′ aỵa′waxa duʟā-hᴀ′sdjidê. //
	\glpreamble	Aandé aÿaawax̱áa du tláa hás jeedé. //
	\gla	{} \rlap{Aandé} @ {} {}
		\rlap{aÿaawax̱áa} @ {} @ {} @ {} @ {} @ {}
		{} du tláa @ \•hás \rlap{jeedé} @ {} {} //
	\glb	{} aan -dé {}
		a- ÿ- wu- i- \rt[²]{x̱a} -μμH
		{} du tláa =hás jee -dé {} //
	\glc	{}[\pr{PP} town -\xx{all} {}]
		\xx{xpl}- \xx{qual}- \xx{pfv}- \xx{stv}- \rt[²]{paddle} -\xx{var}
		{}[\pr{PP} \xx{3h·pss} mother =\xx{plh} poss’n -\xx{all} {}] //
	\gld	{} town -to {}
		\rlap{\xx{zcnj}.\xx{pfv}.paddle} {} {} {} {} {}
		{} his mother =s poss’n -to {} //
	\glft	‘He paddled it to town into his mothers’ possession.’
		//
\endgl
\xe

\FIXME{The verb \fm{aÿaawax̱áa} isn’t actually transitive – note \fm{a-} \xx{xpl} – though the English translation suggests this. The adjunct PP \fm{du tláa hás jeedé} is ambiguous whether it is the head or the protagonist that goes into his mothers’ possession. Pragmatic knowledge allows us to conclude that it is the head, but then the events of handling and transfer are all implicit.}

\ex\label{ex:91-280-shred-face}%
\exmn{279.2}%
\begingl
	\glpreamble	Aỵakā′wahᴀn. //
	\glpreamble	Aÿakaawahán. //
	\gla	\rlap{Aÿakaawahán.} @ {} @ {} @ {} @ {} @ {} @ {} //
	\glb	a- ÿ- k- wu- i- \rt[²]{han} -μH //
	\glc	\xx{arg}- face- \xx{qual}- \xx{pfv}- \xx{stv}- \rt[²]{cut·strips} -\xx{var} //
	\gld	\rlap{3>3.face.\xx{zcnj}.\xx{pfv}.cut·strips} {} {} {} {} {} {} //
	\glft	‘They shredded its face.’
		//
\endgl
\xe

\ex\label{ex:91-281-toast-with-crap}%
\exmn{279.3}%
\begingl
	\glpreamble	Hāʟ! tîn hᴀs aỵā′wat!us!. //
	\glpreamble	Háatlʼ tin has aÿaawatʼúsʼ. //
	\gla	{} {} \rlap{Háatlʼ} @ {} {} tin {}
		has @ \rlap{aÿaawatʼúsʼ.} @ {} @ {} @ {} @ {} @ {} //
	\glb	{} {} \rt[²]{hatlʼ} -μμH {} tin {}
		has= a- ÿ- wu- i- \rt[²]{tʼusʼ} -μH //
	\glc	{}[\pr{PP} {}[\pr{NP} \rt[²]{crap} -\xx{var} {}] \xx{instr} {}]
		\xx{plh}= \xx{arg}- face- \xx{pfv}- \xx{stv}- \rt[²]{toast} -\xx{var} //
	\gld	{} {} \rlap{crap} {} {} with {}
		they \rlap{3>3.face.\xx{zcnj}.\xx{pfv}.toast} {} {} {} {} {} //
	\glft	‘They toasted its face with crap.’
		//
\endgl
\xe

\section{Paragraph 19}\label{sec:91-para-19}

\ex\label{ex:91-282-whenever-done-telling-people-say}%
\exmn{279.4}%
\begingl
	\glpreamble	ʟāgu yên qᴀx duł-nîgî′n ye qoỵanaqe′tc, //
	\glpreamble	Tlaagú yan kax̱dulnígín yéi ḵuÿanaḵéich //
	\gla	{} {} Tlaagú {}
			yan @ \rlap{kax̱dulnígín} @ {} @ {} @ {} @ {} @ {} @ {} @ {} @ {} {}
		yéi @ \rlap{ḵuÿanaḵéich} @ {} @ {} @ {} @ {} @ {} //
	\glb	{} {} tlaagú {} ÿán= k- {} g̱- du- d- l- \rt[²]{nik} -μH -ín {}
		yéi= ḵu- ÿ- n- \rt[¹]{ḵa} -eμH -ch //
	\glc	{}[\pr{CP} {}[\pr{DP} old·story {}]
			\xx{term}= \xx{qual}- \xx{zcnj}\· \xx{mod}-
				\xx{4h·s}- \xx{mid}- \xx{xtn}- \rt[²]{relate}
					-\xx{var} -\xx{ctng} {}]
		thus= \xx{4h·s}- \xx{qual}- \xx{ncnj}- \rt[¹]{say} -\xx{var} -\xx{rep} //
	\gld	{} {} old·story {} done
			\rlap{\xx{zcnj}.\xx{ctng}.ppl.tell} {} {} {} {} {} {} {} {} {}
		thus \rlap{ppl.\xx{hab}.say} {} {} {} {} {} //
	\glft	‘Whenever they are done telling an old story people would say’
		//
\endgl
\xe

\FIXME{Is this the first instance of \fm{tlaagú}? If so, discuss its meaning and etymology here.}

\ex\label{ex:91-283-done-coughing}%
\exmn{279.4}%
\begingl
	\glpreamble	“Hūtc! qêłqᴀ′x.” //
	\glpreamble	«\!Hóochʼ x̱á alḵáx̱\!». //
	\gla	«\!Hóochʼ x̱á \rlap{alḵáx̱\!».} @ {} @ {} @ {} @ {} //
	\glb	\pqp{}hóochʼ x̱á a- l- \rt[²]{ḵax̱} -μH {} //
	\glc	\pqp{}finished \xx{pcl} \xx{xpl}- \xx{xtn}- \rt[²]{cough} -μμH \·\xx{nmz}? //
	\gld	\pqp{}finished indeed \rlap{\xx{zcnj}.\xx{impfv}.cough} {} {} {} {} //
	\glft	‘“Done coughing”.’
		//
\endgl
\xe

\FIXME{Discuss this idiom \parencite[f01/61]{leer:1973}.}