%!TEX root = ../swanton-texts.tex
%%
%% 93. Strong Man (289-291)
%%

\resetexcnt
\chapter{Kaháasʼi: Strong Man}\label{ch:93-strong-man}

This narrative was told to \citeauthor{swanton:1909} by \fm{Deikeenaakʼw} John Morris in Sitka.
In the original publication it is number 93, running from page 289 to 291 and totalling 39 lines of glossed transcription.
\fm{Ḵeixwnéi} Nora Marks Dauenhauer transliterated this text twice, once in the late 1960s or early 1970s into the first Naish-Story orthography, and then again in the mid 1970s in the Revised Popular orthography; both versions are archived at the Alaska Native Language Archive \parencite{dauenhauer:1971b}, and both have been consulted for this retranscription.
\citeauthor{swanton:1909}’s original title is “Kāhā′s!î, The Strong Man” which influenced later English titles for this story.
This name \fm{Kaháasʼi} is reflected in \fm{Ḵaadashaan}’s and \fm{Ḵaa Chʼaatí} Peter Lawrence’s versions as \fm{Atkaháasʼi} \parencites[146]{swanton:1909}[890]{de-laguna:1972} and in \fm{Shgwínde} William Paul’s summary from \fm{Shaadaaxʼ} Robert Zuboff (see below).
This story is more often known in Tlingit by the main character’s other name \fm{Duktʼootlʼ} which is a variation on \fm{dook} ‘skin’ + \fm{tʼoochʼ} ‘charcoal, black’, as for example in \cite[136–151]{dauenhauer:1987}.
But because the name \fm{Duktʼootlʼ} does not appear anywhere in this narrative the name \fm{Kaháasʼi} is used instead.

This particular narrative has some odd ambiguities and its flow is uneven with apparent gaps in the plot.
This could be attributed to the transcription task: \citeauthor{swanton:1909} was writing down each sentence individually so that the narrator would have to pause while \citeauthor{swanton:1909} transcribed it and then confirmed his transcription.
Because other narratives from \fm{Deikeenaakʼw} are more fluent, it seems likely that this was one of the first of \citeauthor{swanton:1909}’s transcriptions with him, and hence one of \citeauthor{swanton:1909}’s first attempts to transcribe Tlingit \parencite[??]{jones:2017}.
As an early attempt at recorded narrative, the disfluencies here would reflect the fact that both men were still developing a process for effectively delivering and capturing spoken narrative on paper.
Even though the discourse structure is somewhat discombobulated, the individual sentences are still excellent examples of complex syntactic structures and some feature rare vocabulary.

This story has been recorded many times and remains popular in Tlingit society today.
\citeauthor{swanton:1909} includes a much longer English version of this story from \fm{Ḵaadashaan} John Kadashan in Wrangell as part of \fm{Ḵaadashaan}’s epic Raven cycle (number 31, pp.\ 145–150).
Perhaps the most well known version in Tlingit was recorded by the Dauenhauers from \fm{Táakw Kʼwátʼi} Frank Johnson \parencite[136–151]{dauenhauer:1987}.
\citeauthor{de-laguna:1972} includes a version in English from \fm{Ḵaa Chʼaatí} Peter Lawrence \parencite[890–892]{de-laguna:1972}.
McClellan records one version from \fm{Yéil Kʼidáa} Jimmy Scotty James \parencite[439–440]{mcclellan-cruikshank:2007b} and another from \fm{Kax̱yéik} Pansy Bailey \parencite[684–687]{mcclellan-cruikshank:2007c}.
\citeauthor{paul:1930}’s copy of \citeauthor{swanton:1909}’s volume includes a loose paper with a typed summary of the story from \fm{Shaadaaxʼ} Robert Zuboff dated 1949 \parencite[289]{paul:1930}; for reference this summary is included after \citeauthor{swanton:1909}’s translation.
Attesting to this story’s continued popularity, there are also several audio versions in Tlingit and English among the Dauenhauer recordings by \fm{Shaadaaxʼ} Robert Zuboff, \fm{Kéet Yanaayí} Willie Marks, \fm{Yeilnaawú} Tom Peters, and \fm{Gunaanastí} Tom Ukas, all of which have yet to be transcribed.

\clearpage
\begin{pairs}
\begin{Leftside}
\beginnumbering
\pstart
\snum{1}Wudishúch áyú yú ḵu.oo, latseen káxʼ.
\snum{2}Aa x̱ooxʼ áyú yéi ÿatee ḵáa, Kaháasʼi.
\snum{3}Tlax̱ ḵʼanashgi\-déi.
\snum{4}Adawóotl káxʼ áyú dashóoch yú aantḵeení.
\snum{5}Hú ḵu.aa ḵʼanashgidéix̱ sitee.
\snum{6}Héendáx̱ daaḵ ag̱a.ádín áwé koodushooḵch.
\snum{7}Koolx̱ʼésch, át áwe nateich.
\snum{8}X̱ách chʼu tle yánde ÿaa anax̱éixʼu áwé chʼu tle nagútch, héende.
\snum{9}Tlax̱ áatʼch g̱ajág̱ín áwé daaḵ ugootch, nateich.
\snum{10}Tle ÿaa ḵeina.éini áwé sh taÿeet akwdax̱eichch du kwási.
\snum{11}Átx̱ áwé du taÿeenáx̱ yóot koodáaych;
\snum{12}Xʼáang̱aa dashooch.
\snum{13}Du x̱oonxʼích kadushtánín woosh dax̱ísht atxʼaayích héenxʼ.
\snum{14}Yéi kootláaÿi aas áwu á.
\snum{15}A daanáx̱ yóot kaawa.áa aas tlʼíli.
\snum{16}Ḵaa latseení áa kdu.aaḵw.
\snum{17}Chʼu tle héendáx̱ daaḵ ÿaa lunagúg̱u áwé héide kei kdulx̱eechch chush x̱ʼanaadáx̱.
\snum{18}«\!Yá ḵáa ḵu.aa x̱ách aas tlʼíli akg̱wa\-lʼéexʼ.\!»
\pend
%2
\pstart
\snum{19}Hóochʼi aaÿí héenx̱ yei kg̱wagoodí, yú ḵáa héenx̱ woogoot.
\snum{20}Cha chʼa aag̱áa yéi ḵoowa.áx̱, a tuwáatx̱ kei gux̱la\-tseen.
\snum{21}Át áyú aseiwa.áx̱.
\snum{22}Latseen yóo duwasáakw.
\snum{23}Chʼu tle ash tʼáat uwagút.
\snum{24}Yéi koogéi du shá;
\snum{25}wudlix̱ʼéx̱ʼ, a yáx̱ ÿatee.
\snum{26}Atxʼaayí du jée.
\snum{27}«\!Haagú dé\!» yóo ash yawsiḵaa,
\snum{28}«\!Ax̱ jeet gú dé\!».
\snum{29}Chʼu tle a jiÿeet uwagút.
\snum{30}Chʼu tle héennáx̱ ash kawlix̱eich.
\snum{31}Dáx̱.aa ash woox̱oox̱.
\snum{32}Aag̱áa áwé tsá latseendéin ash woox̱ísht.
\snum{33}«\!X̱át áyá Latseen.
\snum{34}I eeg̱áa x̱at woosoo\!» yéi ash yawsiḵaa.
\snum{35}«\!Yú aan aÿadultseen át xʼwán nalʼéexʼ.
\snum{36}A ÿeetéet xʼwán aklalóoxʼ.
\snum{37}Aan át g̱eeÿtsaaḵ.\!»
\pend
%3
\pstart
\snum{38}Taat áwé át áa wjixeex.
\snum{39}Du x̱oonxʼích tléil wuskú.
\snum{40}Átx̱ ÿaa ḵeiga.áa áwé du x̱oonxʼí aadé loowagooḵ.
\snum{41}Tléil wuduskú awulʼéexʼi.
\snum{42}Goosú yan yoo ḵux̱échgi?
\snum{43}Chʼu shuxʼaa aaÿích áwé tle aax̱ woolʼéexʼ.
\snum{44}Chʼu tle wduwawóosʼ
\snum{45}«\!Aadóo\-ch sá woolʼéexʼ aas tlʼíli?\!»
\snum{46}Chʼu tle yéi ÿaw\-du\-dzi\-ḵaa
\snum{47}«\!Kaháasʼich áyú woolʼéexʼ.\!»
\snum{48}Chʼa kawduwashooḵ áyú l koox̱áalgich.
\snum{49}Aadáx̱ áwé tle ÿawdudziḵoox̱ latseen du ÿeeg̱áa.
\snum{50}Goosóoyú tléil xʼáan shagóon ḵaa jee aag̱áa.
\snum{51}Ách áyú, a shagóon káxʼ áyú dashóoch.
\snum{52}Héen táa g̱aḵéiych.
\snum{53}Yú litseeni át, yú taan, tlél ách g̱wadudlijaag̱i át ḵaa jee.
\snum{54}Wáa nanée sáwé yú leengít tleiních yéi has ÿawsiḵaa «\!Hú tsú\!».
\snum{55}Yéi ÿawdudziḵaa «\!Yan nax̱dux̱aa\!».
\snum{56}Tléil ách aa x̱dudlijaag̱i át ḵoostí yú taan.
\snum{57}Du een át wuduwaxoon yéi ÿawdudziḵaa.
\snum{58}Cha chʼa aag̱áa tsá yéi ÿawdudziḵaa
\snum{59}«\!Yan nax̱dux̱aa, Kaháasʼi\!».
\snum{60}Aag̱áa tsá du ÿáa wduwasáa ‹\!Kaháasʼi\!›.
\snum{61}A daat ÿawduwax̱áa yú taan xʼáatʼi.
\snum{62}Chʼu tle déix̱ áwé ashaawatleiḵw yú taan.
\snum{63}Yú sʼatʼnax̱.aa taÿaḵʼáatlʼ kát.
\snum{64}Sheeynax̱.aa ḵu.aa wóoshdáx̱ aawasʼéilʼ.
\snum{65}Ḵʼanash\-gi\-déix̱ wusitee yú ḵáa;
\snum{66}ách áyú ldakát át yéide Latseen du eeg̱áa woosoo.
\snum{67}Tléil de yoo ÿawdudlág̱wákw du x̱oonxʼí.
\snum{68}Tléil naa.át náa yei du.úx̱xʼun adawóotl káxʼ.
\snum{69}Yeewooyáatʼ aag̱áa ash jée yéi wooteeÿi ÿé du latseení.
\snum{70}Chʼu yá ÿeedát tsú wududzikóo.
\snum{71}Du latseení átx̱ dulyéx̱ nooch.
\snum{72}Átch ḵuÿaduljéchkw nooch.
\snum{73}Dutee nooch du latseení.
\snum{71}Hóochʼ áwé.
\pend
\endnumbering
\end{Leftside}
%%
%% Column break.
%%
\begin{Rightside}
\beginnumbering
\pstart
\snum{1}It is that they bathed, those people, for stre\-ngth.
\snum{2}It is among them that there is a man, \fm{Kaháasʼi}.
\snum{3}He is very poor.
\snum{4}It is in anticipation of trouble that they bathe, those townspeople.
\snum{5}He however is of poverty.
\snum{6}Whenever they came ashore out of the water they would always laugh at him.
\snum{7}It always stank of urine, there where he always slept.
\snum{8}Actually it is when people are almost asleep that he always goes to the water.
\snum{9}It is whenever the cold had really killed him that he always goes ashore and sleeps.
\snum{10}Then it is when it is dawning that he always dumps it in his bed, his urine.
\snum{11}It is after that that it would flow out there along his bed.
\snum{12}They bathe for war.
\snum{13}His relatives had the habit of beating each other with boughs in the water.
\snum{14}There was a wide tree there.
\snum{15}Along its bark grew out the tree’s penis.
\snum{16}People test their strength there.
\snum{17}Then it is as they are running out of the water that people always shoved him over out of their way.
\snum{18}\qqk{}“This man however is actually going to break the tree’s penis.”
\pend
%2
\pstart
\snum{19}When the last one is going to go down into the water, that man went into the water.
\snum{20}Well then it is around there that he heard someone, that from it he will become strong.
\snum{21}It is there that he heard its voice.
\snum{22}People call it Strength.
\snum{23}Just then it came up behind him.
\snum{24}His head was yay big;
\snum{25}it was shrivelled up, it is like that.
\snum{26}He had tree boughs.
\snum{27}\qqk{}“Come here” he said to him,
\snum{28}\qqk{}“Come to me now”.
\snum{29}So then he went below its hand.
\snum{30}Then he threw him in the water.
\snum{31}He called him a second time.
\snum{32}Then it is after that that he whipped him strongly.
\snum{33}\qqk{}“I am Strength.
\snum{34}I have given supernatural help to you” he said to him.
\snum{35}\qqk{}“Break that thing with which people are exercising.
\snum{36}Urinate on the place where it is.
\snum{37}With that poke it there.”
\pend
%3
\pstart
\snum{38}At night he ran around over there.
\snum{39}His relatives did not know about it.
\snum{40}After that, it having dawned, his relatives ran there.
\snum{41}People did not know that he had broken it.
\snum{42}Where is that which gives people a hard time?
\snum{43}It was just the first one then that broke it off of there.
\snum{44}So then people asked
\snum{45}\qqk{}“Who is it that broke the tree’s penis?”
\snum{46}So then people said
\snum{47}“It is Kaháasʼi who broke it”.
\snum{48}People just laughed at him because he is not interesting.
\snum{49}After that then it brought strength for him.
\snum{50}It was somewhere that people did not have weapons of war around there.
\snum{51}That is why, it is for power that they bathe.
\snum{52}They were always sitting in water.
\snum{53}A thing which is strong, those sealions,
		people did not have something which people could kill them with.
\snum{54}At some point the big man said to them “Him too”.
\snum{55}He said to them “People should take him ashore”.
\snum{56}No thing which people could kill some of them with existed, those sealions.
\snum{57}He told them to creep along around it with him.
\snum{58}Then just right after that he said to them
\snum{59}\qqk{}“People should take him ashore, Kaháasʼi”.
\snum{60}Just after that people named him ‘Kaháasʼi’.
\snum{61}They paddled him around it, that sealion island.
\snum{62}Just then it was two that he snatched up, those sealions.
\snum{63}The left one (he threw) on top of a flat rock.
\snum{64}The right one however he tore apart.
\snum{65}He was of poverty, that man;
\snum{66}that is why with everything that way Strength gave him supernatural help.
\snum{67}They did not beat him now, his relatives.
\snum{68}He did not wear clothing in the face of war.
\snum{69}It was a long time after that way that it was in his possession, his strength.
\snum{70}Even now people also know it.
\snum{71}People always build his strength into things.
\snum{72}People always surprise people with this thing.
\snum{73}People always imitate his strength.
\snum{74}It is finished.
\pend
\endnumbering
\end{Rightside}
\end{pairs}
\Columns

\vspace{1\baselineskip}

\section{Swanton’s abstract}\label{sec:093-swanton-abstract}

In a certain town two persons were bathing for strength in order to kill sea lions.
One of these, the town chief, bathed in public accompanied by all of the town people, while his nephew bathed during the night only, and lay in bed all day, pretending that he was a weakling.
Finally a being called Strength came to the latter and made him so powerful that he was able to accomplish the feats the chief had set himself, namely, to pull the stump of a branch out of a tree and twist another tree down to the base.
Having done so, however, he put them into their original positions, and when the chief tried them next he thought that he had become strong.
When they started out for the sea-lion islands, they let Kāhā′s!î go along also, and, while the chief was killed, Kāhā′s!î destroyed two big sea lions, one with each hand.

\section{Swanton’s translation}\label{sec:093-swanton-translation}

\snum{2}Among some \snum{1}people bathing for strength \snum{2}was a man named Kāhā′s!î.
\snum{3}He was very poor.
\snum{4}The people bathed continually in preparation for war.
\snum{5}He, however, was very miserable.
\snum{6}When the others came out of the water they always laughed at him.
\snum{7}He kept urinating in his sleep.
\snum{7}He was always turned over on one side.
\snum{8}It was when all were asleep that he went down to the water.
\snum{9}When he got very cold he came ashore and went to sleep.
\snum{10}And when daylight was coming on he threw his urine under him.
\snum{11}Then it always ran out from under him.
\snum{12}They kept bathing for strength in war.
\snum{13}His friends used to whip each other in the water with boughs.
\snum{16}They tried their strength on \snum{14}a big tree having a \snum{15}dead branch growing out from it which they called the tree-penis.
\snum{17}And when they ran ashore out of the water they always kicked him (Kāhā′s!î) out of their way.
\snum{18}\qqk{}“When will this man break off the tree-penis?” [they said].

\snum{19}The man went into the water the last time he was going to enter it.
\snum{20}At that very time he heard some one down in it from whom he was going to get his strength.
\snum{22}Strength was his name.
\snum{23}Then the person came out behind him.
\snum{24}He had a large head \snum{25}covered with curly hair.
\snum{26}He held boughs.
\snum{27}\qqk{}“Now,” he (Strength) said to him, \snum{28}\qqk{}“come up to me.” \snum{29}Then he went to him.
\snum{30}He knocked him into the water.
\snum{31}Twice he called him.
\snum{32}At once he whipped him hard.
\snum{33}\qqk{}“I am Strength.
\snum{34}I come to help you,” he said to him.
\snum{35}\qqk{}“Break off the thing the people are trying their strength on.
\snum{37}Put it back again \snum{36}along with some urine.” (¶3) \snum{38}Then he ran there in the night.
\snum{39}His friends did not know it.
\snum{40}After day had begun to dawn his friends ran hither.
\snum{41}It was not known that he had broken it off.
\snum{42}Why had it never been broken off before?
\snum{43}The very first one now broke it off.
\snum{44}Then they inquired, \snum{45}\qqk{}“Who broke off the tree-penis?” \snum{46}and people said, \snum{47}\qqk{}“It was Kāhā′s!î who broke it off.” \snum{48}They laughed at him because [they thought] he was not strong.
\snum{49}Then they started off with the strength they had waited for.
\snum{50}At that time [the Indians] had no fighting ammunition.
\snum{51}This is why they always bathed for ammunition, \snum{52}sitting in the water.
\snum{53}The strong men had nothing at all with which to kill the sea lions.
\snum{54}At once the head man said as follows, “Take him also.” \snum{55}They said, “Take him there.” \snum{56}They had nothing with which to kill the sea lions.
\snum{57}Then they told him that they would take him along.
\snum{58}They said, \snum{59}\qqk{}“Take Kāhā′s!î there.” \snum{60}It was at that time that they gave him his proper name.
\snum{61}They took him out to the sea-lion island.
\snum{62}Then he caught up two sea lions.
\snum{63}The one on the left he threw upon a flat rock, \snum{64}but the one on the right he tore in pieces.
\snum{66}All kinds of strength came to \snum{65}the poor man \snum{66}to help him, \snum{67}and his friends never beat him afterward.
\snum{68}He never put on clothes in time of war.
\snum{69}His strength continued for a long time.
\snum{70}It came to be known even down to this day.
\snum{71}People always use his strength \snum{72}with which to suprise other people, \snum{73}and they always imitate his strength. [Footnote: That is, it is used as a crest and imitated at feasts.]

\snum{74}This is all.

\section{Paul’s summary from Zuboff}\label{sec:093-paul-summary}

Kuh-ha-zi shortened from Uh-kuh-ha-zi “urinates on himself” \&\ Dook-duL “black-skin” were the nicknames of xec-ka-de (Kgec-ka-de).
The cudʼc (pushed) man was Ckuh-guł-wedt (shortened to Guł-wet by Swanton), or Ckuh-kuł-wedt.
The uncle’s name was Guhl-wats.
Dook-duL asked the uncle’s wife for marten fur to bind his head as he went to the canoe and lazily rolled into the canoe after when he pulled it back “to just bail the canoe”.
He walked thru two benches breaking them w his shins before he stept on the next bench and as he walked he asked ques.\ “who twisted the tree?
Who pulled or broke the ahs-Ly-łi (ahs-łu-wu)”.
After killing the sea-lions, he pulled out their teeth with his fingers and when he got home, he made a hat trimmed with them called “tahn-uX-Xoo-zowx”.

per Robert Zuboff June 1, 1949

\begin{itemize}
\item	Kuh-ha-zi — \fm{Kaháasʼi}
\item	Uh-kuh-ha-zi — \fm{Akaháasʼi}
\item	Dook-duL — \fm{Duktʼootlʼ}
\item	xec-ka-de — \fm{X̱ʼeishkaadé}
\item	cudʼc — \fm{shúch} ‘bathe’
\item	Ckuh-guł-wedt — \fm{Shkag̱alwéitʼ}
\item	Guł-wet — \fm{G̱alwéitʼ}
\item	Guhl-wats — \fm{Galwátsʼ}?
\item	ahs-Ly-łi — \fm{aas tlʼíli}
\item	ahs-łu-wu — \fm{aas loowú}
\item	tahn-uX-Xoo-zowx — \fm{taan oox̱ú sʼáaxw}
\end{itemize}

\section{Paragraph 1}\label{sec:093-para-1}

\ex\label{ex:93-1-bathe-for-strength}%
\exmn{289.1}%
\begingl
	\glpreamble	Wudêcu′djayu yuqo′ō łatsī′n kᴀq! //
	\glpreamble	Wudishúch áyú yú ḵu.oo, latseen káxʼ. //
	\gla	\rlap{Wudishúch} @ {} @ {} @ {} @ {} \rlap{áyú} @ {}
		{} yú \rlap{ḵu.oo,} @ {} @ {} {}
		{} \rlap{latseen} @ {} @ {} @ {} \rlap{káxʼ.} @ {} {} //
	\glb	wu- d- i- \rt[²]{shuch} -μH á -yú
		{} yú ḵu- \rt[²]{.u} -μμL {}
		{} l- \rt[¹]{tsin} -μμL {} ká -xʼ {} //
	\glc	\xx{pfv}- \xx{mid}- \xx{stv}- \rt[²]{bathe} -\xx{var} \xx{foc} -\xx{dist}
		{}[\pr{DP} \xx{dist} \xx{areal}- \rt[²]{own} -\xx{var} {}]
		{}[\pr{PP} \xx{xtn}- \rt[¹]{alive} -\xx{var} \·\xx{nmz} \xx{hsfc} -\xx{loc} {}] //
	\gld	\rlap{\xx{pfv}.self.bathe} {} {} {} {} \rlap{it.is} {}
		{} those \rlap{dwellers} {} {} {}
		{} \rlap{strength} {} {} {} atop -at {} //
	\glft	‘It is that they bathed, those people, for strength.’
		//
\endgl
\xe

\citeauthor{swanton:1909} runs (\lastx) and (\nextx) together as a single sentence, but grammatically they appear to be two separate sentences.
This is because any given main clause occurs with one focus particle, or possibly two but only if the first focus particle immediately follows a sentence initial temporal sequencing PP like \fm{aadáx̱} ‘after that’ or \fm{aag̱áa} ‘around then’.
Since (\lastx) has a focus particle with \fm{wudishúch áyú} ‘it is that they bathed’ and (\nextx) has a focus particle with \fm{a x̱ooxʼ áyú} ‘it is among them’, these are grammatically separate main clauses and thus separate sentences.

\ex\label{ex:93-2-among-them}%
\exmn{289.1}%
\begingl
	\glpreamble	ā′xoq! ayu′ yē′ỵatî qā Kāhā′s!î. //
	\glpreamble	Aa x̱ooxʼ áyú yéi ÿatee ḵáa, Kaháasʼi. //
	\gla	{} Aa \rlap{x̱ooxʼ} @ {} {} \rlap{áyú} @ {}
		yéi @ \rlap{ÿatee} @ {} @ {}
		{} ḵáa, {} 
		{} Kaháasʼi. {} //
	\glb	{} aa x̱oo -xʼ {} á -yú
		yéi= i- \rt[¹]{tiʰ} -μμL
		{} ḵáa {} 
		{} Kaháasʼi {} //
	\glc	{}[\pr{PP} \xx{part·pss} among -\xx{loc} {}] \xx{foc} -\xx{dist}
		thus= \xx{stv}- \rt[¹]{be} -\xx{var}
		{}[\pr{DP} man {}]
		{}[\pr{DP} \xx{name} {}] //
	\gld	{} some among -at {} \rlap{it.is} {}
		thus \rlap{\xx{imfpv}.be} {} {}
		{} man {}
		{} \xx{name} {} //
	\glft	‘It is among some of them that there is a man, Kaháasʼi.’
		//
\endgl
\xe

The sentence in (\lastx) gives the main character’s name as \fm{Kaháasʼi} [\ipa{kʰà.ˈháː.sʼì}].
This name is based on the root \fm{\rt[¹]{hasʼ}} ‘vomit, nauseated’ which is related to \fm{\rt[¹]{hatlʼ}} ‘crap, defecate’ and \fm{\rt[¹]{hachʼ}} ‘disgusting, shameful’ \parencites[01/88, 90–92, 97–98]{leer:1973}[2]{leer:1978b}.
The morphological structure is \fm{k-\rt[¹]{hasʼ}-μH-í} where \fm{k-} is a qualifier and \fm{-í} is probably a nominalizer.
This character is better known today by the name \fm{Duktʼootlʼ} [\ipa{tùk.tʼùːtɬʼ}] ‘Black Skin’ \parencites[890–892]{de-laguna:1972}[136–151]{dauenhauer:1987}[439–440]{mcclellan-cruikshank:2007b}[684–687]{mcclellan-cruikshank:2007c}.
This alternative name is derived from \fm{dook} [\ipa{tùːk}] ‘skin’ and a variant form of \fm{tʼoochʼ} [\ipa{tʼùːtʃʼ}] ‘charcoal; black’.
The phonological change from \fm{chʼ} /\ipa{tʃʼ}/ to \fm{tlʼ} /\ipa{tɬʼ}/ in this name is unexplained but it is similar to the parallels between \fm{\rt[¹]{hasʼ}} ‘vomit, nauseated’, \fm{\rt[¹]{hatlʼ}} ‘crap, defecate’, and \fm{\rt[¹]{hachʼ}} ‘disgusting, shameful’ \parencites{leer:1990a}{crippen:2010}.

\ex\label{ex:93-3-very-poor}%
\exmn{289.2}%
\begingl
	\glpreamble	ʟᴀx q!anā′ckîdê. //
	\glpreamble	Tlax̱ ḵʼanashgidéi. //
	\gla	Tlax̱ \rlap{ḵʼanashgidéi.} @ {} @ {} @ {} @ {} @ {} //
	\glb	tlax̱ ḵʼe- n- sh- \rt{git} -i =yé //
	\glc	very mouth- \xx{ncnj}- \xx{pej}- \rt{\xx{unkn}} -\xx{rel} =way //
	\gld	very \rlap{poor} {} {} {} {} {} //
	\glft	‘He is very poor.’
		//
\endgl
\xe

The word \fm{ḵʼanashgidéi} ‘poor’ in (\lastx) is structurally enigmatic.
It is given by \citeauthor{leer:1973} with the forms \fm{ḵʼanashgidéi} [\ipa{qʼà.nàʃ.kì.ˈtéː}], \fm{ḵʼanashgudéi} [\ipa{qʼà.nàʃ.kʷù.ˈtéː}], and Tongass \fm{ḵʼanashgidei} [\ipa{qʼa.naʃ.ki.ˈteː}] \parencite[05/87]{leer:1973}.
He lists it under \orth{D} in his stem list implying that \fm{–déi} is the stem \parencite[21]{leer:1978b} although he does not give a morphological breakdown.
The \fm{-éi} portion appears to be a contraction of the relative clause \fm{-i} and the noun \fm{yé} ‘way, manner, place’, suggesting that the stem is actually \fm{–git} which entails either a root \fm[?]{\rt{git}} or a root \fm[?]{\rt{gi}} with a suffix \fm{-t}.
The root vowel could also plausibly be \fm{e} /\ipa{e}/ instead of \fm{i} /\ipa{i}/.
There is no root \fm{\rt{gi}} attested in any other vocabulary so this is unlikely and can be ruled out.
The root \fm{\rt[¹]{git}} ‘(anim.)\ move rapidly, fall; behave’ is a plausible candidate, known from several verb phrases like \fm{wudzigeet} ‘s/he fell’, \fm{awsigeet} ‘s/he made him/her fall’, \fm{ḵwáaḵt wudzigít} ‘s/he made a mistake’, and \fm{a géidei wdzigeet} ‘s/he went against it (custom, law)’ \parencites[f05/91–93]{leer:1973}[658–659]{leer:1976}.
The same root probably occurs in the poorly documented phrase \fm{ax̱ waḵshiyeexʼ has koosgídáḵw} ‘they’re showing off in front of me’ \parencite[f05/94]{leer:1973} and it might also be identified as the final element in \fm{lingít} ‘person, Tlingit’ \parencite[f05/96]{leer:1973}.
If the root vowel is \fm{e} then it might be compared to \fm{\rt[¹]{get}} ‘tiptoe’ \parencite[f05/75]{leer:1973}, itself connected to \fm{\rt[¹]{gut}} ‘sg.\ go (by foot)’ \parencite[f05/129]{leer:1973}.
Other possible comparanda include \fm{\rt[¹]{ge}} ‘big’ \parencite[f05/52]{leer:1973}, \fm{\rt[¹]{ge}} ‘stingy, cheap’ \parencite[f05/59]{leer:1973}, \fm{\rt[¹]{ge}} ‘wise, understanding’ \parencite[f05/61]{leer:1973}, or \fm{géi} ‘against’ \parencite[f05/64]{leer:1973}.
The connection with \fm{\rt[¹]{git}} ‘(anim.)\ move rapidly, fall; behave’ seems to be the most plausible root but still requires more investigation.
The remaining morphology can be identified as shown in (\lastx) when \fm{\rt{git}} is isolated in \fm{ḵʼanashgidéi} ‘poor’, though a covert \fm{d-} between \fm{n-} and \fm{sh-} is possible.

\ex\label{ex:93-4-bathe-for-trouble}%
\exmn{289.2}%
\begingl
	\glpreamble	Adawū′ʟ kᴀ′q!ayu dacū′tc yū′āntqēnî. //
	\glpreamble	Adawóotl káxʼ áyú dashóoch yú aantḵeiní. //
	\gla	{} {} \rlap{Adawóotl} @ {} @ {} @ {} {} \rlap{káxʼ} @ {} {}
		\rlap{áyú} @ {}
		\rlap{dashóoch} @ {} @ {} +
		{} yú {} \rlap{aantḵeiní.} @ {} @ {} @ {} @ {} @ {} {} {} //
	\glb	{} {} a- d- \rt[²]{wutl} -μμH {} ká -xʼ {}
		á -yú
		d- \rt[²]{shuch} -μμH
		{} yú {} aan- d- \rt[¹]{ḵi} -μμL -n -í {} //
	\glc	{}[\pr{PP} {}[\pr{NP} \xx{arg}- \xx{mid}- \rt[²]{agitated} -\xx{var} {}] \xx{hsfc} -\xx{loc} {}]
		\xx{foc} -\xx{dist}
		\xx{mid}- \rt[²]{bathe} -\xx{var}
		{}[\pr{DP} \xx{dist} {}[\pr{NP} town- \xx{mid}- \rt[¹]{sit·\xx{pl}} -\xx{var} -\xx{nsfx} -\xx{nmz} {}] {}] //
	\gld	{} {} \rlap{trouble} {} {} {} {} atop -at {}
		\rlap{it.is} {}
		\rlap{\xx{impfv}.self.bathe} {} {}
		{} those {} \rlap{townspeople} {} {} {} {} {} {} {} //
	\glft	‘It is in anticipation of trouble that they bathe, those townspeople.’
		//
\endgl
\xe


\ex\label{ex:93-5-he-is-of-poverty}%
\exmn{289.3}%
\begingl
	\glpreamble	Hō qo′a q!anackidē′x sêtê′. //
	\glpreamble	Hú ḵu.aa ḵʼanashgidéix̱ sitee. //
	\gla	Hú ḵu.aa
		{} {} \rlap{ḵʼanashgidéix̱} @ {} @ {} @ {} @ {} @ {} {} {} {}
		\rlap{sitee.} @ {} @ {} @ {} //
	\glb	hú ḵu.aa
		{} {} ḵʼe- n- sh- \rt{git} -i =yé {} -x̱ {}
		s- i- \rt[¹]{tiʰ} -μμL //
	\glc	\xx{3h} \xx{contr}
		{}[\pr{PP} {}[\pr{NP} mouth- \xx{ncnj}- \xx{pej}- \rt{\xx{unkn}} -\xx{rel} =way {}] -\xx{pert} {}]
		\xx{appl}- \xx{stv}- \rt[¹]{be} -\xx{var} //
	\gld	him however
		{} {} \rlap{poor} {} {} {} {} {} {} -of {}
		\rlap{\xx{impfv}.\xx{appl}.be} {} {} {} //
	\glft	‘He however is of poverty.’
		//
\endgl
\xe

\citeauthor{swanton:1909}’s transcription \orth{sêtê′} for \fm{sitee} in (\lastx) is not actually a mishearing.
The usual pronunciation of this verb is \fm{sitee} [\ipa{sì.ˈtʰìː}], but both the short [\ipa{ì}] and long [\ipa{ìː}] are sometimes lowered by speakers of various Northern dialect varieties.
The same pronunciation of \fm{sitee} as [\ipa{sɛ̀.ˈtʰɛ̀}] in utterance-final position has been audio recorded from several speakers in the late 20th century.
The lexically specific sound change of \fm{i} > \fm{e} in roots like \fm{\rt[¹]{niʰ}} ‘occur, happen’ and \fm{\rt[²]{tʼiʰ}} ‘find’ is probably due to the same process, but it is still not clear why this has not applied to the lexical entry of \fm{\rt[¹]{tiʰ}} ‘be’.

\ex\label{ex:93-6-laugh-out-of-water}%
\exmn{289.3}%
\begingl
	\glpreamble	Hīn dᴀx dāq ag̣ā′ᴀīnawe qūdūcū′qtc. //
	\glpreamble	Héendáx̱ daaḵ ag̱a.ádín áwé koodushooḵch. //
	\gla	{} {} \rlap{Héendáx̱} @ {} {}
			daaḵ @ \rlap{ag̱a.ádín} @ {} @ {} @ {} @ {} @ {} {}
		\rlap{áwé} @ {}
		\rlap{koodushooḵch.} @ {} @ {} @ {} @ {} @ {} //
	\glb	{} {} héen -dáx̱ {}
			daaḵ= a- {} g̱- \rt[¹]{.at} -μH -ín {}
		á -wé
		k- u- du- \rt[²]{shuḵ} -μμL -ch //
	\glc	{}[\pr{CP} {}[\pr{PP} water -\xx{abl} {}]
			\xx{abmar}= \xx{4h·s}- \xx{zcnj}\· \xx{mod}- \rt[¹]{go·\xx{pl}} -\xx{var} -\xx{ctng} {}]
		\xx{foc} -\xx{mdst}
		\xx{qual}- \xx{zpfv}- \xx{4h·s}- \rt[²]{laugh} -\xx{var} -\xx{rep} //
	\gld	{} {} water -from {}
			ashore \rlap{people.\xx{ctng}.go·\xx{pl}} {} {} {} {} -when {}
		\rlap{it.is} {}
		\rlap{\xx{hab}.people.laugh} {} {} {} {} {} //
	\glft	‘Whenever they came ashore out of the water they would always laugh at him.’
		//
\endgl
\xe

\ex\label{ex:93-7-urine-stink-sleep}%
\exmn{289.4}%
\begingl
	\glpreamble	Kūłq!ê′stc.
Adawē′ nate′tc. //
	\glpreamble	Koolx̱ʼésch, át áwé nateich. //
	\gla	\rlap{Koolx̱ʼésch,} @ {} @ {} @ {} @ {} @ {} 
		{} \rlap{át} @ {} {}
		\rlap{áwé} @ {}
		\rlap{nateich.} @ {} @ {} @ {} //
	\glb	k- u- l- \rt[¹]{x̱ʼis} -μH -ch
		{} á -t {}
		á -wé
		n- \rt[¹]{taʰ} -eμL -ch //
	\glc	\xx{qual}- \xx{zpfv}- \xx{xtn}- \rt[¹]{urine·stink} -\xx{var} -\xx{rep}
		{}[\pr{PP} \xx{3n} -\xx{pnct} {}]
		\xx{foc} -\xx{mdst}
		\xx{ncnj}- \rt[¹]{sleep·\xx{sg}} -\xx{var} -\xx{rep} //
	\gld	\rlap{\xx{hab}.urine·stink} {} {} {} {} {}
		{} there -at {}
		\rlap{it.is} {}
		\rlap{\xx{hab}.sleep·\xx{sg}} {} {} {} //
	\glft	‘It always stank of urine, there where he always slept.’
		//
\endgl
\xe

\citeauthor{swanton:1909} gives (\lastx) as two separate sentences with inappropriate glosses.
He glosses \orth{Kūłq!ê′stc} as “He always urinated in bed” which seems to be a misinterpretation of the verb as referring to an activity (urinating in bed) rather than a result state (smell of urine).
The root \fm{\rt[¹]{x̱ʼis}} ‘urine smell’ is relatively obscure, documented as “\fm{k-ł-x̱ʼis*} smell like pee” in \cite[77]{leer:1978b} and as a noun \fm{kax̱ʼees} ‘strong urine smell’ in \cite[\textsc{t}·34]{leer:2001}.\footnote{\textcite[77]{leer:1978b} notes a possible connection between \fm{\rt[¹]{x̱ʼis}} (< PT \fm[*]{\rt{χʼis}} \~\ \fm[*]{\rt{qʼic}}) and Eyak \fm{qʼiʰǯ} ‘rancid, sour, spoiled’ \parencite[1700]{krauss:1970} and Proto-Dene \fm[*]{qʼųˀčʼ} ‘sour’ \parencite[84/187]{leer:1996a}.} \citeauthor{swanton:1909} glosses \orth{Adawē′ nate′tc} as “He was always [turned] on his side” which suggests that his consultant imitated the act of lying down to sleep and that \citeauthor{swanton:1909} misinterpreted this as lying on one’s side.
With the interpretations adjusted, the resulting utterance appears to be a single sentence with one of the two verbs being an unmarked adjunct clause.
Since neither clause is marked for clause type, it is not clear which of the two – \fm{koolx̱ʼésch} or \fm{át áwé nateich} – is the main clause and which is the adjunct.
Probably the adjunct is \fm{koolx̱ʼésch} since adjunct clauses are not known to contain focused phrases, but focus within adjuncts has not been explored so the alternative analysis with \fm{koolx̱ʼésch} as the main clause is still plausible.

\ex\label{ex:93-8-sleeping-bathe}%
\exmn{289.5}%
\begingl
	\glpreamble	Xᴀtc tc!uʟe′ yê′ndî ỵaanaxᴀ′q!ᵒawe tc!uʟe′ nagu′ttc hī′ndî. //
	\glpreamble	X̱ách chʼu tle yánde ÿaa anax̱éixʼu áwé chʼu tle nagútch, héende. //
	\gla	X̱ách 
		{} chʼu tle
			{} \rlap{yánde} @ {} {} 
			ÿaa @ \rlap{anax̱éixʼu} @ {} @ {} @ {} @ {} {}
		\rlap{áwé} @ {} +
		chʼu tle \rlap{nagútch} @ {} @ {} @ {}
		{} \rlap{héende.} @ {} {} //
	\glb	x̱áju
		{} chʼu tle
			{} yán -dé {}
			ÿaa= a- n- \rt[¹]{x̱exʼw} -μμH -í {}
		á -wé
		chʼu tle
		n- \rt[¹]{gut} -μH -ch
		{} héen -dé {} //
	\glc	actually
		{}[\pr{CP} just then
			{}[\pr{PP} \xx{term} -\xx{all} {}]
			along= \xx{4h·s}- \xx{ncnj}- \rt[¹]{sleep·\xx{pl}} -\xx{var} -\xx{sub} {}]
		\xx{foc} -\xx{mdst}
		just then
		\xx{ncnj}- \rt[¹]{go·\xx{sg}} -\xx{var} -\xx{rep}
		{}[\pr{PP} water -\xx{all} {}] //
	\gld	actually
		{} just then
			{} \rlap{done} {} {}
			along \rlap{people.\xx{prog}.sleep·\xx{pl}} {} {} {} -when {}
		\rlap{it.is} {}
		just then
		\rlap{\xx{hab}.go·\xx{sg}} {} {} {}
		{} water -to {} //
	\glft	‘Actually it is when people are almost asleep that he always goes to the water.’
		//
\endgl
\xe

\ex\label{ex:93-9-cold-killed}%
\exmn{289.5}%
\begingl
	\glpreamble	ʟāx ā′tᴀtc g̣adjᴀ′g̣înawe dāq ugu′ttc, natē′tc. //
	\glpreamble	Tlax̱ áatʼch g̱ajág̱ín áwé daaḵ ugootch, nateich. //
	\gla	{} {} Tlax̱ {} \rlap{áatʼch} @ {} @ {} {}
				\rlap{g̱ajág̱ín} @ {} @ {} @ {} @ {} @ {} {} \rlap{áwé} @ {} +
			daaḵ @ \rlap{ugootch,} @ {} @ {} @ {} {} 
		{} \rlap{nateich.} @ {} @ {} @ {} {} //
	\glb	{} {} tlax̱ {} \rt[⁰]{.atʼ} -μμH -ch {}
				ⱥ- {} g̱- \rt[²]{jaḵ} -μH -ín {} á -wé
			daaḵ= u- \rt[¹]{gut} -μμL -ch {}
		{} n- \rt[¹]{taʰ} -eμL -ch {} //
	\glc	{}[\pr{CP} [\pr{CP} very {}[\pr{DP} \rt[⁰]{cold} -\xx{var} -\xx{erg} {}]
				\xx{arg}- \xx{zcnj}\· \xx{mod}- \rt[²]{kill} -\xx{var} -\xx{ctng} {}] \xx{foc} -\xx{mdst}
			\xx{abmar}= \xx{zpfv}- \rt[¹]{go·\xx{sg}} -\xx{var} -\xx{rep} {}]
		{}[\pr{CP} \xx{ncnj}- \rt[¹]{sleep·\xx{sg}} -\xx{var} -\xx{rep} {}] //
	\gld	{} {} very {} \rlap{coldness} {} {} {}
				\rlap{3>3.\xx{ctng}.kill} {} {} {} {} -when {} \rlap{it.is} {}
			ashore \rlap{\xx{hab}.go·\xx{sg}} {} {} {} {}
		{} \rlap{\xx{hab}.sleep·\xx{sg}} {} {} {} {} //
	\glft	‘It is whenever the cold had really killed him that he always goes ashore and sleeps.’
		//
\endgl
\xe

The syntactic structure of (\lastx) shows two main clauses that are linked together in parataxis.
Parataxis is essentially a coordination of two hierarchically equal clauses without any overt indication of the coordination, though in speech there is usually some kind of prosodic indication of the connection between them.
In (\lastx) the parataxis is of two main clauses, one with the verb phrase \fm{daaḵ ugootch} ‘s/he always goes ashore/inland’ and the second with the verb \fm{nateich} ‘s/he always sleeps’; both are habituals and have no overt clause type morphology.
The first clause is complex because it contains an adjunct clause \fm{tlax̱ áatʼch g̱ajág̱ín} expressing a contingent ‘whenever cold had really killed him’ as a focused phrase, whereas the second clause is simple because it consists of a single verb.
Translating the parataxis literally in English sounds strange – ‘… he always goes ashore, he always sleeps’ – so it represented instead by coordination with subject elision.

\ex\label{ex:93-10-dump-urine}%
\exmn{289.6}%
\begingl
	\glpreamble	ʟe ỵaqēnaē′nî awe′ ctaỵī′t akᵘdaxē′tctc dukoa′si. //
	\glpreamble	Tle ÿaa ḵeina.éini áwé sh taÿeet akwdax̱eichch du kwási. //
	\gla	{} Tle ÿaa @ \rlap{ḵeina.éini} @ {} @ {} @ {} @ {} @ {} {}
		\rlap{áwé} @ {}
		{} sh \rlap{taÿeet} @ {} @ {} {}
		\rlap{akwdax̱eichch} @ {} @ {} @ {} @ {} @ {} @ {}
		{} du \rlap{kwási.} @ {} {} //
	\glb	{} tle ÿaa= ḵee- n- \rt[¹]{.a} -eμH -n -í {}
		á -wé
		{} sh tá- ÿee -t {}
		a- k- u- d- \rt[²]{x̱ich} -μμL -ch
		{} du kwás -í {} //
	\glc	{}[\pr{CP} then along= dawn- \xx{ncnj}- \rt[¹]{extend} -\xx{var} -\xx{nsfx} -\xx{sub} {}]
		\xx{foc} -\xx{mdst}
		{}[\pr{PP} \xx{rflx·pss} sleep- below -\xx{pnct} {}]
		\xx{arg}- \xx{qual}- \xx{zpfv}- \xx{mid}- \rt[²]{throw·filled} -\xx{var} -\xx{rep}
		{}[\pr{DP} \xx{3h·pss} urine -\xx{pss} {}] //
	\gld	{} then along \rlap{dawn.\xx{prog}.extend} {} {} {} {} -when {}
		\rlap{it.is} {}
		{} self’s \rlap{bed} {} -at {}
		\rlap{3>3.\xx{hab}.dump} {} {} {} {} {} {}
		{} his \rlap{urine} {} {} //
	\glft	‘Then it is when it is dawning that he always dumps it in his bed, his urine.’
		//
\endgl
\xe

The root \fm{\rt[²]{x̱ich}} \~\ \fm{\rt[²]{x̱ech}} in (\lastx) has three distinct but related meanings depending on its morphological and semantic context.
The meaning here is ‘throw filled container’, with a qualification of the object similar to \fm{\rt[²]{.in}} ‘handle filled container’.
This is exemplified in sentences like \fm{héen gáant kax̱waax̱ích} ‘I threw the water outside’ \parencite[227.3218]{story-naish:1973}, \fm{ax̱ xʼeesháyi ḵut x̱waax̱eech} ‘I lost my bucket’ \parencite[129.1716]{story-naish:1973}, and \fm{kóox yax̱ akaawax̱ích} ‘he spilled the rice’ \parencite[203.2841]{story-naish:1973}.
Another meaning is ‘throw singular animate’ which can be seen in sentences like \fm{keitl gáant aawax̱ích} ‘he threw the dog outside’ \parencite[227.3216]{story-naish:1973}, \fm{yoo ayax̱íchk} ‘he is shaking him (to rouse him)’ \parencite[186.2569]{story-naish:1973}, and \fm{kát shax̱wdix̱ích} ‘I surface-dived’ \parencite[71.848]{story-naish:1973}.
A third meaning ‘club, hit with stick’ is found both with \fm{\rt[²]{x̱ich}} \~\ \fm{\rt[²]{x̱ech}} as in \fm{cháatl ashaawax̱ích} ‘he hit the halibut on the head’ \parencite[109.1425]{story-naish:1973}, \fm{x̱ʼaan kát x̱eech wé ḵáasʼ} ‘throw that stick in the fire’ \parencite[f02/59]{leer:1973}, and \fm{hít kawdudlix̱eech} ‘they tore down the house’ \parencite[225.3169]{story-naish:1973}.
This latter meaning also occurs with the similar root \fm{\rt[²]{x̱ish}} – e.g.\ (\ref{ex:93-13-relatives-beating-each-other}) – suggesting that the two roots may have been conflated at some point in the past.

\ex\label{ex:93-11-flow-out}%
\exmn{289.6}%
\begingl
	\glpreamble	ᴀtxā′we dutā′ỵenᴀx yūt kᵘdā′îtc. //
	\glpreamble	Átx̱ áwé du taÿeenáx̱ yóot koodáaych. //
	\gla	{} \rlap{Átx̱} @ {} {} \rlap{áwé} @ {}
		{} du \rlap{taÿeenáx̱} @ {} @ {} {}
		{} \rlap{yóot} @ {} {}
		\rlap{koodáaych.} @ {} @ {} @ {} @ {} @ {} //
	\glb	{} á -dáx̱ {} á -wé
		{} du tá- ÿee -náx̱ {}
		{} yú -t {}
		k- u- \rt[¹]{da} -μμH -ÿ -ch //
	\glc	{}[\pr{PP} \xx{3n} -\xx{abl} {}] \xx{foc} -\xx{mdst}
		{}[\pr{PP} \xx{3h·pss} sleep- below -\xx{perl} {}]
		{}[\pr{PP} \xx{dist} -\xx{pnct} {}]
		\xx{hsfc}- \xx{zpfv}- \rt[¹]{flow} -\xx{var} -\xx{ÿsfx} -\xx{rep} //
	\gld {} that -after {} \rlap{it.is} {}
		{} his \rlap{bed} {} -along {}
		{} there -to {}
		\rlap{\xx{hab}.flow} {} {} {} {} {} //
	\glft	‘It is after that that it would flow out there along his bed.’
		//
\endgl
\xe

\citeauthor{swanton:1909}’s transcription \orth{dutā′ỵenᴀx} in (\lastx) shows an interesting contrast with his \orth{ctaỵī′t} in (\ref{ex:93-10-dump-urine}).
These are both forms of the noun \fm{taÿee} ‘sleeping place; bed’, but the transcriptions suggest different stress patterns and consequent changes to vowel length and quality.
The transcription \orth{ctaỵī′t} is \fm{sh taÿeet} which was probably pronounced [\ipa{ʃ.tʰà.ˈɰìːt}]. with primary stress on the final \fm{ÿeet} syllable.
The transcription \orth{dutā′ỵenᴀx} is \fm{du taÿeenáx̱} but \citeauthor{swanton:1909} has \orth{ỵe} rather than \orth{ỵī} and the stress mark \orth{′} is in \orth{tā′} and not in \orth{ỵī′}.
This suggests that \orth{dutā′ỵenᴀx} is \fm{du taÿeenáx̱} was pronounced like [\ipa{tù.ˌtʰà.ɰì.ˈnáχ}] with primary stress on the final \fm{náx̱} syllable, secondary stress on \fm{ta}, and the unstressed vowel of \fm{ÿee} reduced to a short [\ipa{ì}] or maybe [\ipa{ə̀}].

\ex\label{ex:93-12-bathe-for-war}%
\exmn{289.7}%
\begingl
	\glpreamble	K!ā′nqā dacū′tc. //
	\glpreamble	Xʼáang̱aa dashóoch. //
	\gla	{} \rlap{Xʼáang̱aa} @ {} {} 
		\rlap{dashóoch.} @ {} @ {} //
	\glb	{} xʼáan -g̱aa {}
		d- \rt[²]{shuch} -μμH //
	\glc	{}[\pr{PP} anger -\xx{ades} {}]
		\xx{mid}- \rt[²]{bathe} -\xx{var} //
	\gld	{} war -for {}
		\rlap{\xx{impfv}.self.bathe} {} {} //
	\glft	‘They bathe for war.’
		//
\endgl
\xe

\citeauthor{swanton:1909}’s gloss of (\lastx) is “They always bathed for strength in war”.
The “always” suggests a habitual but the transcription is \orth{dacū′tc} and not something like \orth{udacū′tctc} \fm{udashoochch} which would have the \fm{∅}-conjugation perfective prefix \fm{u-} and the repetitive suffix \fm{-ch} expected for the habitual of a \fm{∅}-conjugation verb.
The form \orth{dacū′tc} is thus \fm{dashóoch} ‘s/he bathes self’ which is an imperfective rather than a habitual.
The PP that \citeauthor{swanton:1909} transcribes as \orth{K!ā′nqā} is probably \fm{xʼáang̱aa} with \fm{xʼáan} ‘anger’ that is related to \fm{\rt[²]{kʼan}} ‘hate; repel, shoo away’ and \fm{x̱ʼaan} ‘fire’.
The adessive postposition \fm{-g̱áa} has a spatial interpretation of ‘nearby, in the area of, around’ but is used metaphorically to mean ‘in order to, for the purpose of, to obtain’.
Thus \fm{xʼáang̱aa} literally means something like ‘in preparation for anger’, calling back to (\ref{ex:93-4-bathe-for-trouble}).

\ex\label{ex:93-13-relatives-beating-each-other}%
\exmn{289.7}%
\begingl
	\glpreamble	Dūxō′nq!î kaductᴀ′nîn wucdaxê′ct atq!ayē′tc hīnq!. //
	\glpreamble	Du x̱oonxʼí kadushtánín woosh dax̱ísht atxʼaayích héenxʼ. //
	\gla	{} Du \rlap{x̱oonxʼí} @ {} @ {} {}
		\rlap{kadushtánín} @ {} @ {} @ {} @ {} @ {} @ {} +
		{} woosh @ \rlap{dax̱ísht} @ {} @ {} @ {}
			{} \rlap{atxʼaayích} @ {} @ {} @ {} {}
			{} \rlap{héenxʼ.} @ {} {} {} //
	\glb	{} du x̱oon -xʼ -í {}
		k- du- d- sh- \rt[²]{tan} -μH -ín
		{} woosh= d- \rt[²]{x̱ish} -μH -t
			{} at= xʼaa -í -ch {}
			{} héen -xʼ {} {} //
	\glc	{}[\pr{DP} \xx{3h·pss} relative -\xx{pl} -\xx{pss} {}]
		\xx{qual}- \xx{4h·s}- \xx{mid}- \xx{pej}- \rt[²]{habit} -\xx{var} -\xx{past}
		{}[\pr{CP} \xx{recip·o}= \xx{mid}- \rt[²]{club} -\xx{var} -\xx{ict}
			{}[\pr{DP} \xx{4n·pss}= tip -\xx{pss} -\xx{instr} {}]
			{}[\pr{PP} water -\xx{loc} {}] {}] //
	\gld	{} his \rlap{relatives} {} {} {}
		\rlap{\xx{impfv}.people.have·habit.\xx{past}} {} {} {} {} {} {}
		{} ea·oth \rlap{\xx{impfv}.club.\xx{rep}} {} {} {}
			{} thing’s\• tip -of -with {}
			{} water -in {} //
	\glft	‘His relatives had the habit of beating each other with boughs in the water.’
		//
\endgl
\xe

The noun \fm{atxʼaayí} in (\lastx) is not documented elsewhere.
\citeauthor{swanton:1909} transcribes it as \orth{atq!ayē′} and glosses it as “tree boughs”, translating it as “boughs”.
There are three lexical items that seem to be related: the root \fm{\rt[¹]{xʼa}} ‘twist to make limber’ \parencite[f04/1–2]{leer:1973}, the noun \fm{xʼaa} ‘point (of land), tip’ \parencite[f04/3–5]{leer:1973}, and \fm{xʼaan} ‘tip, point (of object)’ \parencite[f04/8]{leer:1973}.
The intended meaning must be something like ‘tips of branches’ because then this describes the traditional practice of whipping the skin with the tips of alder branches after emerging from a cold bath for strength \parencites[516–517]{de-laguna:1972}[300]{mcclellan:1975a}[358]{mcclellan:1975b}[335]{emmons:1991}.
In this context we should expect \fm{xʼaan} over \fm{xʼaa} given documentation like \fm{wasʼxʼaan tléig̱u} ‘salmonberry’ (lit.\ ‘bush tip berry’) and \fm{a xʼaan} ‘its sharp edge, tip, peak, tips of its branches’ \parencite[f04/8]{leer:1973}, but \citeauthor{swanton:1909}’s \orth{y} is hard to construe as a recording of [\ipa{n}] so \fm{xʼaa} is presumably the actual noun used here.

\citeauthor{swanton:1909} glosses and translates \fm{du x̱oonxʼí} in (\lastx) as “his friends”.
Given the relationship so far between the protagonist and the other people in this story, the noun \fm{x̱oon} here is probably better translated as ‘relative’.

\ex\label{ex:93-14-thick-tree}%
\exmn{289.8}%
\begingl
	\glpreamble	Yēkᵘʟā′ ās ā′wua //
	\glpreamble	Yéi kootláaÿi aas áwu á. //
	\gla	{} {} Yéi @ \rlap{kootláaÿi} @ {} @ {} @ {} @ {} @ {} {} aas {}
		\rlap{áwu} @ {}
		{} á. {} //
	\glb	{} {} yéi= k- u- i- \rt[¹]{tlaʰ} -μμH -i {} aas {}
		á -ú
		{} á {} //
	\glc	{}[\pr{DP} {}[\pr{CP} thus= \xx{cmpv}- \xx{irr}- \xx{stv}- \rt[¹]{girth} -\xx{var} -\xx{rel} {}] tree {}]
		\xx{cpl} -\xx{locp}
		{}[\pr{DP} \xx{3n} {}] //
	\gld	{} {} thus \rlap{\xx{cmpv}.big·around} {} {} {} {} -that {} tree {} 
		it.is -at
		{} it {} //
	\glft	‘There was a wide tree there.’
		//
\endgl
\xe

\citeauthor{swanton:1909}’s \orth{Yēkᵘʟā′ ās} in (\lastx) is peculiar.
It seems to be a relative clause with the verb modifying the noun \fm{aas}.
But given that this verb is a comparative state imperfective form it should have the stative prefix \fm{i-} and the relative clause suffix \fm{-i}.
Both of these are apparently missing in in \citeauthor{swanton:1909}’s \orth{Yēkᵘʟā′ ās} that instead reflects something like \fm[*]{yéi kwtláa aas} which is ungrammatical.
The form given here has been adjusted under the assumption that \citeauthor{swanton:1909}’s transcription is flawed.

\ex\label{ex:93-15-trees-penis}%
\exmn{289.8}%
\begingl
	\glpreamble	adanᴀ′x yut qā′waᴀ as-ʟ!ē′łî  //
	\glpreamble	A daanáx̱ yóot kaawa.áa aas tlʼíli. //
	\gla	{} A \rlap{daanáx̱} @ {} {}
		{} \rlap{yóot} @ {} {}
		\rlap{kaawa.áa} @ {} @ {} @ {} @ {}
		{} aas \rlap{tlʼíli.} @ {} {} //
	\glb	{} a daa -náx̱ {}
		{} yú -t {}
		k- wu- i- \rt[¹]{.a} -μμH
		{} aas tlʼíl -í {} //
	\glc	{}[\pr{PP} \xx{3n·pss} around -\xx{perl} {}]
		{}[\pr{PP} \xx{dist} -\xx{pnct} {}]
		\xx{qual}- \xx{pfv}- \xx{stv}- \rt[¹]{extend} -\xx{var}
		{}[\pr{DP} tree penis -\xx{pss} {}] //
	\gld	{} its bark -along {}
		{} there -to {}
		\rlap{\xx{pfv}.grow} {} {} {} {} 
		{} tree penis -of {} //
	\glft	‘Along its bark grew out the tree’s penis.’
		//
\endgl
\xe

The phrase \fm{aas tlʼíli} ‘tree’s penis’ in (\lastx) is a characteristic part of this story.
In the version told by \fm{Táakw Kʼwátʼi} Frank Johnson to the Dauenhauers the whole tree is referred to as \fm{aan loowú} ‘village’s nose’ \parencite[142.96]{dauenhauer:1987}.
\fm{Ḵeixwnéi} Nora Marks Dauenhauer says that he explained after the recording that the name was actually \fm{aan laawú} ‘tree’s penis’ with \fm{laaw} [\ipa{ɬàːw}] ‘penis’ (Dauenhauer p.c.\ 2010).
Both \fm{laaw} and \fm{tlʼíl} are used for ‘penis’ though today \fm{laaw} seems to be more common.
\citeauthor{leer:1978b} does not have \fm{laaw} in his stem collection \parencite{leer:1973} but he does list it in his stem list \parencite[29]{leer:1978b} where it is lacking related vocabulary.
He also has \fm{–tlʼíli} ‘penis’ which he connects to \fm{tlʼíl} ‘hammer’ \parencite[35]{leer:1978b} but this latter word is absent in his stem collection \parencite{leer:1973}.
\citeauthor{swanton:1909} transcribes \orth{ʟ!ē′łî} which implies \fm{tlʼéili} [\ipa{tɬʼéː.ɬì}] but this normally refers to a male fish’s milt (semen) or guts \parencites[08/231]{leer:1973}[35]{leer:1978b}; this is probably related but is not usually considered to be the same referent.
Phonologically similar lexical items include \fm{\rt[²]{lʼilʼ}} ‘defecate’ and \fm{\rt[²]{lilʼ}} ‘pull along length’ but these are not obviously related.

\ex\label{ex:93-16-test-strength}%
\exmn{289.9}%
\begingl
	\glpreamble	qāłatsī′ne ā′kdoaq. //
	\glpreamble	Ḵaa latseení áa kdu.aaḵw. //
	\gla	{} Ḵaa {} \rlap{latseení} @ {} @ {} @ {} {} {} {}
		{} \rlap{áa} @ {} {}
		\rlap{kdu.aaḵw.} @ {} @ {} @ {} //
	\glb	{} Ḵaa {} l- \rt[¹]{tsin} -μμL {} {} -í {}
		{} á -μ {}
		k- du- \rt[²]{.aḵw} -μμL //
	\glc	{}[\pr{DP} \xx{4h·pss} {}[\pr{NP} \xx{xtn}- \rt[¹]{alive} -\xx{var} \·\xx{nmz} {}] -\xx{pss} {}]
		{}[\pr{PP} \xx{3n} -\xx{loc} {}]
		\xx{qual}- \xx{4h·s}- \rt[²]{attempt} -\xx{var} //
	\gld	{} one’s {} \rlap{strength} {} {} {} {} {} {}
		{} there -at {}
		\rlap{\xx{impfv}.people.test} {} {} {} //
	\glft	‘People test their strength there.’
		//
\endgl
\xe

The pronouns \fm{ḵaa} ‘someone’s, people’s’ and \fm{du-} ‘someone, people’ in (\lastx) can be interpreted either as conjoint reference where both pronouns refer to the same entity, or as disjoint reference where each pronoun refers to different entities.
\citeauthor{swanton:1909}’s translation “They tried their strength” suggests that the two pronouns should be conjoint: each potential referent of the subject \fm{du-} is identical to a referent of the possessor \fm{ḵaa}.
But his gloss of \orth{qāłatsī′ne} \fm{ḵaa latseení} is “human strength”.
This suggests that the referent of \fm{ḵaa} is generic rather than the same specific entity referred to by \fm{du-}.
The translation here follows \citeauthor{swanton:1909}’s translation rather than his gloss, and thus implies conjoint reference.
Disjoint reference translations following \citeauthor{swanton:1909}’s gloss are e.g.\ ‘people test one’s strength there’ or ‘they test people’s strength there’.

\ex\label{ex:93-17-out-of-water-push}%
\exmn{289.9}%
\begingl
	\glpreamble	Tc!uʟe′ hīn dᴀx dāq āłunag̣o′qo awe′ hē′deqekdułx̣ī′tc tcūc q!ānādᴀ′x. //
	\glpreamble	Chʼu tle héendáx̱ daaḵ ÿaa lunagúg̱u áwé héide kei kdulx̱eechch chush x̱ʼanaadáx̱. //
	\gla	{} Chʼu tle {} \rlap{héendáx̱} @ {} {}
			daaḵ @ ÿaa @ \rlap{lunagúg̱u} @ {} @ {} @ {} @ {} {} 
		\rlap{áwé} @ {} +
		{} \rlap{héide} @ {} {}
		kei @ \rlap{kdulx̱eechch} @ {} @ {} @ {} @ {} @ {} @ {} @ {}
		{} chush \rlap{x̱ʼanaadáx̱.} @ {} @ {} {} //
	\glb	{} chʼu tle {} héen -dáx̱ {}
			daaḵ= ÿaa= lu- n- \rt[²]{guḵ} -μH -í {}
		á -wé
		{} hé -dé {}
		kei= k- u- du- d- l- \rt[²]{x̱ich} -μμL -ch
		{} chush x̱ʼé- niÿaa -dáx̱ {} //
	\glc	{}[\pr{CP} just then {}[\pr{DP} water -\xx{abl} {}]
			\xx{abmar}= along= nose- \xx{ncnj}- \rt[²]{push} -\xx{var} -\xx{sub} {}]
		\xx{foc} -\xx{mdst}
		{}[\pr{PP} \xx{mprx} -\xx{all} {}]
		up= \xx{qual}- \xx{zpfv}- \xx{4h·s}- \xx{mid}- \xx{xtn}- \rt[²]{throw·anim} -\xx{var} -\xx{rep}
		{}[\pr{PP} \xx{rflx·pss} mouth- dir’n -\xx{abl} {}] //
	\gld	{} just then {} water -from {}
			inland along \rlap{\xx{prog}.run·\xx{pl}} {} {} {} -when {}
		\rlap{it.is} {}
		{} here -to {}
		up \rlap{\xx{hab}.people.shove} {} {} {} {} {} {} {}
		{} self’s mouth- dir’n -from {} //
	\glft	‘Then it is as they are running out of the water that people always shoved him over out of their way.’
		//
\endgl
\xe

Both verbs in (\lastx) are relatively straightforward, common vocabulary but they are semantically interesting in this context.
The verb \fm{ÿaa lunagúḵ} ‘when they are running’ specifically describes plural entities running and contrasts with \fm{ÿaa nashíx} (\fm{\rt[¹]{xix}}) ‘when s/he is running’ which describes a singular entity running.
This is an idiom formed with the root \fm{\rt[²]{guḵ}} ‘shove, push’ and the incorporate \fm{lu-} from \fm{lú} ‘nose’ \parencites[f05/180–182]{leer:1973}[674–677]{leer:1976}.
\citeauthor{leer:1991} argues that this idiom arose from \fm{\rt[²]{guḵ}} as ‘move elongated end forward’ with \fm{lu-} as ‘end, point’ so that the meaning was originally ‘move point of elongated group forward’ which he says “evokes the elongated ovoid characteristic of a group running” \parencite[51]{leer:1991}.
\citeauthor{swanton:1909} actually has \orth{dāq āłunag̣o′qo} which if taken literally implies \fm{daaḵ aa lunagúg̱u}, but this would be ungrammatical and the idiosyncratic lenition or deletion of \fm{ÿ} after another consonant is fairly common.

The verb phrase \fm{héide kei kdulx̱eechch} in (\lastx) is a habitual aspect form with a couple of motion derivations applied to it.
The root is \fm{\rt[²]{x̱ich}} ‘throw animate’ as discussed earlier in (\ref{ex:93-10-dump-urine}).
\textcite[149]{boas:1917} identifies the same root here so \citeauthor{swanton:1909}’s gloss “kicked” is mistaken.
The \fm{kei=} ‘up’ preverb indicates the motion derivation \vbderiv{kei}{∅}{\fm{-ch} repetitive}{upward}.
The \fm{-ch} repetitive suffix could either indicate the habitual aspect or it could reflect the repetitive imperfective specified by this motion derivation.
The habitual of a \fm{∅}-conjugation verb would predictably have the \fm{∅}-conjugation perfective prefix \fm{u-} except that when the \fm{du-} subject is present the \fm{u-} may disappear \parencite[110]{leer:1991}.
A repetitive imperfective would predictably lack an aspectual prefix.
Here \citeauthor{swanton:1909}’s gloss and translation “they always kicked him” supports the habitual analysis.
The PP \fm{héide} ‘over here; aside’ may reflect another motion derivation \vbderiv{héide ÿaa= \~\ ÿ-u-}{∅}{\fm{-ch} repetitive}{over that way, aside, out of the way}.
\citeauthor{swanton:1909}’s transcription does not admit the recovery of either \fm{ÿaa=} or \fm{ÿ-u-} so it is also possible that \fm{héide} is simply an adjunct and does not reflect a motion derivation operation.

\ex\label{ex:93-18-break-trees-penis}%
\exmn{289.9}%
\begingl
	\glpreamble	“Yāqā′ qo′a xas ās-ʟ!ē′łî aqᵒgwał!ī′q!.” //
	\glpreamble	«\!Yá ḵáa ḵu.aa x̱ách aas tlʼíli akg̱walʼéexʼ.\!» //
	\gla	{} \llap{«\!}Yá ḵáa {} ḵu.aa x̱ách
		{} aas \rlap{tlʼíli} @ {} {}
		\rlap{akg̱walʼéexʼ.\!»} @ {} @ {} @ {} @ {} @ {} //
	\glb	{} yá ḵáa {} ḵu.aa x̱áju
		{} aas tlʼíl -í {}
		a- w- g- g̱- \rt[²]{lʼixʼ} -μμH //
	\glc	{}[\pr{DP} \xx{prox} man {}] \xx{contr} actually
		{}[\pr{DP} tree penis -\xx{pss} {}]
		\xx{arg}- \xx{irr}- \xx{gcnj}- \xx{mod} \rt[²]{break} -\xx{var} //
	\gld	{} this man {} however actually
		{} tree penis -of {}
		\rlap{3>3.\xx{prsp}.break} {} {} {} {} {} //
	\glft	‘“This man however is actually going to break the tree’s penis.”’
		//
\endgl
\xe

\citeauthor{swanton:1909} glosses his \orth{xas} as “when” but this is nonsense.
Given the context his transcription could be a mishearing of \fm{x̱ách} ‘actually’.
This particle is more fully \fm{x̱áju} or \fm{ḵáju} \parencite[74]{leer:1978b}, but it is often shortened to \fm{ḵách} or \fm{x̱ách} as also seen earlier in (\ref{ex:93-8-sleeping-bathe}).

\citeauthor{swanton:1909} gives the sentence in (\lastx) in quotation maks and translates it with an editorial addition “[they said]” at the end of the sentence.
This implies that the sentence is spoken by the townspeople, and hence that the speaker is mocking the protagonist.
There is an alternative interpretation, however.
Rather than something spoken by a character, the sentence in (\lastx) could be an interstitial comment from the narrator which anticipates the later outcome of the story where \fm{Kaháasʼi} does actually break the tree’s penis.
The translation here maintains \citeauthor{swanton:1909}’s original presentation but this should not be taken to rule out the interpretation of this sentence as background commentary.

\section{Paragraph 2}\label{sec:093-para-2}

\ex\label{ex:93-19-into-the-water}%
\exmn{290.1}%
\begingl
	\glpreamble	Hū′tc!îaỵe hīn xēqgwagū′di yuqā′ hīnx ugū′t. //
	\glpreamble	Hóochʼi aaÿí héenx̱ yei kg̱wagoodí, yú ḵáa héenx̱ woogoot. //
	\gla	{} {} Hóochʼi \rlap{aaÿí} @ {} {}
			{} \rlap{héenx̱} @ {} {}
			yei @ \rlap{kg̱wagoodí} @ {} @ {} @ {} @ {} @ {} {} +
		{} yú ḵáa {}
		{} \rlap{héenx̱} @ {} {}
		\rlap{woogoot.} @ {} @ {} @ {} //
	\glb	{} {} hóochʼ aa -í {}
			{} héen -x̱ {}
			yei= w- g- g̱- \rt[¹]{gut} -μμL -í {}
		{} yú ḵáa {}
		{} héen -x̱ {}
		wu- i- \rt[¹]{gut} -μμH //
	\glc	{}[\pr{CP} {}[\pr{DP} last \xx{part} -\xx{pss} {}]
			{}[\pr{PP} water -\xx{pert} {}]
			down= \xx{irr}- \xx{gcnj}- \xx{mod}- \rt[¹]{go·\xx{sg}} -\xx{var} -\xx{sub} {}]
		{}[\pr{DP} \xx{dist} man {}]
		{}[\pr{PP} water -\xx{pert} {}]
		\xx{pfv}- \xx{stv}- \rt[¹]{go·\xx{sg}} -\xx{var} //
	\gld	{} {} last one -of {}
			{} water -into {}
			down \rlap{\xx{prsp}.go·\xx{sg}} {} {} {} {} -when {}
		{} that man {}
		{} water -into {}
		\rlap{\xx{pfv}.go·\xx{sg}} {} {} {} //
	\glft	‘When the last one is going to go down into the water, that man went into the water.’
		//
\endgl
\xe

\ex\label{ex:93-20-heard-someone}%
\exmn{290.1}%
\begingl
	\glpreamble	Tca′tc!a ag̣ā′awe āg̣a′ yêk ū′waᴀx atūwā′tx qeg̣o′xłatsīn //
	\glpreamble	Cha chʼa aag̱áa yéi ḵoowa.áx̱, a tuwáatx̱ kei gux̱latséen. //
	\gla	Cha chʼa {} \rlap{aag̱áa} @ {} {} \rlap{áwé} @ {}
		yéi @ \rlap{ḵoowa.áx̱,} @ {} @ {} @ {} @ {} +
		{} {} a \rlap{tuwáatx̱} @ {} @ {} {}
			kei @ \rlap{gux̱latséen.} @ {} @ {} @ {} @ {} @ {} @ {} {} //
	\glb	cha chʼa {} á -g̱áa {} á -wé
		yéi= ḵu- wu- i- \rt[²]{.ax̱} -μH
		{} {} a tú- ÿá -dáx̱ {}
			kei= w- g- g̱- l- \rt[¹]{tsin} -μμH {} {} //
	\glc	\xx{interj} just {}[\pr{PP} \xx{3n} -\xx{ades} {}] \xx{foc} -\xx{mdst}
		thus= \xx{4h·o}- \xx{pfv}- \xx{stv}- \rt[²]{hear} -\xx{var}
		{}[\pr{CP} {}[\pr{PP} \xx{3n·pss} mind- face -\xx{abl} {}]
			up= \xx{irr}- \xx{gcnj}- \xx{mod}- \xx{xtn}- \rt[¹]{alive} -\xx{var} \·\xx{sub} {}] //
	\gld	well just {} there -near {} \rlap{it.is} {}
		thus \rlap{someone.\xx{pfv}.hear} {} {} {} {}
		{} {} its mind- face -from {}
			up \rlap{\xx{prsp}.strong} {} {} {} {} {} -that {} //
	\glft	‘Well then it is around there that he heard someone, that from it he will become strong.’
		//
\endgl
\xe

\ex\label{ex:93-21-heard-someone}%
\exmn{290.2}%
\begingl
	\glpreamble	adayu′ asᴀ′ ᴀwaᴀ′x. //
	\glpreamble	Át áyú aseiwa.áx̱. //
	\gla	{} \rlap{Át} @ {} {} \rlap{áyú} @ {}
		\rlap{aseiwa.áx̱.} @ {} @ {} @ {} @ {} @ {} //
	\glb	{} á -t {} á -yú
		a- se- wu- i- \rt[²]{.ax̱} -μH //
	\glc	{}[\pr{PP} \xx{3n} -\xx{pnct} {}] \xx{foc} -\xx{dist}
		\xx{arg}- voice- \xx{pfv}- \xx{stv}- \rt[²]{hear} -\xx{var} //
	\gld	{} there -at {} \rlap{it.is} {}
		\rlap{3>3.voice.\xx{pfv}.hear} {} {} {} {} {} //
	\glft	‘It is there that he heard its voice.’
		//
\endgl
\xe

\ex\label{ex:93-22-called-strength}%
\exmn{290.2}%
\begingl
	\glpreamble	Łatsī′n yūdowasākᵘ. //
	\glpreamble	Latseen yóo duwasáakw. //
	\gla	{} \rlap{Latseen} @ {} @ {} @ {} {} 
		yóo @ \rlap{duwasáakw.} @ {} @ {} @ {} @ {} //
	\glb	{} l- \rt[¹]{tsin} -μμL {} {}
		yóo= du- i- \rt[²]{sa} -μμH -kw //
	\glc	{}[\pr{DP} \xx{xtn}- \rt[¹]{alive} -\xx{var} \·\xx{nmz} {}]
		\xx{quot}= \xx{4h·s}- \xx{stv}- \rt[²]{call} -\xx{var} -\xx{rep} //
	\gld	{} \rlap{strength} {} {} {} {}
		thus \rlap{\xx{impfv}.people.be.call.\xx{rep}} {} {} {} {} //
	\glft	‘People call it Strength.’
		//
\endgl
\xe

\ex\label{ex:93-23-came-behind}%
\exmn{290.3}%
\begingl
	\glpreamble	Tc!uʟe′ act!ā′t uwagu′t. //
	\glpreamble	Chʼu tle ash táat uwagút. //
	\gla	Chʼu tle {} ash \rlap{tʼáat} @ {} {}
		\rlap{uwagút.} @ {} @ {} @ {} //
	\glb	chʼu tle {} ash tʼáaᵏ -t {}
		u- i- \rt[¹]{gut} -μH //
	\glc	just then {}[\pr{PP} \xx{3prx·pss} behind -\xx{pnct} {}]
		\xx{zpfv}- \xx{stv}- \rt[¹]{go·\xx{sg}} -\xx{var} //
	\gld	just then {} his behind -to {}
		\rlap{\xx{pfv}.go·\xx{sg}} {} {} {} //
	\glft	‘Just then it came behind him.’
		//
\endgl
\xe

\ex\label{ex:93-24-head-big}%
\exmn{290.3}%
\begingl
	\glpreamble	Yē′kᵘge ducᴀ′ //
	\glpreamble	Yéi koogéi du shá; // 
	\gla	Yéi @ \rlap{koogéi} @ {} @ {} @ {} @ {}
		{} du shá {} //
	\glb	yéi= k- u- i- \rt[¹]{ge} -μμH
		{} du shá {} //
	\glc	thus= \xx{cmpv}- \xx{irr}- \xx{stv}- \rt[¹]{big} -\xx{var}
		{}[\pr{DP} \xx{3h·pss} head {}] //
	\gld	yay \rlap{\xx{impfv}.\xx{cmpv}.be.big} {} {} {} {}
		{} his head {} //
	\glft	‘His head was yay big;’
		//
\endgl
\xe

\ex\label{ex:93-25-shriveled-like-that}%
\exmn{290.4}%
\begingl
	\glpreamble	wu′ʟ̣ēq!ᴀq!a yêx ỵate′. //
	\glpreamble	wudlix̱ʼéx̱ʼ, a yáx̱ ÿatee. //
	\gla	\rlap{wudlix̱ʼéx̱ʼ,} @ {} @ {} @ {} @ {} @ {}
		{} a yáx̱ {}
		\rlap{ÿatee.} @ {} @ {} //
	\glb	wu- d- l- i- \rt[¹]{x̱ʼix̱ʼ} -μH
		{} á yáx̱ {}
		i- \rt[¹]{tiʰ} -μμL //
	\glc	\xx{pfv}- \xx{psv}- \xx{csv}- \xx{stv}- \rt[¹]{burnt} -\xx{var}
		{}[\pr{PP} \xx{3n} \xx{sim} {}]
		\xx{stv}- \rt[¹]{be} -\xx{var} //
	\gld	 \rlap{\xx{pfv}.\xx{pasv}.make.shrivel} {} {} {} {} {}
		{} it like {}
		\rlap{\xx{impfv}.be} {} {} //
	\glft	‘it was shrivelled up, it is like that.’
		//
\endgl
\xe

The verb \fm{wudlix̱ʼéx̱ʼ} in (\lastx) is a passivized causativized result perfective literally meaning ‘it has been burned, shriveled by heat’.
The root \fm{\rt[¹]{x̱ʼix̱}} \~\ \fm{\rt[¹]{x̱ʼex̱ʼ}} is found in a few different verbs that describe disfigurement by heat or direct flame: \fm{wé atdoogú wudix̱ʼéx̱ʼ} ‘that skin has been burned, shrivelled up by heat’ \parencite[38.362]{story-naish:1973}, \fm{ax̱ jín x̱walix̱ʼéx̱ʼ, yú sdoox tóot ax̱a.aagí} ‘I burned my hand as I was making a fire in the stove’ \parencite[388.364]{story-naish:1973}, \fm{yá kaxʼásʼdi g̱agaanch yaa kanalx̱ʼéx̱ʼ} ‘that lumber, the sun is warping it’ \parencite[242]{story-naish:1973}, \fm{awlix̱ʼíx̱ʼ} ‘he seared it on coals, raw’ \parencite[f01/188]{leer:1973}.
\citeauthor{swanton:1909} glosses this as “curly” and then extrapolates from this for his translation “He had a large head covered with curly hair”.
The latter half of (\lastx) – \fm{a yáx̱ ÿatee} ‘it is like that’ – implies that the narrator was pointing out something to \citeauthor{swanton:1909} during his telling.
This could plausibly have been an animal skin shrivelled by fire, with \citeauthor{swanton:1909} interpreting the gesture as referring to the fur (perhaps wool?)\ where its intended meaning was instead the skin.
In any case, \fm{\rt[¹]{x̱ʼix̱ʼ}} \~\ \fm{\rt[¹]{x̱ʼex̱ʼ}} is not normally used for curly hair which is instead \fm{\rt[¹]{gwalʼ}} ‘curl’ or \fm{\rt[¹]{kuchʼ}} ‘curl’.

\ex\label{ex:93-26-boughs}%
\exmn{290.4}%
\begingl
	\glpreamble	Atq!āyē′ dutcī′. //
	\glpreamble	Atxʼaayí du jée. //
	\gla	{} \rlap{Atxʼaayí} @ {} @ {} {} 
		{} du \rlap{jée.} @ {} {} //
	\glb	{} at= xʼaa -í {}
		{} du jee -H {} //
	\glc	{} \xx{4n·pss}= tip -\xx{pss} {}
		{} \xx{3h·pss} poss’n -\xx{loc} {} //
	\gld	{} thing’s\• tip -of {}
		{} his poss’n -in {} //
	\glft	‘He had boughs.’
		//
\endgl
\xe

\ex\label{ex:93-27-cmere}%
\exmn{290.4}%
\begingl
	\glpreamble	“Hākᵘ de” yū′aciaosiqa, //
	\glpreamble	«\!Haagú dé\!» yóo ash yawsiḵaa, //
	\gla	\rlap{«\!Haagú} @ {} dé\!» yóo @ ash @ \rlap{yawsiḵaa,} @ {} @ {} @ {} @ {} @ {} //
	\glb	\pqp{}haa= gú dé
		yóo= ash= ÿ- wu- s- i- \rt[¹]{ḵa} -μμL //
	\glc	\pqp{}\xx{cis}= go·\xx{imp} now
		\xx{quot}= \xx{3prx·o}= \xx{qual}- \xx{pfv}- \xx{csv}- \xx{stv}- \rt[¹]{say} -\xx{var} //
	\gld	\pqp{}\rlap{come·here} {} now
		thus him \rlap{\xx{pfv}.say} {} {} {} {} {} //
	\glft	‘“Come here” he said to him,’
		//
\endgl
\xe

The speech verb in (\lastx) is an excellent example of the proximate third person used in narrative.
The newly introduced \fm{Latseen} ‘Strength’ is another third person alongside the protagonist \fm{Kaháasʼi}.
If the verb used ordinary third person marking with the three-on-three \fm{a-} as in \fm{yóo ayawsiḵaa} ‘s/he said to him/her’ then the subject would be ambiguous: it could be either \fm{Kaháasʼi} or \fm{Latseen}.
The third person proximate \fm{ash=} disambiguates this situation because it can only refer to the protagonist as the foreground third person.
Compare (\ref{ex:93-23-came-behind}) where the third person proximate possessive pronoun is used for the same effect.

\ex\label{ex:93-28-come-to-me-now}%
\exmn{290.4}%
\begingl
	\glpreamble	“Ā′xdjīt gu de.” //
	\glpreamble	«\!Ax̱ jeet gú de\!». //
	\gla	{} \llap{«\!}Ax̱ \rlap{jeet} @ {} {}
		\rlap{gú} @ {} @ {} @ {}
		dé\!». //
	\glb	{} ax̱ jee -t {}
		{} {} \rt[¹]{gut} -⊗
		dé //
	\glc	{}[\pr{PP} \xx{1sg·pss} poss’n -\xx{pnct} {}]
		\xx{zcnj}\· \xx{2sg·s}\· \rt[¹]{go·\xx{sg}} -\xx{var}
		now //
	\gld	{} my hand -to {}
		\rlap{\xx{imp}.you.go·\xx{sg}} {} {} {}
		now //
	\glft	‘“Come to me now”.’
		//
\endgl
\xe

The PP \fm{ax̱ jeet} in (\lastx) is literally ‘to my possession’ with the relational noun \fm{jee} [\ipa{tʃ\!ìː}] ‘possession’.
The noun \fm{jín} [\ipa{tʃín}] ‘hand’ might be expected here, but because both nouns are semantically related the noun \fm{jee} is occasionally used instead of \fm{jín} as seems to be the case here.
The English translation avoids the interpretive problem of choosing ‘possession’ or ‘hand’ by using the object pronoun \fm{me}.

\ex\label{ex:93-29-went-to-it}%
\exmn{290.5}%
\begingl
	\glpreamble	Tc!uʟe′ adjīỵī′t ūwagu′t. //
	\glpreamble	Chʼu tle a jiÿeet uwagút. //
	\gla	Chʼu tle {} a \rlap{jiÿeet} @ {} @ {} {}
		\rlap{uwagút.} @ {} @ {} @ {} //
	\glb	chʼu tle {} a jín- ÿee -t {} 
		u- i- \rt[¹]{gut} -μH //
	\glc	just then {}[\pr{PP} \xx{3n·pss} hand- below -\xx{pnct} {}]
		\xx{zpfv}- \xx{stv}- \rt[¹]{go·\xx{sg}} -\xx{var} //
	\gld	just then {} its hand- below -to {}
		\rlap{\xx{pfv}.go·\xx{sg}} {} {} {} //
	\glft	‘So then he went below its hand.’
		//
\endgl
\xe

The phrase \fm{a jiÿeet} in (\lastx) is literally ‘to the below of its hand’.
This has two idiomatic interpretations, one being ‘ready and waiting for him/her to use’ (compare English ‘at hand’) and the other being ‘under the burden of it, suffering from it’.
Neither seems appropriate in this case, so the literal meaning must apply.
A loose approximation in English might be ‘he went to its hand’, as in the phrase ‘come to my hand’ which idiomatically means the same as ‘come to me’.

\ex\label{ex:93-30-threw-in-water}%
\exmn{290.5}%
\begingl
	\glpreamble	Tc!uʟe′ hīn nᴀx ac ᴀqa′ołīxetc. //
	\glpreamble	Chʼu tle héennáx̱ ash kawlix̱eich. //
	\gla	Chʼu tle {} \rlap{héennáx̱} @ {} {}
		ash @ \rlap{kawlix̱eich.} @ {} @ {} @ {} @ {} @ {} //
	\glb	chʼu tle {} héen -náx̱ {}
		ash= k- wu- l- i- \rt[²]{x̱ich} -μμL //
	\glc	just then {}[\pr{PP} water -\xx{perl} {}]
		\xx{3prx·o}= \xx{qual}- \xx{pfv}- \xx{xtn}- \xx{stv}- \rt[²]{throw·anim} -\xx{var} //
	\gld	just then {} water -into {}
		him \rlap{\xx{pfv}.throw} {} {} {} {} {} //
	\glft	‘Then he threw him in the water.’
		//
\endgl
\xe

As in (\ref{ex:93-23-came-behind}) and (\ref{ex:93-27-cmere}), the third person proximate pronoun in (\lastx) avoids the potential ambiguity between the two third person referents.
The spirit could be referred to in English with the pronoun ‘it’ to avoid the ambiguity, but this would be misleading because (\ref{ex:93-24-head-big}) and (\ref{ex:93-26-boughs}) both use the third person human possessive pronoun \fm{du} ‘his, hers’ to refer to the spirit, although (\ref{ex:93-29-went-to-it}) uses the third person nonhuman possesive \fm{a} ‘its’.

\ex\label{ex:93-31-second-time}%
\exmn{290.6}%
\begingl
	\glpreamble	Dᴀxa′ ᴀc wuxō′x. //
	\glpreamble	Dáx̱.aa ash woox̱oox̱. //
	\gla	{} \rlap{Dáx̱.aa} @ {} {} ash @ \rlap{woox̱oox̱.} @ {} @ {} @ {} //
	\glb	{} déix̱- .aa {} ash= wu- i- \rt[²]{x̱ux̱} -μμL //
	\glc	{}[\pr{NP} two- \xx{part} {}] \xx{3prx·o}= \xx{pfv}- \xx{stv}- \rt[²]{summon} -\xx{var} //
	\gld	{} second- time {} him \rlap{\xx{pfv}.call} {} {} {} //
	\glft	‘He called him a second time.’
		//
\endgl
\xe

The noun phrase \fm{dáx̱.aa} in (\lastx) was translated by \citeauthor{swanton:1909} as “twice” but it is not actually \fm{deix̱dahéen} [\ipa{ˌtèːχ.tà.ˈhíːn}] ‘twice, two times’ with the \fm{-dahéen} ‘time(s)’ suffix for enumerated eventualities.
The phrase \fm{dáx̱.aa} [\ipa{ˈtáχ.ˌʔàː}] usually means ‘second one’ because of the partitive pronoun \fm{aa} ‘one (of count noun), some (of mass noun)’.
The use of \fm{dáx̱.aa} treats the summoning eventuality as a countable entity and the closest English approximation is ‘a second time’.

\ex\label{ex:93-32-beat-him-strongly}%
\exmn{290.6}%
\begingl
	\glpreamble	Ag̣ā′awe tsa łatsī′n dên ᴀc wuxî′ct. //
	\glpreamble	Aag̱áa áwé tsá latseendéin ash woox̱ísht. //
	\gla	{} \rlap{Aag̱áa} @ {} {} \rlap{áwé} @ {} tsá
		{} \rlap{latseendéin} @ {} @ {} @ {} {}
		ash @ \rlap{woox̱ísht.} @ {} @ {} @ {} @ {} //
	\glb	{} á -g̱áa {} á -wé tsá
		{} l- \rt[¹]{tsin} -μμL -déin {}
		ash= wu- i- \rt[²]{x̱ish} -μH -t //
	\glc	{}[\pr{PP} \xx{3n} -\xx{ades} {}] \xx{foc} -\xx{mdst} just
		{}[\pr{AdvP} \xx{xtn}- \rt[¹]{alive} -\xx{var} -\xx{adv} {}]
		\xx{3prx·o}= \xx{pfv}- \xx{stv}- \rt[²]{club} -\xx{var} -\xx{ict} //
	\gld	{} that -after {} \rlap{it.is} {} just
		{} \rlap{strong} {} {} -ly {}
		him \rlap{\xx{pfv}.whip.\xx{rep}} {} {} {} {} //
	\glft	‘Then it is just after that that he whipped him strongly.’
		//
\endgl
\xe

\ex\label{ex:93-33-i-am-strength}%
\exmn{290.6}%
\begingl
	\glpreamble	“Xā′daya Łatsī′n. //
	\glpreamble	«\!X̱át áyá Latseen. //
	\gla	{} \llap{«\!}X̱át {} \rlap{áyá} @ {}
		{} \rlap{Latseen.} @ {} @ {} @ {} {} //
	\glb	{} x̱át {} á -yá
		{} l- \rt[¹]{tsin} -μμL {} {} //
	\glc	{}[\pr{DP} \xx{1sg} {}] \xx{cpl} -\xx{prox}
		{}[\pr{DP} \xx{xtn}- \rt[¹]{alive} -\xx{var} \·\xx{nmz} {}] //
	\gld	{} me {} \rlap{it.is} {}
		{} \rlap{Strength} {} {} {} {} //
	\glft	‘“It is me, Strength.’
		//
\endgl
\xe

\ex\label{ex:93-34-i-will-help-you-he-said}%
\exmn{290.7}%
\begingl
	\glpreamble	Ī′īg̣ā xat wusu′,” ye acia′osîqa. //
	\glpreamble	I eeg̱áa x̱at woosoo\!» yéi ash yawsiḵaa. //
	\gla	{} {} I \rlap{eeg̱áa} @ {} {}
			x̱at @ \rlap{woosoo} @ {} @ {} @ {} {}
		yéi @ ash @ \rlap{yawsiḵaa.} @ {} @ {} @ {} @ {} @ {} //
	\glb	{} {} i ee -g̱áa {}
			x̱at= wu- i- \rt[¹]{suʰ} -μμL {}
		yéi= ash= ÿ- wu- s- i- \rt[¹]{ḵa} -μμL //
	\glc	{}[\pr{CP} {}[\pr{PP} \xx{2sg} \xx{base} -\xx{ades} {}]
			\xx{1sg·o}= \xx{pfv}- \xx{stv}- \rt[¹]{sup·help} -\xx{var} {}]
		thus= \xx{3prx·o}= \xx{qual}- \xx{pfv}- \xx{csv}- \xx{stv}- \rt[¹]{say} -\xx{var} //
	\gld	{} {} you {} -for {}
			me \rlap{\xx{pfv}.super·help} {} {} {} {}
		thus him \rlap{\xx{pfv}.say·to} {} {} {} {} {} //
	\glft	‘I have given supernatural help to you” he said to him.’
		//
\endgl
\xe

\ex\label{ex:93-35-break-that-thing}%
\exmn{290.8}%
\begingl
	\glpreamble	“Yū′ān a′ỵadᴀłtsī′n ᴀt q!wᴀn nᴀʟ!ī′q! //
	\glpreamble	Yú aan aÿadultseen át xʼwán nalʼéexʼ. //
	\gla	{} \llap{«\!}Yú {} {} \rlap{aan} @ {} {}
			\rlap{aÿadultseen} @ {} @ {} @ {} @ {} @ {} @ {} @ {} {} át {}
		xʼwán
		\rlap{nalʼéexʼ.} @ {} @ {} @ {} //
	\glb	{} yú {} {} á -n {}
			a- ÿ- du- d- l- \rt[¹]{tsin} -μμL {} {} át {}
		xʼwán
		n- {} \rt[²]{lʼixʼ} -μμH //
	\glc	{}[\pr{DP} \xx{dist} {}[\pr{CP} {}[\pr{PP} \xx{3n} -\xx{inst} {}]
			\xx{xpl}- \xx{qual}- \xx{4h·s}- \xx{mid}- \xx{xtn}- \rt[¹]{alive} -\xx{var} \·\xx{rel} {}] \xx{4n} {}]
		\xx{imp}
		\xx{ncnj}- \xx{2sg·s}\· \rt[²]{break} -\xx{var} //
	\gld	{} that {} {} it -with {}
			\rlap{\xx{impfv}.people.exercise} {} {} {} {} {} {} -which {} thing {}
		\xx{imp}
		\rlap{\xx{imp}.you·\xx{sg}.break} {} {} {} //
	\glft	‘“Break that thing with which people are exercising.’
		//
\endgl
\xe

\ex\label{ex:93-36-piss-on-it}%
\exmn{290.8}%
\begingl
	\glpreamble	aỵitī′t q!wan ᴀkłałū′q! //
	\glpreamble	A ÿeetéet xʼwán aklalóoxʼ. //
	\gla	{} A \rlap{ÿeetéet} @ {} {}
		xʼwán
		\rlap{aklalóoxʼ} @ {} @ {} @ {} @ {} @ {} @ {} //
	\glb	{} a ÿeetée -t {}
		xʼwán
		a- k- {} {} l- \rt[²]{luxʼ} -μμH //
	\glc	{}[\pr{PP} \xx{3n·pss} remains -\xx{pnct} {}]
		\xx{imp}
		\xx{xpl}- \xx{qual}- \xx{zcnj}\· \xx{2sg·s}\· \rt[²]{urinate} -\xx{var} //
	\gld	{} its remains -on {}
		\xx{imp}
		\rlap{\xx{imp}.you·\xx{sg}.urinate} {} {} {} {} {} {} //
	\glft	‘Urinate on the place where it is.’
		//
\endgl
\xe

The noun \fm{ÿeetí} [\ipa{ɰìː.ˈtʰí}] ‘remains of, space of’ in (\lastx) is unusual in that it is normally documented as \fm{eetí} [\ipa{ʔìː.ˈtʰí}] with an initial glottal stop /\ipa{ʔ}/ and not with an initial /\ipa{ɰ}/.
It is unclear whether this is regular variation that has disappeared in modern dialects or if it is idiosyncratic and limited to this speaker.
There is exactly one instance from \fm{Ḵaadashaan} in his \fm{Aakʼwtaatseen} narrative (line \ref{ex:100-12-flung-up-off}, p.\ \pageref{ex:100-12-flung-up-off}) which suggests it may have been widespread in the past since \fm{Deikeenaakʼw} is from Sitka and \fm{Ḵaadashaan} is from Wrangell.
Both had heritage in Klukwan as noted by \textcite{jones:2017} so this could reflect a Chilkat dialect feature, but there is no sign of it in transcriptions of Shotridge \parencite{boas:1917}.

\ex\label{ex:93-37-poke-it-there}%
\exmn{290.8}%
\begingl
	\glpreamble	ān ᴀ′tg̣e iỵatsa′q.” //
	\glpreamble	Aan át g̱eeÿtsaaḵ.\!» //
	\gla	{} \rlap{Aan} @ {} {}
		{} \rlap{át} @ {} {}
		\rlap{g̱eeÿtsaaḵ.\!»} //
	\glb	{} á -n {}
		{} á -t {}
		//
	\glft	‘With that poke it there.”’
		//
\endgl
\xe

The transcription in (\lastx) is difficult to interpret.
\citeauthor{swanton:1909}’s gloss is “with it” for \orth{ān} and “put it into it” for \orth{ᴀ′tg̣e iỵatsa′q.}.
This does not match \citeauthor{swanton:1909}’s translation “Put it back again along with some urine” so his translation is probably creative.
The initial \orth{ān} is, going by \citeauthor{swanton:1909}’s gloss, the PP \fm{aan} ‘with it’.
The remainder of the sentence is not as obvious.
The \orth{ᴀ′t} portion is likely the PP \fm{át} ‘on/in there, at it’ corresponding with \citeauthor{swanton:1909}’s gloss “into it”.
This leaves \orth{g̣e iỵatsa′q}, of which the \orth{tsa′q} portion can be identified as a verb stem \fm{tsaaḵ} [\ipa{tsʰàːq}] or \fm{tsáḵ} [\ipa{tsʰáq}].
This is based on the root \fm{\rt[²]{tsaḵ}} ‘push, poke, stick (with long object)’ as is also found in the nouns \fm{tsaag̱álʼ} [\ipa{tsʰàː.qáɬʼ}] ‘spear’ and \fm{wootsaag̱áa} [\ipa{wùː.tsʰàː.qáː}] ‘cane, walking stick’, and presumably corresponds with \citeauthor{swanton:1909}’s gloss of “put”.

The last portion of \orth{ᴀ′tg̣e iỵatsa′q.} in (\lastx) to interpret is \orth{g̣e iỵa}, as well as the length and tone of \orth{tsa′q}.
If \orth{g̣e iỵatsa′q} were retranscribed literally the result would be something like \fm{g̱e iÿatsaaḵ} [\ipa{qè.ʔì.ɰà.tsàːq}] or \fm{g̱eiÿatsaaḵ} [\ipa{qèː.ɰà.tsʰàːq}] but these are nonsense.
The two preceding sentences spoken by the spirit \fm{Latseen} are imperatives so we can reasonably expect this sentence also from \fm{Latseen} to be an imperative.
This excludes the interpretation of \orth{iỵatsa′q} as a perfective with a second person subject \fm{iÿatsáḵ} ‘you pushed, poked, stuck it’; \fm{Latseen} is giving an instruction, not describing an event that has already happened.
If the root \fm{\rt[²]{tsaḵ}} is used for a \fm{∅}-conjugation verb then its imperative form with a second person subject and third person object would predictably be either \fm{tsaaḵ} or \fm{tsáḵ} ‘push, poke, stick it’.

\section{Paragraph 3}\label{sec:093-para-3}

\citeauthor{swanton:1909} runs the preceding paragraph together with this one, resulting in a narrative with one  moderate length paragraph and one very long paragraph.
The sentence in (\nextx) jumps forward in time to the night and so the scene changes.
This is a relatively natural point for a paragraph break so one has been inserted here.

\ex\label{ex:93-38-ran-around-there}%
\exmn{290.9}%
\begingl
	\glpreamble	Tā′dawe ada′odjîx̣īx̣. //
	\glpreamble	Taat áwé át áa wjixeex. //
	\gla	{} Taat {} \rlap{áwé} @ {}
		{} \rlap{át} @ {} {}
		{} \rlap{áa} @ {} {}
		\rlap{wjixeex.} @ {} @ {} @ {} @ {} @ {} //
	\glb	{} taat {} á -wé
		{} á -t {}
		{} á -μ {}
		wu- d- sh- i- \rt[¹]{xix} -μμL //
	\glc	{}[\pr{NP} night {}] \xx{foc} -\xx{mdst}
		{}[\pr{PP} \xx{3n} -\xx{pnct} {}]
		{}[\pr{PP} \xx{3n} -\xx{loc} {}]
		\xx{pfv}- \xx{mid}- \xx{pej}- \xx{stv}- \rt[¹]{fall} -\xx{var} //
	\gld	{} night {} \xx{it.is} {}
		{} there -to {}
		{} there -around {}
		\rlap{\xx{pfv}.run·\xx{sg}} {} {} {} {} {} //
	\glft	‘At night he ran around over there.’
		//
\endgl
\xe

The sentence in (\lastx) appears deceptively simple but includes some interesting complexities: a focused bare NP adverb, two motion derivations applied to one verb, one of which is morphologically ambiguous, and an irregular and very idiomatic verb.
The noun \fm{taat} ‘night’ could be mistaken for an argument since it appears to be a lone noun, but the meaning would then have to be ‘night ran over there’.
Instead this is a bare NP adverb which is a noun phrase used as an adverb without the addition of any postpositions.
Tlingit uses many bare NP adverbs and the sentence in (\lastx) shows that these can be focused just like other adverbs.

The PPs \fm{át} ‘to there’ and \fm{áa} ‘around there’ in (\lastx) are associated with motion derivations, specifically with \vbderiv{NP-\{t,x̱,dé\}}{∅}{\fm{-μ} repetitive}{arriving at NP} and \fm{(NP) áa}{n}{no repetitive}{around (near NP)}.
The first with \fm{át} is a \fm{∅}-conjugation motion derivation and predicting a short high tone stem \fm{xíx} [\ipa{xíx}] for the perfective aspect.
But the second with \fm{áa} is \fm{n}-conjugation predicting a long low tone stem \fm{xeex} [\ipa{xìːx}] for the perfective aspect.
\citeauthor{swanton:1909}’s transcription \orth{x̣īx̣} suggests a long vowel and thus the long low tone stem \fm{xeex}, and thus \fm{n}-conjugation.
This is unexpected because usually it is the leftmost (highest) bound PP that determines the final conjugation class in a stack of motion derivations.
Alternatively \fm{át} could reflect the \fm{n}-conjugation motion derivation \vbderiv{NP-t}{n}{\fm{yoo=i-…-k} repetitive}{around NP}, but it is unclear what this should contribute that is different from the \fm{áa} PP.

The verb \fm{wujixeex} ‘s/he ran’ in (\lastx) is frequently encountered basic vocabulary, but it is notably irregular and idiomatic.
The root \fm{\rt[¹]{xix}} usually refers to a single entity falling or otherwise moving more or less uncontrolled through space.
Examples of this include \fm{wooxeex} ‘it fell’, \fm{du lú yan uwaxíx} ‘his nose ran’ \parencite[38.354]{story-naish:1973}, \fm{hít seiwaxeex} ‘the house collapsed’ \parencite[51.560]{story-naish:1973}, and \fm{tléixʼ yakyee káa yaawaxeex} ‘it took place on Monday; lit.\ it fell on day one’ \parencite[223.3134]{story-naish:1973}.
When \fm{\rt[¹]{xix}} is combined with \fm{d-} and \fm{sh-} it has the unique idiomatic meaning of ‘sg.\ run’, i.e.\ for a singular entity (usually a human) to run: \fm{tláakw ḵux̱ wujixíx du aat x̱ánde} ‘she ran quickly back by her aunt’ \parencite[177.2444]{story-naish:1973}, \fm{nursg̱aa neesheex!} ‘run for a nurse!’ \parencite[88.1104]{story-naish:1973}, and \fm{anax̱ yéi x̱wjixíx} \parencite[f03/99]{leer:1973}.
The \fm{d-} is probably self-affecting middle voice indicating that the agent entity is identical to the affected entity and so is a kind of reflexive.
The \fm{sh-} was probably originally pejorative in meaning, describing e.g.\ children running around chaotically, but it no longer has a discernable pejorative implication.
Compare e.g.\ \fm{x̱wajikʼéin} ‘I jumped’ for a similar combination of \fm{d-} and \fm{sh-}.

\ex\label{ex:93-39-relatives-dont-know}%
\exmn{290.9}%
\begingl
	\glpreamble	Doxō′nq!itc ʟēł wu′sko. //
	\glpreamble	Du x̱oonxʼích tléil wuskú. //
	\gla	{} Du \rlap{x̱oonxʼích} @ {} @ {} @ {} {}
		tléil \rlap{wuskú.} @ {} @ {} @ {} @ {} @ {} //
	\glb	{} du x̱oon -xʼ -í -ch {}
		tléil ⱥ- u- wu- s- \rt[²]{kuʰ} -μH //
	\glc	{}[\pr{DP} \xx{3h·pss} relative -\xx{pl} -\xx{pss} -\xx{erg} {}]
		\xx{neg} \xx{arg}- \xx{irr}- \xx{pfv}- \xx{xtn}- \rt[²]{know} -\xx{var} //
	\gld	{} his \rlap{relatives} {} {} {} {}
		not \rlap{3>3.\xx{pfv}.know} {} {} {} {} {} //
	\glft	‘His relatives did not know about it.’
		//
\endgl
\xe

\ex\label{ex:93-40-dawned-they-ran-there}%
\exmn{290.9}%
\begingl
	\glpreamble	Ātx ỵaqē′ga ā′awe doxō′nq!î adē′ łiwag̣u′q. //
	\glpreamble	Átx̱ ÿaa ḵeiga.áa áwé du x̱oonxʼí aadé loowagooḵ. //
	\gla	{} \rlap{Átx̱} @ {} {}
		{} ÿaa @ \rlap{ḵeiga.áa} @ {} @ {} @ {} @ {} {}
		\rlap{áwé} @ {}
		{} du \rlap{x̱oonxʼí} @ {} @ {} {} +
		{} \rlap{aadé} @ {} {}
		\rlap{loowagooḵ.} @ {} @ {} @ {} @ {} //
	\glb	{} á -dáx̱ {}
		{} ÿaa= ḵei- g- \rt[¹]{.a} -μμH {} {}
		á -wé
		{} du x̱oon -xʼ -í {}
		{} á -dé {}
		lu- wu- i- \rt[¹]{guḵ} -μμL //
	\glc	{}[\pr{PP} \xx{3n} -\xx{abl} {}]
		{}[\pr{CP} along= dawn- \xx{gcnj}- \rt[¹]{extend} -\xx{var} \·\xx{sub} {}]
		\xx{foc} -\xx{mdst}
		{}[\pr{DP} \xx{3h·pss} relative -\xx{pl} -\xx{pss} {}]
		{}[\pr{PP} \xx{3n} -\xx{all} {}]
		nose- \xx{pfv}- \xx{stv}- \rt[¹]{push} -\xx{var} //
	\gld	{} that -after {}
		{} along \rlap{dawn.\xx{csec}.extend} {} {} {} -when {}
		\rlap{it.is} {}
		{} his \rlap{relatives} {} {} {}
		{} there -to {}
		\rlap{\xx{pfv}.run·\xx{pl}} {} {} {} {} //
	\glft	‘After that, it having dawned, his relatives ran there.’
		//
\endgl
\xe

The syntactic position of the PP \fm{átx̱} ‘after that’ in (\lastx) is ambiguous: as given it is part of the main clause, but it could alternatively be part of the adjunct (consecutive) clause with the verb phrase \fm{ÿaa ḵeiga.áa} ‘it having dawned’.
An alternative English translation with \fm{átx̱} in the main clause is ‘It having dawned, after that his relatives ran there’.
But if the \fm{átx̱} ‘after that’ were part of the adjunct clause then a close English approximation would be ‘It having dawned after that, his relatives ran there’.

\ex\label{ex:93-41-didnt-know-broke}%
\exmn{290.10}%
\begingl
	\glpreamble	ʟēł wudusko′ āwuʟ!ī′q!î. //
	\glpreamble	Tléil wuduskú awulʼéexʼi. //
	\gla	Tléil \rlap{wuduskú} @ {} @ {} @ {} @ {} @ {} @ {}
		{} \rlap{awulʼéexʼi.} @ {} @ {} @ {} @ {} {} //
	\glb	tléil u- wu- du- d- s- \rt[²]{ku} -μH
		{} a- wu- \rt[²]{lʼixʼ} -μμH -i {} //
	\glc	\xx{neg} \xx{irr}- \xx{pfv}- \xx{4h·s}- \xx{mid}- \xx{xtn}- \rt[²]{know} -\xx{var}
		{}[\pr{CP} \xx{arg}- \xx{pfv}- \rt[²]{break} -\xx{var} -\xx{sub} {}] //
	\gld	not \rlap{\xx{pfv}.people.know} {} {} {} {} {} {}
		{} \rlap{3>3.\xx{pfv}.break} {} {} {} -that {} //
	\glft	‘People did not know that he had broken it.’
		//
\endgl
\xe

\ex\label{ex:93-42-where-is-hard-time}%
\exmn{290.10}%
\begingl
	\glpreamble	Gūsu′ yên yuqᵒxê′tc gî. //
	\glpreamble	Goosú yan yoo ḵux̱échgi? //
	\gla	\rlap{Goosú} @ {} @ {} 
		{} yan @ yoo @ \rlap{ḵux̱échgi?\!»} @ {} @ {} @ {} @ {} {} //
	\glb	goo -sá -ú 
		{} ÿán= yoo= ḵu- \rt[²]{x̱ich} -μH -k -í {} //
	\glc	where -\xx{q} -\xx{locp}
		{}[\pr{DP} ground= \xx{alt}= \xx{4h·o}- \rt[²]{throw·anim} -\xx{var} -\xx{rep} -\xx{nmz} {}] //
	\gld	where -? -is.at
		{} ground \xx{alt} \rlap{people.\xx{impfv}.throw·anim} {} {} {} \·which {} //
	\glft	‘Where is that which gives people a hard time?’
		//
\endgl
\xe

The phrase that \citeauthor{swanton:1909} transcribes \orth{yên yuqᵒxê′tc gî} in (\lastx) is somewhat unusual.
His gloss “where there it was never broken off ?” is incorrect.
The \fm{goosú} part is indeed ‘where is it’.
The remainder of the sentence is a nominalization based on the root \fm{\rt[²]{x̱ich}} \~\ \fm{\rt[²]{x̱ech}} ‘throw animate/container’ which was discussed earlier in the context of (\ref{ex:93-10-dump-urine}).
The specific verb is an idiom \vblex{yan aawax̱ích}{∅}{achievement, \fm{yoo=i-…-k} repetitive}{s/he/it gave him/her a hard time, stumped him/her, got the best of him/her}, based on a metaphor of a person being thrown (perhaps repeatedly) to the ground by some other entity.\footnote{Following \textcite[803]{leer:1976} the root is interpreted here as ‘throw animate’, but it could instead be ‘beat, club, strike’.} This is illustrated by examples like \fm{yan x̱at uwax̱ích} ‘it’s got me stumped’, \fm{yan yoo x̱at yax̱íchk} ‘it gets the best of me’, and \fm{yan yoo iyax̱íchk} ‘it’s giving you a hard time’ \parencite[f02/59]{leer:1973}.
The same verb is also euphemistically used for rape as in \fm{yan wuduwax̱ích} ‘she was raped’ \parencite[f02/59]{leer:1973}, literally ‘someone/people threw him/her down’ \parencite[803]{leer:1976}.
This verb reflects the motion derivation \fm{ÿan=yoo=}{∅}{\fm{yoo=i-…-k} repetitive}{moving up and down (from surface)}, where the preverb \fm{ÿán=} retains its original meaning of ‘ground’ rather than the more common metaphoric meanings ‘ashore’ and ‘finishing, ending, terminating’.
The nominalization \fm{yan yoo ḵux̱échgi} ‘that which gives people a hard time’ is based on the \fm{yoo=i-…-k} repetitive imperfective \fm{yan yoo ḵux̱échk} ‘s/he/it repeatedly gets the best of people’ with the nominalization suffix \fm{-í} and concomitant suppression of the stative \fm{i-} prefix.

\ex\label{ex:93-43-first-one-broke-off}%
\exmn{290.11}%
\begingl
	\glpreamble	Dju cuq!oa′aỵī′tcawe ʟē āx wuʟ!ē′q!. //
	\glpreamble	Chʼu shuxʼaa aaÿích áwé tle aax̱ woolʼéexʼ. //
	\gla	Chʼu {} \rlap{shuxʼaa} @ {} \rlap{aaÿích} @ {} @ {} {}
		\rlap{áwé} @ {}
		tle {} \rlap{aax̱} @ {} {}
		\rlap{woolʼéexʼ.} @ {} @ {} @ {} @ {} //
	\glb	chʼu {} shú- xʼaa aa -í -ch {}
		á -wé
		tle {} á -dáx̱ {}
		ⱥ- wu- i- \rt[²]{lʼixʼ} -μμH //
	\glc	just {}[\pr{DP} end- tip \xx{part} -\xx{pss} -\xx{erg} {}]
		\xx{foc} -\xx{mdst}
		then
		{}[\pr{PP} \xx{3n} -\xx{abl} {}]
		\xx{arg}- \xx{pfv}- \xx{stv}- \rt[²]{break} -\xx{var} //
	\gld	just {} \rlap{first} {} one -of {} {}
		\rlap{it.is} {}
		then {} it -from {}
		\rlap{3>3.\xx{pfv}.break} {} {} {} {} //
	\glft	‘It was just the first one then that broke it off of there.’
		//
\endgl
\xe

\ex\label{ex:93-44-first-one-broke-off}%
\exmn{290.11}%
\begingl
	\glpreamble	Tc!uʟe′ uduwawū′s! //
	\glpreamble	Chʼu tle wduwawóosʼ //
	\gla	Chʼu tle \rlap{wduwawóosʼ} @ {} @ {} @ {} @ {} //
	\glb	chʼu tle wu- du- i- \rt[²]{wuͣsʼ} -μμH //
	\glc	just then \xx{pfv}- \xx{4h·s}- \xx{stv}- \rt[²]{ask} -\xx{var} //
	\gld	just then \rlap{\xx{pfv}.people.ask} {} {} {} {} {} //
	\glft	‘So then people asked’
		//
\endgl
\xe

\ex\label{ex:93-45-who-broke-trees-penis}%
\exmn{290.12}%
\begingl
	\glpreamble	“Adō′tsᴀ wuʟ!ī′q! ās-ʟ!ē′łî.” //
	\glpreamble	«\!Aadóoch sá woolʼéexʼ aas tlʼíli?\!» //
	\gla	{} \llap{«\!}\rlap{Aadóoch} @ {} {} sá 
		\rlap{woolʼéexʼ} @ {} @ {} @ {} @ {}
		{} aas \rlap{tlʼíli?\!»} @ {} {} //
	\glb	{} aadóo -ch {} sá 
		ⱥ- wu- i- \rt[²]{lʼixʼ} -μμH
		{} aas tlʼíl -í {} //
	\glc	{}[\pr{DP} who -\xx{erg} {}] \xx{q}
		\xx{arg}- \xx{pfv}- \xx{stv}- \rt[²]{break} -\xx{var}
		{}[\pr{DP} tree penis -\xx{pss} {}] //
	\gld	{} who {} {} ?
		\rlap{3>3.\xx{pfv}.break} {} {} {} {}
		{} tree penis -of {} //
	\glft	‘“Who is it that broke the tree’s penis?”.’
		//
\endgl
\xe

\ex\label{ex:93-46-people-said}%
\exmn{290.12}%
\begingl
	\glpreamble	Tc!uʟe′ yē ỵa′odūdzîqa, //
	\glpreamble	Chʼu tle yéi ÿawdudziḵaa //
	\gla	Chʼu tle yéi @ \rlap{ÿawdudziḵaa} @ {} @ {} @ {} @ {} @ {} @ {} @ {} //
	\glb	chʼu tle yéi= ÿ- wu- du- d- s- i- \rt[¹]{ḵa} -μμL //
	\glc	just then thus= \xx{qual}- \xx{pfv}- \xx{4h·s}- \xx{mid}- \xx{csv}- \xx{stv}- \rt[²]{say} -\xx{var} //
	\glft	‘So then people said’
		//
\endgl
\xe

\ex\label{ex:93-47-people-said}%
\exmn{290.13}%
\begingl
	\glpreamble	“Kāhā′s!îdjayu′ wuʟ!ī′q!.” //
	\gla	{} \llap{«\!}\rlap{Kaháasʼich} @ {} {}
		\rlap{áyú} @ {}
		\rlap{woolʼéexʼ.\!»} @ {} @ {} @ {} @ {} //
	\glb	{} Kaháasʼi -ch {}
		á -yú
		ⱥ- wu- i- \rt[²]{lʼixʼ} -μμH //
	\glc	{}[\pr{DP} \xx{name} -\xx{erg} {}]
		\xx{foc} -\xx{dist}
		\xx{arg}- \xx{pfv}- \xx{stv}- \rt[²]{break} -\xx{var} //
	\gld	{} \xx{name} {} {}
		\rlap{it.is} {}
		\rlap{3>3.\xx{pfv}.break} {} {} {} {} //
	\glft	‘“It is Kaháasʼi who broke it”.’
		//
\endgl
\xe

\ex\label{ex:93-48-laugh-uninteresting}%
\exmn{290.13}%
\begingl
	\glpreamble	Tca kaodō′wacūg̣ayu łkūx̣ᴀ′ʟ!gītc. //
	\glpreamble	Chʼa kawduwashooḵ áyú l koox̱áalgich. //
	\gla	Chʼa \rlap{kawduwashooḵ} @ {} @ {} @ {} @ {} @ {}
		\rlap{áyú} @ {} +
		{} {} l \rlap{koox̱áalgich.} @ {} @ {} @ {} @ {} @ {} {} {} {} //
	\glb	chʼa k- wu- du- i- \rt[²]{shuḵ} -μμL
		á -yú
		{} {} l k- u- \rt[²]{x̱al} -μμH -k -í {} -ch {} //
	\glc	just \xx{qual}- \xx{pfv}- \xx{4h·s}- \xx{stv}- \rt[²]{laugh} -\xx{var}
		\xx{foc} -\xx{dist}
		{}[\pr{PP} {}[\pr{CP} \xx{neg} \xx{qual}- \xx{irr}- \rt[²]{interesting} -\xx{var} -\xx{rep} -\xx{sub} {}]
			-\xx{erg} {}] //
	\gld	just \rlap{\xx{pfv}people.laugh·at} {} {} {} {} {}
		\rlap{it.is} {}
		{} {} not \rlap{\xx{impfv}.be.interesting} {} {} {} {} -that {} -because {} //
	\glft	‘People just laughed at him because he is not interesting.’
		//
\endgl
\xe

The verb in (\lastx) is apparently fairly rare, being absent in \cite{story-naish:1973} and documented only by \citeauthor{leer:1976} with the state imperfective form \fm{ash koox̱áal} “(s/he/it) is fascinating, intriguing to him (making him want to go see it)” and the perfective form \fm{ash kaawax̱aal} “(s/he/it) intrigues him” \parencite[789]{leer:1976}.
It may have originally also appeared in \citeauthor{leer:1973}’s stem collection \parencite{leer:1973} but some pages for “\fm{x̣a}” are missing in the archive copy.
Because this is so little documented the conjugation class cannot be determined, but the perfective stem \fm{–x̱aal} with a long vowel and low tone means that it is not \fm{∅}-conjugation.
The documented forms imply that the root is bivalent with the subject being the entity that triggers interest or fascination and the object being the experiencer thereof.
The form in (\lastx) occurs without the expected third person object prefix \fm{a-}; this could be a transcription error or it could reflect variation in the root’s valency.

\ex\label{ex:93-49-brought-strength}%
\exmn{290.14}%
\begingl
	\glpreamble	Adᴀ′xawe ʟē ỵaodu′dzîqōx łᴀtsī′n duỵîg̣a′. //
	\glpreamble	Aadáx̱ áwé tle ÿawdudziḵoox̱ latseen du ÿeeg̱áa. //
	\gla	{} \rlap{Aadáx̱} @ {} {}
		\rlap{áwé} @ {}
		tle \rlap{ÿawdudziḵoox̱} @ {} @ {} @ {} @ {} @ {} @ {} @ {}
		{} \rlap{latseen} @ {} @ {} @ {} {}
		{} du \rlap{ÿeeg̱áa.} @ {} {} //
	\glb	{} á -dáx̱ {}
		á -wé
		tle ÿ- wu- du- d- s- i- \rt[¹]{ḵux̱} -μμL
		{} l- \rt[¹]{tsin} -μμL {} {}
		{} du ee -g̱áa {} //
	\glc	{}[\pr{PP} \xx{3n} -\xx{abl} {}]
		\xx{foc} -\xx{mdst}
		then \xx{qual}- \xx{pfv}- \xx{4h·s}- \xx{mid}- \xx{csv}- \xx{stv}- \rt[¹]{go·boat} -\xx{var}
		{}[\pr{DP} \xx{xtn}- \rt[¹]{alive} -\xx{var} \·\xx{nmz} {}]
		{}[\pr{PP} \xx{3h} \xx{base} -\xx{ades} {}] //
	\gld	{} that -after {}
		\rlap{it.is} {}
		then \rlap{\xx{pfv}.someone.cause.go·boat} {} {} {} {} {} {} {}
		{} \rlap{strength} {} {} {} {}
		{} him {} -for {} //
	\glft	‘After that then it brought strength for him.’
		//
\endgl
\xe

\ex\label{ex:93-50-no-weapons}%
\exmn{290.14}%
\begingl
	\glpreamble	Gūsūyu′ ʟēł q!ān-cagū′n qā′djî ā′g̣a. //
	\glpreamble	Goosóoyú tléil xʼáan shagóon ḵaa jee aag̱áa //
	\gla	\rlap{Goosóoyú} @ {} @ {} @ {}
		tléil
		{} xʼáan shagóon {}
		{} ḵaa \rlap{jee} @ {} {}
		{} \rlap{aag̱áa.} @ {} {} //
	\glb	goo -sá -ú -yú
		tléil
		{} xʼáan shagóon {}
		{} ḵaa jee {} {}
		{} á -g̱áa {} //
	\glc	where -\xx{q} -\xx{locp} -\xx{dist}
		\xx{neg}
		{}[\pr{DP} anger element:\xx{inal} {}]
		{}[\pr{PP} \xx{4h·pss} poss’n -\xx{locp} {}]
		{}[\pr{PP} \xx{3n} -\xx{ades} {}] //
	\gld	where -some -is.at -it.was
		not
		{} war weapon {}
		{} people’s poss’n -in {}
		{} there -near {} //
	\glft	‘It was somewhere that people did not have weapons of war around there.’
		//
\endgl
\xe

\FIXME{Weird use of \fm{goosú}, \citeauthor{swanton:1909} interprets it as temporal rather than spatial.
Also collapsed with focus?}

\FIXME{Discuss \fm{xʼáan shagóon}.
Other examples?
Problems with translating \fm{shagóon}, point forward to (\ref{ex:93-51-bathe-for-shagoon}).}

\FIXME{Probably temporal \fm{aag̱áa}.}

\ex\label{ex:93-51-bathe-for-shagoon}%
\exmn{291.1}%
\begingl
	\glpreamble	 ᴀdjayu′ ᴀcagū′n kᴀ′q!ayu ducū′tc. //
	\glpreamble	Ách áyú, a shagóon káxʼ áyú dashóoch. //
	\gla	{} \rlap{Ách} @ {} {}
		\rlap{áyú,} @ {}
		{} a shagóon \rlap{káxʼ} @ {} {}
		\rlap{áyú} @ {}
		\rlap{dashóoch.} @ {} @ {} //
	\glb	{} á -ch {}
		á -yú
		{} a shagóon ká -xʼ {}
		á -yú
		d- \rt[²]{shuch} -μμH //
	\glc	{}[\pr{PP} \xx{3n} -\xx{erg} {}]
		\xx{foc} -\xx{dist}
		{}[\pr{PP} \xx{3n·pss} element:\xx{inal} \xx{hsfc} -\xx{loc} {}]
		\xx{foc} -\xx{dist}
		\xx{apsv}- \rt[²]{bathe} -\xx{var} //
	\gld	{} it -why {}
		\rlap{it.is} {}
		{} its power atop -at {}
		\rlap{it.is} {}
		\rlap{\xx{impfv}.bathe} {} {} //
	\glft	‘That is why, it is for power that they bathe.’
		//
\endgl
\xe

\FIXME{Explain translation of \fm{shagóon} here as ‘power’.}

\ex\label{ex:93-52-always-sit-in-water}%
\exmn{291.1}%
\begingl
	\glpreamble	hīn tā′g̣aqētc. //
	\glpreamble	Héen táa g̱aḵéiych. //
	\gla	{} Héen \rlap{táa} @ {} {}
		\rlap{g̱aḵéiych.} @ {} @ {} @ {} @ {} //
	\glb	{} héen táaᵏ \·∅ {}
		g̱- \rt[¹]{ḵi} -μμH -ÿ -ch //
	\glc	{}[\pr{PP} water bottom -\xx{loc} {}]
		\xx{g̱cnj}- \rt[¹]{sit·\xx{pl}} -\xx{var} -\xx{ÿsfx} -\xx{rep} //
	\gld	{} water bottom -on {}
		\rlap{\xx{hab}.sit·\xx{pl}} {} {} {} {} //
	\glft	‘They were always sitting in water.’
		//
\endgl
\xe

\ex\label{ex:93-53-nothing-to-kill-them}%
\exmn{291.2}%
\begingl
	\glpreamble	Yū′łîtsīne-ᴀt yū′tān ʟē′łᴀtc g̣adu′ʟ̣îdjāg̣e ᴀt qā′djî. //
	\glpreamble	Yú litseeni át, yú taan, tléil ách g̱wadudlijaag̱i át ḵaa jee. //
	\gla	{} Yú {} \rlap{litseeni} @ {} @ {} @ {} @ {} {} át, {}
		{} yú taan, {}
		tléil
		{} {} {} \rlap{ách} @ {} {}
				\rlap{g̱wadudlijaag̱i} @ {} @ {} @ {} @ {} @ {} @ {} @ {} @ {} @ {} {}
			át {}
		{} ḵaa \rlap{jee} @ {} {} //
	\glb	{} yú {} l- i- \rt[¹]{tsin} -μμL -i {} át {}
		{} yú taan {}
		tléil 
		{} {} {} á -ch {}
				u- {} g̱- du- d- l- i- \rt[²]{jaḵ} -μμL -i {}
			át {}
		{} ḵaa jee {} {} //
	\glc	{}[\pr{DP} \xx{dist} {}[\pr{CP} \xx{xtn}- \xx{stv}- \rt[¹]{alive} -\xx{var} -\xx{rel} {}] thing {}]
		{}[\pr{DP} \xx{dist} sealion {}]
		\xx{neg}
		{}[\pr{DP} {}[\pr{CP} {}[\pr{PP} \xx{3n} -\xx{instr} {}]
				\xx{irr}- \xx{zcnj}\· \xx{mod}- \xx{4h·s}- \xx{mid}- \xx{appl}- \xx{stv}- \rt[²]{kill}
					-\xx{var} -\xx{rel} {}]
			thing {}]
		{}[\pr{PP} \xx{4h·pss} poss’n -\xx{locp} {}] //
	\gld	{} that {} \rlap{\xx{impfv}.be.strong} {} {} {} -which {} thing {}
		{} those sealions {}
		not
		{} {} {} it -with {}
				\rlap{\xx{pot}.people.kill} {} {} {} {} {} {} {} {} -which {}
			thing {}
		{} people’s poss’n -is.in {} //
	\glft	‘A thing which is strong, those sealions,
		people did not have something which people could kill them with.’
		//
\endgl
\xe

\ex\label{ex:93-54-big-man-said}%
\exmn{291.2}%
\begingl
	\glpreamble	Wananī′sawe yułīngî′t ʟēnī′tc ye hᴀs ỵa′osiqa, “Hu ts!u.” //
	\glpreamble	Wáa nanée sáwé yú leengít tleiních yéi has ÿawsiḵaa «\!Hú tsú\!». //
	\gla	{} Wáa \rlap{nanée} @ {} @ {} @ {} {}
		\rlap{sáwé} @ {} @ {}
		{} yú leengít \rlap{tleiních} @ {} {}
		yéi @ has @ \rlap{ÿawsiḵaa} @ {} @ {} @ {} @ {} @ {}
		{} \llap{«\!}Hú tsú\!». {} //
	\glb	{} wáa n- \rt[¹]{ni} -μμH {} {} 
		s= á -wé
		{} yú leengít tlein -ch {}
		yéi= has= ÿ- wu- s- i- \rt[¹]{ḵa} -μμL
		{} hú tsú {} //
	\glc	{}[\pr{CP} how \xx{ncnj}- \rt[¹]{happen} -\xx{var} \·\xx{sub} {}]
		\xx{q}= \xx{foc} -\xx{mdst}
		{}[\pr{DP} \xx{dist} person big -\xx{erg} {}]
		thus= \xx{plh}= \xx{qual}- \xx{pfv}- \xx{csv}- \xx{stv}- \rt[¹]{say} -\xx{var}
		{}[\pr{CP} \xx{3h} also {}] //
	\gld	{} how \rlap{\xx{csec}.happen} {} {} \·while {}
		some\· \rlap{it.is} {}
		{} that man big {} {}
		thus them \rlap{\xx{pfv}.say·to} {} {} {} {} {} 
		{} him too {} //
	\glft	‘At some point the big man said to them “Him too”.’
		//
\endgl
\xe

\ex\label{ex:93-55-take-him-ashore}%
\exmn{291.3}%
\begingl
	\glpreamble	Yē ỵa′odudziqa, “Yên nᴀx duxā′.” //
	\glpreamble	Yéi ÿawdudziḵaa «\!Yan nax̱dux̱aa\!». //
	\gla	Yéi @ \rlap{ÿawdudziḵaa} @ {} @ {} @ {} @ {} @ {} @ {} @ {}
		{} \llap{«\!}Yan \rlap{nax̱dux̱aa.\!»} @ {} @ {} @ {} @ {} {} //
	\glb	yéi= ÿ- wu- du- d- s- i- \rt[¹]{ḵa} -μμL
		{} ÿán= n- g̱- du- \rt[²]{x̱a} -μμL {} //
	\glc	thus= \xx{qual}- \xx{pfv}- \xx{4h·s}- \xx{mid}- \xx{csv}- \xx{stv}- \rt[¹]{say} -\xx{var}
		{}[\pr{CP} \xx{term}= \xx{ncnj}- \xx{mod}- \xx{4h·s}- \rt[²]{paddle} -\xx{var} {}] //
	\gld	thus \rlap{\xx{pfv}.one.say·to} {} {} {} {} {} {} {}
		{} ashore \rlap{\xx{hort}.people.paddle} {} {} {} {} {} //
	\glft	‘He said to them “People should take him ashore”.’
		//
\endgl
\xe

\FIXME{Discuss switch from third person in (\ref{ex:93-54-big-man-said}) to fourth person in (\ref{ex:93-55-take-him-ashore}) along with the disappearance of \fm{has=}.}

\ex\label{ex:93-56-no-thing-with}%
\exmn{291.4}%
\begingl
	\glpreamble	ʟē′łᴀtc āx duʟ̣îdja′g̣e ᴀt qō′ostî′ yū′tān. //
	\glpreamble	Tléil ách aa x̱dudlijaag̱i át ḵoostí yú taan. //
	\gla	Tléil
		{} {} {} \rlap{ách} @ {} {}
				aa @ \rlap{x̱dudlijaag̱i} @ {} @ {} @ {} @ {} @ {} @ {} @ {} @ {} @ {} {}
			át {}
		\rlap{ḵoostí} @ {} @ {} @ {} @ {}
		{} yú taan. {} //
	\glb	tléil
		{} {} {} á -ch {}
				aa= u- {} g̱- du- d- l- i- \rt[²]{jaḵ} -μμL -i {}
			át {}
		ḵu- u- s- \rt[¹]{tiʰ} -μH
		{} yú taan {} //
	\glc	tléil
		{}[\pr{DP} {}[\pr{CP} {}[\pr{PP} \xx{3n} -\xx{erg} {}]
				\xx{part·o}= \xx{irr}- \xx{zcnj}\· \xx{mod}- \xx{4h·s}- \xx{mid}- \xx{appl}- \xx{stv}-
					\rt[²]{kill} -\xx{var} -\xx{rel} {}]
			thing {}]
		\xx{areal}- \xx{irr}- \xx{appl}- \rt[¹]{be} -\xx{var}
		{}[\pr{DP} \xx{dist} sealion {}] //
	\gld	not
		{} {} {} it -with {}
				some \rlap{\xx{pot}.people.kill} {} {} {} {} {} {} {} {} -which {}
			thing {}
		\rlap{\xx{stv}·\xx{impfv}.exist} {} {} {} {}
		{} those sealions {} //
	\glft	‘No thing which people could kill some of them with existed, those sealions.’
		//
\endgl
\xe

\ex\label{ex:93-57-tell-them-go-with-him}%
\exmn{291.5}%
\begingl
	\glpreamble	Duī′n ᴀt wudū′waxūn yē ỵa′odudzîqa. //
	\glpreamble	Du een át wuduwaxoon yéi ÿawdudziḵaa. //
	\gla	{} {} Du \rlap{een} @ {} {}
			{} \rlap{át} @ {} {}
			\rlap{wuduwaxoon} @ {} @ {} @ {} @ {} @ {} {} +
		yéi @ \rlap{ÿawdudziḵaa.} @ {} @ {} @ {} @ {} @ {} @ {} @ {} //
	\glb	{} {} du ee -n {}
			{} á -t {}
			wu- du- i- \rt[¹]{xun} -μμL {} {}
		yéi= ÿ- wu- du- d- s- i- \rt[¹]{ḵa} -μμL //
	\glc	{}[\pr{CP} {}[\pr{PP} \xx{3h} \xx{base} -\xx{instr} {}]
			{}[\pr{PP} \xx{3n} -\xx{pnct} {}]
			\xx{pfv}- \xx{4h·s}- \xx{stv}- \rt[¹]{drift} -\xx{var} \·\xx{sub} {}]
		thus= \xx{qual}- \xx{pfv}- \xx{4h·s}- \xx{mid}- \xx{csv}- \xx{stv}- \rt[¹]{say} -\xx{var} //
	\gld	{} {} him {} -with {}
			{} there -around {}
			\rlap{\xx{pfv}.people.creep} {} {} {} {} {} {}
		thus \rlap{\xx{pfv}.one.say·to} {} {} {} {} {} {} {} //
	\glft	‘He told them to creep along around it with him.’
		//
\endgl
\xe

\citeauthor{swanton:1909} glosses the \fm{yéi} ‘thus’ in (\lastx) as “as precedes”.
This peculiar interpretation makes no sense in this context and is a strange error.
It might however be explained by \citeauthor{swanton:1909}’s process of working with his consultants.
If he asked “What does \orth{yē} mean?” without elaborating on the verb and its context in (\lastx) then \fm{Deikeenaakʼw} might have offered a translation like “that way” or “like so”, or perhaps “like what was just said before it”.
\citeauthor{swanton:1909}’s gloss “as precedes” would thus dutifully reflect the consultant’s actual explanation given an inappropriate lack of context for interpretation.

\FIXME{Discuss \fm{\rt{xun}} meanings.
It’s used for ‘prepare’ in Ḵaadashaan’s Aakʼwtaatseen and is used for ‘show face’ in Jiwduwanág̱i Ḵáa.
Where is ‘drift’ from?
And is it monovalent or bivalent?
Looks monovalent here.
Also logjam?}

\ex\label{ex:93-58-right-then-he-said}%
\exmn{291.5}%
\begingl
	\glpreamble	Tcatc!aāg̣a′tsa yē ỵa′odudzîqa, //
	\glpreamble	Cha chʼa aag̱áa tsá yéi ÿawdudziḵaa //
	\gla	Cha chʼa {} \rlap{aag̱áa} @ {} {} tsá
		yéi @ \rlap{ÿawdudziḵaa} @ {} @ {} @ {} @ {} @ {} @ {} @ {} //
	\glb	cha chʼa {} á -g̱áa {} tsá
		yéi ÿ- wu- du- d- s- i- \rt[¹]{ḵa} -μμL //
	\glc	\xx{interj} just {}[\pr{PP} \xx{3n} -\xx{ades} {}] just
		thus= \xx{qual}- \xx{pfv}- \xx{4h·s}- \xx{mid}- \xx{csv}- \xx{stv}- \rt[¹]{say} -\xx{var} //
	\gld	well just {} that -after {} just
		thus \rlap{\xx{pfv}.one.say·to} {} {} {} {} {} {} {} //
	\glft	‘Then just right after that he said to them’
		//
\endgl
\xe

\ex\label{ex:93-59-take-him-ashore}%
\exmn{291.6}%
\begingl
	\glpreamble	“Yên nax dōxā′, Kāhā′s!î.” //
	\glpreamble	«\!Yan nax̱dux̱aa, Kaháasʼi.\!» //
	\gla	«\!Yan @ \rlap{nax̱dux̱aa} @ {} @ {} @ {} @ {}
		{} Kaháasʼi. {} //
	\glb	\pqp{}ÿán= n- g̱- du- \rt[²]{x̱a} -μμL
		{} Kaháasʼi {} //
	\glc	\pqp{}\xx{term}= \xx{ncnj}- \xx{mod}- \xx{4h·s}- \rt[²]{paddle} -\xx{var}
		{}[\pr{DP} \xx{name} {}] //
	\gld	\pqp{}ashore \rlap{\xx{hort}.people.paddle} {} {} {} {}
		{} Kaháasʼi {} //
	\glft	‘“People should take him ashore, Kaháasʼi.”’
		//
\endgl
\xe

\ex\label{ex:93-60-named-him}%
\exmn{291.6}%
\begingl
	\glpreamble	Ag̣ā′tsa duỵa′odowasa, Kāhā′s!î. //
	\glpreamble	Aag̱áa tsá du ÿáa wduwasáa ‹\!Kaháasʼi\!›. //
	\gla	{} \rlap{Aag̱áa} @ {} {} tsá
		{} du \rlap{ÿáa} @ {} {}
		\rlap{wduwasáa} @ {} @ {} @ {} @ {}
		{} \llap{‹\!}Kaháasʼi\!›. {} //
	\glb	{} á -g̱áa {} tsá
		{} du ÿá -μ {}
		wu- du- i- \rt[²]{sa} -μμH
		{} Kaháasʼi {} //
	\glc	{}[\pr{PP} \xx{3n} -\xx{ades} {}] just
		{}[\pr{PP} \xx{3h·pss} face -\xx{loc} {}]
		\xx{pfv}- \xx{4h·s}- \xx{stv}- \rt[²]{name} -\xx{var}
		{}[\pr{DP} \xx{name} {}] //
	\gld	{} that -after {} just
		{} his face -on {}
		\rlap{\xx{pfv}.people.name} {} {} {} {}
		{} Kaháasʼi {} //
	\glft	‘Just after that people named him Kaháasʼi.’
		//
\endgl
\xe

\ex\label{ex:93-61-paddled-him-around}%
\exmn{291.7}%
\begingl
	\glpreamble	Adā′t ỵaodowaxā′ yū′tān q!ā′t!î. //
	\glpreamble	A daat ÿawduwax̱áa yú taan xʼáatʼi. //
	\gla	{} A \rlap{daat} @ {} {}
		\rlap{ÿawduwax̱aa} @ {} @ {} @ {} @ {} @ {}
		{} yú taan \rlap{xʼáatʼi.} @ {} {} //
	\glb	{} a daa -t {}
		ÿ- wu- du- i- \rt[²]{x̱a} -μμL
		{} yú taan xʼáatʼ -í {} //
	\glc	{}[\pr{PP} \xx{3n·pss} around -\xx{pnct} {}]
		\xx{qual}- \xx{pfv}- \xx{4h·s}- \xx{stv}- \rt[²]{paddle} -\xx{var}
		{}[\pr{DP} \xx{dist} sealion island -\xx{pss} {}] //
	\gld	{} its around -around {}
		\rlap{\xx{pfv}.people.paddled} {} {} {} {} {}
		{} that sealion island -of {} //
	\glft	‘They paddled him around it, that sealion island.’
		//
\endgl
\xe

\ex\label{ex:93-62-snatch-up-two}%
\exmn{291.7}%
\begingl
	\glpreamble	Tc!uʟe′ dē′xawe acā′waʟēq yū′tān. //
	\glpreamble	Chʼu tle déix̱ áwé ashaawatleiḵw yú taan. //
	\gla	Chʼu tle {} déix̱ {}
		\rlap{áwé} @ {}
		\rlap{ashaawatleiḵw} @ {} @ {} @ {} @ {} @ {}
		{} yú taan. {} //
	\glb	chʼu tle {} déix̱ {}
		á -wé
		a- sha- wu- i- \rt[²]{tleḵw} -μμL
		{} yú taan {} //
	\glc	just then {}[\pr{DP} two {}]
		\xx{foc} -\xx{mdst}
		\xx{arg}- head- \xx{pfv}- \xx{stv}- \rt[²]{pick} -\xx{var}
		{}[\pr{DP} \xx{dist} sealion {}] //
	\gld	just then {} two {}
		\rlap{it.is} {}
		\rlap{3>3.\xx{pfv}.grab·\xx{pl}} {} {} {} {} {}
		{} those sealions {} //
	\glft	‘Just then it was two that he snatched up, those sealions.’
		//
\endgl
\xe

\FIXME{Discuss verb \fm{ashaawatleiḵw} based on \fm{\rt[²]{tleḵw}} as plural counterpart of \fm{aawasháat} versus association with gluttony and berries.}

\ex\label{ex:93-63-left-one-on-rock}%
\exmn{291.8}%
\begingl
	\glpreamble	Yūs!ᴀt!nᴀ′xa ta ỵaqā′c kᴀt. //
	\glpreamble	Yú sʼatʼnax̱.aa, taÿaḵʼáatlʼ kát. //
	\gla	{} Yú \rlap{sʼatʼnax̱.aa,} @ {} @ {} {} 
		{} \rlap{taÿaḵʼáatlʼ} @ {} @ {} \rlap{kát.} @ {} {} //
	\glb	{} yú sʼátʼ- náx̱= aa {}
		{} té- ÿá- ḵʼáatlʼ ká -t {} //
	\glc	{}[\pr{DP} \xx{dist} left- side= \xx{part} {}]
		{}[\pr{PP} rock- face- flat \xx{hsfc} -\xx{pnct} {}] //
	\gld	{} that \rlap{left} {} one {}
		{} rock- face- flat atop -to {} //
	\glft	‘The left one (he threw) on top of a flat rock.’
		//
\endgl
\xe

The phrase that \citeauthor{swanton:1909} transcribes as \orth{ta ỵaqā′c kᴀt} in (\lastx) is difficult to interpret.
His gloss is “he threw upon a flat rock”, but there is no verb to be identified in his transcription.
The final \orth{kᴀt} probably reflects the common postposition phrase \fm{ká-t} ‘on top of (it), against (it)’.
The lack of a verb could be explained by ellipsis, perhaps accompanied by a gesture from the narrator.
The \orth{ta} part is probably \fm{té} ‘stone, rock’ or its compounding form \fm{ta-}, matching \citeauthor{swanton:1909}’s gloss “rock”.
The remaining \orth{ỵaqā′c} is the most difficult part to identify.
It naively would be \fm{ÿaḵáash} or \fm{ÿaḵaash}, of which the former could be identified as \fm{ÿá} ‘face’ + \fm{ḵáash} ‘hip’ but the latter would be nonsense.
The interpretation ‘hip face’ would give a translation ‘that left one against a rock hip face’ which is bizarre.
There is a compound noun \fm{té ḵáasʼ} or \fm{taḵáasʼ} ‘rock crevice, fissure in rock’ with \fm{ḵáasʼ} ‘stick, splinter’ which suggests \fm{taÿaḵáasʼ} ‘crack in the rock face’ that could be plausible in this context.
But \citeauthor{swanton:1909}’s gloss “a flat rock” suggests instead that \orth{qā′c} could be a mistranscription of \fm{ḵʼáatlʼ} ‘flat object’.
This noun is found in a number of compounds like \fm{sʼíxʼ ḵʼáatlʼ} ‘plate, platter’ with \fm{sʼíxʼ} ‘dish’, \fm{lʼée ḵʼáatlʼ} ‘felt’ with \fm{lʼée} ‘blanket’, \fm{lítaa ḵʼáatlʼ} ‘table/butter knife’ with \fm{lítaa} ‘knife’, and \fm{luḵʼáatlʼ} ‘northern shoveler’ (\species{Spatula}{clypeata}[L.]) with \fm{lú} ‘nose’.
The noun \fm{ḵʼáachʼ} ‘red seaweed’ (\species{Cryptopleura}{ruprechtiana}[(J.Agardh) Kylin 1924]) may be etymologically connected to \fm{ḵʼáatlʼ} given that red seaweed grows with broad flat leaves.

\ex\label{ex:93-64-right-one-tore-apart}%
\exmn{291.8}%
\begingl
	\glpreamble	Cinaxā′a qo′a wū′cdᴀx āwas!ē′ʟ!. //
	\glpreamble	Sheeynax̱.aa ḵu.aa wóoshdáx̱ aawasʼéilʼ. //
	\gla	{} \rlap{Sheeynax̱.aa} @ {} @ {} {}
		ḵu.aa
		{} \rlap{wóoshdáx̱} @ {} {}
		\rlap{aawasʼéil.} @ {} @ {} @ {} @ {} //
	\glb	{} sheeÿ- náx̱ =aa {}
		ḵu.aa
		{} wóosh =dáx̱ {}
		a- wu- i- \rt[²]{sʼelʼ} -μμH //
	\glc	{}[\pr{DP} right- side= \xx{part} {}]
		\xx{contr}
		{}[\pr{PP} \xx{recip} =\xx{abl} {}]
		\xx{arg}- \xx{pfv}- \xx{stv}- \rt[²]{tear} -\xx{var} //
	\gld	{} \rlap{right} {} one {}
		however
		{} ea·oth =from {}
		\rlap{3>3.\xx{pfv}.tear} {} {} {} {} //
	\glft	‘The right one however he tore apart.’
		//
\endgl
\xe

\ex\label{ex:93-65-he-was-of-poverty-that-man}%
\exmn{291.9}%
\begingl
	\glpreamble	Q!ᴀnᴀckîdē′x wusîte′ yuqā′ //
	\glpreamble	Ḵʼanashgidéix̱ wusitee yú ḵáa;  //
	\gla	{} \rlap{Ḵʼanashgidéix̱} @ {} @ {} @ {} @ {} @ {} @ {} {}
		\rlap{wusitee} @ {} @ {} @ {} @ {}
		{} yú ḵáa; {} //
	\glb	{} ḵʼe- n- sh- \rt{git} -i =yé -x̱ {}
		wu- s- i- \rt[¹]{tiʰ} -μμL
		{} yú ḵáa {} //
	\glc	{}[\pr{PP} mouth- \xx{ncnj}- \xx{pej}- \rt{\xx{unkn}} -\xx{rel} =way -\xx{perl} {}]
		\xx{pfv}- \xx{appl}- \xx{stv}- \rt[¹]{be} -\xx{var}
		{}[\pr{DP} \xx{dist} man {}] //
	\gld	{} \rlap{poor} {} {} {} {} {} -of {}
		\rlap{\xx{pfv}.\xx{appl}.be} {} {} {} {}
		{} that man {} //
	\glft	‘He was of poverty, that man;’
		//
\endgl
\xe

\ex\label{ex:93-66-why-strength-helped-him}%
\exmn{291.9}%
\begingl
	\glpreamble	aca′ yūłdakᴀ′t-ᴀt yē′de łatsī′n duig̣a′ wūsū′. //
	\glpreamble	ách áyú ldakát át yéide Latseen du eeg̱áa woosoo. //
	\gla	{} \rlap{ách} @ {} {}
		\rlap{áyú} @ {}
		{} ldakát át {}
		{} \rlap{yéide} @ {} {}
		{} \rlap{Latseen} @ {} @ {} @ {} {} +
		{} du \rlap{eeg̱áa} @ {} {}
		\rlap{woosoo.} @ {} @ {} @ {} //
	\glb	{} á -ch {}	
		á -yú
		{} ldakát át {}
		{} yé -dé {}
		{} l- \rt[¹]{tsin} -μμL {} {}
		{} du ee -g̱áa {}
		wu- i- \rt[¹]{suʰ} -μμL //
	\glc	{}[\pr{PP} \xx{3n} -\xx{erg} {}]
		\xx{foc} -\xx{dist}
		{}[\pr{DP} all thing {}]
		{}[\pr{PP} thus -\xx{all} {}]
		{}[\pr{DP} \xx{xtn}- \rt[¹]{alive} -\xx{var} \·\xx{nmz} {}]
		{}[\pr{PP} \xx{3h} \xx{base} -\xx{ades} {}]
		\xx{pfv}- \xx{stv}- \rt[¹]{sup·help} -\xx{var} //
	\gld	{} it -why {}
		\rlap{it.is} {}
		{} all thing {}
		{} that -way {}
		{} \rlap{strength} {} {} {} {}
		{} him {} -for {}
		\rlap{\xx{pfv}.super·help} {} {} {} //
	\glft	‘that is why with everything that way Strength gave him supernatural help.’
		//
\endgl
\xe

\ex\label{ex:93-67-dont-beat-him-now}%
\exmn{291.10}%
\begingl
	\glpreamble	ʟēł de yuỵaoduʟ̣′ᴀqᴀk doxōnq!î. //
	\glpreamble	Tléil de yoo ÿawdudlág̱wákw du x̱oonxʼí. //
	\gla	Tléil de 
		yoo @ \rlap{ÿawdudlág̱wákw} @ {} @ {} @ {} @ {} @ {} @ {} @ {}
		{} du \rlap{x̱oonxʼí.} @ {} @ {} {} //
	\glb	tléil de
		yoo= ÿ- wu- du- d- l- \rt[²]{dlaḵw} -μH -k
		{} du x̱oon -xʼ -í {} //
	\glc	\xx{neg} now
		\xx{alt}= \xx{qual}- \xx{pfv}- \xx{4h·s}- \xx{mid}- \xx{xtn}- \rt[²]{win} -\xx{var} -\xx{rep}
		{}[\pr{DP} \xx{3h·pss} relative -\xx{pl} -\xx{pss} {}] //
	\gld	not now
		\xx{alt} \rlap{\xx{pfv}.they.beat.\xx{rep}} {} {} {} {} {} {} {}
		{} his relative -s {} {} //
	\glft	‘They did not beat him now, his relatives.’
		//
\endgl
\xe

\ex\label{ex:93-68-not-clothed-in-war}%
\exmn{291.10}%
\begingl
	\glpreamble	ʟēł naᴀ′t nayē′duo′xq!un adawū′ʟyᴀq!. //
	\glpreamble	Tléil naa.át náa yei du.úx̱xʼún adawóotl yáxʼ. //
	\gla	Tléil {} \rlap{naa.át} @ {} {}
		{} \rlap{náa} {}
		yei @ \rlap{du.úx̱xʼún} @ {} @ {} @ {} @ {} @ {} +
		{} \rlap{adawóotl} @ {} @ {} @ {} \rlap{yáxʼ.} @ {} {} //
	\glb	tléil {} náa- át {}
		{} náa {}
		yei= du- \rt[²]{.u} -μH -x̱ -xʼ -ín
		{} a- d- \rt[²]{wutl} -μμH ÿá -xʼ {} //
	\glc	\xx{neg} {}[\pr{DP} covering- thing {}]
		{}[\pr{NP} covering {}]
		down= \xx{4h·s}- \rt[²]{wear} -\xx{var}- \xx{rep} -\xx{pl} -\xx{past}
		{}[\pr{PP} \xx{xpl}- \xx{mid}- \rt[²]{trouble} -\xx{var} face -\xx{loc} {}] //
	\gld	not {} \rlap{clothing} {} {}
		{} draped {}
		down \rlap{\xx{impfv}.wear.\xx{rep}.\xx{pl}.\xx{past}} {} {} {} {} {}
		{} \rlap{war} {} {} {} face -at {} //
	\glft	‘He did not wear clothing in the face of war.’
		//
\endgl
\xe

\ex\label{ex:93-69-strength-in-his-possession}%
\exmn{291.11}%
\begingl
	\glpreamble	Yīwuyā′t! ag̣ā′ ᴀcdjiyē′ wu′tiỵiỵᴀ dułatsī′ne. //
	\glpreamble	Yeewooyáatʼ aag̱áa ash jée yéi wooteeÿi ÿé, du latseení. //
	\gla	\rlap{Yeewooyáatʼ} @ {} @ {} @ {} @ {}
		{} \rlap{aag̱áa} @ {} {}
		{} {} {} ash \rlap{jée} @ {} {} +
			yéi @ \rlap{wooteeÿi} @ {} @ {} @ {} @ {} {} ÿé, {}
		{} du \rlap{latseení.} @ {} @ {} @ {} @ {} {} //
	\glb	ÿee= wu- i- \rt[¹]{ÿatʼ} -μμH
		{} á -g̱áa {}
		{} {} {} ash jee -H {}
			yéi= wu- i- \rt[¹]{tiʰ} -μμL -i {} ÿé {}
		{} du l- \rt[¹]{tsin} -μμL {} -í {} //
	\glc	time= \xx{pfv}- \xx{stv}- \rt[¹]{long} -\xx{var}
		{}[\pr{PP} \xx{3n} -\xx{ades} {}]
		{}[\pr{DP} {}[\pr{CP} {}[\pr{PP} \xx{3prx·pss} poss’n -\xx{loc} {}]
			thus= \xx{pfv}- \xx{stv}- \rt[¹]{be} -\xx{var} -\xx{rel} {}] way {}]
		{}[\pr{DP} \xx{3h·pss} \xx{xtn}- \rt[¹]{alive} -\xx{var} \·\xx{nmz} -\xx{pss} {}] //
	\gld	\rlap{time.\xx{pfv}.be.long} {} {} {} {}
		{} that -after {}
		{} {} {} his poss’n -in {}
			thus \rlap{\xx{pfv}.be} {} {} {} -that {} way {}
		{} his \rlap{strength} {} {} {} {} {} //
	\glft	‘It was a long time after that the way that it was in his possession, his strength.’
		//
\endgl
\xe

\ex\label{ex:93-70-even-now-people-know-it}%
\exmn{291.12}%
\begingl
	\glpreamble	Tc!uya′ỵidᴀt ts!u wudu′dziku. //
	\glpreamble	Chʼu yá ÿeedát tsú wududzikóo. //
	\gla	Chʼu {} yá ÿeedát {} tsú
		\rlap{wududzikóo} @ {} @ {} @ {} @ {} @ {} @ {} //
	\glb	chʼu {} yá ÿeedát {} tsú
		wu- du- d- s- i- \rt[²]{ku} -μμH //
	\glc	just {}[\pr{DP} \xx{prox} moment {}] also
		\xx{pfv}- \xx{4h·s}- \xx{xtn}- \xx{stv}- \rt[²]{know} -\xx{var} //
	\gld	just {} this moment {} also
		\rlap{\xx{pfv}.people.know} {} {} {} {} {} {} //
	\glft	‘Even now people also know it.’
		//
\endgl
\xe

\ex\label{ex:93-71-make-his-strength-things}%
\exmn{291.12}%
\begingl
	\glpreamble	Dułᴀtsī′nī ᴀtx dułiᴀ′xnutc. //
	\glpreamble	Du latseení átx̱ dulyéx̱ nooch. //
	\gla	{} Du \rlap{latseení} @ {} @ {} @ {} @ {} {}
		{} \rlap{átx̱} @ {} {}
		\rlap{dulyéx̱} @ {} @ {} @ {} @ \•nooch //
	\glb	{} du l- \rt[¹]{tsin} -μμL {} -í {}
		{} át -x̱ {}
		du- l- \rt[²]{yex̱} -μH =nooch //
	\glc	{}[\pr{DP} \xx{3h·pss} \xx{xtn}- \rt[¹]{alive} -\xx{var} \·\xx{nmz} -\xx{pss} {}]
		{} thing -\xx{pert} {}
		\xx{4h·s}- \xx{xtn}- \rt[²]{build} -\xx{var} =\xx{hab·aux} //
	\gld	{} his \rlap{strength} {} {} {} {} {}
		{} thing -of {}
		\rlap{\xx{impfv}.people.build} {} {} {} \•always //
	\glft	‘People always build his strength into things.’
		//
\endgl
\xe

\ex\label{ex:93-72-make-his-strength-things}%
\exmn{291.13}%
\begingl
	\glpreamble	Ātc qoỵadudłdjᴀ′tckunutc. //
	\glpreamble	Átch ḵuÿaduljéchkw nooch. //
	\gla	{} \rlap{Átch} @ {} {}
		\rlap{ḵuÿaduljéchkw} @ {} @ {} @ {} @ {} @ {} @ {} @ {} @ \•nooch. //
	\glb	{} át -ch {}
		ḵu- ÿ- du- d- l- \rt[¹]{jech} -μH -kw =nooch //
	\glc	{}[\pr{DP} thing -\xx{instr} {}]
		\xx{4h·o}- \xx{qual}- \xx{4h·s}- \xx{mid}- \xx{appl}- \rt[¹]{surprise} -\xx{var} -\xx{rep} =\xx{hab·aux} //
	\gld	{} thing -with {}
		\rlap{people.\xx{impfv}.people.\xx{appl}.make.surprised} {} {} {} {} {} {} {} \•always //
	\glft	‘People always surprise people with this thing.’
		//
\endgl
\xe

\ex\label{ex:93-73-imitate-his-strength}%
\exmn{291.13}%
\begingl
	\glpreamble	Dutī′nutc dułᴀtsī′ne. //
	\glpreamble	Dutee nooch du latseení. //
	\gla	\rlap{Dutee} @ {} @ {} @ {} @ \•nooch
		{} du \rlap{latseení.} @ {} @ {} @ {} @ {} {} //
	\glb	u- du- \rt[²]{tiʰ} -μμL =nooch
		{} du l- \rt[¹]{tsin} -μμL {} -í {}  //
	\glc	\xx{irr}- \xx{4h·s}- \rt[²]{imitate} -\xx{var} =\xx{hab·aux}
			{}[\pr{DP} \xx{3h·pss} \xx{xtn}- \rt[¹]{alive} -\xx{var} \·\xx{nmz} -\xx{pss} {}] //
	\gld	\rlap{\xx{stv}.\xx{impfv}.ppl.imitate} {} {} {} \•always
		{} his \rlap{strength} {} {} {} {} {} //
	\glft	‘People always imitate his strength.’
		//
\endgl
\xe

\ex\label{ex:93-74-thats-it}%
\exmn{291.14}%
\begingl
	\glpreamble	Hū′tc!awe. //
	\glpreamble	Hóochʼ áwé. //
	\gla	Hóochʼ \rlap{áwé.} @ {} //
	\glb	hóochʼ á -wé //
	\glc	finished \xx{cpl} -\xx{mdst} //
	\gld	finished \rlap{it.is} {} //
	\glft	‘It is finished.’
		//
\endgl
\xe
