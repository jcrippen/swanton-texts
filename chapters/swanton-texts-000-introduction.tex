%!TEX root = ../swanton-texts.tex
%%
%% Introduction.
%%

\resetexcnt
\chapter{Introduction}\label{ch:introduction}

This book is a collection of narratives, oratory, and songs that were originally recorded by John R.\ Swanton from Tlingit speakers in \fm{Sheetʼká} Sitka and \fm{Ḵaachx̱an.áakʼw} Wrangell, Alaska from January through April 1904 \parencite[1]{swanton:1909}.
They have been converted and corrected from \citeauthor{swanton:1909}’s unreliable transcription into the modern orthography to make them more accessible to Tlingit learners and researchers.
The Tlingit texts are accompanied by new translations into English that closely reflect the structure and meaning of the original Tlingit forms.
Each sentence of every text is provided with a detailed linguistic segmentation and gloss informed by current research and documentation of the Tlingit language, and most sentences also include commentary on their linguistic structure and interpretation.

The texts in this book were originally published as part of \citeauthor{swanton:1909}’s \textit{Tlingit Myths and Texts}, a book issued in 1909 as Bulletin 39 of the Bureau of American Ethnology of the Smithsonian Institution \parencite{swanton:1909}.
The majority of that volume consists of narratives recorded in English (pp.\ 3–251).
Judging by their vocabulary and style, these English narratives are probably elaborations from outlines that \citeauthor{swanton:1909} took down from his consultants during his fieldwork.
He says that some of his consultants spoke no English so much of his English material must have been obtained either through English interpreters or in Chinook Jargon from the consultants themselves.
All this means that \citeauthor{swanton:1909}’s English narratives have been subject to extensive modification and are relatively distant from what his consultants understood about their traditions.

The Tlingit language narratives, oratory, and songs are about 40 percent of \citeauthor{swanton:1909}’s \textit{Tlingit Myths and Texts} (pp.\ 252–415).
These were recorded directly from Tlingit speakers and represent their actual speech.
This means that these texts have been much less modified by their recording and transmission and so more closely reflect the language and intellectual culture of Tlingit people in 1904.
\citeauthor{swanton:1909}’s English translations of these stories are somewhat less removed from their origin than the texts he recorded in English, but they also show signs of being modified and elaborated.
Some of this modification is apparently editorial to fit \citeauthor{swanton:1909}’s ideas of adequate English prose, but some of his changes are misleading or outright wrong.
These errors reflect his very limited understanding of the Tlingit language: he only worked with Tlingit speakers for four months.

\citeauthor{swanton:1909}’s Tlingit transcriptions are the most detailed and faithful renditions of individual Tlingit people’s language and thought at the turn of the 20th century.
Yet they are at the same time the least accessible of \citeauthor{swanton:1909}’s materials.
As discussed below, \citeauthor{swanton:1909}’s transcriptions are far removed from modern Tlingit publications in their linguistic accuracy and his transcription system is opaque to modern linguists.
Although \citeauthor{swanton:1909}’s transcriptions are an invaluable window through which we can view late 19th century Tlingit life, the view is still clouded and hazy.
This book atttempts to wash the window clean, providing us the best available record of Tlingit language and culture at the time directly from the mouths of Tlingit people themselves.

During the four months \citeauthor{swanton:1909} worked with Tlingit speakers, he developed only a partial comprehension of the phoneme inventory of Tlingit.
\citeauthor{swanton:1909}’s transcription was inaccurate: he confused uvular and velar consonants, ejective and non-ejective consonants, and low and high vowels.
He also completely failed to detect contrastive lexical tone, but he did indicate stress somewhat reliably.\footnote{\citeauthor{swanton:1909}’s record of stress actually puts him ahead of some other linguists who have studied Tlingit.
From \textcite{boas:1917} onward stress has only been occasionally mentioned and very rarely indicated in transcription.} His phonetic and grammatical description of the Tlingit language was published as a chapter in \citeauthor{boas:1911}’s \textit{Handbook of American Indian languages} \parencite{swanton:1911}.
It is a retrospective analysis based on his flawed transcriptions and inaccurate translations although it is slightly more consistent than his original data.
The problems with \citeauthor{swanton:1909}’s analysis were noted early on; \citeauthor{boas:1913a} regretted his publication of \citeauthor{swanton:1909}’s Tlingit description in a letter to Edward Sapir about Sapir’s Na-Dene research:

\begin{quote}\small
When Swanton sent me his Tlingit sketch, I hesitated very much to accept it, and have always very seriously doubted the interpretation of many of the elements.
Of course, I could not go behind his examples at the time; and my conclusion, on the whole, was that the identification of most of them was correct, but that the interpretation was forced right through.
I find now that not only the interpretations are untenable from beginning to end, but that also a greeat many things have been brought together that do not belong together.
I wish now very much that the grammar were out of the book…
\sourceatright{\parencite{boas:1913a}}
\end{quote}

\citeauthor{boas:1917} would go on to work with \fm{Stoowuḵáa} Louis Shotridge in 1915.
Together they established the first accurate phoneme inventory of Tlingit and detailed many basic grammatical properties that have formed the foundation for all future research on the language.
But \citeauthor{boas:1917} and later researchers never revisited \citeauthor{swanton:1909}’s transcriptions even though most linguists have referred to these texts more or less explicitly.
For example, in his \textit{Grammatical notes on the language of the Tlingit Indians} \parencite{boas:1917}, \citeauthor{boas:1917} regularly cited forms in \citeauthor{swanton:1909}’s text collection but he never revised the transcription or analysis.
There have been a few attempts to convert some of \citeauthor{swanton:1909}’s texts into a more legible form \parencites{dauenhauer:1971b}{leer:1977}{littlefield-makinen:2003}, but none have covered the whole inventory of his materials and none have attempted any detailed linguistic analysis.
This book is thus the first comprehensive retranscription and reanalysis of \citeauthor{swanton:1909}’s Tlingit language texts.

In \citeauthor{swanton:1909}’s defence, some of the mistranscriptions in \citeauthor{swanton:1909}’s publication may not be his fault, instead being introduced during the complex process of transmission from his original manuscript notes to the published book.
In the 19th and early 20th centuries, letterpress printing was the province of expert tradesmen.
There were two printing processes at the time: cold metal type with precast metal type slugs for individual letters and hot metal type with mechanized on-demand casting of metal type slugs in prearranged blocks.
In both processes an expert arranges metal type by hand into in a frame that is then inked and pressed onto paper.
The image on each type slug is necessarily backward with respect to the image on paper and is usually also upside down so that the printed paper can be flipped off the press; printers thus read the metal type upside down and backwards.
A handwritten ‘fair copy’ would be read by the composer for typesetting, often passing through a few intermediary workers in the hot metal process.
Large documents might be divided up into sections with separate teams typesetting them in parallel, and each person involved would have different levels of care and accuracy in their work.
After the initial typesetting was done a set of unbound ‘proof sheets’ would be sent to the author for correction.
Later these corrected proofs would be incorporated into the final printing for bound volumes.
Each stage of the typesetting process was an opportunity for errors, and the more unusual the text the more likely that errors would occur.
\citeauthor{swanton:1909} had relatively illegible handwriting and his materials were exceedingly unusual even in comparison with other academic publications at the time, so it is entirely plausible that many of the errors in the final publication were caused by typesetters and not \citeauthor{swanton:1909} himself.

One example of the problem in typesetting and subsequently interpreting \citeauthor{swanton:1909}’s materials is his extensive use of the ‘dot below’ diacritic.
The symbols \orth{x} and \orth{y} are differentiated from \orth{x̣} and \orth{ỵ} by the absence or presence of a small dot underneath the symbol.
Such dots are easily mistaken for dirt or ink blotches, called ‘flyspecks’ in the printing trade.
These could be introduced by ink spatter while writing his notes in the field, errors in reading his notes for the fair copy manuscript, mistakes in setting the metal type from his manuscript, problems while inking the metal type, dirt on or in the paper, and mishandling of the printed material before and after binding.
Although \citeauthor{swanton:1909} might have either misheard or mistranscribed an incorrect diacritic in his original notes, it is often equally likely that the mistaken diacritic could have been introduced or lost in preparation of the fair copy or anywhere along the chain of printing tasks.

If we had access to \citeauthor{swanton:1909}’s original manuscript notes from his time in Sitka and Wrangell then we could conceivably correct the published form to maximally approximate his original transcriptions.
Unfortunately, although some of \citeauthor{swanton:1909}’s Tlingit notes are available at the Smithsonian, the bulk of his fieldwork on Tlingit has been lost due to still unknown circumstances.
\FIXME{Review where his notes are and the lack of our target materials.} \FIXME{Corrected against multiple original copies so we know that all introduced errors must precede the bookbinding process.}

\FIXME{Community engagement with \citeauthor{swanton:1909}’s materials.
References from De Laguna, McClellan, and Kan about the use of them.
Note Paul’s annotated copy deposited in ANLA.}

\FIXME{This book does not include any exegesis of the stories and cultural contextualization is minimal.
Two reasons: (i) the book would be at least double in size and (ii) exegetical study must come after linguistic analysis.}

\section{About the speakers}\label{sec:intro-speakers}

\FIXME{Some biographical details about each speaker.
Summarize info from \cite{jones:2017}.}

\FIXME{Transcribed consultants: Don Cameron of the Chilkat Kaagwaantaan (101, 102, 103), Ḵʼalyaan of the Sitka Kiks.ádi (96, 97), Ḵʼadasteen of Yakutat (105), Deikeenaakʼw of the Sitka Ḵookhittaan Kaagwaantaan (89–93, 95, 98, 99, 104), Ḵaadashaan of the Wrangell Ḵaasx̱ʼagweidí (100, 106, all the speeches).}

\FIXME{Other consultants recorded only in English: Ḵaadishaan’s mother Léekʼ, Kasáankʼ of Kake.}

\FIXME{Sources of songs are unclear.}

\subsection{Deikeenaakʼw}\label{sec:intro-speakers-deikeenaakw}

\subsection{Ḵaadashaan}\label{sec:intro-speakers-katishan}

\textcite{jones:2017} says that \citeauthor{swanton:1909} left from Sitka on 20 March 1904 and arrived at Wrangell – \fm{Ḵaachx̱an.áakʼw} – on 22 March \parencite[112 ff.]{jones:2017}.
He soon met Ḵaadashaan John Kadashan and \citeauthor{jones:2017} supposes that most of \citeauthor{swanton:1909}’s work was done at Ḵaadashaan’s house.
Ḵaadashaan was the leader of the Kaasx̱ʼagweidí clan \parencite[]{swanton:1908}, a Raven moiety clan local to Wrangell.

This is the same “Kadachan” that \textcite{muir:1915} and \textcite{young:1915} travelled with some twenty-five years earlier \parencite[160–165, 276 n.\ 36]{cruikshank:2005}. 

\section{About Swanton}\label{sec:intro-swanton}

\FIXME{Some biographical details about Swanton.
Summarize info from \cites{jones:2017}{fenton:1959}{steward:1960}.
Discuss the time period that he worked on Tlingit and connect to his Haida work.
Point to sources on his life.}

\section{Format of chapters}\label{sec:intro-format}

There are three kinds of chapters in this book aside from this introduction: narrative, oratory, and song lyrics.
The majority of the chapters are narratives, and these are presented first in the same order as in \citeauthor{swanton:1909}’s original presentation.
After the narratives there are two oratory chapters and \FIXME{number} song chapters, again following \citeauthor{swanton:1909}’s ordering.

Each narrative chapter opens with a background discussion for the story.
The first paragraph details the context of the narrative in \citeauthor{swanton:1909}’s collection as well as the title given by \citeauthor{swanton:1909}, the title given here, and other widely known titles of the same story.
The second paragraph lists other versions of the same story in the Tlingit ethnological literature, either in Tlingit or in English, and occasionally notes similar stories from neighbouring peoples like the Haida, Coast Tsimshian, Tahltan, Southern Tutchone, and Ahtna.
Additional introductory paragraphs discuss issues in the contextualization and interpretation specific to the narrative.

The narratives are presented in parallel columns of Tlingit and English.
Each Tlingit paragraph contains numbered sentences (or parts of long sentences) with corresponding translations in the English column.
The same sentence numbering is used in the linguistic analysis.
Sentence numbers are continuous throughout the story, ignoring paragraph and page divisions.
Following the running text are two sections of material directly from \citeauthor{swanton:1909}: his abstract of the story if any and then his English translation of the story.
\citeauthor{swanton:1909}’s English translation is accompanied by sentence numbering to match the sentences of the original Tlingit as well as possible, but because his translation practice is imprecise there may be gaps or reordering.

The body of each narrative chapter is a series of sections corresponding to the paragraphs in the running text.
Each numbered sentence is presented with a detailed morphological segmentation, grammatical analysis, and gloss; see section \ref{sec:intro-seg-gloss} for discussion.
Many sentences are followed by a detailed discussion of grammatical phenomena that occur within the sentence.
Some of these are essentially translator’s notes, discussing problems of interpretation with \citeauthor{swanton:1909}’s original transcription and problems of mismatch between Tlingit and English grammar.
Other discussions highlight interesting or unusual grammatical phenomena that are particularly well illustrated by the specific sentence.
A few discussions include etymological interpretation of vocabulary that appears in the sentence, focusing on names and on complex nouns whose meaning has drifted from their etymological origins.
Some sentences have no accompanying discussion; this implies that they are not especially significant in their grammar, discourse, or vocabulary, and thus that they can be considered more or less ‘ordinary’ utterances in Tlingit.

\FIXME{Oratory chapters}

\FIXME{Song chapters}

\section{Segmentation and glossing}\label{sec:intro-seg-gloss}

All of the texts are divided into numbered units which correspond roughly to sentences in the original transcriptions.
Some numbered units are smaller than sentences; these sub-sentential units are either (i) quoted speech, (ii) individual clauses from a long sequence of clauses in a single sentence, or (iii) other sub-parts of a very long sentence which has been divided for easier analysis.
Each numbered unit is presented with a number in parentheses and a page and line number reference in the margin.
The page and line number reference has the format “(\fm{xxx}.\fm{yy})” where \fm{xxx} is the page number in the original publication and \fm{yy} is the line number at the start of the unit.

The linguistic analysis of numbered units in each text is composed of seven lines as illustrated in (\ref{ex:intro-seg-gloss-stone-axe}).
The first line of the unit is always the original form of \citeauthor{swanton:1909}’s transcription, including all of his punctuation, capitalization, and diacritics.
The intent is to allow interested readers to easily verify the retranscriptions without needing to refer to the original publication.
All of these original forms have been carefully double-checked against multiple printed copies of \cite{swanton:1909}.

\ex\label{ex:intro-seg-gloss-stone-axe}%
\exmn{369.11}%
\begingl
	\glpreamble\exrtcmt{original transcription – line 1}%
		Yū′taỵīs q!ᴀłitsī′n. //
	\glpreamble\exrtcmt{retranscription – line 2}%
		Yú taÿees x̱ʼalitseen. //
	\gla\exrtcmt{word segmentation – line 3}%
		{} Yú \rlap{taÿees} @ {} {} \rlap{x̱ʼalitseen.} @ {} @ {} @ {} @ {} //
	\glb\exrtcmt{morpheme segmentation – line 4}%
		{} yú té- ÿees {} x̱ʼe- l- i- \rt[¹]{tsin} -μ //
	\glc\exrtcmt{\normalsize morpheme gloss – line 5}%
		{}[\pr{DP} \xx{dist} stone- wedge {}] mouth- \xx{xtn}- \xx{stv}- \rt[¹]{animate} -\xx{var} //
	\gld\exrtcmt{\normalsize word gloss – line 6}%
		{} those \rlap{stone·axe} {} {} \rlap{mouth.\xx{stv}·\xx{impfv}.strong} {} {} {} {} //
	\glft\exrtcmt{translation – line 7}%
		‘Those stone axes are valuable.’
		//
\endgl
\xe

The second line in each numbered unit is the retranscription as given in the parallel text and translation.
Compare \citeauthor{swanton:1909}’s \orth{Yū′taỵīs q!ᴀłitsī′n.} on the first line of (\ref{ex:intro-seg-gloss-stone-axe}) with the retranscription \fm{Yú taÿees x̱ʼalitseen.} on the second line of (\ref{ex:intro-seg-gloss-stone-axe}).
Most of the time the correspondence between these two lines is very close and there is no discussion of their differences.
Occasionally the retranscription features additional material not present \citeauthor{swanton:1909}’s transcription, and even more rarely the reverse where the retranscrpition lacks material present in \citeauthor{swanton:1909}’s transcription.
Additions and deletions from \citeauthor{swanton:1909}’s original are generally accompanied by some explicit discussion except where the same material is added many times throughout a text, in which case only the first instance of the addition is discussed.

The third, fourth, fifth, and sixth lines in each numbered unit present the detailed linguistic analysis.
These lines together form the analysis section of the numbered unit.
The third line contains the same material as the second line but spaced out into orthographic words.
The fourth, fifth, and sixth lines are then vertically aligned with the third line so that the segmented and glossed material forms columns with the orthographic forms.
The whole analysis section is set off from the second and seventh lines by a small amount of vertical space as can be seen by the line labels in (\ref{ex:intro-seg-gloss-stone-axe}).
When the material in the analysis section runs longer than the width of the page, the whole section is broken off at the end and continued in another group below, separated by some vertical space.
This is illustrated in (\ref{ex:intro-seg-gloss-jumper-stood-up}) where the analysis section of lines 3–6 is broken into two chunks, one chunk for \fm{Aadáx̱ áyú ḵáju kei wutáani ásgéyú} and a second chunk for \fm{yaakwnáx̱ wudihaan}.

\ex\label{ex:intro-seg-gloss-jumper-stood-up}%
\exmn{315.6}%
\begingl
	\glpreamble\exrtcmt{1}%
		Adᴀ′xayu qᴀdu′ ke wutā′nî ᴀsgē′yu yākᵘ nᴀx wudîhā′n. //
	\glpreamble\exrtcmt{2}%
		Aadáx̱ áyú ḵáju kei wutáani ásgéyú yaakwnáx̱ wudihaan. //
	\gla\exrtcmt{3}%
		{} \rlap{Aadáx̱} @ {} {} \rlap{áyú} @ {} 
		ḵáju
		{} kei @ \rlap{wutáani} @ {} @ {} @ {} {}
		\rlap{ásgéyú} @ {} @ {} @ {} +
		\exrtcmt{3}%
		{} \rlap{yaakwnáx̱} @ {} {}
		\rlap{wudihaan} @ {} @ {} @ {} @ {} //
	\glb\exrtcmt{4}%
		{} á -dáx̱ {} á -yú
		ḵáju
		{} kei= wu- \rt[¹]{taʼn} -μH -í {}
		á -sí -gí -yú
		\exrtcmt{4}%
		{} yaakw -náx̱ {}
		wu- d- i- \rt[¹]{han} -μ //
	\glc\exrtcmt{\normalsize 5}%
		{}[\pr{PP} \xx{3n} -\xx{abl} {}] \xx{foc} -\xx{dist} 
		actually
		{}[\pr{DP} up= \xx{pfv}- \rt[¹]{fish·jump} -\xx{var} -\xx{nmz} {}]
		\xx{foc} -\xx{dub} -\xx{yn} -\xx{dist}
		\exrtcmt{\normalsize 5}%
		{}[\pr{PP} boat -\xx{perl} {}]
		\xx{pfv}- \xx{mid}- \xx{stv}- \rt[¹]{stand·\xx{sg}} -\xx{var} //
	\gld\exrtcmt{\normalsize 6}%
		{} that -after {} \rlap{it.is} {}
		actually
		{} up \rlap{jumping·fish} {} {} {} {}
		it.is \·\rlap{maybe} {} {}
		\exrtcmt{\normalsize 6}%
		{} boat -along {}
		\rlap{\xx{pfv}.stand·\xx{sg}} {} {} {} {} //
	\glft\exrtcmt{7}%
		‘Then actually it’s a jumper who apparently stood up alongside the boat.’
		//
\endgl
\xe

The fourth and fifth lines present the detailed segmentation and gloss of the Tlingit forms.
The fourth line gives the segmented elements in their conventional underlying forms.
Elements that are known to be deleted in the surface form – e.g.\ the \fm{d-} prefix of the classifier – are represented explicitly despite their surface absence.
Null morphology which never has a surface realization – e.g.\ the non-past tense contrasting with past tense \fm{-ín} – is generally not represented in the segmentation.
In contrast, some null morphology is explicitly represented by a gloss with a blank segment.
This is particularly done for (i) second person singular subjects of imperatives, (ii) \fm{∅}-conjugation prefixes where the selection of \fm{∅} is lexically or derivationally specified (e.g.\ imperatives, consecutives), (iii) null complementizers in embedded (not main) clauses, and (iv) irregular null locative postpositions.
The lack of aspect marking for imperfectives is not indicated, nor is the lack of voice or valency classifier prefixes.

\FIXME{Examples of deleted \fm{d-}, second singular vs.\ second plural imperative, \fm{∅}-conjugation, null complementizer in subordinate clause, null complementizer in relative clause, irregular null locative postposition.}

The fifth line includes labelled bracketing for phrase structures.
The level of bracketing detail varies depending on the analytical complexity and the discussion of the form.
Minimally every top-level determiner phrase (DP) and postposition phrase (PP) is bracketed in a given sentence.
Complementizer phrases (CP) forming embedded clauses (relative clauses, complement clauses, adjunct clauses) are also always bracketed.
Deeper levels of bracketing – e.g.\ noun phrases (NP) within DPs or DPs within PPs − may be included if the structural complexity is significant, if there is syntactic or semantic ambiguity, if the form represents an unusual pattern, or if the extra detail is useful for the analytical discussion.
The top level CP of each sentence is usually not represented, but CPs of quoted speech are usually indicated.

The sixth line is a coarser gloss with larger divisions corresponding mostly to the orthographic word divisions in the third line.
The sixth line is thus a of word-by-word gloss whereas the fifth line is a morpheme-by-morpheme gloss.
This coarse gloss uses more English-like representations and less linguistic abbreviations so it is more approachable for language learners.
Glossing of whole verbs includes abbreviations that represent the aspect and tense categories which can be difficult to recover from the segmented morphemes without expert knowledge of the grammar.
The verb in (\ref{ex:intro-seg-gloss-unhappy}) for example is an irrealis form of a state imperfective with a habitual auxiliary.
The first orthographic word of this verb is glossed ‘mind.\xx{stv}·\xx{impfv}.good’ and the second orthographic word – the habitual auxiliary – is glossed as ‘always’.

\ex\label{ex:intro-seg-gloss-unhappy}%
\exmn{369.9}%
\begingl
	\glpreamble	ʟᴀkᵘ ʟēł tūcqē′nutc. //
	\glpreamble	tlákw tléil tooshkʼéi nuch. //
	\gla	tlákw tléil \rlap{tooshkʼéi} @ {} @ {} @ {} @ {} @ \•nuch. //
	\glb	tlákw tléil tu- u- sh- \rt[¹]{kʼe} -μH =nuch //
	\glc	always \xx{neg} mind- \xx{irr}- \xx{pej}- \rt[¹]{good} -\xx{var} =\xx{hab·aux} //
	\gld	always not \rlap{mind.\xx{stv}·\xx{impfv}.good} {} {} {} {} \•always //
	\glft	‘he is always unhappy.’
		//
\endgl
\xe

The seventh line in each numbered unit is the English translation.
This translation is exactly the same as the translation given in the running text aside from some minor formatting differences.
There is one significant exception: untranslated names and other words are given in italics in the parallel section but are given in upright (roman) form in the seventh line of the linguistic analysis.
This difference is because italics in the analysis could be naively misinterpreted as focus or other kinds of intonation in English, and also because the untranslated status of these words is relatively obvious in this context.

\citeauthor{swanton:1909}’s transcriptions occasionally contain a vowel which would normally be absent or lack a vowel which would normally be present in the speech of modern speakers.
These unexpectedly present or absent vowels are not errors, instead reflecting phonological processes that are more or less absent in today’s varieties of Tlingit.
They shed light on how the phonology of the language has changed over the last century and are thus reflected in the retranscription.
Unexpectedly present vowels are indicated in the retranscription with a haček below the vowel as shown by (\ref{ex:intro-seg-gloss-knew-fathers-house}).
Here \fm{Awu̬sikóo} represents the form \citeauthor{swanton:1909} transcribed as \orth{Awusikū′} implying \fm{Awusikóo} [\ipa{ʔà.wù.sì.ˈkʰʷúː}] but which today would be \fm{Awsikóo} [\ipa{ʔàw.sì.ˈkʰʷúː}].

\ex\label{ex:intro-seg-gloss-knew-fathers-house}%
\exmn{255.7}%
\begingl
	\glpreamble	Aw\gm{u}sikū′ duī′c hî′tî. //
	\glpreamble	Aw\gm{u̬}sikóo du éesh hídi. //
	\gla	\rlap{Aw\gm{u̬}sikóo} @ {} @ {} @ {} @ {} @ {}
		{} du éesh \rlap{hídi.} @ {} {} //
	\glb	a- wu- s- i- \rt[²]{kuʰ} -μH
		{} du éesh hít -í {} //
	\glc	\xx{arg}- \xx{pfv}- \xx{xtn}- \xx{stv}- \rt[²]{know} -\xx{var}
		{}[\pr{DP} \xx{3h·pss} father house -\xx{pss} {}] //
	\gld	\rlap{3>3.\xx{pfv}.know} {} {} {} {} {}
		{} her father house -of {} //
	\glft	‘She knew her father’s house.’
		//
\endgl
\xe

Unexpectedly absent vowels are indicated in the retranscription with a ring below the vowel as shown by (\ref{ex:intro-seg-gloss-thus-said-inside}).
Here \fm{yóo x̱ʼḁÿaḵá} represents the form \citeauthor{swanton:1909} transcribed as \orth{yū′q!ỵaqᴀ} implying \fm{yóo x̱ʼÿaḵá} [\ipa{júːχʼ.ɰà.ˈkʰá}] but which today would be \fm{yóo x̱ʼaÿaḵá} [\ipa{júː.χʼà.ɰà.ˈkʰá}].
The haček below \orth{◌̬} and the ring below \orth{◌̥} are the IPA diacritics for voicing and devoicing, respectively \parencites{international-phonetic-association:1999}{international-phonetic-association:2018}.

\ex\label{ex:intro-seg-gloss-thus-said-inside}%
\exmn{317.3}%
\begingl
	\glpreamble	yū′q!ỵaqᴀ yuyē′k dutū′q!. //
	\glpreamble	yóo x̱ʼ\gm{ḁ}ÿaḵá yú yéik du tóoxʼ. //
	\gla	yóo @ \rlap{x̱ʼ\gm{ḁ}ÿaḵá} @ {} @ {} @ {}
		{} yú yéik {}
		{} du \rlap{tóoxʼ.} @ {} {} //
	\glb	yóo= x̱ʼe- ÿ- \rt[¹]{ḵa} -H
		{} yú yéik {}
		{} du tú -xʼ {} //
	\glc	thus= mouth- \xx{qual}- \rt[¹]{say} -\xx{var}
		{}[\pr{DP} \xx{d}·\xx{dist} spirit {}]
		{}[\pr{PP} \xx{3h}·\xx{pss} inside -\xx{loc} {}] //
	\gld	thus \rlap{\xx{impfv}.say} {} {} {}
		{} that spirit {}
		{} his inside -at {} //
	\glft	‘thus it speaks, that spirit inside him.’
		//
\endgl
\xe

\FIXME{Discuss uvular lowering.
Move discussion here from Ḵaadishaan texts.}

\FIXME{Discuss annotation of roots: \fm{\rt{niͤʰ}} versus \fm{\rt{neͥʰ}} ‘occur’; \fm{\rt{nikw}} versus \fm{\rt{nuͥk}} ‘feel’; \fm{\rt{g̱ixʼ}} versus \fm{\rt{g̱eͥxʼ}} ‘throw inanimate’.
The \fm{eͥ} indicates lexicalized of /\ipa{i}/ to /\ipa{e}/.
The \fm{uͥ} indicates lexicalized labialization of /\ipa{i}/ to /\ipa{u}/.
The \fm{iͤ} indicates a /\ipa{i}/ which could unpredictably be lowered to /\ipa{e}/ depending on dialect (i.e.\ not uvular lowering).} 

\FIXME{Discuss line breaking within long examples.
Usually arbitrary, but sometimes reflecting structure.}

\ex\label{ex:intro-seg-gloss-place-where-you-cooked}%
\exmn{314.4}%
\begingl
	\glpreamble	“Isîkū′ gî yū′a ᴀt gaỵîsī′ỵiỵê? //
	\glpreamble	«\!Isikóo gí yú áa at gaÿi̬dzi.éeÿi ÿé? //
	\gla	«\!\rlap{Isikóo} @ {} @ {} @ {} @ {} gí +
		{} yú {} {} \rlap{áa} @ {} {} 
			at @ \rlap{gaÿi̬dzi.éeÿi} @ {} @ {} @ {} @ {} @ {} @ {} @ {} @ {} {} ÿé? {} //
	\glb	\pqp{}i- s- i- \rt[²]{kuʰ} -μH gí
		{} yú {} {} á -μ {} 
			at= g- ÿu- i- d- s- i- \rt[¹]{.i} -μμH -i {} yé {} //
	\glc	\pqp{}\xx{2sg·s}- \xx{xtn}- \xx{stv}- \rt[²]{know} -\xx{var} \xx{yn}
		{}[\pr{DP} \xx{dist} {}[\pr{CP} {}[\pr{PP} \xx{3n} -\xx{loc} {}]
			\xx{4n·o}= \xx{sben}- \xx{pfv}- \xx{2sg·o}- \xx{mid}- \xx{csv}- \xx{stv}- 
				\rt[¹]{cook} -\xx{var} -\xx{rel} {}]
		place {}] //
	\gld	\pqp{}\rlap{\xx{zcnj}.\xx{impfv}.you·\xx{sg}.know} {} {} {} {} ?
		{} that {} {} it -at {}
			thing \rlap{for·self.\xx{zcnj}.\xx{pfv}.you·\xx{sg}.make.cooked} {} {} {} {} {} {} {} -where {} place {} //
	\glft	‘“Do you know that place where you cooked things for yourself?’
		//
\endgl
\xe
