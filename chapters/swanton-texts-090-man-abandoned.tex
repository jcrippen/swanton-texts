%!TEX root = ../swanton-texts.tex
%%
%% 90. Man abandoned (262-266)
%%

\resetexcnt
\chapter{Jiwduwanág̱i Ḵáa: Man Abandoned}\label{ch:90-man-abandoned}

This narrative was told to \citeauthor{swanton:1909} by Deikeenaakʼw in Sitka in 1904.
In the original publication it is number 90, running from page 262 to 266 and totalling 65 lines of glossed transcription.
\citeauthor{swanton:1909}’s original title is “The Man Who Was Abandoned”.
The title given here in Tlingit is \fm{Jiwduwanág̱i Ḵáa} ‘Man Whom People Abandoned’ which is adapted from \citeauthor{swanton:1909}’s title.
The English equivalent used here is a small clause for brevity.

The same story was recorded in English as “Abandoned Boy” from \fm{Yéil Kʼidáa} Jimmy Scotty James (\fm{G̱aanax̱teidí}) by Catharine McClellan in 1950 in Carcross \parencite[384–390]{mcclellan-cruikshank:2007b}.
Though James’s version shares much of the same plot, his version is much richer and differs from this version by Deikeenaakʼw in many details.
No similar stories are found in \textcite{veniaminov:1846}, \textcite{judson:1911}, \textcite{jones:1914}, \textcites{velten:1939}{velten:1944}, \textcite{salisbury:1962}, \textcite{olson:1967}, \textcite{de-laguna:1972}, \textcite{dauenhauer:1987}, or \textcite{emmons:1991}.
\FIXME{check Kamenskii, Krause}

Above the title of this story in his copy \citeauthor{swanton:1909}’s book, \citeauthor{paul:1930} writes “Very much like the Tsimshean story – an eagle fed the lazy boy \&\ his grandmother” \parencite[262]{paul:1930}.
\citeauthor{paul:1930} is apparently referring here to the Nisg̱aʼa story of “Little-eagle” which was told by Moses to Boas in 1894 \parencite[169–187]{boas:1902}; \citeauthor{paul:1930} is known to have had a copy of this book.
This Nisg̱aʼa narrative does indeed appear to be the same story though many details are different.
Another similar story not mentioned by \citeauthor{paul:1930} is the Coast Tsimshian narrative “The prince who was deserted” recounted by Henry Tate \parencite[225–232]{boas-tate:1916}; this also has a relatively similar plot but again shows many differences in the details.
Both of the Nisg̱aʼa and Coast Tsimshian versions are much closer to the version told by Jimmy Scotty James (see above); this connection is reinforced by James’s statement that his story is about \fm{Tsʼootsxán} ‘Tsimshian’ people.\footnote{\textcite[384]{mcclellan-cruikshank:2007b} misinterpret what they transcribe as \orth{djutsqwan} as \orth{cqʼAt ʻqwan} but that is \fm{Shxʼat Ḵwáan} (cf.\ \orth{Cq!ᴀt qoan} in \cite[396]{swanton:1908}) which is an alternative name for the \fm{Shtaxʼhéen Ḵwáan} ‘Stikine People’ who are the Tlingit ḵwáan resident around Wrangell.} \FIXME{Note the similar “\fm{Tsʼᴇnslâ′ek·} (Abandoned one)” recorded by \citeauthor{boas:1912a} in \cite[589–594]{boas:2002}.}

This narrative, along with \citeauthor{swanton:1909}’s numbers 89 and 91 were transliterated into the modern Tlingit orthography by Jeff Leer in an unpublished typescript filed in the Alaska Native Language Archive as \textsc{\MakeLowercase{TL962L1977}}a \parencite{leer:1977}.
\citeauthor{leer:1977}’s analysis informs this interpretation of the text in the version presented here, but this version departs from his  in many places due to the greatly improved lexical and grammatical documentation available today.
Dialect differences like uvular lowering are also maintained in this version.

\clearpage
\begin{pairs}
\begin{Leftside}
\beginnumbering
\pstart
\snum{1}Ḵuwa.óo áyú aantḵeiní.
\snum{2}Aan kulayátʼixʼ áyá áxʼ ḵoowa.éin.
\snum{3}Du káak x̱ánxʼ yéi ÿatee, yú ḵáa.
\snum{4}Dáx̱náx̱ ÿatee du káak shát, yú ḵáa.
\snum{5}Áa áwé yaandéin ḵoowanei.
\snum{6}Chʼa tlákw nateich, yú ḵáa;
\snum{7}oodzikaa.
\snum{8}Tlax̱ ḵaa x̱ʼax̱ánt at shuxéex áwé naa wligáasʼ, du náḵ, yú oodzikaaÿi aa ḵu.aa.
\snum{9}Diÿée\-náx̱ aa du káak shátch áwé gáasʼ kʼi.eetéede at woog̱éixʼ, atx̱ʼéeshi, du x̱ʼéis.
\snum{10}Chʼa aadé woosh kát wudigút diÿéede.
\snum{11}Áyú yéi ash yawsiḵaa
\snum{12}«\!I x̱ʼéis gáasʼ kʼi.eetéede atx̱ʼéeshi x̱waag̱éexʼ\!».
\snum{13}Yéi kwgéi áwé washtóode andatéech.
\snum{14}Aax̱ kei ag̱atéen chʼa tlákw nateich.
\snum{15}Shanáa oo\-dasʼítch.
\pend
%2
\pstart
\snum{16}Wáa nanée sáwé du ÿeex̱ánde yéi ÿawdudziḵaa
\snum{17}«\!I eeg̱áa x̱at woosoo\!».
\snum{18}Tléikʼ, gwáayá l wudaa át át alg̱ein.
\snum{19}Wáa nanée sáwé táach uwajáḵ.
\snum{20}De du kaanáx̱ yáx̱ ÿatee, yú laaxw.
\snum{21}Wáa nanée sáwé yan aÿaawayék.
\snum{22}Yú yéi ash daaÿaḵá át, yéi kuligéiÿi át áwé du ÿeex̱ánt wujixeex.
\snum{23}Du oox̱ yéi kwdiyáatʼ.
\snum{24}Chʼu tle aax̱ aawasháat.
\snum{25}Du at.sʼéilʼi tóoxʼ áwé aÿaawashát.
\snum{26}Tle táach uwajáḵ.
\snum{27}Yéi ajóon «\!Héenxʼ xʼwán yan x̱at shát\!» yóo ash daaÿaḵá.
\snum{28}Ÿaa ḵeina.éini áwé a yáx̱ awsinei.
\snum{29}Éig̱i aan ÿeiḵ uwagút.
\snum{30}Du jintáaxʼ áwé kei ax̱échch.
\snum{31}«\!X̱aag̱áa eewasoo\!» yéi aÿanasḵáa áwé héennáx̱ aawax̱eich.
\snum{32}Wuduwasʼéḵ áwé át aawax̱éji yé.
\snum{33}Tle yánde ÿaa xeena.ádi áwé shanáa awdisʼít.
\snum{34}Chʼu tle ÿaa ḵeina.éini tin áwé aawa.áx̱ yéil sé du eeg̱ayáade.
\snum{35}Cháatl gwáaÿá yáanáx̱ yan awsigúḵ.
\snum{36}A téisʼi kaadé x̱ʼaÿaxát yú ash eeg̱áa woosoowu át.
\pend
%3
\pstart
\snum{37}Tsʼayóokʼ áwé hít yáx̱ jeewanei.
\snum{38}Aatlein awliyéx̱.
\snum{39}Tsʼootaat áwé éiḵ aan ÿeiḵ uwagút.
\snum{40}Ajeewanáḵ tsu.
\snum{41}Aax̱ ÿaa ḵeiga.áa áwé tsu éiḵ\-de wuduwa.áx̱ yéil sé.
\snum{42}Áa ÿeiḵ wujixeex.
\snum{43}Chʼu tle keejín áwé wooch shóode kawdiháa yú tsaa du eeg̱aÿáaxʼ.
\snum{44}Keejín aa lidíx̱ʼ aadé áwé x̱ʼayaxát.
\snum{45}Wé ash yeeg̱áa naseich át a daatx̱ a káa aawajél.
\snum{46}Tléil duteen dé du hídi a ÿee, yú kalóox̱jaach.
\snum{47}Yú ash náḵ wuligáasʼi du káak hás ḵu.aa éindéin woonee.
\pend
%4
\pstart
\snum{48}Wáa nanée sáwé du kináade kawdikʼítʼ jánwu.
\snum{49}X̱oodé ajeewanáḵ.
\snum{50}Tle ldakát daak kaawasóos.
\snum{51}Chʼu tle tléil wuduteen dé du hídi ÿee at shaÿala\-héin tlein.
\pend
%5
\pstart
\snum{52}Du toowúch láaxw áwé, du káakch aadé ḵu\-kaawaḵaa du eeg̱áa kag̱aax̱dusgaant áyú.
\snum{53}Du goox̱xʼú aadé akaawaḵaa.
\snum{54}Átx̱ áwé du x̱ánt uwaḵúx̱.
\snum{55}Wé goox̱xʼ neildé aawax̱oox̱.
\snum{56}\hspace{-0.0625em}Daaḵ awsi.át.
\hspace*{-0.1875em}\snum{57}X̱ʼéix̱ at téex̱.
\hspace*{-0.1875em}\snum{58}\hspace{-0.125em}Tléixʼ ash x̱áni uwax̱éi.
\snum{59}Ásíwé ÿátkʼw aÿa.óo yú goox̱.
\snum{60}Yéi sdu daaÿaḵá
\snum{61}«\!Líl kei aa ÿishátjiḵ xʼwán\!».
\snum{62}Á ásíwé at tóode aa woog̱éixʼ yú goox̱kʼúch.
\snum{63}«‹\!De kei wutusigán\!› yóo xʼwán sh kaneelneek i sʼaatí tin.\!»
\snum{64}Yóo yan ashukaawajáa.
\pend
%6
\pstart
\snum{65}Taat aan has ḵóox̱ áwé kei kawdig̱áx̱ du ÿátkʼu.
\snum{66}«\!Taayákʼw éi, taayákʼw éi\!» yóo kda\-g̱áax̱ yú goox̱ ÿátkʼu.
\snum{67}Ḵoowa.éin yú aan, yú át naa wligásʼi ÿé.
\snum{68}A x̱oo aa láxwtʼ.
\snum{69}Á áwé aawaḵeit yú aanḵáawuch, yú aadé kdag̱áx̱ ÿé, yú goox̱ ÿátkʼu.
\snum{70}Tsʼas a káa kei akanatéen.
\snum{71}Yóo kdag̱áax̱ yú goox̱ ÿátkʼu
\snum{72}«\!taaÿákʼw éi, taayákʼw éi\!».
\snum{73}Yóo kdag̱áax̱ yú goox̱ ÿátkʼu.
\snum{74}«\!Gáalʼ g̱eiÿí áwé ÿéi aÿasáakw\!» du tláa yéi ÿaawaḵaa.
\snum{75}X̱ách kichyát áwé aa datéen, yú goox̱ch du ÿátkʼu x̱ʼéis.
\snum{76}Aan shawdiḵei.
\snum{77}Du x̱ʼadaa wulitétlʼ.
\snum{78}Wáa nanée sáwé aan yan akaawaník.
\snum{79}Yéi aan akaawaneek du sʼaatí tin:
\snum{80}«\!Áwu hú.
\snum{81}Aatlein at shaÿalahéin áwé du jée.\!»
\pend
%7
\pstart
\snum{82}Chʼu tle naa x̱ʼakawduwanáa aadé.
\snum{83}Chʼa a yáx̱ gwáayú aatlein at shaÿalahéin gwáayú du jée.
\snum{84}Dloowkát sh daadané, tléixʼ ÿateeÿi aa du káak shát, ashikʼáani aa.
\snum{85}Du yadaa alg̱éigu áyú a tóox̱ at wooxeex.
\snum{86}Du washká aawaḵʼékʼw.
\snum{87}Wé du x̱ʼéis gáasʼ kʼi.eetéede at woog̱éixʼi aa ḵu.aa kʼédéin át tuwditán.
\snum{88}Chʼu tle du x̱ánt naa wligáasʼ yú aantḵeiní.
\snum{89}Yú du káak ḵu.aa yéi át tuwditán yú atx̱á l x̱ʼéi akoonax̱danoogú, ḵa du káak shát.
\snum{90}Chʼa aadé taÿeedé áwé kawlitʼík, du káak ḵa du káak shát.
\snum{91}Yú ash eet wudishéeÿi aa du káak shát tle aawasháa.
\snum{92}Yú aan du eeg̱áa ḵoowasoowu át ḵu.aa áwé goox̱g̱áa aawahoon.
\snum{93}Du eetx̱ yax̱ ÿawdudzi.óo aantḵeiní.
\snum{94}Yan kuda\-gáa áwé a daakeitkʼí yéi awsinei yú ash eeg̱áa woosoowu át.
\snum{95}Tléil aadóoch sá yei oostínch.
\snum{96}Chʼa ḵaa waḵwantʼéixʼ áyú.
\pend
%8
\pstart
\snum{97}Deikéex̱ ÿaa ÿandaxún yáay.
\snum{98}Aadé ajeewanáḵ.
\snum{99}Tsʼootaat áyú éiḵ ig̱ayáanáx̱ yan akaawaháa yú yáay tlénxʼ.
\snum{100}Chʼu a daat ḵu\-ÿa\-wustaag̱í áyú, yú yáay, a kát seiwaxʼáḵw.
\snum{101}Yú hóochʼi aaÿí daadé x̱ʼaxádi a kát seiwa\-xʼáḵw.
\snum{102}Daaḵ kadujéil áyú yú yáay.
\snum{103}Chʼu tle a kát sawuxʼaag̱úch áyú ḵútx̱ shoowa\-xeex yú aantḵeiní.
\snum{104}Ách áyú yéi at g̱waakóo:
\snum{105}«\!A náḵ naa wligáasʼi ooskaa sʼaatí yáx̱ xʼwán eeng̱waatee.\!».
\snum{106}Yú a.eeni át, ldakát ḵux̱ wudi.át.
\snum{107}Héende aa át kaawa.át ḵa dáḵde.
\snum{108}Hóochʼ, ldakát ḵútx̱ shoowaxeex yú aant\-ḵeiní.
\pend
\endnumbering
\end{Leftside}
%%
%% Column break.
%%
\begin{Rightside}
\beginnumbering
\pstart
\snum{1}They dwell, the townspeople.
\snum{2}Along the length of the town people were starving there.
\snum{3}He lived with his uncle, that man.
\snum{4}They are two, his uncle’s wives, that man.
\snum{5}It is there that people had gone hungry.
\snum{6}He was always sleeping, that man;
\snum{7}he was lazy.
\snum{8}People’s food having completely run out, the clan moved away from him,
		that one who is lazy, however.
\snum{9}It is the younger of his uncle’s wives who threw something in a housepost pit, dryfish, for him to eat.
\snum{10}Somehow they came upon each other below.
\snum{11}So she said to him
\snum{12}\qqk{}“I threw dryfish in a housepost pit for you to eat”.
\snum{13}It is a moderate amount that he always put in his cheek.
\snum{14}Whenever he took it out of there, he always slept.
\snum{15}He always wrapped up his head.
\pend
%2
\pstart
\snum{16}At some point down by him someone said
\snum{17}\qqk{}“I have given supernatural help to you”.
\snum{18}No, apparently he is looking at things which are not unfamiliar.
\snum{19}At some point he fell asleep.
\snum{20}Already he is like he is overcome, the starvation.
\snum{21}At some point he finally watched for it.
\snum{22}That thing talking to him, it is a small thing which was running around down by him.
\snum{23}His teeth were so long.
\snum{24}Just then he grabbed it up from there.
\snum{25}It was within his rags that he captured it.
\snum{26}Then he fell asleep.
\snum{27}He dreams that “Put me down in the water” it says to him.
\snum{28}It was as dawn came that he did so.
\snum{29}He went down to the beach with it.
\snum{30}He bounces it in his palm.
\snum{31}Having said “You gave supernatural help to me”, he threw it in the water.
\snum{32}It became smoky, the place where he threw it to.
\snum{33}Then it was as it was getting toward dusk that he wrapped up his head.
\snum{34}It was just with the dawning that he heard it, a raven’s voice toward the beach below him.
\snum{35}It was apparently a halibut that it had pushed ashore nearby.
\snum{36}Hanging on its flab was that thing which had given supernatural help to him.
\pend
%3
\pstart
\snum{37}Immediately he worked up something like a house.
\snum{38}He built a big one.
\snum{39}In the morning he went down to the beach with it.
\snum{40}He released it again.
\snum{41}After that, it was as it was dawning that again there was heard toward the beach a raven’s voice.
\snum{42}He ran there to the beach.
\snum{43}Just then there were five that appeared end to end, those seals, on the beach below him.
\snum{44}The fifth one’s neck, it is to there that it is fastened.
\snum{45}That thing that gave supernatural help to him, he carried it there from around it.
\snum{46}Now it could not be seen, the inside of his house, for the dripping.
\snum{47}Those uncles of his who had abandoned him however were suffering from famine.
\pend
%4
\pstart
\snum{48}At some point above him they were spread out, mountain goats.
\snum{49}He released it among them.
\snum{50}Then they all fell down.
\snum{51}Now the inside of his house could not be seen for the great multiplication of things.
\pend
%5
\pstart
\snum{52}Believing him to have starved, his uncle ordered people to go there for him to burn him up.
\snum{53}He ordered his slaves there.
\snum{54}So then they boated near to him.
\snum{55}The slaves called out to him inside.
\snum{56}He had them come up.
\snum{57}He gives them things to eat.
\snum{58}They overnighted once with him.
\snum{59}Apparently she has a little child, the slave.
\snum{60}He says to them
\snum{61}\qqk{}“Do not take anything”.
\snum{62}Then apparently he tossed some into something, that little slave.
\snum{63}\qqk{}“Tell your master ‘we already burned him up’.”
\snum{64}Thus he instructed him.
\pend
%6
\pstart
\snum{65}At night having boated back with it, he cried out, her little child.
\snum{66}\qqk{}“Little fat hey, little fat hey” cried that slave’s little child.
\snum{67}People were starving at that town, the place where the clan had moved to.
\snum{68}Some among them are starving to death.
\snum{69}It was then that the chief began to suspect him, the way that he is crying, that slave’s little child.
\snum{70}Apparently he was just adding to it.
\snum{71}Thus he cries, that slave’s little child
\snum{72}\qqk{}“Little fat hey, little fat hey”.
\snum{73}Thus he cries, that slave’s little child.
\snum{74}\qqk{}“The inside of a clam is what he is calling it” his mother said.
\snum{75}Actually she had some at her side, that slave, for her baby to eat.
\snum{76}They sat up with it.
\snum{77}Around his mouth it was fatty.
\snum{78}At some point she admitted it to him.
\snum{79}Thus she told about him to him, her master:
\snum{80}\qqk{}“He is there.
\snum{81}He has much in his possession.”
\pend
%7
\pstart
\snum{82}So then the clan was ordered there.
\snum{83}Apparently it is just like he seemingly has much in his possession.
\snum{84}She makes herself up carefully, the first one of his uncle’s wives, the one who hates him.
\snum{85}It was as she is wiping her face that something fell into it.
\snum{86}She cut the surface of her cheek.
\snum{87}That one who had thrown something in a housepost pit for him to eat however, he thought well of her.
\snum{88}So then the clan moved to be near him, those townspeople.
\snum{89}His uncle however decided that he had no sense of taste for the food, and his uncle’s wife.
\snum{90}It was somehow in bed that they became stiff, his uncle and his uncle’s wife.
\snum{91}That one of his uncle’s wives who had helped him, he then married her.
\snum{92}It is that stuff that he was given supernatural help with however that he sold for slaves.
\snum{93}They bought it all up from him, the townspeople.
\snum{94}It was after a while that he made a container for it, that thing that gave supernatural help to him.
\snum{95}Nobody ever saw it.
\snum{96}It is out of sight.
\pend
%8
\pstart
\snum{97}Out at sea whales are showing their faces.
\snum{98}He released it there.
\snum{99}It was in the morning on the beach that it had moved them along the beach below, those big whales.
\snum{100}So it was while people were taking care of them, those whales, that he forgot about it.
\snum{101}Hanging on the outside of that last one, he forgot about it.
\snum{102}So they are lugging them up, those whales.
\snum{103}So then it was because he had forgotten about it that they died off, those townspeople.
\snum{104}That is why there is a saying:
\snum{105}\qqk{}“You may be like that master of laziness whose clan abandoned him.”
\snum{105}Those things that he kills, they all came back.
\snum{106}Some wandered to the water, and some inland.
\snum{107}It is finished, they all died out, those townspeople.
\pend
\endnumbering
\end{Rightside}
\end{pairs}
\Columns

\vspace{1\baselineskip}

\section{Swanton’s abstract}\label{sec:90-swanton-abstract}

A lazy man was abandoned by his townspeople, who left him nothing except a piece of dried fish which one of his uncle’s wives dropped into a post hole.
After that a small animal killed all kinds of game for him, and he became wealthy, while the other people were starving.
By and by some slaves were sent to burn his body and were feasted by him.
They were told not to say anything about him, but one of them concealed a piece of fat for her child and the cries of the infant over this food let the truth be discovered.
Then they went to him and he became a great chief.
He married the woman who had been good to him, but killed his uncle’s other wife and her husband.

\section{Swanton’s translation}\label{sec:90-swanton-translation}

\snum{1}People living \snum{2}in a long town were suffering from famine.
\snum{3}A certain man stayed with his uncle, \snum{4}who had two wives.
\snum{5}The people were very hungry.
\snum{6}This man was always sleeping, \snum{7}for he was lazy.
\snum{8}When their food was all gone, they started away from the lazy man to camp, \snum{9}but his uncle’s wife threw some dried fish into a hole beside the house post for him, \snum{10}while she was walking around back of the fire.
\snum{11}Then she said to him, \snum{12}\qqk{}“I threw a piece of dried fish into the post hole for you.” \snum{13}He would put a small piece of this into his mouth.
\snum{14}When he took it out, he would go to sleep.
\snum{15}He always had his head covered.

\snum{16}Suddenly something said to him, \snum{17}\qqk{}“I am come to help you.” \snum{18}When he looked there was nothing there.
\snum{19}At once he fell asleep.
\snum{20}Hunger was overcoming him.
\snum{21}At once he prepared himself for it.
\snum{22}What was speaking to him was a small thing running around him.
\snum{23}Its teeth were long.
\snum{24}Then he took it away.
\snum{25}He put it among his rags, \snum{26}and fell asleep again.
\snum{27}Then he dreamed that it said to him, “Put me into the water.” \snum{28}When it was getting light he did so.
\snum{29}He went down into the water with it.
\snum{30}He kept throwing it up and down in his hands.
\snum{31}Saying, “You came to help me,” he threw it into the water.
\snum{32}Where he threw it in [the water] smoked.
\snum{33}And when it was getting dark he covered his head.
\snum{34}When day was beginning to dawn he heard the cry of the raven below him.
\snum{35}A halibut had drifted ashore there, \snum{36}and the thing that was helping him was at its heart.

\snum{37}Quickly he built a house.
\snum{38}He built a big one.
\snum{39}In the morning he went down to the beach with his helper \snum{40}and let it go.
\snum{41}Toward daylight he again heard the raven’s call at the beach, \snum{42}and he ran down.
\snum{43}Then five seals were floating below him, one behind another.
\snum{44}His helper hung around the neck of the fifth, \snum{45}and he took it off.
\snum{46}One could not see about inside of his house on account of the drippings.
\snum{47}His uncles who had left him, however, were suffering from famine.

\snum{48}Suddenly some mountain sheep came out above him.
\snum{49}He let it go among them.
\snum{50}Then all fell down.
\snum{51}The inside of his house could not be seen on account of the great abundance of food.

\snum{52}Now when his uncle thought that he had died he sent some one thither to burn his body.
\snum{53}His slaves that he told to go after him \snum{54}came thither, \snum{55}and he called the slaves into the house.
\snum{56}They came up.
\snum{57}He gave them things to eat, \snum{58}and they remained with him one night.
\snum{59}One of these slaves had a child.
\snum{60}Then he said to them, \snum{61}\qqk{}“Do not take away anything.” \snum{62}The little slave, however, threw a piece inside of something.
\snum{63}\qqk{}“Tell your household that you burned me up.” \snum{64}He left those directions with them.

\snum{65}When they reached home that night the baby began to cry: \snum{66}\qqk{}“Little fat.
Little fat,” the slave's child began to cry out.
\snum{67}There was a great famine in the town whither the people had moved.
\snum{68}Some among them had died.
\snum{69}Then the chief thought about the way the slave's baby was crying.
\snum{71}He kept crying \snum{70}louder: \snum{72}\qqk{}“Little fat, Little fat,” \snum{73}he cried.
\snum{74}His mother said, “He is crying for the inside of a clam.” \snum{75}But the slave had a piece of fat on her side for her baby.
\snum{76}She sat up with it.
\snum{77}Its mouth was greasy all over.
\snum{78}At once she confessed to him.
\snum{79}She said to her master, \snum{80}\qqk{}“He is there.
\snum{81}The things that he has are many.”

\snum{82}Then all started thither.
\snum{83}Indeed it was a great quantity of things that he had.
\snum{84}The wife of his uncle who had hated him tried to make herself look pretty, \snum{85}but when she wiped her face something got inside of the rag \snum{86}and she cut her face.
\snum{87}But the one who had thrown something into the post hole for him, he thought kindly toward.
\snum{88}Then the people moved to him.
\snum{89}He willed, however, that the food should not fill his uncle or his uncle’s wife.
\snum{90}Just where they lay, his uncle and his uncle’s wife were dead.
\snum{91}So he married the other wife that helped him.
\snum{92}The food his helper obtained for him, however, he sold for slaves.
\snum{93}The people came to him to buy everything.
\snum{94}Afterward he fixed a little box for the thing that had helped him.
\snum{95}No one ever saw it because \snum{96}it was kept out of sight.

\snum{97}One day a whale came along, moving up and down, \snum{98}and he let his helper go at it.
\snum{99}In the morning the big whale floated up below on the beach.
\snum{100}When all were busy with the whale he forgot his helper.
\snum{101}It was hanging to the last piece.
\snum{102}When they took up the whale he forgot it.
\snum{103}And because he forgot it all of the people were destroyed.
\snum{104}This is why people say to a lazy man even now: \snum{105}\qqk{}“You will be like the man that was abandoned.” \snum{106}All the things that had been killed came to life.
\snum{107}Some ran into the water and some into the woods.
\snum{108}The people were completely destroyed.

\clearpage
\section{Paragraph 1}\label{sec:90-para-1}

\ex\label{ex:90-1-townspeople-dwell}%
\exmn{262.1}%
\begingl
	\glpreamble	Qowau′wau āntqenî′ //
	\glpreamble	Ḵuwa.óo áyú, aantḵeiní. //
	\gla	\rlap{Ḵuwa.óo} @ {} @ {} @ {}
		\rlap{áyú} @ {}
		{} \rlap{aantḵeiní.} @ {} @ {} @ {} @ {} @ {} {} //
	\glb	ḵu- i- \rt[²]{.u} -μμH
		á -yú
		{} aan- d- \rt[¹]{ḵi} -μμL -n -í {} //
	\glc	\xx{areal}- \xx{stv}- \rt[²]{own} -\xx{var}
		\xx{foc} -\xx{dist}
		{}[\pr{DP} town- \xx{mid}- \rt[¹]{sit·\xx{pl}} -\xx{var} -\xx{nsfx} -\xx{nmz} {}] //
	\gld	\rlap{\xx{impfv}.be.dwelling} {} {} {}
		\rlap{it.is} {}
		{} \rlap{townspeople} {} {} {} {} {} {} //
	\glft	‘They dwell, the townspeople.’
		//
\endgl
\xe

The word \citeauthor{swanton:1909} transcribes as \orth{Qowau′wau} appears to be composed of the verb \fm{ḵuwa.óo} and the focus particle \fm{áyú}.
The \orth{Qowau′} part is the verb and the immediately following \orth{w} is probably a non-phonemic offglide from the vowel [\ipa{u}] at the end of the verb, indicating that the speaker did not insert a glottal stop.
\citeauthor{swanton:1909}‘s transcription of the focus particle as \orth{au′} rather than say \orth{ayu′} suggests that the speaker either had [\ipa{ɰ}] rather than the expected [\ipa{j}], or that \citeauthor{swanton:1909} simply did not hear the [\ipa{j}].
Since \fm{áyú} is not otherwise attested with [\ipa{ɰ}] as \fm[*]{áÿú}, it is more likely that \citeauthor{swanton:1909} missed the [\ipa{j}].
A possible alternative interpretation is a locative predicate  \fm{áwu} with \fm{-ú}, but that would be ungrammatical in this context.

\ex\label{ex:90-2-along-length-starving}%
\exmn{262.1}%
\begingl
	\glpreamble	ān qołayᴀ′tq! aaya′ ᴀq! qō′waēn. //
	\glpreamble	Aan kulayátʼixʼ áyá áxʼ ḵoowa.éin. //
	\gla	{} {} Aan \rlap{kulayátʼixʼ} @ {} @ {} @ {} @ {} @ {} {} {} {}
		\rlap{áyá} @ {}
		{} \rlap{áxʼ} @ {} {} 
		\rlap{ḵoowa.éin.} @ {} @ {} @ {} @ {} //
	\glb	{} {} aan k- u- l- \rt[¹]{ÿatʼ} -μH -í {} -xʼ {} 
		á -yá
		{} á -xʼ {}
		ḵu- wu- i- \rt[²]{.eʼn} -μμH //
	\glc	{}[\pr{PP} {}[\pr{DP} town \xx{cmpv}- \xx{irr}- \xx{xtn}- \rt[¹]{long} -\xx{var} -\xx{pss} {}] -\xx{loc} {}]
		\xx{foc} -\xx{prox}
		{}[\pr{PP} \xx{3n} -\xx{loc} {}]
		\xx{areal}- \xx{pfv}- \xx{stv}- \rt[¹]{starve} -\xx{var} //
	\gld	{} {} town \rlap{length} {} {} {} {} -of {} -at {}
		\rlap{it.is} {}
		{} there -at {}
		\rlap{people.\xx{pfv}.starve} {} {} {} {} //
	\glft	‘Along the length of the town, people were starving there.’
		//
\endgl
\xe

Sentences (\ref{ex:90-1-townspeople-dwell}) and (\ref{ex:90-2-along-length-starving}) are run together by \citeauthor{swanton:1909} into a single sentence in both his transcription and translation, but their morphology and syntax suggest that they are two separate sentences.
Specifically, (\ref{ex:90-1-townspeople-dwell}) has a main clause verb form \fm{ḵuwa.óo} ‘people dwell’ and a focus particle \fm{áyú}, together implying that these are a single sentence.
Likewise, (\ref{ex:90-2-along-length-starving}) includes a focused phrase preceding the focus particle \fm{áyá} as well as a main clause verb form \fm{ḵoowa.éin} ‘people have starved’.

\ex\label{ex:90-3-lived-beside-uncle}%
\exmn{262.1}%
\begingl
	\glpreamble	Dukā′k xᴀnq! yē′ỵatî, yuqā. //
	\glpreamble	Du káak x̱ánxʼ yéi ÿatee, yú ḵáa. //
	\gla	{} Du káak \rlap{x̱ánxʼ} @ {} {}
		yéi @ \rlap{ÿatee} @ {} @ {} 
		{} yú ḵáa. {} //
	\glb	{} du káak x̱án -xʼ {}
		yéi= i- \rt[¹]{tiʰ} -μμL
		{} yú ḵáaʷ {} //
	\glc	{}[\pr{PP} \xx{3h·pss} mat·uncle near -\xx{loc} {}]
		thus= \xx{stv}- \rt[¹]{be} -\xx{var}
		{}[\pr{DP} \xx{dist} man {}] //
	\gld	{} his mat·uncle beside -at {}
		thus \rlap{\xx{impfv}.be} {} {} 
		{} that man {} //
	\glft	‘He lived with his uncle, that man.’
		//
\endgl
\xe

\ex\label{ex:90-4-wives-are-two}%
\exmn{262.2}%
\begingl
	\glpreamble	Dᴀxnᴀ′x ÿatî′ dukā′k cᴀt.
Yuqā′ //
	\glpreamble	Dáx̱náx̱ ÿatee du káak shát, yú ḵáa. //
	\gla	{} \rlap{Dáx̱náx̱} @ {} {}
		\rlap{ÿatee} @ {} @ {}
		{} du káak shát, {}
		{} yú ḵáa. {} //
	\glb	{} déix̱ -náx̱ {}
		i- \rt[¹]{tiʰ} -μμL
		{} du káak shát {}
		{} yú ḵáaʷ {} //
	\glc	{}[\pr{NP} two -\xx{hum} {}]
		\xx{stv}- \rt[¹]{be} -\xx{var}
		{}[\pr{DP} \xx{3h·pss} uncle:\xx{inal} wife:\xx{inal} {}]
		{}[\pr{DP} \xx{dist} man {}] //
	\gld	{} \rlap{two} {} {}
		\rlap{\xx{impfv}.be} {} {}
		{} his uncle wife {}
		{} that man {} //
	\glft	‘They are two, his uncle’s wives, that man.’
		//
\endgl
\xe

\citeauthor{swanton:1909} give the final \fm{yú ḵáa} ‘that man’ of (\lastx) as the beginning of (\nextx).
This is ungrammatical because the verb in (\nextx) is intransitive and its lone argument is explicitly indicated by \fm{ḵu-} ‘someone, people’.
Analyzing \fm{yú ḵáa} as part of (\lastx) works since the phrase makes explicit the referent of \fm{du} ‘his’ and forms a parallel with the ending of the sentence in (\ref{ex:90-3-lived-beside-uncle}).

\ex\label{ex:90-5-there-went-hungry}%
\exmn{262.2}%
\begingl
	\glpreamble	ā′awe yān dēnkū′wane. //
	\glpreamble	Áa áwé yaandéin ḵoowanei. //
	\gla	{} \rlap{Áa} @ {} {}
		\rlap{áwé} @ {}
		{} \rlap{yaandéin} @ {} {}
		\rlap{ḵoowanei.} @ {} @ {} @ {} @ {}  //
	\glb	{} á -μ {}
		á -wé
		{} yaan -déin {}
		ḵu- wu- i- \rt[¹]{niʰ} -μμL //
	\glc	{}[\pr{PP} \xx{3n} -\xx{loc} {}]
		\xx{foc} -\xx{mdst}
		{}[\pr{AdvP} hunger -\xx{adv} {}]
		\xx{areal}- \xx{pfv}- \xx{stv}- \rt[¹]{happen} -\xx{var} //
	\gld	{} there -at {}
		\rlap{it.is} {}
		{} hunger -ly {}
		\rlap{people.\xx{pfv}.happen} {} {} {} {} //
	\glft	‘It is there that people had gone hungry.’
		//
\endgl
\xe

The English translation of the sentence in (\lastx) is deliberately loose.
A more literal translation is ‘it is there that it happened hungrily to people’.
The noun \fm{yaan} is only rarely found today outside of the phrase \fm{ax̱ eet yaan uwaháa} ‘I have gotten hungry’ (lit.\ ‘hunger has appeared to me’), but its appearance in \fm{yaandéin} ‘hungrily’ shows that it was more productive in the past.

\ex\label{ex:90-6-always-sleeping}%
\exmn{262.3}%
\begingl
	\glpreamble	Tc!a ʟᴀkᵘ natē′tc yuqā′ //
	\glpreamble	Chʼa tlákw nateich, yú ḵáa; //
	\gla	Chʼa tlákw \rlap{nateich} @ {} @ {} @ {}
		{} yú ḵáa; {} //
	\glb	chʼa tlákw n- \rt[¹]{taʰ} -eμL -ch
		{} yú ḵáaʷ {} //
	\glc	just always \xx{ncnj}- \rt[¹]{sleep·\xx{sg}} -\xx{var} -\xx{rep}
		{}[\pr{DP} \xx{dist} man {}] //
	\gld	just always \rlap{\xx{hab}.sleep} {} {} {}
		{} that man {} //
	\glft	‘He was always sleeping, that man;’
		//
\endgl
\xe

\ex\label{ex:90-7-lazy}%
\exmn{262.3}%
\begingl
	\glpreamble	ūdzîka′. //
	\glpreamble	oodzikaa. //
	\gla	\rlap{oodzikaa.} @ {} @ {} @ {} @ {} @ {} @ {} //
	\glb	a- u- d- s- i- \rt[²]{kaʰ} -μμL //
	\glc	\xx{xpl}- \xx{irr}- \xx{mid}- \xx{xtn}- \xx{stv}- \rt[²]{lazy} -\xx{var} //
	\gld	\rlap{\xx{impfv}.be.lazy} {} {} {} {} {} {} //
	\glft	‘he is lazy.’
		//
\endgl
\xe

The sentences in (\ref{ex:90-6-always-sleeping}) and (\ref{ex:90-7-lazy}) are an instance of parataxis.
The sentence in (\ref{ex:90-7-lazy}) is an explanation for the behaviour described in (\ref{ex:90-6-always-sleeping}), and (\ref{ex:90-7-lazy}) is also short since it consists of a single verb word.
This kind of explanatory parataxis without any explicit discourse particles linking the two sentences is fairly common in Tlingit conversation and oratory, though perhaps less so in narrative.

\ex\label{ex:90-8-food-out-moved-away}%
\exmn{262.3}%
\begingl
	\glpreamble	ʟᴀx qāq!axᴀ′nt acux̣ī′x̣awe naołig̣ā′s! dunᴀ′q yū′udzikaỵa-qua. //
	\glpreamble	Tlax̱ ḵaa x̱ʼax̱ánt at shuxéex áwé naa wligáasʼ, du náḵ, yú oodzikaaÿi aa ḵu.aa. //
	\gla	{} Tlax̱ 
			{} ḵaa \rlap{x̱ʼax̱ánt} @ {} @ {} {} 
			at @ \rlap{shuxéex} @ {} @ {} @ {} @ {} {}
		\rlap{áwé} @ {} +
		naa @ \rlap{wligáasʼ,} @ {} @ {} @ {} @ {}
		{} du náḵ, {} +
		{} yú
			{} \rlap{oodzikaaÿi} @ {} @ {} @ {} @ {} @ {} @ {} @ {} {}
			aa {} ḵu.aa. //
	\glb	{} tlax̱
			{} ḵaa x̱ʼé- x̱án -t {}
			at= shu- {} \rt[¹]{xix} -μμH {} {}
		á -wé
		naa= wu- l- i- \rt[¹]{gasʼ} -μμH
		{} du náḵ {}
		{} yú 
			{} a- u- d- s- i- \rt[¹]{kaʰ} -μμL -i {}
			aa {} ḵu.aa //
	\glc	{}[\pr{CP} very
			{}[\pr{PP} \xx{4h·pss} mouth- near -\xx{pnct} {}]
			\xx{4n·o}= end- \xx{zcnj}\· \rt[¹]{fall} -\xx{var} \·\xx{sub} {}]
		\xx{foc} -\xx{mdst}
		clan= \xx{pfv}- \xx{xtn}- \xx{stv}- \rt[¹]{extend} -\xx{var}
		{}[\pr{PP} \xx{3h} \xx{elat} {}]
		{}[\pr{DP} \xx{dist}
			{}[\pr{CP} \xx{xpl}- \xx{irr}- \xx{mid}- \xx{xtn}- \xx{stv}- \rt[²]{lazy} -\xx{var} -\xx{rel} {}]
			\xx{part} {}] \xx{contr} //
	\gld	{} very
			{} one’s mouth- beside -at {}
			thing \rlap{\xx{csv}.run·out} {} {} {} -when {}
		\rlap{it.is} {}
		clan \rlap{\xx{pfv}.relocate} {} {} {} {}
		{} him away {}
		{} that
			{} \rlap{\xx{impfv}.be.lazy} {} {} {} {} {} {} -who {}
			one {} however //
	\glft	‘People’s food having completely run out, the clan moved away from him,
		that one who is lazy, however.’
		//
\endgl
\xe

\ex\label{ex:90-9-lower-wife-dryfish}%
\exmn{262.4}%
\begingl
	\glpreamble	Dīỵī′nᴀx aa dukā′k cᴀ′ttcawe g̣ās!-k!î ite′dê ᴀt wug̣ēq! ᴀtq!ē′cî doq!ē′s //
	\glpreamble	Diÿéenáx̱ aa du káak shátch áwé gáasʼ kʼi.eetéede at woog̱éixʼ, atx̱ʼéeshi, du x̱ʼéis. //
	\gla	{} {} \rlap{Diÿéenáx̱} @ {} {} aa du káak \rlap{shátch} @ {} {}
		\rlap{áwé} @ {} +
		{} gáasʼ \rlap{kʼi.eetéede} @ {} @ {} {}
		at @ \rlap{woog̱éixʼ,} @ {} @ {} @ {} +
		{} \rlap{atx̱ʼéeshi,} @ {} @ {} @ {} {}
		{} du \rlap{x̱ʼéis.} @ {} {} //
	\glb	{} {} diÿée -náx̱ {} aa du káak shát -ch {}
		á -wé
		{} gáasʼ kʼí- eetí -dé {}
		at= wu- i- \rt[²]{g̱ixʼ} -μμH
		{} at= \rt[²]{x̱ʼish} -μμH -í {}
		{} du x̱ʼé -ÿís {} //
	\glc	{}[\pr{DP} {}[\pr{PP} below -\xx{perl} {}] \xx{part} \xx{3h·pss} mat·uncle wife -\xx{erg} {}]
		\xx{foc} -\xx{mdst}
		{}[\pr{PP} housepost base- remains -\xx{all} {}]
		\xx{4n·o}= \xx{pfv}- \xx{stv}- \rt[²]{throw·\xx{sg}} -\xx{var}
		{}[\pr{DP} \xx{4n·o}= \rt[²]{dry·fish} -\xx{var} -\xx{nmz} {}]
		{}[\pr{PP} \xx{3h·pss} mouth -\xx{ben} {}] //
	\gld	{} {} below -along {} one his mat·uncle wife {} {}
		\rlap{it.is} {}
		{} housepost base- hole -to {}
		thing \rlap{\xx{pfv}.throw·\xx{sg}} {} {} {}
		{} \rlap{dryfish} {} {} {} {}
		{} his mouth- for {} //
	\glft	‘It is the younger of his uncle’s wives who threw something in a housepost pit,
		dryfish, for him to eat.’
		//
\endgl
\xe

The noun phrase \fm{diÿéenáx̱ aa} in (\lastx) literally means ‘lower one’ and describes a thing that is below some other implicit thing.
Here this is used as a modifier of \fm{du káak shát} ‘his maternal uncle’s wife’ to pick out the younger of the uncle’s two wives that were established in (\ref{ex:90-4-wives-are-two}).
This is because the younger wife is metaphorically below the older wife, and so would be physically positioned below the older wife in many situations.
\citeauthor{swanton:1909}’s translation “back of the fire” is erroneous and probably comes from a misinterpretation of the narrator’s clarifying gesture toward the back side of the house’s firepit where an aristocratic woman would traditionally sit.

The fourth person nonhuman object \fm{at=} ‘something, stuff’ like in \fm{at woog̱éixʼ} ‘s/he threw something’ does not normally occur with a coreferential DP, but in (\lastx) it appears to be coreferential with the noun \fm{atx̱ʼéeshi} ‘dryfish’. (See chapter \ref{ch:100-salmon-boy-wrg} at sentence (\ref{ex:100-8-ask-for-dryfish}) on page \ref{note:100-dryfish-discussion} for a discussion of the word \fm{atx̱ʼéeshi} ‘dryfish’.) The position of the DP \fm{atx̱ʼéeshi} after the verb word suggests that it is given but no dryfish has been previously mentioned in this narrative.
Instead \fm{atx̱ʼéeshi} seems to be an ‘afterthought’, a phrase added to clarify ambiguity earlier in the sentence.

\ex\label{ex:90-10-somehow-encountered}%
\exmn{262.5}%
\begingl
	\glpreamble	Tca adê′ wuckᴀ′t wudigu′t diyē′dî. //
	\glpreamble	Chʼa aadé woosh kát wudigút diÿéede. //
	\gla	{} Chʼa \rlap{aadé} @ {} {} 
		{} woosh \rlap{kát} @ {} {}
		\rlap{wudigút} @ {} @ {} @ {} @ {}
		{} \rlap{diÿéede.} @ {} {} //
	\glb	{} chʼa á -dé {}
		{} woosh ká -t {}
		wu- d- i- \rt[¹]{gut} -μH
		{} diÿée -dé {} //
	\glc	{}[\pr{PP} just \xx{3n} -\xx{all} {}]
		{}[\pr{PP} \xx{recip·pss} \xx{hsfc} -\xx{pnct} {}]
		\xx{pfv}- \xx{mid}- \xx{stv}- \rt[¹]{go·\xx{sg}} -\xx{var}
		{}[\pr{PP} below -\xx{all} {}] //
	\gld	{} just it -to {}
		{} ea·oth’s atop -to {}
		\rlap{\xx{pfv}.go·\xx{sg}} {} {} {} {}
		{} below -to {} //
	\glft	‘Somehow they came upon each other below.’
		//
\endgl
\xe

\citeauthor{swanton:1909} glosses \fm{diÿéede} in (\lastx) as “at the rear of the fire” but just as in (\ref{ex:90-9-lower-wife-dryfish}) the noun \fm{diÿée} simply means ‘below’.
The origo (point of reference) for this noun is implicit and could just as well be the house, in which case \fm{diÿéede} would refer to the beach in front of the house.

\ex\label{ex:90-11-so-she-said}%
\exmn{262.6}%
\begingl
	\glpreamble	Ayu′ ye′ acia′osîqa //
	\glpreamble	Áyú yéi ash yawsiḵaa //
	\gla	\rlap{Áyú} @ {}
		yéi @ ash @ \rlap{yawsiḵaa} @ {} @ {} @ {} @ {} @ {} //
	\glb	á -yú
		yéi= ash= ÿ- wu- s- i- \rt[¹]{ḵa} -μμL //
	\glc	\xx{foc} -\xx{dist}
		thus= \xx{3prx·o}= \xx{qual}- \xx{pfv}- \xx{csv}- \xx{stv}- \rt[¹]{say} -\xx{var} //
	\gld	\rlap{it.is} {}
		thus him \rlap{\xx{pfv}.say·to} {} {} {} {} {} //
	\glft	‘So she said to him’
		//
\endgl
\xe

\ex\label{ex:90-12-i-threw-dryfish}%
\exmn{262.6}%
\begingl
	\glpreamble	“Eq!ē′s g̣ās!-k!î ītī′dî ᴀtq!ē′ci xâg̣ē′q!.” //
	\glpreamble	«\!I x̱ʼéis gáasʼ kʼi.eetéede atx̱ʼéeshi x̱waag̱éexʼ\!». //
	\gla	{} \llap{«\!}I \rlap{x̱ʼéis} @ {} {}
		{} gáasʼ \rlap{kʼi.eetéede} @ {} @ {} {}
		{} \rlap{atx̱ʼéeshi} @ {} @ {} @ {} {}
		\rlap{x̱waag̱éexʼ\!».} @ {} @ {} @ {} @ {} //
	\glb	{} i x̱ʼé -ÿís {}
		{} gáasʼ kʼí- eetí -dé {}
		{} at= \rt[²]{x̱ʼish} -μμH -í {}
		wu- x̱- i- \rt[²]{g̱ixʼ} -μμH //
	\glc	{}[\pr{PP} \xx{2sg·pss} mouth -\xx{ben} {}]
		{}[\pr{PP} housepost base- remains -\xx{all} {}]
		{}[\pr{DP} \xx{4n·o}= \rt[²]{dry·fish} -\xx{var} -\xx{nmz} {}]
		\xx{pfv}- \xx{1sg·s}- \xx{stv}- \rt[²]{throw·\xx{sg}} -\xx{var} //
	\gld	{} your mouth -for {}
		{} housepost base- hole -to {}
		{} \rlap{dryfish} {} {} {} {}
		\rlap{\xx{pfv}.I.throw·\xx{sg}} {} {} {} {} //
	\glft	‘“I threw dryfish in a housepost pit for you to eat”.’
		//
\endgl
\xe

\ex\label{ex:90-13-small-piece-in-cheek}%
\exmn{262.7}%
\begingl
	\glpreamble	Yē′kᵘg̣e awe′ wuctū′dî ᴀndatī′tc. //
	\glpreamble	Yéi kwgéi áwé washtóode andatéech. //
	\gla	{} Yéi @ \rlap{kwgéi} @ {} @ {} @ {} {}
		\rlap{áwé} @ {}
		{} {} \rlap{washtóode} @ {} @ {} {}
		\rlap{andatéech.} @ {} @ {} @ {} @ {} @ {}  //
	\glb	{} yéi= k- w- \rt[¹]{ge} -μμH {}
		á -wé
		{} {} wásh- tú -dé {}
		a- n- d- \rt[²]{ti} -μμH -ch //
	\glc	{}[\pr{NP} thus= \xx{cmpv}- \xx{irr}- \rt[¹]{big} -\xx{var} {}]
		\xx{foc} -\xx{mdst}
		{}[\pr{PP} \xx{rflx·pss} cheek- inside -\xx{all} {}]
		\xx{arg}- \xx{ncnj}- \xx{mid}- \rt[¹]{handle} -\xx{var} -\xx{rep} //
	\gld	{} thus \rlap{\xx{cmpv}.\xx{impfv}.big} {} {} {} {}
		\rlap{it.is} {}
		{} self’s cheek- inside -to {}
		\rlap{3>3.\xx{hab}.handle} {} {} {} {} {} //
	\glft	‘It is a moderate amount that he always put in his cheek.’
		//
\endgl
\xe

The transcription \orth{Yē′kᵘg̣e} suggests a form \fm{yéi kwg̱éi} but this is nonsense.
Instead the actual form seems to be \fm{yéi kwgéi} which is a comparative dimension state verb based on the root \fm{\rt[¹]{ge}} ‘big; much’.
This fits with \citeauthor{swanton:1909}’s gloss “A small amount”.
The translation and syntactic context suggest that this should be a deverbal noun hence the analysis as an NP; the only indication of nominalization is the lack of the stative prefix \fm{i-}.
An alternative is that the original form was actually \fm{yéi koogéiyi aa} ‘some which is moderately big’, but this would require significant loss in \citeauthor{swanton:1909}’s transcription.

\ex\label{ex:90-14-take-out-sleep}%
\exmn{262.8}%
\begingl
	\glpreamble	Āx ke ag̣atī′n tc!a ʟᴀkᵘ natē′tc. //
	\glpreamble	Aax̱ kei ag̱atéen chʼa tlákw nateich. //
	\gla	{} {} \rlap{Aax̱} @ {} {}
			kei @ \rlap{ag̱atéenín} @ {} @ {} @ {} @ {} @ {} @ {} {}
		chʼa tlákw \rlap{nateich.} @ {} @ {} @ {} //
	\glb	{} {} á -dáx̱ {}
			kei= a- {} g̱- \rt[²]{tin} -μμH -n -ín {}
		chʼa tlákw n- \rt[¹]{taʰ} -eμL -ch //
	\glc	{}[\pr{CP} {}[\pr{PP} \xx{3n} -\xx{abl} {}]
			up= \xx{arg}- \xx{zcnj}\· \xx{mod}- \rt[²]{handle} -\xx{var} -\xx{nsfx} -\xx{ctng} {}]
		just always \xx{ncnj}- \rt[¹]{sleep·\xx{sg}} -\xx{var} -\xx{rep} //
	\gld	{} {} it -from {}
			up \rlap{3>3.\xx{ctng}.handle} {} {} {} {} {} {} {}
		just always \rlap{\xx{hab}.sleep·\xx{sg}} {} {} {} //
	\glft	‘Whenever he took it out of there, he always slept.’
		//
\endgl
\xe

\ex\label{ex:90-15-wrap-head}%
\exmn{262.8}%
\begingl
	\glpreamble	Cᴀna′odas!î′ttc. //
	\glpreamble	Shanáa oodasʼítch. //
	\gla	{} {} \rlap{Shanáa} @ {} {}
		\rlap{oodasʼítch.} @ {} @ {} @ {} @ {} @ {} //
	\glb	{} {} shá- náa {}
		a- u- d- \rt[²]{sʼit} -μH -ch //
	\glc	{}[\pr{DP} \xx{rflx·pss} head- cover {}]
		\xx{arg}- \xx{zpfv}- \xx{mid}- \rt[²]{wrap} -\xx{var} -\xx{rep} //
	\gld	{} self’s head- cover {}
		\rlap{3>3.\xx{hab}.wrap} {} {} {} {} {} //
	\glft	‘He always wrapped up his head.’
		//
\endgl
\xe

\clearpage
\section{Paragraph 2}\label{sec:90-para-2}

\ex\label{ex:90-16-something-says-below}%
\exmn{262.9}%
\begingl
	\glpreamble	Wananī′sawe du ỵîxᴀ′ndî yē ỵa′odudzîqa //
	\glpreamble	Wáa nanée sáwé du ÿeex̱ánde yéi ÿawdudziḵaa //
	\gla	{} Wáa \rlap{nanée} @ {} @ {} @ {} {}
		\rlap{sáwé} @ {} @ {}
		{} du \rlap{ÿeex̱ánde} @ {} @ {} {}
		yéi @ \rlap{ÿawdudziḵáa} @ {} @ {} @ {} @ {} @ {} @ {} @ {} //
	\glb	{} wáa n- \rt[¹]{ni} -μμH {} {} 
		s= á -wé
		{} du ÿee -x̱án -dé {}
		yéi= ÿ- wu- du- d- s- i- \rt[¹]{ḵa} -μμL //
	\glc	{}[\pr{CP} how \xx{ncnj}- \rt[¹]{happen} -\xx{var} \·\xx{sub} {}]
		\xx{q}= \xx{foc} -\xx{mdst}
		{}[\pr{PP} \xx{3h·pss} below- near -\xx{all} {}]
		thus= \xx{qual}- \xx{pfv}- \xx{4h·s}- \xx{mid}- \xx{csv}- \xx{stv}- \rt[¹]{say} -\xx{var} //
	\gld	{} how \rlap{\xx{csec}.happen} {} {} \·while {}
		some\· \rlap{it.is} {}
		{} his below- near -to {}
		thus \rlap{\xx{pfv}.someone.say} {} {} {} {} {} {} {} //
	\glft	‘At some point down by him someone said’
		//
\endgl
\xe

\ex\label{ex:90-17-supernatural-help}%
\exmn{262.9}%
\begingl
	\glpreamble	“Iīg̣ā′ xᴀt wūsu′.” //
	\glpreamble	«\!I eeg̱áa x̱at woosoo\!». //
	\gla	{} \llap{«\!}I \rlap{eeg̱áa} @ {} {}
		x̱at @ \rlap{woosoo.} @ {} @ {} @ {} //
	\glb	{} i ee -g̱áa {}
		x̱at= wu- i- \rt[¹]{suʰ} -μμL //
	\glc	{}[\pr{PP} \xx{2sg} \xx{base} -\xx{ades} {}]
		\xx{1sg·o}= \xx{pfv}- \xx{stv}- \rt[¹]{sup·help} -\xx{var} //
	\gld	{} you {} -for {}
		me \rlap{\xx{pfv}.super.help} {} {} {} //
	\glft	‘“I have given supernatural help to you”.’
		//
\endgl
\xe

\ex\label{ex:90-18-unfamiliar}%
\exmn{262.10}%
\begingl
	\glpreamble	 ʟēk! gwâ′yᴀł wudaᴀ′t ᴀt ᴀłg̣ē′n. //
	\glpreamble	Tléikʼ, gwáayá l wudaa át át alg̱ein. //
	\gla	Tléik, \rlap{gwáayá} @ {} @ {}
		{} {} l \rlap{wudaa} @ {} @ {} @ {} @ {} {} át {} +
		{} \rlap{át} @ {} {}
		\rlap{alg̱ein.} @ {} @ {} @ {} @ {} //
	\glb	tléikʼ gwá= á -yá
		{} {} l u- wu- \rt[¹]{da} -μμL {} {} át {}
		{} á -t {} 
		a- d- l- \rt[²]{g̱in} -μμL  //
	\glc	no \xx{mir}= \xx{foc} -\xx{mdst}
		{}[\pr{DP} {}[\pr{CP} \xx{neg} \xx{irr}- \xx{pfv}- \rt[¹]{eyed} -\xx{var} \·\xx{rel} {}] thing {}]
		{}[\pr{PP} \xx{3n} -\xx{pnct} {}]
		\xx{xpl}- \xx{mid}- \xx{xtn}- \rt[²]{look·at} -\xx{var} //
	\gld	no apparently\· it is
		{} {} not \rlap{\xx{pfv}.familiar} {} {} {} \·which {} thing {}
		{} it -at {}
		\rlap{\xx{impfv}.look·at} {} {} {} {} //
	\glft	‘But no, apparently he is looking at things which are not unfamiliar.’
		//
\endgl
\xe

\ex\label{ex:90-19-sleep-kills}%
\exmn{262.10}%
\begingl
	\glpreamble	Wananī′sawe tātc uwadjᴀ′q. //
	\glpreamble	Wáa nanée sáwé táach uwajáḵ. //
	\gla	{} Wáa \rlap{nanée} @ {} @ {} @ {} {}
		\rlap{sáwé} @ {} @ {}
		{} \rlap{táach} @ {} {}
		\rlap{uwajáḵ.} @ {} @ {} @ {} @ {} //
	\glb	{} wáa n- \rt[¹]{ni} -μμH {} {} 
		s= á -wé
		{} tá -ch {}
		ⱥ- u- i- \rt[²]{jaḵ} -μH //
	\glc	{}[\pr{CP} how \xx{ncnj}- \rt[¹]{happen} -\xx{var} \·\xx{sub} {}]
		\xx{q}= \xx{foc} -\xx{mdst}
		{}[\pr{DP} sleep -\xx{erg} {}]
		\xx{arg}- \xx{zpfv}- \xx{stv}- \rt[²]{kill} -\xx{var} //
	\gld	{} how \rlap{\xx{csec}.happen} {} {} \·while {}
		some\· \rlap{it.is} {}
		{} sleep {} {}
		\rlap{3>3.\xx{pfv}.kill} {} {} {} {} //
	\glft	‘At some point he fell asleep.’
		//
\endgl
\xe

The phrase \fm{táach uwajáḵ} ‘sleep killed him/her/it’ as in (\lastx) is a conventional idiom for falling asleep in Tlingit.
The verb \fm{tá} ‘s/he sleeps’ \~\ \fm{wootaa} ‘s/he slept’ describes an activity that takes place over time whereas \fm{táach uwajáḵ} ‘sleep killed him/her/it’ describes an achievement that culminates instantaneously, just like the base verb \fm{aawajáḵ} ‘s/he killed him/her/it’.
Thus the idiom is not only a circumlocution but also provides an event structure organization different from the ordinary verb.
This idiom also reflects the  metaphysical idea that sleep is parallel to death except that \fm{ḵaa toowú} ‘one’s soul/mind’ returns after having left the body \parencites[759]{de-laguna:1972}[56, 105]{kan:2016}.

\ex\label{ex:90-20-starvation}%
\exmn{262.11}%
\begingl
	\glpreamble	Dadukᴀ′nᴀx yᴀx ỵatî′ yū′łāx̣ᵘ. //
	\glpreamble	De du kaanáx̱ yáx̱ ÿatee, yú laaxw. //
	\gla	De {} {} du \rlap{kaanáx̱} @ {} {} yáx̱ {}
		\rlap{ÿatee} @ {} @ {}
		{} yú \rlap{laaxw.} @ {} {} //
	\glb	de {} {} du ká -náx̱ {} yáx̱ {}
		i- \rt[¹]{tiʰ} -μμL
		{} yú \rt[¹]{laxw} -μμL {} //
	\glc	de {}[\pr{PP} {}[\pr{PP} \xx{3h·pss} \xx{hsfc} -\xx{perl} {}] \xx{sim} {}]
		\xx{stv}- \rt[¹]{be} -\xx{var}
		{}[\pr{DP} \xx{dist} \rt[¹]{starve} -\xx{var} {}] //
	\gld	already {} {} his \rlap{overcome} {} {} like {}
		\rlap{\xx{impfv}.be} {} {}
		{} that \rlap{starvation} {} {} //
	\glft	‘Already he is like he is overcome, the starvation.’
		//
\endgl
\xe

The phrase \fm{du kaanáx̱} in (\lastx) is apparently an obscure and poorly documented idiom.
It is attested elsewhere by the phrase \fm{du kaanáx̱ at wootee} ‘he is exhausted’ \parencite[f06/2]{leer:1973} which is more literally ‘something became along over him’.
Tlingit normally prohibits the stacking of postpositions, so the syntactic context of \fm{du kaanáx̱} in (\lastx) is unusual because it is a PP but it occurs within another PP headed by \fm{yáx̱} ‘like, similar’.
This syntactic irregularity emphasises the idomaticity of the phrase because it means that \fm{du kaanáx̱} behaves like an unanalyzed chunk rather than a syntactically active PP.
Syntactically it may be more like an adverb since there are other examples of adverb + \fm{yáx̱}, including adverbs that have a frozen \fm{-náx̱} like \fm{kagéináx̱} ‘slowly, carefully, cautiously’.

\ex\label{ex:90-21-watch-for}%
\exmn{262.11}%
\begingl
%	\glpreamble	Wananī′sawe yên aỵā′wayᴀk. //
	\glpreamble	Wáa nanée sáwé yan aÿaawayék. //
	\gla	{} Wáa \rlap{nanée} @ {} @ {} @ {} {}
		\rlap{sáwé} @ {} @ {}
		yan @ \rlap{aÿaawayék.} @ {} @ {} @ {} @ {} @ {} //
	\glb	{} wáa n- \rt[¹]{ni} -μμH {} {} 
		s= á -wé
		ÿán= a- ÿ- wu- i- \rt[²]{yek} -μH //
	\glc	{}[\pr{CP} how \xx{ncnj}- \rt[¹]{happen} -\xx{var} \·\xx{sub} {}]
		\xx{q}= \xx{foc} -\xx{mdst}
		\xx{term}= \xx{arg}- \xx{qual}- \xx{pfv}- \xx{stv}- \rt[²]{watch·for} -\xx{var} //
	\gld	{} how \rlap{\xx{csec}.happen} {} {} \·while {}
		some\· \rlap{it.is} {}
		finally\· \rlap{3>3.\xx{pfv}.watch·for} {} {} {} {} {}  //
	\glft	‘At some point he finally watched for it.’
		//
\endgl
\xe

%This sentence provides an excellent example of both a topic phrase \fm{yú yéi ash daaÿaḵá át} ‘that thing speaking to him’ and a focus phrase \fm{yéi kwligeiyi át áwé} ‘it’s a small thing’ with an unergative intransitive  \fm{du ÿeex̱ánt wujixeex} ‘(which) was running around down by him’.
Both the topic and focus are relative clauses headed by \fm{át} ‘thing’, and both are coreferential with the subject of the verb.

\ex\label{ex:90-22-talking-running-around}%
\exmn{263.1}%
\begingl
	\glpreamble	Yū′ye-ᴀct-āỵaqa-ᴀt yekułigā′ỵi-ᴀtawe′ duyāxᴀ′nt wudjîx̣ī′x̣. //
	\glpreamble	Yú yéi ash daaÿaḵá át, yéi kuligéiÿi át áwé du yeex̱ánt wujixeex. //
	\gla	{} Yú {} yéi @ ash @ \rlap{daaÿaḵá} @ {} @ {} @ {} @ {} {} át, {} +
		{} {} yéi @ \rlap{kuligéiyi} @ {} @ {} @ {} @ {} @ {} @ {} {} át {}
		\rlap{áwé} @ {} +
		{} du \rlap{ÿeex̱ánt} @ {} @ {} {}
		\rlap{wujixeex.} @ {} @ {} @ {} @ {} @ {} //
	\glb	{} yú {} yéi= ash= daa- ÿ- \rt[²]{ḵa} -μH {} {} át {}
		{} {} yéi= k- u- l- i- \rt[¹]{ge} -μμH -i {} át {}
		á -wé
		{} du ÿee- x̱án -t {}
		wu- d- sh- i- \rt[¹]{xix} -μμL //
	\glc	{}[\pr{DP} \xx{dist} {}[\pr{CP} thus= \xx{3prx·o}= around- \xx{qual}- \rt[²]{say} -\xx{var} \·\xx{rel} {}]
			thing {}]
		{}[\pr{DP} {}[\pr{CP} thus= \xx{cmpv}- \xx{irr}- \xx{xtn}- \xx{stv}- \rt[¹]{big} -\xx{var} -\xx{rel} {}]
			thing {}]
		\xx{foc} -\xx{mdst}
		{}[\pr{PP} \xx{3h·pss} below- near -\xx{pnct} {}]
		\xx{pfv}- \xx{mid}- \xx{pej}- \xx{stv}- \rt[¹]{fall} -\xx{var} //
	\gld	{} that {} thus him \rlap{around.\xx{impfv}.say} {} {} {} \·that {} thing {}
		{} {} thus \rlap{\xx{cmpv}.be.big} {} {} {} {} {} -that {} thing {}
		\rlap{it.is} {}
		{} his below- near -around {}
		\rlap{\xx{pfv}.run·\xx{sg}} {} {} {} {} {} //
	\glft	‘That thing talking to him, it is a small thing which was running around down by him.’
		//
\endgl
\xe

The sentence in (\ref{ex:90-22-talking-running-around}) provides an excellent example of a topic and a focus in the same sentence.
The topic is the relative clause \fm{yú yéi ash daaÿaḵá át} ‘that thing talking to him’ which appears at the beginning of the sentence.
The focus is the relative clause \fm{yéi kwligéiÿi át} ‘a thing which is moderately large’ followed by the focus particle \fm{áwé}.
The topic and focus refer to the same entity in the discourse, namely the supernatural being that is described in the following sentences, and both are coreferential with the subject of the main verb \fm{wujixeex} ‘it ran’.
But the topic and focus are different constituents with different semantic interpretations.
Notably the topic contains a transitive verb where the subject is the ultimate referent but the proximate object pronominal refers to the protagonist, where in contrast the focus contains an intransitive verb describing only the supernatural helper.

\ex\label{ex:90-23-tooth-long}%
\exmn{263.1}%
\begingl
	\glpreamble	Duū′x yēkdiyā′t!. //
	\glpreamble	Du oox̱ yéi kwdiyáatʼ. //
	\gla	{} Du oox̱ {}
		yéi @ \rlap{kwdiyáatʼ.} @ {} @ {} @ {} @ {} @ {} //
	\glb	{} du oox̱ {}
		yéi= k- w- d- i- \rt[¹]{ÿatʼ} -μμH //
	\glc	{}[\pr{DP} \xx{3h·pss} tooth {}]
		thus= \xx{cmpv}- \xx{irr}- \xx{mid}- \xx{stv}- \rt[¹]{long} -\xx{var} //
	\gld	{} his tooth {}
		thus \rlap{\xx{cmpv}.\xx{pl}.be.long} {} {} {} {} {} //
	\glft	‘His teeth were so long.’
		//
\endgl
\xe

The sentence in (\lastx) has two phenomena that deserve comment.
One is that the pronoun \fm{du} ‘his/hers’ refers specifically to the supernatural helper introduced first in (\ref{ex:90-16-something-says-below}) and elaborated on below.
This is significant because it means that the narrator intends for the supernatural helper to be viewed as human or human-like since \fm{du} is normally used only for human entities.
The other phenomenon is that the verb contains \fm{d-}.
State verbs that describe dimensions (length, weight, size, etc.)\ feature an unusual alternation involving the \fm{d-} prefix for middle voice.
Specifically, when a dimension state verb refers to a plural entity the \fm{d-} prefix appears in the verb form and the plural suffix \fm{-xʼ} may also appear on the end of the verb form.
Here in (\lastx) the verb with \fm{d-} indicates that the phrase \fm{du oox̱} must be plural ‘his teeth’ and not singular ‘his tooth’.

\ex\label{ex:90-24-grab-up-from-there}%
\exmn{263.2}%
\begingl
	\glpreamble	Tc!uʟe′ ax ā′wacāt. //
	\glpreamble	Chʼu tle aax̱ aawasháat. //
	\gla	Chʼu tle {} \rlap{aax̱} @ {} {}
		\rlap{aawasháat.} @ {} @ {} @ {} @ {} //
	\glb	chʼu tle {} á -dáx̱ {}
		a- wu- i- \rt[²]{shaʼt} -μμH //
	\glc	just then {}[\pr{PP} \xx{3n} -\xx{abl} {}]
		\xx{arg}- \xx{pfv}- \xx{stv}- \rt[²]{grab} -\xx{var} //
	\gld	just then {} it -from {}
		\rlap{3>3.\xx{pfv}.grab} {} {} {} {} //
	\glft	‘Just then he grabbed it up from there.’
		//
\endgl
\xe

\ex\label{ex:90-25-grab-up-from-there}%
\exmn{263.2}%
\begingl
	\glpreamble	Doᴀts!ī′ʟ! tū′q!awe aỵā′wacᴀt. //
	\glpreamble	Du at.sʼéili tóoxʼ áwé aÿaawashát. //
	\gla	{} Du {} \rlap{at.sʼéilʼi} @ {} @ {} @ {} {} \rlap{tóoxʼ} @ {} {}
		\rlap{áwé} @ {}
		\rlap{aÿaawashát.} @ {} @ {} @ {} @ {} @ {} //
	\glb	{} du {} at= \rt[²]{sʼelʼ} -μμH -í {} tú -xʼ {}
		á -wé
		a- ÿ- wu- i- \rt[²]{shaʼt} -μH //
	\glc	{}[\pr{PP} \xx{3h·pss} {}[\pr{NP} \xx{4n·o}= \rt[²]{tear} -\xx{var} -\xx{nmz} {}] inside -\xx{loc} {}]
		\xx{foc} -\xx{mdst}
		\xx{arg}- \xx{qual}- \xx{pfv}- \xx{stv}- \rt[²]{grab} -\xx{var} //
	\gld	{} his {} \rlap{rags} {} {} {} {} inside -at {}
		\rlap{it.is} {}
		\rlap{3>3.\xx{pfv}.capture} {} {} {} {} {} //
	\glft	‘It was within his rags that he captured it.’
		//
\endgl
\xe

The sentence in (\ref{ex:90-25-grab-up-from-there}) notably has \fm{du} ‘his/hers’ now referring to the protagonist (who is also the subject) rather than the supernatural helper as in (\ref{ex:90-23-tooth-long}) previously.
This does not mean that the viewpoint has shifted or that the status of foreground versus background has changed because the proximate \fm{ash=} ‘him/her’ in (\ref{ex:90-27-put-me-in-water}) once again refers to the protagonist and this same character is also the experiencer in (\ref{ex:90-26-fell-asleep}).
The \fm{du} pronoun is actually neutral with regard to foreground or background status, so that its use in (\ref{ex:90-23-tooth-long}) and (\ref{ex:90-25-grab-up-from-there}) is compatible with both the status of both characters.

\ex\label{ex:90-26-fell-asleep}%
\exmn{263.3}%
\begingl
	\glpreamble	ʟe tātc uwadjᴀ′q. //
	\glpreamble	Tle táach uwajáḵ. //
	\gla	Tle {} \rlap{táach} @ {} {}
		\rlap{uwajáḵ.} @ {} @ {} @ {} @ {} //
	\glb	tle {} tá -ch {}
		ⱥ- u- i- \rt[²]{jaḵ} -μH //
	\glc	just {}[\pr{DP} sleep -\xx{erg} {}]
		\xx{arg}- \xx{zpfv}- \xx{stv}- \rt[²]{kill} -\xx{var} //
	\gld	just {} sleep {} {}
		\rlap{3>3.\xx{pfv}.kill} {} {} {} {} //
	\glft	‘Then he fell asleep.’
		//
\endgl
\xe

\ex\label{ex:90-27-put-me-in-water}%
\exmn{263.3}%
\begingl
	\glpreamble	Ye adjū′n, “Hīnq! q!wan yên xᴀt cᴀt,” yu-act-ā′ỵaqa. //
	\glpreamble	Yéi ajóon «\!Héenxʼ xʼwán yan x̱at shát\!» yóo ash daaÿaḵá. //
	\gla	Yéi @ \rlap{ajóon} @ {} @ {}
		{} {} {} \llap{«\!}\rlap{Héenxʼ} @ {} {} 
					xʼwán
					yan @ x̱at @ \rlap{shát\!»} @ {} @ {} @ {} {} +
			yóo @ ash @ \rlap{daaÿaḵá.} @ {} @ {} @ {} {} //
	\glb	yéi= a- \rt[²]{jun} -μμH
		{} {} {} héen -xʼ {}
					xʼwán
					ÿán= x̱at= {} {} \rt[²]{shaʼt} -μH {}
			yóo= ash= daa- ÿ- \rt[²]{ḵa} -μH {} //
	\glc	thus= \xx{arg}- \rt[²]{dream} -\xx{var}
		{}[\pr{CP} {}[\pr{CP} {}[\pr{PP} water -\xx{loc} {}]
					\xx{imp}
					\xx{term}= \xx{1sg·o}= \xx{zcnj}\· \xx{2sg·s}\· \rt[²]{grab} -\xx{var} {}]
			\xx{quot}= \xx{3prx·o}= about- \xx{qual}- \rt[²]{say} -\xx{var} {}] //
	\gld	thus \rlap{3>3.\xx{impfv}.dream} {} {}
		{} {} {} water -in {}
					\xx{imp}
					down me \rlap{\xx{imp}.you.grab} {} {} {} {} 
			thus him \rlap{about.\xx{impfv}.say} {} {} {} {} //
	\glft	‘He dreams that “Put me down in the water” it says to him.’
		//
\endgl
\xe

The initial phrase \fm{yéi ajóon} ‘s/he dreams thus’ in (\lastx) could be taken by its position to be either an unmarked adjunct of the main clause or the main clause itself.
The analysis given in (\lastx) treats it as the main clause.
If it were analyzed as an unmarked adjunct then the translation would be something like ‘While he is dreaming, “Put me down in the water” it says to him’.
The form \fm{yéi ajóon} could be mistaken for a consecutive aspect form, but the activity verb \fm{ajóon} ‘s/he dreams’ is \fm{n}-conjugation so the consecutive form would have to be \fm{yéi anajóon} ‘him having dreamed’.

The quoted speech \fm{héenxʼ xʼwán yan x̱at shát} ‘put me down in the water’ in (\lastx) is an imperative form of the verb \fm{aawasháat} ‘s/he grabbed, caught him/her/it’ which is a \fm{g}-conjugation achievement verb.
The imperative would ordinarily be expected to have the \fm{g-} conjugation prefix, but it is absent in (\lastx) because it has been superseded by a motion derivation.
The specific motion derivation can be detected by the absence of an overt conjugation prefix (i.e.\ \fm{∅}-conjugation) and the presence of the preverb \fm{ÿan=} ‘ashore; end, terminate’.
This is the motion derivation \vblex{ÿán= \~\ ÿáx̱= \~\ ÿánde=}{∅}{\fm{-μμL} repetitive}{ashore; finishing, ending, terminating}.
This motion derivation also has the idiomatic interpretation of ‘down’ with handling verbs because people often put things down after handling them.
The verb in (\lastx) then literally means something like ‘hold me down’ or ‘grab me down’, but \citeauthor{swanton:1909}’s translation “Put me into the water” and gloss “in the water there me put” and the continuing narrative suggest that the intended meaning is more ‘put’ than ‘hold’ or ‘grab’.

\ex\label{ex:90-28-dawning-did-so}%
\exmn{263.4}%
\begingl
	\glpreamble	Ỵaqēnaē′nî awe′ ayᴀ′x a′osîne. //
	\glpreamble	Ÿaa ḵeina.éini áwé a yáx̱ awsinei. //
	\gla	{} Ÿaa @ \rlap{ḵeina.éini} @ {} @ {} @ {} @ {} @ {} {}
		\rlap{áwé} @ {}
		{} a yáx̱ {}
		\rlap{awsinei.} @ {} @ {} @ {} @ {} @ {} //
	\glb	{} ÿaa= ḵee- n- \rt[¹]{.a} -eμH -n -í {}
		á -wé
		{} a yáx̱ {}
		a- wu- s- i- \rt[¹]{niʰ} -μμL //
	\glc	{}[\pr{CP} along= dawn- \xx{ncnj}- \rt[¹]{extend} -\xx{var} -\xx{ncnj} -\xx{sub} {}]
		\xx{foc} -\xx{mdst}
		{}[\pr{PP} \xx{3n} \xx{sim} {}]
		\xx{arg}- \xx{pfv}- \xx{csv}- \xx{stv}- \rt[¹]{happen} -\xx{var} //
	\gld	{} along \rlap{dawn.\xx{prog}.extend} {} {} {} {} -while {}
		\rlap{it.is} {}
		{} it like {}
		\rlap{3>3.\xx{pfv}.make.happen} {} {} {} {} {} //
	\glft	‘It was as dawn came that he did so.’
		//
\endgl
\xe

\ex\label{ex:90-29-down-to-beach}%
\exmn{263.4}%
\begingl
	\glpreamble	Eg̣e ān ỵēq uwagu′t. //
	\glpreamble	Éig̱i aan ÿeiḵ uwagút. //
	\gla	{} \rlap{Éig̱i} @ {} {}
		{} \rlap{aan} @ {} {}
		ÿeiḵ @ \rlap{uwagút.} @ {} @ {} @ {} //
	\glb	{} éeḵ -í {}
		{} á -n {}
		ÿeeḵ= u- i- \rt[¹]{gut} -μH //
	\glc	{}[\pr{PP} beach -\xx{loc} {}]
		{}[\pr{PP} \xx{3n} -\xx{instr} {}]
		beach= \xx{zpfv}- \xx{stv}- \rt[¹]{go·\xx{sg}} -\xx{var} //
	\gld	{} beach -to {}
		{} it -with {}
		beach \rlap{\xx{pfv}.go·\xx{sg}} {} {} {} //
	\glft	‘He went down to the beach with it.’
		//
\endgl
\xe

\ex\label{ex:90-30-bounce-in-hand}%
\exmn{263.5}%
\begingl
	\glpreamble	Dudjī′n tā′q!awe ke āxê′tctc. //
	\glpreamble	Du jintáaxʼ áwé kei ax̱échch. //
	\gla	{} Du \rlap{jíntáaxʼ} @ {} @ {} {}
		\rlap{áwé} @ {}
		kei @ \rlap{ax̱échch.} @ {} @ {} @ {} //
	\glb	{} du jín- táaᵏ -xʼ {}
		á -wé
		kei= a- \rt[²]{x̱ich} -μH -ch //
	\glc	{}[\pr{PP} \xx{3h·pss} hand- bottom -\xx{loc} {}]
		\xx{foc} -\xx{mdst}
		up= \xx{arg}- \rt[²]{throw·anim} -\xx{var} -\xx{rep} //
	\gld	{} his hand- palm -in {}
		\rlap{it.is} {}
		up \rlap{3>3.\xx{impfv}.throw·anim.\xx{rep}} {} {} {} //
	\glft	‘He bounces it in his palm.’
		//
\endgl
\xe

The verb in (\lastx) is based on the root \fm{\rt[²]{x̱ich}} which has a few different interpretations.
Here the meaning is presumably the common ‘throw animate’ since the supernatural helper is an animate being.
A potential alternative is ‘throw filled container’ if the supernatural helper is actually contained in some other object, but no mention has been made of a container so this is less obvious.
The third alternative meaning ‘club, beat, spank’ is probably inapplicable in this context given \citeauthor{swanton:1909}’s translation “He kept throwing it up and down in his hands”.
The “up and down” suggests that the motion described is not quite the same as English ‘throw’ in this context, but instead more like ‘bounce’, perhaps reflecting the protagonist testing the weight in his hand before throwing a long distance.

\ex\label{ex:90-31-came-help-saying-threw}%
\exmn{263.5}%
\begingl
	\glpreamble	“Xāg̣a′ iwasu′,” yē′aỵênᴀsqā′awe hīn nᴀx āwaxē′tc. //
	\glpreamble	«\!X̱aag̱áa eewasoo\!» yéi aÿanasḵáa áwé héennáx̱ aawax̱eich. //
	\gla	{} {} {} \llap{«\!}\rlap{X̱aag̱áa} @ {} {}
				\rlap{eewasoo\!»} @ {} @ {} @ {} @ {} {} +
			yéi @ \rlap{aÿanasḵáa} @ {} @ {} @ {} @ {} @ {} @ {} {}
		\rlap{áwé} @ {} +
		{} \rlap{héennáx̱} @ {} {}
		\rlap{aawax̱eich.} @ {} @ {} @ {} @ {} //
	\glb	{} {} {} x̱á -g̱áa {}
				i- wu- i- \rt[¹]{suʰ} -μμL {}
			yéi= a- ÿ- n- s- \rt[¹]{ḵa} -μμH {} {}
		á -wé
		{} héen -náx̱ {}
		a- wu- i- \rt[²]{x̱ich} -μμL //
	\glc	{}[\pr{CP} {}[\pr{CP} {}[\pr{PP} \xx{1sg} -\xx{ades} {}]
				\xx{2sg·o}- \xx{pfv}- \xx{stv}- \rt[¹]{sup·help} -\xx{var} {}]
			thus= \xx{arg}- \xx{qual}- \xx{ncnj}- \xx{csv}- \rt[¹]{say} -\xx{var} \·\xx{sub} {}]
		\xx{foc} -\xx{mdst}
		{}[\pr{PP} water -\xx{perl} {}]
		\xx{arg}- \xx{pfv}- \xx{stv}- \rt[²]{throw·anim} -\xx{var} //
	\gld	{} {} {} me -for {}
				\rlap{you.\xx{pfv}.super·help} {} {} {} {} {}
			thus \rlap{3>3.\xx{csec}.say} {} {} {} {} {} \·when {}
		\rlap{it.is} {}
		{} water -thru {}
		\rlap{3>3.\xx{pfv}.throw·anim} {} {} {} {} //
	\glft	‘Having said “You gave supernatural help to me”, he threw it into the water.’
		//
\endgl
\xe

\ex\label{ex:90-32-smoke-on-the-water}%
\exmn{263.6}%
\begingl
	\glpreamble	Wuduwas!ᴀ′qawe ᴀt āwaxê′tciỵa. //
	\glpreamble	Wuduwasʼéḵ áwé át aawax̱éji ÿé. //
	\gla	\rlap{Wuduwasʼéḵ} @ {} @ {} @ {} @ {}
		\rlap{áwé} @ {} +
		{} {} {} \rlap{át} @ {} {}
			\rlap{aawax̱éji} @ {} @ {} @ {} @ {} @ {} {} ÿé {} //
	\glb	wu- du- i- \rt[¹]{sʼiḵ} -μH
		á -wé
		{} {} {} á -t {}
			a- wu- i- \rt[²]{x̱ich} -μH -i {} ÿé {} //
	\glc	\xx{pfv}- \xx{xpl}- \xx{stv}- \rt[¹]{smoke} -\xx{var}
		\xx{foc} -\xx{mdst}
		{}[\pr{NP} {}[\pr{CP} {}[\pr{PP} \xx{3n} -\xx{pnct} {}]
			\xx{arg}- \xx{pfv}- \xx{stv}- \rt[²]{throw·anim} -\xx{var} -\xx{rel} {}] place {}] //
	\gld	\rlap{\xx{pfv}.smoke} {} {} {} {}
		\rlap{it.is} {}
		{} {} {} there -at {}
			\rlap{3>3.\xx{pfv}.throw·\xx{anim}} {} {} {} {} -where {} place {} //
	\glft	‘It became smoky, the place where he threw it to.’
		//
\endgl
\xe

The sentence in (\lastx) is syntactically unusual.
The verb \fm{wuduwasʼéḵ} ‘it became smoky’ is normally impersonal like a weather verb and so takes neither an object nor a subject; the \fm{du-} is a meaningless dummy pronoun to satisfy the monovalency of the verb root \fm{\rt[¹]{sʼiḵ}} ‘smoke’.
The relative clause and noun phrase \fm{át aawax̱éji ÿé} ‘place where he threw it to’ therefore cannot be the argument of the verb because the verb cannot take a subject or object.
There is a related \fm{wudisʼíḵ} ‘it (thing) smoked’ which takes an object  \parencite[524]{leer:1976}, but \citeauthor{swanton:1909}’s transcription \orth{Wuduwas!ᴀ′q} is unambiguous.

\ex\label{ex:90-33-dusk-wrap-head}%
\exmn{263.7}%
\begingl
	\glpreamble	ʟe yê′ndiỵax x̣înᴀ′-ᴀ′tî awe′ cana′odîs!ît. //
	\glpreamble	Tle yánde ÿaa xeena.ádi áwé shanáa awdisʼít. //
	\gla	{} Tle {} \rlap{yánde} @ {} {}
			ÿaa @ \rlap{xeena.ádi} @ {} @ {} @ {} @ {} {}
		\rlap{áwé} @ {} +
		{} {} \rlap{shanáa} @ {} {}
		\rlap{awdisʼít.} @ {} @ {} @ {} @ {} @ {} //
	\glb	{} tle {} ÿán -dé {}
			ÿaa= xee- n- \rt[¹]{.at} -μH -í {}
		á -wé
		{} {} shá- náa {}
		a- wu- d- i- \rt[²]{sʼit} -μH //
	\glc	{}[\pr{CP} then {}[\pr{PP} \xx{term} -\xx{all} {}]
			along= dusk- \xx{ncnj}- \rt[¹]{go·\xx{pl}} -\xx{var} -\xx{sub} {}]
		\xx{foc} -\xx{mdst}
		{}[\pr{NP} \xx{rflx·pss} head- cover {}]
		\xx{arg}- \xx{pfv}- \xx{mid}- \xx{stv}- \rt[²]{wrap} -\xx{var} //
	\gld	{} then {} done -to {}
			along \rlap{dusk.\xx{prog}.go·\xx{pl}} {} {} {} -while {}
		\rlap{it.is} {}
		{} self’s head- cover {}
		\rlap{3>3.\xx{pfv}.wrap} {} {} {} {} {} //
	\glft	‘Then it was as it was getting toward dusk that he wrapped up his head.’
		//
\endgl
\xe

\ex\label{ex:90-34-dawn-hear-raven-beach}%
\exmn{263.7}%
\begingl
	\glpreamble	Tc!uʟe′ aqē′naē′nî tî′nawe ā′waᴀx yēł sa duīg̣aya′dê. //
	\glpreamble	Chʼu tle ÿaa ḵeina.éini tin áwé aawa.áx̱ yéil sé du eeg̱ayáade. //
	\gla	{} Chʼu tle
			{} ÿaa @ \rlap{ḵeina.éini} @ {} @ {} @ {} @ {} @ {} {}
			tin {}
		\rlap{áwé} @ {}
		\rlap{aawa.áx̱} @ {} @ {} @ {} @ {}
		{} yéil sé {}
		{} du \rlap{eeg̱ayáade.} @ {} @ {} {} //
	\glb	{} chʼu tle
			{} ÿaa= ḵei- n- \rt[¹]{.a} -eμH -n -í {}
			tin {}
		á -wé
		a- wu- i- \rt[²]{.ax̱} -μH
		{} yéil sé {}
		{} du eeḵ- ÿáᵏ -de {} //
	\glc	{}[\pr{PP} just then
			{}[\pr{CP} along= dawn- \xx{ncnj}- \rt[¹]{extend} -\xx{var} -\xx{nsfx} -\xx{sub} {}]
			\xx{instr} {}]
		\xx{foc} -\xx{mdst}
		\xx{arg}- \xx{pfv}- \xx{stv}- \rt[²]{hear} -\xx{var}
		{}[\pr{DP} raven voice:\xx{inal} {}]
		{}[\pr{PP} \xx{3h·pss} beach- face -\xx{all} {}] //
	\gld	{} just then
			{} along \rlap{dawn.\xx{prog}.extend} {} {} {} {} -ing {}
			with {}
		\rlap{it.is} {}
		\rlap{3>3.\xx{pfv}.hear} {} {} {} {}
		{} raven voice {}
		{} his beach- face -to {} //
	\glft	‘It was just with the dawning that he heard it, a raven’s voice toward the beach below him.’
		//
\endgl
\xe

\ex\label{ex:90-35-halibut-pushed-ashore}%
\exmn{263.8}%
\begingl
	\glpreamble	Tcāʟ gwâ′ỵa yā′nᴀx yên a′sîguq. //
	\glpreamble	Cháatl gwáaÿá yáanax̱ yan awsigúḵ. //
	\gla	{} Cháatl {}
		\rlap{gwáaÿá} @ {} @ {}
		{} \rlap{yáanáx̱} @ {} {}
		yan @ \rlap{awsigúḵ.} @ {} @ {} @ {} @ {} @ {} //
	\glb	{} cháatl {}
		gwá= á -ÿá
		{} yá -náx̱ {}
		ÿán= a- wu- s- i- \rt[²]{guḵ} -μH //
	\glc	{}[\pr{DP} halibut {}]
		\xx{mir}= \xx{foc} -\xx{prox}
		{}[\pr{PP} \xx{prox} -\xx{perl} {}]
		\xx{term}= \xx{arg}- \xx{pfv}- \xx{xtn}- \xx{stv}- \rt[²]{push} -\xx{var} //
	\gld	{} halibut {}
		apparently\• \xx{it.is} {}
		{} here -along {}
		ashore \rlap{3>3.\xx{pfv}.push} {} {} {} {} {}  //
	\glft	‘It was apparently a halibut that it had pushed ashore nearby.’
		//
\endgl
\xe

\ex\label{ex:90-36-hang-on-flab-helper}%
\exmn{263.9}%
\begingl
	\glpreamble	ᴀtē′s!î kade′q! aỵax̣ᴀ′t yū′-ᴀcīg̣ā′-wusū′wu-ᴀt. //
	\glpreamble	A téisʼi kaadé x̱ʼaÿaxát yú ash eeg̱áa woosoowu át. //
	\gla	{} A \rlap{téisʼi} @ {} \rlap{kaadé} @ {} {}
		\rlap{x̱ʼaÿaxát} @ {} @ {} @ {} +
		{} yú {} {} ash \rlap{eeg̱áa} @ {} {}
			\rlap{woosoowu} @ {} @ {} @ {} @ {} {} át. {} //
	\glb	{} a téisʼ -í ká -dé {}
		x̱ʼe- i- \rt[¹]{xat} -μH
		{} yú {} {} ash ee -g̱áa {}
			wu- i- \rt[¹]{suʰ} -μμL -i {} át {} //
	\glc	{}[\pr{PP} \xx{3n·pss} flab -\xx{pss} \xx{hsfc} -\xx{all} {}]
		mouth- \xx{stv}- \rt[¹]{suspend} -\xx{var}
		{}[\pr{DP} \xx{dist} {}[\pr{CP} {}[\pr{PP} \xx{3prx} \xx{base} -\xx{ades} {}]
			\xx{pfv}- \xx{stv}- \rt[¹]{sup·help} -\xx{var} -\xx{rel} {}] thing {}] //
	\gld	{} its flab {} atop -to {}
		\rlap{\xx{impfv}.be.fastened} {} {} {}
		{} that {} {} him {} -for {}
			\rlap{\xx{pfv}.super·help} {} {} {} -that {} thing {} //
	\glft	‘Hanging on its flab was that thing that gave supernatural help to him.’
		//
\endgl
\xe

\citeauthor{swanton:1909}’s gloss of \orth{ᴀtē′s!î} in (\lastx) is “Its heart”, and thus his translation of (\lastx) is “the thing that was helping him was at its heart”.
The transcription suggests \fm{téisʼ} ‘flab, limp flesh’, but it is possible that \citeauthor{swanton:1909}’s \orth{ᴀtē′s!î} is a misreading of e.g.\ \orth{ᴀtē′x!î}.
If this is the case then the noun is actually \fm{téix̱ʼ} ‘heart’ as implied by his gloss and translation.
To verify this error we need to review \citeauthor{swanton:1909}’s manuscript transcription; the change from \orth{x} or \orth{x̣} to \orth{s} is plausibly due to either his misreading in preparing a manuscript copy for printing, or to the printer’s misreading of his manuscript copy.

\section{Paragraph 3}\label{sec:90-para-3}

\ex\label{ex:90-37-work-like-house}%
\exmn{263.10}%
\begingl
	\glpreamble	Ts!ayu′k!awe hît yᴀx dji′wᴀne. //
	\glpreamble	Tsʼayóokʼ áwé hít yáx̱ jeewanei. //
	\gla	\rlap{Tsʼayóokʼ} @ {} @ {} \rlap{áwé} @ {} 
		{} hít yáx̱ {}
		\rlap{jeewanei.} @ {} @ {} @ {} @ {} //
	\glb	tsʼa= yú -kʼ á -wé
		{} hít yáx̱ {}
		ji- wu- i- \rt[²]{niʰ} -μμL //
	\glc	just= \xx{dist} -\xx{loc} \xx{foc} -\xx{mdst}
		{}[\pr{PP} house \xx{sim} {}]
		hand- \xx{pfv}- \xx{stv}- \rt[¹]{work} -\xx{var} //
	\gld	\rlap{immediately} {} {} \rlap{it.is} {}
		{} house like {}
		\rlap{\xx{pfv}.work} {} {} {} {} //
	\glft	‘Immediately he worked up something like a house.’
		//
\endgl
\xe

\FIXME{Discuss analysis of \fm{tsʼayóokʼ} and its relationship with \fm{chʼa}.}

\ex\label{ex:90-38-built-big-one}%
\exmn{263.10}%
\begingl
	\glpreamble	Aʟē′n aoliyᴀ′x. //
	\glpreamble	Aatlein awliyéx̱. //
	\gla	\rlap{Aatlein} @ {}
		\rlap{awliyéx̱.} @ {} @ {} @ {} @ {} @ {} //
	\glb	aa =tlein
		a- wu- l- i- \rt[²]{yex̱} -μH //
	\glc	\xx{part} =big
		\xx{arg}- \xx{pfv}- \xx{xtn}- \xx{stv}- \rt[²]{make} -\xx{var} //
	\gld	\rlap{big.one} {}
		\rlap{3>3.\xx{pfv}.build} {} {} {} {} {} //
	\glft	‘He built a big one.’
		//
\endgl
\xe

The word \fm{aatlein} in (\lastx) is usually an adverb or adjective that translates to ‘much, lots’.
Here it instead seems to be describing size instead of quantity.
The quantity interpretation makes no sense in the continuing narrative context because there is only a single house.
The syntactic category of \fm{aatlein} is also at issue; usually this word is a modifier but here it seems to be an argument.
An alternative analysis is that \fm{aatlein} is still a modifier, but that it is modifying a covert noun.
It is not yet known if it is possible to elide a modified noun; compare English \fm[*]{I ate a big} versus \fm{I ate a big one}.
One other possible analysis is that the verb word is actually \fm{aa wliyéx̱} where \fm{aa} is the partitive object pronoun ‘one, some’, and then \fm{aatlein} is a modifier of this pronoun.
Once again however, it is not known if pronouns can be modified like nouns.

\ex\label{ex:90-39-morning-beach-with-it}%
\exmn{263.10}%
\begingl
	\glpreamble	Ts!utā′dawe ēq an ỵēq uwagu′t, //
	\glpreamble	Tsʼootaat áwé éiḵ aan ÿeiḵ uwagút. //
	\gla	{} Tsʼootaat {} \rlap{áwé} @ {} 
		{} éiḵ {} 
		{} \rlap{aan} @ {} {}
		ÿeiḵ @ \rlap{uwagút.} @ {} @ {} @ {} //
	\glb	{} tsʼootaat {} á -wé
		{} éeḵ {}
		{} á -n {}
		ÿeeḵ= u- i- \rt[¹]{gut} -μH //
	\glc	{}[\pr{NP} morning {}] \xx{foc} -\xx{mdst}
		{}[\pr{NP} beach {}]
		{}[\pr{PP} \xx{3n} -\xx{instr} {}]
		beach= \xx{zpfv}- \xx{stv}- \rt[¹]{go·\xx{sg}} -\xx{var} //
	\gld	{} morning {} \rlap{it.is} {}
		{} beach {}
		{} it -with {}
		beach \rlap{\xx{pfv}.go·\xx{sg}} {} {} {} //
	\glft	‘Then in the morning he went down to the beach with it.’
		//
\endgl
\xe

\ex\label{ex:90-40-let-it-go-again}%
\exmn{263.10}%
\begingl
	\glpreamble	ᴀdjiwanᴀ′q ts!u. //
	\glpreamble	Ajeewanáḵ tsu. //
	\gla	\rlap{Ajeewanáḵ} @ {} @ {} @ {} @ {} @ {} tsu. //
	\glb	a- ji- wu- i- \rt[²]{naḵ} -μH tsu //
	\glc	\xx{arg}- hand- \xx{pfv}- \xx{stv}- \rt[²]{abandon} -\xx{var} again //
	\gld	\rlap{3>3.\xx{pfv}.release} {} {} {} {} {} again //
	\glft	‘He released it again.’
		//
\endgl
\xe

\citeauthor{swanton:1909} glosses the final \orth{ts!u} in (\lastx) as “also”.
This would be the focus particle \fm{tsú} ‘also, too, in addition’, but it does not make sense in this context.
Instead the word is probably \fm{tsu} ‘again, once more; still, yet’ with low tone rather than high tone.
The syntactic distribution of the two particles is different: \fm{tsu} ‘again’ can occur either before or after the phrase that it modifies, but \fm{tsú} can only occur after the phrase it modifies and usually in the second position of the sentence since it is a focus particle.

\ex\label{ex:90-41-again-raven-voice}%
\exmn{263.11}%
\begingl
	\glpreamble	Āx ỵaqē′ga a′awe ts!u eq de wudū′waᴀx yēł sa. //
	\glpreamble	Aax̱ ÿaa ḵeiga.áa áwé tsu éiḵde wuduwa.áx̱ yéil sé. //
	\gla	{} \rlap{Aax̱} @ {} {}
		{} ÿaa @ \rlap{ḵeiga.áa} @ {} @ {} @ {} @ {} {} 
		\rlap{áwé} @ {}
		tsu
		{} \rlap{éiḵde} @ {} {}
		\rlap{wuduwa.áx̱} @ {} @ {} @ {} @ {}
		{} yéil sé. {} //
	\glb	{} á -dáx̱ {}
		{} ÿaa= ḵee- g- \rt[¹]{.a} -μμH {} {}
		á -wé
		tsu
		{} éeḵ -dé {}
		wu- du- i- \rt[²]{.ax̱} -μH
		{} yéil sé {} //
	\glc	{}[\pr{PP} \xx{3n} -\xx{abl} {}]
		{}[\pr{CP} along= dawn- \xx{gcnj}- \rt[¹]{extend} -\xx{var} \·\xx{sub} {}]
		\xx{foc} -\xx{mdst}
		again
		{}[\pr{PP} beach -\xx{all} {}]
		\xx{pfv}- \xx{4h·s}- \xx{stv}- \rt[²]{hear} -\xx{var}
		{}[\pr{DP} raven voice:\xx{inal} {}] //
	\gld	{} that -after {}
		{} along \rlap{dawn.\xx{csec}.extend} {} {} {} \·when {}
		\rlap{it.is} {}
		again
		{} beach -to {}
		\rlap{\xx{pfv}.people.hear} {} {} {} {}
		{} raven voice {} //
	\glft	‘After that, it was as it was dawning that again there was heard toward the beach a raven’s voice.’
		//
\endgl
\xe

\ex\label{ex:90-42-ran-down-there}%
\exmn{263.12}%
\begingl
	\glpreamble	Ā′ỵēq wudjix̣ī′x̣. //
	\glpreamble	Áa ÿeiḵ wujixeex. //
	\gla	{} \rlap{Áa} @ {} {}
		ÿeiḵ @ \rlap{wujixeex.} @ {} @ {} @ {} @ {} @ {} //
	\glb	{} á -μ {}
		ÿeiḵ= wu- d- sh- i- \rt[¹]{xix} -μμL //
	\glc	{}[\pr{PP} \xx{3n} -\xx{loc} {}]
		beach= \xx{pfv}- \xx{mid}- \xx{pej}- \xx{stv}- \rt[¹]{fall} -\xx{var} //
	\gld	{} there -to {}
		beach \rlap{\xx{pfv}.run·\xx{sg}} {} {} {} {} {} //
	\glft	‘He ran there to the beach.’
		//
\endgl
\xe

\ex\label{ex:90-43-ran-down-there}%
\exmn{263.12}%
\begingl
	\glpreamble	Tc!uʟe′ kīdjî′nawe wudcu′ta ka′odiha yū′tsa duīg̣aỵā′q!. //
	\glpreamble	Chʼu tle keejín áwé wooch shóode kawdiháa yú tsaa du eeg̱aÿáaxʼ. //
	\gla	Chʼu tle keejín \rlap{áwé} @ {}
		{} wooch \rlap{shóode} @ {} {}
		\rlap{kawdiháa} @ {} @ {} @ {} @ {} @ {} +
		{} yú tsáa {} 
		{} du \rlap{eeg̱aÿáaxʼ.} @ {} @ {} {} //
	\glb	chʼu tle keijín á -wé
		{} wooch shú -dé {}
		k- wu- d- i- \rt[¹]{ha} -μμH
		{} yú tsáa {}
		{} du éeḵ- ÿáᵏ -xʼ {} //
	\glc	just then five \xx{foc} -\xx{mdst}
		{}[\pr{PP} \xx{recip·pss} end -\xx{all} {}]
		\xx{qual}- \xx{pfv}- \xx{mid}- \xx{stv}- \rt[¹]{appear} -\xx{var}
		{}[\pr{DP} \xx{dist} seal {}]
		{}[\pr{PP} \xx{3h·pss} beach- face -\xx{loc} {}] //
	\gld	just then five \rlap{it.is} {}
		{} ea·oth’s end -to {}
		\rlap{\xx{pfv}.appear} {} {} {} {} {}
		{} that seal {}
		{} his beach- face -at {} //
	\glft	‘Just then there were five that appeared end to end, those seals, on the beach below him.’
		//
\endgl
\xe

The phrase that \citeauthor{swanton:1909} transcribes as \orth{wudcu′ta ka′odiha} in (\lastx) is not immediately obvious.
The second word appears to be the verb \fm{kawdiháa} ‘it appeared’ \parencite[12]{leer:1976}.
This leaves \orth{wudcu′ta} which would literally be \fm{wudshóota} which is nonsense.
The \orth{cu′} is probably from \fm{shú} ‘end (of it)’ and then \orth{ta} could be the allative postposition \fm{-dé} ‘to, toward’.
The remaining \orth{wud} is the least clear, but working from \citeauthor{swanton:1909}’s gloss “one behind another” this could be a mangled transcription of the reciprocal possessive pronoun \fm{wooch} ‘each other’s’.
The resulting phrase \fm{wooch shóode} would mean roughly ‘toward each other’s end’ and thus ‘end to end’ or more loosely ‘head to tail’.

\ex\label{ex:90-44-fifth-neck-fastened}%
\exmn{263.13}%
\begingl
	\glpreamble	Kidjî′na łēdᴀ′q! adē′awe q!ayax̣ᴀ′t. //
	\glpreamble	Keejín aa lidíx̱ʼ, aadé áwé x̱ʼayaxát. //
	\gla	{} Keejín aa \rlap{lidíx̱ʼ} @ {} {}
		{} \rlap{aadé} @ {} {} \rlap{áwé} @ {} 
		\rlap{x̱ʼayaxát.} @ {} @ {} @ {} //
	\glb	{} keijín aa lé- díx̱ʼ {}
		{} á -dé {} á -wé
		x̱ʼe- i- \rt[¹]{xat} -μH //
	\glc	{}[\pr{DP} five \xx{part} throat- spine {}]
		{}[\pr{PP} \xx{3n} -\xx{all} {}] \xx{foc} -\xx{mdst}
		mouth- \xx{stv}- \rt[¹]{suspend} -\xx{var} //
	\gld	{} five one \rlap{neck} {} {}
		{} there -to {} \rlap{it.is} {}
		\rlap{\xx{impfv}.be.fastened} {} {} {} //
	\glft	‘The fifth one’s neck, it is to there that it is fastened.’
		//
\endgl
\xe

\ex\label{ex:90-45-carry-suphelp-thing}%
\exmn{263.14}%
\begingl
	\glpreamble	We′aciye g̣ānasē′tc-ᴀt daᴀtxakā′ awadjᴀ′ł. //
	\glpreamble	Wé ash yeeg̱áa naseich át, a daatx̱ a káa aawajél. //
	\gla	{} Wé
			{} {} ash \rlap{yeeg̱áa} @ {} {}
				\rlap{naseich} @ {} @ {} @ {} @ {} {}
			át, {} +
		{} a \rlap{daatx̱} @ {} {}
		{} a \rlap{káa} @ {} {}
		\rlap{aawajél.} @ {} @ {} @ {} @ {} //
	\glb	{} wé
			{} {} ash ee -g̱áa {}
				n- \rt[¹]{suʰ} -eμL -ch {} {}
			át {}
		{} a daa -dáx̱ {}
		{} a ká -μ {}
		a- wu- i- \rt[²]{jel} -μH //
	\glc	{}[\pr{DP} \xx{mdst}
			{}[\pr{CP} {}[\pr{PP} \xx{3prx} \xx{base} -\xx{ades} {}]
				\xx{ncnj}- \rt[¹]{sup·help} -\xx{var} -\xx{rep} \·\xx{rel} {}]
			thing {}]
		{}[\pr{PP} \xx{3n·pss} around -\xx{abl} {}]
		{}[\pr{PP} \xx{3n} \xx{hsfc} -\xx{loc} {}]
		\xx{arg}- \xx{pfv}- \xx{stv}- \rt[²]{carry·load} -\xx{var} //
	\gld	{} that
			{} {} him {} -for {}
				\rlap{\xx{hab}.super·help} {} {} {} \·that {}
			thing {}
		{} its around -from {}
		{} its atop -at {}
		\rlap{3>3.\xx{pfv}.carry·load} {} {} {} {} //
	\glft	‘That thing that gave supernatural help to him, he carried it there from around it.’
		//
\endgl
\xe

\ex\label{ex:90-46-inside-cant-be-seen}%
\exmn{263.14}%
\begingl
	\glpreamble	ʟēł dutī′n de duhî′tî aỵî yuqᴀłū′xtcātc. //
	\glpreamble	Tléil duteen dé du hídi a ÿee, yú kalóox̱jaach. //
	\gla	Tléil \rlap{duteen} @ {} @ {} @ {}
		dé
		{} du \rlap{hídi} @ {} a ÿee, {} +
		{} yú \rlap{kalóox̱jaach} @ {} @ {} @ {} @ {} @ {} {} //
	\glb	tléil u- du- \rt[²]{tin} -μμL
		dé
		{} du hít -í a ÿee {}
		{} yú k- \rt[¹]{lux̱} -μμH -ch -aa -ch {} //
	\glc	\xx{neg} \xx{irr}- \xx{4h·s}- \rt[²]{see} -\xx{var}
		now
		{}[\pr{DP} \xx{3h·pss} house -\xx{pss} \xx{3n·pss} below:\xx{inal} {}]
		{}[\pr{PP} \xx{dist} \xx{qual}- \rt[¹]{drip} -\xx{var} -\xx{rep} -\xx{nmz} -\xx{erg} {}] //
	\gld	not \rlap{\xx{impfv}.one.can·see} {} {} {}
		now
		{} his house {} its inside {}
		{} the \rlap{dripping} {} {} {} {} -because {} //
	\glft	‘Now it could not be seen, the inside of his house, for the dripping.’
		//
\endgl
\xe

\FIXME{Note partial parallelism between (\ref{ex:90-46-inside-cant-be-seen}) and (\ref{ex:90-51-cant-be-seen-for-plenty}).}

\ex\label{ex:90-47-uncles-famine}%
\exmn{264.1}%
\begingl
	\glpreamble	Yu acnᴀ′q-wułîg̣ā′s!î dukā′k-hᴀs qo yaē′n dēn wū′nî. //
	\glpreamble	Yú ash náḵ wuligáasʼi du káak hás ḵu.aa éindéin woonee. //
	\gla	{} Yú {} {} ash náḵ {}
				\rlap{wuligáasʼi} @ {} @ {} @ {} @ {} @ {} {} 
			du káak @ \•hás {}
		ḵu.aa
		\rlap{éindéin} @ {} @ {}
		\rlap{woonee} @ {} @ {} @ {} //
	\glb	{} yú {} {} ash náḵ {}
				wu- l- i- \rt[¹]{gasʼ} -μμH -i {}
			du káak =hás {}
		ḵu.aa
		\rt[¹]{.en} -μμH -déin
		wu- i- \rt[¹]{niʰ} -μμL //
	\glc	{}[\pr{DP} \xx{dist} {}[\pr{CP} {}[\pr{PP} \xx{3prx} \xx{elat} {}]
				\xx{pfv}- \xx{xtn}- \xx{stv}- \rt[¹]{extend} -\xx{var} -\xx{rel} {}]
			\xx{3h·pss} mat·uncle =\xx{plh} {}]
		\xx{contr}
		\rt[¹]{famine} -\xx{var} -\xx{adv}
		\xx{pfv}- \xx{stv}- \rt[¹]{happen} -\xx{var} //
	\gld	{} those {} {} him away {}
				\rlap{\xx{pfv}.relocate} {} {} {} {} -who {}
			his mat·uncle =s {}
		however
		\rlap{famine} {} -ly 
		\rlap{\xx{pfv}.happened} {} {} {} //
	\glft	‘Those uncles of his who had abandoned him however were suffering from famine.’
		//
\endgl
\xe

\section{Paragraph 4}\label{sec:90-para-4}

\ex\label{ex:90-48-mtn-goats}%
\exmn{264.3}%
\begingl
	\glpreamble	Wananī′sawe dukînā′da ka′odîk!ît! djᴀ′nwu. //
	\glpreamble	Wáa nanée sáwé du kináade kawdikʼítʼ jánwu. //
	\gla	{} Wáa \rlap{nanée} @ {} @ {} @ {} {}
		\rlap{sáwé} @ {} @ {}
		{} du \rlap{kináade} @ {} {}
		\rlap{kawdikʼítʼ} @ {} @ {} @ {} @ {} @ {}
		{} jánwu. {} //
	\glb	{} wáa n- \rt[¹]{ni} -μμH {} {} 
		s= á -wé
		{} du kináa -dé {}
		k- wu- d- i- \rt[¹]{kʼítʼ} -μH
		{} jánwu {} //
	\glc	{}[\pr{CP} how \xx{ncnj}- \rt[¹]{happen} -\xx{var} \·\xx{sub} {}]
		\xx{q}= \xx{foc} -\xx{mdst}
		{}[\pr{PP} \xx{3h·pss} above -\xx{all} {}]
		\xx{qual}- \xx{pfv}- \xx{pasv}- \xx{stv}- \rt[²]{use·up} -\xx{var}
		{}[\pr{DP} mtn·goat {}] //
	\gld	{} how \rlap{\xx{csec}.happen} {} {} \·while {}
		some\· \rlap{it.is} {}
		{} his above -to {}
		\rlap{\xx{pfv}.spread·out} {} {} {} {} {}
		{} mtn·goat {} //
	\glft	‘At some point above him they were spread out, mountain goats.’
		//
\endgl
\xe

The verb \fm{kawdikʼítʼ} in (\lastx) is somewhat tricky to interpret.
The root \fm{\rt[²]{kʼitʼ}} which can mean variously ‘eat all of’, ‘pick berries’, ‘spread out’, ‘(group) die off’, ‘(group) capsize from boat’, and ‘run out of’ \parencites[f04/118–121]{leer:1973}[776]{leer:1976}.
\citeauthor{swanton:1909}’s gloss “came out” does not suggest the extinction or exhaustion meanings, so the most likely interpretation is ‘spread out’.
If instead it was ‘die off’ then the mountain goats would be dead before the supernatural helper was released among them in (\nextx).
If the mountain goats were all eaten up then this would similarly conflict with (\nextx).
The ‘pick berries’ and ‘capsize’ meanings also do not make sense in this context.

\ex\label{ex:90-49-let-loose-among}%
\exmn{264.3}%
\begingl
	\glpreamble	Xō′de adjī′wanᴀq. //
	\glpreamble	X̱oodé ajeewanáḵ. //
	\gla	{} {} \rlap{x̱oodé} @ {} {}
		\rlap{ajeewanáḵ.} @ {} @ {} @ {} @ {} @ {}  //
	\glb	{} {} x̱oo -dé {}
		a- ji- wu- i- \rt[²]{naḵ} -μH //
	\glc	{}[\pr{PP} \xx{3n·pss} among -\xx{all} {}]
		\xx{arg}- hand- \xx{pfv}- \xx{stv}- \rt[²]{abandon} -\xx{var} //
	\gld	{} its among -to {}
		\rlap{3>3.\xx{pfv}.release} {} {} {} {} {} //
	\glft	‘He released it among them.’
		//
\endgl
\xe

The phrase \fm{x̱oodé} in (\lastx) is a bit unusual.
Normally the relational noun \fm{x̱oo} ‘among’ would be expected to have a possessor, but \citeauthor{swanton:1909}’s transcription unambiguously starts the sentence without one.
The analysis includes a covert possessor, but it is equally possible that \citeauthor{swanton:1909}’s transcription is faulty and there should be an overt possessor \fm{a} ‘its’.
There are other cases where the possessor of a relational noun is covert, but these always include a middle voice \fm{d-} in the verb which is absent in \fm{ajeewanáḵ} in (\lastx).

\ex\label{ex:90-50-all-fall-down}%
\exmn{264.4}%
\begingl
	\glpreamble	ʟē łdakᴀ′t dāq kawasū′s. //
	\glpreamble	Tle ldakát daak kaawasóos. //
	\gla	Tle ldakát daak @ \rlap{kaawasóos.} @ {} @ {} @ {} @ {} //
	\glb	tle ldakát daak= k- wu- i- \rt[¹]{suʼs} -μμH //
	\glc	then all \xx{admar}= \xx{sro}- \xx{pfv}- \xx{stv}- \rt[¹]{fall·\xx{pl}} -\xx{var} //
	\gld	then all down \rlap{round.\xx{pfv}.fall·\xx{pl}} {} {} {} {} //
	\glft	‘Then they all fell down.’
		//
\endgl
\xe

\citeauthor{swanton:1909} transcribes \orth{dāq} in (\lastx) which suggests \fm{daaḵ} ‘inland, up from sea; away from open; off of fire’.
In this context the mountain goats of (\ref{ex:90-48-mtn-goats}) would be expected to fall down from the mountainside as implied by \citeauthor{swanton:1909}’s gloss “down”.
It seems likely that \orth{dāq} is a mishearing of \fm{daak} ‘seaward, out from shore; into open; onto fire’; note the final velar [\ipa{tàːk}] rather than the uvular of \fm{daaḵ} [\ipa{tàːq}].
This would be in keeping with motion down the mountainside toward the shore and also from the inland toward the ocean.
Compare for example the use of \fm{daak} in \fm{séew daak wusitán} ‘rain fell’ where \fm{daak} describes motion down from the clouds.

\ex\label{ex:90-51-cant-be-seen-for-plenty}%
\exmn{264.4}%
\begingl
	\glpreamble	Tc!uʟe′ ʟēł wudutī′n de duhî′tîỵî-ᴀt cā′ỵełahēn ʟēn. //
	\glpreamble	Chʼu tle tléil wuduteen dé du hídi ÿee at shaÿalahéin tlein. //
	\gla	Chʼu tle tléil
		\rlap{wuduteen} @ {} @ {} @ {} @ {}
		dé
		{} du \rlap{hídi} @ {} ÿee {} +
		{} at @ \rlap{shaÿalahéin} @ {} @ {} @ {} @ {} @ {} tlein. {} //
	\glb	chʼu tle tléil
		u- wu- du- \rt[²]{tin} -μμL
		dé
		{} du hít -í ÿee {}
		{} at= sha- ÿ- l- \rt[¹]{hen} -μμH {} tlein {} //
	\glc	just then \xx{neg}
		\xx{irr}- \xx{pfv}- \xx{4h·s}- \rt[²]{see} -\xx{var}
		now
		{}[\pr{DP} \xx{3h·pss} house -\xx{pss} below {}]
		{}[\pr{NP} \xx{4n·o}= head- \xx{qual}- \xx{csv}- \rt[¹]{many} -\xx{var} \·\xx{nmz} big {}] //
	\gld	just then not
		\rlap{\xx{pfv}.people.can·see} {} {} {} {}
		now
		{} his house {} inside {}
		{} things= \rlap{\xx{impfv}.make.many} {} {} {} {} {} big {} //
	\glft	‘Now the inside of his house could not be seen for the great multiplication of things.’
		//
\endgl
\xe

The phrase \fm{at shaÿalahéin tlein} deserves some explanation.
This is a nominalization of the transitive imperfective state verb \fm{ashaÿalihéin} ‘s/he increases it, makes them many’; the unaccusative \fm{shaÿadihéin} ‘it is much, they are many’ is the most well known verb based on the same root \fm{\rt[¹]{hen}} ‘many’.
The nominalization in (\lastx) is not indicated by an overt suffix but can be detected by two other facts: (i) the stative \fm{i-} prefix is missing which reflects its regular suppression in nominalizations and subordinate clauses, and (ii) the postnominal modifier \fm{tlein} is not normally grammatical with a verb but should be unproblematic with nominalizations.

\clearpage
\section{Paragraph 5}\label{sec:90-para-5}

\ex\label{ex:90-52-sent-people-to-burn-him}%
\exmn{264.6}%
\begingl
	\glpreamble	Dutuwū′tc łā′xawe dukā′ktc ade′ kukā′waqa duīg̣a′ qᴀg̣ā′x dusqā′ndayu. //
	\glpreamble	Du toowúch láaxw áwé, du káakch aadé ḵukaawakaa du eeg̱áa kag̱aax̱dusgaant áyú. //
	\gla	{} {} Du \rlap{toowúch} @ {} @ {} {}
			\rlap{láaxw} @ {} @ {} @ {} {} \rlap{áwé,} @ {} +
		{} du \rlap{káakch} @ {} {}
		{} \rlap{aadé} @ {} {}
		\rlap{ḵukaawaḵaa} @ {} @ {} @ {} @ {} @ {}
		{} du \rlap{eeg̱áa} @ {} {}
		{} {} \rlap{kag̱aax̱dusgaant} @ {} @ {} @ {} @ {} @ {} @ {} @ {} @ {} {} {} {} 
		\rlap{áyú.} @ {} //
	\glb	{} {} du tú -í -ch {}
			{} \rt[¹]{laxw} -μμH {} {} á -wé
		{} du káak -ch {}
		{} á -dé {}
		ḵu- k- wu- i- \rt[²]{ḵa} -μμL
		{} du ee -g̱áa {}
		{} {} k- g̱- g̱- du- d- s- \rt[¹]{gan} -μμL {} {} -t {}
		á -yú //
	\glc	{}[\pr{CP} {}[\pr{PP} \xx{3h·pss} mind -\xx{pss} -\xx{instr} {}]
			\xx{zcnj}\· \rt[¹]{starve} -\xx{var} \·\xx{sub} {}] \xx{foc} -\xx{mdst}
		{}[\pr{DP} \xx{3h·pss} mat·uncle -\xx{erg} {}]
		{}[\pr{PP} \xx{3n} -\xx{all} {}]
		\xx{4h·o}- \xx{qual}- \xx{pfv}- \xx{stv}- \rt[²]{say} -\xx{var}
		{}[\pr{PP} \xx{3h} \xx{base} -\xx{pnct} {}]
		{}[\pr{PP} {}[\pr{CP}
			\xx{qual}- \xx{g̱cnj}- \xx{mod}- \xx{4h·s}- \xx{mid}- \xx{csv}-
				\rt[¹]{burn} -\xx{var} \·\xx{sub} {}] -\xx{pnct} {}]
		\xx{foc} -\xx{dist} //
	\gld	{} {} his mind {} -by {}
			\rlap{\xx{csec}.starve} {} {} \·when {} \rlap{it.is} {}
		{} his mat·uncle {} {}
		{} there -to {}
		\rlap{people.\xx{pfv}.order} {} {} {} {} {}
		{} him {} -for {}
		{} {} \rlap{\xx{hort}.people.make.burn} {} {} {} {} {} {} {} {} {} -to {}
		\rlap{it.is} {} //
	\glft	‘Believing him to have starved, his uncle ordered people to go there for him to burn him up.’
		//
\endgl
\xe

The phrase \fm{du toowúch} at the beginning of (\lastx) looks like it contains the ergative \fm{-ch} and would thus be the subject DP of a transitive verb.
But the consecutive verb form \fm{láaxw} ‘having starved’ is an unergative intransitive and so cannot have a subject.
Instead \fm{du toowúch} is actually an adjunct PP where \fm{-ch} is instrumental; the literal translation is ‘by his mind’ or more loosely ‘according to his belief/opinion/desire’.

The phrase \fm{kagaax̱dusgaant} in (\lastx) is a special kind of adjunct PP formed with the punctual postposition \fm{-t} and an unmarked subordinate clause whose verb is inflected for the hortative mood.
These structures express the purpose for or intended situation arising from the situation in the matrix clause and so are conventionally known as ‘purposive clauses’ \parencites[91–92]{naish:1966}[106, 187]{story:1966}[427–428]{leer:1991}.
Purposive clauses can often be translated as ‘in order to’ or ‘so that’.

The adjunct PP \fm{du eeg̱áa} preceding the purposive clause in (\lastx) is analyzed as part of the matrix clause, but it alternatively could be analyzed as part of the purposive clause.
As an adjunct in the matrix clause it would mean that the uncle ordered people for his nephew, i.e.\ to go somewhere to deal with him.
As an adjunct in the purposive clause, the PP would mean that burning is to be done for the uncle’s benefit.
The difference lies in the interpretation of the pronoun \fm{du}: in the matrix where the uncle is the subject the \fm{du} would refer to the nephew, but in the purposive clause where the nephew is the object the \fm{du} would refer to the uncle.
In both cases the English translation is the same ‘for him’.

\ex\label{ex:90-53-ordered-his-slaves}%
\exmn{264.7}%
\begingl
	\glpreamble	Dugū′x!o a′de akā′waqa //
	\glpreamble	Du goox̱xʼú aadé akaawaḵaa. //
	\gla	{} Du \rlap{goox̱xʼú} @ {} @ {} {}
		{} \rlap{aadé} @ {} {}
		\rlap{akaawaḵaa.} @ {} @ {} @ {} @ {} @ {} //
	\glb	{} du goox̱ -xʼ -í {}
		{} á -dé {}
		a- k- wu- i- \rt[²]{ḵa} -μμL //
	\glc	{}[\pr{DP} \xx{3h·pss} slave -\xx{pl} -\xx{pss} {}]
		{}[\pr{PP} \xx{3n} -\xx{all} {}]
		\xx{arg}- \xx{qual}- \xx{pfv}- \xx{stv}- \rt[²]{say} -\xx{var} //
	\gld	{} his slave -s {} {}
		{} there -to {}
		\rlap{3>3.\xx{pfv}.order} {} {} {} {} {} //
	\glft	‘He ordered his slaves there.’
		//
\endgl
\xe

\ex\label{ex:90-54-boated-near}%
\exmn{264.7}%
\begingl
	\glpreamble	ᴀtxawe′t doxᴀ′nt uwaqo′x. //
	\glpreamble	Átx̱ áwé du x̱ánt uwaḵúx̱. //
	\gla	{} \rlap{Átx̱} @ {} {}
		\rlap{áwé} @ {}
		{} du \rlap{x̱ánt} @ {} {}
		\rlap{uwaḵúx̱.} @ {} @ {} @ {} //
	\glb	{} á -dáx̱ {}
		á -wé
		{} du x̱án -t {}
		u- i- \rt[¹]{ḵux̱} -μH //
	\glc	{}[\pr{PP} \xx{3n} -\xx{abl} {}]
		\xx{foc} -\xx{mdst}
		{}[\pr{PP} \xx{3h·pss} near -\xx{pnct} {}]
		\xx{zpfv}- \xx{stv}- \rt[¹]{go·boat} -\xx{var} //
	\gld	{} that -after {}
		\rlap{it.is} {}
		{} his near -to {}
		\rlap{\xx{pfv}.go·boat} {} {} {} //
	\glft	‘So then they boated near to him.’
		//
\endgl
\xe

\ex\label{ex:90-55-called-inside}%
\exmn{264.8}%
\begingl
	\glpreamble	Wē′guxq! nēłde′ awaxō′x. //
	\glpreamble	Wé goox̱xʼ neildé aawax̱oox̱. //
	\gla	{} Wé \rlap{goox̱xʼ} @ {} {}
		{} \rlap{neildé} @ {} {}
		\rlap{aawax̱oox̱.} @ {} @ {} @ {} @ {} //
	\glb	{} wé goox̱ -xʼ {}
		{} neil -dé {}
		a- wu- i- \rt[²]{x̱ux̱} -μμL //
	\glc	{}[\pr{DP} \xx{mdst} slave -\xx{pl} {}]
		{}[\pr{PP} home -\xx{all} {}]
		\xx{arg}- \xx{pfv}- \xx{stv}- \rt[²]{call·out} -\xx{var} //
	\gld	{} those slave -s {}
		{} inside -to {}
		\rlap{3>3.\xx{pfv}.call·out} {} {} {} {} {} //
	\glft	‘The slaves called out to him inside.’
		//
\endgl
\xe

\ex\label{ex:90-56-come-up}%
\exmn{264.8}%
\begingl
	\glpreamble	Dāq a′osiᴀt. //
	\glpreamble	Daaḵ awsi.át. //
	\gla	Daaḵ @ \rlap{awsi.át.} @ {} @ {} @ {} @ {} @ {} //
	\glb	daaḵ= a- wu- s- i- \rt[¹]{.at} -μH //
	\glc	inland= \xx{arg}- \xx{pfv}- \xx{csv}- \xx{stv}- \rt[¹]{go·\xx{pl}} -\xx{var} //
	\gld	up \rlap{3>3.\xx{pfv}.make.go·\xx{pl}} {} {} {} {} {} //
	\glft	‘He had them come up.’
		//
\endgl
\xe

\ex\label{ex:90-57-give-to-eat}%
\exmn{264.8}%
\begingl
	\glpreamble	Q!ēx ᴀt tīx. //
	\glpreamble	X̱ʼéix̱ at téex̱. //
	\gla	{} {} \rlap{X̱ʼéix̱} @ {} {}
		at @ \rlap{téex̱} @ {} @ {} //
	\glb	{} {} x̱ʼé -x̱ {}
		at= \rt[²]{ti} -μμH -x̱ //
	\glc	{}[\pr{PP} \xx{3h·pss} mouth -\xx{pert} {}]
		\xx{4n·o}= \rt[²]{handle} -\xx{var} -\xx{rep} //
	\gld	{} their mouth -at {}
		thing \rlap{\xx{impfv}.handle.\xx{rep}} {} {} //
	\glft	‘He gives them things to eat.’
		//
\endgl
\xe

The phrase \fm{x̱ʼéix̱} in (\lastx) is another instance of a relational noun without an overt possessor.
Compare the similar absence of a third person possessor in (\ref{ex:90-49-let-loose-among}).
If it were overt – or if \citeauthor{swanton:1909} missed it while transcribing – the expected form would be \fm{hasdu} ‘their’ since the slaves are referred to as plural in (\ref{ex:90-53-ordered-his-slaves})–(\ref{ex:90-56-come-up}).

\ex\label{ex:90-58-overnight-once}%
\exmn{264.9}%
\begingl
	\glpreamble	ʟēq! ᴀcxᴀ′nî uwaxe′. //
	\glpreamble	Tléixʼ ash x̱áni uwax̱éi. //
	\gla	{} Tléixʼ {} 
		{} ash \rlap{x̱áni} @ {} {}
		\rlap{uwax̱éi.} @ {} @ {} @ {} //
	\glb	{} tléixʼ {}
		{} ash x̱án -í {}
		u- i- \rt[¹]{x̱i} -μμH //
	\glc	{}[\pr{NP} one {}]
		{}[\pr{PP} \xx{3prx·pss} near -\xx{loc} {}]
		\xx{zpfv}- \xx{stv}- \rt[¹]{overnight} -\xx{var} //
	\gld	{} one {}
		{} his near -at {}
		\rlap{\xx{pfv}.overnight} {} {} {} //
	\glft	‘They overnighted once with him.’
		//
\endgl
\xe

\ex\label{ex:90-59-slave-has-child}%
\exmn{264.9}%
\begingl
	\glpreamble	A′siwe ỵê′tk!waỵa u yū′gux. //
	\glpreamble	Ásíwé ÿátkʼw aÿa.óo yú goox̱. //
	\gla	\rlap{Ásíwé} @ {} @ {}
		{} \rlap{ÿátkʼw} @ {} {}
		\rlap{aÿa.óo} @ {} @ {} @ {} 
		{} yú goox̱. {} //
	\glb	á -sí -wé
		{} ÿát -kʼw {}
		a- i- \rt[²]{.uʰ} -μμH
		{} yú goox̱ {} //
	\glc	\xx{foc} -\xx{dub} -\xx{mdst}
		{}[\pr{DP} child -\xx{dim} {}]
		\xx{arg}- \xx{stv}- \rt[²]{own} -\xx{var}
		{}[\pr{DP} \xx{dist} slave {}] //
	\glft	‘Apparently she has a little child, the slave.’
		//
\endgl
\xe

The sentence in (\lastx) is a good example of the relatively uncommon object-verb-subject word order.
The subject of the verb \fm{aÿa.óo} ‘s/he owns, has him/her/it’ is the DP \fm{yú goox̱} ‘the slave’ which occurs after the verb, and the object is the DP \fm{ÿátkʼw} ‘little child’.
The object is also an example of the use of the noun \fm{ÿát} without possession; this is now dispreferred in modern Tlingit but was apparently unremarkable in \citeauthor{swanton:1909}’s era since it occurs fairly often in his materials.

The translation of (\lastx) represents the subject with the feminine pronoun ‘she’.
The Tlingit sentence makes no mention of whether the slave is male or female since Tlingit has no grammatical gender.
\citeauthor{swanton:1909}’s translation of (\lastx) as “One of these slaves had a child” also offers no hint for the gender of the subject.
But later in (\ref{ex:90-74-inside-of-clam}) the mother of the child is explicitly mentioned as \fm{du tláa} ‘his mother’ so it seems likely that the intended referent here is a woman.

\ex\label{ex:90-60-says-to-them}%
\exmn{264.10}%
\begingl
	\glpreamble	Yē′sdo-daỵaqa, //
	\glpreamble	Yéi sdu daaÿaḵá //
	\gla	Yéi @ \rlap{sdu} @ {} @ \rlap{daaÿaḵá} @ {} @ {} @ {} //
	\glb	yéi= s= du= daa- ÿ- \rt[²]{ḵa} -μH //
	\glc	thus= \xx{plh}= \xx{3h·o}= around- \xx{qual}- \rt[²]{say} -\xx{var} //
	\gld	thus \rlap{them·to} {} \rlap{\xx{impfv}.say} {} {} {} //
	\glft	‘He says to them’
		//
\endgl
\xe

The verb in (\lastx) is interesting because it has the third person plural pronoun \fm{sdu} as an object.
Normally \fm{sdu} \~\ \fm{hasdu} is the third person plural possessive pronoun ‘their’ as in \fm{hasdu éesh} ‘their father’ or \fm{hasdu yaagú} ‘their boat’.
Here it appears to be the object of the verb, supported both by the discourse context and by \citeauthor{swanton:1909}’s gloss “He said to them”.
The typical form in this context would be \fm{yéi has adaayaḵá} with the plural human modifier \fm{has=} and the argument-marking prefix \fm{a-} to indicate a third person object acted on by a third person subject.
A similar example of \fm{has} and \fm{a-} is \fm{Atxʼaan sákw áyú yéi has adaané} ‘They are preparing it for dryfish’ \parencite[201.133]{dauenhauer:1987}.
The factors that give rise to a possessive pronoun instead of an object pronoun here are not clear, but it is probably significant that the incorporated noun \fm{daa-} corresponds to the inalienable noun \fm{daa} ‘around, about’.
The possessive \fm{ax̱=} ‘my’ sometimes occurs instead of \fm{x̱at=} ‘me’ with other inalienable nouns such as in \fm{Ax̱ shawlixaash} ‘s/he cut my hair’ with \fm{sha-} from \fm{shá} ‘head’.

\ex\label{ex:90-61-dont-take-anything}%
\exmn{264.10}%
\begingl
	\glpreamble	“Łî′ł ke ai′cᴀ′tdjīk q!wᴀn.” //
	\glpreamble	«\!Líl kei aa ishátjiḵ xʼwán.\!» //
	\gla	«\!Líl kei @ aa @ \rlap{ishátjiḵ} @ {} @ {} @ {} @ {} xʼwán.\!» //
	\glb	\pqp{}líl kei= aa= i- \rt[²]{shaʼt} -μH -ch -ḵ xʼwán //
	\glc	\pqp{}\xx{phib} up= \xx{part·o}= \xx{2sg·s}- \rt[²]{grab} -\xx{var} -\xx{rep} -\xx{phib} \xx{imp} //
	\glft	‘“Do not take anything”.’
		//
\endgl
\xe

\citeauthor{swanton:1909}’s transcription \orth{ai′cᴀ′tdjīk} clearly suggets \fm{aa ishátjiḵ} which has the second person singular subject \fm{i-} ‘you sg.’ and not the second person plural subject \fm{ÿi-} which would instead be \fm{aa ÿishátjiḵ} (e.g.\ \orth{āỵi′cᴀ′tdjēk}).
This is strange because the discourse from (\ref{ex:90-53-ordered-his-slaves}) through (\ref{ex:90-60-says-to-them}) has clearly referred to plural slaves visiting the protagonist, but here the protagonist apparently is speaking to a single person.
This does not seem to be a one-off mistake because (\ref{ex:90-63-tell-master-burned-him-up}) is also explicitly directed at a single person with singular \fm{i-} ‘you sg.’ and (\ref{ex:90-64-thus-instructed}) implicitly refers to a single person since \fm{has=} is absent.

The English translation of (\lastx) is somewhat loose.
A more literal translation would be something like ‘Do not grab up some’, reflecting that the combination of \fm{kei=} ‘up’ and \fm{\rt[²]{shaʼt}} ‘grab, catch, capture, hold’ means something like ‘grab up, snatch’ and that the partitive object \fm{aa=} means ‘one of, some of’.
The referent of \fm{aa=} is probably meant to be the foodstuffs that the protagonist obtained with supernatural help, but this is not explicit.
The verb in (\ref{ex:90-62-something-into-something}) has the same \fm{aa=} and presumably refers to the same thing.

\ex\label{ex:90-62-something-into-something}%
\exmn{264.10}%
\begingl
	\glpreamble	Aᴀ′siwe ᴀt tū′de ā′wug̣ēq! yū′guxk!utc. //
	\glpreamble	Á ásíwé at tóode aa woog̱éixʼ yú goox̱kʼúch. //
	\gla	{} Á {} \rlap{ásíwé} @ {} @ {}
		{} at \rlap{tóode} @ {} {}
		aa @ \rlap{woog̱éixʼ} @ {} @ {} @ {} +
		{} yú \rlap{goox̱kʼúch.} @ {} @ {} {} //
	\glb	{} á {} á -sí -wé
		{} at tú -dé {}
		aa= wu- i- \rt[²]{g̱ixʼ} -μμH
		{} yú goox̱ -kʼ -ch {} //
	\glc	{}[\pr{DP} \xx{3n} {}] \xx{foc} -\xx{dub} -\xx{mdst}
		{}[\pr{PP} \xx{4n·pss} inside -\xx{all} {}]
		\xx{part·o}= \xx{pfv}- \xx{stv}- \rt[²]{throw·\xx{sg}} -\xx{var}
		{}[\pr{DP} \xx{dist} slave -\xx{dim} -\xx{erg} {}] //
	\gld	{} it {} \rlap{it.is.apparently} {} {}
		{} sth’s inside -to {}
		some \rlap{\xx{pfv}.throw·\xx{sg}} {} {} {}
		{} that slave -little {} {} //
	\glft	‘Then it is apparently that he tossed some into something, that little slave.’
		//
\endgl
\xe

\ex\label{ex:90-63-tell-master-burned-him-up}%
\exmn{264.11}%
\begingl
	\glpreamble	“Dekē′wu tusigᴀ′n yu-q!wᴀ′n-ckan-īłnīk- îts!ā′tītîn //
	\glpreamble	«\!‹\!De kei wutusigán\!› yóo xʼwán sh kaneelneek i sʼaatí tin.\!» //
	\gla	«\!‹\!De
		kei @ \rlap{wutusigán\!›} @ {} @ {} @ {} @ {} @ {}
		yóo xʼwán
		sh @ \rlap{kaneelneek} @ {} @ {} @ {} @ {} @ {} @ {}
		{} i sʼaatí tin.\!» {} //
	\glb	\phantom{«\!‹\!}de
		kei= wu- tu- s- i- \rt[¹]{gan} -μH
		yóo xʼwán
		sh= k- n- i- d- l- \rt[²]{nik} -μμL
		{} i sʼaatí teen {} //
	\glc	\phantom{«\!‹\!}already
		up= \xx{pfv}- \xx{1pl·s}- \xx{csv}- \xx{stv}- \rt[¹]{burn} -\xx{var}
		\xx{quot} \xx{imp}
		\xx{rflx·o}= \xx{qual}- \xx{ncnj}- \xx{2sg·s}- \xx{mid}- \xx{xtn}- \rt[²]{tell} -\xx{var}
		{}[\pr{PP} \xx{2sg·pss} master \xx{instr} {}] //
	\gld	\phantom{«\!‹\!}already
		up \rlap{\xx{pfv}.we.make.burn} {} {} {} {} {}
		\xx{quot} \xx{imp}
		self \rlap{\xx{imp}.you·\xx{sg}.tell} {} {} {} {} {} {}
		{} your master with {} //
	\glft	‘“Tell your master ‘we already burned him up’.”’
		//
\endgl
\xe

\ex\label{ex:90-64-thus-instructed}%
\exmn{264.12}%
\begingl
	\glpreamble	Yū′yên ᴀcukā′wadja. //
	\glpreamble	Yóo yan ashukaawajáa. //
	\gla	Yóo @ yan @ \rlap{ashukaawajáa.} @ {} @ {} @ {} @ {} @ {} @ {} //
	\glb	yóo= ÿán= a- shu- k- wu- i- \rt[²]{ja} -μμH //
	\glc	\xx{quot}= \xx{term}= \xx{arg}- end- \xx{qual}- \xx{pfv}- \xx{stv}- \rt[²]{instruct} -\xx{var} //
	\glft	‘Thus he instructed him.’
		//
\endgl
\xe

\section{Paragraph 6}\label{sec:90-para-6}

\ex\label{ex:90-65-cried-out}%
\exmn{264.13}%
\begingl
	\glpreamble	Tāt ān hᴀs qō′xawe ke ka′odigᴀx duỵêtk!ᵒ. //
	\glpreamble	Taat aan has ḵóox̱ áwé kei kawdig̱áx̱ du ÿátkʼu. //
	\gla	{} Taat {}
		{} {} \rlap{aan} @ {} {}
			has @ \rlap{ḵóox̱} @ {} @ {} @ {} {}
		\rlap{áwé} @ {}
		kei @ \rlap{kawdig̱áx̱} @ {} @ {} @ {} @ {} @ {} 
		{} du \rlap{ÿátkʼu.} @ {} @ {} {} //
	\glb	{} taat {}
		{} {} á -n {}
			has= {} \rt[¹]{ḵux̱} -μμH {} {}
		á -wé
		kei= k- wu- d- i- \rt[¹]{g̱ax̱} -μH
		{} du ÿát -kʼw -í {} //
	\glc	{} night {}
		{}[\pr{CP} {}[\pr{PP} \xx{3n} -\xx{instr} {}]
			\xx{plh}= \xx{zcnj}\· \rt[¹]{go·boat} -\xx{var} \·\xx{sub} {}]
		\xx{foc} -\xx{mdst}
		up= \xx{qual}- \xx{pfv}- \xx{mid}- \xx{stv}- \rt[¹]{cry} -\xx{var}
		{}[\pr{DP} \xx{3h·pss} child -\xx{dim} -\xx{pss} {}] //
	\gld	{} night {}
		{} {} it -with {}
			they\• \rlap{\xx{csec}.go·boat} {} {} {} {}
		\rlap{it.is} {}
		up \rlap{\xx{pfv}.cry·out} {} {} {} {} {}
		{} her child -little {} {} //
	\glft	‘At night having boated back with it, he cried out, her little child.’
		//
\endgl
\xe

\citeauthor{swanton:1909}’s gloss of \orth{duỵêtk!ᵒ} is “the baby” but this obviously reflects \fm{du ÿátkʼu} and therefore must be ‘her little child’.
The possessive pronoun \fm{du} is translated as feminine following the discussion about pronominal gender for (\ref{ex:90-59-slave-has-child}).

\ex\label{ex:90-66-little-fat}%
\exmn{264.13}%
\begingl
	\glpreamble	“Tayê′k!we, tayê′k!we,” yūk dag̣¯á′x yū′gux ỵêtk!ᵒ. //
	\glpreamble	«\!Taayákʼw éi, taayákʼw éi\!» yóo kdag̱áax̱ yú goox̱ ÿátkʼu. //
	\gla	«\!\rlap{Taayákʼw} @ {} éi \rlap{taayákʼw} @ {} éi\!»
		yóo @ \rlap{kdag̱áax̱} @ {} @ {} @ {}
		{} yú goox̱ \rlap{ÿátkʼu.} @ {} @ {} {} //
	\glb	\pqp{}taaÿ -kʼw éi taaÿ -kʼw éi
		yóo= k- d- \rt[¹]{g̱ax̱} -μμH
		{} yú goox̱ ÿát -kʼw -í {} //
	\glc	\pqp{}fat -\xx{dim} hey fat -\xx{dim} hey
		\xx{quot}= \xx{qual}- \xx{mid}- \rt[¹]{cry} -\xx{var}
		{}[\pr{DP} \xx{dist} slave child -\xx{dim} -\xx{pss} {}] //
	\gld	\pqp{}fat -little hey fat -little hey
		thus \rlap{\xx{impfv}.cry·out} {} {} {}
		{} that slave child -little -of {} //
	\glft	‘“Little fat hey, little fat hey” cries that slave’s little child.’
		//
\endgl
\xe

\ex\label{ex:90-67-little-fat}%
\exmn{264.14}%
\begingl
	\glpreamble	Qō′waēn yū′ān yū-ᴀt-naołigᴀ′s!îỵa. //
	\glpreamble	Ḵoowa.éin yú aan, yú át naa wligásʼi ÿé. //
	\gla	\rlap{Ḵoowa.éin} @ {} @ {} @ {} @ {}
		{} yú aan {} +
		{} yú 
			{} {} \rlap{át} @ {} {}
				naa @ \rlap{wligásʼi} @ {} @ {} @ {} @ {} @ {} {}
			ÿé. {} //
	\glb	ḵu- wu- i- \rt[¹]{.eʼn} -μμH
		{} yú aan {}
		{} yú
			{} {} á -t {}
				naa= wu- l- i- \rt[¹]{gasʼ} -μH -i {}
			ÿé {} //
	\glc	\xx{areal}- \xx{pfv}- \xx{stv}- \rt[¹]{starve} -\xx{var}
		{}[\pr{DP} \xx{dist} town {}]
		{}[\pr{DP} \xx{dist}
			{}[\pr{CP} {}[\pr{PP} \xx{3n} -\xx{pnct} {}]
				clan= \xx{pfv}- \xx{xtn}- \xx{stv}- \rt[¹]{extend} -\xx{var} -\xx{rel} {}]
			place {}] //
	\gld	\rlap{people.\xx{pfv}.starve} {} {} {} {}
		{} that town {}
		{} that
			{} {} there -at {}
				clan \rlap{\xx{pfv}.relocate} {} {} {} {} \·where {}
			place {} //
	\glft	‘People were starving at that town, the place where the clan had moved to.’
		//
\endgl
\xe

\ex\label{ex:90-68-little-fat}%
\exmn{265.1}%
\begingl
	\glpreamble	ᴀ′xo ałᴀx̣ᵘt!. //
	\gla	{} A x̱oo aa {} \rlap{láxwtʼ.} @ {} @ {} //
	\glb	{} a x̱oo aa {} \rt[¹]{laxw} -μH -tʼ //
	\glc	{}[\pr{DP} \xx{3n·pss} among \xx{part} {}] \rt[¹]{starve} -\xx{var} -\xx{rep} //
	\gld	{} them among some {} \rlap{\xx{impfv}.starve.\xx{rep}} {} {} //
	\glft	‘Some among them are starving to death.’
		//
\endgl
\xe

The verb \fm{láxwtʼ} in (\lastx) is imperfective with the repetitive suffix \fm{-tʼ}.
This particular repetitive suffix is associated with the destruction of entities.
Examples include \fm{has náatʼ} ‘they die off’ \parencite[243]{leer:1976}, \fm{asgántʼ} ‘s/he burns them up’ \parencite[645]{leer:1976}, and \fm{ashkélʼtʼ} ‘s/he makes ash of it’ \parencite[700]{leer:1976}.

\ex\label{ex:90-69-chief-suspect}%
\exmn{265.1}%
\begingl
	\glpreamble	Ā′we ā′waqēt yūanqā′wutc yu-adê′q-dag̣ᴀ′xỵa yū′gux ỵê′tk!ᵒ. //
	\glpreamble	Á áwé aawaḵeit yú aanḵáawuch, yú aadé kdag̱áx̱ ÿé, yú goox̱ ÿátkʼu. //
	\gla	{} Á {}
		\rlap{áwé} @ {}
		\rlap{aawaḵeit} @ {} @ {} @ {} @ {}
		{} yú \rlap{aanḵáawuch} @ {} @ {} @ {} {} +
		{} yú
			{} {} \rlap{aadé} @ {} {}
				\rlap{kdag̱áx̱} @ {} @ {} @ {} @ {} {}
			ÿé {}
		{} yú goox̱ \rlap{ÿátkʼu.} @ {} @ {} {} //
	\glb	{} á {}
		á -wé
		a- wu- i- \rt[²]{ḵit} -μμL
		{} yú aan- ḵáaʷ -í -ch {}
		{} yú
			{} {} á -dé {}
				k- d- \rt[¹]{g̱ax̱} -μH {} {}
			ÿé {}
		{} yú goox̱ ÿát -kʼw -í {} //
	\glc	{}[\pr{DP} \xx{3n} {}]
		\xx{foc} -\xx{mdst}
		\xx{arg}- \xx{pfv}- \xx{stv}- \rt[²]{suspect} -\xx{var}
		{}[\pr{DP} \xx{dist} land- man -\xx{pss} -\xx{erg} {}]
		{}[\pr{DP} \xx{dist}
			{}[\pr{CP} {}[\pr{PP} \xx{3n} -\xx{all} {}]
				\xx{qual}- \xx{mid}- \rt[¹]{cry} -\xx{var} \·\xx{rel} {}]
			way {}]
		{}[\pr{DP} \xx{dist} slave child -\xx{dim} -\xx{pss} {}] //
	\glft	‘It was then that the chief began to suspect him, the way that he is crying, that slave’s little child.’
		//
\endgl
\xe

\ex\label{ex:90-70-adding-to-it}%
\exmn{265.2}%
\begingl
	\glpreamble	Ts!ᴀs aka′ ke akᴀnatī′n. //
	\glpreamble	Tsʼas a káa kei akanatéen. //
	\gla	\rlap{Tsʼas} @ {}
		{} a \rlap{káa} @ {} {}
		kei @ \rlap{akanatéen.} @ {} @ {} @ {} @ {} @ {} //
	\glb	tsʼa =s
		{} a ká -μ {}
		kei= a- k- n- \rt[²]{ti} -μμH -n //
	\glc	just =\xx{dub}
		{}[\pr{PP} \xx{3n·pss} \xx{hsfc} -\xx{loc} {}]
		up= \xx{arg}- \xx{sro}- \xx{ncnj}- \rt[²]{handle} -\xx{var} -\xx{nsfx} //
	\gld	just =maybe
		{} it atop -on {}
		up \rlap{3>3.\xx{prog}.handle} {} {} {} {} {} //
	\glft	‘Apparently he was just adding to it.’
		//
\endgl
\xe

The translation in (\lastx) is fairly loose, following \citeauthor{swanton:1909}’s lead.
A literal translation is ‘he is just putting it (small round object) on top of it’.
This describes the little child metaphorically putting its behaviour as a small round object on top of the pile of stressors – e.g.\ starvation – that are already affecting the community.
\citeauthor{swanton:1909}’s gloss note “he was making it louder” is misleading; the loudness of the little child’s crying is not being increased by anything, and instead it is the crying that is making a bad situation worse.

\ex\label{ex:90-71-slaves-child-cries}%
\exmn{265.2}%
\begingl
	\glpreamble	Yūk dag̣ā′x yū′gux ỵê′tk!ᵒ: //
	\glpreamble	Yóo kdag̱áax̱ yú goox̱ ÿátkʼu //
	\gla	Yóo @ \rlap{kdag̱áax̱} @ {} @ {} @ {}
		{} yú goox̱ \rlap{ÿátkʼu} @ {} @ {} {} //
	\glb	yóo= k- d- \rt[¹]{g̱ax̱} -μμH
		{} yú goox̱ ÿát -kʼw -í {}  //
	\glc	\xx{quot}= \xx{qual}- \xx{mid}- \rt[¹]{cry} -\xx{var}
		{}[\pr{DP} \xx{dist} slave child -\xx{dim} -\xx{pss} {}] //
	\gld	thus \rlap{\xx{impfv}.cry·out} {} {} {}
		{} that slave child -little -of {} //
	\glft	‘Thus he cries, that slave’s little child’
		//
\endgl
\xe

\ex\label{ex:90-72-little-fat-hey}%
\exmn{265.3}%
\begingl
	\glpreamble	“Tā′ỵak!we, tā′yak!we.” //
	\glpreamble	«\!taaÿákʼw éi, taayákʼw éi\!». //
	\gla	«\!\rlap{taaÿákʼw} @ {} éi \rlap{taayákʼw} @ {} éi\!». //
	\glb	\pqp{}taaÿ -kʼw éi taaÿ -kʼw éi //
	\glc	\pqp{}fat -\xx{dim} hey fat -\xx{dim} hey //
	\gld	\pqp{}fat -little hey fat -little hey //
	\glft	‘“little fat hey, little fat hey”.’
		//
\endgl
\xe

\ex\label{ex:90-73-thus-he-cries}%
\exmn{265.3}%
\begingl
	\glpreamble	Yūk dag̣ā′x yū′gux ỵêtk!ᵒ. //
	\glpreamble	Yóo kdag̱áax̱ yú goox̱ ÿátkʼu. //
	\gla	Yóo @ \rlap{kdag̱áax̱} @ {} @ {} @ {}
		{} yú goox̱ \rlap{ÿátkʼu.} @ {} @ {} {} //
	\glb	yóo= k- d- \rt[¹]{g̱ax̱} -μμH
		{} yú goox̱ ÿát -kʼw -í {}  //
	\glc	\xx{quot}= \xx{qual}- \xx{mid}- \rt[¹]{cry} -\xx{var}
		{}[\pr{DP} \xx{dist} slave child -\xx{dim} -\xx{pss} {}] //
	\gld	thus \rlap{\xx{impfv}.cry·out} {} {} {}
		{} that slave child -little -of {} //
	\glft	‘Thus he cries, that slave’s little child.’
		//
\endgl
\xe

The phrases in (\ref{ex:90-71-slaves-child-cries}) and (\ref{ex:90-73-thus-he-cries}) are identical brackets around the little child’s speech in (\ref{ex:90-72-little-fat-hey}).
Syntactically either one of (\ref{ex:90-71-slaves-child-cries}) or (\ref{ex:90-73-thus-he-cries}) could be the quotation verb introducing the speech in (\ref{ex:90-72-little-fat-hey}).
\citeauthor{swanton:1909}’s punctuation of (\ref{ex:90-71-slaves-child-cries}) with a final colon \orth{:} suggests that this sentence introduces the quotation and hence that (\ref{ex:90-73-thus-he-cries}) is a repetition for narrative effect.
The run of (\ref{ex:90-71-slaves-child-cries})–(\ref{ex:90-73-thus-he-cries}) is almost a chiasm except that the order of phrases in (\ref{ex:90-73-thus-he-cries}) is not the exact reverse \fm{yú goox̱ ÿátkʼu yóo kdag̱áax̱} of (\ref{ex:90-71-slaves-child-cries}) or vice versa.

\ex\label{ex:90-74-inside-of-clam}%
\exmn{265.4}%
\begingl
	\glpreamble	“G̣a′ʟg̣eỵī′awe ỵīaỵasākᵘ,” duʟāˈ ye ỵawaqa′. //
	\glpreamble	«\!Gáalʼ g̱eiÿí áwé ÿéi aÿasáakw\!» du tláa yéi ÿaawaḵaa. //
	\gla	{} \llap{«\!}Gáalʼ \rlap{g̱eiÿí} @ {} {}
		\rlap{áwé} @ {}
		ÿéi @ \rlap{aÿasáakw\!»} @ {} @ {} @ {} @ {}
		{} du tláa {}
		yéi @ \rlap{ÿaawaḵaa.} @ {} @ {} @ {} @ {} //
	\glb	{} gáalʼ g̱ei -í {}
		á -wé
		ÿéi= a- i- \rt[²]{sa} -μμH -kw
		{} du tláa {}
		yéi= ÿ- wu- i- \rt[¹]{ḵa} -μμL //
	\glc	{}[\pr{DP} clam between -\xx{pss} {}]
		\xx{foc} -\xx{mdst}
		thus= \xx{arg}- \xx{stv}- \rt[²]{call} -\xx{var} -\xx{rep}
		{}[\pr{DP} \xx{3h·pss} mother:\xx{inal} {}]
		thus= \xx{qual}- \xx{pfv}- \rt[¹]{say} -\xx{var} //
	\gld	{} clam inside -of {}
		\rlap{it.is} {}
		thus \rlap{3>3.\xx{impfv}.be.call.\xx{rep}} {} {} {} {}
		{} his mother {}
		thus \rlap{\xx{pfv}.say} {} {} {} {} // 
	\glft	‘“The inside of a clam is what he is calling it” his mother said.’
		//
\endgl
\xe

\ex\label{ex:90-75-some-along-side}%
\exmn{265.5}%
\begingl
	\glpreamble	Xᴀtc kîtcyê′dawe ā′datīn yū′guxtc duỵê′tk!ᵒ q!ēs. //
	\glpreamble	X̱ách kichyát áwé aa datéen, yú goox̱ch du ÿátkʼu x̱ʼéis. //
	\gla	X̱ách
		{} {} \rlap{kichyát} @ {} @ {} {}
		aa @ \rlap{datéen} @ {} @ {} @ {}
		{} yú \rlap{goox̱ch} @ {} {} +
		{} du \rlap{ÿátkʼu} @ {} @ {} \rlap{x̱ʼéis.} @ {} {} //
	\glb	x̱áju
		{} {} kích- ÿá -t {}
		aa= d- \rt[²]{ti} -μμH -n
		{} yú goox̱ -ch {}
		{} du ÿát -kʼw -í x̱ʼé -ÿís {} //
	\glc	actually
		{}[\pr{PP} \xx{rflx·pss} wing- face -\xx{pnct} {}]
		\xx{part·o}= \xx{mid}- \rt[²]{handle} -\xx{var} -\xx{nsfx}
		{}[\pr{DP} \xx{dist} slave -\xx{erg} {}]
		{}[\pr{PP} \xx{3h·pss} child -\xx{dim} -\xx{pss} mouth:\xx{inal} -\xx{ben} {}] //
	\gld	actually
		{} self’s \rlap{side} {} -at {}
		some \rlap{\xx{pos}·\xx{impfv}.handle} {} {} {}
		{} that slave {} {}
		{} her child -little {} mouth -for {} //
	\glft	‘Actually she has some at her side, that slave, for her baby to eat.’
		//
\endgl
\xe

The noun \fm{kichyá} ‘side’ in (\lastx) is relatively uncommon and is elsewhere only documented from Inland Tlingit varieties \parencite[\textsc{t}·38]{leer:2001}, but its appearance here confirms that it must have been more widespread in the past.
It is derived from the more common inalienable noun \fm{kích} ‘wing’ compounded with the inalienable noun \fm{ÿá} ‘face, vertical surface’ and so originally meant something like ‘wing face’ for which it can still be used when talking about birds.
For humans this noun describes the area along the torso that is typically in contact with the inside of the upper arm, i.e.\ the side of the ribcage.
There is probably an etymological connection with \fm{kík} ‘side, symmetric half of, one of a pair’ \parencite[f06/123]{leer:1973}.
The homophonous noun \fm{kích} ‘fine powder’ is probably not directly related, but Proto-Dene \fm[*]{kʸuːdz} \~\ \fm[*]{kʸʊšg} ‘down feathers, fluff’ \parencites[62]{leer:1978b}[\fm{kʸe-kʸu}/42]{leer:1996i} suggests an ancient connection in Proto-Na-Dene.

The phrase \fm{kichyát áwé aa datéen} in (\lastx) features a couple of unrelated grammatical phenomena that can be confusing.
The noun \fm{kichÿá} is inalienable and so requires a possessor, but there is no overt possessor since the preceding word \fm{x̱ách} is a discourse particle.
The answer lies in the verb \fm{datéen}: the presence of \fm{d-} indicates that the noun is actually possessed by an unpronounced reflexive possessor.
The \fm{-t} in \fm{kichyát} and the \fm{-n} in \fm{datéen} together signal a different feature, specifically that the verb is a positional imperfective.
These are imperfective forms derived from handling or motion verbs that describe the position or configuration of an entity at a location in space.
Here \fm{kichyát aa datéen} says that the slave has some of the fat positioned along the side of her body under her arm.

\ex\label{ex:90-76-sit-up-with-it}%
\exmn{265.5}%
\begingl
	\glpreamble	Ān ca′odîqe. //
	\glpreamble	Aan shawdiḵei. //
	\gla	{} \rlap{Aan} @ {} {}
		\rlap{shawdiḵei.} @ {} @ {} @ {} @ {} @ {} //
	\glb	{} á -n {}
		sha- wu- d- i- \rt[¹]{ḵi} -μμL //
	\glc	{}[\pr{PP} \xx{3n} -\xx{instr} {}]
		head- \xx{pfv}- \xx{mid}- \xx{stv}- \rt[¹]{sit·\xx{pl}} -\xx{var} //
	\gld	{} it -with {}
		\rlap{\xx{pfv}.self.sit·\xx{pl}} {} {} {} {} {} //
	\glft	‘They sat up with it.’
		//
\endgl
\xe

\ex\label{ex:90-77-around-mouth-fat}%
\exmn{265.6}%
\begingl
	\glpreamble	Dōq!wᴀ da′wᴀłîtêtʟ!. //
	\glpreamble	Du x̱ʼadaa wulitétlʼ. //
	\gla	{} Du \rlap{x̱ʼadaa} @ {} {}
		\rlap{wulitétlʼ.} @ {} @ {} @ {} @ {} //
	\glb	{} du x̱ʼé- daa {}
		wu- l- i- \rt[¹]{tetlʼ} -μH //
	\glc	{}[\pr{DP} \xx{3h·pss} mouth- around {}]
		\xx{pfv}- \xx{xtn}- \xx{stv}- \rt[¹]{fat} -\xx{var} //
	\gld	{} his mouth -around {}
		\rlap{\xx{pfv}.fat} {} {} {} {} //
	\glft	‘Around his mouth it was fatty.’
		//
\endgl
\xe

The verb in (\lastx) is remarkable because it is not otherwise documented.
The root \fm{\rt[¹]{tetlʼ}} ‘fat’ is attested by the verb \fm{ÿatéitlʼ} ‘it (animal) is fat’ \parencite[378]{leer:1976} and by the nouns \fm{téitlʼ} ‘animal fat’ and \fm{suḵtéitlʼ} ‘goosetongue (\species{Plantago}{maritima}[L.])’ lit.\ ‘fat grass’ with \fm{sooḵ} ‘wide beach grass’ \parencite[06/131]{leer:1973}.
The verb in (\lastx) with \fm{l-} is unknown elsewhere, but it is logically predictable given that \fm{l-} adds an extensional meaning where the fat is smeared along a path or around an area.
The root \fm{\rt[¹]{tetlʼ}} ‘fat’ is etymologically related to the root \fm{\rt[¹]{tesʼ}} ‘flabby, limp’ as seen in the noun \fm{téisʼ} ‘flab’ in (\ref{ex:90-36-hang-on-flab-helper}).

\ex\label{ex:90-78-she-admitted-it-to-him}%
\exmn{265.6}%
\begingl
	\glpreamble	Wananī′sawe ān yên akā′wanîk. //
	\glpreamble	Wáa nanée sáwé aan yan akaawaník. //
	\gla	{} Wáa \rlap{nanée} @ {} @ {} @ {} {}
		\rlap{sáwé} @ {} @ {}
		{} \rlap{aan} @ {} {}
		yan @ \rlap{akaawaník.} @ {} @ {} @ {} @ {} @ {} //
	\glb	{} wáa n- \rt[¹]{ni} -μμH {} {} 
		s= á -wé
		{} á -n {}
		ÿán= a- k- wu- i- \rt[²]{nik} -μH //
	\glc	{}[\pr{CP} how \xx{ncnj}- \rt[¹]{happen} -\xx{var} \·\xx{sub} {}]
		\xx{q}= \xx{foc} -\xx{mdst}
		{}[\pr{PP} \xx{3n} -\xx{instr} {}]
		\xx{term}= \xx{arg}- \xx{qual}- \xx{pfv}- \xx{stv}- \rt[²]{tell} -\xx{var} //
	\gld	{} how \rlap{\xx{csec}.happen} {} {} \·while {}
		some\· \rlap{it.is} {}
		{} him -to {}
		done \rlap{3>3.\xx{pfv}.tell·about} {} {} {} {} {} //
	\glft	‘At some point she admitted it to him.’
		//
\endgl
\xe

\ex\label{ex:90-79-she-told-him-about-him}%
\exmn{265.7}%
\begingl
	\glpreamble	Ye ān akā′wanîk dus!atī′ tîn. //
	\glpreamble	Yéi aan akaawaneek du sʼaatí tin: //
	\gla	Yéi {} \rlap{aan} @ {} {}
		\rlap{akaawaneek} @ {} @ {} @ {} @ {} @ {}
		{} du sʼaatí tin: {} //
	\glb	yéi {} á -n {}
		a- k- wu- i- \rt[²]{nik} -μμL
		{} du sʼaatí tin {} //
	\glc	thus {}[\pr{PP} \xx{3n} -\xx{instr} {}]
		\xx{arg}- \xx{qual}- \xx{pfv}- \xx{stv}- \rt[²]{tell} -\xx{var}
		{}[\pr{PP} \xx{3h·pss} master:\xx{inal} \xx{instr} {}] //
	\gld	thus {} him\ix{i} -to {}
		\rlap{3\ix{j}>3\ix{k}.\xx{pfv}.tell·about} {} {} {} {} {}
		{} her\ix{j} master\ix{i} to {} //
	\glft	‘Thus she\ix{j} told about him\ix{k} to him\ix{i}, her\ix{j} master\ix{i}:’
		//
\endgl
\xe

The interpretation of (\ref{ex:90-79-she-told-him-about-him}) is potentially confusing because there are three third person pronouns.
The subject \fm{akaawaneek} ‘she told about him’ is the slave who is the mother of the little child.
The referent of \fm{aan} ‘to him’ is the same as the referent of \fm{du sʼaatí tin} ‘to her master’, thus the chief mentioned in (\ref{ex:90-69-chief-suspect}).
The object of the verb \fm{akaawaneek} ‘she told about him’ is the protagonist.
The indices \textit{i}, \textit{j}, and \textit{k} in the gloss of (\ref{ex:90-79-she-told-him-about-him}) are meant to clarify this ambiguity: \textit{i} tracks the chief, \textit{j} tracks the slave, and \textit{k} tracks the protagonist.

\ex\label{ex:90-80-hes-there}%
\exmn{265.7}%
\begingl
	\glpreamble	“Ā′wu ho. //
	\glpreamble	«\!Áwu hú. //
	\gla	«\!\rlap{Áwu} @ {} {} hú. {} //
	\glb	\pqp{}á -ú {} hú {} //
	\glc	\pqp{}\xx{3n} -\xx{locp} {}[\pr{DP} \xx{3h} {}] //
	\gld	\pqp{}there -is.at {} he {} //
	\glft	‘“He is there.’
		//
\endgl
\xe

\ex\label{ex:90-81-hes-got-a-lot}%
\exmn{265.7}%
\begingl
	\glpreamble	Aʟ̣ē′n ᴀt cā′ỵᴀlahē′nawe dudjī′.” //
	\glpreamble	Aatlein at shaÿalahéin áwé du jée.\!» //
	\gla	{} \rlap{Aatlein} @ {} at @ \rlap{shaÿalahéin} @ {} @ {} @ {} @ {} @ {} {}
		\rlap{áwé} @ {}
		{} du \rlap{jée.\!»} @ {} {} //
	\glb	{} aa =tlein at= sha- ÿ- l- \rt[²]{hen} -μμH {} {}
		á -wé
		{} du jee -H {} //
	\glc	{}[\pr{DP} \xx{part} =big \xx{4n·o}= head- \xx{qual}- \xx{csv}- \rt[¹]{many} -\xx{var} \·\xx{nmz} {}]
		\xx{cpl} -\xx{mdst}
		{}[\pr{PP} \xx{3h·pss} poss’n -\xx{loc} {}] //
	\gld	{} \rlap{big·one} {} things= \rlap{\xx{impfv}.make.many} {} {} {} {} {} {}
		\rlap{it.is} {}
		{} his poss’n -in {} //
	\glft	‘He has much in his possession.”’
		//
\endgl
\xe

The translation for (\lastx) is fairly loose.
The phrase \fm{aatlein at shaÿalahéin} is a nominalization of the imperfective state verb \fm{ashaÿalihéin} ‘s/he increases it, makes them many’.
This same nominalization occurred previously in (\ref{ex:90-51-cant-be-seen-for-plenty}).
A literal translation of this would be something like ‘it is a lot that is multiplied in his possession’ or ‘it is what is greatly increased in his possession’, but a direct translation of this nominalization is difficult to construct in English because there is a mismatch in eventuality class.

\section{Paragraph 7}\label{sec:90-para-7}

\ex\label{ex:90-82-clan-ordered-there}%
\exmn{265.8}%
\begingl
	\glpreamble	Tc!uʟe′ nāq! ka′odowᴀna′adê. //
	\glpreamble	Chʼu tle naa x̱ʼakawduwanáa aadé. //
	\gla	Chʼu tle
		{} naa {}
		\rlap{x̱ʼakawduwanáa} @ {} @ {} @ {} @ {} @ {} @ {}
		{} \rlap{aadé.} @ {} {} //
	\glb	chʼu tle
		{} naa {}
		x̱ʼe- k- wu- du- i- \rt[²]{na(ʼÿ)} -μμH
		{} á -dé {} //
	\glc	just then
		{}[\pr{DP} clan {}]
		mouth- \xx{qual}- \xx{pfv}- \xx{4h·s}- \xx{stv}- \rt[²]{tell·go} -\xx{var}
		{}[\pr{PP} \xx{3n} -\xx{all} {}] //
	\gld	just then
		{} clan {}
		\rlap{\xx{pfv}.one.tell·to·go} {} {} {} {} {} {}
		{} there -to {} //
	\glft	‘So then the clan was ordered there.’
		//
\endgl
\xe

\citeauthor{swanton:1909} glosses his \orth{nāq!} as “thither”, but this does not obviously connect with known vocabulary.
One possibility is \fm{naa} < \fm{niÿaa} ‘direction’, but this is inalienable and so requires a possessor.
Another possibility is \fm{naa} ‘upriver’ \parencite[04/31]{leer:1973}, but there has been no mention of a river or other current flow so far in this narrative.
If we ignore \citeauthor{swanton:1909}’s gloss then one more possibility is \fm{naa} ‘clan’ which was seen earlier in (\ref{ex:90-8-food-out-moved-away}) and (\ref{ex:90-67-little-fat}); then the entire clan has been ordered to go to the location expressed by the PP \fm{aadé} ‘to there’.
This leaves \citeauthor{swanton:1909}’s \orth{q!} unaccounted for.
This could be the plural suffix \fm{-xʼ} on \fm{naa} ‘clan’, but pluralization of this noun is rare and the narrative otherwise only refers to a single clan.
The verb \fm{akaawanáa} ‘s/he told him/her to go’ is well documented.
Although this verb is not known to occur with \fm{x̱ʼe-} ‘mouth’, its addition is plausible since \fm{x̱ʼe-} is frequently used in speech verbs.

The root \fm{\rt[²]{na(ʼÿ)}} ‘tell to go, send, order away’ seen in (\lastx) is unusual in that it occurs in some contexts as \fm{\rt[²]{na}} and in other contexts as \fm{\rt[²]{naʼÿ}}.
One example of the \fm{\rt[²]{na}} form is \fm{Neildé kawduwanáa atyátxʼi} ‘The children were ordered home’ \parencite[143.1934]{story-naish:1973}, and an example of the \fm{\rt[²]{naʼÿ}} form is \fm{Du sée gáande akawlináay} ‘He kept ordering his daughter outside’ \parencite[143.1938]{story-naish:1973}.
Both root forms apparently occur in the same dialect, but it is not clear from the extant documentation whether they are both available for a single speaker.
The semantic difference between these two root forms, if any, is unknown.

\ex\label{ex:90-83-apparently-he-has-lots}%
\exmn{265.8}%
\begingl
	\glpreamble	Tcayᴀ′x gwâ′yu aʟē′n ᴀt-caỵᴀ′łahēn gwâyū′ dudjī′. //
	\glpreamble	Chʼa a yáx̱ gwáayú aatlein at shaÿalahéin gwáayú du jée. //
	\gla	Chʼa {} a yáx̱ {}
		\rlap{gwáayú} @ {} @ {} +
		{} \rlap{aatlein} @ {} at @ \rlap{shaÿalahéin} @ {} @ {} @ {} @ {} @ {} {}
		\rlap{gwáayú} @ {} @ {} +
		{} du \rlap{jée.} @ {} {} //
	\glb	chʼa {} a yáx̱ {}
		gwá= á -yú
		{} aa =tlein at= sha- ÿ- l- \rt[²]{hen} -μμH {} {}
		gwá= á -yú
		{} du jee -H {} //
	\glc	just {}[\pr{PP} \xx{3n} \xx{sim} {}]
		\xx{mir}= \xx{foc} -\xx{dist}
		{}[\pr{DP} \xx{part} =big \xx{4n·o}= head- \xx{qual}- \xx{csv}- \rt[¹]{many} -\xx{var} \·\xx{nmz} {}]
		\xx{mir}= \xx{cpl} -\xx{dist}
		{}[\pr{PP} \xx{3h·pss} poss’n -\xx{loc} {}] //
	\gld	just {} it like {}
		apparently\• \rlap{it.is} {}
		{} \rlap{big·one} {} things= \rlap{\xx{impfv}.make.many} {} {} {} {} {} {}
		apparently\• \rlap{it.is} {}
		{} his poss’n -in {} //
	\glft	‘Apparently it is just like he seemingly has much in his possession.’
		//
\endgl
\xe

\ex\label{ex:90-84-makes-herself-up}%
\exmn{265.9}%
\begingl
	\glpreamble	ʟ̣ukᴀtctā′dᴀna ʟēq! ᴀtī′ỵia dukā′k cᴀt ᴀcak!ā′ne-a. //
	\glpreamble	Dloowkát sh daadané, tléixʼ ÿateeÿi aa du káak shát, ashikʼáani aa. //
	\gla	Dloowkát
		sh @ \rlap{daadané,} @ {} @ {} @ {}
		{} {} {} tléixʼ \rlap{ÿateeÿi} @ {} @ {} @ {} {} aa {}
			du káak shát, {}
		{} {} \rlap{ashikʼáani} @ {} @ {} @ {} @ {} @ {} {} aa. {} //
	\glb	dleew̃kát
		sh= daa- d- \rt[²]{ne} -μH
		{} {} {} tléixʼ i- \rt[¹]{tiʰ} -μμL -i {} aa {}
			du káak shát {}
		{} {} a- sh- i- \rt[²]{kʼan} -μμH -i {} aa {} //
	\glc	carefully
		\xx{rflx·o}= around- \xx{mid}- \rt[²]{work} -\xx{var}
		{}[\pr{DP} {}[\pr{NP} {}[\pr{CP} one \xx{stv}- \rt[¹]{be} -\xx{var} -\xx{rel} {}] \xx{part} {}]
			\xx{3h·pss} mat·uncle wife {}]
		{}[\pr{DP} {}[\pr{CP} \xx{arg}- \xx{pej}- \xx{stv}- \rt[²]{hate} -\xx{var} -\xx{rel} {}] \xx{part} {}] //
	\gld	carefully
		self \rlap{around.\xx{impfv}.work} {} {} {}
		{} {} {} first \rlap{\xx{impfv}.be} {} {} -which {} one {}
			his mat·uncle wife {}
		{} {} \rlap{3>3.\xx{impfv}.be.hate} {} {} {} {} -who {} one {} //
	\glft	‘She makes herself up carefully, the first one of his uncle’s wives, the one who hates him.’
		//
\endgl
\xe

\ex\label{ex:90-85-wipe-face-fall}%
\exmn{265.10}%
\begingl
	\glpreamble	Du′yeda ᴀłg̣ē′gu ayu′ ᴀtū′x ᴀt wux̣ī′x̣. //
	\glpreamble	Du yadaa alg̱éigu áyú a tóox̱ at wooxeex. //
	\gla	{} {} Du \rlap{yadaa} @ {} {}
			\rlap{alg̱éigu} @ {} @ {} @ {} @ {} @ {} {}
		\rlap{áyú} @ {} +
		{} a \rlap{tóox̱} @ {} {}
		at @ \rlap{wooxeex.} @ {} @ {} @ {} //
	\glb	{} {} du ÿá -daa {}
			a- l- \rt[²]{g̱u} -eμH -k -í {}
		á -yú
		{} a tú -x̱ {}
		at= wu- i- \rt[¹]{xix} -μμL //
	\glc	{}[\pr{CP} {}[\pr{DP} \xx{3h·pss} face- around {}]
			\xx{arg}- \xx{xtn}- \rt[²]{wipe} -\xx{var} -\xx{rep} -\xx{sub} {}]
		\xx{foc} -\xx{dist}
		{}[\pr{PP} \xx{3n·pss} inside -\xx{pert} {}]
		\xx{4n·o}= \xx{pfv}- \xx{stv}- \rt[¹]{fall} -\xx{var} //
	\gld	{} {} her face- around {}
			\rlap{3>3.\xx{impfv}.wipe.\xx{rep}} {} {} {} {} \·when {} \rlap{it.is} {}
		{} its inside -at {} sth\• \rlap{\xx{pfv}.fall} {} {} {} //
	\glft	‘It was as she is wiping her face that something fell into it.’
		//
\endgl
\xe

The \fm{a} ‘its’ of \fm{a tóox̱} ‘at its inside’ in (\lastx) is not explicitly mentioned.
It is probably some kind of rag used for face wiping, what can be called a \fm{yag̱wéinaa} ‘face-wiper’ in Tlingit.
Whatever that fell into this is the item that cuts her cheek in (\nextx).

\ex\label{ex:90-86-cut-cheek}%
\exmn{265.10}%
\begingl
	\glpreamble	Duwᴀckᴀ′ awak!ᴀ′k!ᵘ. //
	\glpreamble	Du washká aawaḵʼékʼw. //
	\gla	{} Du \rlap{washká} @ {} {}
		\rlap{aawaḵʼékʼw.} @ {} @ {} @ {} @ {} //
	\glb	{} du wásh -ká {}
		a- wu- i- \rt[²]{ḵʼekʼw} -μH //
	\glc	{}[\pr{DP} \xx{3h·pss} cheek- \xx{hsfc} {}]
		\xx{arg}- \xx{pfv}- \xx{stv}- \rt[²]{knife·wound} -\xx{var} //
	\gld	{} her cheek -surface {}
		\rlap{3>3.\xx{pfv}.knife·wound} {} {} {} {} //
	\glft	‘She cut the surface of his cheek.’
		//
\endgl
\xe

The verb \orth{awak!ᴀ′k!ᵘ} in \citeauthor{swanton:1909}’s transcription and his gloss “she cut” are enough to identify this verb as the one that describes a knife wound or other slashing cut into the surface of something.
The root is not especially well documented and what little documentation there is suggests some variation in its pronunciation.
\citeauthor{story-naish:1973} have it as \fm{\rt[²]{ḵʼekʼw}} with an onset uvular and a coda velar \parencite[61.711]{story-naish:1973}.
\citeauthor{leer:1973} has the same from Tongass Tlingit, but then also gives a noun \fm{ḵʼéiḵʼw} ‘knife wound’ with a coda uvular \parencite[f01/230]{leer:1973}.
Later in his stem list \citeauthor{leer:1978b} instead has only \fm{ḵʼéikʼw} \parencite[79]{leer:1978b}, but then in his noun database (circa 1995) he once again has \fm{ḵʼéiḵʼw}.
It is unclear if this variation in documentation reflects dialect differences or other variation in pronunciation, or if instead this is due to errors in transcription.
At least the onset is always uvular, so \citeauthor{swanton:1909}’s transcription has been interpreted accordingly.
The nearly homophonous noun \fm{ḵʼeiḵʼw} ‘sea pigeon, tern’ (Sternini) is probably unrelated but it could plausibly be a source of lexical interference with the pronunciation of the verb root.

\ex\label{ex:90-87-thought-well-of}%
\exmn{265.11}%
\begingl
	\glpreamble	Wē′doq!es g̣ā′s!-k!î ītī′dî ᴀt wug̣ē′q!êa qo′a k!edē′n ᴀt tō′ditᴀn. //
	\glpreamble	Wé du x̱ʼéis gáasʼ kʼi.eetéede at woog̱éixʼi aa ḵu.aa, kʼédéin át tuwditán. //
	\gla	{} Wé {} {} du \rlap{x̱ʼéis} @ {} {}
				{} gáasʼ \rlap{kʼi.eetéede} @ {} @ {} {} +
				at @ \rlap{woog̱éixʼi} @ {} @ {} @ {} @ {} {} aa {}
		ḵu.aa +
		\rlap{kʼédéin} @ {} @ {}
		{} \rlap{át} @ {} {}
		\rlap{tuwditán.} @ {} @ {} @ {} @ {} @ {} //
	\glb	{} wé {} {} du x̱ʼé -ÿís {}
				{} gáasʼ kʼí- eetí -dé {}
				at= wu- i- \rt[²]{g̱ixʼ} -μμH -i {} aa {}
		ḵu.aa
		\rt[¹]{kʼe} -μH -déin
		{} á -t {}
		tu- wu- d- i- \rt[²]{tan} -μH //
	\glc	{}[\pr{DP} \xx{mdst} {}[\pr{CP} {}[\pr{PP} \xx{3h·pss} mouth -\xx{ben} {}]
				{}[\pr{PP} housepost base- remains -\xx{all} {}]
				\xx{4n·o}= \xx{pfv}- \xx{stv}- \rt[²]{throw·\xx{sg}} -\xx{var} -\xx{rel} {}] \xx{part} {}]
		\xx{contr}
		\rt[¹]{good} -\xx{var} -\xx{adv}
		{}[\pr{PP} \xx{3n} -\xx{pnct} {}]
		inside- \xx{pfv}- \xx{mid}- \xx{stv}- \rt[²]{hdl·w/e} -\xx{var} //
	\gld	{} that {} {} his mouth -for {}
				{} housepost base- hole -to {}
				sth \rlap{\xx{pfv}.throw·\xx{sg}} {} {} {} {} {} one {}
		however
		\rlap{well} {} {}
		{} her -on {}
		\rlap{\xx{pfv}.think} {} {} {} {} {} //
	\glft	‘That one who had thrown something in a housepost pit for him to eat however,
		he thought well of her.’
		//
\endgl
\xe

\ex\label{ex:90-88-townspeople-relocate}%
\exmn{265.12}%
\begingl
	\glpreamble	Tc!uʟe′ doxᴀ′nt naołîg̣ᴀ′s! yū′āntqenî. //
	\glpreamble	Chʼu tle du x̱ánt naa wligáasʼ yú aantḵeiní. //
	\gla	Chʼu tle
		{} du \rlap{x̱ánt} @ {} {}
		naa @ \rlap{wligáasʼ} @ {} @ {} @ {} @ {} +
		{} yú \rlap{aantḵeiní.} @ {} @ {} @ {} @ {} @ {} {} //
	\glb	chʼu tle
		{} du x̱án -t {}
		naa= wu- l- i- \rt[¹]{gasʼ} -μμH
		{} yú aan- d- \rt[¹]{ḵi} -μμL -n -í {} //
	\glc	just then
		{}[\pr{PP} \xx{3h·pss} near -\xx{pnct} {}]
		clan= \xx{pfv}- \xx{csv}- \xx{stv}- \rt[¹]{extend} -\xx{var}
		{}[\pr{DP} \xx{dist} town- \xx{mid}- \rt[¹]{sit·\xx{pl}} -\xx{var} -\xx{nsfx} -\xx{nmz} {}] //
	\gld	just then
		{} his near -to {}
		clan \rlap{\xx{pfv}.relocate} {} {} {} {}
		{} those \rlap{townspeople} {} {} {} {} {} {} // 
	\glft	‘So then the clan moved near to him, those townspeople.’
		//
\endgl
\xe

\ex\label{ex:90-89-no-taste-for-food}%
\exmn{265.12}%
\begingl
	\glpreamble	Yudukā′k qo′a ye ᴀtū′dîtᴀn, yū′ᴀtxā łq!ē′a kū′nᴀx du′nugu qa dukā′k cāt. //
	\glpreamble	Yú du káak ḵu.aa yéi át tuwditán yú atx̱á l x̱ʼéi akoonax̱danoogú, ḵa du káak shát. //
	\gla	{} Yú du káak {}
		ḵu.aa
		yéi {} \rlap{át} @ {} {}
		\rlap{tuwditán} @ {} @ {} @ {} @ {} @ {} +
		{} {} yú atx̱á {} 
			l {} {} \rlap{x̱ʼéi} @ {} {}
			\rlap{akoonax̱danoogú,} @ {} @ {} @ {} @ {} @ {} @ {} @ {} @ {} @ {} +
		ḵa {} du káak shát. {} //
	\glb	{} yú du káak {}
		ḵu.aa
		yéi {} á -t {}
		tu- wu- d- i- \rt[²]{tan} -μH
		{} {} yú atx̱á {}
			l {} {} x̱ʼé -μ {}
			a- k- u- n- g̱- d- \rt[²]{nikw} -μμL -í {}
		ḵa {} du káak shát {} //
	\glc	{}[\pr{DP} \xx{dist} \xx{3h·pss} mat·uncle {}]
		\xx{contr}
		thus {}[\pr{PP} \xx{3n} -\xx{pnct} {}]
		mind- \xx{pfv}- \xx{mid}- \xx{stv}- \rt[²]{hdl·w/e} -\xx{var}
		{}[\pr{CP} {}[\pr{DP} \xx{dist} food {}]
			\xx{neg} {}[\pr{PP} \xx{rflx·pss} mouth -\xx{loc} {}]
			\xx{arg}- \xx{qual}- \xx{irr}- \xx{ncnj}- \xx{mod}- \xx{mid}- \rt[²]{feel} -\xx{var} -\xx{sub} {}]
		and {}[\pr{DP} \xx{3h·pss} mat·uncle wife {}] //
	\gld	{} that his mat·uncle {}
		however
		thus {} it -on {}
		\rlap{\xx{pfv}.decide} {} {} {} {} {}
		{} {} that food {}
			not {} self’s mouth -to {}
			\rlap{3>3.\xx{pot}.feel} {} {} {} {} {} {} {} -that {}
		and {} his mat·uncle’s wife {} //
	\glft	‘His uncle however decided that he had no sense of taste for the food, and his uncle’s wife.’
		//
\endgl
\xe

The sentence in (\lastx) is a bit odd with its coordination at the end.
In typical  modern Tlingit we would expect something like \fm{… ḵa du káak shát tsú} ‘and his uncle’s wife also’ or just \fm{… du káak shát tsú} ‘his uncle’s wife too’.
The lack of the additive focus particle \fm{tsú} here is grammatically strange, but the sentence is still interpretable.
This grammaticality problem suggests that there is an ellipsis or raising phenomenon with coordination here that is normally prohibited, but this needs more investigation.

\ex\label{ex:90-90-rigor-mortis}%
\exmn{265.13}%
\begingl
	\glpreamble	Tc!a a′dê taỵē′dî awe′ kaołît!î′k dukā′k qa dukā′k cᴀt. //
	\glpreamble	Chʼa aadé taÿeedé áwé kawlitʼík, du káak ḵa du káak shát. //
	\gla	Chʼa {} \rlap{aadé} @ {} {}
		{} \rlap{taÿeedé} @ {} @ {} {}
		\rlap{áwé} @ {}
		\rlap{kawlitʼík,} @ {} @ {} @ {} @ {} @ {} +
		{} du káak {} 
		ḵa
		{} du káak shát. {} //
	\glb	chʼa {} á -dé {}
		{} tá- ÿee -dé {}
		á -wé
		k- wu- l- i- \rt[¹]{tʼik} -μH
		{} du káak {}
		ḵa
		{} du káak shát {} //
	\glc	just {}[\pr{PP} \xx{3n} -\xx{all} {}]
		{}[\pr{PP} bed- below -\xx{all} {}]
		\xx{foc} -\xx{mdst}
		\xx{qual}- \xx{pfv}- \xx{xtn}- \xx{stv}- \rt[¹]{stiff} -\xx{var}
		{}[\pr{DP} \xx{3h·pss} mat·uncle {}]
		and
		{}[\pr{DP} \xx{3h·pss} mat·uncle wife {}] //
	\gld	just {} it -way {}
		{} bed- below -to {}
		\rlap{it.is} {}
		\rlap{\xx{pfv}.rigor·mortis} {} {} {} {} {}
		{} his mat·uncle {}
		and
		{} his mat·uncle’s wife {} //
	\glft	‘It was somehow in bed that they became stiff, his uncle and his uncle’s wife.’
		//
\endgl
\xe

The verb \fm{kawlitʼík} ‘s/he/it became stiff’ in (\lastx) specifically refers to rigor mortis, the stiffening of the body after death.
This can be used for humans as in this context, but it is also used for other animals such as fish in e.g.\ \fm{x̱áat kawlitʼík} ‘the salmon was stiff (after death)’ \parencite[212.2981]{story-naish:1973}.
The root \fm{\rt[¹]{tʼik}} is also used in other verbs to describe stiffness as in \fm{daa.ittunéekwch du jín wusitʼík} ‘her hand has become stiff with arthritis’ \parencite[212.2980]{story-naish:1973}.
The root \fm{\rt[¹]{tʼixʼ}} ‘hard, frozen’ is etymologically related although the historical details are still unclear.
The homophonous root \fm{\rt[¹]{tʼik}} ‘steer with paddle’ is not obviously related.

\ex\label{ex:90-91-marry-uncles-wife}%
\exmn{265.14}%
\begingl
	\glpreamble	Yu-ᴀcī′t-wudacī′ỵia dukā′k cᴀt ʟe ā′waca. //
	\glpreamble	Yú ash eet wudishéeÿi aa du káak shát tle aawasháa. //
	\gla	{} Yú {} {} {} ash \rlap{eet} @ {} {}
				\rlap{wudishéeÿi} @ {} @ {} @ {} @ {} @ {} {} aa {}
			du káak + shát {}
		tle
		\rlap{aawasháa.} @ {} @ {} @ {} @ {} //
	\glb	{} yú {} {} {} ash ee -t {}
				wu- d- i- \rt[¹]{shiʰ} -μμH -i {} aa {}
			du káak shát {}
		tle
		a- wu- i- \rt[²]{shaʷ} -μμH //
	\glc	{}[\pr{DP} \xx{dist} {}[\pr{NP} {}[\pr{CP} {}[\pr{PP} \xx{3prx} \xx{base} -\xx{pnct} {}]
				\xx{pfv}- \xx{mid}- \xx{stv}- \rt[¹]{reach} -\xx{var} -\xx{rel} {}] \xx{part} {}]
			\xx{3h·pss} mat·uncle wife {}]
		then
		\xx{arg}- \xx{pfv}- \xx{stv}- \rt[²]{woman} -\xx{var} //
	\gld	{} that {} {} {} him {} -to {}
				\rlap{\xx{pfv}.help} {} {} {} {} -who {} one {}
			his mat·uncle’s wife {}
		then
		\rlap{3>3.\xx{pfv}.marry} {} {} {} {} //
	\glft	‘That one of his uncle’s wives who had helped him, he then married her.’
		//
\endgl
\xe

\ex\label{ex:90-92-stuff-sold-for-slaves}%
\exmn{266.1}%
\begingl
	\glpreamble	Yū′-ān-duī-g̣ā′-qowasū′-ᴀt qo′a awe′ gūx g̣ā awahū′n. //
	\glpreamble	Yú aan du eeg̱áa ḵoowasoowu át ḵu.aa áwé goox̱g̱áa aawahoon. //
	\gla	{} Yú {} {} \rlap{aan} @ {} {}
				{} du \rlap{eeg̱áa} @ {} {}
				\rlap{ḵoowasoowu} @ {} @ {} @ {} @ {} @ {} {} át {}
		ḵu.aa
		\rlap{áwé} @ {}
		{} \rlap{goox̱g̱áa} @ {} {}
		\rlap{aawahoon.} @ {} @ {} @ {} @ {} //
	\glb	{} yú {} {} á -n {}
				{} du ee -g̱áa {}
				ḵu- wu- i- \rt[¹]{su} -μμL -i {} át {}
		ḵu.aa
		á -wé
		{} goox̱ -g̱áa {}
		a- wu- i- \rt[²]{hun} -μμL //
	\glc	{}[\pr{DP} \xx{dist} {}[\pr{CP} {}[\pr{PP} \xx{3n} -\xx{instr} {}]
				{}[\pr{PP} \xx{3h} \xx{base} -\xx{ades} {}]
				\xx{4h·o}- \xx{pfv}- \xx{stv}- \rt[¹]{sup·help} -\xx{var} -\xx{rel} {}] thing {}]
		\xx{contr}
		\xx{foc} -\xx{mdst}
		{}[\pr{PP} slave -\xx{ades} {}]
		\xx{arg}- \xx{pfv}- \xx{stv}- \rt[²]{sell} -\xx{var} //
	\gld	{} that {} {} it -with {}
				{} him {} -for {}
				\rlap{one.\xx{pfv}.super·help} {} {} {} {} -that {} thing {}
		however
		\rlap{it.is} {}
		{} slave -for {}
		\rlap{3>3.\xx{pfv}.sell} {} {} {} {} //
	\glft	‘It is that stuff that he was given supernatural help with however that he sold for slaves.’
		//
\endgl
\xe

The relative clause \fm{yú aan du eeg̱áa ḵoowasoowu át} in (\lastx) translates literally to ‘that thing that someone did supernatural help for him with’.
The supernatural helper is not explicitly mentioned here, instead being referred to in the background by the fourth person (indefinite, nonspecific) object \fm{ḵu-} ‘someone, people”.
The thing described is not the supernatural helper, but rather the wealth of food magically provided to the protagonist.
This is reflected by \citeauthor{swanton:1909}’s gloss of the clause as “the food his helper got for him”, though \fm{át} is literally just ‘thing’.
The English translation of the clause is passive rather than active with ‘someone’ as a subject because unlike in Tlingit it is unusual in English to use an indefinite pronoun for an already established referent.

\ex\label{ex:90-93-townspeople-bought-it}%
\exmn{266.2}%
\begingl
	\glpreamble	Duī′tx yᴀx ỵa′odudzî-ū āntqeni′. //
	\glpreamble	Du eetx̱ yax̱ ÿawdudzi.óo antḵeiní. //
	\gla	{} Du \rlap{eetx̱} @ {} {}
		yax̱ @ \rlap{ÿawdudzi.óo} @ {} @ {} @ {} @ {} @ {} @ {} @ {} +
		{} \rlap{aantḵeiní.} @ {} @ {} @ {} @ {} @ {} {} //
	\glb	{} du ee -dáx̱ {}
		ÿáx̱= ÿ- wu- du- d- s- i- \rt[²]{.uʰ} -μμH
		{} aan- d- \rt[¹]{ḵi} -μμL -n -í {} //
	\glc	{}[\pr{PP} \xx{3h} \xx{base} -\xx{abl} {}]
		\xx{exh}= \xx{qual}- \xx{pfv}- \xx{4h·s}- \xx{mid}- \xx{xtn}- \xx{stv}- \rt[²]{buy} -\xx{var}
		{}[\pr{DP} town- \xx{mid}- \rt[¹]{sit·\xx{pl}} -\xx{var} -\xx{nsfx} -\xx{nmz} {}] //
	\gld	{} him {} -from {}
		all \rlap{\xx{pfv}.people.buy} {} {} {} {} {} {} {}
		{} \rlap{townspeople} {} {} {} {} {} {} //
	\glft	‘They bought it all up from him, the townspeople.’
		//
\endgl
\xe

The verb in (\lastx) illustrates the application of a form of the exhaustive derivation, here specifically \fm{ÿáx̱=} + \fm{ÿ-} + \fm{s-}.
Exhaustives indicate that the event somehow uses up most if not all of the object, such as the contrast between \fm{x̱waax̱áa} ‘I ate it’ versus \fm{yax̱ yax̱wsix̱áa} ‘I ate it up’ \parencite[see][219]{leer:1991}.
The object in (\lastx) is not explicitly stated, but can be inferred as identical to the object in (\ref{ex:90-92-stuff-sold-for-slaves}), i.e.\ the stuff given through supernatural help to the protagonist which he sold for slaves.
Thus (\ref{ex:90-92-stuff-sold-for-slaves}) establishes the sale and (\ref{ex:90-93-townspeople-bought-it}) describes the complete (exhaustive) purchase by the townspeople of all of the sold material.

The use of fourth person human \fm{du-} ‘someone, people; they’ in (\lastx) is syntactically interesting.
The \fm{du-} subject is typically indefinite, but occasionally it refers to a definite entity who is in the discourse background (what \cite{leer:1991} calls ‘recessive’).
This backgrounding use of \fm{du-} normally has the same syntax as the indefinite use of \fm{du-} where it saturates the subject argument and so no coreferential DP occurs in the core clause.
The sentence in (\lastx) is a rare example of an overt DP \fm{aantḵeiní} ‘townspeople’ in the same sentence and coreferential with \fm{du-}.
Since the DP is in the right periphery rather than the core clause preceding the verb word, it is presumably not the argument of the verb and so it does not compete with \fm{du-} for saturating the argument.

\ex\label{ex:90-94-made-a-container}%
\exmn{266.2}%
\begingl
	\glpreamble	Yên kudagā′awe ᴀdake′tk!e aosîne′ yu-ᴀcīg̣ā′-wusuwu′-ᴀt. //
	\glpreamble	Yan kudagáa áwé a daakeitkʼí yéi awsinei, yú ash eeg̱áa woosoowu át. //
	\gla	{} Yan @ \rlap{kudagáa} @ {} @ {} @ {} @ {} @ {} @ {} {}
		\rlap{áwé} @ {} +
		{} a \rlap{daakeitkʼí} @ {} @ {} @ {} @ {} {}
		yéi @ \rlap{awsinei,} @ {} @ {} @ {} @ {} @ {} +
		{} yú {} {} ash \rlap{eeg̱áa} @ {} {}
				\rlap{woosoowu} @ {} @ {} @ {} @ {} {} át. {} //
	\glb	{} ÿán= k- u- {} d- \rt[²]{ga} -μμH {} {}
		á -wé
		{} a daa- ká- át -kʼ -í {}
		yéi= a- wu- s- i- \rt[²]{niʰ} -μμL
		{} yú {} {} ash ee -g̱áa {}
			wu- i- \rt[¹]{suʰ} -μμL -i {} át {} //
	\glc	{}[\pr{CP} \xx{term}= \xx{qual}- \xx{irr}- \xx{zcnj}\· \xx{mid}- \rt[²]{wait} -\xx{var} \·\xx{sub} {}]
		\xx{foc} -\xx{mdst}
		{}[\pr{DP} \xx{3n·pss} around- \xx{hsfc}- thing -\xx{dim} -\xx{pss} {}]
		thus= \xx{arg}- \xx{pfv}- \xx{csv}- \xx{stv}- \rt[¹]{happen} -\xx{var}
		{}[\pr{DP} \xx{dist} {}[\pr{CP} {}[\pr{PP} \xx{3prx} \xx{base} -\xx{ades} {}]
				\xx{pfv}- \xx{stv}- \rt[¹]{sup·help} -\xx{var} -\xx{rel} {}] thing {}] //
	\gld	{} end \rlap{\xx{csec}.delay} {} {} {} {} {} \·when {}
		\rlap{it.is} {}
		{} its \rlap{container} {} {} -little {} {}
		thus \rlap{3>3.\xx{pfv}.make.happen} {} {} {} {} {}
		{} that {} {} him {} -for {}
				\rlap{\xx{pfv}.super·help} {} {} {} -that {} thing {} //
	\glft	‘It was after a while that he made a container for it, that thing that gave supernatural help to him.’
		//
\endgl
\xe

\ex\label{ex:90-95-nobody-ever-saw-it}%
\exmn{266.3}%
\begingl
	\glpreamble	ʟēł adu′tsa ye ustî′ntc //
	\glpreamble	Tléil aadóoch sá yei oostínch. //
	\gla	Tléil {} {} \rlap{aadóoch} @ {} {} sá {}
		yei @ \rlap{oostínch.} @ {} @ {} @ {} @ {} @ {} //
	\glb	tléil {} {} aadóo -ch {} sá {}
		yei= a- u- s- \rt[²]{tin} -μH -ch //
	\glc	\xx{neg} {}[\pr{QP} {}[\pr{DP} who -\xx{erg} {}] \xx{q} {}]
		down= \xx{arg}- \xx{zpfv}- \xx{xtn}- \rt[²]{see} -\xx{var} -\xx{rep} //
	\gld	not {} {} who {} {} ever {}
		down \rlap{3>3.\xx{hab}.see} {} {} {} {} {} //
	\glft	‘Nobody ever saw it.’
		//
\endgl
\xe

\ex\label{ex:90-96-out-of-sight}%
\exmn{266.3}%
\begingl
	\glpreamble	tcaqā′wᴀq wᴀnt!ē′q!ayu. //
	\glpreamble	Chʼa ḵaa waḵwantʼéixʼ áyú. //
	\gla	Chʼa {} ḵaa \rlap{waḵwantʼéixʼ} @ {} @ {} @ {} {} \rlap{áyú.} @ {} //
	\glb	chʼa {} ḵaa waaḵ- wán- tʼéiᵏ -xʼ {} á -yú //
	\glc	just {}[\pr{PP} \xx{4h·pss} eye- edge- behind -\xx{loc} {}] \xx{cpl} -\xx{dist} //
	\gld	just {} one’s vision- edge- behind -at {} \rlap{it.is} {} //
	\glft	‘It is out of sight.’
		//
\endgl
\xe

\section{Paragraph 8}\label{sec:90-para-8}

\ex\label{ex:90-97-whales-show-faces}%
\exmn{266.5}%
\begingl
	\glpreamble	Dekī′x ỵā′ỵê′ndaxun yā′î. //
	\glpreamble	Deikéex̱ ÿaa ÿandaxún yáay. //
	\gla	{} \rlap{Deikéex̱} @ {} {}
		ÿaa @ \rlap{ÿandaxún} @ {} @ {} @ {} @ {}
		{} yáay. {} //
	\glb	{} deikée -x̱ {}
		ÿaa= ÿ- n- d- \rt[¹]{xun} -μH
		{} yáaÿ {} //
	\glc	{}[\pr{PP} offshore -\xx{pert} {}]
		along= face- \xx{ncnj}- \xx{mid}- \rt[¹]{drift·\xx{pl}} -\xx{var}
		{} whale {} //
	\gld	{} offshore -at {}
		along \rlap{face.\xx{prog}.show·\xx{pl}} {} {} {} {}
		{} whale {} //
	\glft	‘Out at sea whales are showing their faces.’
		//
\endgl
\xe

\ex\label{ex:90-98-release-it-there}%
\exmn{266.5}%
\begingl
	\glpreamble	Ā′de ᴀtcī′wanᴀq. //
	\glpreamble	Aadé ajeewanáḵ. //
	\gla	{} \rlap{Aadé} @ {} {}
		\rlap{ajeewanáḵ.} @ {} @ {} @ {} @ {} @ {} //
	\glb	{} á -dé {}
		a- ji- wu- i- \rt[²]{naḵ} -μH //
	\glc	{} \xx{3n} -\xx{all} {}
		\xx{arg}- hand- \xx{pfv}- \xx{stv}- \rt[²]{abandon} -\xx{var} //
	\gld	{} there -to {}
		\rlap{3>3.\xx{pfv}.release} {} {} {} {} {} //
	\glft	‘He released it there.’
		//
\endgl
\xe

\ex\label{ex:90-99-move-whales-along-beach}%
\exmn{266.5}%
\begingl
	\glpreamble	Ts!ūtā′tayu eqêg̣ayā′nᴀx yên akā′waha yuyā′î ʟᴀnq!. //
	\glpreamble	Tsʼootaat áyú éiḵ ig̱ayáanáx̱ yan akaawaháa yú yáay tlénxʼ. //
	\gla	{} Tsʼootaat {} \rlap{áyú} @ {}
		{} éiḵ {}
		{} \rlap{ig̱ayáanáx̱} @ {} @ {} {}
		yan @ \rlap{akaawaháa} @ {} @ {} @ {} @ {} @ {}
		{} yú yáay \rlap{tlénxʼ.} @ {} {} //
	\glb	{} tsʼootaat {} á -yú
		{} éeḵ {}
		{} eeḵ- ÿáᵏ -náx̱ {}
		ÿán= a- k- wu- i- \rt[²]{ha} -μμH
		{} yú yáaÿ tlein -xʼ {} //
	\glc	{}[\pr{NP} morning {}] \xx{foc} -\xx{dist}
		{}[\pr{NP} beach {}]
		{}[\pr{PP} beach- face -\xx{perl} {}]
		\xx{term}= \xx{arg}- \xx{sro}- \xx{pfv}- \xx{stv}- \rt[²]{mv·mass} -\xx{var}
		{}[\pr{DP} \xx{dist} whale big -\xx{pl} {}] //
	\gld	{} morning {} \rlap{it.is} {}
		{} beach {}
		{} beach- face -along {}
		done\• \rlap{3>3.\xx{zcnj}.\xx{pfv}.move·mass} {} {} {} {} {}
		{} those whale big -\xx{pl} {} //
	\glft	‘It was in the morning on the beach that it had moved them along the beach below,
		those big whales.’
		//
\endgl
\xe

The preverb \fm{ÿán=} in (\lastx) is semantically ambiguous in this context.
It is typically used for termination as part of the motion derivation \vblex{ÿán= \~\ ÿáx̱= \~\ ÿán-de}{∅}{\fm{-μμL} repetitive}{finishing, ending}.
But as discussed in detail on page \pageref{note:100-shore-discussion} in chapter \ref{ch:100-salmon-boy-wrg}, the noun \fm{ÿán} from which the preverb developed normally refers to the shoreline of a body of water, this from an earlier \fm[*]{ŋan} ‘ground, earth’ (PND \fm[*]{ŋənˀ}) whose vestiges are still seen in some noun compounds and in a few uncommon usages.
In (\lastx) the \fm{ÿán=} preverb could be plausibly interpreted as either the terminative or as the literal ‘shoreline’ since the mass of whales are offshore according to (\ref{ex:90-97-whales-show-faces}).
The English translation takes the liberty of not reflecting either interpretation, leaving both implicitly possible.

\ex\label{ex:90-100-while-busy-forgot}%
\exmn{266.6}%
\begingl
	\glpreamble	Tc!u atā′t qoỵa′ostāg̣e ayu′ yuyā′î akᴀ′tsiwᴀq!ᴀkᵘ. //
	\glpreamble	Chʼu a daat ḵuÿawustaag̱í áyú yú yáay, a kát seiwaxʼáḵw. //
	\gla	{} Chʼu {} a \rlap{daat} @ {} {}
			\rlap{ḵuÿawustaag̱í} @ {} @ {} @ {} @ {} @ {} @ {} {}
		\rlap{áyú} @ {} +
		{} yú yáay, {}
		{} a \rlap{kát} @ {} {}
		\rlap{seiwaxʼáḵw} @ {} @ {} @ {} @ {} //
	\glb	{} chʼu {} a daa -t {}
			ḵu- ÿ- wu- s- \rt[¹]{taḵ} -μμL -í {}
		á -yú
		{} yú yáaÿ {}
		{} a ká -t {}
		se- wu- i- \rt[¹]{xʼaḵw} -μH //
	\glc	{}[\pr{CP} just {}[\pr{PP} \xx{3n} around -\xx{pnct} {}]
			\xx{4h·o}- \xx{qual}- \xx{pfv}- \xx{xtn}- \rt[¹]{attend} -\xx{var} -\xx{sub} {}]
		\xx{foc} -\xx{dist}
		{}[\pr{DP} \xx{dist} whale {}]
		{}[\pr{PP} \xx{3n} \xx{hsfc} -\xx{pnct} {}]
		voice- \xx{pfv}- \xx{stv}- \rt[¹]{die·off} -\xx{var} //
	\gld	{} just {} them about -to {}
			\rlap{people.\xx{pfv}.attend} {} {} {} {} {} -while {}
		\rlap{it.is} {}
		{} those whales {}
		{} it \rlap{about} {} {}
		\rlap{\xx{pfv}.forget} {} {} {} {} //
	\glft	‘So it was while people were taking care of them, those whales, that he forgot about it.’
		//
\endgl
\xe

\ex\label{ex:90-101-last-one-hanging-there-forgot}%
\exmn{266.8}%
\begingl
	\glpreamble	Yuhū′tc!î-ā′ỵê da′de q!ax̣ᴀ′tî.
Akᴀ′tsiwaq!ᴀkᵘ //
	\glpreamble	Yú hóochʼi aaÿí daadé x̱ʼaxádi, a kát seiwaxʼáḵw. //
	\gla	{} {} {} Yú hóochʼi \rlap{aaÿí} @ {} {} \rlap{daadé} @ {} {}
			\rlap{x̱ʼaxádi,} @ {} @ {} @ {} {} +
		{} a \rlap{kát} @ {} {}
		\rlap{seiwaxʼáḵw.} @ {} @ {} @ {} @ {} //
	\glb	{} {} {} yú hóochʼ aa -í {} daa -dé {}
			x̱ʼe- \rt[¹]{xat} -μH -í {}
		{} a ká -t {}
		se- wu- i- \rt[¹]{xʼaḵw} -μH //
	\glc	{}[\pr{CP} {}[\pr{PP} {}[\pr{DP} \xx{dist} last \xx{part} -\xx{pss} {}] around -\xx{all} {}]
			mouth- \rt[¹]{suspend} -\xx{var} -\xx{sub} {}]
		{}[\pr{PP} \xx{3n} \xx{hsfc} -\xx{pnct} {}]
		voice- \xx{pfv}- \xx{stv}- \rt[¹]{die·off} -\xx{var} //
	\gld	{} {} {} that last one -of {} outside -to {}
			\rlap{\xx{impfv}.be·fastened} {} {} {} {}
		{} it \rlap{about} {} {}
		\rlap{\xx{pfv}.forget} {} {} {} {} //
	\glft	‘Hanging on the outside of that last one, he forgot about it.’
		//
\endgl
\xe

\ex\label{ex:90-102-theyre-taking-it-up-that-whale}%
\exmn{266.8}%
\begingl
	\glpreamble	dāq kᴀdudjē′łayu yuyā′î. // 
	\glpreamble	Daaḵ kadujéil áyú yú yáay. //
	\gla	Daaḵ @ \rlap{kadujéil} @ {} @ {} @ {}
		\rlap{áyú} @ {}
		{} yú yáay. {} //
	\glb	daaḵ= k- du- \rt[²]{jel} -μμH
		á -yú
		{} yú yáaÿ {} //
	\glc	inland= \xx{qual}- \xx{4h·s}- \rt[²]{lug} -\xx{var}
		\xx{foc} -\xx{dist}
		{}[\pr{DP} \xx{dist} whale {}] //
	\gld	inland \rlap{\xx{impfv}.people.lug} {} {} {}
		\rlap{it.is} {}
		{} those whales {} //
	\glft	‘So they are lugging them up, those whales.’
		//
\endgl
\xe

\ex\label{ex:90-103-because-forgot-died-out}%
\exmn{266.8}%
\begingl
	\glpreamble	Tc!uʟe′ akᴀ′ tsa wu!ag̣ō′djayu qot cū′wax̣ix̣ yū′antqenî. //
	\glpreamble	Chʼu tle a kát sawuxʼaag̱úch áyú ḵutx̱ shoowaxeex yú aantḵeiní. //
	\gla	Chʼu tle {} {} {} a \rlap{kát} @ {} {}
			\rlap{sawuxʼaag̱úch} @ {} @ {} @ {} @ {} {} {} {}
		\rlap{áyú} @ {} +
		{} \rlap{ḵutx̱} @ {} {} \rlap{shoowaxeex} @ {} @ {} @ {} @ {}
		{} yú \rlap{aantḵeiní.} @ {} @ {} @ {} @ {} @ {} {} //
	\glb	chʼu tle {} {} {} a ká -t {}
			se- wu- \rt[¹]{xʼaḵw} -μμL -í {} -ch {}
		á -yú
		{} ḵú -dáx̱ {} shu- wu- i- \rt[¹]{xix} -μμL
		{} yú aan- d- \rt[¹]{ḵi} -μμL -n -í {} //
	\glc	just then {}[\pr{PP} {}[\pr{CP} {}[\pr{PP} \xx{3n·pss} \xx{hsfc} -\xx{pnct} {}]
			voice- \xx{pfv}- \rt[¹]{die·off} -\xx{var} -\xx{sub} {}] -\xx{erg} {}]
		\xx{foc} -\xx{dist}
		{}[\pr{PP} \xx{areal} -\xx{abl} {}] end- \xx{pfv}- \xx{stv}- \rt[¹]{fall} -\xx{var}
		{}[\pr{DP} \xx{dist} town- \xx{mid}- \rt[¹]{sit·\xx{pl}} -\xx{var} -\xx{nsfx} -\xx{nmz} {}] //
	\gld	just then {} {} {} its atop -on {}
			\rlap{\xx{pfv}.forget} {} {} {} -that {} -because {}
		\rlap{it.is} {}
		{} \rlap{too.much} {} {} \rlap{\xx{pfv}.run·out} {} {} {} {}
		{} those \rlap{townspeople} {} {} {} {} {} {} //
	\glft	‘So then it was because he had forgotten about it that they died off, those townspeople.’
		//
\endgl
\xe

\ex\label{ex:90-104-thats-why-saying}%
\exmn{266.9}%
\begingl
	\glpreamble	ᴀtcayu′ “Ye′-ᴀtg̣âku- //
	\glpreamble	Ách áyú yéi at g̱waakóo: //
	\gla	{} \rlap{Ách} @ {} {}
		\rlap{áyú} @ {}
		yéi @ at @ \rlap{g̱waakóo:} @ {} @ {} @ {} @ {} @ {} //
	\glb	{} á -ch {}
		á -yú
		yéi= at= u- {} g̱- i- \rt[²]{ku} -μμH //
	\glc	{}[\pr{PP} \xx{3n} -\xx{erg} {}]
		\xx{foc} -\xx{dist}
		thus= \xx{4n·o}= \xx{irr}- \xx{zcnj}\· \xx{mod}- \xx{stv}- \rt[²]{know} -\xx{var} //
	\gld	{} that -why {}
		\rlap{it.is} {}
		thus thing \rlap{\xx{pot}.be.known} {} {} {} {} {} //
	\glft	‘That is why there is a saying:’
		//
\endgl
\xe

\ex\label{ex:90-105-you-may-be-like-him}%
\exmn{266.10}%
\begingl
	\glpreamble	nᴀq-naołîg̣ā′s!î wūckā′djātê yᴀx q!wᴀn īng̣â′te.” //
	\glpreamble	«\!A náḵ naa wligáasʼi ooskaa sʼaatí yáx̱ xʼwán eeng̱waatee.\!» //
	\gla	{} {} {} {} «\!A náḵ {}
					naa @ \rlap{wligáasʼi} @ {} @ {} @ {} @ {} @ {} {} +
				{} \rlap{ooskaa} @ {} @ {} @ {} @ {} @ {} @ {} {}
				sʼaatí {}
			yáx̱ {}
		xʼwán
		\rlap{eeng̱waatee.\!»} @ {} @ {} @ {} @ {} @ {} @ {} //
	\glb	{} {} {} {} a náḵ {}
					naa= wu- l- i- \rt[¹]{gasʼ} -μμH -i {}
				{} a- u- d- s- \rt[²]{ka} -μμL {} {}
				sʼaatí {}
			yáx̱ {}
		xʼwán
		i- u- n- g̱- i- \rt[¹]{tiʰ} -μμL //
	\glc	{}[\pr{PP} {}[\pr{DP} {}[\pr{CP} {}[\pr{PP} \xx{3n} \xx{elat} {}]
					clan= \xx{pfv}- \xx{csv}- \xx{stv}- \rt[¹]{extend} -\xx{var} -\xx{rel} {}]
				{}[\pr{NP} \xx{xpl}- \xx{irr}- \xx{mid}- \xx{xtn}- \rt[²]{lazy} -\xx{var} \·\xx{nmz} {}]
				master:\xx{inal} {}]
			\xx{sim} {}]
		\xx{imp}
		\xx{2sg·o}- \xx{irr}- \xx{ncnj}- \xx{mod}- \xx{stv}- \rt[¹]{be} -\xx{var} //
	\gld	{} {} {} {} him from {}
					clan \rlap{\xx{pfv}.relocate} {} {} {} {} -that {}
					{} \rlap{\xx{impfv}.be.lazy} {} {} {} {} {} -ing {}
				master·of {}
			like {}
		\xx{imp}
		\rlap{you·\xx{sg}.\xx{pot}.be} {} {} {} {} {} {} //
	\glft	‘“You may be like that master of laziness whose clan abandoned him.”’
		//
\endgl
\xe

\ex\label{ex:90-106-things-came-back}%
\exmn{266.11}%
\begingl
	\glpreamble	Yū′a-inî-ᴀt łdakᴀ′t qox wu′dîᴀt. //
	\glpreamble	Yú a.eeni át, ldakát ḵux̱ wudi.át. //
	\gla	{} Yú {} \rlap{a.eeni} @ {} @ {} @ {} {} át, {}
		ldakát
		\rlap{ḵux̱} @ {} @ \rlap{wudi.át} @ {} @ {} @ {} @ {} //
	\glb	{} yú {} a- \rt[²]{.in} -μμL -i {} át {}
		ldakát
		ḵúx̱ {} wu- d- i- \rt[¹]{.at} -μH //
	\glc	{}[\pr{DP} \xx{dist} {}[\pr{CP} \xx{arg}- \rt[²]{gather} -\xx{var} -\xx{rel} {}] thing {}]
		all
		\xx{rev} -\xx{pnct}= \xx{pfv}- \xx{mid}- \xx{stv}- \rt[¹]{go·\xx{pl}} -\xx{var} //
	\gld	{} those {} \rlap{3>3.\xx{impfv}.kill·\xx{pl}} {} {} -that {} things {}
		all
		\rlap{back} {} \rlap{\xx{pfv}.go·\xx{pl}} {} {} {} {} //
	\glft	‘Those things that he kills, they all came back.’
		//
\endgl
\xe

\ex\label{ex:90-107-to-water-and-inland}%
\exmn{266.11}%
\begingl
	\glpreamble	Hī′nde a ᴀt kā′waᴀt qā dᴀ′qdî. //
	\glpreamble	Héende aa át kaawa.át ḵa dáḵde. //
	\gla	{} \rlap{Héende} @ {} {}
		aa {} \rlap{át} @ {} {} \rlap{kaawa.át} @ {} @ {} @ {} @ {}
		ḵa {} \rlap{dáḵde.} @ {} {} //
	\glb	{} héen -dé {}
		aa= {} á -t {} k- wu- i- \rt[¹]{.at} -μH
		ḵa {} dáaḵ -dé {} //
	\glc	{}[\pr{PP} water -\xx{all} {}]
		\xx{part·o}= {}[\pr{PP} \xx{3n} -\xx{pnct} {}] \xx{qual}- \xx{pfv}- \xx{stv}- \rt[¹]{go·\xx{pl}} -\xx{var}
		and {}[\pr{PP} inland -\xx{all} {}]  //
	\gld	{} water -to {}
		some {} there -to {} \rlap{\xx{pfv}.stroll·\xx{pl}} {} {} {} {}
		and {} inland -to {} //
	\glft	‘Some wandered to the water, and some inland.’
		//
\endgl
\xe

\ex\label{ex:90-108-done-all-died-out}%
\exmn{266.12}%
\begingl
	\glpreamble	Hūtc! łdakᴀ′t qotx cū′wax̣īx̣ yū′antqenî. //
	\glpreamble	Hóochʼ, ldakát ḵútx̱ shoowaxeex yú aantḵeiní. //
	\gla	Hóochʼ,
		ldakát
		{} \rlap{ḵutx̱} @ {} {} \rlap{shoowaxeex} @ {} @ {} @ {} @ {} +
		{} yú \rlap{aantḵeiní.} @ {} @ {} @ {} @ {} @ {} {} //
	\glb	hóochʼ
		ldakát
		{} ḵú -dáx̱ {} shu- wu- i- \rt[¹]{xix} -μμL
		{} yú aan- d- \rt[¹]{ḵi} -μμL -n -í {} //
	\glc	finished
		all
		{}[\pr{PP} \xx{areal} -\xx{abl} {}] end- \xx{pfv}- \xx{stv}- \rt[¹]{fall} -\xx{var}
		{}[\pr{DP} \xx{dist} town- \xx{mid}- \rt[¹]{sit·\xx{pl}} -\xx{var} -\xx{nsfx} -\xx{nmz} {}] //
	\gld	finished
		all
		{} \rlap{too.much} {} {} \rlap{\xx{pfv}.run·out} {} {} {} {}
		{} those \rlap{townspeople} {} {} {} {} {} {} //
	\glft	‘It is finished, they all died out, those townspeople.’
		//
\endgl
\xe
