%!TEX root = ../swanton-texts.tex
%%
%% 104. Kaagwaantaan History (326–346)
%%

\resetexcnt
\chapter{Kaagwaantaan Daat: Kaagwaantaan History}\label{ch:104-kaagwaantaan-history}

\section{Swanton’s abstract}\label{sec:104-swanton-abstract}

A noted hunter named Qakē′q!ᵘtê killed the sleep bird, and along with it all his own people.
Being unable to sleep himself, he wandered north to the mouth of Alsek riveer where he tried to trap a ground hog, but found a frog in his trap instead.
He thought he saw some people but found they were stones.
Then he went up the river and came among the Athapascans, whose good will he obtained by teaching them how to catch eulachon, thus preserving them from starvation.
In spring they accompanied him back to his own peeople, bringing loads of furs with them.
They came first to the Grass people, but these were afraid and sent them away, so they went to the Kā′gwᴀntān who opened trade with them and became rich.
The Athapascans tradeed particularly for a kind fo seaweed.

From the wealth thus obtained the Kā′gwᴀntān built Shadow house, and had a great feast.
By and by the chief’s daughter, who was menstruant, said something to anger the glacier, and it extended itself over the town, driving the people to Kᴀq!nuwū′, while the T!ᴀ′q!dentān settled opposite.
Later on the people warred with the Łuqā′xᴀdî of Alsek riveer and captured the Wolf post from them.
A Łuqā′xᴀdî shaman was attacked by some warriors and few away.
He flew around for somee time until a menstruant woman looked at him, making him fall into a pond.
The warriors who had attacked him began to tamper with his spirit paraphernalia, and all but one of them were destroyed.
Then the Kā′gwᴀntān erected another house, which they named Wolf house, and carved its posts like the Wolf post they had captured.
They invited people to the feast from Chilkat, Sitka, and Killisnoo.
Slave’s valley then received its name from some slaves who came to life after having been killed and thrown down into it, supposedly dead.
Afterward two parties of young people contended with each other going after firewood, and later on pushed the house fire over on each other until the great beams caught.
As a result of this fight, the family scattered, and some moved to Sitka.
From that time, too, they came to be known as Burnt-house people (Kā′gwᴀntān).

\section{Swanton’s translation}\label{sec:104-swanton-translation}

\section{Paragraph 1}\label{sec:104-para-1}

%\clearpage
