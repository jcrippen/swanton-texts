%!TEX root = ../swanton-texts.tex
%%
%% 94. Wealth Woman (292–293)
%%

\resetexcnt
\chapter{Tlʼanaxéedáḵw: Wealth Woman}\label{ch:94-wealth-woman}

This narrative was told to \citeauthor{swanton:1909} by \fm{Daawoolsʼéesʼ} Don Cameron in Sitka in 1904.
In \citeauthor{swanton:1909}’s original publication it is number 94, running from page 292 to 293 and totalling 16 lines of glossed transcription.
The Tlingit name of this story is \fm{Tlʼanaxéedáḵw} which is the name of the mythical figure featured in this story.
\citeauthor{swanton:1909} did not translate the name, giving the story as “The ʟ!ênᴀxx̣ī′dᴀq” using his transcription.
Some common names in English for \fm{Tlʼanaxéedáḵw} are ‘Wealth Woman’ \FIXME{cite}, ‘Lucky Lady’ \FIXME{cite}, and ‘Property Woman’ \FIXME{cite}, but none of these is a literal translation.
The name’s meaning is no longer clear in modern Tlingit; see the discussion later in this introduction.

Like several other narratives in \citeauthor{swanton:1909}’s collection, this narrative is quite short.
Because of this it is difficult to interpret on its own.
But the story is very well known and has been recorded many times from other Tlingit people.
\citeauthor{swanton:1909} has a more detailed version of the same story in English from \fm{Kasáankʼw} as number 35 in his collection, recorded in Wrangell in 1904 \parencite[173–175]{swanton:1909}.
\citeauthor{swanton:1909} also records a mention of her at the end of his number 105 for which see \FIXME{xref to ch.\ \ref{ch:105-kwaashkikhwaan-history}}. 
Anatolii Kamenskii recorded another version in Russian some time before 1906 from an unnamed consultant in Sitka \parencites{kamenskii:1906}[68–70]{kamenskii-kan:1985}.
Viola Garfield provides some notes in English about \fm{Tlʼanaxéedáḵw} from an unknown consultant in 1939, given in the context of a totem pole in Klawock that also features the \fm{G̱unakadeit} mythical creature \parencite[117–118]{garfield-forrest:1948}.
Frederica de Laguna has a version from \fm{Ḵaajaaḵw} Jack Ellis and another from \fm{Shaawát Ḵʼwás} Emma Ellis, both in English, which she recorded in Yakutat in 1954 \parencite[884]{de-laguna:1972}.
Jeff Leer published a version in Tlingit by \fm{Seidayaa} Elizabeth Nyman \parencite[218–255]{nyman:1993}.
Catharine McClellan recorded five versions in English, one from \fm{Ltʼaanéekaneek} Patsy Henderson \parencite[238–241]{mcclellan-cruikshank:2007b}, two from \fm{Stéew} Angela Sidney \parencite[344–348, 348–353]{mcclellan-cruikshank:2007b}, one from \fm{Yeilnaawú} Tom Peters \parencite[715–719]{mcclellan-cruikshank:2007c}, and one from \fm{Neildayéen} Edgar Sidney \parencite[747–753]{mcclellan-cruikshank:2007c}.
In addition McClellan recorded a personal story of \fm{Ltʼaanéekaneek} Patsy Henderson hearing the \fm{Tlʼanaxéedáḵw} \parencite[241–244]{mcclellan-cruikshank:2007b}.
Aside from these published versions there are also unpublished audio recordings;
the Dauenhauer collection has at least four recordings of the story that await transcription: \fm{Yeilnaawú} Tom Peters in 1973 (tape 42 side \textsc{a}), \fm{Keiḵóokʼw} George Jim in 1970 (tape 91 side \textsc{b}), and \fm{Kéet Yanaayí} Willie Marks in 1976 (tape 75) and 1980 (tape 324 side \textsc{a}).
\FIXME{Check Eyak, Southern Tutchone, Tahltan collections}

In a footnote \citeauthor{swanton:1909} explicitly connects \fm{Tlʼanaxéedáḵw} to the Haida “Skîl djā′adai” that he translates as “Property Woman” \parencite[292 fn.\ a]{swanton:1909}.
\citeauthor{enrico:2005} gives this name as \fm{Skils Jaadaay} [\ipa{skils tʃaːtaːj}] in Skidegate and \fm{Skil Jaadee} [\ipa{skil tʃaːteː}] in Masset, both meaning ‘Wealth Woman’ \parencite[525]{enrico:2005}.
\citeauthor{swanton:1905a} offers some details about her in his ethnography of the Haida \parencite[29]{swanton:1905a} and she is mentioned in a couple of stories he recorded \parencites{swanton:1905b}{swanton:1908a} though none of these are strictly about her.
\FIXME{cite specific narratives, elaborate, merge with following paragraph}

\citeauthor{boas:1916} notes that the Tsimshian narrative from Henry Tate called “Plucking Out Eyes” \parencite[154–158]{boas:1916} is “practically identical” to the Tlingit narratives that \citeauthor{swanton:1909} collected \parencite[746]{boas:1916}.
\citeauthor{boas:1916} also says that the same incident of the child taking out people’s eyes happens as part of the Skidegate Raven stories \parencite[111]{swanton:1905b} and the Masset Raven stories \parencite[143]{swanton:1905b}, but this is not directly connected to the Wealth Woman.
\FIXME{“The being who plucked the eyes of the people” \cite[74–77]{barbeau-beynon:1987a}.}

The name \fm{Tlʼanaxéedáḵw} is opaque in modern Tlingit.
\citeauthor{mcclellan:1975b} says that “the Inland Tlingit know her by the Tlingit name, \fm{tłᴇnax̣idᴀq}, which they cannot translate” \parencite[572]{mcclellan:1975b}.
\citeauthor{de-laguna:1972} does not give any translations of the name from her consultants, but she says “Because of her association with food at low tide, I have wondered whether her name bight not be \fm{łen} (low tide) \fm{nᴀx̣} (from) \fm{hi} or \fm{xi} (?), \fm{dᴀq} (appeared), which is how one informant pronounced it” \parencite[821]{de-laguna:1972}.
\citeauthor{de-laguna:1972}’s speculation suggests that her consultants did not offer any interpretations of the name; her suggested analysis is morphologically and lexically implausible.

\citeauthor{leer:1973} lists \fm{Tlʼanaxéedáḵw} after his entry for \fm{\rt[²]{xit}} ‘move stick, furrow, scratch’ which implies that the stem is \fm{xéet} \parencite[f03/70]{leer:1973}.
But he does not give an analysis of the name and rather than a gloss or translation he gives a description of the being from an unnamed consultant: “looks like a black bear, fingers are all claws; you are supposed to take off the scab and it’s good luck” \parencite[f03/70]{leer:1973}.
He later lists the form “\fm{tlʼanaxi·daḵw}” in his comprehensive stem list under the stem “\fm{-xiˋt*} make furrow, plow”, giving it the gloss “legendary clawed creature resembling bear” which is crossed out with “Property Woman” written below \parencite[53]{leer:1978b}.

If the stem of \fm{Tlʼanaxéedáḵw} is indeed \fm{xéet} based on \fm{\rt[²]{xit}} ‘move stick, furrow, scratch’ then the following \fm{-áḵw} must be a suffix.
This is probably the epenthesized allomorph of the deprivative derivational suffix \fm{-ḵ} \~\ \fm{-áḵw} ‘lacking, deprived of’ as in \fm{kaltéelḵ} ‘shoeless’ with \fm{téel} ‘shoe’ and \fm{ḵutx̱ naaxʼáḵw} ‘extinction of clan’ with \fm{naa} ‘clan’ and \fm{-xʼ} plural.
A few other nouns containing \fm{-áḵw} include \fm{gantáḵw} ‘lupine’ (\fm{\rt{gan}} ‘burn’ + \fm{-t}?), \fm{laanáḵw} ‘barren female ruminant’ (\fm{laan} unknown but \fm{g̱áx̱ laan} ‘blanket of rabbit fur strips’), \fm{katʼéitʼáḵw} ‘berries remaining on stem after freezing’ (\fm{\rt{tʼa}} ‘hot; ripe’ + \fm{-tʼ}), \fm{gooshúḵ} \~\ \fm{goosháḵw} ‘nine’ (\fm{goosh} ‘thumb; dorsal fin’), \fm{tʼaawáḵ} ‘goose’ (\fm{tʼaaw} ‘feather’), and \fm{gaawáḵ} ‘saskatoonberry’ (\fm{gaaw} ‘noise’).
It is also etymologically part of the postposition \fm{náḵ} ‘leaving behind, abandoning’ and its compounds like \fm{jináḵ} ‘taking away’, \fm{x̱ʼanáḵ} ‘withdrawing food’, and \fm{waḵnáḵ} ‘avoiding/leaving vision’.
The suffix \fm{-áḵw} is generally frozen and uninterpretable now, but was probably more productive in the recent past.
If the stem \fm{xéet} in \fm{Tlʼanaxéedáḵw} does reflect the root \fm{\rt[²]{xit}} ‘move stick, furrow, scratch’ then \fm{xéedáḵw} could mean something like ‘scratchless’ or ‘without a scratch’ which is plausibly relatable to the scratching and collection of scabs reported widely in connection with the story \parencites[821]{de-laguna:1972}[572]{mcclellan:1975b}.

The first part of the name \fm{Tlʼanaxéedáḵw} is difficult to identify and this probably contributes the most to its opacity in modern Tlingit.
\citeauthor{swanton:1909}’s transcription \orth{ʟ!ê′nᴀxx̣ī′dᴀq} provides some important clues.
His representation suggests something like \fm{Tlʼenax̱xéedáḵ} which indicates that the initial vowel can be \fm{e} rather than \fm{a} and that the second syllable can be \fm{nax̱} rather than just \fm{na}.
This suggests a stem like \fm{tlʼei} or \fm{tlʼéi} with a suffix \fm{-náx̱}.
There is no open syllable root like \fm{\rt{tlʼe}} but there are several with a coda consonant: \fm{\rt{tlʼen}} ‘impure, unclean’, \fm{\rt{tlʼetʼ}} ‘grasp, clasp’, \fm{\rt{tlʼel}} ‘milt’, \fm{\rt{tlʼetlʼ}} ‘moonfish’, \fm{\rt{tlʼexw}} ‘ill-fitting, ugly’, \fm{\rt{tlʼekw}} ‘recoil, dodge’, \fm{\rt{tlʼekw}} ‘eat raw seafood’, \fm{\rt{tlʼex̱}} ‘\textit{Usnea} lichen’, \fm{\rt{tlʼex̱ʼ}} ‘soil’, and \fm{\rt{tlʼiḵ}} \~\ \fm{\rt{tlʼeḵ}} ‘finger’ \parencite[35]{leer:1978b}.
The scratching implied by \fm{\rt{xit}} suggests a connection with \fm{\rt{tlʼiḵ}} \~\ \fm{\rt{tlʼeḵ}} ‘finger’, but impurity (\fm{\rt{tlʼen}}) and recoiling (\fm{\rt{tlʼekw}}) are also reasonable candidates.

As for \fm{-náx̱} there are two productive \fm{-náx̱} suffixes we can identify.
One \fm{-náx̱} is the perlative postposition as in \fm{héennáx̱} ‘along, across, via the river’ with \fm{héen} ‘fresh water, river’.
The other \fm{-náx̱} is the human numeral modifier as in \fm{dáx̱náx̱} ‘two humans’ with \fm{déix̱} ‘two’.
In addition there is an unproductive \fm{-náx̱} found in a variety of more or less opaque placenames associated with bays such as \fm{G̱aanáx̱} Port Stewart \parencite[187 \#149]{thornton:2012}, \fm{Sʼiknáx̱} Limestone Inlet \parencite[77 \#142]{thornton:2012}, and \fm{Skanáx̱} Saginaw Bay \parencite[125, 136 \#133]{thornton:2012}; this seems to originally refer to a kind of bay or harbour as in \fm{Nax̱ Tlein} ’Big Harbor’ \parencite[170 \#119]{thornton:2012}.
Of these three only the perlative postposition \fm{-náx̱} ‘via, along, across, through’ is compositionally plausible.
If we then take \fm{\rt{tlʼiḵ}} \~\ \fm{\rt{tlʼeḵ}} ‘finger’ as a basis we have something like \fm{tlʼeiḵnáx̱ xéedáḵw} as the original form of the name which is morphologically \fm{\rt{tlʼeḵ}-μμL-náx̱ \rt{xit}-μμH-áḵw}.
The meaning of this phrase is still not entirely clear, but it should literally be something like ‘scratchless via finger’.

Although the story is often located by narrators at \fm{Áakʼw} Auke Lake near Juneau, there are a few placenames containing the name \fm{Tlʼanaxéedáḵw} which suggest that it may also have been situated in other locales.
One example is \fm{Tlʼanaxéedáḵw Xʼáatʼákʼw} (\fm{xʼáatʼ-ákʼw} ‘island-little’) near Danger Passage at the south end of Mary Island in \fm{Taantʼá Ḵwáan} country \parencite[202 \#675]{thornton:2012}.
\FIXME{Any other placenames?}

\section{Swanton’s abstract}\label{sec:94-swanton-abstract}

A man saw a woman and two children floating in Auk lake, and he captured one of the children and brought it home.
During the night the child gouged out the eyes of all the people living in the village except one woman, and ate them.
This woman killed the child, and taking on her back her own child, to which she had just given birth, she went up into the woods and became the ʟ!ê′naxx̣ī′dᴀq.
As she went along she ate mussels and fitted the shells together.

\section{Swanton’s translation}\label{sec:94-swanton-translation}

\pgnum{292}
\snum{1–2}A man at Auk went out on the lake after firewood.
\snum{3}On the way round it he saw a woman floating about.
Her hair was long.
Looking at her for some time, he saw that her little ones were with her.
He took one of the children home.
When it became dark they went to sleep.
It was the child of the ʟ!ê′nᴀxx̣ī′dᴀq, and that night it went through the town picking out people’s eyes.
Toward morning a certain woman bore a child.
In the morning, when she was getting up, this [the ʟ!ê′nᴀxx̣ī′dᴀq’s child] came in to her into the house.
The small boy had a big belly full of eyes.
He had taken out the eyes of all the people.
That woman to whom the small boy came had a cane.
He kept pointing at her eyes.
Then she pushed him away with the cane.
When he had done it twice, she pushed it into him.
He was all full of eyes.
After she had killed him the woman went through the
\pgnum{293}
houses.
Then she began to dress herself up.
She took her child up on her back to start wandering.
She said, “I am going to be the ʟ!ê′nᴀxx̣ī′dᴀq.”
When she came down on the beach she kept eating mussels.
She put the shells inside of one another.
As she walks along she nurses her little child.

%\clearpage
\section{Paragraph 1}\label{sec:94-para-1}

\ex\label{ex:94-1-at-auke-lake}%
\exmn{292.1}%
\begingl
	\glpreamble	Āk!ᵘq!ayu′ yē ỵatî //
	\glpreamble	Áakʼwxʼ áyú yéi ÿatee; //
	\gla	{} \rlap{Áakʼwxʼ} @ {} @ {} {} \rlap{áyú} @ {}
		yéi @ \rlap{ÿatee;} @ {} @ {} //
	\glb	{} áaʷ -kʼw -xʼ {} á -yú
		yéi= ÿa- \rt[¹]{tiᴸ} -μμL //
	\glc	{}[\pr{PP} lake -\xx{dim} -\xx{loc} {}] \xx{foc} -\xx{dist}
		thus= \xx{stv}- \rt[¹]{be} -\xx{var} //
	\gld	{} lake -little -at {} \rlap{it.is} {}
		thus \rlap{\xx{ncnj}.\xx{s·impfv}.be} {} {} //
	\glft	‘It is over at Auke Lake where it is;’
		//
\endgl
\xe

\citeauthor{swanton:1909} gives sentences (\lastx) and (\nextx) as a single sentence.
They are however separate main clauses with clausally unmarked verb forms and not subordinate-marked forms of verbs: \fm{yéi ÿatee} not \fm{yéi teeÿí} and \fm{woogoot} not \fm{wugoodí}.
Presumably \citeauthor{swanton:1909} ran them together because they were uttered in sequence without pause by the speaker, so as a compromise they are given as a single capitalized unit divided by a semicolon.

\ex\label{ex:94-2-man-went-there-for-firewood}%
\exmn{292.1}%
\begingl
	\glpreamble	qā akadē′ wugu′t g̣ᴀ′ng̣ā. //
	\glpreamble	ḵáa a kaadé woogoot gáng̱aa. //
	\gla	{} ḵáa {} {} a \rlap{kaadé} @ {} {}
		\rlap{woogoot} @ {} @ {} @ {}
		{} \rlap{gáng̱aa.} @ {} {} //
	\glb	{} ḵáaʷ {} {} a kaa -dě {}
		wu- μ- \rt[¹]{gut} -μμL
		{} gán -g̱ǎa {} //
	\glc	{}[\pr{NP} man {}] {}[\pr{PP} \xx{3n·pss} \xx{hsfc} -\xx{all} {}]
		\xx{pfv}- \xx{stv}- \rt[¹]{go·\xx{sg}} -\xx{var}
		{}[\pr{PP} firewood -\xx{ades} {}] //
	\gld	{} man {} {} its atop -to {}
		\rlap{\xx{ncnj}.\xx{pfv}.go·\xx{sg}} {} {} {}
		{} firewood -for {} //
	\glft	‘a man went there for firewood.’
		//
\endgl
\xe

\ex\label{ex:94-3-going-edge-woman-floating-middle}%
\exmn{292.1}%
\begingl
	\glpreamble	A′yᴀxde ỵanagudī′ayu aosītī′n cāwᴀ′t yū′adīgīgā cwū′ʟ̣īx̣āc. //
	\glpreamble	A yaax̱dé ÿaa nagúdi áyú awsiteen shaawát yú a digeeygéi sh wudlihaash. //
	\gla	{} {} A \rlap{yaax̱dé} @ {} {}
			ÿaa @ \rlap{nagúdi} @ {} @ {} @ {} {}
			\rlap{áyú} @ {}
		\rlap{awsiteen} @ {} @ {} @ {} @ {} @ {} +
		{} {} \rlap{shaawát} @ {} {}
			{} yú a \rlap{digeeygéi} @ {} {}
			sh @ \rlap{wudlihaash.} @ {} @ {} @ {} @ {} @ {} @ {} {}  //
	\glb	{} {} a yaax̱ -dě {} 
			ÿaa= n- \rt[¹]{gut} -μH -ǐ {}
			á -yú
		a- w- s- i- \rt[²]{tin} -μμL
		{} {} sháaʷ- ÿát {}
			{} yú a digeeÿgé -μ {}
			sh= wu- d- l- i- \rt[¹]{hash} -μμL {} {} //
	\glc	{}[\pr{CP} {}[\pr{PP} \xx{3n·pss} edge -\xx{all} {}]
			along= \xx{ncnj}- \rt[¹]{go·\xx{sg}} -\xx{var} -\xx{sub} {}]
			\xx{foc} -\xx{dist}
		\xx{arg}- \xx{pfv}- \xx{xtn}- \xx{stv}- \rt[²]{see} -\xx{var}
		{}[\pr{CP} {}[\pr{NP} woman- child {}]
			{}[\pr{PP} \xx{dist} \xx{3n·pss} middle -\xx{loc} {}]
			\xx{rflx}= \xx{pfv}- \xx{mid}- \xx{csv}- \xx{stv}-
				\rt[¹]{float} -\xx{var} \·\xx{sub} {}] //
	\gld	{} {} its edge -to {}
			along \rlap{\xx{ncnj}.\xx{prog}.go·\xx{sg}} {} {} -while {}
			\rlap{it.is} {} 
		\rlap{3>3.\xx{g̱cnj}.\xx{pfv}.see} {} {} {} {} {} 
		{} {} \rlap{woman} {} {} 
			{} that its middle -at {}
			self \rlap{\xx{ncnj}.\xx{pfv}.float} {} {} {} {} \·that {} //
	\glft	‘It was while he is going to its edge that he saw that a woman had floated herself out in the middle of it.’
		//
\endgl
\xe

The sentence in (\lastx) is a nice example of a relatively complex sentence structure.
It begins with an adjunct clause \fm{a yaax̱dé ÿaa nagúdi} ‘while he is going to its edge’ that contains a progressive aspect verb with a PP.
This adjunct clase is set off from the rest of the clause by a focus marker \fm{áyú} which uses the distal \fm{yú} ‘that, there’ in common with sentence (\ref{ex:94-1-at-auke-lake}) and the later distal determiner in this sentence’s complement clause.
The use of the distal in (\ref{ex:94-1-at-auke-lake}) puts the scene far away from the speaker’s location (Sitka).
The use of the distal in the focus marker could be spatial, placing the event of the man’s travel to the lake edge at a distance from the middle of the lake.
It could also be perspectival, placing the event of the man’s travel at a mental distance from the event of his seeing the woman in the middle of the lake.

The remainder of the sentence in (\lastx) is the main clause.
The verb in this main clause is \fm{awsiteen} ‘he saw it’ which takes the following embedded clause as its complement.
The complement clause starts with an NP \fm{shaawát} ‘woman’ which is the subject of the clausally unmarked verb \fm{sh wudlihaash} ‘she floated herself’.
This verb is based on the verb root \fm{\rt[¹]{hash}} ‘float’ which supports an unaccusative intransitive verb \fm{x̱at wulihaash} ‘I floated’ and a transitive causativized verb \fm{awlihaash} ‘s/he/it made him/her/it float’ \parencite[45]{leer:1976}.
Some verb forms based on this root are \fm{n}-conjugation class members: \fm{ḵúx̱de yaa nalhásh} “he’s drifting back” \parencite[93.1184]{story-naish:1973}, \fm{du jeedé x̱wlihaash} “I let him do it, have it” lit.\ ‘I floated it into his possession’ \parencite[01/94]{leer:1973}, and the nominalization \fm{nalháashadi} ‘floating thing (esp.\ log)’.
Other verb forms are apparently \fm{g}-conjugation class members: \fm{g̱ayéisʼ tléil kei ulháshch} “iron doesn’t float” \parencite[93.1183]{story-naish:1973}.
The verb \fm{sh wudlihaash} in (\lastx) could be in either class, but it has been assigned \fm{n}-conjugation because this is relatively more common.
This verb notably has a reflexive object \fm{sh} ‘self’ which means that it is causative and thus the woman has intentionally floated herself into the middle of the lake.

The complement clause in (\lastx) has a PP separating the subject NP and the verb, namely \fm{yú a digeeygéi} ‘in that middle of it’, referring to the middle of \fm{Áakʼw} Auke Lake.
The final \orth{ā} in \orth{dīgīgā} is probably a misreading of handwritten \orth{ē}.
The postposition in this phrase is the vowel lengthening allomorph \fm{-μ} of the locative better known by its consonant suffix form \fm{-xʼ}.
The noun \fm{digeeygé} ‘middle’ has the apperance of something structurally complex but it is unanalyzable in modern Tlingit.
\FIXME{Finish discussion of \fm{digeeÿgé} and note its occurrence in \#89 and \#99.}

\ex\label{ex:94-4-hair-long}%
\exmn{292.2}%
\begingl
	\glpreamble	Dūcaxāwu′ yek!u′ʟ̣îyāt!. //
	\glpreamble	Du shax̱aawú yéi kudliyáatʼ. //
	\gla	{} Du \rlap{shax̱aawú} @ {} @ {} {} 
		yéi @ \rlap{kudliyáatʼ.} @ {} @ {} @ {} @ {} @ {} @ {} //
	\glb	{} du sha- x̱aaw -ǔ {}
		yéi= k- u- d- l- i- \rt[¹]{ÿatʼ} -μμH //
	\glc	{}[\pr{NP} \xx{3h·pss} head- fur -\xx{pss} {}]
		thus= \xx{cmpv}- \xx{irr}- \xx{mid}- \xx{xtn}- \xx{stv}- \rt[¹]{long} -\xx{var} //
	\gld	{} her head- hair {} {}
		thus \rlap{\xx{cmpv}.\xx{gcnj}.\xx{s·impfv}.long} {} {} {} {} {} {} //
	\glft	‘Her hair is comparatively long.’
		//
\endgl
\xe

\ex\label{ex:94-5-watching-long-saw-small-ones}%
\exmn{292.3}%
\begingl
	\glpreamble	Tc!ākᵘ āłtî′nî a′ya aosītī′n yē′kᵘts!îgā′ỵî a. //
	\glpreamble	Chʼáakw altíni áyá awsiteen yéi kwdzigéiÿi aa. //
	\gla	{} Chʼáakw \rlap{altíni} @ {} @ {} @ {} @ {} {}
			\rlap{áyá} @ {}
		\rlap{awsiteen} @ {} @ {} @ {} @ {} @ {} +
		{} {} yéi @ \rlap{kwdzigéiÿi} @ {} @ {} @ {} @ {} @ {} @ {} @ {} {} aa. {} //
	\glb	{} chʼáakw a- l- \rt[²]{tin} -μH -ǐ {}
			á -yá
		a- w- s- i- \rt[²]{tin} -μμL
		{} {} yéi= k- w- d- s- i- \rt[¹]{ge} -μμH -i {}
			aa {} //
	\glc	{}[\pr{CP} long·time \xx{arg}- \xx{xtn}- \rt[²]{see} -\xx{var} -\xx{sub} {}]
			\xx{foc} -\xx{prox}
		\xx{arg}- \xx{pfv}- \xx{xtn}- \xx{stv}- \rt[²]{see} -\xx{var}
		{}[\pr{NP} {}[\pr{CP} thus= \xx{cmpv}- \xx{irr}- \xx{mid}- \xx{xtn}- \xx{stv}-
			\rt[¹]{big} -\xx{var} -\xx{rel} {}]
			\xx{part} {}] //
	\gld	{} long·time \rlap{3>3.\xx{zcnj}.\xx{impfv}.watch} {} {} {} {} {}
			\rlap{it.is} {}
		\rlap{3>3.\xx{g̱cnj}.\xx{pfv}.see} {} {} {} {} {}
		{} {} thus \rlap{\xx{ncnj}.\xx{s·impfv}.\xx{pl}.small} {} {} {} {} {} {} -that {}
			one {} //
	\glft	‘Watching for a long time, he saw them, the ones that are small.’
		//
\endgl
\xe

\FIXME{\citeauthor{swanton:1909}’s \orth{ā} in \orth{yē′kᵘts!îgā′ỵî} is presumably an error for \orth{ē}.}

\FIXME{Discuss \fm{ka-u-} comparative with \fm{d-…-xʼ} plural versus \fm{d-} plural.} 

\ex\label{ex:94-6-grab-some-kids-home}%
\exmn{292.3}%
\begingl
	\glpreamble	ᴀt ỵᴀ′tq!î ᴀx ā′wucāt nēłde′. //
	\glpreamble	Atÿátxʼi aax̱ aa woosháat neildé. //
	\gla	{} \rlap{Atÿátxʼi} @ {} @ {} @ {} {}
		{} \rlap{aax̱} @ {} {}
		aa @ \rlap{woosháat} @ {} @ {} @ {}
		{} \rlap{neildé.} @ {} {} //
	\glb	{} at= ÿát -xʼ -ǐ {}
		{} aa -x̱ {}
		aa= wu- μ- \rt[²]{shaᴴt} -μμH
		{} neil -dě {} //
	\glc	{}[\pr{NP} \xx{4n·pss}= child -\xx{pl} -\xx{pss} {}]
		{}[\pr{PP} \xx{3n} -\xx{abl} {}]
		\xx{part}= \xx{pfv}- \xx{stv}- \rt[²]{grab} -\xx{var}
		{}[\pr{PP} home -\xx{all} {}] //
	\gld	{} \rlap{children} {} {} {} {}
		{} her -from {}
		some \rlap{\xx{gcnj}.\xx{pfv}.grab} {} {} {}
		{} home -to {} //
	\glft	‘He grabbed some children from her, homeward.’
		//
\endgl
\xe

\FIXME{Discuss \orth{ā′wucāt} as \fm{aa woosháat} versus \fm{aawasháat}.}

\ex\label{ex:94-7-got-dark-sleep}%
\exmn{292.4}%
\begingl
	\glpreamble	Yên qōqacg̣ēt ayu′ āwaxē′q!ᵘ. //
	\glpreamble	Yan ḵukashg̱éit áyú aawax̱éixʼw. //
	\gla	{} Yan @ \rlap{ḵukashg̱éit} @ {} @ {} @ {} @ {} @ {} @ {} @ {} {}
			\rlap{áyú} @ {}
		\rlap{aawax̱éixʼw.} @ {} @ {} @ {} @ {} //
	\glb	{} ÿán= ḵu- ka- {} d- sh- \rt[¹]{g̱eͥᴴt} -μμH {} {}
			á -yú
		a- wμ- ÿa- \rt[¹]{x̱eͥxʼw} -μμH //
	\glc	{}[\pr{CP} \xx{term}= \xx{areal}- \xx{qual}- \xx{zcnj}\· \xx{mid}- \xx{pej}-
			\rt[¹]{dark} -μμH \·\xx{sub} {}]
			\xx{foc} -\xx{dist}
		\xx{4h·s}- \xx{pfv}- \xx{stv}- \rt[¹]{sleep·\xx{pl}} -\xx{var} //
	\gld	{} done \rlap{\xx{areal}.\xx{csec}.dark} {} {} {} {} {} {} {} {}
			\rlap{it.is} {}
		\rlap{ppl.\xx{ncnj}.\xx{pfv}.sleep·\xx{pl}} {} {} {} {} //
	\glft	‘Having gotten dark, people slept.’
		//
\endgl
\xe

\ex\label{ex:94-8-actually-wealth-woman}%
\exmn{292.4}%
\begingl
	\glpreamble	Xᴀtc ʟ!ênᴀxx̣ī′dᴀq ỵê′tî asiyu′ //
	\glpreamble	X̱ách Tlʼenax̱xéedáḵw ÿádi ásíyú; //
	\gla	X̱ách
		{} {} \rlap{Tlʼenax̱xéedáḵw} @ {} @ {} @ {} @ {} {} \rlap{ÿádi} @ {} {}
		\rlap{ásíyú;} @ {} @ {} //
	\glb	x̱áju
		{} {} tlʼeͥḵ -náx̱= \rt[²]{xit} -μμH -áḵw {} ÿát -ǐ {} 
		á -sí -yú //
	\glc	\xx{mir}
		{}[\pr{NP} {}[\pr{NP} finger -\xx{perl}= \rt[²]{scratch} -\xx{var} -\xx{dprv} {}] child -\xx{pss} {}]
		\xx{foc} -\xx{dub} -\xx{dist} //
	\gld	actually {} {} \xx{name} {} {} {} {} {} child -of {}
		\rlap{it.is.apparently} {} {} //
	\glft	‘Actually it is apparently the child of Tlʼanaxéedáḵw;’
		//
\endgl
\xe

\FIXME{Discuss splitting of sentences in (\ref{ex:94-8-actually-wealth-woman}) through (\ref{ex:94-10-took-peoples-eyes}).
This seems to happen a few times suggesting that \citeauthor{swanton:1909} was not as good with this transcription and so possibly this was one of his first in Tlingit.
That could also explain the fragmentary and short qualities.
Discuss this in the introduction.}

\FIXME{Discuss \fm{ÿatee} versus \fm{yéi ÿatee}.}

\FIXME{Point to introduction for discussion of name.}

\ex\label{ex:94-9-night-went-town-rim}%
\exmn{291.5}%
\begingl
	\glpreamble	tā′dawe yū′ānq!ᴀtūx ỵā′wagut //
	\glpreamble	taat áwé yú aan x̱ʼatóox̱ ÿaawagút; //
	\gla	taat \rlap{áwé} @ {} {} yú aan \rlap{x̱ʼatóox̱} @ {} @ {} {}
		\rlap{ÿaawagút;} @ {} @ {} @ {} @ {} //
	\glb	taat á -wé {} yú aan x̱ʼé- tú -x̱ {}
		ÿa- wμ- ÿa- \rt[¹]{gut} -μH //
	\glc	night \xx{foc} -\xx{mdst} {}[\pr{PP} \xx{dist} town mouth- inside -\xx{pert} {}]
		face- \xx{pfv}- \xx{stv}- \rt[¹]{go·\xx{sg}} -\xx{var} //
	\gld	night \rlap{it.is} {} {} that town \rlap{rim} {} -at {}
		\rlap{oblique.\xx{zcnj}.\xx{pfv}.go·\xx{sg}} {} {} {} {} //
	\glft	‘it is at night that he went around along the rim of town;’
		//
\endgl
\xe

\FIXME{Discuss \fm{x̱ʼatú} ‘edge, rim (vessel, boat, open container); inside edge; eyehole (something woven)’.}

\FIXME{Discuss \fm{NP-x̱} (\fm{∅}; \fm{-ch} repetitive) ‘obliquely, circuitously along NP’ motion derivation.}

\ex\label{ex:94-10-took-peoples-eyes}%
\exmn{292.5}%
\begingl
	\glpreamble	qāwa′q āx kē akawadjᴀ′ł. //
	\glpreamble	ḵaa waaḵ aax̱ kei akaawajél. //
	\gla	{} ḵaa waaḵ {} {} \rlap{aax̱} @ {} {}
		kei @ \rlap{akaawajél.} @ {} @ {} @ {} @ {} @ {} //
	\glb	{} ḵaa waaḵ {} {} aa -dáx̱ {}
		kei= a- ka- wμ- ÿa- \rt[²]{jel} -μH //
	\glc	{}[\pr{NP} \xx{4h·pss} eye {}] {}[\pr{PP} \xx{3n} -\xx{abl} {}]
		up= \xx{arg}- \xx{qual}- \xx{pfv}- \xx{stv}- \rt[²]{grope} -\xx{var} //
	\gld	{} ppl’s eye {} {} it -from {}
		up\• \rlap{\xx{zcnj}.\xx{pfv}.carry·load} {} {} {} {} {} //
	\glft	‘he gathered up people’s eyes from there.’
		//
\endgl
\xe

\FIXME{Discuss \fm{k-\rt[²]{jel}} ‘carry in loads’ versus \fm{\rt[²]{jel}} ‘grope’.}

\ex\label{ex:94-11-dawn-woman-have-child}%
\exmn{292.6}%
\begingl
	\glpreamble	Ỵaqē′gaa yucā′wᴀt ỵᴀt ā′wa-u. //
	\gla	{} Ÿaa \rlap{ḵeiga.áa} @ {} @ {} @ {} @ {} {}
		{} yú \rlap{shaawát} @ {} {}
		{} ÿát {} 
		\rlap{aawa.oo.} @ {} @ {} @ {} @ {} //
	\glb	{} ÿaa= ḵei- g- \rt[¹]{.a} -μμH {} {}
		{} yú sháaʷ- ÿát {}
		{} ÿát {}
		a- wμ- ÿa- \rt[²]{.u} -μμL //
	\glc	{}[\pr{CP} along= dawn- \xx{gcnj}- \rt[¹]{extend} -\xx{var} \·\xx{sub} {}]
		{}[\pr{DP} \xx{dist} woman- child {}]
		{}[\pr{NP} child {}]
		\xx{arg}- \xx{pfv}- \xx{stv}- \rt[²]{own} -\xx{var} //
	\gld	{} along \rlap{dawn.\xx{csec}.extend} {} {} {} {} {}
		{} that \rlap{woman} {} {}
		{} child {}
		\rlap{3>3.\xx{ncnj}.\xx{pfv}.own} {} {} {} {} //
	\glft	‘Having dawned, that woman had a child.’
		//
\endgl
\xe

The verb that \citeauthor{swanton:1909} transcribes as \orth{Ỵaqē′gaa} in (\lastx) is interesting because it shows an unexpected conjugation class.
This is a consecutive clause form of the verb \fm{ḵeewa.aa} \~\ \fm{ḵeiwa.aa} (\fm{n}; achievement) ‘it dawned’ which is based on the root \fm{\rt[¹]{.a}} ‘extend’ that is found in a variety of verbs meaning things like ‘plant grow’, ‘water gush’, ‘fish migrate’, ‘earthquake’, and ‘move face; inspect’ \parencite[72–78]{leer:1976}.
The verb \fm{ḵeewa.aa} \~\ \fm{ḵeiwa.aa} ‘it dawned’ includes the incorporated noun \fm{ḵee} \~\ \fm{ḵei} ‘dawn’, so the verb literally means ‘dawn extended’.
The noun \fm{ḵee} \~\ \fm{ḵei} ‘dawn’ has generally been replaced by the nominalization \fm{ḵee.á} \~\ \fm{ḵei.á} derived from the verb, but the original noun can still be found in the compound \fm{ḵeex̱ʼé} \~\ \fm{ḵeix̱ʼé} ‘mouth of dawn’ which is one reported origin of the placename \fm{Ḵéex̱ʼ} \~\ \fm{Ḵéix̱ʼ} ‘Kake’ \parencites{hope:2003}[123]{thornton:2012}.
The verb \fm{ḵeewa.aa} \~\ \fm{ḵeiwa.aa} ‘it dawned’ is documented as a member of the \fm{n}-conjugation class based on forms which have an overt \fm{n-} conjugation prefix like the consecutive clause \fm{ḵeena.áa} ‘having dawned’ and the conditional clause \fm{ḵeena.éini} ‘when it dawns’ \parencite[02/10]{leer:1973} with support from other forms that have indirect evidence for \fm{n}-conjugation class like the progressive \fm{yaa ḵeina.éin} ‘it is dawning’ and the repetitive imperfective \fm{yoo ḵeiya.éik} ‘it repeatedly dawns’ \parencite[3]{leer:1963}.
But in (\lastx) the consecutive clause form \fm{ÿaa ḵeiga.áa} clearly has the \fm{g-} conjugation prefix and not \fm{n-}.
This could be a transcription error by say misreading handwritten \orth{n} as \orth{g}, but it is semantically plausible as well.
\FIXME{Discuss motion derivations and epiaspect.
Also note \fm{ḵei.éich} ‘it always dawns’ \parencite[3]{leer:1963}.}

\FIXME{Discuss idiomatic meaning of \fm{ÿát aÿa.óo} and ambiguous ‘have a child’ in English.
Also note difference between \fm{\rt[²]{.u}} ‘own’ and \fm{\rt[²]{.u}} ‘buy’.}

\ex\label{ex:94-12-while-waking-went-inside-near}%
\exmn{292.6}%
\begingl
	\glpreamble	Ts!ūtā′t ayū′ ỵacᴀ′ndanukᵘ doxᴀ′nq!ᵘ nēł ū′wagut. //
	\glpreamble	Tsʼootaat áyú ÿaa shandanúk du x̱ánxʼ neil uwagút. //
	\gla	{} {} Tsʼootaat @ {} {} \rlap{áyú} @ {} 
			ÿaa @ \rlap{shandanúk} @ {} @ {} @ {} @ {} @ {} {} +
		{} du \rlap{x̱ánxʼ} @ {} {}
		{} neil @ {} {}
		\rlap{uwagút.} @ {} @ {} @ {} //
	\glb	{} {} tsʼootaat {} {} á -yú
		ÿaa= sha- n- d- \rt[¹]{nuk} -μH {} {}
		{} du x̱án -xʼ {}
		{} neil {} {}
		u- ÿa- \rt[¹]{gut} -μH //
	\glc	{}[\pr{CP} {}[\pr{PP} morning \·\xx{loc} {}] \xx{foc} -\xx{dist}
			along= head- \xx{ncnj}- \xx{mid}- \rt[¹]{stand·\xx{sg}} -\xx{var} \·\xx{sub} {}]
		{}[\pr{PP} \xx{3h·pss} near -\xx{loc} {}]
		{}[\pr{PP} home \·\xx{pnct} {}]
		\xx{zpfv}- \xx{stv}- \rt[¹]{go·\xx{sg}} -\xx{var} //
	\gld	{} {} morning -in {} \rlap{it.is}
			along \rlap{head.\xx{zcnj}.\xx{prog}.stand·\xx{sg}} {} {} {} {} {} {} {}
		{} his near -at {}
		{} inside \·to {}
		\rlap{\xx{zcnj}.\xx{pfv}.go·\xx{sg}} {} {} {} //
	\glft	‘It was in the morning while she was getting up that he went inside near her.’
		//
\endgl
\xe

\ex\label{ex:94-13-boy-big-stomach}%
\exmn{292.7}%
\begingl
	\glpreamble	ᴀtk!ᴀˈtsk!ᵒ yē′q!ołkułige //
	\glpreamble	Atkʼátskʼu yéi x̱ʼulʼkuligéi;  //
	\gla	{} \rlap{atkʼátskʼu,} @ {} @ {} @ {} @ {} {}
		yéi @ \rlap{x̱ʼulʼkuligéi;} @ {} @ {} @ {} @ {} @ {} @ {} //
	\glb	{} at= kʼí- ÿáts -kʼʷ -í {}
		yéi= x̱ʼulʼ- k- u- l- i- \rt[¹]{ge} -μμH //
	\glc	{}[\pr{DP} \xx{4n·pss}= base- child -\xx{dim} -\xx{pss} {}]
		thus= stomach- \xx{cmpv}- \xx{irr}- \xx{xtn}- \xx{stv}- \rt[¹]{big} -\xx{var} //
	\gld	{} \rlap{boy} {} {} {} {} {}
		thus= \rlap{stomach.\xx{cmpv}.\xx{gcnj}.\xx{s·impfv}.big} {} {} {} {} {} //
	\glft	‘The boy was big in the stomach;’
		//
\endgl
\xe

\FIXME{Discuss incorporation.}

\FIXME{Discuss splitting of sentences.}

\ex\label{ex:94-14-filled-with-eyes}%
\exmn{292.7}%
\begingl
	\glpreamble	xātc qā′wag̣e ᴀsīyu′ aca′ołīhîk. //
	\glpreamble	x̱ách ḵaa waag̱í ásíyú ashawlihík. //
	\gla	X̱ách {} ḵaa \rlap{waag̱í} @ {} {} \rlap{ásíyú} @ {} @ {}
		\rlap{ashawlihík.} @ {} @ {} @ {} @ {} @ {} @ {} //
	\glb	x̱ách {} ḵaa waaḵ -í {} á -sí -yú
		a- sha- w- l- i- \rt[¹]{hik} -μH //
	\glc	actually {}[\pr{NP} \xx{4h·pss} eye -\xx{pss} {}] \xx{foc} -\xx{dub} -\xx{dist}
		\xx{arg}- head- \xx{pfv}- \xx{csv}- \xx{stv}- \rt[¹]{filled} -\xx{var} //
	\gld	actually {} ppl’s eye {} {} \rlap{it.is.maybe} {} {}
		\rlap{3>3.head.\xx{zcnj}.\xx{pfv}.make.filled} //
	\glft	‘Actually it was apparently people’s eyes that had filled him.’
		//
\endgl
\xe

\FIXME{Discuss interpretation of subject and object in (\lastx).
Crucially not ‘he filled it with eyes’ or ‘he was filled with eyes’.}

\ex\label{ex:94-15-picked-up-peoples-eyes}%
\exmn{292.8}%
\begingl
	\glpreamble	Łdᴀkᴀ′t yu′qᵒu qā′wag̣e ayu′ āx kē akā′wadjêł. //
	\glpreamble	Ldakát yú ḵu.oo, ḵaa waag̱í áyú aax̱ kei akaawajél. //
	\gla	Ldakát {} yú \rlap{ḵu.oo,} @ {} @ {} {}
		{} ḵaa \rlap{waag̱í} @ {} {}
		\rlap{áyú} @ {}
		{} \rlap{aax̱} @ {} {}
		kei @ \rlap{akaawajél.} @ {} @ {} @ {} @ {} @ {} //
	\glb	ldakát {} yú ḵu- \rt[¹]{.u} -μμL {}
		{} ḵaa waaḵ -ǐ {}
		á -yú
		{} a -μx̱ {}
		kei= a- ka- μw- ÿa- \rt[²]{jel} -μH //
	\glc	every {}[\pr{DP} \xx{dist} \xx{4h·o}- \rt[²]{own} -\xx{var} {}]
		{}[\pr{NP} \xx{4h·pss} eye -\xx{pss} {}]
		\xx{foc} -\xx{dist}
		{}[\pr{PP} \xx{3n} -\xx{abl} {}]
		up= \xx{arg}- \xx{qual}- \xx{pfv}- \xx{stv}- \rt[²]{lug} -\xx{var} //
	\gld	all {} that \rlap{dweller} {} {} {}
		{} ppl’s eye {} {}
		\rlap{it.is} {}
		{} it -from {}
		up\• \rlap{3>3.\xx{zcnj}.\xx{pfv}.carry·load} {} {} {} {} {} //
	\glft	‘All of those people, it was people’s eyes that he had picked up from them.’
		//
\endgl
\xe

\FIXME{Discuss reference of \fm{ḵaa} and \fm{a} in (\lastx).}

\ex\label{ex:94-16-she-had-a-cane}%
\exmn{292.9}%
\begingl
	\glpreamble	Wuts!ā′g̣a ᴀcdjī′ hu yu-cāwᴀ′t //
	\glpreamble	Wootsaag̱áa ash jeewú, yú shaawát; //
	\gla	{} \rlap{Wootsaag̱áa} @ {} @ {} @ {} {}
		{} ash \rlap{jeewú,} @ {} {}
		{} yú \rlap{shaawát;} @ {} {} //
	\glb	{} wu- \rt[²]{tsaḵ} -μμL -áa {}
		{} ash jee -wú {}
		{} yú sháaʷ- ÿát {} //
	\glc	{}[\pr{NP} \xx{unkn}- \rt[²]{poke} -\xx{var} -\xx{nmz} {}]
		{}[\pr{PP} \xx{3prx·pss} poss’n -\xx{locp} {}]
		{}[\pr{DP} \xx{dist} woman -child {}] //
	\gld	{} \rlap{cane} {} {} {} {}
		{} her poss’n -is.at {}
		{} that \rlap{woman} {} {} //
	\glft	‘A cane was in her possession, that woman;’
		//
\endgl
\xe

The noun \fm{wootsaag̱áa} in (\lastx) can be translated as ‘staff’, ‘baton’, and ‘walking stick’ as well as ‘cane’.
It can refer to a stick that is taller than an adult as well as to one that is only about as high as an adult’s waist, and such a stick can be a walking aid, a ceremonial object (with elaborate carving and painting), or a weapon.
The translation ‘cane’ was used by \citeauthor{swanton:1909} and is maintained here, but given other descriptions and depictions of \fm{Tlʼanaxéedáḵw} with a walking stick at head height it might be better translated as ‘staff’.

The identity of the \fm{wu-} in \fm{wootsaag̱áa} is uncertain.
It looks like the perfective prefix but there is no clear reason why perfective aspect should be used rather than say imperfective \fm[?]{tsaag̱áa}.
It could instead be a combination of the qualifier \fm{ÿa-} \~\ \fm{ÿ-} (often meaning ‘face’) and the irrealis \fm{u-}, but the meaning of this combination is equally puzzling.
\citeauthor{leer:1973} lists \fm{wootsaag̱áa} with the note “Rez.\ gigitsak-a” \parencite[09/146]{leer:1973}.
This implies that Rezanov transcribed the noun as something like \orth{гигицак-а} in the early 19th century which could represent something like \fm[?]{ÿiÿitsaag̱áa} but this does little to clarify the problem.
For now \fm{wu-} has been given as unknown.

The phrase that \citeauthor{swanton:1909} transcribes as \orth{ᴀcdjī′ hu} in (\lastx) is almost certainly \fm{ash jeewú} ‘in her possession’.
His use of \orth{h} is unexpected though; it could just be a misreading of \orth{w} or the speaker could have said something like [\ipa{ʔàʃ ˈtʃìː.ɰú}] or [\ipa{ʔàʃ ˈtʃìː.ɦú}] where \citeauthor{swanton:1909} interpreted the [\ipa{ɰ}] or [\ipa{ɦ}] as an unusual /\ipa{h}/.
The marking of stress with \orth{′} on the syllable \fm{jee} and not the syllable \fm{wú} even though the latter has high tone is one of many indications that \citeauthor{swanton:1909} paid more attention to stress patterns than he did to tone.

\citeauthor{swanton:1909} gives (\lastx) and (\nextx) as a single sentence, but again this is not grammatical.
The phrase \fm{ash jeewú} ‘in her possession’ in (\lastx) is a predicate as is the verb \fm{uwagút} ‘went’ in (\nextx).
Both of these have the form of main clauses and taking either of them as an adjunct clause for the other is semantically problematic.
Both clauses also have a similar phrase in the right periphery that indicates structure repetition which is a common stylistic technique for sequences of sentences in narratives.

\ex\label{ex:94-17-he-came-inside-near-her}%
\exmn{292.9}%
\begingl
	\glpreamble	ᴀcxᴀ′nīnēł uwagu′t yu-ᴀtk!ᴀ′tsk!ᵒ. //
	\glpreamble	ash x̱áni neil uwagút, yú atkʼátskʼu. //
	\gla	{} ash \rlap{x̱áni} @ {} {}
		{} neil @ {} {}
		\rlap{uwagút,} @ {} @ {} @ {} +
		{} yú \rlap{atkʼátskʼu,} @ {} @ {} @ {} @ {} {} //
	\glb	{} ash x̱án -ǐ {}
		{} neil -t {}
		u- ÿa- \rt[¹]{gut} -μH
		{} yú at= kʼí- ÿáts -kʼʷ -í {} //
	\glc	{}[\pr{PP} \xx{3prx·pss} near -\xx{loc} {}]
		{}[\pr{PP} inside -\xx{pnct} {}]
		\xx{zpfv}- \xx{stv}- \rt[¹]{go·\xx{sg}} -\xx{var}
		{}[\pr{DP} \xx{dist} \xx{4n·pss}= base- child -\xx{dim} -\xx{pss} {}] //
	\gld	{} her near -at {}
		{} inside -to {}
		\rlap{\xx{zcnj}.\xx{pfv}.go·\xx{sg}} {} {} {}
		{} that \rlap{boy} {} {} {} {} {} //
	\glft	‘he came inside near her, that boy.’
		//
\endgl
\xe
