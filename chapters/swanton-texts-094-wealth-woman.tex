%!TEX root = ../swanton-texts.tex
%%
%% 94. Wealth Woman (292–293)
%%

\resetexcnt
\chapter{Tlʼanaxéedáḵw: Wealth Woman}\label{ch:94-wealth-woman}

This narrative was told to \citeauthor{swanton:1909} by \fm{Daawoolsʼéesʼ} Don Cameron in Sitka in 1904.
In \citeauthor{swanton:1909}’s original publication it is number 94, running from page 292 to 293 and totalling 16 lines of glossed transcription.
The Tlingit name of this story is \fm{Tlʼanaxéedáḵw} which is the name of the mythical figure featured in this story.
\citeauthor{swanton:1909} did not translate the name, giving the story as “The ʟ!ênᴀxx̣ī′dᴀq” using his transcription.
Some common names in English for \fm{Tlʼanaxéedáḵw} are ‘Wealth Woman’, ‘Lucky Lady’, and ‘Property Woman’, but none of these is a literal translation.

Like several other narratives in \citeauthor{swanton:1909}’s collection, this narrative is quite short.
It is however well known and has been recorded many times from other Tlingit people.
\citeauthor{swanton:1909} has a more detailed version of the same story in English from \fm{Kasáankʼw} as number 35 in his collection, recorded in Wrangell in 1904 \parencite[173–175]{swanton:1909}.
\citeauthor{swanton:1909} also records a mention of her at the end of his number 105 for which see \FIXME{xref to ch.\ \ref{ch:105-kwaashkikhwaan-history}}. 
Anatolii Kamenskii recorded another version in Russian some time before 1906 from an unnamed consultant in Sitka \parencites{kamenskii:1906}[68–70]{kamenskii-kan:1985}.
Viola Garfield provides some notes in English about \fm{Tlʼanaxéedáḵw} from an unknown consultant in 1939, given in the context of a totem pole in Klawock that also features the \fm{G̱unakadeit} mythical creature \parencite[117–118]{garfield-forrest:1948}.
Frederica de Laguna has a version from \fm{Ḵaajaaḵw} Jack Ellis and another from \fm{Shaawát Ḵʼwás} Emma Ellis, both in English, which she recorded in Yakutat in 1954 \parencite[884]{de-laguna:1972}.
Jeff Leer published a version in Tlingit which he recorded in 1988 from \fm{Seidayaa} Elizabeth Nyman \parencite[218–255]{nyman:1993}.
Catharine McClellan recorded five versions in English, one from \fm{Ltʼaanéekaneek} Patsy Henderson \parencite[238–241]{mcclellan-cruikshank:2007b}, two from \fm{Stéew} Angela Sidney \parencite[344–348, 348–353]{mcclellan-cruikshank:2007b}, one from \fm{Yeilnaawú} Tom Peters \parencite[715–719]{mcclellan-cruikshank:2007c}, and one from \fm{Neildayéen} Edgar Sidney \parencite[747–753]{mcclellan-cruikshank:2007c}.
In addition McClellan recorded a personal story of \fm{Ltʼaanéekaneek} Patsy Henderson hearing the \fm{Tlʼanaxéedáḵw} \parencite[241–244]{mcclellan-cruikshank:2007b}.
Aside from these published versions there are also unpublished audio recordings;
the Dauenhauer collection has at least four recordings of the story that await transcription: \fm{Yeilnaawú} Tom Peters in 1973 (tape 42 side \textsc{a}), \fm{Keiḵóokʼw} George Jim in 1970 (tape 91 side \textsc{b}), and \fm{Kéet Yanaayí} Willie Marks in 1976 (tape 75) and 1980 (tape 324 side \textsc{a}).

In a footnote \citeauthor{swanton:1909} explicitly connects \fm{Tlʼanaxéedáḵw} to the Haida “Skîl djā′adai” that he translates as “Property Woman” \parencite[292 fn.\ a]{swanton:1909}.
\citeauthor{enrico:2005} gives this name as \fm{Skils Jaadaay} [\ipa{skils tʃaːtaːj}] in Skidegate and \fm{Skil Jaadee} [\ipa{skil tʃaːteː}] in Masset, both meaning ‘Wealth Woman’ \parencite[525]{enrico:2005}.
\citeauthor{swanton:1905a} offers some details about her in his ethnography of the Haida \parencite[29]{swanton:1905a} and she is mentioned in a couple of stories he recorded \parencites{swanton:1905b}{swanton:1908a} though none of these are strictly about her.

\citeauthor{boas:1916} notes that the Tsimshian narrative from Henry Tate called “Plucking Out Eyes” \parencite[154–158]{boas:1916} is “practically identical” to the Tlingit narratives that \citeauthor{swanton:1909} collected \parencite[746]{boas:1916}.
\citeauthor{boas:1916} also says that the same incident of the child taking out people’s eyes happens as part of the Skidegate Raven stories \parencite[111]{swanton:1905b} and the Masset Raven stories \parencite[143]{swanton:1905b}, but this is not directly connected to the Wealth Woman.

\section{Swanton’s abstract}\label{sec:94-swanton-abstract}

A man saw a woman and two children floating in Auk lake, and he captured one of the children and brought it home.
During the night the child gouged out the eyes of all the people living in the village except one woman, and ate them.
This woman killed the child, and taking on her back her own child, to which she had just given birth, she went up into the woods and became the ʟ!ê′naxx̣ī′dᴀq.
As she went along she ate mussels and fitted the shells together.

\section{Swanton’s translation}\label{sec:94-swanton-translation}

\pgnum{292}
\snum{1–2}A man at Auk went out on the lake after firewood.
\snum{3}On the way round it he saw a woman floating about.
Her hair was long.
Looking at her for some time, he saw that her little ones were with her.
He took one of the children home.
When it became dark they went to sleep.
It was the child of the ʟ!ê′nᴀxx̣ī′dᴀq, and that night it went through the town picking out people’s eyes.
Toward morning a certain woman bore a child.
In the morning, when she was getting up, this [the ʟ!ê′nᴀxx̣ī′dᴀq’s child] came in to her into the house.
The small boy had a big belly full of eyes.
He had taken out the eyes of all the people.
That woman to whom the small boy came had a cane.
He kept pointing at her eyes.
Then she pushed him away with the cane.
When he had done it twice, she pushed it into him.
He was all full of eyes.
After she had killed him the woman went through the
\pgnum{293}
houses.
Then she began to dress herself up.
She took her child up on her back to start wandering.
She said, “I am going to be the ʟ!ê′nᴀxx̣ī′dᴀq.”
When she came down on the beach she kept eating mussels.
She put the shells inside of one another.
As she walks along she nurses her little child.

%\clearpage
\section{Paragraph 1}\label{sec:94-para-1}

\ex\label{ex:94-1-at-auke-lake}%
\exmn{291.1}%
\begingl
	\glpreamble	Āk!ᵘq!ayu′ yē ỵatî //
	\glpreamble	Áakʼwxʼ áyú yéi ÿatee; //
	\gla	{} \rlap{Áakʼwxʼ} @ {} @ {} {} \rlap{áyú} @ {}
		yéi @ \rlap{ÿatee;} @ {} @ {} //
	\glb	{} áaʷ -kʼ -xʼ {} á -yú
		yéi= ÿa- \rt[¹]{tiˋ} -μμL //
	\glc	{}[\pr{PP} lake -\xx{dim} -\xx{loc} {}] \xx{foc} -\xx{dist}
		thus= \xx{stv}- \rt[¹]{be} -\xx{var} //
	\gld	{} lake -little -at {} \rlap{it.is} {}
		thus \rlap{\xx{ncnj}.\xx{impfv}.be} {} {} //
	\glft	‘It is over at Auke Lake where it is;’
		//
\endgl
\xe

\citeauthor{swanton:1909} gives sentences (\lastx) and (\nextx) as a single sentence.
They are however separate main clauses with clausally unmarked verb forms and not subordinate-marked forms of verbs: \fm{yéi ÿatee} not \fm{yéi teeÿí} and \fm{woogoot} not \fm{wugoodí}.
Presumably \citeauthor{swanton:1909} ran them together because they were uttered in sequence without pause by the speaker, so as a compromise they are given as a single capitalized unit divided by a semicolon.

\ex\label{ex:94-2-man-went-there-for-firewood}%
\exmn{291.1}%
\begingl
	\glpreamble	qā akadē′ wugu′t g̣ᴀ′ng̣ā. //
	\glpreamble	ḵáa a kaadé woogoot gáng̱aa. //
	\gla	{} ḵáa {} {} a \rlap{kaadé} @ {} {}
		\rlap{woogoot} @ {} @ {} @ {}
		{} \rlap{gáng̱aa.} @ {} {} //
	\glb	{} ḵáaʷ {} {} a kaa -dě {}
		wu- μ- \rt[¹]{gut} -μμL
		{} gán -g̱ǎa {} //
	\glc	{}[\pr{NP} man {}] {}[\pr{PP} \xx{3n·pss} \xx{hsfc} -\xx{all} {}]
		\xx{pfv}- \xx{stv}- \rt[¹]{go·\xx{sg}} -\xx{var}
		{}[\pr{PP} firewood -\xx{ades} {}] //
	\gld	{} man {} {} its atop -to {}
		\rlap{\xx{ncnj}.\xx{pfv}.go·\xx{sg}} {} {} {}
		{} firewood -for {} //
	\glft	‘a man went there for firewood.’
		//
\endgl
\xe

\ex\label{ex:94-3-going-edge-woman-floating-middle}%
\exmn{291.1}%
\begingl
	\glpreamble	A′yᴀxde ỵanagudī′ayu aosītī′n cāwᴀ′t yū′adīgīgā cwū′ʟ̣īx̣āc. //
	\glpreamble	A yaax̱dé ÿaa nagúdi áyú awsiteen shaawát yú a digeeygéi sh wudlihaash. //
	\gla	{} {} A \rlap{yaax̱dé} @ {} {}
			ÿaa @ \rlap{nagúdi} @ {} @ {} @ {} {}
			\rlap{áyú} @ {}
		\rlap{awsiteen} @ {} @ {} @ {} @ {} @ {} +
		{} {} \rlap{shaawát} @ {} {}
			{} yú a \rlap{digeeygéi} @ {} {}
			sh @ \rlap{wudlihaash.} @ {} @ {} @ {} @ {} @ {} @ {} {}  //
	\glb	{} {} a yaax̱ -dě {} 
			ÿaa= n- \rt[¹]{gut} -μH -ǐ {}
			á -yú
		a- w- s- i- \rt[²]{tin} -μμL
		{} {} sháaʷ- ÿát {}
			{} yú a digeeÿgé -μ {}
			sh= wu- d- l- i- \rt[¹]{hash} -μμL {} {} //
	\glc	{}[\pr{CP} {}[\pr{PP} \xx{3n·pss} edge -\xx{all} {}]
			along= \xx{ncnj}- \rt[¹]{go·\xx{sg}} -\xx{var} -\xx{sub} {}]
			\xx{foc} -\xx{dist}
		\xx{arg}- \xx{pfv}- \xx{xtn}- \xx{stv}- \rt[²]{see} -\xx{var}
		{}[\pr{CP} {}[\pr{NP} woman- child {}]
			{}[\pr{PP} \xx{dist} \xx{3n·pss} middle -\xx{loc} {}]
			\xx{rflx}= \xx{pfv}- \xx{mid}- \xx{csv}- \xx{stv}-
				\rt[¹]{float} -\xx{var} \·\xx{sub} {}] //
	\gld	{} {} its edge -to {}
			along \rlap{\xx{ncnj}.\xx{prog}.go·\xx{sg}} {} {} -while {}
			\rlap{it.is} {} 
		\rlap{3>3.\xx{g̱cnj}.\xx{pfv}.see} {} {} {} {} {} 
		{} {} \rlap{woman} {} {} 
			{} that its middle -at {}
			self \rlap{\xx{ncnj}.\xx{pfv}.float} {} {} {} {} \·that {} //
	\glft	‘It was while he is going to its edge that he saw that a woman had floated herself out in the middle of it.’
		//
\endgl
\xe

The sentence in (\lastx) is a nice example of a relatively complex sentence structure.
It begins with an adjunct clause \fm{a yaax̱dé ÿaa nagúdi} ‘while he is going to its edge’ that contains a progressive aspect verb with a PP.
This adjunct clase is set off from the rest of the clause by a focus marker \fm{áyú} which uses the distal \fm{yú} ‘that, there’ in common with sentence (\ref{ex:94-1-at-auke-lake}) and the later distal determiner in this sentence’s complement clause.
The use of the distal in (\ref{ex:94-1-at-auke-lake}) puts the scene far away from the speaker’s location (Sitka).
The use of the distal in the focus marker could be spatial, placing the event of the man’s travel to the lake edge at a distance from the middle of the lake.
It could also be perspectival, placing the event of the man’s travel at a mental distance from the event of his seeing the woman in the middle of the lake.

The remainder of the sentence in (\lastx) is the main clause.
The verb in this main clause is \fm{awsiteen} ‘he saw it’ which takes the following embedded clause as its complement.
The complement clause starts with an NP \fm{shaawát} ‘woman’ which is the subject of the clausally unmarked verb \fm{sh wudlihaash} ‘she floated herself’.
This verb is based on the verb root \fm{\rt[¹]{hash}} ‘float’ which supports an unaccusative intransitive verb \fm{x̱at wulihaash} ‘I floated’ and a transitive causativized verb \fm{awlihaash} ‘s/he/it made him/her/it float’ \parencite[45]{leer:1976}.
Some verb forms based on this root are \fm{n}-conjugation class members: \fm{ḵúx̱de yaa nalhásh} “he’s drifting back” \parencite[93.1184]{story-naish:1973}, \fm{du jeedé x̱wlihaash} “I let him do it, have it” lit.\ ‘I floated it into his possession’ \parencite[01/94]{leer:1973}, and the nominalization \fm{nalháashadi} ‘floating thing (esp.\ log)’.
Other verb forms are apparently \fm{g}-conjugation class members: \fm{g̱ayéisʼ tléil kei ulháshch} “iron doesn’t float” \parencite[93.1183]{story-naish:1973}.
The verb \fm{sh wudlihaash} in (\lastx) could be in either class, but it has been assigned \fm{n}-conjugation because this is relatively more common.
This verb notably has a reflexive object \fm{sh} ‘self’ which means that it is causative and thus the woman has intentionally floated herself into the middle of the lake.

The complement clause in (\lastx) has a PP separating the subject NP and the verb, namely \fm{yú a digeeygéi} ‘in that middle of it’, referring to the middle of \fm{Áakʼw} Auke Lake.
The final \orth{ā} in \orth{dīgīgā} is probably a misreading of handwritten \orth{ē}.
The postposition in this phrase is the vowel lengthening allomorph \fm{-μ} of the locative better known by its consonant suffix form \fm{-xʼ}.
The noun \fm{digeeygé} ‘middle’ has the apperance of something structurally complex but it is unanalyzable in modern Tlingit.
\FIXME{Finish discussion of \fm{digeeÿgé} and note its occurrence in \#89 and \#99.}
