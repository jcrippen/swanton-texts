%!TEX root = ../swanton-texts.tex
%%
%% 94. Wealth Woman (292–293)
%%

\resetexcnt
\chapter{Tlʼanaxéedáḵw: Wealth Woman}\label{ch:94-wealth-woman}

This narrative was told to \citeauthor{swanton:1909} by \fm{Daawoolsʼéesʼ} Don Cameron in Sitka in 1904.
In \citeauthor{swanton:1909}’s original publication it is number 94, running from page 292 to 293 and totalling 16 lines of glossed transcription.
The Tlingit name of this story is \fm{Tlʼanaxéedáḵw} which is the name of the mythical figure featured in this story.
\citeauthor{swanton:1909} did not translate the name, giving the story as “The ʟ!ênᴀxx̣ī′dᴀq” using his transcription.
Some common names in English for \fm{Tlʼanaxéedáḵw} are ‘Wealth Woman’, ‘Lucky Lady’, and ‘Property Woman’, but none of these is a literal translation.

Like several other narratives in \citeauthor{swanton:1909}’s collection, this narrative is quite short.
It is however well known and has been recorded many times from other Tlingit people.
\citeauthor{swanton:1909} has a more detailed version of the same story in English from \fm{Kasáankʼw} as number 35 in his collection, recorded in Wrangell in 1904 \parencite[173–175]{swanton:1909}.
Anatolii Kamenskii recorded another version in Russian some time before 1906 from an unnamed consultant in Sitka \parencites{kamenskii:1906}[68–70]{kamenskii-kan:1985}.
Viola Garfield provides some notes in English about \fm{Tlʼanaxéedáḵw} from an unknown consultant in 1939, given in the context of a totem pole in Klawock that also features the \fm{G̱unakadeit} mythical creature \parencite[117–118]{garfield-forrest:1948}.
Frederica de Laguna has a version from \fm{Ḵaajaaḵw} Jack Ellis and another from \fm{Shaawát Ḵʼwás} Emma Ellis, both in English, which she recorded in Yakutat in 1954 \parencite[884]{de-laguna:1972}.
Jeff Leer published a version in Tlingit which he recorded in 1988 from \fm{Seidayaa} Elizabeth Nyman \parencite[218–255]{nyman:1993}.
Catharine McClellan recorded five versions in English, one from \fm{Ltʼaanéekaneek} Patsy Henderson \parencite[238–241]{mcclellan-cruikshank:2007b}, two from \fm{Stéew} Angela Sidney \parencite[344–348, 348–353]{mcclellan-cruikshank:2007b}, one from \fm{Yeilnaawú} Tom Peters \parencite[715–719]{mcclellan-cruikshank:2007c}, and one from \fm{Neildayéen} Edgar Sidney \parencite[747–753]{mcclellan-cruikshank:2007c}.
In addition McClellan recorded a personal story of \fm{Ltʼaanéekaneek} Patsy Henderson hearing the \fm{Tlʼanaxéedáḵw} \parencite[241–244]{mcclellan-cruikshank:2007b}.
Aside from these published versions there are also unpublished audio recordings;
the Dauenhauer collection has at least four recordings of the story that await transcription: \fm{Yeilnaawú} Tom Peters in 1973 (tape 42 side \textsc{a}), \fm{Keiḵóokʼw} George Jim in 1970 (tape 91 side \textsc{b}), and \fm{Kéet Yanaayí} Willie Marks in 1976 (tape 75) and 1980 (tape 324 side \textsc{a}).

\section{Swanton’s abstract}\label{sec:94-swanton-abstract}

A man saw a woman and two children floating in Auk lake, and he captured one of the children and brought it home.
During the night the child gouged out the eyes of all the people living in the village except one woman, and ate them.
This woman killed the child, and taking on her back her own child, to which she had just given birth, she went up into the woods and became the ʟ!ê′naxx̣ī′dᴀq.
As she went along she ate mussels and fitted the shells together.

\section{Swanton’s translation}\label{sec:94-swanton-translation}

\section{Paragraph 1}\label{sec:94-para-1}

%\clearpage
