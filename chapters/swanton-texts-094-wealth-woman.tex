%!TEX root = ../swanton-texts.tex
%%
%% 94. Wealth Woman (292–293)
%%

\resetexcnt
\chapter{Tlʼanaxéedáḵw: Wealth Woman}\label{ch:94-wealth-woman}

This narrative was told to \citeauthor{swanton:1909} by \fm{Daawoolsʼéesʼ} Don Cameron in Sitka in 1904.
In \citeauthor{swanton:1909}’s original publication it is number 94, running from page 292 to 293 and totalling 16 lines of glossed transcription.
The Tlingit name of this story is \fm{Tlʼanaxéedáḵw} which is the name of the mythical figure featured in this story.
\citeauthor{swanton:1909} did not translate the name, giving the story as “The ʟ!ênᴀxx̣ī′dᴀq” using his transcription.
Some common names in English for \fm{Tlʼanaxéedáḵw} are ‘Wealth Woman’, ‘Lucky Lady’, and ‘Property Woman’, but none of these is a literal translation.

\section{Swanton’s abstract}\label{sec:94-swanton-abstract}

A man saw a woman and two children floating in Auk lake, and he captured one of the children and brought it home.
During the night the child gouged out the eyes of all the people living in the village except one woman, and ate them.
This woman killed the child, and taking on her back her own child, to which she had just given birth, she went up into the woods and became the ʟ!ê′naxx̣ī′dᴀq.
As she went along she ate mussels and fitted the shells together.

\section{Swanton’s translation}\label{sec:94-swanton-translation}

\section{Paragraph 1}\label{sec:94-para-1}

%\clearpage
