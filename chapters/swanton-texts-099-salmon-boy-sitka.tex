%!TEX root = ../swanton-texts.tex
%%
%% 99. Salmon Boy (301-310)
%%

\resetexcnt
\chapter{Aakʼwtaatseen: The Salmon Boy}\label{ch:099-salmon-boy-sitka}

This narrative was told to \citeauthor{swanton:1909} by Deikeenaakʼw in Sitka.
In the original publication it is number 99, running from page 301 to 310 and totalling 131 lines of glossed transcription.
\citeauthor{swanton:1909}’s original title is “Moldy-End”; this is a loose translation of the name \fm{Shanyaakʼutlaax̱} by which the protagonist is called at one point in this story.
Tlingit people today more often refer to the protagonist by his more prestigious name \fm{Aakʼwtaatseen}, so the title has been changed to fit.
The phrase “The Salmon Boy” is not a translation from Tlingit; it is instead a customary English description of the protagonist that is apparently first recorded by \textcite{de-laguna:1972}.

This text was retranscribed into the modern orthography and translated into English by \fm{Koolyéiḵ} Roby Littlefield and \fm{Daasdiyaa} Ethel Makinen in 2003, with editing by \fm{Ḵudeishgé} Lydia George, \fm{Ḵeixwnéi} Nora Marks Dauenhauer, and \fm{Xwaayeenáḵ} Richard Dauenhauer  \parencite{littlefield-makinen:2003}.
This transliteration and translation has not been archived to my knowledge; I was given copies by both Roby Littlefield and Richard Dauenhauer.
The Littlefield et al.\ version includes significant changes to the text that depart from \citeauthor{swanton:1909}’s original.
For example, the initial sentence that \citeauthor{swanton:1909} transcribed as \orth{Dax̣ē′tayu ᴀnaē′tc Kîksᴀ′dî} is given as \fm{Daxéit áyú áa yéi ḵutíx̱xʼun, Kiks.ádi} in the Littlefield et al.\ version which is a different verb phrase and different tense from the original.
The retranscription here is as faithful as possible to \citeauthor{swanton:1909}’s original transcription.

The same story was later told to \citeauthor{swanton:1909} by Ḵaadashaan in Wrangell, for which see chapter \ref{ch:100-salmon-boy-wrg}.
As a well-known story, several other narratives of this story have been published.
Frederica De Laguna in Yakutat recorded a version by \fm{Shaawát Ḵʼwás} Emma Ellis and another by \fm{Ḵaajax̱daḵeinaa} Sheldon James \parencite[889–890]{de-laguna:1972}.
Catharine McClellan recorded two in Carcross by \fm{Chʼuunèhtèʼ Maa Stéew} Angela Sidney \parencite[279–282]{mcclellan-cruikshank:2007b} and \fm{Yéil Kʼidáa} Jimmy Scotty James \parencite[413–415]{mcclellan-cruikshank:2007b} as well as one in Teslin by \fm{Neildayéen} Edgar Sidney \parencite[765–773]{mcclellan-cruikshank:2007c}.
Ronald Olson recorded a version from \textit{Shagóok} Billy Johnson of Saxman as a story from the \fm{Neix̱.ádi} clan \parencite[34–35]{olson:1967}.
Viola Garfield recorded a version from an unknown consultant (perhaps \fm{Gunyaa} George Gunyah) in Klawock which refers to the \fm{Téelʼ Hít} of the \fm{Lʼeeneidí} clan in Tuxekan \parencite[146–147]{garfield-forrest:1948}.
Kai Birket-Smith and Frederica De Laguna recorded two versions from Eyak consultants in Cordova, one from Old Man Dude \parencite[274]{birket-smith-de-laguna:1938} and one from Galushia Nelson \parencite[272–273]{birket-smith-de-laguna:1938}.

\FIXME{Review Boas’s \textit{Sagen}, Nisg̱aʼa, Coast Tsimshian, and Haida literature.}

\clearpage
\clearpage
%% Column widths (§4.2.1, p. 9).
% X + (0.01875 × 2 = 0.0375) + Y
%\setlength{\Lcolwidth}{0.53125\textwidth}
\setlength{\Lcolwidth}{0.471875\textwidth}
%\setlength{\Rcolwidth}{0.5625\textwidth}
\setlength{\Rcolwidth}{0.490625\textwidth}

\begin{pairs}
\begin{Leftside}
\beginnumbering
% 1
\pstart\noindent
\snum{1}Daxéit áyú áa yana.éich Kiks.ádi;
\snum{2}x̱áat áa yéi s adaanéi nuch.
\snum{3}Áa áwé áxʼ yánde át yaa at naduxʼán.
\snum{4}Yú x̱áat atx̱ʼéeshi sákw daadus.aax̱w de.
\snum{5}Á áwé kéidladi yaÿeexʼ yéi adaané dáasʼaa;
\snum{6}.
\snum{7}A shóotx̱ neil góot áwé óot yaan uwaháa.
\snum{8}«\!Atléi, x̱áat yaan uwaháa.
\snum{9}Atx̱'éeshi ax̱ jeet tí.\!»
\snum{10}Ash jeet aawatée yú atx̱ʼéeshi.
\snum{11}A shanyaa wuditlaax̱.
\snum{12}Yéi aÿawsiḵaa yú atx̱ʼéeshi,
\snum{13}«\!Tsʼas shanÿaakʼw\-tlaax̱ ḵaa x̱ʼéix̱ atéex̱ nuch.
\snum{14}Yantʼéidáx̱ daadus.aax̱w nuch.\!»
\snum{15}Atx̱ʼéeshi aaÿí yéi aÿawsiḵaa.
\snum{16}Chʼu tle a tóotx̱ áwé tʼaaÿawduwaḵaa
\snum{17}«\!I dáasʼayi, a waaḵt uwagút, kéidladi.\!»
\snum{18}Chʼu tle yoo akujeek áwé aadé daak wujixeex.
\snum{19}Chʼu tle a kaadé héenx̱ wujixeex du dáasʼaÿi.
\snum{20}Héen digeeÿgéi daak hóo áwé héende wuduwax̱óotʼ.
\snum{21}A yáx̱ woonee, yú ÿádákʼw.
\snum{22}Ldakát yú x̱áat yéi adaanéiÿi ÿéeÿi áwé du tʼáat loowagúḵ.
\snum{23}Ḵudushée du eeg̱áa.
\snum{24}Chʼu tle tléil wuduskú wáa sá wooneeyi ÿé.
\pend
% 2
\pstart
\snum{25}Chʼu tle yú x̱áat ḵu.aa áyú toowú klig̱éi.
\pend
\endnumbering
\end{Leftside}
%%
%% Column break.
%%
\begin{Rightside}
\beginnumbering
%1
\pstart\noindent
\snum{1}It is Daxéit that the Kiks.ádi always migrate to,
\snum{2}they always prepare fish there.
\snum{3}It is there that people are there finishing up smoking things around there.
\snum{4}People are already bundling salmon for dryfish.
\snum{5}So it is that he works on it for seagulls, a snare;
\snum{6}.
\snum{7}Having gone home afterward, hunger came to him.
\snum{8}\qqk{}“Mommy, hunger has come to me.
\snum{9}Give me dryfish.”
\snum{10}She gave it to him, that dryfish.
\snum{11}The end part of it was mouldy.
\snum{12}He said to that dryfish
\snum{13}\qqk{}“I guess they always give people a mouldy end bit.
\snum{14}They always bind it up from out of sight.”
\snum{15}He said so to the dryfish piece.
\snum{16}\!Just then from inside it someone said back
\snum{17}“Your snare, it has gone to its eye, a seagull.”
\snum{18}Being curious about it just then, he ran out to it.
\snum{19}Just then he ran up to it in the water, his snare.
\snum{20}Having waded out into the middle of the water, he was dragged into the water.
\snum{21}It happened like that, to that little child.
\snum{22}It was all of those who had been working on fish that ran behind him.
\snum{23}They are searching for him.
\snum{24}Just then people did not know how the way that it happened.
\pend
% 2
\pstart
\snum{25}Then those salmon however, they are proud.
\pend
\endnumbering
\end{Rightside}
\end{pairs}
\Columns

\section{Swanton’s abstract}\label{sec:099-swanton-abstract}

A small boy made an angry remark about a piece of moldy salmon and was carried off by the salmon people to their town.
When he became hungry he began eating the salmon eggs lying upon the beach, but was told that they were salmon dung.
Finding that he was homesick, his salmon father diverted him by sending him to Amusement creek and placing his arms around two sand-hill cranes.
By and by they started back with him, and passed through something called sīt which opens and closes, and scars those salmon which are caught in it.
When they camped they made other scars by throwing hot rocks upon one another, as if cooking.
Then they met the herring tribe, with which they had a verbal contest, and finally announced what creeks they would enter.
The boy’s father went to Dax̣ē′t, where the boy let his human father spear him.
When his mother began to cut him open she discovered his copper necklace, and concluded it was her son.
His father put him into a basket and placed it upon the roof, where his spirit began to work in him, and he turned back into a man.
Then he became a great shaman and told the people what had happened to him.
By and by he tested his spirits by sending a raft load of his people over a waterfall under the sea.
The next morning it came up with all the people safe.
He sent his clothes-man to spear land-otter, and, although he had him throw his spear across a point at an invisible animal, it struck the land otter on the tip of the tail and killed it.
He lived to be more than a hundred.

\section{Swanton’s translation}\label{sec:099-swanton-translation}

\snum{1}The Kîksᴀ′dî used to live at Dax̣ē′t, \snum{2}where they dried salmon.
\snum{3}After they had gotten through drying it \snum{4}they tied it up there.
\snum{5}So he (a small boy) was baiting a snare for sea gulls.
\snum{7}When he came into the house afterward he was very hungry.
\snum{8}\qqk{}“Mother, I am hungry.
\snum{9}Give me some dried salmon.” \snum{10}So she gave him a piece of dried salmon \snum{11}which had begun to mold on the corner.
\snum{12}Then he said, \snum{13}\qqk{}“You always give me moldy-cornered ones.” \snum{14}They always began tying up from the corner of the house.
\snum{15}He spoke to the dried salmon.
\snum{16}Just then some one shouted out, \snum{17}\qqk{}“There is a sea gull in your snare.” \snum{18}So he ran down to it.
\snum{19}He ran out into the water to his snare.
\snum{20}When he got out into the midst of the water he looked as if he were pulled down into it.
\snum{22}Then all of the drying salmon ran down to him.
\snum{23}Now the people were hunting for him, but he was nowhere to be seen.
\snum{24}It was not known what had happened to him.
\snum{25}The salmon, however, began feeling very high.
They began to rush about at the mouth of the creek.
It was the salmon people that had done it.
Then the salmon people went out to sea with him.
They went seaward with him toward their homes.
To him it looked as if they were in a canoe.
A chief among these salmon had made him his son.
The sea gull that he had followed out went along with him.
Then he stayed with them in the salmon people’s town.
He was among them for one year.
Well out from that town fish eggs were heaped up.
He began to take up and swallow some of them without asking anybody.
Then the people shouted out, “Moldy-end is eating the townspeople’s dung.” At that time they gave him the name.
Afterward he discovered that the salmon tribe had saved him.
Then he went to lie down and remained in that position.
In the morning his father said, “What did they say to you, my son?” He went out and spoke. “Take him up to Amusement creek.
Put his hands around the necks of the sand-hill cranes at the mout of it.” There he saw two sand-hill cranes jumping up and down, facing each other, at the mouth of the creek.
All creatures, such as brants, could be heard making a noise down in this creek.
This is why it was called Amusement creek.
Where was it that he had been feeling badly?
It all got out of him.

The salmon people all knew the salmon month had come up here which was their month for returning.
They always spawn up here among us.
At once they started back with him.
They started up this way.
Then the cohoes people broke their canoe.
This is why the cohoes come up last.
The ʟ!ūk!nᴀx′dî were going to have the cohoes as an emblem, and this is why the ʟ!ūk!nᴀx′dî are also very slow people.
At once all started, dog salmon and humpbacks.
They started up this way with Lively-frog-in-pond (the boy’s name).
The big salmon people started up thither.
Very soon the salmon tribe came to the “sīt.” It is this sīt which gives scars to whichever one happens to get caught in it.
After all got through, the people looking could see a cloud far down on the horizon which appeared like a canoe.
In the evening they went ashore to camp.
They dug holes in the ground and made flat sticks to stick into the ground.
The salmon tribe always does that way.
Then the salmon people would throw hot rocks upon one another.
Their bodies vibrated with the heat.
It is that that leaves scars on the skin of the salmon.
It was Lively-frog-in-pond that let people know what the salmon people do to one another.

At once they started hitherward up this coast.
The salmon tribe came against the herring tribe.
In the canoes of the salmon tribe one stood up.
He said to them, “When did your cheek-flesh ever fill a man?” The others stood by one another.
The herring tribe said in reply, ”We fed them before you.
Our eggs are our cheek-flesh.
When will the space around your backbone not be dirty?” The salmon tribe started off for the outside coasts of these islands.
When they got outside of them the salmon chief said, “To what creek are you going?” Having held a conference, the salmon people named their choices.
The humpbacks said, “We will go to Saliva creek,”, but the one among them who had taken the man, mentioned Dax̣ē′t.
The salmon people called it Right-to-the-town.
Then they came in sight of the mouth of the creek.
They called the point Floating point, and the smoke house that was there a fort.
It looekd like that in the eyes of the salmon people.
The salmon called human beingsx “seal-children’s dog salmon.” When they first came into the mouth of the creek the people sharpened poles for them to flal on when they jumped.
Then the boys always said, “Upon my father’s.” At once they jumped upon it, where before they had not killed any.
At that they (the people) were very happy.

Now they saw his father plainly coming down from far uip the creek.
They said to him (the boy), “Stand up.” He jumped up. “Very fine,” said his mother.
His mother called him a fine salmon.
After that the salmon swam up the creek.
The women who were cutting salmon were always seated by Dax̣ē′t with their backs downstream.
The salmon, however, were always rushing about down in the creek.
The salmon tribe shouted about those who were cutting.
When they were partly through drying the salmon people said to him, “Go to your mother.” His mother was cutting salmon on the beach.
The canoe floated below her on the back current.
So he floated there with his head sticking out from under it.
Then she called her husband’s attention to it. “A fine salmon is floating here with its head out.” His father took up a hook, for he did not know that it was his son.
It swam out from him.
He never expected [to see] his son again.
One year had passed since he had disappeared.
At once he swam out in front of his father.
When he had hooked it he pulled it out on a sandy bar.
He hit it on the head in order to keep it fresh.
Then he threw it to his wife. “Cut it up.
We will cook it,” [he said].
So she put the salmon down to cut it up in the usual manner.

The Tlingit obtained copper in ancient times.
A chain of twisted copper was around the young man’s neck, for head had gone into the water with it on.
After she had tried to cut around his neck for a while, and found that she could not, she looked at her knife.
There were bits of copper on her knife.
Then she called out to her husband, “Come here.” So they began to examine it.
It was the copper chain that used to hang aorund his son’s neck.
Anciently the people used to have a fine woven basket called łīt!.
As soon as he knew this he threw it into such a basket. [He spit upon it] and blew on eagle’s down.
Then he put the basket enclosing the salmon on the roof of the house.
Toward morning there was a noise inside of it.
His (the boy’s) spirit began to work inside of it.
At daybreak he went up to look at it, and a large man lay where the salmon had been.

The took their things out of all the houses.
When they brought what had been a salmon inside a man went out and spoke to the many Kîksᴀ′dî. “Let all the people go with their heads down.” So it was given out.
They brought up salt and devil’s clubs.
As soon as they had drunk it down in accordance with his directions they vomited.
The devil’s club and sea water were vomited out.
Toward evening the shaman bathed.
Below this town is a little pond named Beating-time-for-shaman lake because he also bathed in that.
In the evening his spirits really came to him, and blood kept running out of his mouth.
The sea gull for which he had gone out came to be his spirit.
THen he showed them all things that were to be done to the salmon down in the creek. [Footnote: That is, the tabus.] “Cut them into four pieces,” he said.
He called [the tabus] Adēỵā′ (“That’s the way”).
After that his spirits said to him, “Tie up a raft over there on the edge of Noisy-waterfall.” He was testing his spirits to see how strong they were.
This waterfall comes down a long distance.
The Kîksᴀ′dî began to get ont he raft, which his spirits named Sea-lion raft.
At once he said “Go.” He began blowing on the raft.
One man was not courageous enough to go down into the waterfall, and when the raft went down he seized the bough of a tree at the edge of the fall.
Then it went under.
It was gone for one night.

Next morning the noise of shamans’ sticks was heard at the mouth of the creek.
The raft came up from underneath.
Meanwhile the one that had saved himself came among his friends and told them that the Kîksᴀ′dî were all destroyed.
Therefore the women were all weeping.
When the shaman saw them he spoke.
His spirits said that the people were not hurt at all.
Nor were their clothes even torn.
This is why a Kîksᴀ′dî is very brave.
The man who jumped out, however, was very much ashamed.
Then they brought the people up from [the place where they had come out].

Now the spirits worked in him, and he sang for another land otter so that the people could see his strength.
He sent out his clothes-man to a point that could be seen below. “Take a spear” [he said].
He went to it.
He saw nothing, and stayed there that night.
Then he came back.
When it was day he (the shaman) said, “Take me down there.” He said, “Go around the point below here.” He said to his clothes-man, “Be brave”.
Then he spit on the end of the spear.
He spoke to get strength.
When he got up after speaking and threw it over the point he hit the land otter in the tail.
Now the shaman sent for it [and said], “Take it round there.” The land otter lay stiff.
The spear was stuck into the end of its tail.
This is why even now the people call that place Point-thrown-across.
He put the shadow of his tongue of the land otter upon it (the shadow).
This is why they named the island Divided-by-motion-of-paddle. [Footnote: By a mere motion of his paddle he cut off the land otter’s tongue.] He fasted eight days on the island, when he cut off the land-otte tongue.
Afterward he came up, and they were going to start home from that place.
He lived for more than a hundred years.
His spirits were of such strength that he lived so long that he could just turn about in one place.

\clearpage
\section{Paragraph 1}\label{sec:099-para-1}

\ex\label{ex:099-1-migrate-to-daxeit}%
\exmn{301.1}%
\begingl
	\glpreamble	Dax̣ē′tayu ᴀnaē′tc Kîksᴀ′dî, //
	\glpreamble	Daxéit áyú áa yana.éich Kiks.ádi; //
	\gla	{} Daxéit {} \rlap{áyú} @ {}
		{} \rlap{áa} @ {} {}
		\rlap{yana.éich} @ {} @ {} @ {} @ {}
		{} Kiks.ádi; {} //
	\glb	{} Daxéit {} á -yú
		{} á -μ {}
		ÿ- n- \rt[¹]{.a} -eμH -ch
		{} Kiks.ádi {} //
	\glc	{}[\pr{DP} \xx{name} {}] \xx{foc} -\xx{dist}
		{}[\pr{PP} \xx{3n} -\xx{loc} {}]
		\xx{qual}- \xx{ncnj}- \rt[¹]{mv·end} -\xx{var} -\xx{rep}
		{}[\pr{DP} \xx{name} {}] //
	\gld	{} Daxéit {} \rlap{it.is} {}
		{} there -at {}
		\rlap{\xx{hab}.migrate} {} {} {} {}
		{} Kiks.ádi {} //
	\glft	‘It is Daxéit that the Kiks.ádi always migrate to,’
		//
\endgl
\xe

\ex\label{ex:099-2-prepare-fish}%
\exmn{301.1}%
\begingl
	\glpreamble	xāt ayē′sᴀtanē′nutc. //
	\glpreamble	x̱áat áa yéi s adaanéi nuch. //
	\gla	x̱áat {} \rlap{áa} @ {} {}
		yéi @ s @ \rlap{adaanéi} @ {} @ {} @ {} @ \•nuch. //
	\glb	x̱áat {} á -μ {}
		yéi= has= a- daa- \rt[²]{ne} -μμH =nuch //
	\glc	salmon {}[\pr{PP} \xx{3n} -\xx{loc} {}]
		thus= \xx{plh}= \xx{arg}- around- \rt[²]{work} -\xx{var} =\xx{hab·aux} //
	\gld	salmon {} there -at {}
		thus they \rlap{3>3.\xx{ncnj}.\xx{impfv}.work} {} {} {} \•always //
	\glft	‘they always prepare salmon there.’
		//
\endgl
\xe

\ex\label{ex:099-3-finish-smoking}%
\exmn{301.1}%
\begingl
	\glpreamble	Ā′awe ᴀq! yê′ndî ᴀt yaᴀtnadu′q!wᴀn //
	\glpreamble	Áa áwé áxʼ yánde át yaa at naduxʼán. //
	\gla	{} \rlap{Áa} @ {} {} \rlap{áwé} @ {}
		{} \rlap{áxʼ} @ {} {}
		{} \rlap{yánde} @ {} {}
		{} \rlap{át} @ {} {}
		yaa @ at @ \rlap{naduxʼán;} @ {} @ {} @ {} //
	\glb	{} á -μ {} á -wé
		{} á -xʼ {}
		{} ÿán -dé {}
		{} á -t {}
		ÿaa= at= n- du- \rt[²]{xʼan} -μH //
	\glc	{}[\pr{PP} \xx{3n} -\xx{loc} {}] \xx{foc} -\xx{mdst}
		{}[\pr{PP} \xx{3n} -\xx{loc} {}]
		{}[\pr{PP} \xx{term} -\xx{all} {}]
		{}[\pr{PP} \xx{3n} -\xx{pnct} {}]
		along= \xx{4n·o}= \xx{ncnj}- \xx{4h·s}- \rt[²]{smoke·food} -\xx{var} //
	\gld	{} there -at {} \rlap{it.is} {}
		{} there -at {}
		{} done -to {}
		{} there -around {}
		along sth \rlap{\xx{zcnj}.\xx{prog}.ppl.smoke·food} {} {} {} //
	\glft	‘It iss there that people are there finishing up smoking things around there.’
		//
\endgl
\xe

\citeauthor{swanton:1909} ran together the sentences in (\lastx) and (\nextx) as a single sentence but this is ungrammatical.
The verb \fm{yaa at naduxʼán} in (\lastx) contains a fourth person nonhuman object pronoun \fm{at=} ‘something, things, stuff’ which means that the phrase \fm{yú x̱áat} ‘that salmon’ in (\nextx) cannot be the object of the verb in (\lastx).

The transcription \orth{yaᴀtnadu′q!wᴀn} in (\lastx) implies \fm{yaa at naduxʼwán} [\ipa{jàː.ʔàt.nà.tù.ˈxʼʷán}] with labialization on the onset [\ipa{xʼʷ}] of the stem syllable.
Although there is a root \fm{\rt{xʼwan}} ‘boot’ and an imperative particle \fm{xʼwán} ‘be sure to’, neither of these makes sense in this context.
Instead the root is \fm{\rt[²]{xʼan}} ‘smoke (food)’ and the labialization has spread from the preceding vowel of [\ipa{tù}].
This labialization spread has been ignored in the retranscription to avoid confusion.

The sentence in (\lastx) has a suprising panoply of spatial PPs, all of which apparently refer to the same location.
This is not ungrammatical but it is semantically somewhat strange.
The reason for this multiplicity of locatives is unclear, though it is possible that the \fm{át} ‘around there’ is instead an anticipatory error for the fourth person nonhuman object \fm{at=}.

\ex\label{ex:099-4-bundling-salmon}%
\exmn{301.2}%
\begingl
	\glpreamble	yuxā′t ᴀtq!ē′cî sᴀkᵘ dādusā′xdê.  //
	\glpreamble	Yú x̱áat atx̱ʼéeshi sákw daadus.aax̱w de. //
	\gla	{} Yú x̱áat {}
		{} {} \rlap{atx̱ʼéeshi} @ {} @ {} @ {} {} sákw {} +
		\rlap{daadus.aax̱w} @ {} @ {} @ {} @ {} @ {} de //
	\glb	{} yú x̱áat {}
		{} {} at= \rt[²]{x̱ʼish} -μμH -í {} sákw {}
		daa- du- d- s- \rt[²]{.ax̱w} -μμH de //
	\glc	{}[\pr{DP} \xx{dist} salmon {}]
		{}[\pr{PP} {}[\pr{DP} \xx{4n·o}= \rt[²]{dry·fish} -\xx{var} -\xx{nmz} {}] \xx{fut} {}]
		around- \xx{4h·s}- \xx{mid}- \xx{xtn}- \rt[²]{tie} -\xx{var}
		already //
	\gld	{} that salmon {}
		{} {} \rlap{dryfish} {} {} {} {} for {}
		\rlap{around.\xx{zcnj}.\xx{impfv}.bundle} {} {} {} {} {} already //
	\glft	‘People are already bundling salmon for dryfish.’
		//
\endgl
\xe

The word \fm{atx̱ʼéeshi} is etymologically complex, being formed from a root \fm{\rt[²]{x̱ʼish}} or \fm{\rt[²]{x̱ʼiʼsh}} that is not otherwise attested in the language but which can be connected to a Proto-Dene-Eyak word \fm[*]{qʼəs} ‘one half, one of a pair’.
See the discussion of this etymology on page \pageref{note:100-dryfish-discussion} for more details.

\ex\label{ex:099-5-works-on-snare}%
\exmn{301.2}%
\begingl
	\glpreamble	A′awe kē′ʟ̣adiyaỵiq! yeadā′na dā′s!a, //
	\glpreamble	Á áwé kéidladi yaÿeexʼ yéi adaané dáasʼaa; //
	\gla	{} Á {} \rlap{áwé} @ {}
		{} kéidladi \rlap{yaÿeexʼ} @ {} @ {} {}
		yéi @ \rlap{adaané} @ {} @ {} @ {} +
		{} \rlap{dáasʼaa;} @ {} @ {} {} //
	\glb	{} á {} á -wé
		{} kéidladi ÿá- ÿee -xʼ {}
		yéi= a- daa- \rt[²]{ne} -μH
		{} \rt[²]{dasʼ} -μμH -aa {} //
	\glc	{}[\pr{DP} \xx{3n} {}] \xx{foc} -\xx{mdst}
		{}[\pr{PP} seagull face- below -\xx{loc} {}]
		thus= \xx{arg}- around- \rt[²]{work} -\xx{var}
		{}[\pr{DP} \rt[²]{snare} -\xx{var} -\xx{nmz} {}] //
	\gld	{} it {} \rlap{it.is} {}
		{} seagull face- below -at {}
		thus\• \rlap{3>3.\xx{ncnj}.\xx{impfv}.work} {} {} {}
		{} \rlap{snare} {} {} {} //
	\glft	‘So it is that he works on it for seagulls, a snare;’
		//
\endgl
\xe

The noun \fm{dáasʼaa} ‘snare’ in (\lastx) is based on the root \fm{\rt[¹]{dasʼ}} ‘wear through by abrasion, cut with wire; snare’.
\FIXME{Cite some ethnographic discussions.
Contrast with the \fm{asnóotʼaa} in chapter \ref{ch:100-salmon-boy-wrg} sentence \ref{ex:100-4-catching-gulls}.}

\FIXME{Discuss meaning of \fm{yayeexʼ} in this context.
Translated as ‘for’ for simplicity.}

\ex\label{ex:099-6-}%
\exmn{301.3}%
\begingl
	\glpreamble	awāq dē′snaaqnutc. //
	\glpreamble	a waaḵdé s ana.aax̱w nuch. //
	\gla	{} a \rlap{waaḵdé} @ {} {}
		s @ \rlap{ana.aax̱w} @ {} @ {} @ {} @ \•nuch //
	\glb	{} a waaḵ -dé {}
		has= a- n- \rt[²]{.ax̱w} -μμL =nuch //
	\glc	{}[\pr{DP} \xx{3n} eye -\xx{all} {}]
		\xx{plh}= \xx{arg}- \xx{ncnj}- \rt[²]{bind} -\xx{var} =\xx{hab·aux} //
	\gld	 //
	\glft	‘’
		//
\endgl
\xe

The verb in (\lastx) is difficult to identify.
\citeauthor{swanton:1909}’s transcription \orth{dē′snaaqnutc} has an initial \fm{-dé} which is the head of the preceding PP \fm{a waaḵdé}.
The remainder \orth{snaaqnutc} clearly ends with \fm{nuch} which is the habitual auxiliary, implying that the verb is probably an activity imperfective since this is the typical aspect that the habitual auxiliary is used with.
The \orth{snaaq} is naively something like \fm{sna.aḵ} because \citeauthor{swanton:1909} usually assumes an implicit glottal stop between vowels.
There is no root \fm[*]{\rt{.aḵ}} and the root \fm{\rt[²]{.aḵw}} ‘plan, attempt, try; order, command’ is not felicitous in this context.
Phonologically similar roots with plausibly applicable meanings are: \fm{\rt[²]{.ak}} ‘weave’, \fm{\rt[¹]{.aʼk}} ‘stagger when wounded’, \fm{\rt[²]{.ax̱}} ‘handle fabric; drape’, and \fm{\rt[²]{.ax̱w}} ‘bind’.
If \orth{snaaq} were instead read as something like \fm{s naaḵ} then the phonologically similar roots are: \fm{\rt[¹]{nak}} ‘shaped’, \fm{\rt{naʼk}} ‘numb’, \fm{\rt[²]{nakw}} ‘medicine’, \fm{\rt[¹]{naḵ}} ‘plural stand’, \fm{\rt[²]{naḵ}} ‘abandon’ (cf.\ postposition \fm{náḵ} ‘leaving behind’), \fm{\rt[¹]{naḵw}} ‘rot (wood)’, \fm{\rt[²]{naḵw}} ‘octopus; fish with octopus as bait’.
Of these \FIXME{discuss options, decide}

\ex\label{ex:099-7-after-home-hunger-appear}%
\exmn{301.3}%
\begingl
	\glpreamble	Acū′tc nēł gū′dawe At yan ūwaxā′. //
	\glpreamble	A shóotx̱ neil góot áwé óot yaan uwaháa. //
	\gla	{} {} A \rlap{shóotx̱} @ {} {} {} neil @ {} {}
			\rlap{góot} @ {} @ {} @ {} {} \rlap{áwé} @ {} +
		{} \rlap{óot} @ {} {} yaan @ \rlap{uwaháa.} @ {} @ {} @ {} //
	\glb	{} {} a shú -dáx̱ {} {} neil {} {}
			{} \rt[¹]{gut} -μμH {} {} á -wé
		{} ú -t {} ÿaan= u- i- \rt[¹]{haʰ} -μμH //
	\glc	{}[\pr{CP} {}[\pr{PP} \xx{3n·pss} end -\xx{abl} {}] {}[\pr{PP} home -\xx{pnct} {}]
			\xx{zcnj}\· \rt[¹]{go·\xx{sg}} -\xx{var} \·\xx{sub} {}] \xx{foc} -\xx{mdst}
		{}[\pr{PP} \xx{3h} -\xx{pnct} {}] hunger= \xx{zpfv}- \xx{stv}- \rt[¹]{mv·invis} -\xx{var} //
	\gld	{} {} it end -from {} {} home -to {}
			\rlap{\xx{csec}.go·\xx{sg}} {} {} \·when {} \rlap{it.is} {}
		{} him -to {} hunger \rlap{\xx{zcnj}.\xx{pfv}.appear} {} {} {} //
	\glft	‘Having gone home afterward, hunger came to him.’
		//
\endgl
\xe

\ex\label{ex:099-8-mommy-hungry}%
\exmn{301.4}%
\begingl
	\glpreamble	“ᴀʟe′ xāt yan uwaha′. //
	\glpreamble	«\!Atléi, x̱áat yaan uwaháa. //
	\gla	«\!Atléi, {} \rlap{x̱áat} @ {} {} yaan @ \rlap{uwaháa.} @ {} @ {} @ {} //
	\glb	\pqp{}atléi {} x̱á -t {} ÿaan= u- i- \rt[¹]{haʰ} -μμH //
	\glc	\pqp{}mommy {}[\pr{PP} \xx{1sg} -\xx{pnct} {}] hunger= \xx{zpfv}- \xx{stv}- \rt[¹]{mv·invis} -\xx{var} //
	\gld	\pqp{}mommy {} me -to {} hunger \rlap{\xx{zcnj}.\xx{pfv}.appear} {} {} {}  //
	\glft	‘“Mommy, hunger has come to me.’
		//
\endgl
\xe

\ex\label{ex:099-9-gimme-dryfish}%
\exmn{301.4}%
\begingl
	\glpreamble	ᴀtq!ē′cî ᴀxdjī′t tê.” //
	\glpreamble	Atx̱'éeshi ax̱ jeet tí.\!» //
	\gla	{} \rlap{atx̱ʼéeshi} @ {} @ {} @ {} {}
		{} ax̱ \rlap{jeet} @ {} {} \rlap{tí.\!»} @ {} @ {} @ {} //
	\glb	{} at= \rt[²]{x̱ʼish} -μμH -í {}
		{} ax̱ jee -t {} {} {} \rt[²]{ti} -μH //
	\glc	{}[\pr{DP} \xx{4n·o}= \rt[²]{dry·fish} -\xx{var} -\xx{nmz} {}]
		{}[\pr{PP} \xx{1sg·pss} poss’n -\xx{pnct} {}] \xx{zcnj}\· \xx{2sg·s}\· \rt[²]{handle} -\xx{var} //
	\gld	{} \rlap{dryfish} {} {} {} {}
		{} my poss’n -to {} \rlap{\xx{imp}.you·\xx{sg}.handle} {} {} {} //
	\glft	‘Give me dryfish.”’
		//
\endgl
\xe

\ex\label{ex:099-10-gave-him-dryfish}%
\exmn{301.5}%
\begingl
	\glpreamble	ᴀcdjī′t ā′wate yuᴀtq!ē′cî. //
	\glpreamble	Ash jeet aawatée yú atx̱ʼéeshi. //
	\gla	{} Ash \rlap{jeet} {} {} \rlap{aawatée} @ {} @ {} @ {} @ {}
		{} yú {} \rlap{atx̱ʼéeshi.} @ {} @ {} @ {} {} {} //
	\glb	{} ash jee -t {} a- wu- i- \rt[²]{ti} -μμH
		{} yú {} at= \rt[²]{x̱ʼish} -μμH -í {} {} //
	\glc	{}[\pr{PP} \xx{3prx·pss} poss’n -\xx{pnct} {}] \xx{arg}- \xx{pfv}- \xx{stv}- \rt[²]{handle} -\xx{var}
		{}[\pr{DP} \xx{dist} {}[\pr{NP} \xx{4n·o}= \rt[²]{dry·fish} -\xx{var} -\xx{nmz} {}] {}] //
	\gld	{} his poss’n -to {} \rlap{3>3.\xx{zcnj}.\xx{pfv}.handle} {} {} {} {}
		{} that {} \rlap{dryfish} {} {} {} {} {} //
	\glft	‘She gave it to him, that dryfish.’
		//
\endgl
\xe

\ex\label{ex:099-11-head-mouldy}%
\exmn{301.5}%
\begingl
	\glpreamble	Acê′nya wudîʟā′x. //
	\glpreamble	A shanyaa wuditlaax̱. //
	\gla	{} A \rlap{shanyaa} @ {} {} \rlap{wuditlaax̱.} @ {} @ {} @ {} @ {} //
	\glb	{} a shá- niÿaa {} wu- d- i- \rt[¹]{tlax̱} -μμL //
	\glc	{}[\pr{DP} \xx{3n·pss} head- direction {}] \xx{pfv}- \xx{mid}- \xx{stv}- \rt[¹]{mould} -\xx{var} //
	\gld	{} its head- direction {} \rlap{\xx{ncnj}?.\xx{pfv}.mould} {} {} {} {} //
	\glft	‘The head end of it was mouldy.’
		//
\endgl
\xe

The verb in (\lastx) was transcribed by \citeauthor{swanton:1909} as \orth{wudîʟā′x} which suggests a form \fm{wuditlaax̱} with a long low tone \fm{-μμL} stem.
The use of a long low tone \fm{-μμL} stem in the perfective aspect implies that the verb is a member of one of the three \fm{n-}, \fm{g̱-}, or \fm{g}-conjugation classes.
But this verb is only documented as a member of the \fm{∅}-conjugation \parencites[08/142]{leer:1973}[471]{leer:1976}[258]{edwards:2009} so that its perfective aspect form predictably has a short high tone \fm{-μH} stem, i.e.\ \fm{wuditláx̱}.
If \citeauthor{swanton:1909} had heard \fm{wuditláx̱} we would expect him to transcribe something like \orth{wudîʟᴀ′x} but he clearly has \orth{ā} not \orth{ᴀ} and it is difficult to see how this could have been a mistake introduced during manuscript preparation or typesetting.
Differences in conjugation class sometimes have meaningful consequences for the semantics of particular verbs, but it is unclear what the different meanings would be for this particular verb.
Despite the temptation to reanalyze this as \fm{wuditláx̱} and thus \fm{∅}-conjugation, the long low tone \fm{-μμL} stem has been retained in the retranscription with the hope that the semantic consequences of this will be clarified by further study.
The conjugation class has been tentatively given in the analysis as \fm{n}-conjugation since \fm{∅}/\fm{n} is the most common pairing when a non-motion verb occurs in two conjugation classes.

\ex\label{ex:099-12-he-says-to-dryfish}%
\exmn{301.5}%
\begingl
	\glpreamble	Ye aỵa′osîqa yuᴀtq!ē′cî, //
	\glpreamble	Yéi aÿawsiḵaa yú atx̱ʼéeshi, //
	\gla	Yéi @ \rlap{aÿawsiḵaa} @ {} @ {} @ {} @ {} @ {} @ {}
		{} yu {} \rlap{atx̱ʼéeshi,} @ {} @ {} @ {} {} {} //
	\glb	yéi= a- ÿ- wu- s- i- \rt[¹]{ḵa} -μμL
		{} yú {} at= \rt[²]{x̱ʼish} -μμH -í {} {} //
	\glc	thus= \xx{arg}- \xx{qual}- \xx{pfv}- \xx{csv}- \xx{stv}- \rt[¹]{say} -\xx{var}
		{}[\pr{DP} \xx{dist} {}[\pr{NP} \xx{4n·o}= \rt[²]{dry·fish} -\xx{var} -\xx{nmz} {}] {}] //
	\gld	thus \rlap{3>3.\xx{ncnj}.\xx{pfv}.say·to} {} {} {} {} {} {}
		{} that {} \rlap{dryfish} {} {} {} {} {} //
	\glft	‘He said to that dryfish’
		//
\endgl
\xe

\ex\label{ex:099-13-always-give-mouldy-end-bit}%
\exmn{301.6}%
\begingl
	\glpreamble	“Ts!ᴀs cᴀnỵā′k!ᵘʟāx qaq!ē′xᴀtexnutc.” //
	\glpreamble	«\!Tsʼas shanÿaakʼwtlaax̱ ḵaa x̱ʼéix̱ atéex̱ nuch. //
	\gla	«\!\rlap{Tsʼas} @ {}
		{} \rlap{shanÿaakʼwtlaax̱} @ {} @ {} @ {} @ {} {}
		{} ḵaa \rlap{x̱ʼéix̱} @ {} {}
		\rlap{atéex̱} @ {} @ {} @ {} @ \•nuch.\!» //
	\glb	\pqp{}tsʼa =sí
		{} shá- niÿaa -kʼw= \rt[¹]{tlax̱} -μμL {}
		{} ḵaa x̱ʼé -x̱ {}
		a- \rt[²]{ti} -μμH -x̱ =nooch //
	\glc	\pqp{}just =\xx{dub}
		{}[\pr{DP} head- direction -\xx{dim}= \rt[¹]{mould} -\xx{var} {}]
		{}[\pr{PP} \xx{4h·pss} mouth -\xx{pert} {}]
		\xx{arg}- \rt[²]{handle} -\xx{var} -\xx{rep} =\xx{hab·aux} //
	\gld	\pqp{}just \•maybe
		{} head- direction -little \rlap{mould} {} {}
		{} ppl’s mouth -at {}
		\rlap{3>3.\xx{zcnj}.\xx{impfv}.handle.\xx{rep}} {} {} {} {} \•always //
	\glft	‘“I guess they always give people a mouldy end bit.’
		//
\endgl
\xe

\ex\label{ex:099-14-bind-up-out-of-sight}%
\exmn{301.6}%
\begingl
	\glpreamble	Yit!ē′dᴀx dādusā′nutc. //
	\glpreamble	Yantʼéidáx̱ daadus.aax̱w nuch.\!» //
	\gla	{} \rlap{Yantʼéidáx̱} @ {} @ {} {}
		\rlap{daadus.aax̱w} @ {} @ {} @ {} @ {} @ {} @ \•nuch.\!» //
	\glb	{} ÿán- tʼéiᵏ -dáx̱ {}
		daa- du- d- s- \rt[²]{.ax̱w} -μμL =nooch //
	\glc	{}[\pr{PP} below- behind -\xx{abl} {}]
		around- \xx{4h·s}- \xx{mid}- \xx{xtn}- \rt[²]{bind} -\xx{var} =\xx{hab·aux} //
	\gld	{} \xx{unkn}- behind -from {}
		\rlap{\xx{zcnj}.\xx{impfv}.ppl.bind} {} {} {} {} {} \•always //
	\glft	‘They always bind it up from out of sight.”’
		//
\endgl
\xe

The phrase that \citeauthor{swanton:1909} transcribes \orth{Yit!ē′dᴀx} is probably a mishearing of \fm{yantʼéidáx̱}.
There is no attested noun \fm{yitʼéiᵏ}, but \citeauthor{leer:1973} gives \fm{yantʼéik} as meaning “behind, out of the way place (can’t be seen)” and “out of the way, out of sight”, including an example sentence \fm{yantʼéide yéi gax̱dusnéi} “they will put it out of the way, out of sight” \parencite[07/123]{leer:1976}.
There is a related \fm{ÿatʼéiᵏ} but this occurs only with a possessor as in \fm{du yatʼéik} “behind his back” and \fm{ḵaa ÿatʼéináx̱} “when no one is looking” \parencite[07/122]{leer:1976} and since (\lastx) lacks a possessor this is a less likely candidate.
The relational noun \fm{tʼéiᵏ} ‘behind’ is etymologically connected to \fm{tʼáaᵏ} ‘behind’.
The \fm{ÿan-} in \fm{ÿantʼéiᵏ} is of uncertain meaning, but it is probably connnected historically to the noun \fm{ÿán} ‘shore’ from Pre-Tlingit \fm[*]{ŋan} < PND \fm[*]{ŋənˀ} ‘ground, earth’; see the discussion on page \pageref{note:100-shore-discussion}.
The \fm{ÿan-} could alternatively be \fm{ÿee-} ‘below, inside of house’ \parencite[cf.][3]{littlefield-makinen:2003}, but the compound \fm[*]{ÿeetʼéiᵏ} or \fm[*]{ÿitʼéiᵏ} is unattested and is grammatically problematic because it would be expected to be a relational noun with an obligatory possessor that does not occur in (\lastx).

\citeauthor{swanton:1909} analyzed the sentence in (\lastx) as not being part of the quoted speech in (\ref{ex:099-13-always-give-mouldy-end-bit}).
His translation of (\lastx) as “They always began tying up from the corner of the house” makes this sentence into a non sequitur.
Reanalyzing it as part of the quoted speech, (\lastx) can be seen as a continuation of the protagonist’s complaint about the mouldy dryfish, specifically accusing people of hiding the mouldy part.
With this interpretation the quoted speech is then naturally bracketed by the two speech verbs in (\ref{ex:099-12-he-says-to-dryfish}) and (\ref{ex:099-15-said-so-to-dryfish}), with both sentences making it clear that the sentences in (\ref{ex:099-13-always-give-mouldy-end-bit}) and (\ref{ex:099-14-bind-up-out-of-sight}) are directed at the piece of dryfish in the protagonist’s possession.

\ex\label{ex:099-15-said-so-to-dryfish}%
\exmn{301.7}%
\begingl
	\glpreamble	ᴀtq!ē′cî aỵî yē aỵao′sîqa. //
	\glpreamble	Atx̱ʼéeshi aaÿí yéi aÿawsiḵaa. //
	\gla	{} {} \rlap{Atx̱ʼéeshi} @ {} @ {} @ {} {} \rlap{aaÿí} @ {} {}
		yéi @ \rlap{aÿawsiḵaa.} @ {} @ {} @ {} @ {} @ {} @ {} //
	\glb	{} {} at= \rt[²]{x̱ʼish} -μμH -í {} aa -í {}
		yéi= a- ÿ- wu- s- i- \rt[¹]{ḵa} -μμL //
	\glc	{}[\pr{DP} {}[\pr{NP} \xx{4n·o}= \rt[²]{dry·fish} -\xx{var} -\xx{nmz} {}] \xx{part} -\xx{pss} {}]
		thus= \xx{arg}- \xx{qual}- \xx{pfv}- \xx{csv}- \rt[¹]{say} -\xx{var} //
	\gld	{} {} \rlap{dryfish} {} {} {} {} part -of {}
		thus \rlap{3>3.\xx{ncnj}.\xx{pfv}.say·to} {} {} {} {} {} {} //
	\glft	‘He said so to the dryfish piece.’
		//
\endgl
\xe

\ex\label{ex:099-16-from-inside-said-back}%
\exmn{301.7}%
\begingl
	\glpreamble	Tc!uʟe′ atū′xawe t!āỵaodowaqa, //
	\glpreamble	Chʼu tle a tóotx̱ áwé tʼaaÿawduwaḵaa //
	\gla	Chʼu tle {} a \rlap{tóotx̱} @ {} {} \rlap{áwé} @ {}
		\rlap{tʼaaÿawduwaḵaa} @ {} @ {} @ {} @ {} @ {} @ {} //
	\glb	chʼu tle {} a tóo -dáx̱ {} á -wé
		tʼaa- ÿ- wu- du- i- \rt[¹]{ḵa} -μμL //
	\glc	just then {}[\pr{PP} \xx{3n·pss} inside -\xx{abl} {}] \xx{foc} -\xx{mdst}
		back- \xx{qual}- \xx{pfv}- \xx{4h·s}- \xx{stv}- \rt[¹]{say} -\xx{var} //
	\gld	just then {} its inside -from {} \rlap{it.is} {}
		\rlap{back.\xx{ncnj}.\xx{pfv}.say} {} {} {} {} {} {} //
	\glft	‘Just then from inside it someone said back’
		//
\endgl
\xe

\ex\label{ex:099-17-snare-eye-seagull}%
\exmn{301.8}%
\begingl
	\glpreamble	“Edā′s!aye awā′q!t uwagu′t kē′ʟ̣adî.” //
	\glpreamble	«\!I dáasʼayi, a waaḵt uwagút, kéidladi.\!» //
	\gla	{} \llap{«\!}I {} \rlap{dáasʼayi,} @ {} @ {} @ {} {} {}
		{} a \rlap{waaḵt} @ {} {}
		\rlap{uwagút,} @ {} @ {} @ {} +
		{} kéidladi. {} //
	\glb	{} i {} \rt[¹]{dasʼ} -μμH -aa -í {} {}
		{} a waaḵ -t {}
		u- i- \rt[¹]{gut} -μH
		{} kéidladi {} //
	\glc	{}[\pr{DP} \xx{2sg·pss} {}[\pr{NP} \rt[²]{snare} -\xx{var} -\xx{nmz} -\xx{pss} {}] {}]
		{}[\pr{PP} \xx{3n·pss} eye -\xx{pnct} {}]
		\xx{zpfv}- \xx{stv}- \rt[¹]{go·\xx{sg}} -\xx{var}
		{}[\pr{DP} seagull {}] //
	\gld	{} your {} \rlap{snare} {} {} {} {} {}
		{} its eye -to {}
		\rlap{\xx{zcnj}.\xx{pfv}.go·\xx{sg}} {} {} {}
		{} seagull {} //
	\glft	‘“Your snare, it has gone to its eye, a seagull.”’
		//
\endgl
\xe

\ex\label{ex:099-18-curious-ran-to-it}%
\exmn{301.8}%
\begingl
	\glpreamble	Tc!uʟe′ akudjī′nawe ādê′ dak wudjix̣ī′x̣. //
	\glpreamble	Chʼu tle yoo akujeek áwé aadé daak wujixeex. //
	\gla	{} Chʼu tle
			yoo @ \rlap{akujeek} @ {} @ {} @ {} @ {} @ {} @ {} {}
		\rlap{áwé} @ {} +
		{} \rlap{aadé} @ {} {}
		daak @ \rlap{wujixeex.} @ {} @ {} @ {} @ {} @ {} //
	\glb	{} chʼu tle
			yoo= a- k- u- \rt[²]{jiʰ} -μμL -k {} {}
		á -wé
		{} á -dé {}
		dáak= wu- d- sh- i- \rt[¹]{xix} -μμL //
	\glc	{}[\pr{CP} just then
			\xx{alt}= \xx{arg}- \xx{qual}- \xx{irr}- \rt[²]{think} -\xx{var} -\xx{rep} \·\xx{sub} {}]
		\xx{foc} -\xx{mdst}
		{}[\pr{PP} \xx{3n} -\xx{all} {}]
		seaward= \xx{pfv}- \xx{mid}- \xx{pej}- \xx{stv}- \rt[¹]{fall} -\xx{var} //
	\gld	{} just then \xx{alt} \rlap{3>3.\xx{zcnj}.\xx{impfv}.curious.\xx{rep}} {} {} {} {} {} when {}
		\xx{it.is} {}
		{} it -to {}
		seaward \rlap{\xx{ncnj}.\xx{pfv}.run·\xx{sg}} {} {} {} {} {} //
	\glft	‘Being curious about it just then, he ran out to it.’
		//
\endgl
\xe

\citeauthor{swanton:1909} has \orth{akudjī′n} in (\lastx) which suggests \fm{akujeen} or \fm{akujéen} but this is nonsense.
The \orth{n} is plausibly a typo for \orth{k}, giving a form like \fm{akujeek}.
This is reminiscent of the verb \vblex{yoo akoojeek}{n}{irrealis repetitive state}{s/he is curious about him/her/it} which is based on the root \fm{\rt[²]{jiʰ}} ‘think’ with the addition of qualifier \fm{k-} and irrealis \fm{u-} and the alternating repetitive state \fm{yoo=i-…-k}.
The resulting adjunct clause \fm{yoo akujeek} has predictably suppressed \fm{i-}, but it is missing the expected subordinate clause suffix \fm{-í} since the form is not \fm{yoo akujeegí}.

\ex\label{ex:099-19-ran-to-snare-in-water}%
\exmn{301.9}%
\begingl
	\glpreamble	Tc!uʟe′ akā′de hīnx wudjix̣ī′x̣ dudā′s!aỵî. //
	\glpreamble	Chʼu tle a kaadé héenx̱ wujixeex du dáasʼaÿi. //
	\gla	Chʼu tle {} a \rlap{kaadé} @ {} {} {} \rlap{héenx̱} @ {} {}
		\rlap{wujixeex} @ {} @ {} @ {} @ {} @ {} +
		{} du {} \rlap{dáasʼaÿi.} @ {} @ {} @ {} {} {} //
	\glb	chʼu tle {} a ká -dé {} {} héen -x̱ {}
		wu- d- sh- i- \rt[¹]{xix} -μμL
		{} du {} \rt[¹]{dasʼ} -μμH -aa -í {} {} //
	\glc	just then {}[\pr{PP} \xx{3n·pss} \xx{hsfc} -\xx{all} {}] {}[\pr{PP} water -\xx{pert} {}]
		\xx{pfv}- \xx{mid}- \xx{pej}- \xx{stv}- \rt[¹]{fall} -\xx{var}
		{}[\pr{DP} \xx{3h·pss} {}[\pr{NP} \rt[²]{snare} -\xx{var} -\xx{nmz} -\xx{pss} {}] {}] //
	\gld	just then {} its atop -to {} {} water -in {}
		\rlap{\xx{ncnj}.\xx{pfv}.run·\xx{sg}} {} {} {} {} {}
		{} his {} \rlap{snare} {} {} {} {} {} //
	\glft	‘Just then he ran up to it in the water, his snare.’
		//
\endgl
\xe

\ex\label{ex:099-20-waded-out-dragged-in}%
\exmn{301.10}%
\begingl
	\glpreamble	Hī′ndî gīỵgē′daqx̣ū′awawe hī′nde wuduwaxō′t! //
	\glpreamble	Héen digeeÿgéi daak hóo áwé héende wuduwax̱óotʼ. //
	\gla	{} {} Héen \rlap{digeeÿgéi} @ {} {} daak @ \rlap{hóo} @ {} @ {} @ {} {}
		\rlap{áwé} @ {} +
		{} \rlap{héende} @ {} {}
		\rlap{wuduwax̱óotʼ.} @ {} @ {} @ {} @ {} //
	\glb	{} {} héen digeeÿigé -μ {} daak= {} \rt[¹]{hu} -μμH {} {}
		á -wé
		{} héen -dé {}
		wu- du- i- \rt[²]{x̱utʼ} -μμH //
	\glc	{}[\pr{CP} {}[\pr{PP} water middle -\xx{loc} {}] seaward= \xx{zcnj}- \rt[¹]{wade} -\xx{var} \·\xx{sub} {}]
		\xx{foc} -\xx{mdst}
		{}[\pr{PP} water -\xx{all} {}]
		\xx{pfv}- \xx{4h·s}- \xx{stv}- \rt[²]{drag} -\xx{var} //
	\gld	{} {} water middle -at {} seaward \rlap{\xx{csec}.wade} {} {} {} {}
		\rlap{it.is} {}
		{} water -to {}
		\rlap{\xx{ncnj}.\xx{pfv}.ppl.drag} {} {} {} {} //
	\glft	‘Having waded out into the middle of the water, he was dragged into the water.’
		//
		
\endgl
\xe

\ex\label{ex:099-21-happened-like-that}%
\exmn{301.10}%
\begingl
	\glpreamble	āyᴀ′x wū′nî yuỵadᴀ′k!ᵘ. //
	\glpreamble	A yáx̱ woonee, yú ÿádákʼw. //
	\gla	{} A yáx̱ {} \rlap{woonee,} @ {} @ {} @ {}
		{} yú \rlap{ÿádákʼw} @ {} {} //
	\glb	{} a yáx̱ {} wu- i- \rt[¹]{niʰ} -μμL
		{} yú ÿát -kʼw {} //
	\glc	{}[\pr{PP} \xx{3n} \xx{sim} {}] \xx{pfv}- \xx{stv}- \rt[¹]{happen} -\xx{var}
		{}[\pr{DP} \xx{dist} child -\xx{dim} {}] //
	\gld	{} it like {} \rlap{\xx{ncnj}.\xx{pfv}.happen} {} {} {}
		{} that child -little {} //
	\glft	‘It happened like that, to that little child.’
		//
\endgl
\xe

\citeauthor{swanton:1909}’s transcription \orth{āyᴀ′x} in (\lastx) suggests something like \fm{áa yáx̱} with a long vowel on the first syllable, but this does not make sense.
The \fm{áa} is a variation of \fm{áxʼ} ‘there’ which is composed of the third person nonhuman pronoun \fm{á} and the locative suffix \fm{-xʼ} that has a vowel lengthening allomorph \fm{-μ} after a word-final short vowel.
The following \fm{yáx̱} ‘like, similar’ is the similative postposition which requires an NP to precede it, not a PP.
The simplest solution followed in (\lastx) is to disregard length on \orth{ā}.
Another possible reading could be \fm{áa a yáx̱} ‘there like that’, but this introduces another syllable which is not attested in \citeauthor{swanton:1909}’s transcription.

\ex\label{ex:099-22-all-working-ran}%
\exmn{301.11}%
\begingl
	\glpreamble	Łdakᴀ′t yuxā′t yedānē′ỵî ỵī′ỵîawe dut!ā′t ʟūwagu′q. //
	\glpreamble	Ldakát yú x̱áat yéi adaanéiÿi ÿéeÿi áwé du tʼáat loowagúḵ. //
	\gla	Ldakát {} yú {} {} x̱áat {}
			yéi @ \rlap{adaanéiÿi} @ {} @ {} @ {} @ {} @ \•ÿéeÿi {} {}
		\rlap{áwé} @ {} +
		{} du \rlap{tʼáat} @ {} {}
		\rlap{loowagúḵ.} @ {} @ {} @ {} //
	\glb	ldakát {} yú {} {} x̱áat {}
			yéi= a- daa- \rt[²]{ne} -μμH -í =ÿéeÿi {} {}
		á -wé
		{} du tʼáaᵏ -t {}
		lu- i- \rt[¹]{guḵ} -μH //
	\glc	all {}[\pr{DP} \xx{dist} {}[\pr{CP} {}[\pr{DP} salmon {}]
			thus= \xx{arg}- around- \rt[²]{work} -\xx{var} -\xx{sub} =\xx{past} {}] {}]
		\xx{foc} -\xx{mdst}
		{}[\pr{PP} \xx{3h·pss} behind -\xx{pnct} {}]
		nose- \xx{pfv}- \xx{stv}- \rt[¹]{push} -\xx{var} //
	\gld	all {} those {} {} fish {} thus \rlap{3>3.\xx{ncnj}.\xx{impfv}.work} {} {} {} which \•\xx{past} {} {}
		\rlap{it.is} {}
		{} his behind -to {}
		\rlap{\xx{zcnj}.\xx{pfv}.run·\xx{pl}} {} {} {} //
	\glft	‘It was all of those who had been working on fish that ran behind him.’
		//
\endgl
\xe

\ex\label{ex:099-23-searching-for-him}%
\exmn{302.1}%
\begingl
	\glpreamble	Qoducī′ duīg̣a′. //
	\glpreamble	Ḵudushée du eeg̱áa. //
	\gla	\rlap{Ḵudushée} @ {} @ {} @ {}
		{} du \rlap{eeg̱áa.} @ {} {} //
	\glb	ḵu- du- \rt[²]{shiʰ} -μμH
		{} du ee -g̱áa {} //
	\glc	\xx{areal}- \xx{4h·s}- \rt[²]{reach·for} -\xx{var}
		{}[\pr{PP} \xx{3h·pss} \xx{base} -\xx{ades} {}] //
	\gld	\rlap{\xx{ncnj}.\xx{impfv}.ppl.search} {} {} {}
		{} him {} -for {} //
	\glft	‘They are searching for him.’
		//
\endgl
\xe

\ex\label{ex:099-24-didnt-know-happened}%
\exmn{302.1}%
\begingl
	\glpreamble	Tc!uʟe′ ʟēł wudusku′ wā′sa wa′niỵe. //
	\glpreamble	Chʼu tle tléil wuduskú wáa sá wooneeyi ÿé. //
	\gla	Chʼu tle tléil \rlap{wuduskú} @ {} @ {} @ {} @ {} @ {} @ {} +
		{} {} {} wáa sá {}
			\rlap{wooneeyi} @ {} @ {} @ {} @ {} {} ÿé {} //
	\glb	chʼu tle tléil u- wu- du- d- s- \rt[²]{kuʰ} -μH
		{} {} {} wáa sá {}
			wu- i- \rt[¹]{niʰ} -μμL -i {} yé {} //
	\glc	just then \xx{neg} \xx{irr}- \xx{pfv}- \xx{4h·s}- \xx{mid}- \xx{xtn}- \rt[²]{know} -\xx{var}
		{}[\pr{DP} {}[\pr{CP} {}[\pr{QP} how \xx{q} {}]
			\xx{pfv}- \xx{stv}- \rt[¹]{happen} -\xx{var} -\xx{rel} {}] way {}] //
	\gld	just then not \rlap{\xx{zcnj}.\xx{pfv}.ppl.know} {} {} {} {} {} {}
		{} {} {} how ever {}
			\rlap{\xx{ncnj}.\xx{pfv}.happen} {} {} {} -that {} way {} //
	\glft	‘Just then people did not know how the way that it happened.’
		//
\endgl
\xe

\section{Paragraph 2}\label{sec:099-para-2}

This paragraph break has been added to the original.
\citeauthor{swanton:1909}’s presentation has the first paragraph run over three pages combining several separate scenes with different participants and backgrounds.
This particular spot in the narrative has a significant change in perspective so it is a suitable location for a new paragraph.

\ex\label{ex:099-25-salmon-proud}%
\exmn{302.2}%
\begingl
	\glpreamble	Tc!uʟe′ yuxā′t qo′a ayu′ tuwu′qłig̣ê. //
	\glpreamble	Chʼu tle yú x̱áat ḵu.aa áyú toowú klig̱éi. //
	\gla	Chʼu tle {} yú x̱áat {} ḵu.aa \rlap{áyú} @ {}
		{} {} \rlap{toowú} @ {} {} \rlap{klig̱éi.} @ {} @ {} @ {} @ {} //
	\glb	chʼu tle {} yú x̱áat {} ḵu.aa á -yú
		{} {} tú -í {} k- l- i- \rt[¹]{g̱e} -μμH //
	\glc	just then {}[\pr{DP} \xx{dist} salmon {}] \xx{contr} \xx{foc} -\xx{dist}
		{}[\pr{DP} \xx{3·pss} inside -\xx{pss} {}] \xx{qual}- \xx{xtn}- \xx{stv}- \rt[¹]{bright} -\xx{var} //
	\gld	just then {} those salmon {} however \rlap{it.is} {}
		{} their mind {} {} \rlap{\xx{gcnj}.\xx{impfv}.bright} {} {} {} {} //
	\glft	‘Then those salmon however, they are proud.’
		//
\endgl
\xe
