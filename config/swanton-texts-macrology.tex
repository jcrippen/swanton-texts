%!TEX root = ../swanton-texts.tex
%%
%% Miscellaneous macrology.
%%

%% The new LaTeX3 generic document command parser.
\RequirePackage{xparse}
%% The new xparse package doesn’t have something built in to test if
%% something is either -NoValue- or just empty, but it’s easy enough to
%% implement, so here we go.
%%
%% FIXME: Detect if this is defined elsewhere, in case someone else
%% implements this obvious function.
\ExplSyntaxOn
\ProvideExpandableDocumentCommand \IfNoValueOrEmptyTF {mmm}
 	{\IfNoValueTF{#1}{#2}
		{\tl_if_empty:nTF {#1} {#2} {#3}}}
\ExplSyntaxOff

%% Gloss abbreviations.
%%
%% NOTE: Requires document authors to define \glossfont usefully.
%%
%% \xx{} is used for abbreviations in text, \zz{} is used for abbreviations in
%% glosses. The latter uses a condensed font so that it takes up less space.
%%
%% Screw people who don’t like two letter macros. These are mine and I don’t
%% care one whit about your namespace.
%%
%% FIXME: Detect lack of \glossfont and print a warning.
%% FIXME: Detect lack of \scshape and print a warning. This is much harder.
\ProvideDocumentCommand \xx { m } {\textsc{#1}}
\ProvideDocumentCommand \zz { m } {{\glossfont\textsc{#1}}}

%% Italicized semantic variables.
%%
%% Semantics not in math mode needs explicit italicization, but this should be
%% presentationally separated from hardcoded \textit{…}. This pair of commands
%% follows the \fm and \sm templates above.
\makeatletter
\ProvideDocumentCommand \sv@cmd { m } {\textit{#1}}
%% The \sm command has an optional star to prevent hyphenation via
%% an \mbox, just like \fm above.
\ProvideDocumentCommand \sv { s m } {%
	\IfBooleanTF{#1}%
		{\mbox{\sv@cmd{#2}}}
		{\sv@cmd{#2}}}
\makeatother

%% Italicized linguistic forms appearing in text.
%%
%% The \fm@cmd command does the underlying formatting.
%% This just calls LaTeX \emph which does the right thing, but it can be replaced.
\makeatletter
\ProvideDocumentCommand \fm@cmd { m } {\emph{#1}}
\makeatother
%% The \fm command is the user interface.
%%
%%  arg 1: star
%%    This wraps everything in an mbox to prevent line breaking
%%    because hyphenation can get ugly especially for suffixes.
%%  arg 2: stigma
%%    An optional argument to appear before the form. This can be a
%%    star (asterisk) since it is always a square bracket argument.
%%  arg 3: body of linguistic form
\makeatletter
\ProvideDocumentCommand \fm { s O{} m } {%
	\IfBooleanTF{#1}%
		{\mbox{#2\fm@cmd{#3}}}
		{#2\nobreak\fm@cmd{#3}}}
\makeatother

%% Roots.
%%
%% The root symbol (U+221A Square Root) in many fonts is taller than
%% the usual cap or ascender height. This \raisebox hack sticks the
%% root into a zero-height box so that it doesn’t cause an increase
%% in leading. Since this symbol often has an excessively large right
%% sidebearing for our purposes because of the angle of the ascender,
%% the negative horizontal space in \rtkern kerns the following
%% letters a bit more closely.
%%
%% The optional argument is meant for a superscript number like ² that
%% indicates the valency of the root. The superscript is rlapped so
%% that it appears above the vee of the root symbol. Since this never
%% works properly, a \raisebox moves the superscript around to get in
%% the right spot. This can be configured using the \rtsuplift length.
%% The \raisebox for the superscript has zero height and depth so it
%% doesn't screw up other spacing.
\newlength{\rtkern}
\setlength{\rtkern}{-0.0625ex}
\newlength{\rtsuplift}
\setlength{\rtsuplift}{0.66ex}
\ProvideDocumentCommand \rt { o m } {%
	\IfNoValueOrEmptyTF{#1}{}%
	{\raisebox{\rtsuplift}[0pt][0pt]{\rlap{\unspace#1\unspace}}}%
	\raisebox{0pt}[0pt][0pt]{√}\hspace*{\rtkern}#2}
%% This command allows the root to hang on the left side. It
%% is meant for column labels where the root symbol otherwise
%% visually misaligns the column label with the rest of the column.
%% Note that the \rt{} command still applies \rtkern despite \llap.
\ProvideDocumentCommand \rtlap { m } {\llap{\rt{}}#1}
%% The Brill font doesn’t need its roots kerned.
\setlength{\rtkern}{0pt}

%% Root as a symbol.
%%
%% This smashes the root into a 1em vertical box, adding slight whitespace on the right.
\ProvideDocumentCommand \rtsym {} {\raisebox{0pt}[0.75em][0.25em]{√\!}}

%% Gloss emphasis.
%%
%% Though some authors use underlining to emphasize elements in glosses,
%% this is a bad idea for Tlingit because of the underscore diacritic (U+0330).
%% I use bolding instead. Optional colouration and background shading can
%% also be turned on and adjusted to taste. Colour uses the xcolor package.
%%
%% This uses LaTeX3 code.
\makeatletter
\ExplSyntaxOn
%% A boolean to toggle bolding on and off.
\bool_new:c {gmboldbool}
%% A boolean to toggle colour on and off.
\bool_new:c {gmcolorbool}
%% A boolean to toggle background colour on and off.
\bool_new:c {gmbgcolorbool}
%% The toggle commands.
\ProvideDocumentCommand \gmboldon { } {\bool_set_true:c {gmboldbool}}
\ProvideDocumentCommand \gmboldoff { } {\bool_set_false:c {gmboldbool}}
\ProvideDocumentCommand \gmcoloron { } {\bool_set_true:c {gmcolorbool}}
\ProvideDocumentCommand \gmcoloroff { } {\bool_set_false:c {gmcolorbool}}
\ProvideDocumentCommand \gmbgcoloron { } {\bool_set_true:c {gmbgcolorbool}}
\ProvideDocumentCommand \gmbgcoloroff { } {\bool_set_false:c {gmbgcolorbool}}
%% The colours.
%% FIXME: Expand to allow different xcolor's other colour models, e.g. CMYK.
\providecolor{gmcolor}{rgb}{0.66,0,0}
\providecolor{gmbgcolor}{rgb}{0.95,0.95,0.95}
%% Command to change the colours.
%% The \definecolor macro overwrites the current value with a new one.
\ProvideDocumentCommand \gmcolorset { m } {\definecolor{gmcolor}{rgb}{#1}}
\ProvideDocumentCommand \gmbgcolorset { m } {\definecolor{gmbgcolor}{rgb}{#1}}
%% A strut with 90 percent of normal height and width for the colorbox.
%% See http://tex.stackexchange.com/questions/74459/remove-space-before-colorbox
\ProvideDocumentCommand \gm@smallstrut {}
	{\vrule width 0pt height 0.9\ht\strutbox depth 0.9\dp\strutbox\relax}
%%
%% The guts.
%%
\ProvideDocumentCommand \gm@bold { m }
	{\bool_if:cTF {gmboldbool}
		{\textbf{#1}}
		{#1}}
\ProvideDocumentCommand \gm@color { m }
	{\bool_if:cTF {gmcolorbool}
		{\textcolor{gmcolor}{#1}}
		{#1}}
\ProvideDocumentCommand \gm@bgcolor { m }
	{\bool_if:cTF {gmbgcolorbool}
		% A colorbox has a frame built into it which we set to
		% zero using \setlength inside a group for scope.
		{\begingroup
		\setlength{\fboxsep}{0pt}%
		\colorbox{gmbgcolor}{\gm@smallstrut{}#1}%
		\endgroup}
		{#1}}
%%
%% The actual gloss emphasis command \gm.
%%
%% NOTE: I’m not sure if the order of operations matters or not. IWFM.
\ProvideDocumentCommand \gm { m }
	{\gm@bgcolor{\gm@color{\gm@bold{#1}}}}
%% Back to normal LaTeX2e.
\ExplSyntaxOff
\makeatother
%% Settings for the \gm gloss emphasis macro.
%%
%% Bold text.
\gmboldoff
%% Foreground colour.
\gmcoloroff
%% Background shading.
\gmbgcoloron
%% Adjust the two colours.
\gmcolorset{0.66,0,0}
\gmbgcolorset{0.9,0.9,0.9}

%% An underscore.
%%
%% Both measurements are font dependent and thus adapt to different font sizes.
%% The thickness measurement was empirically determined.
\newlength{\uscorethickness}
\setlength{\uscorethickness}{0.15625ex}
\newlength{\uscorewidth}
\setlength{\uscorewidth}{1.25ex}
\ProvideDocumentCommand \_ {} {\rule{\uscorewidth}{\uscorethickness}}
%% The Brill font needs a wider underscore.
\setlength{\uscorewidth}{0.875em}

%% Notes for later changes.
%%
%% FIXME: Detect whether font has U+27E6 & U+27E7, use them instead.
\newlength{\dblbrackkern}
\setlength{\dblbrackkern}{-0.325ex}
\ProvideDocumentCommand \dblbrackleft {} {[\hspace{\dblbrackkern}[}
\ProvideDocumentCommand \dblbrackright {} {]\hspace{\dblbrackkern}]}
\ProvideDocumentCommand \FIXME { +m }
	{\mbox{\dblbrackleft}\textsc{Fixme}: #1\mbox{\dblbrackright}}
%% Use real double brackets for fixme comments.
\renewcommand*{\dblbrackleft}{⟦}
\renewcommand*{\dblbrackright}{⟧}

%% A tilde.
%\renewcommand*{\~}{\textasciitilde}
%% For some reason the \textasciitilde macro has recently (circa
%% February 2017) started producing U+02DC Small Tilde instead
%% of U+007E Tilde. This forces the latter character.
\DeclareDocumentCommand \~ {} {\char`~}

%% Phrase labels.
%%
%% FIXME: Provide star to toggle fake subscripts or real ones. This
%% should be the inverse of the star for \textsubscript.
\ProvideDocumentCommand \pr { m } {\textsubscript*{#1}}

%% Phonetics.
%%
%% FIXME: Detect lack of \ipafont and print warning.
\ProvideDocumentCommand \ipa { m } {{\ipafont #1}}

%% Subscript italic indices.
%%
%% FIXME: Detect whether font has a specific subscript (e.g. U+1D62).
\ProvideDocumentCommand \ix { m } {\textsubscript*{\textit{#1}}}

%% Orthographic representation.
\ProvideDocumentCommand \orth { m } {⟨#1⟩}

%% Phantom (invisible) hyphens for alignment in examples.
\ProvideDocumentCommand \· {} {\phantom{-}}

%% Phantom (invisible) equals signs for alignment in examples.
\ProvideDocumentCommand \• {} {\phantom{=}}

%% Small spaces.
%%
%% These override the builtin ones which are only defined for math mode and
%% which I don't use.
\DeclareDocumentCommand \! {} {\nolinebreak\hspace{0.2ex}}
\DeclareDocumentCommand \: {} {\nolinebreak\hspace{-0.2ex}}

%% Stigmata macros.
\ProvideDocumentCommand \† {} {\llap{\textsuperscript*{†}}}

%% Command to force lining numerals instead of old style.
\ProvideDocumentCommand \liningnumerals {m}
	{\begingroup\addfontfeatures{Letters=Uppercase,Numbers=Lining}#1\endgroup}

%% Second language abbreviation.
\ProvideDocumentCommand \Lone {} {\liningnumerals{L1}}
\ProvideDocumentCommand \Ltwo {} {\liningnumerals{L2}}

%% Orthography abbreviation.
%\ProvideDocumentCommand \NSone {} {\liningnumerals{NS1}}
%\ProvideDocumentCommand \NStwo {} {\liningnumerals{NS2}}

%% Empty spacing for gloss abbreviation (\glc) line.
\DeclareDocumentCommand {\glcsp} {s}
	{\hspace{1em}\IfBooleanTF{#1}{\hspace{1em}}}

%% Fake spacing for gloss text line (\gla), using ExPex’s length.
\DeclareDocumentCommand {\glasp} {} {\hspace\lingglspace}

%% Verb ephemera.
\ProvideDocumentCommand \vbeph { m m }
	{(\IfNoValueOrEmptyTF{#1}{}{\fm{#1}\IfNoValueOrEmptyTF{#2}{}{; }}\IfNoValueOrEmptyTF{#2}{}{#2})}

%% Verb lexical entry.
\ProvideDocumentCommand \vblex { m m m m }
	{\IfNoValueOrEmptyTF{#1}{}{\fm{#1} }\vbeph{#2}{#3}\IfNoValueOrEmptyTF{#4}{}{ ‘#4’}}

%% Verb derivation.
\ProvideDocumentCommand \vbderiv { m m m m }
	{\vblex{#1}{#2}{#3}{#4}}

%% Verb lexical entry in examples.
\ProvideDocumentCommand \exvblex { s m m m m }
	{\IfBooleanTF{#1}{}{\newline}\footnotesize\vblex{#2}{#3}{#4}{#5}\normalsize}

%% Verb derivation in examples.
\ProvideDocumentCommand \exvbderiv { s m m m m }
	{\IfBooleanTF{#1}{}{\newline}\footnotesize\vbderiv{#2}{#3}{#4}{#5}\normalsize}

%% Binominal species names.
\ProvideDocumentCommand \species { s m m o} 
	{\textit{#2}~\textit{#3}\IfNoValueOrEmptyTF{#4}{}{~\IfNoValueTF{#1}{(}{}#4\IfNoValueTF{#1}{)}{}}}

%% Some memoir configuration for marginal notes.
\marginparmargin{right}
\setmarginnotes{1ex}{4em}{0.5ex}
%% Margin note for example number.
\DeclareDocumentCommand {\exmn} {m} {\marginpar{\footnotesize{}(#1)}}

%% Kerning for sentence numbers before double quotes.
\ProvideDocumentCommand \qqk {}
	{\nolinebreak\hspace{-0.5ex}}

%% Pre-guillemet phantom spacing.
%% For use in glosses wherethe \gla line has a guillemet quote.
\ProvideDocumentCommand \pqp {}
	{\phantom{\normalfont«\!}}