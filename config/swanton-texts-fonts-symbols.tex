%!TEX root = ../swanton-texts.tex
%!TEX encoding = UTF-8 Unicode
%%
%% Font support for special symbols (e.g. math).
%%

%% Slashed zero for the nul or zero symbol.
%%
%% This bagbiting cruft is font dependent because most fonts have a circle
%% and line for their null set rather than a slashed zero. Why? Because Unicode
%% doesn’t specify a unique slashed zero character. The distinction is
%% supposedly a ‘presentation form’ issue according to the punctilious
%% mathematicians that know better than The Rest of Us. The OpenType
%% specification gives a [zero] feature but that only turns on slashed zero
%% for regular zero. Font designers have, to date, not included an alternate
%% form of the circular null set, despite my repeated suggestions.
%%
%% The \texorpdfstring macro from hyperref should convert the printed
%% form (some flavour of U+0030 Digit Zero) into the real null symbol
%% character (U+2205 Empty Set) for the PDF text stream that is given
%% with cut-and-paste and in PDF bookmarks. The first argument of
%% the \texorpdfstring macro is the visual representation, the second is
%% the text stream representation. Put that in your presentation forms
%% pipe and smoke it.
%%
%% If you ask me “Can’t you just use U+0030 and U+0338 Combining
%% Long Solidus Overlay?” then I will curse your name unto the ends of
%% the Earth.
%%
%% If \zerofont is undefined then we just hope for the best and use U+2205.
\makeatletter
\@ifundefined{zerofont}%
	{\ProvideDocumentCommand \nul { } {∅}}
	%% Else \texorpdfstring is defined by the hyperref package. If not,
	%% then we just define \nul without it.
	{\@ifundefined{texorpdfstring}%
		% Just do the \zerofont.
		{\ProvideDocumentCommand \nul { } {{\zerofont 0}}}
		% Use \texorpdfstring.
		% FIXME: Using \ProvideDocumentCommand doesn’t work here. Why?
		{\newcommand{\nul}{\texorpdfstring{{\zerofont 0}}{∅}}}}
\makeatother

% https://tex.stackexchange.com/questions/367773/command-textsuperscript-unavailable-in-encoding-tu?noredirect=1
\DeclareTextSymbol{\texttilde} {TU} {"007E}

%% Short command for a superscript question mark.
\ProvideDocumentCommand \supq {} {\textsuperscript*{?}}

%% Load the newunicodechar package.
%%
%% This package lets you define arbitrary macros to run for particular
%% Unicode characters (or any character, really). This allows you to
%% switch fonts per character, and thus use different fonts for e.g.
%% mathematical symbols, Hebrew letters, etc. It should not be used
%% for long stretches of text, which instead should be done with a
%% specialized command or environment that changes the font in a
%% more general way.
%%
%% All that the \newunicodechar command is really doing under the
%% hood is something like plain TeX
%%   \catcode`\X=\active
%%   \defX{\textbf{\char"2323}}
%% but it's much easier to use and so less error-prone.
\usepackage{newunicodechar}

%% This font is specifically for the check symbol that indicates
%% grammaticality or felicity. The macro can be used in contexts
%% where the character can't, with the same printed result. E.g.
%% ExPex can use \ck to determine judgement width, but not ✓.
%%
%% NOTE: You need two braces here because of macro processing.
%% You can also use \begingroup ... \endgroup. It's a TeX thing.
\ProvideDocumentCommand \ck {} {{\ckfont ✓}}
%% Two braces again, here and elsewhere below.
\newunicodechar{✓}{{\ck}}

%% This font is for the X symbol that contrasts with the check above.
%% The same principles apply.
%\newfontfamily{\badxfont}[Scale=1]{Zapf Dingbats}
\newfontfamily{\badxfont}{Zapf Dingbats}
\ProvideDocumentCommand \badx {} {{\badxfont ✓}}
\newunicodechar{✗}{{\badxfont ✗}}

%% The slashed zero as the null symbol.
%% The macro is defined in crippenmacros.sty.
\newunicodechar{∅}{{\nul}}

%% The Brill font has Greek-style default versions of {β, θ, λ, χ} and
%% Roman-style versions available with ss20. But I only want
%% the roman θ and not the others. Brill 2.02 doesn’t implement
%% character variant features (cvXX) for these, so the only way to access
%% them is through ss20. I use newunicodechar to automatically add
%% ss20 whenever θ appears.
%%
%% This solution completely ignores the current font. That means that ss20 will
%% be applied in every situation, no matter what. So even if the font is no
%% longer Brill for some reason, θ will still show up with ss20. It works fine if
%% you’re consistent and careful, but it isn’t ideal. cvXX would be ideal.
\newunicodechar{θ}{{\addfontfeature{StylisticSet=20}θ}}

%% Mathematical symbols.
%%
%% If you have things with diacritics (e.g. math negation) then if
%% these are decomposed (NFD) it will make TeX barf. YHBW.
%%
%% The double braces are on purpose, with the inner pair
%% restricting the scope of the font switching commands.
%%
%% FIXME: This should be in an array with a loop instantiating each
%% character, rather than repeating the same code over and over again.
%% It wouldn't make it any more efficient, but it'd be easier to modify.
%% Actually it would be a couple array+loops, one for each font.

%% Set stuff.
\newunicodechar{∈}{{\mthfont ∈}}	% element
\newunicodechar{∉}{{\mthfont ∉}}	% not element
\newunicodechar{⊂}{{\mthfont ⊂}}	% proper subset
\newunicodechar{⊃}{{\mthfont ⊃}}	% proper superset
\newunicodechar{⊆}{{\mthfont ⊆}}	% total subset
\newunicodechar{⊇}{{\mthfont ⊇}}	% total superset
\newunicodechar{⊊}{{\mthfont ⊊}}	% nontotal subset
\newunicodechar{⊋}{{\mthfont ⊋}}	% nontotal superset
\newunicodechar{⊈}{{\mthfont ⊈}}	% not total subset
\newunicodechar{⊉}{{\mthfont ⊉}}	% not total superset
\newunicodechar{∪}{{\mthfont ∪}}	% union
\newunicodechar{∩}{{\mthfont ∩}}	% intersection
\newunicodechar{⋃}{{\mthfont ⋃}}	% big union

%% Logic and quantification.
\newunicodechar{≡}{{\mthfont ≡}}	% triple equals
\newunicodechar{≢}{{\mthfont ≢}}	% triple not equals
\newunicodechar{≅}{{\mthfont ≅}}	% triple approximately equals
\newunicodechar{∧}{{\mthfont ∧}}	% logical and
\newunicodechar{∨}{{\mthfont ∨}}	% logical or
\newunicodechar{∀}{{\mthfont ∀}}	% for all
\newunicodechar{∃}{{\mthfont ∃}}	% exists
\newunicodechar{∄}{{\mthfont ∄}}	% not exists

%% Orderings.
\newunicodechar{⊤}{{\mthfont ⊤}}	% supremum, top
\newunicodechar{⊥}{{\mthfont ⊥}}	% infimum, bottom
\newunicodechar{⊑}{{\mthfont ⊑}}	% precedes
\newunicodechar{⊒}{{\mthfont ⊒}}	% succedes
\newunicodechar{⊏}{{\mthfont ⊏}}	% strictly precedes
\newunicodechar{⊐}{{\mthfont ⊐}}	% strictly succedes
\newunicodechar{⋢}{{\mthfont ⋢}}	% not succeeds
\newunicodechar{≪}{{\mthfont ≪}}	% much less than
\newunicodechar{≫}{{\mthfont ≫}}	% much greater than
\newunicodechar{≶}{{\mthfont ≶}}	% less than over greater than
\newunicodechar{≷}{{\mthfont ≷}}	% greater than over less than
\newunicodechar{≸}{{\mthfont ≸}}	% neither less than nor greater than
\newunicodechar{≹}{{\mthfont ≹}}	% neither greater than nor less than
\newunicodechar{≼}{{\mthfont ≼}}	% curved precedes
\newunicodechar{≽}{{\mthfont ≽}}	% curved succedes
\newunicodechar{≺}{{\mthfont ≺}}	% curved strictly precedes
\newunicodechar{≻}{{\mthfont ≻}}	% curved strictly succedes
\newunicodechar{⊀}{{\mthfont ⊀}}	% not curved strictly succeeds

%% Arrows.
\newunicodechar{↮}{{\mthfont ↮}} % left right arrow with stroke
\newunicodechar{⇒}{{\mthfont ⇒}} % double east arrow
\newunicodechar{⇐}{{\mthfont ⇐}} % double west arrow
\newunicodechar{⇔}{{\mthfont ⇔}} % double east-west arrow
\newunicodechar{⇏}{{\mthfont ⇏}} % not double east arrow
\newunicodechar{⇍}{{\mthfont ⇍}} % not double west arrow
\newunicodechar{⇎}{{\mthfont ⇎}} % not double east-west arrow
\newunicodechar{↦}{{\mthfont ↦}} % maps to
\newunicodechar{↤}{{\mthfont ↤}} % maps from
\newunicodechar{⇦}{{\mthfont ⇦}} % leftwards white arrow
\newunicodechar{⇴}{{\mthfont ⇴}} % right arrow with small circle

%% Phonetic stuff.
%\newunicodechar{ᵃ}{{\phonfont ᵃ}}	% superscript a
%\newunicodechar{ᵊ}{{\phonfont ᵊ}}	% superscript e
%\newunicodechar{ᵒ}{{\phonfont ᵒ}}	% superscript o
%\newunicodechar{ᵏ}{{\phonfont ᵏ}}	% superscript k
%\newunicodechar{ᵘ}{{\phonfont ᵘ}}	% superscript u
%\newunicodechar{ᵉ}{{\phonfont ᵉ}}	% superscript e
%\newunicodechar{ᵗ}{{\phonfont ᵗ}}	% superscript t
%\newunicodechar{ꜝ}{{\phonfont ꜝ}}	% superscript !
%\newunicodechar{ꜜ}{{\phonfont ꜜ}}	% superscript down arrow
%\newunicodechar{ꜛ}{{\phonfont ꜛ}}	% superscript up arrow
\newunicodechar{ₚ}{{\phonfont ₚ}}	% subscript p
%\newunicodechar{ₐ}{{\phonfont ₐ}}	% subscript a
%\newunicodechar{ᴴ}{{\phonfont ᴴ}}	% superscript capital H

%% Miscellanea.
\newunicodechar{¬}{{\mthfont ¬}}		% not: Brill has this but it’s too low
\newunicodechar{⟨}{{\mthfont ⟨}}		% math left angle bracket
\newunicodechar{⟩}{{\mthfont ⟩}}		% math right angle bracket
\newunicodechar{〈}{{\mthfont 〈}}	% left-pointing angle bracket
\newunicodechar{〉}{{\mthfont 〉}}	% right-pointing angle bracket
\newunicodechar{⋮}{{\mthfont ⋮}}		% vertical ellipsis
\newunicodechar{⋯}{{\mthfont ⋯}}	% middle horizontal ellipsis
\newunicodechar{⊗}{{\upshape\mthfont ⊗}}	% circle times
\newunicodechar{⬚}{{\symfont ⬚}}	% dotted box
